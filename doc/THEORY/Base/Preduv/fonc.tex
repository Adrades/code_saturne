%                      Code_Saturne version 1.3
%                      ------------------------
%
%     This file is part of the Code_Saturne Kernel, element of the
%     Code_Saturne CFD tool.
% 
%     Copyright (C) 1998-2007 EDF S.A., France
%
%     contact: saturne-support@edf.fr
% 
%     The Code_Saturne Kernel is free software; you can redistribute it
%     and/or modify it under the terms of the GNU General Public License
%     as published by the Free Software Foundation; either version 2 of
%     the License, or (at your option) any later version.
% 
%     The Code_Saturne Kernel is distributed in the hope that it will be
%     useful, but WITHOUT ANY WARRANTY; without even the implied warranty
%     of MERCHANTABILITY or FITNESS FOR A PARTICULAR PURPOSE.  See the
%     GNU General Public License for more details.
% 
%     You should have received a copy of the GNU General Public License
%     along with the Code_Saturne Kernel; if not, write to the
%     Free Software Foundation, Inc.,
%     51 Franklin St, Fifth Floor,
%     Boston, MA  02110-1301  USA
%
%-----------------------------------------------------------------------
%

\programme{preduv}
%
\vspace{1cm}
%%%%%%%%%%%%%%%%%%%%%%%%%%%%%%%%%%
%%%%%%%%%%%%%%%%%%%%%%%%%%%%%%%%%%
\section{Fonction}
%%%%%%%%%%%%%%%%%%%%%%%%%%%%%%%%%%
%%%%%%%%%%%%%%%%%%%%%%%%%%%%%%%%%%
Dans ce sous-programme, on effectue l'\'etape de pr\'ediction de la vitesse
$\vect{u}$. Ceci consiste � r\'esoudre l'�quation de quantit\'e de
mouvement (Q.D.M.) en traitant la pression $p$ de mani�re explicite. La solution en vitesse-pression est obtenue apr�s une �tape de correction sur la pression
effectu�e dans le sous-programme \fort{resolp}, en utilisant la loi de conservation de la masse :
\begin{equation}
\frac{\partial \rho } {\partial t}+ \dive(\rho \underline{u}) = \Gamma, 
\end{equation}
o\`u $\Gamma$ est le terme source de masse\footnote{ en $kg.m^{-3}.s^{-1}$ }.\\  
L'�quation de conservation de la quantit� de mouvement moyenne obtenue par application 
du th�or�me fondamental de la dynamique est :
\begin{equation}
\frac {\partial (\rho \underline {u})} {\partial t }+
\dive(\rho \underline{u} \otimes \underline{u}) =
\dive(\underline{\underline{\sigma}}) + \underline{S} - \dive{(\rho\,\tens{R})}\end{equation}
o� : 
\begin{equation}
\underline{\underline{\sigma}} = - p \underline{\underline{Id}} + \underline{\underline{\tau }}
\end{equation}
avec pour les �coulements newtoniens, la relation lin�aire suivante :
\begin{equation}
\begin{array}{lcl}
&\displaystyle \underline{\underline{\tau}} = 2\ \mu\ \underline{\underline{D}}
+\, 
 \lambda\ tr(\underline{\underline{D }})\ \underline{\underline{Id}} &\\
&\displaystyle \underline{\underline{D}}=\frac{1}{2}\ (\ggrad \underline
{u} +\ ^t\ggrad \underline {u})
\end{array}
\end{equation}


$\tens{\sigma}$ repr\'esente le tenseur de contraintes, $\tens{\tau}$ le tenseur
des contraintes visqueuses, $\mu$ la viscosit\'e dynamique (mol\'eculaire et
\'eventuellement turbulente), $\tens{D}$
 le tenseur taux de d\'eformation\footnote{\`A ne pas confondre, malgr\'e la m\^eme notation $D$, avec les flux
diffusifs d\'ecrits dans le sous-programme \fort{navsto}},
$\tens{R}$ le tenseur de Reynolds qui appara\^\i t lors de l'application de
l'op\'erateur moyenne \`a l'\'equation instantan\'ee, $\underline{S}$ les termes
sources.\\
$\lambda$ est le second coefficient de viscosit�. Il est reli� � la viscosit� de
volume $\kappa$ par la relation
\begin{equation}
\lambda=\kappa-\frac{2}{3}\mu
\end{equation}
Quand l'hypoth�se de Stokes est v�rifi�e, la viscosit� de volume $\kappa$ est
nulle, soit $3\lambda+2\mu=0$. Cette hypoth�se est implicite dans le code et
dans le reste du doument, sauf en compressible.\\


L'\'equation de conservation de la quantit� de mouvement s'\'ecrit finalement
 :
\begin{equation}
\begin{array}{lcl}
&\displaystyle \rho\,
\frac{\partial \underline {u} } {\partial t} = -\ 
\underbrace {\dive(\rho \underline{u} \otimes \underline{u})}_{\text{
convection}} +\ \underbrace {\dive (\mu\ \ggrad \underline {u})}_{\text{
diffusion}} &\\
&\displaystyle \underbrace { +\ \dive (\mu \,^t\ggrad \underline {u}) }_{\text{
terme en gradient transpos\'e}}
\underbrace { - \ \frac {2} {3}\ \grad (\mu\ \dive \underline {u})}_{\text{
viscosit� secondaire}}\ \ - \dive{(\rho \tens{R})}
 -\ \grad(p) + (\rho -\rho_0)\,\underline {g} + 
\underline{u}\,\dive (\rho\,\underline {u})&\\
&\displaystyle +\underbrace {\Gamma
(\underline{u}_{\,i}-\underline{u})}_{\text{terme source de Q.D.M. d� � la source
de masse}}- \underbrace {\rho\ 
\tens{K}_{\,pdc} \underline {u}}_{\text{perte
de charge}} + \underbrace { \underline{T}_{\,s}^{\,exp}+
T_{s}^{\,imp}\ \underline{u}}_{\text{autres termes sources de
Q.D.M.}}
\label{Base_Preduv_eqqdm}

\end{array}
\end{equation}
avec $p$ d�finissant l'�cart � la pression hydrostatique de r\'ef\'erence (la
pression hydrostatique r\'eelle \'etant calcul\'ee avec la masse volumique $\rho$
et non $\rho_{\,0}$) :
\begin{equation}
p=p^*-\rho_{\,0}\ \underline{g}\,.\,\underline{X}
\end{equation}
(\underline{X} \'etant le vecteur de composantes $x$, $y$ et $z$).\\
$\mu_t$, $\tens{K}_{\,pdc}$, $\underline{u}_{\,i}$ repr\'esentent respectivement
la viscosit� dynamique turbulente, le tenseur des pertes de charge et la valeur de la variable associ�e � la source de
masse.\\
La divergence du tenseur des contraintes de Reynolds s'\'ecrit :
\begin{equation}
-\dive{(\rho\,\tens{R})}=
\begin{cases}
\vect{0} & \text{en laminaire}, \\
 -\displaystyle\frac {2} {3}\, \grad (\mu_t\ \dive \underline {u})+\dive (\mu_t\ (\ggrad \underline {u}+ \,^t\ggrad \underline {u}))-\frac {2}{3}\,\grad (\rho\, k) & \text{pour les mod�les}\\
 & \text{� viscosit� turbulente}, \\
 -\dive(\rho\,\tens{R})& \text{pour les mod�les}\\
 & \text{au second ordre},\\
-\displaystyle\frac {2} {3}\, \grad (\mu_t\ \dive \underline {u})+\dive (\mu_t\ (\ggrad \underline {u}+ \,^t\ggrad \underline {u})) & \text{en  LES}\\
\end{cases}
\end{equation}
Le terme source de masse fait intervenir la vitesse locale $\underline {u}$ et
aussi une vitesse $\underline {u}_{\,i}$ associ\'ee \`a la masse inject\'ee (ou retir\'ee). 
Lorsque $\Gamma<0$, on \^ote de la masse au syst\`eme et on a donc 
$\underline{u}_{\,i} = \underline{u}$. Le terme est nul (\emph{i.e.} $\Gamma
(\underline{u}_{\,i}-\underline{u})= \underline{0} $). Quand $\Gamma>0$, on a un
terme non nul si $\underline{u}_{\,i} \ne \underline{u}$.
Dans ce sous-programme, tous les termes intervenant dans
l'�quation de conservation de la
quantit� de mouvement, except� les termes de convection et diffusion, sont
calcul�s et transmis au sous-programme \fort{codits} qui r�sout l'�quation compl�te
(convection-diffusion avec termes sources).


