%                      Code_Saturne version 1.3
%                      ------------------------
%
%     This file is part of the Code_Saturne Kernel, element of the
%     Code_Saturne CFD tool.
%
%     Copyright (C) 1998-2007 EDF S.A., France
%
%     contact: saturne-support@edf.fr
%
%     The Code_Saturne Kernel is free software; you can redistribute it
%     and/or modify it under the terms of the GNU General Public License
%     as published by the Free Software Foundation; either version 2 of
%     the License, or (at your option) any later version.
%
%     The Code_Saturne Kernel is distributed in the hope that it will be
%     useful, but WITHOUT ANY WARRANTY; without even the implied warranty
%     of MERCHANTABILITY or FITNESS FOR A PARTICULAR PURPOSE.  See the
%     GNU General Public License for more details.
%
%     You should have received a copy of the GNU General Public License
%     along with the Code_Saturne Kernel; if not, write to the
%     Free Software Foundation, Inc.,
%     51 Franklin St, Fifth Floor,
%     Boston, MA  02110-1301  USA
%
%-----------------------------------------------------------------------
%

\programme{recvmc}

\vspace{1cm}
%%%%%%%%%%%%%%%%%%%%%%%%%%%%%%%%%%
%%%%%%%%%%%%%%%%%%%%%%%%%%%%%%%%%%
\section{Fonction}
%%%%%%%%%%%%%%%%%%%%%%%%%%%%%%%%%%
%%%%%%%%%%%%%%%%%%%%%%%%%%%%%%%%%%
Le but de ce sous-programme est de calculer la vitesse au centre des cellules
\`a partir du flux de masse aux faces, par moindres carr\'es. Utilis\'ee apr\`es
l'\'etape de correction de pression ({\it cf.}~\fort{navsto}) cette m\'ethode est une
alternative \`a la technique de reconstruction \`a partir du gradient de
l'incr\'ement de pression (technique standard).
Elle est activ\'ee quand l'indicateur \var{IREVMC} vaut~1 ou 2.

On rappelle que, � la fin de l'�tape de correction de pression, le flux de masse aux faces vaut :
\begin{equation}
(\rho \vect{u})^{n+1}_{\,ij}\text{.}\vect{S}_{\,ij} =
(\rho
\vect{\widetilde{u}})^{n+1}_{\,ij}.\,\vect{S}_{\,ij}
-\vect{D}_{\,ij}(\Delta t^n,\delta P^{n+\theta})
+\text{RC}_{\,ij}
\end{equation}
o� $\vect{\widetilde{u}}$ est la vitesse issue de l'�tape de pr�diction, $D_{\,ij}$ un op�rateur de gradient aux faces
et $\text{RC}_{\,ij}$ le terme d'Arakawa (cf. \fort{navsto} pour une d�finition pr�cise des notations).
Une premi�re m�thode, activ�e par \var{IREVMC} = 2, consiste � partir directement de
$(\rho \vect{u})^{n+1}_{\,ij}\text{.}\vect{S}_{\,ij}$ pour calculer $\vect{u}^{n+1}$ par moindres carr�s. Son utilisation a
montr\'e qu'elle semblait plus diffusive que la m\'ethode standard (par exemple, dans le cas de la cavit\'e entra\^\i n\'ee)
et pouvait conduire � des r�sultats erron�s sur des maillages ne comportant pas uniquement des t�tra�dres
(ou des prismes � base triangulaire en ``2D'') et des pav�s (hexa�dres orthogonaux).\\
On note que, dans la m�thode ci-dessus, on est parti d'une vitesse $\vect{\widetilde{u}}$ au centre des cellules, qu'on a projet�e aux faces pour obtenir le flux de masse, et qu'on ram�ne au centre des cellules par moindres carr�s. Fort de cette constatation, une m�thode alternative est disponible, activ�e par \var{IREVMC} = 1. Elle consiste � n'appliquer la m�thode des moindres carr�s qu'� la partie $-\vect{D}_{\,ij}(\Delta t^n,\delta P^{n+\theta}) +\text{RC}_{\,ij}$ du flux de masse et � rajouter directement
$\vect{\widetilde{u}}$ (connu au centre des cellules) au r�sultat obtenu\footnote{cette derni�re �tape est faite dans
\fort{navsto}.}. Cette m�thode donne des r�sultats sensiblement meilleurs.


