%                      Code_Saturne version 1.3
%                      ------------------------
%
%     This file is part of the Code_Saturne Kernel, element of the
%     Code_Saturne CFD tool.
%
%     Copyright (C) 1998-2007 EDF S.A., France
%
%     contact: saturne-support@edf.fr
%
%     The Code_Saturne Kernel is free software; you can redistribute it
%     and/or modify it under the terms of the GNU General Public License
%     as published by the Free Software Foundation; either version 2 of
%     the License, or (at your option) any later version.
%
%     The Code_Saturne Kernel is distributed in the hope that it will be
%     useful, but WITHOUT ANY WARRANTY; without even the implied warranty
%     of MERCHANTABILITY or FITNESS FOR A PARTICULAR PURPOSE.  See the
%     GNU General Public License for more details.
%
%     You should have received a copy of the GNU General Public License
%     along with the Code_Saturne Kernel; if not, write to the
%     Free Software Foundation, Inc.,
%     51 Franklin St, Fifth Floor,
%     Boston, MA  02110-1301  USA
%
%-----------------------------------------------------------------------
%
\programme{covofi}
%
\vspace{1cm}
%%%%%%%%%%%%%%%%%%%%%%%%%%%%%%%%%%
%%%%%%%%%%%%%%%%%%%%%%%%%%%%%%%%%%
\section{Fonction}
%%%%%%%%%%%%%%%%%%%%%%%%%%%%%%%%%%
%%%%%%%%%%%%%%%%%%%%%%%%%%%%%%%%%%
Dans ce sous-programme, on r�sout : \\
{\tiny$\bigstar$} soit l'�quation de convection-diffusion d'un scalaire en
pr�sence de termes sources :
%\begin{equation}\label{Base_Covofi_EQ_cvd)
\begin{equation}
\frac {\partial  (\rho a)}{\partial t} +
\underbrace{\,\dive\,((\rho \underline{u})\,a)}_{\text{convection}}
- \underbrace{\,\dive\,(K \grad a)}_{\text{diffusion}} = T_s^{\,imp} a
+T_s^{\,exp} +\Gamma\,a_i
\end{equation}
Ici $a$ repr�sente la valeur instantan�e du scalaire en approche laminaire ou,
en approche RANS, sa moyenne de Reynolds $\widetilde{a}$. Les deux approches
�tant exclusives et les �quations obtenues similaires, on utilisera le plus
souvent aussi la notation $a$ pour $\widetilde{a}$.\\
{\tiny$\bigstar$} soit, dans le cas d'une mod�lisation RANS, la variance de la
fluctuation d'un scalaire en pr�sence de termes sources\footnote{Davroux A. et
Archambeau F. : Calcul de la variance des fluctuations
d'un scalaire dans le solveur commun. Application � l'exp�rience du CEGB dite
``Jet in Pool'', HE-41/99/043.} :
\begin{equation}
\begin{array}{lcl}
&\displaystyle
 \frac {\partial  (\rho \widetilde{{a"}^2})}{\partial t} +
\underbrace{\dive\,((\rho\,\underline{u})\ \widetilde{{a"}^2})}_{\text{convection}}
- \underbrace{\dive\,(K\ \grad \widetilde{{a"}^2})}_{\text{diffusion}} = T_s^{\,imp} \widetilde{{a"}^2}
+T_s^{\,exp} +\ \Gamma\,\widetilde{{a"}^2}_i \\
&\displaystyle \underbrace {+ 2\,\frac{\mu_t}{\sigma_t}(\grad \widetilde{a})^2 -
\frac{\rho\,\varepsilon}{R_f k}\ \widetilde{{a"}^2}}_{\text{termes de production et
de dissipation dus � la turbulence moyenne}}
\end{array}
\end{equation}
$\widetilde{{a"}^2}$ repr\'esente ici la moyenne du carr\'e des fluctuations\footnote{$a$ et
$\widetilde{{a"}^2}$, sous forme discr\`ete en espace, correspondent donc en
fait \`a des vecteurs dimensionn\'es \`a \var{NCELET} de composantes $a_I$ et $\widetilde{{a"}^2}_{I}$
respectivement, I d\'ecrivant l'ensemble des cellules.} de $a$. $K$, $\Gamma$,
$T_s^{imp}$ et  $T_s^{exp}$ repr�sentent respectivement le coefficient de
diffusion, la valeur du terme source de masse, les termes sources implicite et
explicite du scalaire $a$ ou $\widetilde{{a"}^2}$. $\mu_t$ et $\sigma_t$
sont respectivement la viscosit� turbulente et le nombre de Schmidt ou de
Prandtl turbulent, $\varepsilon$ est la dissipation de l'�nergie turbulente $k$
et $R_f$ d�finit le rapport constant entre les �chelles dissipatives de $k$ et
de $\widetilde{{a"}^2}$ ($R_f$ est constant selon le mod�le assez simple adopt� ici).\\
On �crit les deux �quations pr�c�dentes sous la forme commune suivante~:
\begin{equation}
\frac {\partial  (\rho f)}{\partial t} + \dive\,((\rho\,\underline{u}) f)
- \dive\,(K \grad f) = T_s^{\,imp} f + T_s^{\,exp} + \Gamma\,f_i + T_s^{\,pd}
\end{equation}
avec, pour $f=a$ ou $f=\widetilde{{a"}^2}$ :\\
\begin{equation}
\begin{array}{lll}
&\displaystyle
T_s^{\,pd}=
\begin{cases}
0 & \text{pour $f=a$}, \\
2\ \displaystyle \frac{\mu_t}{\sigma_t}(\grad \widetilde{a})^2 -
\displaystyle \frac{\rho\,\varepsilon}{R_f k}\ \widetilde{{a"}^2} & \text{pour $f=\widetilde{{a"}^2}$ }
\end{cases}
\end{array}
\end{equation}

Le terme $\displaystyle \frac {\partial  (\rho f)}{\partial t}$ est d�compos� de la sorte :
\begin{equation}
\frac {\partial  (\rho f)}{\partial t}=\rho \frac {\partial f}{\partial t} + f
\frac {\partial \rho}{\partial t}
\end{equation}
En utilisant l'�quation de conservation de la masse (cf. \fort{preduv}),
l'�quation pr�c�dente s'�crit finalement :\\
\begin{equation}\label{Base_Covofi_Eq_cv_scal}
\rho\ \displaystyle \frac {\partial f}{\partial t} +
\dive\,((\rho\,\underline{u})\,f) - \dive\,(K\ \grad f)
= T_s^{\,imp} f + T_s^{\,exp} + \Gamma (f_i - f) + T_s^{\,pd} + f\,\dive\,(\rho\,\underline{u})
\end{equation}
