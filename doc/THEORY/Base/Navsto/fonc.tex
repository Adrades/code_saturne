%                      Code_Saturne version 1.3
%                      ------------------------
%
%     This file is part of the Code_Saturne Kernel, element of the
%     Code_Saturne CFD tool.
% 
%     Copyright (C) 1998-2007 EDF S.A., France
%
%     contact: saturne-support@edf.fr
% 
%     The Code_Saturne Kernel is free software; you can redistribute it
%     and/or modify it under the terms of the GNU General Public License
%     as published by the Free Software Foundation; either version 2 of
%     the License, or (at your option) any later version.
% 
%     The Code_Saturne Kernel is distributed in the hope that it will be
%     useful, but WITHOUT ANY WARRANTY; without even the implied warranty
%     of MERCHANTABILITY or FITNESS FOR A PARTICULAR PURPOSE.  See the
%     GNU General Public License for more details.
% 
%     You should have received a copy of the GNU General Public License
%     along with the Code_Saturne Kernel; if not, write to the
%     Free Software Foundation, Inc.,
%     51 Franklin St, Fifth Floor,
%     Boston, MA  02110-1301  USA
%
%-----------------------------------------------------------------------
%


\programme{navsto}

\vspace{1cm}
On s'int\'eresse \`a la r\'esolution du syst\`eme d'\'equations de Navier-Stokes
tridimensionnelles monophasiques, \`a une pression, instationnaires, en
incompressible ou faiblement dilatable, bas\'ees sur une discr\'etisation
temporelle de type Euler implicite d'ordre 1 ou Crank-Nicolson d'ordre 2 et sur
une discr\'etisation spatiale  par volumes finis colocalis\'ee. 


%%%%%%%%%%%%%%%%%%%%%%%%%%%%%%%%%%
%%%%%%%%%%%%%%%%%%%%%%%%%%%%%%%%%%
\section{Fonction}
%%%%%%%%%%%%%%%%%%%%%%%%%%%%%%%%%%
%%%%%%%%%%%%%%%%%%%%%%%%%%%%%%%%%%

  Dans ce sous-programme sont calcul\'ees, \`a un pas de temps donn\'e, les
variables vitesse et pression de ce probl\`eme en proc\'edant en
deux  \'etapes issues d'une d\'ecomposition des op\'erateurs (m\'ethode \`a
pas fractionnaires).\\
Les variables sont donc suppos\'ees connues \`a
l'instant ${t^n}$ et on cherche \`a les d\'eterminer \`a l'instant\footnote{La pression est suppos�e connue � l'instant $t^{n-1+\theta}$ et recherch�e en $t^{n+\theta}$, avec $\theta=1$ ou $1/2$ suivant le sch�ma en temps consid�r�.} ${t^{n+1}}$. Soit ${\Delta {t^n} ={t^{n+1}- {t^n}}}$ le pas de temps associ\'e. Dans un premier temps, on r\'ealise l'\'etape de
pr\'ediction de la vitesse en r\'esolvant l'\'equation de quantit\'e de
mouvement avec une pression explicite. Suit l'\'etape de correction de la
pression (ou projection de la vitesse) qui permet d'obtenir un champ de vitesse \`a divergence nulle.\\\\
Les \'equations en continu sont donc :   
\begin{equation}
\left\{\begin{array}{l}
\displaystyle\frac{\partial}{\partial t}(\rho \vect{u}) + \dive(\rho\, \vect{u} \otimes \vect{u})
=\dive(\tens{\sigma}) + \vect{TS} - \tens{K}\,\vect{u}\\ 
\dive(\rho \vect{u}) = \Gamma
\end{array}\right.
\end{equation}

%(plus tard $\frac{\partial \rho}{\partial t} + \dive(\rho \vect{u}) = \Gamma$)



avec $\rho$ la masse volumique, $\vect{u}$ le champ de vitesse, 
$[\,\vect{TS}-\tens{K}\,\vect{u}\,]$ les autres termes sources ($\tens{K}$~est un 
tenseur diagonal positif par d\'efinition), $\tens{\sigma}$ le tenseur
de contraintes, $\tens{\tau}$ le tenseur des contraintes visqueuses, $\mu$ la
viscosit\'e dynamique (mol\'eculaire et \'eventuellement turbulente), $\kappa$
la viscosit� de 
volume (usuellement nulle et n�glig�e dans le code et dans la suite du document,
sauf en compressible),
$\tens{D}$ le tenseur taux de d\'eformation\footnote{\`A ne pas confondre, malgr\'e la m\^eme notation $D$,
avec les flux diffusifs $\vect{D}_{\,ij}$ et $\vect{D}_{\,{b}_{ik}}$ d\'ecrits par la suite dans ce
sous-programme.}, $\Gamma$ le terme source de masse.
\begin{equation}
\left\{\begin{array}{l}
\tens{\sigma} = \tens{\tau} - P\tens{Id}  \\
\tens{\tau} = 2\,\mu\ \tens{D} +\ (\kappa\ - \frac{2}{3}\mu)\  tr({\tens{D}})\
\tens{Id}  \\
\tens{D} = \frac{1}{2}(\ggrad\vect{u}+\,^{t}\ggrad\vect{u})
\end{array}\right.
\end{equation}
 \\

On rappelle la d\'efinition des notations employ\'ees\footnote{en
utilisant la convention de sommation d'Einstein.}~:
\begin{equation}\notag
\left\{\begin{array}{lll}
\left[\ggrad{\vect{a}}\right]_{ij} &=& \partial_j a_i\\
\left[\dive(\tens{\sigma})\right]_i &=& \partial_j \sigma_{ij}\\
\left[\vect{a}\otimes\vect{b}\right]_{ij} &= &
a_i\,b_j\\
\end{array}\right.
\end{equation}
et donc :
\begin{equation}\notag
\begin{array}{lll}
\left[\dive(\vect{a}\otimes\vect{b})\right]_i &= &
\partial_j (a_i\,b_j)
\end{array} 
\end{equation}

\minititre{Remarque}
Dans le cas de la prise en compte d'une masse volumique variable, l'�quation de continuit� s'�crit :
$$\frac{\partial \rho}{\partial t} + \dive{\,(\rho\,\vect{u})} = \Gamma  $$
Cette �quation n'est pas prise en compte dans l'�tape de projection (on continue � r�soudre
seulement
$\displaystyle \dive(\,{\rho\,\vect{u}}) = \Gamma$) alors que le terme
$\displaystyle \frac{\partial \rho}{\partial t}$ appara\^{\i}t lors de l'�tape de pr\'ediction de la vitesse
dans le sous-programme \fort{preduv}. Si ce terme joue un r�le sensible, l'algorithme compressible
de \CS\ (qui r�sout l'�quation compl�te) est alors sans doute plus adapt�.
