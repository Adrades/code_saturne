%                      Code_Saturne version 1.3
%                      ------------------------
%
%     This file is part of the Code_Saturne Kernel, element of the
%     Code_Saturne CFD tool.
% 
%     Copyright (C) 1998-2007 EDF S.A., France
%
%     contact: saturne-support@edf.fr
% 
%     The Code_Saturne Kernel is free software; you can redistribute it
%     and/or modify it under the terms of the GNU General Public License
%     as published by the Free Software Foundation; either version 2 of
%     the License, or (at your option) any later version.
% 
%     The Code_Saturne Kernel is distributed in the hope that it will be
%     useful, but WITHOUT ANY WARRANTY; without even the implied warranty
%     of MERCHANTABILITY or FITNESS FOR A PARTICULAR PURPOSE.  See the
%     GNU General Public License for more details.
% 
%     You should have received a copy of the GNU General Public License
%     along with the Code_Saturne Kernel; if not, write to the
%     Free Software Foundation, Inc.,
%     51 Franklin St, Fifth Floor,
%     Boston, MA  02110-1301  USA
%
%-----------------------------------------------------------------------
%

\programme{matrix}

\vspace{1cm}
%%%%%%%%%%%%%%%%%%%%%%%%%%%%%%%%%%
%%%%%%%%%%%%%%%%%%%%%%%%%%%%%%%%%%
\section{Fonction}
%%%%%%%%%%%%%%%%%%%%%%%%%%%%%%%%%%
%%%%%%%%%%%%%%%%%%%%%%%%%%%%%%%%%%

Le but de ce sous-programme, appel\'e par \fort{codits} et \fort{covofi}, est de construire la
matrice de convection/diffusion, incluant les contributions ad\'equates des termes sources,
intervenant dans le membre de gauche d'\'equations discr\'etis\'ees telles que
celle de la
quantit\'e de mouvement, d'une \'equation de convection diffusion d'un scalaire
ou de mod\`ele de turbulence.\\
Le type d'\'equation consid\'er\'ee est, pour la variable scalaire $a$ :
\begin{equation}\notag
\displaystyle \frac{\partial a}{\partial t} + \dive (\,(\rho \vect{u})\, a) -
\displaystyle \frac{\partial }{\partial x}\left(\beta\,\frac{\partial a}{\partial x}\right) = 0
\end{equation}
La matrice ne s'applique qu'aux termes non reconstruits, les autres \'etant pris en compte au second membre et
trait\'es dans le sous-programme \fort{bilsc2}. La partie
convective, lorsqu'elle existe, est issue du sch\'ema upwind pur, quelque soit
le type de sch\'ema convectif choisi par l'utilisateur. En effet, c'est, \`a
l'heure actuelle, la seule fa\c con d'obtenir un op\'erateur lin\'eaire �
diagonale dominante.\\\\ 
La matrice est donc associ\'ee \`a $\mathcal{EM_{\it{scal}}}$, op\'erateur
agissant sur un scalaire $a$ (inspir\'e de celui vectoriel $\mathcal{EM}$
d\'efini dans \fort{navsto}) d'expression actuelle, pour tout cellule $\Omega_i$ de
centre $I$  :
\begin{equation}\notag
\begin{array}{lll}
\mathcal{EM_{\it{scal}}}(a,I) &=  f_s^{imp}\ a_I\ \\ 
&+\sum\limits_{j\in Vois(i)}{F^{\,amont}_{\,ij}((\rho\vect{u})^n,a)}  
+\sum\limits_{k\in {\gamma_b(i)}} {F^{\,amont}_{\,{b}_{ik}}((\rho
\vect{u})^n,a)}\\
&-\sum\limits_{j\in Vois(i)}{D^{NRec}_{\,ij}(\beta,a)}
-\sum\limits_{k\in {\gamma_b(i)}} {D^{NRec}_{\,{b}_{ik}}(\beta,a)}\\ 
\end{array}
\end{equation}
avec~:\\
$\bullet$ $f_s^{imp}$ le coefficient issu du terme instationnaire
$\displaystyle\frac{\rho \ |\Omega_i|}{\Delta t}$, s'il y a lieu, et de
l'implicitation de certains termes sources (contribution d\'ecoulant de la prise
en compte de la
variation $\displaystyle\frac{\partial \rho }{\partial t}$ de
la masse volumique $\rho$ au cours du temps, diagonale du tenseur de pertes de
charges par exemple...): cette initialisation est en fait effectu\'ee en amont
de ce sous-programme, \\
$\bullet$ $F^{\,amont}_{\,ij}$ le flux num\'erique convectif scalaire
d\'ecentr\'e amont calcul\'e \`a la face interne $ij$ de la cellule $\Omega_i$,\\
$\bullet$ $F^{\,amont}_{\,b_{ik}}$ le flux num\'erique convectif scalaire
d\'ecentr\'e amont associ\'e calcul\'e \`a la face de bord $ik$ de la cellule $\Omega_i$
jouxtant le bord du domaine $\Omega$,\\
$\bullet$ $D^{\,NRec}_{\,ij}$ le flux num\'erique diffusif scalaire non
reconstruit associ\'e calcul\'e \`a la face interne $ij$ de la cellule $\Omega_i$,\\
$\bullet$ $D^{\,NRec}_{\,{b}_{ik}}$ le flux num\'erique diffusif scalaire non
reconstruit associ\'e calcul\'e \`a la face de bord $ik$ de la cellule $\Omega_i$ jouxtant le bord du domaine $\Omega$,\\
$\bullet$ $Vois(i)$ repr\'esente toujours l'ensemble des cellules
voisines de ${\Omega_i}$ et $\gamma_b(i)$ l'ensemble des faces de
bord de ${\Omega_i}$.\\
Du fait de la r\'esolution en incr\'ements, $a$ est un incr\'ement et ses
conditions aux limites associ\'ees sont donc de type Dirichlet homog\`ene ou de
type 
Neumann homog\`ene.

