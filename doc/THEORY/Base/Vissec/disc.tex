%                      Code_Saturne version 1.3
%                      ------------------------
%
%     This file is part of the Code_Saturne Kernel, element of the
%     Code_Saturne CFD tool.
%
%     Copyright (C) 1998-2007 EDF S.A., France
%
%     contact: saturne-support@edf.fr
%
%     The Code_Saturne Kernel is free software; you can redistribute it
%     and/or modify it under the terms of the GNU General Public License
%     as published by the Free Software Foundation; either version 2 of
%     the License, or (at your option) any later version.
%
%     The Code_Saturne Kernel is distributed in the hope that it will be
%     useful, but WITHOUT ANY WARRANTY; without even the implied warranty
%     of MERCHANTABILITY or FITNESS FOR A PARTICULAR PURPOSE.  See the
%     GNU General Public License for more details.
%
%     You should have received a copy of the GNU General Public License
%     along with the Code_Saturne Kernel; if not, write to the
%     Free Software Foundation, Inc.,
%     51 Franklin St, Fifth Floor,
%     Boston, MA  02110-1301  USA
%
%-----------------------------------------------------------------------
%

%%%%%%%%%%%%%%%%%%%%%%%%%%%%%%%%%%
%%%%%%%%%%%%%%%%%%%%%%%%%%%%%%%%%%
\section{Discr\'etisation}
%%%%%%%%%%%%%%%%%%%%%%%%%%%%%%%%%%
%%%%%%%%%%%%%%%%%%%%%%%%%%%%%%%%%%
L'int�gration des termes de gradient transpos�
 $\dive\,(\mu_{\,tot}\,^t\,\ggrad(\underline{v}))$ et de viscosit�
secondaire \\
$-\displaystyle \frac{2}{3}\,\grad (\mu_{\,tot}\,\dive(\underline{v}))$ est la suivante\footnote{la viscosit� de volume
$\kappa$ est suppos�e nulle, cf. \fort{navsto}} :
\begin{equation}\notag
\begin{array}{llll}
&\displaystyle \int_{\Omega_i}\dive (\mu_{\,tot}
\,^t\,\ggrad(\underline{v})\,)\,d\Omega \\
&= \sum\limits_{l=x,y,z}\left[ \sum\limits_{j \in
Vois(i)} \mu_{\,tot,ij} ((\displaystyle \frac {\partial v_x}{\partial
l})_{\,moy,ij}\,n_{\,ij}^{\,x} +(\frac {\partial v_y}
 {\partial l})_{\,moy,ij}\,n_{\,ij}^{\,y}+ \displaystyle(\frac {\partial v_z}
{\partial l})_{\,moy,ij}\,n_{\,ij}^{\,z}) S_{\,ij}\right.\\
&+\left.\displaystyle \sum\limits_{k \in \gamma_b(i)} \mu_{\,tot,\,b_{ik}}
((\frac {\partial v_x}{\partial l})_{\,moy,\,b_{ik}}\,n_{\,b_{ik}}^{\,x} +(\frac
{\partial v_y}{\partial l})_{\,moy,\,b_{ik}}\,n_{\,b_{ik}}^{\,y}+
\displaystyle(\frac {\partial v_z}{\partial
l})_{\,moy,\,b_{ik}}\,n_{\,b_{ik}}^{\,z}) S_{\,b_{ik}}\right] \underline
{e}_{\,l} \\\\
&-\displaystyle \frac{2}{3}\,\int_{\Omega_i}\grad (\mu_{\,tot} \dive\,(\underline{v}))
\,d\Omega \\
&= -\displaystyle \frac{2}{3}\,\sum\limits_{l=x,y,z}\left[\sum\limits_{j \in Vois(i)}(\mu_{\,tot}\,\dive(\underline{v}))_{\,ij} S_{\,ij}^{\,l} + \displaystyle \sum_{k \in \gamma_b(i)}(\mu_{\,tot}\,\dive(\underline{v}))_{\,b_{ik}} S_{\,b_{ik}}^{\,l}\right]\,\underline {e}_{\,l}
\end{array}
\end{equation}
Le terme de viscosit� $\mu_{\,tot}$ est calcul� en fonction du mod�le de turbulence utilis� :
\begin{equation}\notag
\mu_{\,tot}=
\begin{cases}
\mu + \mu_t & \text{pour les mod�les � viscosit� turbulente ou en LES}, \\
\mu& \text{pour les mod�les au second ordre ou en laminaire}.
\end{cases}
\end{equation}
o� $\mu$ et $\mu_t$ repr\'esentent respectivement la viscosit\'e dynamique mol�culaire et la viscosit� dynamique turbulente.
