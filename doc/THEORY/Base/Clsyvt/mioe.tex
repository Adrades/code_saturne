%                      Code_Saturne version 1.3
%                      ------------------------
%
%     This file is part of the Code_Saturne Kernel, element of the
%     Code_Saturne CFD tool.
%
%     Copyright (C) 1998-2007 EDF S.A., France
%
%     contact: saturne-support@edf.fr
%
%     The Code_Saturne Kernel is free software; you can redistribute it
%     and/or modify it under the terms of the GNU General Public License
%     as published by the Free Software Foundation; either version 2 of
%     the License, or (at your option) any later version.
%
%     The Code_Saturne Kernel is distributed in the hope that it will be
%     useful, but WITHOUT ANY WARRANTY; without even the implied warranty
%     of MERCHANTABILITY or FITNESS FOR A PARTICULAR PURPOSE.  See the
%     GNU General Public License for more details.
%
%     You should have received a copy of the GNU General Public License
%     along with the Code_Saturne Kernel; if not, write to the
%     Free Software Foundation, Inc.,
%     51 Franklin St, Fifth Floor,
%     Boston, MA  02110-1301  USA
%
%-----------------------------------------------------------------------
%

%%%%%%%%%%%%%%%%%%%%%%%%%%%%%%%%%%
%%%%%%%%%%%%%%%%%%%%%%%%%%%%%%%%%%
\section{Mise en \oe uvre}
%%%%%%%%%%%%%%%%%%%%%%%%%%%%%%%%%%
%%%%%%%%%%%%%%%%%%%%%%%%%%%%%%%%%%
\label{Base_Clsyvt_prg_meo}%
\etape{D\'ebut de boucle}
D\'ebut de boucle sur toutes les faces de bord \var{IFAC} avec des conditions de
sym\'etrie. La face est consid\'er\'ee comme une face de sym\'etrie quand
\var{ICODCL(IFAC,IU(IPHAS))} vaut 4, sachant que les tests dans \fort{vericl}
sont tels que \var{ICODCL} vaut 4 pour \var{IU} si et seulement si il vaut aussi
4 pour les autres composantes de la vitesse et les composantes de $\tens{R}$ (le
cas \'ech\'eant).\\
La valeur 0 est alors affect\'ee \`a \var{ISYMPA}, ce qui identifie la face
comme une face de paroi ou de sym\'etrie, c'est-\`a-dire o\`u le flux de masse
sera forc\'e \`a 0 (cf. \fort{inimas}).

\etape{Calcul des vecteurs de base}
La normale $\vect{n}$ est stock\'ee dans \var{(RNX,RNY,RNZ)}.\\
$\vect{u}_{I'}$, calcul\'e dans \fort{CONDLI} et pass\'e {\em via} \var{COEFU}
est stock\'e dans \var{(UPX,UPY,UPZ)}.

\etape{Cas du $R_{ij}-\varepsilon$}
Dans le cas o\`u on est en $R_{ij}-\varepsilon$ (\var{ITURB}=30 ou 31),
alors les
vecteurs $\vect{t}$ et $\vect{b}$ doivent \^etre calcul\'es explicitement
(on utilise $\matt{P}$ et pas simplement $\matt{A}$).
Ils sont stock\'es respectivement dans \var{(TX,TY,TZ)} et
\var{(T2X,T2Y,T2Z)}.\\
La matrice de passage $\matt{P}$ est ensuite calcul\'ee et stock\'ee dans le
tableau \var{ELOGLO}.\\
On appelle alors le sous-programme \fort{clca66} qui calcule la matrice
r\'eduite $\matt{D}$, qu'il stocke dans \var{ALPHA}. \fort{clca66} est appel\'e
avec un param\`etre \var{CLSYME} qui vaut 1 et qui correspond au param\`etre
$\omega$ de l'\'equation \ref{Base_Clsyvt_eq_clRij}.


\etape{Remplissage des tableaux \var{COEFA} et \var{COEFB}}
On remplit les tableaux \var{COEFA} et \var{COEFB} en suivant directement les
\'equations \ref{Base_Clsyvt_eq_clU}, \ref{Base_Clsyvt_eq_clRimp} et \ref{Base_Clsyvt_eq_clRexp}.\\
\var{RIJIPB(IFAC,.)} correspond au vecteur $\mat{S}_{I'}^\prime$, calcul\'e dans
\fort{condli} et pass\'e en argument \`a \var{clsyvt}.

\etape{Remplissage des tableaux \var{COEFAF} et \var{COEFBF}}
Dans le cas o\`u ils sont d\'efinis, les tableaux \var{COEFAF} et \var{COEFBF}
sont remplis. Ils contiennent les m\^emes valeurs que \var{COEFA} et
\var{COEFB}.


%%%%%%%%%%%%%%%%%%%%%%%%%%%%%%%%%%
%%%%%%%%%%%%%%%%%%%%%%%%%%%%%%%%%%
\section{Annexe A}
%%%%%%%%%%%%%%%%%%%%%%%%%%%%%%%%%%
%%%%%%%%%%%%%%%%%%%%%%%%%%%%%%%%%%
\minititre{D\'emonstration de l'orthogonalit\'e de la matrice $\matt{A}$}

On conserve toutes les notations du paragraphe 2. On a :
\begin{eqnarray}
(^t\matt{A}\,.\,\matt{A})_{ij}
& = & \sum_{k=1}^9\comp{A}_{ki}\comp{A}_{kj}\nonumber\\
& = & \sum_{k=1}^9
\comp{P}_{q(k)q(i)}\comp{P}_{r(k)r(i)}\comp{P}_{q(k)q(j)}\comp{P}_{r(k)r(j)}
\end{eqnarray}
Or, quand $k$ varie de 1 \`a 3, $q(k)$ reste \'egal \`a 1 et $r(k)$ varie de 1
\`a 3. On a donc :
\begin{eqnarray}
\sum_{k=1}^3
\comp{P}_{q(k)q(i)}\comp{P}_{r(k)r(i)}\comp{P}_{q(k)q(j)}\comp{P}_{r(k)r(j)}
&=&\comp{P}_{1q(i)}\comp{P}_{1q(j)}\sum_{k=1}^3
\comp{P}_{r(k)r(i)}\comp{P}_{r(k)r(j)}\nonumber\\
&=&\comp{P}_{1q(i)}\comp{P}_{1q(j)}\sum_{k=1}^3
\comp{P}_{kr(i)}\comp{P}_{kr(j)}\\
&=&\comp{P}_{1q(i)}\comp{P}_{1q(j)}\delta_{r(i)r(j)}\qquad\text{(par
orthogonalit\'e de $\matt{P}$)}\nonumber
\end{eqnarray}
On fait de m\^eme pour $k$ variant de 4 \`a 6 ou de 7 \`a 9, $q(k)$ valant alors
respectivement 2 ou 3. On obtient alors :
\begin{eqnarray}
(^t\matt{A}\,.\,\matt{A})_{ij}
&=&
\sum_{k=1}^9
\comp{P}_{q(k)q(i)}\comp{P}_{r(k)r(i)}\comp{P}_{q(k)q(j)}\comp{P}_{r(k)r(j)}
\nonumber\\
&=&
\sum_{k=1}^3\comp{P}_{kq(i)}\comp{P}_{kq(j)}\delta_{r(i)r(j)}\\
&=&\delta_{q(i)q(j)}\delta_{r(i)r(j)}\nonumber\\
&=&\delta_{ij}\qquad
\text{(par bijectivit\'e de $(q,r)$)}\nonumber
\end{eqnarray}

Donc $^t\matt{A}\,.\,\matt{A}=\matt{Id}$. De m\^eme, on montre que
$\matt{A}\,.\,^t\matt{A}=\matt{Id}$. $\matt{A}$ est donc bien une matrice
orthogonale.


%%%%%%%%%%%%%%%%%%%%%%%%%%%%%%%%%%
%%%%%%%%%%%%%%%%%%%%%%%%%%%%%%%%%%
\section{Annexe B}
%%%%%%%%%%%%%%%%%%%%%%%%%%%%%%%%%%
%%%%%%%%%%%%%%%%%%%%%%%%%%%%%%%%%%
\minititre{Calcul de la matrice $\matt{D}$}

On conna\^\i t la relation liant les matrices de dimension $9\times1$
des composantes de $\tens{R}$ dans le rep\`ere $\mathcal{R}$ en $F$ et en $I'$
(matrices $\mat{S}_F$ et $\mat{S}_{I'}$) :
\begin{equation}
\mat{S}_F=\matt{C}\,.\,\mat{S}_{I'}
\end{equation}
avec
\begin{equation}
\comp{C}_{ij}=\sum_{k=1}^9
\comp{P}_{q(i)q(k)}\comp{P}_{r(i)r(k)}\comp{P}_{q(j)q(k)}\comp{P}_{r(j)r(k)}
(\delta_{k1}+\omega\delta_{k3}+\delta_{k5}+\omega\delta_{k7}+\delta_{k9})
\end{equation}

Pour passer de $\mat{S}$ \`a la matrice r\'eduite $6\times 1$ $\mat{S}^\prime$,
on introduit l'application $s$ de $\{1,2,3,4,5,6,7,8,9\}$ dans
$\{1,2,3,4,5,6\}$ prenant les valeurs suivantes :
\begin{center}
\begin{tabular}{|c|c|c|c|c|c|c|c|c|c|}
\hline
$i$&1&2&3&4&5&6&7&8&9\\
\hline
$s(i)$&1&4&5&4&2&6&5&6&3\\
\hline
\end{tabular}
\end{center}
Par construction, on a $\comp{S}_i=\comp{S}^\prime_{s(i)}$ pour tout $i$ entre 1
et 9.

Pour calculer $\comp{D}_{ij}$, on peut choisir une valeur $m$ telle que
$s(m)=i$ et sommer tous les $\comp{C}_{mn}$ tels que $s(n)=j$. Le choix de $m$
est indiff\'erent. De mani\`ere plus sym\'etrique, on peut aussi sommer sur tous
les $m$ tels que $s(m)=i$ et diviser par le nombre de telles valeurs de $m$. C'est
cette derni\`ere m\'ethode que nous allons utiliser.

On d\'efinit $N(i)$ le nombre d'entiers entre 1 et 9 tels que
$s(m)=i$. D'apr\`es ce qui pr\'ec\`ede, on a donc :

\begin{eqnarray}
\comp{D}_{ij}&=&\frac{1}{N(i)}\sum_{s(m)=i \atop s(n)=j}\comp{C}_{mn}\nonumber\\
&=&\frac{1}{N(i)}\sum_{{s(m)=i \atop s(n)=j}\atop 1\leqslant k\leqslant 9}
\comp{P}_{q(m)q(k)}\comp{P}_{r(m)r(k)}\comp{P}_{q(n)q(k)}\comp{P}_{r(n)r(k)}
(\delta_{k1}+\omega\delta_{k3}+\delta_{k5}+\omega\delta_{k7}+\delta_{k9})
\end{eqnarray}

\vspace{1cm}
$\bullet\ ${\sc Premier cas} : $i\leqslant 3$ et $j\leqslant 3$\\
Dans ce cas, on a forc\'ement $N(i)=N(j)=1$. De plus, si $s(m)=i$ et $s(n)=j$,
alors $q(m)=r(m)=i$ et $q(n)=r(n)=j$. Donc \\
\begin{equation}
\comp{D}_{ij}=\sum_{k=1}^9
\comp{P}_{iq(k)}\comp{P}_{ir(k)}\comp{P}_{jq(k)}\comp{P}_{jr(k)}
(\delta_{k1}+\omega\delta_{k3}+\delta_{k5}+\omega\delta_{k7}+\delta_{k9})
\end{equation}
Quand $k$ balaye $\{1,5,9\}$, $q(k)=r(k)$ balaye $\{1,2,3\}$. Et pour $k=3$ ou
$k=7$, $q(k)=1$ et $r(k)=3$, ou vice-versa (et pour $k$ pair le facteur en somme
de symboles de Kronecker est nul). On a donc finalement :
\begin{equation}
\comp{D}_{ij}=\sum_{k=1}^3\comp{P}_{ik}^2\comp{P}_{jk}^2
+2\omega\comp{P}_{j1}\comp{P}_{i3}\comp{P}_{i1}\comp{P}_{j3}
\end{equation}

\vspace{1cm}
$\bullet\ ${\sc Deuxi\`eme cas} : $i\leqslant 3$ et $j\geqslant 4$\\
On a encore $N(i)=1$, et si $s(m)=i$ alors $q(m)=r(m)=i$.\\
Par contre, on a $N(j)=2$, les deux possibilit\'es \'etant $m_1$ et $m_2$.\\
\begin{itemize}
\item[-] si $j=4$, alors $m_1=2$ et $m_2=4$,
$q(m_1)=r(m_2)=1$ et $r(m_1)=q(m_2)=2$. On pose alors
$m=1$ et $n=2$.

\item[-] si $j=5$, alors $m_1=3$ et $m_2=7$,
$q(m_1)=r(m_2)=1$ et $r(m_1)=q(m_2)=3$. On pose alors
$m=1$ et $n=3$.

\item[-] si $j=6$, alors $m_1=6$ et $m_2=8$,
$q(m_1)=r(m_2)=2$ et $r(m_1)=q(m_2)=3$. On pose alors
$m=2$ et $n=3$.
\end{itemize}

Et on a :
\begin{equation}
\comp{D}_{ij}=\sum_{k=1}^9
\comp{P}_{iq(k)}\comp{P}_{ir(k)}\left[
\comp{P}_{mq(k)}\comp{P}_{nr(k)}+\comp{P}_{nq(k)}\comp{P}_{mr(k)}\right]
(\delta_{k1}+\omega\delta_{k3}+\delta_{k5}+\omega\delta_{k7}+\delta_{k9})
\end{equation}

Or quand $k$ balaye $\{1,5,9\}$, $q(k)=r(k)$ balaye $\{1,2,3\}$. Donc :
\begin{equation}
\comp{D}_{ij}=2\sum_{k=1}^3
\comp{P}_{ik}^2\comp{P}_{mk}\comp{P}_{nk}
+\omega\sum_{k=1}^9
\comp{P}_{iq(k)}\comp{P}_{ir(k)}\left[
\comp{P}_{mq(k)}\comp{P}_{nr(k)}+\comp{P}_{nq(k)}\comp{P}_{mr(k)}\right]
(\delta_{k3}+\delta_{k7})
\end{equation}

Et pour $k=3$ ou $k=7$, $q(k)=1$ et $r(k)=3$, ou vice-versa. On a donc
finalement :
\begin{equation}
\comp{D}_{ij}=2\left[\sum_{k=1}^3
\comp{P}_{ik}^2\comp{P}_{mk}\comp{P}_{nk}
+\omega\comp{P}_{i1}\comp{P}_{i3}\left(
\comp{P}_{m1}\comp{P}_{n3}+\comp{P}_{n1}\comp{P}_{m3}\right)
\right]
\end{equation}
avec $(m,n)=(1,2)$ si $j=4$, $(m,n)=(1,3)$ si $j=5$ et $(m,n)=(2,3)$ si
$j=6$.

\vspace{1cm}
$\bullet\ ${\sc Troisi\`eme cas} : $i\geqslant 4$ et $j\leqslant 3$\\
Par sym\'etrie de $\matt{C}$, on obtient un r\'esultat sym\'etrique du
deuxi\`eme cas, sauf que $N(i)$ vaut maintenant 2. Donc :
\begin{equation}
\comp{D}_{ij}=\sum_{k=1}^3
\comp{P}_{jk}^2\comp{P}_{mk}\comp{P}_{nk}
+\omega\comp{P}_{j1}\comp{P}_{j3}\left(
\comp{P}_{m1}\comp{P}_{n3}+\comp{P}_{n1}\comp{P}_{m3}\right)
\end{equation}
avec $(m,n)=(1,2)$ si $i=4$, $(m,n)=(1,3)$ si $i=5$ et $(m,n)=(2,3)$ si
$i=6$.

\vspace{1cm}
$\bullet\ ${\sc Quatri\`eme cas} : $i\geqslant 4$ et $j\geqslant 4$\\
Alors $N(i)=N(j)=2$.\\
On pose $(m,n)=(1,2)$ si $i=4$, $(m,n)=(1,3)$ si $i=5$ et
$(m,n)=(2,3)$ si $i=6$. On fait de m\^eme pour d\'efinir $m^\prime$ et
$n^\prime$ en fonction de $j$. On a alors :
\begin{eqnarray}
\comp{D}_{ij}&=&\frac{1}{2}\sum_{k=1}^9
\left(\comp{P}_{mq(k)}\comp{P}_{nr(k)}+\comp{P}_{nq(k)}\comp{P}_{mr(k)}\right)
\left(\comp{P}_{m^\prime q(k)}\comp{P}_{n^\prime r(k)}
+\comp{P}_{n^\prime q(k)}\comp{P}_{m^\prime r(k)}\right)\nonumber\\
&&\qquad\qquad\qquad\qquad\qquad\qquad\qquad\qquad\qquad\times
(\delta_{k1}+\omega\delta_{k3}+\delta_{k5}+\omega\delta_{k7}+\delta_{k9})\nonumber\\
&=&\frac{1}{2}\left[
\sum_{k=1}^3 4\comp{P}_{mk}\comp{P}_{nk}
\comp{P}_{m^\prime k}\comp{P}_{n^\prime k}
+2\omega\left(\comp{P}_{m1}\comp{P}_{n3}+\comp{P}_{n1}\comp{P}_{m3}\right)
\left(\comp{P}_{m^\prime 1}\comp{P}_{n^\prime 3}
+\comp{P}_{n^\prime 1}\comp{P}_{m^\prime 3}\right)\right]
\end{eqnarray}

soit au final :
\begin{equation}
\comp{D}_{ij}=
2\sum_{k=1}^3 \comp{P}_{mk}\comp{P}_{nk}
\comp{P}_{m^\prime k}\comp{P}_{n^\prime k}
+\omega\left(\comp{P}_{m1}\comp{P}_{n3}+\comp{P}_{n1}\comp{P}_{m3}\right)
\left(\comp{P}_{m^\prime 1}\comp{P}_{n^\prime 3}
+\comp{P}_{n^\prime 1}\comp{P}_{m^\prime 3}\right)
\end{equation}
avec $(m,n)=(1,2)$ si $i=4$, $(m,n)=(1,3)$ si $i=5$ et $(m,n)=(2,3)$ si
$i=6$\\
et $(m^\prime ,n^\prime )=(1,2)$ si $j=4$, $(m^\prime ,n^\prime )=(1,3)$
si $j=5$ et $(m^\prime ,n^\prime )=(2,3)$ si $j=6$.


