%                      Code_Saturne version 1.3
%                      ------------------------
%
%     This file is part of the Code_Saturne Kernel, element of the
%     Code_Saturne CFD tool.
%
%     Copyright (C) 1998-2007 EDF S.A., France
%
%     contact: saturne-support@edf.fr
%
%     The Code_Saturne Kernel is free software; you can redistribute it
%     and/or modify it under the terms of the GNU General Public License
%     as published by the Free Software Foundation; either version 2 of
%     the License, or (at your option) any later version.
%
%     The Code_Saturne Kernel is distributed in the hope that it will be
%     useful, but WITHOUT ANY WARRANTY; without even the implied warranty
%     of MERCHANTABILITY or FITNESS FOR A PARTICULAR PURPOSE.  See the
%     GNU General Public License for more details.
%
%     You should have received a copy of the GNU General Public License
%     along with the Code_Saturne Kernel; if not, write to the
%     Free Software Foundation, Inc.,
%     51 Franklin St, Fifth Floor,
%     Boston, MA  02110-1301  USA
%
%-----------------------------------------------------------------------
%

%%%%%%%%%%%%%%%%%%%%%%%%%%%%%%%%%%
%%%%%%%%%%%%%%%%%%%%%%%%%%%%%%%%%%
\section{Discr\'etisation}
%%%%%%%%%%%%%%%%%%%%%%%%%%%%%%%%%%
%%%%%%%%%%%%%%%%%%%%%%%%%%%%%%%%%%
Afin d'expliquer la proc�dure utilis�e pour traiter les difficult�s dues aux
termes de reconstruction et de test de pente dans les termes de
convection-diffusion, on note, de fa\c con analogue \`a ce qui est d\'efini dans
\fort{navsto} mais sans discr\'etisation spatiale associ\'ee, $\mathcal{E}_{n}$ l'op\'erateur :
\begin{equation}\label{Base_Codits_Eq_ref_small}
\begin{array}{c}
\mathcal{E}_{n}(a) = f_s^{\,imp}\,a + \theta\,\, \dive((\rho
\underline{u})\,a) - \theta\,\, \dive(\mu_{\,tot}\,\grad a)\\
- f_s^{\,exp} -  f_s^{\,imp}\,a^{n} +(1-\theta)\,\,\dive((\rho
\underline{u})\, a^n) - (1-\theta)\,\, \dive(\mu_{\,tot}\,\grad a^n)
\end{array}
\end{equation}
L'\'equation (\ref{Base_Codits_eq_ref}) s'\'ecrit donc :
\begin{equation}
\mathcal{E}_{n}(a^{n+1}) = 0
\end{equation}
La quantit\'e  $\mathcal{E}_{n}(a^{n+1})$ comprend donc :\\
\hspace*{1.cm} $\rightsquigarrow$ $f_s^{\,imp}\,a^{n+1}$, contribution des
termes diff\'erentiels d'ordre $0$ lin\'eaire en $a^{n+1}$,\\
\hspace*{1.cm} $\rightsquigarrow$ $\theta\,\,
\dive((\rho\underline{u})\,a^{n+1})
- \theta\,\, \dive(\mu_{\,tot}\,\grad a^{n+1})$, termes de convection-diffusion
implicites complets (termes non reconstruits + termes de reconstruction)
lin\'eaires\footnote{Lors de la discr�tisation en espace, le caract�re lin�aire
de ces termes pourra cependant �tre perdu, notamment � cause du test de pente.}
en $a^{n+1}$,\\
\hspace*{1.cm} $\rightsquigarrow$ $f_s^{\,exp}- f_s^{\,imp}\,a^n$ et
$(1-\theta)\,\,\dive((\rho
\underline{u})\,a^n) - (1-\theta)\,\, \dive(\mu_{\,tot}\,\grad a^n)$ l'ensemble
des termes explicites (y compris la partie explicite provenant du sch\'ema en
temps appliqu\'e \`a la convection diffusion).\\\\

De m\^eme, on introduit un op\'erateur $\mathcal{EM}_{n}$ approch\'e de
$\mathcal{E}_{n}$, lin\'eaire et simplement inversible, tel que son
expression contient :\\
\hspace*{1.cm}$\rightsquigarrow$ la prise en compte des termes lin\'eaires en $a$,\\
\hspace*{1.cm}$\rightsquigarrow$ la convection int�gr�e par un sch�ma d�centr� amont
(upwind) du premier ordre en espace,\\
\hspace*{1.cm}$\rightsquigarrow$ les flux diffusifs non reconstruits.\\
\begin{equation}
\mathcal{EM}_{n}(a) = f_s^{\,imp}\,a + \theta\,\,[\dive((\rho
\underline{u})\,a)]^{\textit{amont}} - \theta\,\, [\dive(\mu_{\,tot}\,\grad a)]^{\textit{N Rec}}
\end{equation}
Cet op\'erateur permet donc de contourner la difficult\'e induite par la pr\'esence d'\'eventuelles non lin\'earit\'es introduites par l'activation du test de pente lors du sch\'ema convectif, et par le remplissage important de la structure de la matrice d\'ecoulant de la pr\'esence des gradients propres \`a la reconstruction.\\
On a la relation\footnote{On pourra se reporter au sous-programme
\fort{matrix} pour plus de d\'etails relativement \`a
$\mathcal{EM_{\it{disc}}}$, op\'erateur discret agissant sur un scalaire $a$.}, pour toute cellule $\Omega_I$ de centre $I$  :
\begin{equation}\notag
\mathcal{EM_{\it{disc}}}(a,I) = \int_{\Omega_i}\mathcal{EM}_{n}(a)  \, d\Omega
\end{equation}
On cherche � r�soudre :
\begin{equation}
0 =\mathcal{E}_{n}(a^{n+1}) =  \mathcal{EM}_{n}(a^{n+1}) +  \mathcal{E}_{n}(a^{n+1}) - \mathcal{EM}_{n}(a^{n+1})
\end{equation}
Soit :
\begin{equation}
\mathcal{EM}_{n}(a^{n+1}) =  \mathcal{EM}_{n}(a^{n+1}) -  \mathcal{E}_{n}(a^{n+1})
\end{equation}
On va pour cela utiliser un algorithme de type point fixe en d�finissant la
suite $(a^{n+1,\,k})_{k\in \grandN}$\footnote{Dans le cas ou le point fixe en
vitesse-pression est utilis� (\var{NTERUP}$>$ 1) $a^{n+1,0}$ est initialis� par
la derni�re valeur obtenue de $a^{n+1}$.}:
\begin{equation}\notag
\left\{\begin{array}{l}
a^{n+1,\,0} = a^{n}\\
a^{n+1,\,k+1} = a^{n+1,\,k} + \delta a^{n+1,\,k+1}
\end{array}\right.
\end{equation}
o� $\delta a^{n+1,\,k+1}$ est solution de :
\begin{equation}
\mathcal{EM}_{n}(a^{n+1,\,k} + \delta a^{n+1,\,k+1}) = \mathcal{EM}_{n}(a^{n+1,\,k}) - \mathcal{E}_{n}(a^{n+1,\,k})
\end{equation}
Soit encore, par lin�arit� de $\mathcal{EM}_{n}$ :
\begin{equation}
\mathcal{EM}_{n}(\delta a^{n+1,\,k+1}) = - \mathcal{E}_{n}(a^{n+1,\,k})
\label{Base_Codits_Eq_Codits}
\end{equation}

Cette suite, coupl\'ee avec le choix de l'op\'erateur $\mathcal{E}_{n}$, permet donc de lever la difficult� induite par la
pr\'esence de la convection (discr\'etis\'ee \`a l'aide de sch\'emas num\'eriques
qui peuvent introduire des non lin\'earit\'es) et les termes de
reconstruction. Le sch\'ema r\'eellement choisi par l'utilisateur pour la
convection (donc \'eventuellement non lin\'eaire si le test de pente est activ\'e) ainsi que les termes de
reconstruction vont \^etre pris \`a l'it�ration $k$ et trait\'es au second membre {\it via} le sous-programme \fort{bilsc2},  alors que les termes
non reconstruits sont pris � l'it�ration $k+1$ et repr\'esentent donc les
inconnues du syst\`eme lin\'eaire r\'esolu par \fort{codits}\footnote{cf. le sous-programme
\fort{navsto}.}.\\

On suppose de plus que cette suite $(a^{n+1,\,k})_k$ converge vers la solution
$a^{n+1}$ de l'\'equation (\ref{Base_Codits_Eq_ref_small}), {\it i.e.}
$\lim\limits_{k\rightarrow\infty} \delta a^{n+1,\,k}\,=\,0$, ceci pour tout $n$ donn\'e.\\
(\ref{Base_Codits_Eq_Codits}) correspond \`a l'\'equation r\'esolue par \fort{codits}. La
matrice $\tens{EM}_{\,n}$, matrice associ\'ee \`a $\mathcal{EM}_{n}$  est
 \`a inverser, les termes non lin\'eaires sont mis au second membre mais sous forme
 explicite (indice $k$ de $a^{n+1,\,k}$) et ne posent donc plus de probl\`eme.

\minititre{Remarque 1}
La viscosit� $\mu_{\,tot}$ prise dans $\mathcal{EM}_{n}$ et dans
$\mathcal{E}_{n}$  d�pend du mod�le de turbulence utilis�. Ainsi on a
 $\mu_{\,tot}=\mu_{\,laminaire} + \mu_{\,turbulent}$
dans $\mathcal{EM}_{n}$ et dans $\mathcal{E}_{n}$ sauf lorsque l'on
utilise un mod�le $R_{ij}-\varepsilon$, auquel cas on a
$\mu_{\,tot}=\mu_{\,laminaire}$.\\
Le choix de $\mathcal{EM}_{n}$ �tant  {\it a
priori} arbitraire ($\mathcal{EM}_{n}$ doit \^etre lin\'eaire et la suite
 $(a^{n+1,\,k})_{k\in\grandN}$ doit converger pour tout $n$ donn\'e), une option des mod�les
$R_{ij}-\varepsilon$ ($\var{IRIJNU}=1$) consiste \`a for\c cer  $\mu_{\,tot}^n$
dans l'expression de $\mathcal{EM}_{n}$ \`a la
valeur $\mu_{\,laminaire}^n + \mu_{\,turbulent}^n$ lors de l'appel \`a
\fort{codits} dans le sous-programme \fort{navsto}, pour l'\'etape de
pr\'ediction de la vitesse. Ceci n'a pas de sens
physique (seul $\mu_{\,laminaire}^n$ �tant cens� intervenir), mais cela peut
dans certains cas avoir un effet stabilisateur, sans que cela modifie pour
autant les valeurs de la limite de la suite $(a^{n+1,\,k})_k$.\\

\minititre{Remarque 2}
Quand \fort{codits} est utilis\'e pour le couplage instationnaire renforc�
vitesse-pression (\var{IPUCOU}=1), on fait une seule it�ration $k$ en initialisant la suite $(a^{n+1,\,k})_{k\in\grandN}$ � z�ro. Les conditions de type Dirichlet sont
annul�es (on a $\var{INC}\,=\,0$) et le second membre est �gal � $\rho |\Omega_i|$.
Ce qui permet d'obtenir une approximation de type diagonal de
$\tens{EM}_{n}$
n\'ecessaire lors de l'�tape de correction de la vitesse\footnote{cf. le sous-programme \fort{resolp}.}.
