%                      Code_Saturne version 1.3
%                      ------------------------
%
%     This file is part of the Code_Saturne Kernel, element of the
%     Code_Saturne CFD tool.
%
%     Copyright (C) 1998-2007 EDF S.A., France
%
%     contact: saturne-support@edf.fr
%
%     The Code_Saturne Kernel is free software; you can redistribute it
%     and/or modify it under the terms of the GNU General Public License
%     as published by the Free Software Foundation; either version 2 of
%     the License, or (at your option) any later version.
%
%     The Code_Saturne Kernel is distributed in the hope that it will be
%     useful, but WITHOUT ANY WARRANTY; without even the implied warranty
%     of MERCHANTABILITY or FITNESS FOR A PARTICULAR PURPOSE.  See the
%     GNU General Public License for more details.
%
%     You should have received a copy of the GNU General Public License
%     along with the Code_Saturne Kernel; if not, write to the
%     Free Software Foundation, Inc.,
%     51 Franklin St, Fifth Floor,
%     Boston, MA  02110-1301  USA
%
%-----------------------------------------------------------------------
%

\programme{resolp}
%
\vspace{1cm}
%%%%%%%%%%%%%%%%%%%%%%%%%%%%%%%%%%
%%%%%%%%%%%%%%%%%%%%%%%%%%%%%%%%%%
\section{Fonction}
%%%%%%%%%%%%%%%%%%%%%%%%%%%%%%%%%%
%%%%%%%%%%%%%%%%%%%%%%%%%%%%%%%%%%
Dans ce sous-programme appel� dans \fort{navsto}, on effectue l'�tape de projection de la vitesse (ou de correction de pression). L'�quation de quantit� de mouvement (pr\'ediction) est r�solue dans \fort{preduv} avec une pression totalement explicite. Il en r�sulte un champ de vitesse qui ne satisfait pas l'\'equation de continuit\'e. Deux algorithmes de correction sont propos�s :
\begin{enumerate}
\item L'algorithme que l'on appellera "couplage faible vitesse-pression". C'est un algorithme largement implant� dans les codes industriels. Il ne couple la vitesse et la pression qu'� travers le terme de masse (c'est l'algorithme propos� par d�faut). C'est un algorithme de type \textit{SIMPLEC} proche du \textit{SIMPLE}. Ce dernier prend en compte, en plus du terme de masse, les diagonales simplifi\'ees de la convection, de la diffusion et des termes source implicites.
\item L'algorithme de couplage vitesse-pression renforc� (option \var{IPUCOU  = 1}). C'est un algorithme qui couple la vitesse et la pression � travers tous les termes (convection, diffusion et termes source implicites) de l'�quation de quantit� de mouvement sans pour autant �tre exact. Il permet en pratique de prendre de grands pas de temps sans d�coupler totalement la vitesse et la pression.
\end{enumerate}

Si $\delta p$ est l'incr�ment de pression ({\it i.e.} $p^{n+1} = p^n+\delta p$) et $\widetilde{u}$ la vitesse issue de l'�tape de pr\'ediction, l'�tape de projection revient d'un point de vue continu � r�soudre une �quation de Poisson :

\begin{equation}
\dive(\,{\tens{T}^n\ \grad{\delta p}}) = \dive(\,{\rho \,\widetilde{u}})
\end{equation}

et \`a corriger la vitesse :

\begin{equation}
\vect{u}^{n+1} = \vect{u}^n - \frac{1}{\rho}\ \tens{T}^n\ \grad{\delta p}
\end{equation}

$\tens{T}^n$ est un tenseur d'ordre 2 dont les termes sont homog�nes � un pas de temps.

