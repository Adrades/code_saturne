%                      Code_Saturne version 1.3
%                      ------------------------
%
%     This file is part of the Code_Saturne Kernel, element of the
%     Code_Saturne CFD tool.
% 
%     Copyright (C) 1998-2007 EDF S.A., France
%
%     contact: saturne-support@edf.fr
% 
%     The Code_Saturne Kernel is free software; you can redistribute it
%     and/or modify it under the terms of the GNU General Public License
%     as published by the Free Software Foundation; either version 2 of
%     the License, or (at your option) any later version.
% 
%     The Code_Saturne Kernel is distributed in the hope that it will be
%     useful, but WITHOUT ANY WARRANTY; without even the implied warranty
%     of MERCHANTABILITY or FITNESS FOR A PARTICULAR PURPOSE.  See the
%     GNU General Public License for more details.
% 
%     You should have received a copy of the GNU General Public License
%     along with the Code_Saturne Kernel; if not, write to the
%     Free Software Foundation, Inc.,
%     51 Franklin St, Fifth Floor,
%     Boston, MA  02110-1301  USA
%
%-----------------------------------------------------------------------
%
\programme{turbke}

\vspace{1cm}
%%%%%%%%%%%%%%%%%%%%%%%%%%%%%%%%%%
%%%%%%%%%%%%%%%%%%%%%%%%%%%%%%%%%%
\section{Fonction}
%%%%%%%%%%%%%%%%%%%%%%%%%%%%%%%%%%
%%%%%%%%%%%%%%%%%%%%%%%%%%%%%%%%%%
Le but de ce sous-programme est de r\'esoudre le syst\`eme des \'equations de
$k$ et $\varepsilon$ de mani\`ere semi-coupl\'ee.\\
Le syst\`eme d'\'equations r\'esolu est le suivant :

\begin{equation}
\left\{\begin{array}{l}
\displaystyle
\rho\frac{\partial k}{\partial t} + 
\dive\left[\rho \vect{u}\,k-(\mu+\frac{\mu_t}{\sigma_k}\grad{k})\right] =
\mathcal{P}+\mathcal{G}-\rho\varepsilon+k\dive(\rho\vect{u})
+\Gamma(k_i-k)\\
\multicolumn{1}{c}{+\alpha_k k +\beta_k}\\
\displaystyle
\rho\frac{\partial \varepsilon}{\partial t} + 
\dive\left[\rho \vect{u}\,\varepsilon-
(\mu+\frac{\mu_t}{\sigma_\varepsilon}\grad{\varepsilon})\right] =
C_{\varepsilon_1}\frac{\varepsilon}{k}\left[\mathcal{P}
+(1-C_{\varepsilon_3})\mathcal{G}\right]
-\rho C_{\varepsilon_2}\frac{\varepsilon^2}{k}
+\varepsilon\dive(\rho\vect{u})\\
\multicolumn{1}{c}{+\Gamma(\varepsilon_i-\varepsilon)
+\alpha_\varepsilon \varepsilon +\beta_\varepsilon}
\end{array}\right.
\end{equation}

$\mathcal{P}$ est le terme de production par cisaillement moyen :
\begin{displaymath}
\begin{array}{rcl}
\mathcal{P} & = & \displaystyle -\rho R_{ij}\frac{\partial u_i}{\partial x_j}
= -\left[-\mu_t \left(\frac{\partial u_i}{\partial x_j} + 
\frac{\partial u_j}{\partial x_i}\right)
+\frac{2}{3}\mu_t\frac{\partial u_k}{\partial x_k}\delta_{ij}
+\frac{2}{3}\rho k\delta_{ij}\right]
\frac{\partial u_i}{\partial x_j}\\
& = & \displaystyle \mu_t \left(\frac{\partial u_i}{\partial x_j} + 
\frac{\partial u_j}{\partial x_i}\right)\frac{\partial u_i}{\partial x_j}
-\frac{2}{3}\mu_t(\dive\vect{u})^2-\frac{2}{3}\rho k \dive(\vect{u})\\
& = & \displaystyle \mu_t \left[
2\left(\frac{\partial u}{\partial x}\right)^2+
2\left(\frac{\partial v}{\partial y}\right)^2+
2\left(\frac{\partial w}{\partial z}\right)^2+
\left(\frac{\partial u}{\partial y}+\frac{\partial v}{\partial x}\right)^2+
\left(\frac{\partial u}{\partial z}+\frac{\partial w}{\partial x}\right)^2+
\left(\frac{\partial v}{\partial z}+\frac{\partial w}{\partial y}\right)^2
\right]\\
\multicolumn{3}{r}%
{\displaystyle -\frac{2}{3}\mu_t(\dive\vect{u})^2-\frac{2}{3}\rho k \dive(\vect{u})}
\end{array}
\end{displaymath}

$\mathcal{G}$ est le terme de production par gravit\'e :
$\displaystyle
\mathcal{G}=-\frac{1}{\rho}\frac{\mu_t}{\sigma_t}
\frac{\partial\rho}{\partial x_i}g_i$

La viscosit\'e turbulente est
$\displaystyle \mu_t=\rho C_\mu\frac{k^2}{\varepsilon}$.

Les constantes sont :\\
$C_\mu=0,09$ ;
$C_{\varepsilon_2}=1,92$ ; $\sigma_k=1$ ; $\sigma_\varepsilon=1,3$\\
$C_{\varepsilon_3}=0$ si $\mathcal{G}\geqslant0$ (stratification instable) et 
$C_{\varepsilon_3}=1$ si $\mathcal{G}\leqslant0$ (stratification stable).

$\Gamma$ est un \'eventuel terme source de masse (tel que l'\'equation de
conservation de masse devienne
$\displaystyle \frac{\partial \rho}{\partial t}+\dive(\rho\vect{u})=\Gamma$).
$\varphi_i$ ($\varphi=k$ ou $\varepsilon$) est la valeur de $\varphi$
associ\'ee \`a la masse inject\'ee ou retir\'ee. Dans le cas o\`u on retire de
la masse ($\Gamma<0$), on a forc\'ement $\varphi_i=\varphi$. De m\^eme, quand on
injecte de la masse, on sp\'ecifie souvent aussi $\varphi_i=\varphi$. Dans ces
deux cas, le terme dispara\^\i t de l'\'equation. Dans la suite du document, on
qualifiera d'{\em injection forc\'ee} les cas o\`u on a $\Gamma>0$ et
$\varphi_i\ne\varphi$.

$\alpha_k$, $\beta_k$, $\alpha_\varepsilon$, $\beta_\varepsilon$ sont des termes
sources utilisateur \'eventuels, conduisant \`a une implicitation partielle, impos\'es le cas
\'ech\'eant par le sous-programme \fort{ustske}.


