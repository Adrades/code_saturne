%                      Code_Saturne version 1.3
%                      ------------------------
%
%     This file is part of the Code_Saturne Kernel, element of the
%     Code_Saturne CFD tool.
%
%     Copyright (C) 1998-2007 EDF S.A., France
%
%     contact: saturne-support@edf.fr
%
%     The Code_Saturne Kernel is free software; you can redistribute it
%     and/or modify it under the terms of the GNU General Public License
%     as published by the Free Software Foundation; either version 2 of
%     the License, or (at your option) any later version.
%
%     The Code_Saturne Kernel is distributed in the hope that it will be
%     useful, but WITHOUT ANY WARRANTY; without even the implied warranty
%     of MERCHANTABILITY or FITNESS FOR A PARTICULAR PURPOSE.  See the
%     GNU General Public License for more details.
%
%     You should have received a copy of the GNU General Public License
%     along with the Code_Saturne Kernel; if not, write to the
%     Free Software Foundation, Inc.,
%     51 Franklin St, Fifth Floor,
%     Boston, MA  02110-1301  USA
%
%-----------------------------------------------------------------------
%

%%%%%%%%%%%%%%%%%%%%%%%%%%%%%%%%%%
%%%%%%%%%%%%%%%%%%%%%%%%%%%%%%%%%%
\section{Mise en \oe uvre}
%%%%%%%%%%%%%%%%%%%%%%%%%%%%%%%%%%
%%%%%%%%%%%%%%%%%%%%%%%%%%%%%%%%%%

\etape{Calcul du terme de production}
On appelle trois fois \fort{grdcel} pour calculer les gradients de $u$, $v$ et
$w$. Au final, on a \\
$\displaystyle \var{TINSTK}=
2\left(\frac{\partial u}{\partial x}\right)^2+
2\left(\frac{\partial v}{\partial y}\right)^2+
2\left(\frac{\partial w}{\partial z}\right)^2+
\left(\frac{\partial u}{\partial y}+\frac{\partial v}{\partial x}\right)^2+
\left(\frac{\partial u}{\partial z}+\frac{\partial w}{\partial x}\right)^2+
\left(\frac{\partial v}{\partial z}+\frac{\partial w}{\partial y}\right)^2$\\
et\\
$\displaystyle \var{DIVU}=
\frac{\partial u}{\partial x}+\frac{\partial v}{\partial y}
+\frac{\partial w}{\partial z}$

(le terme $div(\vect{u})$ n'est pas calcul\'e par \fort{divmas}, pour
correspondre exactement \`a la trace du tenseur des d\'eformations qui est
calcul\'e pour la production)


\etape{Lecture des termes sources utilisateur}
Appel de \fort{ustske} pour charger les termes sources utilisateurs. Ils sont
stock\'es dans les tableaux suivants :\\
$\var{W7}=\Omega\beta_k$\\
$\var{W8}=\Omega\beta_\varepsilon$\\
$\var{DAM}=\Omega\alpha_k$\\
$\var{W9}=\Omega\alpha_\varepsilon$

Puis on ajoute le terme en $(div\vect{u})^2$ \`a \var{TINSTK}. On a donc \\
$\var{TINSTK}=\tilde{\mathcal{P}}$

\etape{Calcul du terme de gravit\'e}
Calcul uniquement si $\var{IGRAKE}=1$.\\
On appelle \fort{grdcel} pour \var{ROM}, avec comme conditions aux limites
$\var{COEFA}=\var{ROMB}$ et \mbox{$\var{COEFB}=\var{VISCB}=0$}.\\
$\var{PRDTUR}=\sigma_t$ est mis \`a 1 si on n'a pas de scalaire temp\'erature.

$\tilde{\mathcal{G}}$ est calcul\'e et les termes sources sont mis \`a jour :\\
$\var{TINSTK}=\tilde{\mathcal{P}}+\tilde{\mathcal{G}}$\\
$\var{TINSTE}=\tilde{\mathcal{P}}+\Max\left[\tilde{\mathcal{G}},0\right]
=\tilde{\mathcal{P}}+(1-C_{\varepsilon_3})\tilde{\mathcal{G}}$

Si $\var{IGRAKE}=0$, on a simplement\\
$\var{TINSTK}=\var{TINSTE}=\tilde{\mathcal{P}}$

\etape{Calcul du terme d'accumulation de masse}
On stocke
$\displaystyle \var{W1}=\Omega\dive(\rho\vect{u})$
calcul\'e par \fort{divmas} (pour correspondre aux termes de convection de la
matrice).

\etape{Calcul des termes sources explicites}
On affecte les termes sources explicites de $k$ et $\varepsilon$ pour la
premi\`ere \'etape.\\
$\displaystyle\var{SMBRK}=\Omega\left(\mu_t(\tilde{\mathcal{P}}+\tilde{\mathcal{G}})
-\frac{2}{3}\rho^{(n)} k^{(n)}\dive{\vect{u}}
-\rho^{(n)} \varepsilon^{(n)}\right)+\Omega k^{(n)}\dive(\rho\vect{u})$\\
$\displaystyle\var{SMBRE}=\Omega\frac{\varepsilon^{(n)}}{k^{(n)}}
\left(C_{\varepsilon_1}\left(
\mu_t(\tilde{\mathcal{P}}+(1-C_{\varepsilon_3})\tilde{\mathcal{G}})
-\frac{2}{3}\rho^{(n)} k^{(n)}\dive{\vect{u}}\right)
-C_{\varepsilon_2}\rho^{(n)}\varepsilon^{(n)}\right)
+\Omega\varepsilon^{(n)}\dive(\rho\vect{u})$

soit $\var{SMBRK}=\Omega S_k^{(n)}+\Omega k^{(n)}\dive(\rho\vect{u})$
et $\var{SMBRE}=\Omega S_\varepsilon^{(n)}+\Omega\varepsilon^{(n)}\dive(\rho\vect{u})$.


\etape{Calcul des termes sources utilisateur}
On ajoute les termes sources utilisateur explicites \`a \var{SMBRK} et
\var{SMBRE}, soit :\\
$\var{SMBRK}=\Omega S_k^{(n)}+\Omega k^{(n)}\dive(\rho\vect{u})+\Omega\alpha_k k^{(n)} +\Omega\beta_k$\\
$\var{SMBRE}=\Omega S_\varepsilon^{(n)}+\Omega\varepsilon^{(n)}\dive(\rho\vect{u})
+\Omega\alpha_\varepsilon \varepsilon^{(n)} +\Omega\beta_\varepsilon$

Les tableaux \var{W7} et \var{W8} sont lib\'er\'es, \var{DAM} et \var{W9} sont
conserv\'es pour \^etre utilis\'es dans la troisi\`eme phase de r\'esolution.

\etape{Calcul des termes de convection/diffusion explicites}
\fort{bilsc2} est appel\'e deux fois, pour $k$ et pour $\varepsilon$, afin
d'ajouter \`a \var{SMBRK} et \var{SMBRE} les termes de convection/diffusion
explicites $D(k^{(n)})$ et $D(\varepsilon^{(n)})$. Ces termes sont d'abord
stock\'es dans \var{W7} et \var{W8}, pour \^etre conserv\'es et r\'eutilis\'es
dans la troisi\`eme phase de r\'esolution.


\etape{Termes source de masse}
Dans le cas d'une injection forc\'ee de mati\`ere, on passe deux fois dans
\fort{catsma} pour ajouter les termes en
$\Omega \Gamma (k_i-k^{(n)})$ et
$\Omega \Gamma (\varepsilon_i-\varepsilon^{(n)})$ \`a \var{SMBRK} et
\var{SMBRE}. La partie implicite ($\Omega\Gamma$) est stock\'ee dans les
variables \var{W2} et \var{W3}, qui seront utilis\'ees lors de la troisi\`eme
phase (les deux variables sont bien n\'ecessaires, au cas o\`u on aurait une
injection forc\'ee sur $k$ et pas sur $\varepsilon$, par exemple).

\etape{Fin de la premi\`ere phase}
Ceci termine la premi\`ere phase. On a \\
$\displaystyle \var{SMBRK}=\Omega \rho^{(n)}\frac{k_e-k^{(n)}}{\Delta t}$\\
$\displaystyle \var{SMBRE}=\Omega \rho^{(n)}\frac{\varepsilon_e-\varepsilon^{(n)}}{\Delta t}$

\etape{Phase de couplage}
(uniquement si $\var{IKECOU}=1$)

On renormalise \var{SMBRK} et \var{SMBRE} qui deviennent les seconds membres du
syst\`eme de couplage.\\
$\displaystyle \var{SMBRK}=\frac{1}{\Omega\rho^{(n)}}\var{SMBRK}
=\frac{k_e-k^{(n)}}{\Delta t}$\\
$\displaystyle \var{SMBRE}=\frac{1}{\Omega\rho^{(n)}}\var{SMBRE}
=\frac{\varepsilon_e-\varepsilon^{(n)}}{\Delta t}$\\
et $\displaystyle \var{DIVP23}=\frac{2}{3}\Max\left[\dive(\vect{u}),0\right]$.

On remplit la matrice de couplage\\
$\displaystyle \var{A11}=\frac{1}{\Delta t}
-2 C_\mu\frac{k^{(n)}}{\varepsilon^{(n)}}
\Min\left[\left(\tilde{\mathcal{P}}+\tilde{\mathcal{G}}\right),0\right]
+\frac{2}{3}\Max\left[\dive(\vect{u}),0\right]$\\
$\displaystyle \var{A12}= 1$\\
$\displaystyle \var{A21}=
- C_{\varepsilon_1} C_\mu \left(\tilde{\mathcal{P}}
+(1-C_{\varepsilon_3})\tilde{\mathcal{G}}\right)
- C_{\varepsilon_2}\left(\frac{\varepsilon^{(n)}}{k^{(n)}}\right)^2$\\
$\displaystyle \var{A22}=\frac{1}{\Delta t}
+\frac{2}{3}C_{\varepsilon_1}\Max\left[\dive(\vect{u}),0\right]
+2 C_{\varepsilon_2}\frac{\varepsilon^{(n)}}{k^{(n)}}$

On inverse le syst\`eme $2\times 2$, et on r\'ecup\`ere :\\
$\displaystyle \var{DELTK}=k_{ts}-k^{(n)}$\\
$\displaystyle \var{DELTE}=\varepsilon_{ts}-\varepsilon^{(n)}$

\etape{Fin de la deuxi\`eme phase}
On met \`a jour les variables \var{SMBRK} et \var{SMBRE}.\\
$\displaystyle \var{SMBRK}=\Omega \rho^{(n)}\frac{k_{ts}-k^{(n)}}{\Delta t}$\\
$\displaystyle \var{SMBRE}=
\Omega \rho^{(n)}\frac{\varepsilon_{ts}-\varepsilon^{(n)}}{\Delta t}$

Dans le cas o\`u on ne couple pas ($\var{IKECOU}=0$), ces deux variables gardent
la m\^eme valeur qu'\`a la fin de la premi\`ere \'etape.

\etape{Calcul des termes implicites}
On retire \`a \var{SMBRK} et \var{SMBRE} la partie en convection diffusion au
temps $n$, qui \'etait stock\'ee dans \var{W7} et \var{W8}.\\
$\displaystyle \var{SMBRK}=\Omega \rho^{(n)}\frac{k_{ts}-k^{(n)}}{\Delta t}
-\Omega D(k^{(n)})$\\
$\displaystyle \var{SMBRE}=
\Omega \rho^{(n)}\frac{\varepsilon_{ts}-\varepsilon^{(n)}}{\Delta t}
-\Omega D(\varepsilon^{(n)})$


On calcule les termes implicites, hors convection/diffusion, qui correspondent
\`a la diagonale de la matrice.\\
$\displaystyle \var{TINSTK}=\frac{\Omega \rho^{(n)}}{\Delta t}
-\Omega\dive(\rho\vect{u})+\Omega\Gamma+\Omega\Max[-\alpha_k,0]$\\
$\displaystyle \var{TINSTE}=\frac{\Omega \rho^{(n)}}{\Delta t}
-\Omega\dive(\rho\vect{u})+\Omega\Gamma+\Omega\Max[-\alpha_\varepsilon,0]$\\
($\Gamma$ n'est pris en compte qu'en injection forc\'ee, c'est-\`a-dire qu'il
est forc\'ement positif et ne risque pas d'affaiblir la diagonale de la matrice).

\etape{R\'esolution finale}
On passe alors deux fois dans \fort{codits}, pour $k$ et $\varepsilon$,
pour r\'esoudre les \'equations du type :

$\var{TINST}\times(\varphi^{(n+1)}-\varphi^{(n)}) = D(\varphi^{(n+1)})+\var{SMBR}$.

\etape{clipping final}
On passe enfin dans la routine \fort{clipke} pour faire un clipping \'eventuel
de $k^{(n+1)}$ et $\varepsilon^{(n+1)}$.



