%                      Code_Saturne version 1.3
%                      ------------------------
%
%     This file is part of the Code_Saturne Kernel, element of the
%     Code_Saturne CFD tool.
%
%     Copyright (C) 1998-2008 EDF S.A., France
%
%     contact: saturne-support@edf.fr
%
%     The Code_Saturne Kernel is free software; you can redistribute it
%     and/or modify it under the terms of the GNU General Public License
%     as published by the Free Software Foundation; either version 2 of
%     the License, or (at your option) any later version.
%
%     The Code_Saturne Kernel is distributed in the hope that it will be
%     useful, but WITHOUT ANY WARRANTY; without even the implied warranty
%     of MERCHANTABILITY or FITNESS FOR A PARTICULAR PURPOSE.  See the
%     GNU General Public License for more details.
%
%     You should have received a copy of the GNU General Public License
%     along with the Code_Saturne Kernel; if not, write to the
%     Free Software Foundation, Inc.,
%     51 Franklin St, Fifth Floor,
%     Boston, MA  02110-1301  USA
%
%-----------------------------------------------------------------------
%
%%%%%%%%%%%%%%%%%%%%%%%%%%%%%%%%%%%%%%%%%%%%%%%%%%%%%%%%%%%%%%%%%%%%%%
% SYNTH�SE
\vspace*{0.1cm}
\begin{center}
\medskip
\textit{SYNTH\`ESE}
\end{center}
\vspace*{1cm}
\pdfbookmark[1]{Synth�se}{synthese}

\CS\ est un syst\`eme de r\'esolution des \'equations de Navier-Stokes
pour des \'ecoulements 2D, 2D axisym\'etriques ou 3D,
stationnaires ou instationnaires, laminaires ou turbulents,
incompressibles ou dilatables, isothermes ou non, avec prise en compte de
scalaires et de fluctuations de scalaires.  Le code comprend en outre
un module lagrangien, un module de rayonnement
semi-transparent, un module ``combustion
gaz'',  un module ``charbon pulv\'eris\'e'', un module ``\'electrique''
(effet Joule et arc \'electrique) et un module compressible.
On d\'esignera dans le pr\'esent
document les potentialit\'es ``combustion gaz'', ``charbon pulv\'eris\'e'',
``\'electrique'' et ``compressible''
sous la d\'enomination commune de ``physiques particuli\`eres''.
La discr\'etisation est de type
volume finis, et permet l'utilisation d'une large gamme de maillages non
structur\'es, qu'ils soient hybrides (contenant des \'el\'ements
de diff\'erents
types) ou qu'ils pr\'esentent des non conformit\'es.

Ce document constitue la documentation th\'eorique et
informatique des parties centrales du noyau de \CS\ version \verscs.
Les modules �lectrique
et compressible sont aussi trait�s.

Afin que la documentation soit directement utilisable par les
d\'eveloppeurs, elle a \'et\'e structur\'ee en sous-parties correspondant
\`a la fois aux \'etapes majeures de l'algorithme et \`a certains
sous-programmes cl\'es du code.
Pour chacune des sous-parties (ou sous-programmes), on donne une description
de la {\bf fonction}, puis on d\'etaille la {\bf discr\'etisation}. \`A
l'attention des d\'eveloppeurs, on pr\'ecise ensuite la {\bf mise en \oe
uvre} informatique et, le cas �ch�ant, on liste dans un dernier paragraphe
les {\bf points \`a traiter} et les am\'eliorations \`a apporter au code.

La documentation est attach\'ee \`a la version du code
correspondante pour favoriser les mises \`a jour. En pratique, les
utilisateurs de \CS\
peuvent acc\'eder \`a la documentation sous\\
\texttt{\$CS\_HOME%$
/doc/NOYAU/}, information qui leur est rappel\'ee par la
commande d'information g\'en\'erale \texttt{info\_cs [theory]}.

\CS\ is free software; you can redistribute it
and/or modify it under the terms of the GNU General Public License
as published by the Free Software Foundation; either version 2 of
the License, or (at your option) any later version.
\CS\ is distributed in the hope that it will be
useful, but WITHOUT ANY WARRANTY; without even the implied warranty
of MERCHANTABILITY or FITNESS FOR A PARTICULAR PURPOSE.  See the
GNU General Public License for more details.

%
%%%%%%%%%%%%%%%%%%%%%%%%%%%%%%%%%%%%%%%%%%%%%%%%%%%%%%%%%%%%%%%%%%%%%%

%
%%%%%%%%%%%%%%%%%%%%%%%%%%%%%%%%%%%%%%%%%%%%%%%%%%%%%%%%%%%%%%%%%%%%%%
% EXECUTIVE SUMMARY
%\passepage
%\vspace*{0.1cm}
%\begin{center}
%\textbf{\large \titreang}\\
%\medskip
%\textit{EXECUTIVE SUMMARY}
%\end{center}
%\vspace*{1cm}
%\pdfbookmark[1]{Excutive summary}{executive summary}
%
%This is the executive summary of the report.
%
%
%%%%%%%%%%%%%%%%%%%%%%%%%%%%%%%%%%%%%%%%%%%%%%%%%%%%%%%%%%%%%%%%%%%%%%

%%% Local Variables:
%%% mode: latex
%%% TeX-master: "rapport"
%%% End:
