%-----------------------------------------------------------------------
%
%     This file is part of the Code_Saturne Kernel, element of the
%     Code_Saturne CFD tool.
%
%     Copyright (C) 1998-2008 EDF S.A., France
%
%     contact: saturne-support@edf.fr
%
%     The Code_Saturne Kernel is free software; you can redistribute it
%     and/or modify it under the terms of the GNU General Public License
%     as published by the Free Software Foundation; either version 2 of
%     the License, or (at your option) any later version.
%
%     The Code_Saturne Kernel is distributed in the hope that it will be
%     useful, but WITHOUT ANY WARRANTY; without even the implied warranty
%     of MERCHANTABILITY or FITNESS FOR A PARTICULAR PURPOSE.  See the
%     GNU General Public License for more details.
%
%     You should have received a copy of the GNU General Public License
%     along with the Code_Saturne Kernel; if not, write to the
%     Free Software Foundation, Inc.,
%     51 Franklin St, Fifth Floor,
%     Boston, MA  02110-1301  USA
%
%-----------------------------------------------------------------------
%
%%%%%%%%%%%%%%%%%%%%%%%%%%%%%%%%%%%%%%%%%%%%%%%%%%%%%%%%%%%%%%%%%%%%%%
% SYNTH�SE
\vspace*{0.1cm}
\begin{center}
\medskip
\textit{ABSTRACT}
\end{center}
\vspace*{1cm}
\pdfbookmark[1]{Summary}{summary}

\CS solves the Navier-Stokes equations for 2D, 2D axisymmetric, or 3D,
steady or unsteady, laminar or turbulent, incompressible or dilatable
flows, with or without heat transfer, and with possible scalar
fluctuations. The code also includes a Lagrangian module, a
semi-transparent radiation module, a gas combustion module, a coal
combustion module, an electric module (Joule effect and electric arc)
and a compressible module. In the present document, the "gas
combustion", "coal combustion", "electric" and "compressible"
capabilities of the code will be referred to as "particular
physics". The code uses a finite volume discretization. A wide range
of unstructured meshes, either hybrid (containing elements of
different types) and/or non-conform, can be used.


This document constitutes the theory and developer's guide associated
with the kernel of \CS. Combustion, electric and compressible modules
are also presented.

To make the documentation immediately suitable to the developers' needs, it
has been organized into sub-sections corresponding to the major steps of the
algorithm and to some important subroutines of the code.
In each sub-section (for each subroutine), the \textbf{function} of the
piece of code considered is indicated. Then, a description of the \textbf{%
discretisation} follows. Finally, and more oriented towards the developers,
details of the \textbf{implementation} are provided and a list of open
problems is given (improvements, limitations...).

To make it easier for the developers to keep the documentation up to date
during the development process, the files have been associated "physically"
with the release of the code (each release of the code includes a directory
containing the whole documentation). In practice,
% fa modification : discard (from the development version 1.1.0.n on),
the users of \CS can access the documentation at the following location \texttt{\$CS\_HOME/doc/NOYAU/}. 
The general command \texttt{info\_cs [theory]} also provides this
information.


\CS is free software; you can redistribute it
and/or modify it under the terms of the GNU General Public License
as published by the Free Software Foundation; either version 2 of
the License, or (at your option) any later version.
\CS is distributed in the hope that it will be
useful, but WITHOUT ANY WARRANTY; without even the implied warranty
of MERCHANTABILITY or FITNESS FOR A PARTICULAR PURPOSE.  See the
GNU General Public License for more details.

%
%%%%%%%%%%%%%%%%%%%%%%%%%%%%%%%%%%%%%%%%%%%%%%%%%%%%%%%%%%%%%%%%%%%%%%

%
%%%%%%%%%%%%%%%%%%%%%%%%%%%%%%%%%%%%%%%%%%%%%%%%%%%%%%%%%%%%%%%%%%%%%%
% EXECUTIVE SUMMARY
%\passepage
%\vspace*{0.1cm}
%\begin{center}
%\textbf{\large \titreang}\\
%\medskip
%\textit{EXECUTIVE SUMMARY}
%\end{center}
%\vspace*{1cm}
%\pdfbookmark[1]{Excutive summary}{executive summary}
%
%This is the executive summary of the report.
%
%
%%%%%%%%%%%%%%%%%%%%%%%%%%%%%%%%%%%%%%%%%%%%%%%%%%%%%%%%%%%%%%%%%%%%%%
