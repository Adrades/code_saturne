%                      Code_Saturne version 1.3
%                      ------------------------
%
%     This file is part of the Code_Saturne Kernel, element of the
%     Code_Saturne CFD tool.
%
%     Copyright (C) 1998-2007 EDF S.A., France
%
%     contact: saturne-support@edf.fr
%
%     The Code_Saturne Kernel is free software; you can redistribute it
%     and/or modify it under the terms of the GNU General Public License
%     as published by the Free Software Foundation; either version 2 of
%     the License, or (at your option) any later version.
%
%     The Code_Saturne Kernel is distributed in the hope that it will be
%     useful, but WITHOUT ANY WARRANTY; without even the implied warranty
%     of MERCHANTABILITY or FITNESS FOR A PARTICULAR PURPOSE.  See the
%     GNU General Public License for more details.
%
%     You should have received a copy of the GNU General Public License
%     along with the Code_Saturne Kernel; if not, write to the
%     Free Software Foundation, Inc.,
%     51 Franklin St, Fifth Floor,
%     Boston, MA  02110-1301  USA
%
%-----------------------------------------------------------------------
%
%%%%%%%%%%%%%%%%%%%%%%%%%%%%%%%%%%%%%%%%%%%%%%%%%%%%%%%%%%%%%%%%%%%%%%
%                                                                    %
%                                                                    %
%                                                                    %
% Titre :           Code_Saturne theory and programmer's guide       %
%                                                                    %
%                                                                    %
%                                                                    %
%%%%%%%%%%%%%%%%%%%%%%%%%%%%%%%%%%%%%%%%%%%%%%%%%%%%%%%%%%%%%%%%%%%%%%
\documentclass[a4paper,10pt,twoside]{report}
%
%%%%%%%%%%%%%%%%%%%%%%%%%%%%%%%%%%%%%%%%%%%%%%%%%%%%%%%%%%%%%%%%%%%%%%
% PACKAGES OBLIGATOIRES
\usepackage{../../style/csdoc}
%
%%%%%%%%%%%%%%%%%%%%%%%%%%%%%%%%%%%%%%%%%%%%%%%%%%%%%%%%%%%%%%%%%%%%%%

%
%%%%%%%%%%%%%%%%%%%%%%%%%%%%%%%%%%%%%%%%%%%%%%%%%%%%%%%%%%%%%%%%%%%%%%
% PACKAGES ET COMMANDES POUR LE DOCUMENTS PDF ET LES HYPERLIENS
\usepackage[pdftex,
            bookmarksopen=true,
            colorlinks=true,
            linkcolor=blue,
            filecolor=blue,
            urlcolor=blue,
            citecolor=blue]{hyperref}
\hypersetup{%
  pdftitle = {Code_Saturne version 1.3 Theory and Programmer's Guide},
  pdfauthor = {MFEE},
  pdfpagemode = UseOutlines
}
\pdfinfo{/CreationDate (D:20030429000000-01 00 )}
%
% Pour avoir les Thumbnails a l'ouverture du document sous ACROREAD :
% pdfpagemode = UseThumbs
%
%%%%%%%%%%%%%%%%%%%%%%%%%%%%%%%%%%%%%%%%%%%%%%%%%%%%%%%%%%%%%%%%%%%%%%
% MACROS SUPPLEMENTAIRES
% \newcommand{/...}{...}
%
\setcounter{tocdepth}{0}
%Compteur de ``programme'' remis a jour dans les part.
\newcounter{prog}[part]
\renewcommand{\theprog}{\arabic{prog}}
\renewcommand{\thesection}{\theprog.\arabic{section}}
\renewcommand{\theequation}{\thepart.\theprog.\arabic{equation}}
\renewcommand{\thefigure}{\thepart.\theprog.\arabic{figure}}
\newcommand{\programme}[1]{%
\passepage
\refstepcounter{prog}
\stepcounter{chapter}
\setcounter{section}{0}
\setcounter{equation}{0}
\setcounter{figure}{0}
\begin{center}
\Huge \bf \theprog - \underline{Sous-programme \fort{#1}}
\end{center}
\addcontentsline{toc}{chapter}{\theprog- Sous-programme #1}}
% repertoire et extension des images
\newcommand{\repgraphics}{../graphics}
\newcommand{\extgraphics}{pdf}
\DeclareGraphicsExtensions{.pdf,.jpg}
%
\newcommand{\bm}[1]{\text{\boldmath $#1$ \unboldmath} \! \!}
\newcommand{\Min}{\text{Min}}
\newcommand{\Max}{\text{Max}}
\newcommand{\nl}{\vspace{1ex}}
\newcommand{\grandN}{\mbox{I\hspace{-.15em}N}}
\newcommand{\grandC}{\mbox{l\hspace{-.47em}C}}
\newcommand{\mat}[1]{\underline{\textrm{#1}}}
\newcommand{\matt}[1]{\underline{\underline{\textrm{#1}}}}
\newcommand{\comp}[1]{\textrm{#1}}
\newcommand{\gradv}{\text{g}\underline{\text{rad}}\ }
\newcommand{\gradt}{\text{g}\underline{\underline{\text{rad}}}\ }
\newcommand{\divs}{\text{div}}
\newcommand{\divv}{\underline{\text{div}}}
\newcommand{\etape}[1]{\vspace{0.3cm}$\bullet\ ${\bf #1}\\}
\newcommand{\fort}[1]{\texttt{#1}}
\newcommand{\var}[1]{\ensuremath{\text{\texttt{#1}}}}
%
%%%%%%%%%%%%%%%%%%%%%%%%%%%%%%%%%%%%%%%%%%%%%%%%%%%%%%%%%%%%%%%%%%%%%%
% INFO POUR PAGES DE GARDES
\titreCS{\CS\ \verscs\ Theory and Programmer's Guide}
\docassociesCS{}
\resumeCS{Ce document constitue la documentation th\'eorique et
informatique des parties centrales du noyau de \CS~\verscs.
La documentation est attach\'ee \`a la version du code
correspondante pour favoriser les mises \`a jour. En pratique, les
utilisateurs de \CS\
peuvent acc\'eder \`a la documentation sous
\texttt{\$CS\_HOME/doc/NOYAU/}, information qui leur est rappel\'ee par la
commande d'information g\'en\'erale \texttt{info\_cs [theory]}.}
%
%%%%%%%%%%%%%%%%%%%%%%%%%%%%%%%%%%%%%%%%%%%%%%%%%%%%%%%%%%%%%%%%%%%%%%

%
%%%%%%%%%%%%%%%%%%%%%%%%%%%%%%%%%%%%%%%%%%%%%%%%%%%%%%%%%%%%%%%%%%%%%%
% DEBUT DU DOCUMENT
\begin{document}
\pdfgraphics
%\def\bibname{R�f�rences}
\def\contentsname{\textbf{\normalsize SOMMAIRE}\pdfbookmark[1]{Sommaire}{contents}}
\def\indexname{Index des variables principales et des mots cl\'es}

\pdfbookmark[1]{Pages de garde}{pdg}
\large
\makepdgCS
\normalsize

\passepage
\input summary

\passepart
\begin{center}\begin{singlespace}
\tableofcontents
\end{singlespace}\end{center}


%\passepage
%\printsymliste\
%
%%%%%%%%%%%%%%%%%%%%%%%%%%%%%%%%%%%%%%%%%%%%%%%%%%%%%%%%%%%%%%%%%%%%%%
% CORPS DU DOCUMENT
%
\passepage
\part{Introduction}
\stepcounter{prog}
%                      Code_Saturne version 1.3
%                      ------------------------
%
%     This file is part of the Code_Saturne Kernel, element of the
%     Code_Saturne CFD tool.
%
%     Copyright (C) 1998-2007 EDF S.A., France
%
%     contact: saturne-support@edf.fr
%
%     The Code_Saturne Kernel is free software; you can redistribute it
%     and/or modify it under the terms of the GNU General Public License
%     as published by the Free Software Foundation; either version 2 of
%     the License, or (at your option) any later version.
%
%     The Code_Saturne Kernel is distributed in the hope that it will be
%     useful, but WITHOUT ANY WARRANTY; without even the implied warranty
%     of MERCHANTABILITY or FITNESS FOR A PARTICULAR PURPOSE.  See the
%     GNU General Public License for more details.
%
%     You should have received a copy of the GNU General Public License
%     along with the Code_Saturne Kernel; if not, write to the
%     Free Software Foundation, Inc.,
%     51 Franklin St, Fifth Floor,
%     Boston, MA  02110-1301  USA
%
%-----------------------------------------------------------------------
%
\programme{vortex}
%
\vspace{1cm}
%%%%%%%%%%%%%%%%%%%%%%%%%%%%%%%%%%
%%%%%%%%%%%%%%%%%%%%%%%%%%%%%%%%%%
\section{Fonction}
%%%%%%%%%%%%%%%%%%%%%%%%%%%%%%%%%%
%%%%%%%%%%%%%%%%%%%%%%%%%%%%%%%%%%
Ce sous-programme est d�di� � la g�n�ration des conditions d'entr�e
turbulente utilis�es en LES.


La m�thode des vortex est bas�e sur une approche de tourbillons
ponctuels. L'id�e de la m�thode consiste � injecter des tourbillons 2D dans le
plan d'entr�e du calcul, puis � calculer le champ de vitesse induit par ces
tourbillons au centre des faces d'entr�e.

%                      Code_Saturne version 1.3
%                      ------------------------
%
%     This file is part of the Code_Saturne Kernel, element of the
%     Code_Saturne CFD tool.
% 
%     Copyright (C) 1998-2007 EDF S.A., France
%
%     contact: saturne-support@edf.fr
% 
%     The Code_Saturne Kernel is free software; you can redistribute it
%     and/or modify it under the terms of the GNU General Public License
%     as published by the Free Software Foundation; either version 2 of
%     the License, or (at your option) any later version.
% 
%     The Code_Saturne Kernel is distributed in the hope that it will be
%     useful, but WITHOUT ANY WARRANTY; without even the implied warranty
%     of MERCHANTABILITY or FITNESS FOR A PARTICULAR PURPOSE.  See the
%     GNU General Public License for more details.
% 
%     You should have received a copy of the GNU General Public License
%     along with the Code_Saturne Kernel; if not, write to the
%     Free Software Foundation, Inc.,
%     51 Franklin St, Fifth Floor,
%     Boston, MA  02110-1301  USA
%
%-----------------------------------------------------------------------
%
%%%%%%%%%%%%%%%%%%%%%%%%%%%%%%%%%%
%%%%%%%%%%%%%%%%%%%%%%%%%%%%%%%%%%
\section{Discr\'etisation}
%%%%%%%%%%%%%%%%%%%%%%%%%%%%%%%%%%
%%%%%%%%%%%%%%%%%%%%%%%%%%%%%%%%%%

Le terme convectif en $\dive(\underline{u} \otimes \rho\,\underline{u})$
introduit une non lin\'earit\'e et un couplage des composantes de la vitesse
$\vect{u}$ dans l'�quation (\ref{Base_Preduv_eqqdm}). Une lin\'earisation et un d\'ecouplage
des trois composantes de la 
vitesse sont r\'ealis\'es lors de la discr\'etisation de cette \'etape de
pr\'ediction.\\
En effet, soit :
\begin{equation}
\vect{\widetilde{u}}= \vect{u}^n + \delta \vect{u} 
\end{equation}
La contribution exacte du terme convectif \`a prendre en compte dans cette
\'etape de pr\'ediction serait :\\
\begin{equation}\label{Base_Preduv_Conv_exact}
\begin{array}{ll}
\dive(\vect{\widetilde{u}} \otimes \rho\,\vect{\widetilde{u}}) =
\dive(\vect{u}^{n} \otimes \rho\,\vect{u}^{n}) + \dive(\delta \vect{u} \otimes
\rho\,\vect{u}^{n}) +  \underbrace { \dive(\vect{u}^{n} \otimes
\rho\,\delta \vect{u})}_{\text {terme couplant lin\'eaire}} +  \underbrace { \dive(\delta \vect{u} \otimes
\rho\,\delta \vect{u})}_{\text {terme couplant et non lin\'eaire}}\\
\end{array} 
\end{equation}
Les deux derniers termes de l'expression (\ref{Base_Preduv_Conv_exact}) sont {\em a priori} n�glig�s
de mani�re � obtenir un syst\`eme en vitesse qui soit d\'ecoupl\'e et donc,
�viter l'inversion d'une matrice pouvant \^etre de tr\`es grande taille. Ces
deux termes peuvent n�anmoins �tre pris en compte de mani�re plus ou moins
approch�e par extrapolation explicite du flux de masse en $n+\theta_F$ (pour le
terme couplant lin�aire provenant de la convection de $\vect{u}^{n}$ par $\delta
\vect{u}$) et utilisation d'un point-fixe par sous it�ration sur le sous
programme \fort{navsto} (pour le terme non-lin�aire, en sp�cifiant $\var{NTERUP}>1$).

L'�quation (\ref{Base_Preduv_eqqdm}) est discr�tis�e au temps $n+\theta$ � l'aide d'un
$\theta$-sch�ma, et le tenseur des pertes de charges d�compos� en une partie
diagonale $\tens{K}_{d}$ et une extradiagonale $\tens{K}_{e}$ (soit
 $\tens{K}_{pdc}=\tens{K}_{d}+\tens{K}_{e}$).\\
$\bullet$ La pression est suppos�e connue en $n-1+\theta$ (d�calage temporel
pression-vitesse) et le gradient naturellement calcul� � cet instant.\\ 
$\bullet$ Les termes sources de viscosit� secondaire, de gradient transpos\'e,
ceux provenant du mod�le de turbulence\footnote{except� $\dive (\mu_t\ (\ggrad
\underline {u}))$}, $\rho\,\tens{K}_{\,e}\ \underline{u}$, $(\rho -\rho_0)
\underline {g}$ ainsi que $\underline{T}_{s}^{\,exp}$ et
$\Gamma\,\underline{u}_{\,i}$ sont pris de mani�re explicite au temps $n$, ou
extrapol�s suivant le sch�ma en temps choisi pour les propri�t�s physique et les
termes sources.\\ 
$\bullet$ Les termes sources $\underline{u}\,\,\dive (\rho\,\underline {u})$,
$\Gamma\,\,\underline{u}$, $T_{s}^{\,imp}\,\,\underline{u}$ et
$-\rho\,\tens{K}_{\,d}\,\,\underline{u}$ sont implicit�s est calcul�s �
l'instant $n+\theta$.\\ 
$\bullet$ Le terme de diffusion $\dive (\mu_{\,tot}\,\ggrad \underline{u})$ est
implicit� : la vitesse est prise � l'instant $n+\theta$ et la viscosit�
explicit�e ou extrapol�e.\\ 
$\bullet$ Enfin, le terme de convection en $\dive(\,\underline{u} \otimes
(\rho\underline{u})\,)$ est implicit� : la composante r�solue de la vitesse est
prise en $n+\theta$, et le flux de masse, explicit�, ou extrapol� en
$n+\theta_F$. 

Par souci de clart�, on suppose, en l'absence d'indication, que les propri�tes
physiques $\Phi$ ($\rho,\,\mu_{tot},\,...$) et le flux de masse
$(\rho\underline{u})$ sont pris respectivement aux instants $n+\theta_\Phi$ et
$n+\theta_F$, o� $\theta_\Phi$ et $\theta_F$ d�pendent des sch�mas en temps
sp�cifiquement utilis�s pour ces grandeurs\footnote{cf. \fort{introd}}. 

La discr�tisation temporelle de l'�quation (\ref{Base_Preduv_eqqdm}) s'�crit alors comme suit : 

\begin{equation}\label{Base_Preduv_eq_di1}
 \begin{array}{c}
\displaystyle \rho\,\ \frac{ \underline {\widetilde{u}}^{n+1} -\underline {u}^{n} }
{\Delta t} + \dive(\,\underline{\widetilde{u}}^{n+\theta} \otimes (\rho\underline{u})\,) -\dive
(\mu_{\,tot}\,\ggrad \underline{\widetilde{u}}^{n+\theta}) =
\\
\displaystyle
 - \grad p^{n-1+\theta} + \dive (\rho\,\underline {u})\,\underline{\widetilde{u}}^{n+\theta} +(\Gamma\,\underline{u}_{\,i})^{n+\theta_S}-\Gamma^n\,\,\underline{\widetilde{u}}^{n+\theta}
\\
\begin{array}{c}
\displaystyle
- \rho\,\tens{K}_{\,d}^{n}\,\,\underline{\widetilde{u}}^{n+\theta} - (\rho\,\tens{K}_{\,e}\ \underline{u})^{n+\theta_S} + (\underline{T}_{s}^{\,exp})^{\,n+\theta_S} + T_{s}^{\,imp}\,\,\underline{\widetilde{u}}^{n+\theta}
\\
\displaystyle
+[\dive (\mu_{\,tot}\,^t\ggrad \underline {u})]^{n+\theta_S}-\frac {2} {3}[\,\grad (\mu_{\,tot}\,\dive \underline {u})]^{n+\theta_S} + (\rho -\rho_0) \underline {g}
 - (\underline{turb})^{n+\theta_S}
\end{array}
\end{array}
\end{equation}
o\`u, par souci de simplification, on a pos\'e :
\begin{equation}
\mu_{\,tot}=
\begin{cases}
\mu+\mu_t & \text{pour les mod�les � viscosit� turbulente ou en LES}, \\
\mu & \text{pour les mod�les au second ordre ou en laminaire}
\end{cases} \ 
\end{equation}
\\
et :
\begin{equation}
\underline{turb}^{n}=
\begin{cases}
\displaystyle\frac {2}{3}\grad (\rho^{n}\,k^{n}) & \text{pour les mod�les � viscosit� turbulente}, \\
\dive(\rho^{n}\,\tens{R}^n) & \text{pour les mod�les au second ordre},\\
0 & \text{en laminaire ou en LES}\\
\end{cases}
\end{equation}
Par analogie avec l'�criture du $\theta$-sch�ma pour une variable scalaire, $\,
\underline {\widetilde{u}}^{n+\theta}$ est interpol�e � partir de la vitesse
pr�dite $\underline {\widetilde{u}}^{n+1}$ de la mani\`ere suivante\footnote{si
$\theta=1/2$, ou qu'une extrapolation est utilis�e, l'ordre 2 n'est obtenu que si
le pas de temps $\Delta t$ est uniforme en temps et en espace.}~: 
\begin{equation}
\underline {\widetilde{u}}^{n+\theta}=\theta\, \underline
{\widetilde{u}}^{n+1}+(1-\theta)\, \underline {u}^{n}\\ 
\end{equation}
Avec :
\begin{equation}
\left\{
\begin{array}{ll}
\theta = 1   & \text{Pour un sch\'ema de type Euler implicite d'ordre 1.}\\
\theta = 1/2 & \text{Pour un sch\'ema de type Cranck-Nicolson d'ordre 2.}\\
\end{array}
\right.
\end{equation}

L'�quation (\ref{Base_Preduv_eq_di1}) est alors r��crite sous la forme :

\begin{equation}\label{Base_Preduv_eq_di2}
\begin{array}{c}
\displaystyle \underbrace{\left(\frac{\rho}{\Delta t} -\theta \,\dive (\rho\,\underline {u}) +\theta \,\, \Gamma^n +
\theta \,\, \rho\,\tens{K}_{\,d}^n-\theta \,T_s^{\,imp} \right)}_{\displaystyle f_s^{imp}}\, (\underline {\,\widetilde{u}}^{n+1} -\underline {u}^{n})
\\
 +\, \theta\, \dive(\underline {\widetilde{u}}^{n+1} \otimes (\rho\underline{u}))-\, \theta\,\dive (\mu_{\,tot}\,\ggrad \underline {\widetilde{u}}^{n+1}) =
\\
-\,(1-\theta)\, \dive(\underline {u}^{n} \otimes (\rho\underline{u})) +\,(1-\theta)\,\dive (\mu_{\,tot}\,\ggrad \underline {u}^{n})
\\
f_s^{exp}\left\{
\begin{array}{c}
\displaystyle 
- \grad p^{n-1+\theta} + \dive (\rho\,\underline {u})\,\underline{u}^{n} +\,(\,\Gamma^{n}\,\underline{u}_{\,i}\,)^{n+\theta_S}- \Gamma^n\,\,\underline{u}^{n}
\\
\displaystyle
-(\,\rho\,\tens{K}_{\,e}\ \underline{u}\,)^{n+\theta_S} -\rho\,\tens{K}_{\,d}^n\ \underline{u}^{n}+ (\underline{T}_{s}^{\,exp})^{\,n+\theta_S} + T_s^{\,imp}\,\,\underline {u}^{n} 
\\
\displaystyle
+[\dive (\mu_{\,tot}\,^t\ggrad \underline {u}\,)]^{n+\theta_S}-\frac {2} {3}[\,\grad (\mu_{\,tot}\,\dive \underline {u}\,)]^{n+\theta_S} + (\rho -\rho_0) \underline {g}-(\underline{turb})^{n+\theta_S}
\end{array}
\right.
\end{array}
\end{equation}

d'o� l'�quation r�solue par le sous-programme \fort{codits} :
\begin{equation}\begin{array}{c}
\displaystyle
f_s^{\,imp}(\underline {\widetilde{u}}^{n+1}-\underline {u}^{n}) + \theta\, \dive(\underline{\widetilde{u}}^{n+1} \otimes (\rho
\underline{u})) - \theta\,\dive (\,\mu_{\,tot}\,\ggrad \underline{\widetilde{u}}^{n+1}) = 
\\\\
\displaystyle
-(1-\theta)\,\dive(\underline{u}^{n} \otimes (\rho \underline{u}))+(1-\theta)\,\dive (\,\mu_{\,tot}\,\ggrad \underline{u}^{n})
+ \underline{f}_{\,s}^{\,exp}
\end{array}
\end{equation}
La m\'ethode de discr\'etisation spatiale est d\'evelopp\'ee dans le sous-programme \fort{codits}.\\



\minititre{Remarques :}
{\tiny$\blacksquare$} Dans le cas standard sans extrapolation, le terme
$-\,T_s^{\,imp}$ n'est ajout� � $f_s^{\,imp}$ que s'il est positif afin de ne
pas affaiblir la dominance de la diagonale de la matrice � inverser.\\ 
{\tiny$\blacksquare$} Si une extrapolation est utilis�e, par contre,
$\,T_s^{\,imp}$ est ajout� � $f_s^{\,imp}$ quel que soit son signe. En effet, l'id�e intuitive qui
consiste � prendre~: 
\begin{equation}
\begin{cases}
\displaystyle
(\underline{T}_{s}^{\,exp} + T_{s}^{\,imp}\,\underline {u})^{\,n+\theta_S} &
\text{si } T_{s}^{\,imp} > 0\\ 
\displaystyle
(\underline{T}_{s}^{\,exp})^{\,n+\theta_S} + T_{s}^{\,imp}\,\underline{u}^{n+\theta} &\text{sinon}\\
\end{cases}
\end{equation} 
aboutit � une incoh�rence dans le traitement si $T_s^{imp}$ change de signe
entre deux pas de temps.\\ 
{\tiny$\blacksquare$} la partie diagonale $\tens{K}_{\,d}$ du terme
de perte de charge est utilis�e dans $f_s^{\,imp}$. En fait, pour \^etre rigoureux,
il faudrait ne retenir que les contributions positives (point signal\'e dans le
sous-programme utilisateur associ\'e \fort{uskpdc}). Cette prise en compte sera \`a am\'eliorer.\\
{\tiny$\blacksquare$} Le terme $\theta\,\Gamma^{n}-\theta\,\dive
(\rho\,\underline {u})$ ne pose pas de probl�me pour la 
dominance de la diagonale de la matrice car il est exactement compens� par le
terme de convection (cf. \fort{covofi}). 


%                      Code_Saturne version 1.3
%                      ------------------------
%
%     This file is part of the Code_Saturne Kernel, element of the
%     Code_Saturne CFD tool.
%
%     Copyright (C) 1998-2007 EDF S.A., France
%
%     contact: saturne-support@edf.fr
%
%     The Code_Saturne Kernel is free software; you can redistribute it
%     and/or modify it under the terms of the GNU General Public License
%     as published by the Free Software Foundation; either version 2 of
%     the License, or (at your option) any later version.
%
%     The Code_Saturne Kernel is distributed in the hope that it will be
%     useful, but WITHOUT ANY WARRANTY; without even the implied warranty
%     of MERCHANTABILITY or FITNESS FOR A PARTICULAR PURPOSE.  See the
%     GNU General Public License for more details.
%
%     You should have received a copy of the GNU General Public License
%     along with the Code_Saturne Kernel; if not, write to the
%     Free Software Foundation, Inc.,
%     51 Franklin St, Fifth Floor,
%     Boston, MA  02110-1301  USA
%
%-----------------------------------------------------------------------
%

%%%%%%%%%%%%%%%%%%%%%%%%%%%%%%%%%%
%%%%%%%%%%%%%%%%%%%%%%%%%%%%%%%%%%
\section{Mise en \oe uvre}
%%%%%%%%%%%%%%%%%%%%%%%%%%%%%%%%%%
%%%%%%%%%%%%%%%%%%%%%%%%%%%%%%%%%%
La num\'ero de la phase trait\'ee fait partie des arguments de \fort{turrij}. On
omettra volontairement de le pr\'eciser dans ce qui suit, on indiquera par $(\ )$ la
notion de tableau s'y rattachant.

\etape{Calcul des termes de production $\tens{\mathcal{P}}$}
\begin{itemize}
\item [$\star$] Initialisation \`a z\'ero du tableau \var{PRODUC} dimensionn\'e \`a $\var{NCEL}\times 6$.
\item [$\star$] On appelle trois fois \fort{grdcel} pour calculer les gradients des composantes de la vitesse $u$, $v$ et
$w$ prises au temps $n$.

Au final, on a :\\
$\displaystyle
\begin{array} {ll}
\var{PRODUC(1,IEL)} = & \displaystyle - 2 \left[ R_{11}^{\,n} \frac{\partial u^{\,n}} {\partial x} +R_{12}^{\,n} \frac{\partial u^{\,n}} {\partial y}+R_{13}^{\,n} \frac{\partial u^{\,n}} {\partial z} \right] \text{        (production de $R_{11}^{\,n}$)}\\
\var{PRODUC(2,IEL)} = & \displaystyle - 2 \left[ R_{12}^{\,n} \frac{\partial v^{\,n}} {\partial x} +R_{22}^{\,n} \frac{\partial v^{\,n}} {\partial y}+R_{23}^{\,n} \frac{\partial v^{\,n}} {\partial z} \right] \text{        (production de $R_{22}^{\,n}$)}\\
\var{PRODUC(3,IEL)} = & \displaystyle - 2 \left[ R_{13}^{\,n} \frac{\partial w^{\,n}} {\partial x} +R_{23}^{\,n} \frac{\partial w^{\,n}} {\partial y}+R_{33}^{\,n} \frac{\partial w^{\,n}} {\partial z} \right] \text{        (production de $R_{33}^{\,n}$)}\\
\var{PRODUC(4,IEL)} = & \displaystyle - \left[ R_{12}^{\,n} \frac{\partial u^{\,n}} {\partial x} +R_{22}^{\,n} \frac{\partial u^{\,n}} {\partial y}+R_{23}^{\,n} \frac{\partial u^{\,n}} {\partial z} \right] \\
& \displaystyle - \left[ R_{11}^{\,n} \frac{\partial v^{\,n}} {\partial x} +R_{12}^{\,n} \frac{\partial v^{\,n}} {\partial y}+R_{13}^{\,n} \frac{\partial v^{\,n}} {\partial z} \right] \text{        (production de $R_{12}^{\,n}$)} \\
\var{PRODUC(5,IEL)} = & \displaystyle - \left[ R_{13}^{\,n} \frac{\partial u^{\,n}} {\partial x} +R_{23}^{\,n} \frac{\partial u^{\,n}} {\partial y}+R_{33}^{\,n} \frac{\partial u^{\,n}} {\partial z} \right] \\
& \displaystyle - \left[ R_{11}^{\,n} \frac{\partial w^{\,n}} {\partial x} +R_{12}^{\,n} \frac{\partial w^{\,n}} {\partial y}+R_{13}^{\,n} \frac{\partial w^{\,n}} {\partial z} \right] \text{        (production de $R_{13}^{\,n}$)} \\
\var{PRODUC(6,IEL)} = & \displaystyle - \left[ R_{13}^{\,n} \frac{\partial v^{\,n}} {\partial x} +R_{23}^{\,n} \frac{\partial v^{\,n}} {\partial y}+R_{33}^{\,n} \frac{\partial v^{\,n}} {\partial z} \right] \\
& \displaystyle - \left[ R_{12}^{\,n} \frac{\partial w^{\,n}} {\partial x} +R_{22}^{\,n} \frac{\partial w^{\,n}} {\partial y}+R_{23}^{\,n} \frac{\partial w^{\,n}} {\partial z} \right]  \text{        (production de $R_{23}^{\,n}$)}
\end{array}
$
\end{itemize}

\etape{Calcul du gradient de la masse volumique $\rho^n$ prise au d\'ebut du pas
de temps courant\footnote{{\it i.e.} calcul\'ee \`a partir des
variables du pas de temps pr\'ec\'edent $n$ si n\'ecessaire.} $(n+1)$}
Ce calcul n'a lieu que si les termes de gravit\'e doivent \^etre pris en compte
($\var{IGRARI()} =1$).
\begin{itemize}
\item [$\star$] Appel de \fort{grdcel}  pour calculer le gradient de $\rho^n$
dans les trois directions de l'espace. Les conditions aux limites sur $\rho^n$
sont des conditions de Dirichlet puisque la valeur de $\rho^n$ aux faces de bord
$ik$ (variable \var{IFAC}) est connue et vaut $\rho_{\,b_{\,ik}}$. Pour \'ecrire les conditions aux limites
sous la forme habituelle, $$\rho_{\,b_{\,ik}} = \var{COEFA} + \var{COEFB}
\,\rho^n_{\,I'}$$ on pose alors $\var{COEFA}=
\var{PROPCE(IFAC,IPPROB(IROM(IPHAS)))}$ et $\var{COEFB} = \var{VISCB} = 0$.\\
$\var{PROPCE(1,IPPROB(IROM(IPHAS)))}$ (resp.$\var{VISCB}$) est utilis\'e en lieu
et place de l'habituel \var{COEFA} ($\var{COEFB}$), lors de l'appel \`a \fort{grdcel}.\\
On a donc :\\
$\displaystyle \var{GRAROX}= \frac{\partial \rho^n}{\partial x}\ $,$\displaystyle \ \var{GRAROY}= \frac{\partial
\rho^n}{\partial y}$ et $
\displaystyle \ \var{GRAROZ}= \frac{\partial \rho^n}{\partial z}\ $.

\end{itemize}

Le gradient de $\rho^n$ servira \`a calculer les termes de production par effets de gravit\'e si ces derniers sont pris en compte.

\etape{Boucle \var{ISOU} de $1$ \`a $6$ sur les tensions de Reynolds}
Pour $\var{ISOU} = 1,2,3,4,5,6$, on r\'esout respectivement et dans
l'ordre  les
\'equations de $R_{11}$, $R_{22}$, $R_{33}$, $R_{12}$, $R_{13}$ et $R_{23}$ par
l'appel au sous-programme \fort{resrij}.\\
La r\'esolution se fait par incr\'ement $\delta {R}_{ij}^{\,n+1,k+1}$ , en utilisant la m\^eme m\'ethode que
celle d\'ecrite dans le sous-programme \fort{codits}. On adopte ici les m\^emes notations.
\var{SMBR} est le second membre du syst\`eme \`a inverser, syst\`eme portant sur
les incr\'ements de la variable. \var{ROVSDT} repr\'esente la diagonale de la
matrice, hors convection/diffusion.\\
On va r\'esoudre l'\'equation (\ref{Base_Turrij_Eq_Temp_Rij}) sous forme incr\'ementale en
utilisant \fort{codits}, soit :
\begin{equation}\label{Base_Turrij_Eq_Temp_deltaRij}
\begin{array}{ll}
&\displaystyle \underbrace{\left(\frac {\rho^n_L}{\Delta t^n}
+ \rho^n_L \,C_1\,\frac{\varepsilon^n_L}{k^n_L}(1-\frac{\delta_{ij}}{3})
 - m^n_{\,lm} + \Gamma_L\,+ max(-\alpha^n_{R_{ij}},0)\right)\,|\Omega_l|}
_{\text {$\var{ROVSDT}$ contribuant
\`a la diagonale de la matrice simplifi\'ee de \fort{matrix}}}\,(\delta{R}_{ij}^{\,n+1,p+1})_{\,L}\\\\
&  \underbrace{+\sum\limits_{m\in Vois(l)}\displaystyle \left[
 m^n_{\,lm} \delta R_{ij,\,f_{\,lm}}^{\,n+1,p+1}
- (\mu^n_{\,lm} + \gamma^n_{\,lm})\
\frac{({\delta R}_{ij}^{\,n+1,p+1})_{M}-({\delta R}_{ij}^{\,n+1,p+1})_{L})}{\overline{L'M'}}\,
S_{\,lm} \right]}_{\text { convection upwind pur et diffusion non reconstruite
relatives \`a la matrice simplifi\'ee de \fort{matrix}\footnotemark}} \\
% voir le texte de la footmark plus bas
&= - \displaystyle\frac {\rho^n_L}{\Delta t^n}\,\left(\,(R^{\,n+1,p}_{ij})_L - (R^{\,n}_{ij})_L\,\right)\\
&-\,\underbrace{\displaystyle\int_{\Omega_l} \left(
\dive\,[\,(\rho\,\vect{u})^n\,R^{\,n+1,p}_{ij} - (\mu^n\,+ \gamma^n\,)\,
\grad{R^{\,n+1,p}_{ij}}\,]\right)\,d\Omega}_{\text {convection et diffusion
trait\'ees par \fort{bilsc2}}}\\
&+\displaystyle \int_{\Omega_l} \left[\,\mathcal{P}^{\,n+1,p}_{ij} + \mathcal{G}^{\,n+1,p}_{ij}
- \displaystyle\rho^n \,C_1\,\frac{\varepsilon^n}{k^n}\left[R^{\,n+1,p}_{ij}-
\frac{2}{3}\,k^n\,\delta_{ij}\right] + \phi^{\,n+1,p}_{ij,2} +
\phi^{\,n+1,p}_{ij,w}\,\right]\, d\Omega \\
& + \displaystyle\int_{\Omega_l} \left[- \frac{2}{3} \rho^n \varepsilon^n \delta_{ij}
 + \Gamma\,(\,R^{\,in}_{ij} - R^{\,n+1,p}_{ij}\,) +
\alpha^n_{R_{ij}}\,R^{\,n+1,p}_{ij}+ \beta^n_{R_{ij}}\right]\, d\Omega\\
&+ \sum\limits_{m\in
Vois(l)}\displaystyle \left[\ \tens{E}^n\,\grad{R}^{\,n+1,p}_{ij} \right]_{\,lm}\,.\,\vect{n}_{\,lm}S_{\,lm}\\
&+ \sum\limits_{m\in Vois(l)}\displaystyle \left[\
\tens{D}^n\,\grad{R}^{\,n+1,p}_{ij} \right]_{\,lm}\,.\,\vect{n}_{\,lm}S_{\,lm}\\
&- \sum\limits_{m\in Vois(l)} \gamma^n_{\,lm} \left( \grad{R}^{\,n+1,p}_{ij}\,.\,\vect{n}_{\,lm} \right)  S_{\,lm}\\
&+ \sum\limits_{m\in Vois(l)}  m^n_{\,lm}\,(R^{\,n+1,p}_{ij})_L\\
\end{array}
\end{equation}
% si on ne fait pas comme ca, il n'apparait pas
\footnotetext[\thefootnote]{Si $\var{IRIJNU} = 1$, on remplace  $\mu^n_{\,lm}$ par $(\mu +
\mu_t)^n_{\,lm}$ dans l'expression de la diffusion non reconstruite
associ\'ee \`a la matrice simplifi\'ee de \fort{matrix} ($\mu_t$ d\'esigne la
viscosit\'e turbulente calcul\'ee comme en $k-\varepsilon$).}

o\`u on rappelle :\\
pour $n$ donn\'e entier positif, on d\'efinit la suite
 $({R}_{ij}^{\,n+1,p})_{p \in \grandN}$
 par :
\begin{equation}\notag
\left\{\begin{array}{l}
{R}_{ij}^{\,n+1,0} = {R}_{ij}^{\,n}\\
{R}_{ij}^{\,n+1,p+1} = {R}_{ij}^{\,n+1,p} + \delta{R}_{ij}^{\,n+1,p+1} \\
\end{array}\right.
\end{equation}
$(\delta{R}_{ij}^{\,n+1,p+1})_{\,L}$ d\'esigne la valeur sur l'\'el\'ement
$\Omega_l$ du $\text{$(\,p+1\,)$-i\`eme}$ incr\'ement de ${R}_{ij}^{\,n+1}$,
$ m^n_{\,lm}$ le flux de masse \`a l'instant $n$ \`a travers la face $lm$,
$\delta R_{ij,\,f_{\,lm}}^{\,n+1,p+1}$ vaut $({\delta
R}_{ij}^{\,n+1,p+1})_{L}$  si $ m^n_{\,lm} \geqslant 0$, $({\delta
R}_{ij}^{\,n+1,p+1})_{M}$ sinon,
$\mathcal{P}^{\,n+1,p}_{ij}$, $\phi^{\,n+1,p}_{ij,2}$, $\phi^{\,n+1,p}_{ij,w}$ les valeurs
des quantit\'es associ\'ees correspondant \`a l'incr\'ement
$(\delta{R}_{ij}^{\,n+1,p})$.\\



Tous ces termes sont calcul\'es comme suit :
\begin{itemize}
\item Terme de gauche de l'\'equation (\ref{Base_Turrij_Eq_Temp_deltaRij})\\
Dans \fort{resrij} est calcul\'ee la variable \var{ROVSDT}. Les autres
termes sont compl\'et\'es par \fort{codits}, lors de la construction de la matrice simplifi\'ee , {\it via} un
appel au sous-programme \fort{matrix}. La quantit\'e
 $(\mu^n_{\,lm} + \gamma^n_{\,lm})$ \`a la face $lm$ est calcul\'ee lors de l'appel \`a
\fort{visort}.\\
\item Second membre de l'\'equation (\ref{Base_Turrij_Eq_Temp_deltaRij})\\
Le premier terme non d\'etaill\'e est calcul\'e par le sous-programme
\fort{bilsc2}, qui applique le sch\'ema convectif choisi par l'utilisateur, qui
reconstruit ou non selon le souhait de l'utilisateur les gradients intervenants
dans la convection-diffusion.\\
Les termes sans accolade sont, eux, compl\`etement explicites et ajout\'es au fur et
\`a mesure dans \var{SMBR} pour former
l'expression $f^{\,exp}_s$ de \fort{codits}.
\end{itemize}
On d\'ecrit ci-dessous les \'etapes de \fort{resrij} :
\begin{itemize}

\item DELTIJ = 1, pour $\var{ISOU} \leqslant 3$ et DELTIJ = 0  Si $\var{ISOU} >
3$. Cette valeur repr\'esente le symbole de Kroeneker $\delta_{ij}$.

\item Initialisation \`a z\'ero de \var{SMBR} (tableau contenant le second
membre) et \var{ROVSDT} (tableau contenant la diagonale de la matrice sauf celle
relative \`a la contribution de la
diagonale des op\'erateurs de convection et de diffusion lin\'earis\'es
\footnote{qui correspondent aux sch\'emas convectif upwind pur et diffusif sans
reconstruction.}), tous deux de dimension $\var{NCEL}$.

\item Lecture et prise en compte des termes sources utilisateur pour la variable $R_{ij}$

Appel \`a \fort{ustsri} pour charger les termes sources utilisateurs. Ils sont
stock\'es comme suit. Pour la cellule $\Omega_l$ de centre $L$, repr\'esent\'ee par $\var{IEL}$, on a :\\
\begin{equation}\notag
\left\{\begin{array}{lll}
&\var{ROVSDT(IEL)}&= |\Omega_l| \ \alpha_{R_{ij}}\\
&\var{SMBR(IEL)}&=|\Omega_l| \ \beta_{R_{ij}}\\
\end{array}\right.
\end{equation}
On affecte alors les valeurs ad\'equates au second membre \var{SMBR} et \`a la
diagonale \var{ROVSDT} comme suit :
\begin{equation}\notag
\left\{\begin{array}{lll}
&\var{SMBR(IEL)} &= \var{SMBR(IEL)} +\ |\Omega_l| \ \alpha_{R_{ij}} \ (R^n_{ij})_L \\
&\var{ROVSDT(IEL)}&= \text{max }(-\ |\Omega_l| \ \alpha_{R_{ij}},0)\\
\end{array}\right.
\end{equation}
La valeur de $ \var{ROVSDT}$ est ainsi calcul\'ee pour des raisons de stabilit\'e
num\'erique. En effet, on ne rajoute sur la diagonale que les valeurs positives,
ce qui correspond physiquement \`a impliciter les termes de rappel uniquement.
\item{Calcul du terme source de masse  si $\Gamma_L > 0$}

Appel de \fort{catsma} et incr\'ementation si n\'ecessaire de \var{SMBR} et
\var{ROVSDT} {\it via} :\\
\begin{equation}\notag
\left\{\begin{array}{lll}
\displaystyle \var{SMBR(IEL)} = \var{SMBR(IEL)} + |\Omega_l| \ \Gamma_L \
\left[(R^{\,in}_{ij})_L - (R^{\,n}_{ij})_L \right] \\
\displaystyle \var{ROVSDT(IEL)}=\var{ROVSDT(IEL)} + |\Omega_l| \ \Gamma_L
\end{array}\right.
\end{equation}
\item Calcul du terme d'accumulation de masse et du terme instationnaire

On stocke $\displaystyle \var{W1}= \int_{\Omega_l}\dive\,(\rho\,\vect{u})\,d\Omega$
calcul\'e par \fort{divmas} \`a l'aide des flux de masse aux faces internes
$ m^n_{\,lm}=\sum\limits_{m\in Vois(l)}{(\rho \vect{u})_{\,lm}^n} \text{.}\,
\vect{S}_{\,lm} $ (tableau \var{FLUMAS}) et des flux de masse aux bords  $ m^n_{\,b_{lk}} = \sum\limits_{k\in{\gamma_b(l)}}{(\rho \vect{u})_{\,{b}_{lk}}^n} \text{.}\,
\vect{S}_{\,{b}_{lk}} $ (tableau \var{FLUMAB}).
On incr\'emente ensuite \var{SMBR} et \var{ROVSDT}.
\begin{equation}\notag
\left\{\begin{array}{lll}
&\var{SMBR(IEL)} &= \var{SMBR(IEL)} + \var{ICONV}\  (R^n_{ij})_L\,(\displaystyle
\int_{\Omega_l}\dive\,(\rho\,\vect{u})\ d\Omega) \\
&\var{ROVSDT(IEL)}& = \var{ROVSDT(IEL)} +  \var{ISTAT}\,\displaystyle
\frac{\rho^n_L \ |\Omega_l|}{\Delta t^n} -  \var{ICONV}\ (\displaystyle
\int_{\Omega_l}\dive\,(\rho\,\vect{u})\ d\Omega) \\
\end{array}\right.
\end{equation}
\item Calcul des termes sources de production, des termes $\displaystyle
\phi_{\,ij,1}+\phi_{\,ij,2}$ et de la dissipation~$\displaystyle-\frac{2}{3} \varepsilon\,\delta_{\,ij}$ :

On effectue une boucle d'indice \var{IEL} sur les cellules $\Omega_l$ de centre $L$ :
\begin{itemize}
\item [$\Rightarrow$] $\displaystyle \var{TRPROD}= \frac{1}{2} (\mathcal{P}^n_{ii})_L = \frac{1}{2} \left[ \var{PRODUC(1,IEL)} +  \var{PRODUC(2,IEL)} +  \var{PRODUC(3,IEL)} \right] $
\item [$\Rightarrow$] $\displaystyle \var{TRRIJ }= \frac{1}{2} (R^n_{ii})_L $
\item [$\Rightarrow$] $\displaystyle \var{SMBR(IEL)} =\ \var{SMBR(IEL)}\ +$\\
$\ \displaystyle\rho^n_L |\Omega_l| \left[ \displaystyle
\frac{2}{3}\,\delta_{\,ij} \left( \ \displaystyle \frac{ C_2}{2}\,(\mathcal{P}^n_{ii})_L\ +
(C_1-1)\ \varepsilon^n_L\, \right)\right.$\\
$ + \left.\ (1-C_2) \ \var{PRODUC(ISOU,IEL)} -
\displaystyle C_1\ \frac{2\,\varepsilon^n_L}{(R^n_{ii})_L}\ (R^n_{ij})_L \right]$
\item [$\Rightarrow$] $\displaystyle \var{ROVSDT(IEL)} = \var{ROVSDT(IEL)} +
\rho^n_L \ |\Omega_l| \ (- \displaystyle \frac{1}{3} \ \,\delta_{\,ij} + 1) \ C_1
\ \frac{2\ \varepsilon^n_L}{(R^n_{ii})_L}$
\end{itemize}
\item Appel de \fort{rijech} pour le calcul des termes d'\'echo de paroi
 $\phi^n_{ij,w}$ si $\var{IRIJEC()}=1$ et ajout dans \var{SMBR}.\\
$\var{SMBR} = \var{SMBR} + \phi^n_{ij,w}$\\
Suivant son mode de calcul (\var{ICDPAR}), la distance � la paroi est directement accessible
par \var{RA(IDIPAR+IEL-1)} (\var{|ICDPAR|} = 1) ou bien
est calcul\'ee \`a partir de $\var{IA(IIFAPA(IPHAS)+IEL - 1)}$,
qui donne pour l'\'el\'ement $\var{IEL}$ le num\'ero de la face de bord
paroi la plus  proche (\var{|ICDPAR|} = 2). Ces tableaux ont \'et\'e renseign\'e une fois pour toutes au
d\'ebut de calcul.

\item  Appel de \fort{rijthe} pour le calcul des termes de gravit\'e $\mathcal{G}^n_{ij}$ et ajout dans \var{SMBR}.

Ce calcul n'a lieu que si $\var{IGRARI()} = 1$.
$ \var{SMBR} = \var{SMBR} + \mathcal{G}^n_{ij}$
\item Calcul de la partie explicite du terme de diffusion
 $\dive{\,\left[\tens{A}\,\grad{R}^{\,n}_{ij}\right]}$, plus pr\'ecis\'ement
des contributions du terme extradiagonal pris aux faces purement internes
(remplissage du tableau \var{VISCF}), puis aux faces de bord (remplissage du
tableau \var{VISCB}).
\begin{itemize}
\item [$\star$] Appel de \fort{grdcel} pour le calcul du gradient de
$R^{\,n}_{ij}$ dans chaque direction. Ces gradients sont respectivement
stock\'es dans les tableaux de travail \var{W1}, \var{W2} et \var{W3}.

\item [$\star$] boucle d'indice \var{IEL} sur les cellules $\Omega_l$ de centre
$L$ pour le
calcul de $\tens{E}^n\,\grad{R}^{\,n}_{ij}$ aux cellules dans un premier temps :\\
\begin{itemize}
\item [$\Rightarrow$] $\displaystyle \var{TRRIJ}= \frac{1}{2} (R^{\,n}_{ii})_L $
\item [$\Rightarrow$] $\displaystyle \var{CSTRIJ} = \rho^n_L\ C_S \ \displaystyle\frac{(R^n_{ii})_L}{2\,\varepsilon^n_L}$
\item [$\Rightarrow$] $\displaystyle \var{W4(IEL)} = \rho^n_L\ C_S\
\displaystyle\frac{(R^n_{ii})_L}{2\,\varepsilon^n_L} \left[\,(R^{\,n}_{12})_L \ \var{W2(IEL)} +
(R^{\,n}_{13})_L \ \var{W3(IEL)}\,\right]$
\item [$\Rightarrow$] $\displaystyle \var{W5(IEL)} = \rho^n_L\ C_S\
\displaystyle\frac{(R^n_{ii})_L}{2\,\varepsilon^n_L} \left[\,(R^{\,n}_{12})_L \ \var{W1(IEL)} +
(R^{\,n}_{23})_L \ \var{W3(IEL)}\,\right]$
\item [$\Rightarrow$] $\displaystyle \var{W6(IEL)} = \rho^n_L\ C_S\
\displaystyle\frac{(R^n_{ii})_L}{2\,\varepsilon^n_L} \left[\,(R^{\,n}_{13})_L \ \var{W1(IEL)} + (R^{\,n}_{23})_L \ \var{W2(IEL)}\,\right]$
\end{itemize}



\item [$\star$] Appel de \fort{vectds}\footnote{Le r\'esultat est stock\'e dans
\var{VISCF} et \var{VISCB}. Dans \fort{vectds}, les valeurs aux faces internes
sont interpol\'ees lin\'eairement sans reconstruction et \var{VISCB} est mis \`a
z\'ero.} pour assembler $\displaystyle\left[ \tens{E}^n\,\grad{R}^{\,n}_{ij}
\right]\,.\,\vect{n}_{\,lm}S_{\,lm}$ aux faces $lm$.
\item [$\star$] Appel de \fort{divmas} pour calculer la divergence du flux d\'efini par \var{VISCF} et \var{VISCB}.
Le r\'esultat est stock\'e dans \var{W4}.\\
Ajout au second membre \var{SMBR}.\\
\var{SMBR} = \var{SMBR} + \var{W4}
\end{itemize}

A l'issue de cette \'etape, seule la partie extradiagonale de la diffusion prise
enti\`erement explicite~:
 $$\sum\limits_{m\in
Vois(l)}\left[\ \tens{E}^n\,\grad{R}^{\,n}_{ij} \right]_{\,lm}\,.\,\vect{n}_{\,lm}S_{\,lm}$$ a \'et\'e calcul\'ee.\\

\item Calcul de la partie diagonale du terme de diffusion\footnote{Seule la
composante normale  du  gradient de $R_{ij}$ aux faces sera implicite.} :\\
Comme on l'a d\'eja vu, une partie de cette contribution sera trait\'ee en
implicite (celle relative \`a la composante normale du gradient) et donc
ajout\'ee au second membre par \fort{bilsc2} ; l'autre
partie sera explicite et prise en compte dans $f_s^{\,exp}$.
\begin{itemize}
\item [$\star$] On effectue une boucle d'indice \var{IEL} sur les cellules
$\Omega_l$ de centre $L$ :
\begin{itemize}
\item [$\Rightarrow$] $\displaystyle \var{TRRIJ }= \frac{1}{2} (R^{\,n}_{ii})_L $
\item [$\Rightarrow$] $\displaystyle \var{CSTRIJ} = \rho^n_L \ C_S \ \frac{(R^{\,n}_{ii})_L}{2\,\varepsilon^n_L}$
\item [$\Rightarrow$] $\displaystyle \var{W4(IEL)} = \rho^n_L \ C_S \
\frac{(R^{\,n}_{ii})_L}{2\,\varepsilon^n_L} \ (R^{\,n}_{11})_L$
\item [$\Rightarrow$] $\displaystyle \var{W5(IEL)} = \rho^n_L \ C_S \ \frac{(R^{\,n}_{ii})_L}{2\,\varepsilon^n_L}\ (R^n_{22})_L$
\item [$\Rightarrow$] $\displaystyle \var{W6(IEL)} = \rho^n_L \ C_S \ \frac{(R^{\,n}_{ii})_L}{2\,\varepsilon^n_L} \ (R^n_{33})_L$
\end{itemize}

%\item Traitement du parall\'elisme et de la p\'eriodicit\'e.

\item [$\star$] On effectue une boucle d'indice \var{IFAC} sur les faces
purement internes $lm$ pour remplir le tableau \var{VISCF} :
\begin{itemize}
\item [$\Rightarrow$] Identification des cellules $\Omega_l$ et $\Omega_m$ de
centre respectif $L$ (variable \var{II}) et $M$ (variable \var{JJ}), se trouvant de chaque c\^ot\'e de la face
$lm$\footnote{La normale \`a la face est orient\'ee de L vers M.}.
\item [$\Rightarrow$] Calcul du carr\'e de la surface de la face. La valeur est
stock\'ee dans le tableau \var{SURFN2}.
\item [$\Rightarrow$] Interpolation du gradient de $R^{\,n}_{ij}$ \`a la face
$lm$ (gradient facette $\left[\grad{R}^{\,n}_{ij}\right]_{\,lm}$) :
\begin{equation}\notag
\left\{\begin{array}{ll}
\var{GRDPX} &= \displaystyle \frac{1}{2} \left(\var{W1(II)} + \var{W1(JJ)}
\right) \\
&\\
\var{GRDPY} &= \displaystyle \frac{1}{2} \left(\var{W2(II)} + \var{W2(JJ)} \right) \\
&\\
\var{GRDPZ} &= \displaystyle \frac{1}{2} \left(\var{W3(II)} + \var{W3(JJ)} \right)
\end{array}\right.
\end{equation}
\item [$\Rightarrow$] Calcul du gradient de $R^{\,n}_{ij}$ normal \`a la face
$lm$, $\left[\grad{R}^{\,n}_{ij}\right]_{\,lm}.\vect{n}_{\,lm}\,S_{\,lm}$.\\

$\displaystyle \var{GRDSN} =  \var{GRDPX} \ \var{SURFAC(1,IFAC)} + \var{GRDPY} \ \var{SURFAC(2,IFAC)} +  \var{GRDPZ} \ \var{SURFAC(3,IFAC)}$
$\var{SURFAC}$ est le vecteur surface de la face \var{IFAC}.


\item [$\Rightarrow$] calcul de
 $\left[\grad{R^{\,n}_{ij}} - (\grad
R^{\,n}_{ij}\,.\,\vect{n}_{\,lm})\vect{n}_{\,lm}\right]$, les vecteurs \'etant
calcul\'es \`a la face $lm$ :
\begin{equation}\notag
\left\{\begin{array}{lll}
&\displaystyle \var{GRDPX} &= \var{GRDPX} - \displaystyle\frac{\var{GRDSN}}{\var{SURFN2}} \ \var{SURFAC(1,IFAC)}\\
&&\\
&\displaystyle \var{GRDPY} &= \var{GRDPY} - \displaystyle\frac{\var{GRDSN}}{\var{SURFN2}} \ \var{SURFAC(2,IFAC)} \\
&&\\
&\displaystyle \var{GRDPZ} &= \var{GRDPZ} - \displaystyle\frac{\var{GRDSN}}{\var{SURFN2}} \ \var{SURFAC(3,IFAC)}
\end{array}\right.
\end{equation}
\item [$\Rightarrow$] finalisation du calcul de l'expression totalement
explicite
 $$\left[ \tens{D}^n\,\left( \grad{R^{\,n}_{ij}} - (\grad R^{\,n}_{ij}\,.\,\vect{n}_{\,lm})\,\vect{n}_{\,lm}\right) \right]\,.\,\vect{n}_{\,lm}$$
\begin{equation}\notag
\begin{array} {ll}
\displaystyle \var{VISCF} = &
 \displaystyle\frac{1}{2} (\ \var{W4(II)} +\ \var{W4(JJ)}) \ \var{GRDPX} \
\var{SURFAC(1,IFAC)})\ + \\
&\\
&  \displaystyle\frac{1}{2} (\ \var{W5(II)} +\ \var{W5(JJ)}) \ \var{GRDPY} \
\var{SURFAC(2,IFAC)})\ + \\
&\\
&  \displaystyle\frac{1}{2} (\ \var{W6(II)} +\ \var{W6(JJ)}) \ \var{GRDPZ} \ \var{SURFAC(3,IFAC)})
\end{array}
\end{equation}
\end{itemize}

\item [$\star$] Mise \`a z\'ero du tableau \var{VISCB}.

\item [$\star$] Appel de \fort{divmas} pour calculer la divergence de~:
 $$\tens{D}^{\,n}\,\left( \grad{R^{\,n}_{ij}} - (\grad R^{\,n}_{ij}\,.\,\vect{n}_{\,lm})\vect{n}_{\,lm}\right)$$ d\'efini aux faces dans \var{VISCF} et \var{VISCB}.

Le r\'esultat est stock\'e dans le tableau \var{W1}.\\
Ajout au second membre \var{SMBR}.\\
$\var{SMBR} = \var{SMBR} + \var{W1}$
\end{itemize}
\item Calcul de la viscosit\'e orthotrope $\gamma^n_{\,lm}$ \`a la face $lm$ de la variable principale
$R^{\,n}_{ij}$\\
Ce calcul permet au sous-programme \fort{codits} de compl\'eter le second membre
\var{SMBR} par :
\begin{equation}
\begin{array} {ll}
& \sum\limits_{m\in Vois(l)}
\mu^n_{\,lm}\,\left(\grad{R}^{\,n}_{ij}\,.\,\vect{n}_{\,lm}\right)S_{\,lm}
 + \sum\limits_{m\in Vois(l)} \left(\grad{R}^{\,n}_{ij}
\,.\,\vect{n}_{\,lm}\right)\left[\tens{D}^{\,n}\,\vect{n}_{\,lm}\right]_{\,lm}\,.\,\vect{n}_{\,lm}\
S_{\,lm}\\
& = \sum\limits_{m\in Vois(l)}(\,\mu^n_{\,lm}\, + \,\gamma^n_{\,lm}\,)\,\left(\grad{R}^{\,n}_{ij}\,.\,\vect{n}_{\,lm}\right)S_{\,lm}
\end{array}
\end{equation}
sans pr\'eciser la nature de la face $lm$, {\it via} l'appel \`a \fort{bilsc2}  et de disposer de la quantit\'e
$(\mu^n_{\,lm}\, + \gamma^n_{\,lm})$ pour construire sa
matrice simplifi\'ee.\\
\begin{itemize}
\item [$\star$] On effectue une boucle d'indice \var{IEL} sur les cellules
$\Omega_l$ :
\begin{itemize}
\item [$\Rightarrow$] $\displaystyle \var{TRRIJ }= \frac{1}{2} (R^{\,n}_{ii})_L $
\item [$\Rightarrow$] $\displaystyle \var{RCSTE} = \rho^n_L \ C_S \ \frac{ (R^{\,n}_{ii})_L}{2\,\varepsilon^n_L} $
\item [$\Rightarrow$] $\displaystyle \var{W1(IEL)} = \mu^n +\rho^n_L \ C_S \ \frac{
(R^{\,n}_{ii})_L}{2\,\varepsilon^n_L}\ (R^n_{11})_L$
\item [$\Rightarrow$] $\displaystyle \var{W2(IEL)} = \mu^n + \rho^n_L \ C_S \ \frac{ (R^{\,n}_{ii})_L}{2\,\varepsilon^n_L}\ (R^n_{22})_L$
\item [$\Rightarrow$] $\displaystyle \var{W3(IEL)} = \mu^n + \rho^n_L \ C_S \ \frac{ (R^{\,n}_{ii})_L}{2\,\varepsilon^n_L}\ (R^n_{33})_L$
\end{itemize}

\item [$\star$] Appel de \fort{visort} pour calculer la viscosit\'e orthotrope
\footnote{Comme dans le sous-programme \fort{viscfa}, on multiplie la viscosit\'e par
$\displaystyle \frac{S_{\,lm}}{\overline{L'M'}}$, o\`u $S_{\,lm}$ et
$\overline{L'M'}$ repr\'esentent respectivement la surface de la face $lm$ et la
mesure alg\'ebrique du segment reliant les projections des centres des cellules
voisines sur la normale \`a la face. On garde dans ce sous-programme  la possibilit\'e d'interpoler la viscosit\'e aux cellules lin\'eairement ou d'utiliser une moyenne harmonique. La viscosit\'e au bord est celle de la cellule de bord correspondante.}
$ \gamma^n_{\,lm} = (\tens{D}^{\,n}\,\vect{n}_{\,lm}).\vect{n}_{\,lm}$ aux faces $lm$

Le r\'esultat est stock\'e dans les tableaux \var{VISCF} et \var{VISCB}.
\end{itemize}

\item appel de \fort{codits} pour la r\'esolution de l'\'equation de
convection/diffusion/termes sources de la variable $R_{ij}$. Le terme source est
\var{SMBR}, la viscosit\'e \var{VISCF} aux faces purement internes (
resp. \var{VISCB} aux faces de bord) et \var{FLUMAS} le flux de masse interne
 ( resp. \var{FLUMAB} flux de masse au bord). Le r\'esultat est la variable $R_{ij}$ au temps
$n+1$, donc $R^{\,n+1}_{ij}$. Elle est stock\'ee dans le tableau \var{RTP} des
variables mises \`a jour.

\end{itemize}

\etape{Appel de \fort{reseps} pour la r\'esolution de la variable $\varepsilon$}

On d\'ecrit ci-dessous le sous-programme \fort{reseps}, les commentaires du sous-programme \fort{resrij} ne seront pas r\'ep\'et\'es puisque les deux sous-programmes ne diff\`erent que par leurs termes sources.

\begin{itemize}
\item Initialisation \`a z\'ero de \var{SMBR} et \var{ROVSDT}.

\item{Lecture et prise en compte des termes sources utilisateur pour la variable $\varepsilon$ :}

Appel de \fort{ustsri} pour charger les termes sources utilisateurs. Ils sont
stock\'es dans les tableaux suivants :\\
pour la cellule $\Omega_l$ repr\'esent\'ee par $\var{IEL}$ de centre $L$, on a :
\begin{equation}\notag
\left\{\begin{array}{lll}
&\var{ROVSDT(IEL)}&= |\Omega_l| \ \alpha_{\varepsilon}\\
&\var{SMBR(IEL)}&=|\Omega_l| \ \beta_{\varepsilon}\\
\end{array}\right.
\end{equation}
On affecte alors les valeurs ad\'equates au second membre \var{SMBR} et \`a la
diagonale \var{ROVSDT} comme suit :
\begin{equation}\notag
\left\{\begin{array}{lll}
&\var{SMBR(IEL)} &= \var{SMBR(IEL)} +\ |\Omega_l| \ \alpha_{\,\varepsilon} \
\varepsilon^n_L \\
&\var{ROVSDT(IEL)}&= \text{max }(-\ |\Omega_l| \ \alpha_{\,\varepsilon},0)\\
\end{array}\right.
\end{equation}

\item{Calcul du terme source de masse si $\Gamma_L > 0$ :
\begin{equation}\notag
\left\{\begin{array}{lll}
&\displaystyle \var{SMBR(IEL)} = \var{SMBR(IEL)} + |\Omega_l| \ \Gamma_L \
(\varepsilon^{\,in}_L -\varepsilon^n_L) \\
&\displaystyle \var{ROVSDT(IEL)}= \var{ROVSDT(IEL)} + |\Omega_l| \ \Gamma_L
\end{array}\right.
\end{equation}
\item Calcul du terme d'accumulation de masse et du terme instationnaire \\
On stocke $\displaystyle \var{W1}= \int_{\Omega_l}\dive\,(\rho\,\vect{u})\,d\Omega$
calcul\'e par \fort{divmas} \`a l'aide des flux de masse internes et aux bords.\\
On incr\'emente ensuite \var{SMBR} et \var{ROVSDT}.
\begin{equation}\notag
\left\{\begin{array}{lll}
&\var{SMBR(IEL)} &= \var{SMBR(IEL)} + \var{ICONV}\ \varepsilon^n_L\,(\displaystyle
\int_{\Omega_l}\dive\,(\rho\,\vect{u})\ d\Omega) \\
&\var{ROVSDT(IEL)}& = \var{ROVSDT(IEL)} +  \var{ISTAT}\,\displaystyle
\frac{\rho^n_L \ |\Omega_l|}{\Delta t^n} -  \var{ICONV}\ (\displaystyle
\int_{\Omega_l}\dive\,(\rho\,\vect{u})\ d\Omega) \\
\end{array}\right.
\end{equation}

\item Traitement du terme de production
 $\displaystyle \rho\,C_{\varepsilon_1}\,\frac{\varepsilon}{k}\,\mathcal{P}$
 et du terme de dissipation $-\,\displaystyle \rho\,C_{\varepsilon_2}\,\frac{\varepsilon}{k}\,\varepsilon$ \\
pour cela, on effectue une boucle d'indice \var{IEL} sur les cellules $\Omega_l$
de centre $L$ :
\begin{itemize}
\item [$\Rightarrow$] $\displaystyle \var{TRPROD}= \frac{1}{2} (\mathcal{P}^n_{ii})_L = \frac{1}{2} \left[ \var{PRODUC(1,IEL)} +  \var{PRODUC(2,IEL)} +  \var{PRODUC(3,IEL)} \right] $
\item [$\Rightarrow$] $\displaystyle \var{TRRIJ }= \frac{1}{2} (R^n_{ii})_L $
\item [$\Rightarrow$] $\displaystyle \var{SMBR(IEL)} = \var{SMBR(IEL)} + \rho^n_L
|\Omega_l| \left[ -C_{\varepsilon_2} \ \frac{2\,(\varepsilon^n_L)^2}{(R^n_{ii})_L} + C_{\varepsilon_1} \ \frac{\varepsilon^n_L}{(R^n_{ii})_L}\ (\mathcal{P}^n_{ii})_L \right] $
\item [$\Rightarrow$] $\displaystyle \var{ROVSDT(IEL)} = \var{ROVSDT(IEL)} + C_{\varepsilon_2} \ \rho^n_L \ |\Omega_l| \ \frac{2\,\varepsilon^n_L}{(R^n_{ii})_L}$
\end{itemize}

\item Appel de \fort{rijthe} pour le calcul des termes de gravit\'e $\mathcal{G}^n_{\varepsilon}$ et ajout dans \var{SMBR}.

$ \var{SMBR} = \var{SMBR} + \mathcal{G}^n_{\varepsilon}$\\
Ce calcul n'a lieu que si $\var{IGRARI()} = 1$.

\item Calcul de la diffusion de $\varepsilon$ \\
 Le terme $\dive \left[\mu\, \grad(\varepsilon) + \tens{A'}\,\grad(\varepsilon)
\right]$ est calcul\'e exactement de la m\^eme mani\`ere que pour les tensions
de Reynolds $R_{ij}$ en rempla\c cant $\tens{A}$ par $\tens{A'}$.

\item Appel de \fort{codits} pour la r\'esolution de l'\'equation de
convection/diffusion/termes sources de la variable principale $\varepsilon$. Le
r\'esultat $\varepsilon^{\,n+1}$ est stock\'e dans le tableau \var{RTP} des
variables mises \`a jour.
}
\end{itemize}

\etape{clippings finaux}
On passe enfin dans le sous-programme  \fort{clprij} pour faire un clipping \'eventuel
des variables $R^{\,n+1}_{ij}$ et $\varepsilon^{\,n+1}$. Le sous-programme
\fort{clprij} est appel\'e\footnote{L'option
$\var{ICLIP} = 1$ consiste en un clipping minimal des variables $R_{ii}$ et
$\varepsilon$ en prenant la valeur absolue de ces variables puisqu'elles ne
peuvent \^etre que positives.} avec $\var{ICLIP} = 2$ . Cette option
\footnote{Quand la valeur des grandeurs $R_{ii}$ ou $\varepsilon$ est
n\'egative, on la remplace par le minimum entre sa valeur absolue et (1,1)
fois la valeur obtenue au pas de temps pr\'ec\'edent.} contient l'option $\var{ICLIP} = 1$  et permet de v\'erifier l'in\'egalit\'e de Cauchy-Schwarz sur les grandeurs extra-diagonales du tenseur $\tens{R}$ (pour $i \neq j$, $|R_{ij}|^2 \le R_{ii} R_{jj}$).


%%%%%%%%%%%%%%%%%%%%%%%%%%%%%%%%%%
%%%%%%%%%%%%%%%%%%%%%%%%%%%%%%%%%%
\section{Points \`a traiter}
%%%%%%%%%%%%%%%%%%%%%%%%%%%%%%%%%%
%%%%%%%%%%%%%%%%%%%%%%%%%%%%%%%%%%
Sauf mention explicite, $\phi$ repr\'esentera une tension de Reynolds ou la dissipation turbulente ($\phi = R_{ij} \ \text{ou} \ \varepsilon$).

\begin{itemize}
\item {La vitesse utilis\'ee pour le calcul de la production est explicite. Est-ce qu'une implicitation peut am\'eliorer la pr\'ecision temporelle de $\phi$ \footnote{Cette remarque peut \^etre g\'en\'eralis\'ee. En effet, peut-on envisager d'actualiser les variables d\'ej\`a r\'esolues (sans r\'eactualiser les variables turbulentes apr\`es leur r\'esolution)? Ceci obligerait \`a modifier les sous-programmes tels que \fort{condli} qui sont appel\'es au d\'ebut de la boucle en temps.} ?}
\item {Dans quelle mesure le terme d'\'echo de paroi est-il valide ? En effet, ce terme est remis en question par certains auteurs.}
\item {On peut envisager la r\'esolution d'un syst\`eme hyperbolique pour les
tensions de Reynolds afin d'introduire un couplage avec le champ de vitesse.}
\item {Le flux au bord \var{VISCB} est annul\'e dans le sous-programme
\fort{vectds}. Peut-on envisager de mettre au bord la valeur de la variable
concern\'ee \`a la cellule de bord correspondant? De m\^eme, il faudrait se
pencher sur les hypoth\`eses sous-jacentes \`a l'annulation des contributions
aux bords de \var{VISCB} lors du calcul de : $$\left[ \tens{D}^n\,\left( \grad{R^{\,n}_{ij}} - (\grad R^{\,n}_{ij}\,.\,\vect{n}_{\,lm})\,\vect{n}_{\,lm}\right) \right]\,.\,\vect{n}_{\,lm}.$$}
\item {Un probl\`eme de pond\'eration appara\^\i t plus g\'en\'eralement. Si on prend la partie explicite de $\tens{D}\,\grad(\phi)$, la pond\'eration aux faces internes utilise le coefficient $\displaystyle\frac{1}{2}$ avec pond\'eration s\'epar\'ee de $\tens{D}$ et $\grad(\phi)$, alors que pour $\tens{E}\,\grad(\phi)$, on calcule d'abord ce terme aux cellules pour ensuite l'interpoler lin\'eairement aux faces \footnote{Cette interpolation se fait dans \fort{vectds} avec des coefficients de pond\'eration aux faces.}. Ceci donne donc deux types d'interpolations pour des termes de m\^eme nature.}
\item {On laisse la possibilit\'e dans \fort{visort} d'utiliser une moyenne
harmonique aux faces. Est-ce que ceci est valable puisque les interpolations
utilis\'ees lors du calcul de la partie explicite de $\tens{A}\,\grad{\phi}$
sont des moyennes arithm\'etiques ?}
\item {Les techniques adopt\'ees lors du clipping sont \`a revoir.}
\item {On utilise dans le cadre du mod\`ele $\displaystyle R_{ij}-\varepsilon $ une semi-implicitation de termes comme $\displaystyle \phi_{ij,1}$ ou $\displaystyle -\rho\,C_{\varepsilon_2}\,\frac{\varepsilon}{k}\,\varepsilon$. On peut envisager le m\^eme type d'implicitation dans \fort{turbke} m\^eme en pr\'esence du couplage $\displaystyle k-\varepsilon$.}
\item L'adoption d'une r\'esolution d\'ecoupl\'ee fait perdre l'invariance par rotation.
\item La formulation et l'implantation des conditions aux limites de paroi
devront \^etre v\'erifi\'ees. En effet, il semblerait que, dans certains cas, des ph\'enom\`enes
``oscillatoires'' apparaissent, sans qu'il soit ais\'e d'en d\'eterminer la cause.
\item L'implicitation partielle (du fait de la r\'esolution d\'ecoupl\'ee) des
conditions aux limites conduit souvent \`a des calculs instables. Il
conviendrait d'en conna\^\i tre la raison. L'implicitation partielle avait
\'et\'e mise en \oe uvre afin de tenter d'utiliser un pas de temps plus grand
dans le cas de jets axisym\'etriques en particulier.

\end{itemize}

\part{Module de base}
%                      Code_Saturne version 1.3
%                      ------------------------
%
%     This file is part of the Code_Saturne Kernel, element of the
%     Code_Saturne CFD tool.
%
%     Copyright (C) 1998-2007 EDF S.A., France
%
%     contact: saturne-support@edf.fr
%
%     The Code_Saturne Kernel is free software; you can redistribute it
%     and/or modify it under the terms of the GNU General Public License
%     as published by the Free Software Foundation; either version 2 of
%     the License, or (at your option) any later version.
%
%     The Code_Saturne Kernel is distributed in the hope that it will be
%     useful, but WITHOUT ANY WARRANTY; without even the implied warranty
%     of MERCHANTABILITY or FITNESS FOR A PARTICULAR PURPOSE.  See the
%     GNU General Public License for more details.
%
%     You should have received a copy of the GNU General Public License
%     along with the Code_Saturne Kernel; if not, write to the
%     Free Software Foundation, Inc.,
%     51 Franklin St, Fifth Floor,
%     Boston, MA  02110-1301  USA
%
%-----------------------------------------------------------------------
%
\programme{vortex}
%
\vspace{1cm}
%%%%%%%%%%%%%%%%%%%%%%%%%%%%%%%%%%
%%%%%%%%%%%%%%%%%%%%%%%%%%%%%%%%%%
\section{Fonction}
%%%%%%%%%%%%%%%%%%%%%%%%%%%%%%%%%%
%%%%%%%%%%%%%%%%%%%%%%%%%%%%%%%%%%
Ce sous-programme est d�di� � la g�n�ration des conditions d'entr�e
turbulente utilis�es en LES.


La m�thode des vortex est bas�e sur une approche de tourbillons
ponctuels. L'id�e de la m�thode consiste � injecter des tourbillons 2D dans le
plan d'entr�e du calcul, puis � calculer le champ de vitesse induit par ces
tourbillons au centre des faces d'entr�e.

%                      Code_Saturne version 1.3
%                      ------------------------
%
%     This file is part of the Code_Saturne Kernel, element of the
%     Code_Saturne CFD tool.
% 
%     Copyright (C) 1998-2007 EDF S.A., France
%
%     contact: saturne-support@edf.fr
% 
%     The Code_Saturne Kernel is free software; you can redistribute it
%     and/or modify it under the terms of the GNU General Public License
%     as published by the Free Software Foundation; either version 2 of
%     the License, or (at your option) any later version.
% 
%     The Code_Saturne Kernel is distributed in the hope that it will be
%     useful, but WITHOUT ANY WARRANTY; without even the implied warranty
%     of MERCHANTABILITY or FITNESS FOR A PARTICULAR PURPOSE.  See the
%     GNU General Public License for more details.
% 
%     You should have received a copy of the GNU General Public License
%     along with the Code_Saturne Kernel; if not, write to the
%     Free Software Foundation, Inc.,
%     51 Franklin St, Fifth Floor,
%     Boston, MA  02110-1301  USA
%
%-----------------------------------------------------------------------
%
%%%%%%%%%%%%%%%%%%%%%%%%%%%%%%%%%%
%%%%%%%%%%%%%%%%%%%%%%%%%%%%%%%%%%
\section{Discr\'etisation}
%%%%%%%%%%%%%%%%%%%%%%%%%%%%%%%%%%
%%%%%%%%%%%%%%%%%%%%%%%%%%%%%%%%%%

Le terme convectif en $\dive(\underline{u} \otimes \rho\,\underline{u})$
introduit une non lin\'earit\'e et un couplage des composantes de la vitesse
$\vect{u}$ dans l'�quation (\ref{Base_Preduv_eqqdm}). Une lin\'earisation et un d\'ecouplage
des trois composantes de la 
vitesse sont r\'ealis\'es lors de la discr\'etisation de cette \'etape de
pr\'ediction.\\
En effet, soit :
\begin{equation}
\vect{\widetilde{u}}= \vect{u}^n + \delta \vect{u} 
\end{equation}
La contribution exacte du terme convectif \`a prendre en compte dans cette
\'etape de pr\'ediction serait :\\
\begin{equation}\label{Base_Preduv_Conv_exact}
\begin{array}{ll}
\dive(\vect{\widetilde{u}} \otimes \rho\,\vect{\widetilde{u}}) =
\dive(\vect{u}^{n} \otimes \rho\,\vect{u}^{n}) + \dive(\delta \vect{u} \otimes
\rho\,\vect{u}^{n}) +  \underbrace { \dive(\vect{u}^{n} \otimes
\rho\,\delta \vect{u})}_{\text {terme couplant lin\'eaire}} +  \underbrace { \dive(\delta \vect{u} \otimes
\rho\,\delta \vect{u})}_{\text {terme couplant et non lin\'eaire}}\\
\end{array} 
\end{equation}
Les deux derniers termes de l'expression (\ref{Base_Preduv_Conv_exact}) sont {\em a priori} n�glig�s
de mani�re � obtenir un syst\`eme en vitesse qui soit d\'ecoupl\'e et donc,
�viter l'inversion d'une matrice pouvant \^etre de tr\`es grande taille. Ces
deux termes peuvent n�anmoins �tre pris en compte de mani�re plus ou moins
approch�e par extrapolation explicite du flux de masse en $n+\theta_F$ (pour le
terme couplant lin�aire provenant de la convection de $\vect{u}^{n}$ par $\delta
\vect{u}$) et utilisation d'un point-fixe par sous it�ration sur le sous
programme \fort{navsto} (pour le terme non-lin�aire, en sp�cifiant $\var{NTERUP}>1$).

L'�quation (\ref{Base_Preduv_eqqdm}) est discr�tis�e au temps $n+\theta$ � l'aide d'un
$\theta$-sch�ma, et le tenseur des pertes de charges d�compos� en une partie
diagonale $\tens{K}_{d}$ et une extradiagonale $\tens{K}_{e}$ (soit
 $\tens{K}_{pdc}=\tens{K}_{d}+\tens{K}_{e}$).\\
$\bullet$ La pression est suppos�e connue en $n-1+\theta$ (d�calage temporel
pression-vitesse) et le gradient naturellement calcul� � cet instant.\\ 
$\bullet$ Les termes sources de viscosit� secondaire, de gradient transpos\'e,
ceux provenant du mod�le de turbulence\footnote{except� $\dive (\mu_t\ (\ggrad
\underline {u}))$}, $\rho\,\tens{K}_{\,e}\ \underline{u}$, $(\rho -\rho_0)
\underline {g}$ ainsi que $\underline{T}_{s}^{\,exp}$ et
$\Gamma\,\underline{u}_{\,i}$ sont pris de mani�re explicite au temps $n$, ou
extrapol�s suivant le sch�ma en temps choisi pour les propri�t�s physique et les
termes sources.\\ 
$\bullet$ Les termes sources $\underline{u}\,\,\dive (\rho\,\underline {u})$,
$\Gamma\,\,\underline{u}$, $T_{s}^{\,imp}\,\,\underline{u}$ et
$-\rho\,\tens{K}_{\,d}\,\,\underline{u}$ sont implicit�s est calcul�s �
l'instant $n+\theta$.\\ 
$\bullet$ Le terme de diffusion $\dive (\mu_{\,tot}\,\ggrad \underline{u})$ est
implicit� : la vitesse est prise � l'instant $n+\theta$ et la viscosit�
explicit�e ou extrapol�e.\\ 
$\bullet$ Enfin, le terme de convection en $\dive(\,\underline{u} \otimes
(\rho\underline{u})\,)$ est implicit� : la composante r�solue de la vitesse est
prise en $n+\theta$, et le flux de masse, explicit�, ou extrapol� en
$n+\theta_F$. 

Par souci de clart�, on suppose, en l'absence d'indication, que les propri�tes
physiques $\Phi$ ($\rho,\,\mu_{tot},\,...$) et le flux de masse
$(\rho\underline{u})$ sont pris respectivement aux instants $n+\theta_\Phi$ et
$n+\theta_F$, o� $\theta_\Phi$ et $\theta_F$ d�pendent des sch�mas en temps
sp�cifiquement utilis�s pour ces grandeurs\footnote{cf. \fort{introd}}. 

La discr�tisation temporelle de l'�quation (\ref{Base_Preduv_eqqdm}) s'�crit alors comme suit : 

\begin{equation}\label{Base_Preduv_eq_di1}
 \begin{array}{c}
\displaystyle \rho\,\ \frac{ \underline {\widetilde{u}}^{n+1} -\underline {u}^{n} }
{\Delta t} + \dive(\,\underline{\widetilde{u}}^{n+\theta} \otimes (\rho\underline{u})\,) -\dive
(\mu_{\,tot}\,\ggrad \underline{\widetilde{u}}^{n+\theta}) =
\\
\displaystyle
 - \grad p^{n-1+\theta} + \dive (\rho\,\underline {u})\,\underline{\widetilde{u}}^{n+\theta} +(\Gamma\,\underline{u}_{\,i})^{n+\theta_S}-\Gamma^n\,\,\underline{\widetilde{u}}^{n+\theta}
\\
\begin{array}{c}
\displaystyle
- \rho\,\tens{K}_{\,d}^{n}\,\,\underline{\widetilde{u}}^{n+\theta} - (\rho\,\tens{K}_{\,e}\ \underline{u})^{n+\theta_S} + (\underline{T}_{s}^{\,exp})^{\,n+\theta_S} + T_{s}^{\,imp}\,\,\underline{\widetilde{u}}^{n+\theta}
\\
\displaystyle
+[\dive (\mu_{\,tot}\,^t\ggrad \underline {u})]^{n+\theta_S}-\frac {2} {3}[\,\grad (\mu_{\,tot}\,\dive \underline {u})]^{n+\theta_S} + (\rho -\rho_0) \underline {g}
 - (\underline{turb})^{n+\theta_S}
\end{array}
\end{array}
\end{equation}
o\`u, par souci de simplification, on a pos\'e :
\begin{equation}
\mu_{\,tot}=
\begin{cases}
\mu+\mu_t & \text{pour les mod�les � viscosit� turbulente ou en LES}, \\
\mu & \text{pour les mod�les au second ordre ou en laminaire}
\end{cases} \ 
\end{equation}
\\
et :
\begin{equation}
\underline{turb}^{n}=
\begin{cases}
\displaystyle\frac {2}{3}\grad (\rho^{n}\,k^{n}) & \text{pour les mod�les � viscosit� turbulente}, \\
\dive(\rho^{n}\,\tens{R}^n) & \text{pour les mod�les au second ordre},\\
0 & \text{en laminaire ou en LES}\\
\end{cases}
\end{equation}
Par analogie avec l'�criture du $\theta$-sch�ma pour une variable scalaire, $\,
\underline {\widetilde{u}}^{n+\theta}$ est interpol�e � partir de la vitesse
pr�dite $\underline {\widetilde{u}}^{n+1}$ de la mani\`ere suivante\footnote{si
$\theta=1/2$, ou qu'une extrapolation est utilis�e, l'ordre 2 n'est obtenu que si
le pas de temps $\Delta t$ est uniforme en temps et en espace.}~: 
\begin{equation}
\underline {\widetilde{u}}^{n+\theta}=\theta\, \underline
{\widetilde{u}}^{n+1}+(1-\theta)\, \underline {u}^{n}\\ 
\end{equation}
Avec :
\begin{equation}
\left\{
\begin{array}{ll}
\theta = 1   & \text{Pour un sch\'ema de type Euler implicite d'ordre 1.}\\
\theta = 1/2 & \text{Pour un sch\'ema de type Cranck-Nicolson d'ordre 2.}\\
\end{array}
\right.
\end{equation}

L'�quation (\ref{Base_Preduv_eq_di1}) est alors r��crite sous la forme :

\begin{equation}\label{Base_Preduv_eq_di2}
\begin{array}{c}
\displaystyle \underbrace{\left(\frac{\rho}{\Delta t} -\theta \,\dive (\rho\,\underline {u}) +\theta \,\, \Gamma^n +
\theta \,\, \rho\,\tens{K}_{\,d}^n-\theta \,T_s^{\,imp} \right)}_{\displaystyle f_s^{imp}}\, (\underline {\,\widetilde{u}}^{n+1} -\underline {u}^{n})
\\
 +\, \theta\, \dive(\underline {\widetilde{u}}^{n+1} \otimes (\rho\underline{u}))-\, \theta\,\dive (\mu_{\,tot}\,\ggrad \underline {\widetilde{u}}^{n+1}) =
\\
-\,(1-\theta)\, \dive(\underline {u}^{n} \otimes (\rho\underline{u})) +\,(1-\theta)\,\dive (\mu_{\,tot}\,\ggrad \underline {u}^{n})
\\
f_s^{exp}\left\{
\begin{array}{c}
\displaystyle 
- \grad p^{n-1+\theta} + \dive (\rho\,\underline {u})\,\underline{u}^{n} +\,(\,\Gamma^{n}\,\underline{u}_{\,i}\,)^{n+\theta_S}- \Gamma^n\,\,\underline{u}^{n}
\\
\displaystyle
-(\,\rho\,\tens{K}_{\,e}\ \underline{u}\,)^{n+\theta_S} -\rho\,\tens{K}_{\,d}^n\ \underline{u}^{n}+ (\underline{T}_{s}^{\,exp})^{\,n+\theta_S} + T_s^{\,imp}\,\,\underline {u}^{n} 
\\
\displaystyle
+[\dive (\mu_{\,tot}\,^t\ggrad \underline {u}\,)]^{n+\theta_S}-\frac {2} {3}[\,\grad (\mu_{\,tot}\,\dive \underline {u}\,)]^{n+\theta_S} + (\rho -\rho_0) \underline {g}-(\underline{turb})^{n+\theta_S}
\end{array}
\right.
\end{array}
\end{equation}

d'o� l'�quation r�solue par le sous-programme \fort{codits} :
\begin{equation}\begin{array}{c}
\displaystyle
f_s^{\,imp}(\underline {\widetilde{u}}^{n+1}-\underline {u}^{n}) + \theta\, \dive(\underline{\widetilde{u}}^{n+1} \otimes (\rho
\underline{u})) - \theta\,\dive (\,\mu_{\,tot}\,\ggrad \underline{\widetilde{u}}^{n+1}) = 
\\\\
\displaystyle
-(1-\theta)\,\dive(\underline{u}^{n} \otimes (\rho \underline{u}))+(1-\theta)\,\dive (\,\mu_{\,tot}\,\ggrad \underline{u}^{n})
+ \underline{f}_{\,s}^{\,exp}
\end{array}
\end{equation}
La m\'ethode de discr\'etisation spatiale est d\'evelopp\'ee dans le sous-programme \fort{codits}.\\



\minititre{Remarques :}
{\tiny$\blacksquare$} Dans le cas standard sans extrapolation, le terme
$-\,T_s^{\,imp}$ n'est ajout� � $f_s^{\,imp}$ que s'il est positif afin de ne
pas affaiblir la dominance de la diagonale de la matrice � inverser.\\ 
{\tiny$\blacksquare$} Si une extrapolation est utilis�e, par contre,
$\,T_s^{\,imp}$ est ajout� � $f_s^{\,imp}$ quel que soit son signe. En effet, l'id�e intuitive qui
consiste � prendre~: 
\begin{equation}
\begin{cases}
\displaystyle
(\underline{T}_{s}^{\,exp} + T_{s}^{\,imp}\,\underline {u})^{\,n+\theta_S} &
\text{si } T_{s}^{\,imp} > 0\\ 
\displaystyle
(\underline{T}_{s}^{\,exp})^{\,n+\theta_S} + T_{s}^{\,imp}\,\underline{u}^{n+\theta} &\text{sinon}\\
\end{cases}
\end{equation} 
aboutit � une incoh�rence dans le traitement si $T_s^{imp}$ change de signe
entre deux pas de temps.\\ 
{\tiny$\blacksquare$} la partie diagonale $\tens{K}_{\,d}$ du terme
de perte de charge est utilis�e dans $f_s^{\,imp}$. En fait, pour \^etre rigoureux,
il faudrait ne retenir que les contributions positives (point signal\'e dans le
sous-programme utilisateur associ\'e \fort{uskpdc}). Cette prise en compte sera \`a am\'eliorer.\\
{\tiny$\blacksquare$} Le terme $\theta\,\Gamma^{n}-\theta\,\dive
(\rho\,\underline {u})$ ne pose pas de probl�me pour la 
dominance de la diagonale de la matrice car il est exactement compens� par le
terme de convection (cf. \fort{covofi}). 


%                      Code_Saturne version 1.3
%                      ------------------------
%
%     This file is part of the Code_Saturne Kernel, element of the
%     Code_Saturne CFD tool.
%
%     Copyright (C) 1998-2007 EDF S.A., France
%
%     contact: saturne-support@edf.fr
%
%     The Code_Saturne Kernel is free software; you can redistribute it
%     and/or modify it under the terms of the GNU General Public License
%     as published by the Free Software Foundation; either version 2 of
%     the License, or (at your option) any later version.
%
%     The Code_Saturne Kernel is distributed in the hope that it will be
%     useful, but WITHOUT ANY WARRANTY; without even the implied warranty
%     of MERCHANTABILITY or FITNESS FOR A PARTICULAR PURPOSE.  See the
%     GNU General Public License for more details.
%
%     You should have received a copy of the GNU General Public License
%     along with the Code_Saturne Kernel; if not, write to the
%     Free Software Foundation, Inc.,
%     51 Franklin St, Fifth Floor,
%     Boston, MA  02110-1301  USA
%
%-----------------------------------------------------------------------
%

%%%%%%%%%%%%%%%%%%%%%%%%%%%%%%%%%%
%%%%%%%%%%%%%%%%%%%%%%%%%%%%%%%%%%
\section{Mise en \oe uvre}
%%%%%%%%%%%%%%%%%%%%%%%%%%%%%%%%%%
%%%%%%%%%%%%%%%%%%%%%%%%%%%%%%%%%%
La num\'ero de la phase trait\'ee fait partie des arguments de \fort{turrij}. On
omettra volontairement de le pr\'eciser dans ce qui suit, on indiquera par $(\ )$ la
notion de tableau s'y rattachant.

\etape{Calcul des termes de production $\tens{\mathcal{P}}$}
\begin{itemize}
\item [$\star$] Initialisation \`a z\'ero du tableau \var{PRODUC} dimensionn\'e \`a $\var{NCEL}\times 6$.
\item [$\star$] On appelle trois fois \fort{grdcel} pour calculer les gradients des composantes de la vitesse $u$, $v$ et
$w$ prises au temps $n$.

Au final, on a :\\
$\displaystyle
\begin{array} {ll}
\var{PRODUC(1,IEL)} = & \displaystyle - 2 \left[ R_{11}^{\,n} \frac{\partial u^{\,n}} {\partial x} +R_{12}^{\,n} \frac{\partial u^{\,n}} {\partial y}+R_{13}^{\,n} \frac{\partial u^{\,n}} {\partial z} \right] \text{        (production de $R_{11}^{\,n}$)}\\
\var{PRODUC(2,IEL)} = & \displaystyle - 2 \left[ R_{12}^{\,n} \frac{\partial v^{\,n}} {\partial x} +R_{22}^{\,n} \frac{\partial v^{\,n}} {\partial y}+R_{23}^{\,n} \frac{\partial v^{\,n}} {\partial z} \right] \text{        (production de $R_{22}^{\,n}$)}\\
\var{PRODUC(3,IEL)} = & \displaystyle - 2 \left[ R_{13}^{\,n} \frac{\partial w^{\,n}} {\partial x} +R_{23}^{\,n} \frac{\partial w^{\,n}} {\partial y}+R_{33}^{\,n} \frac{\partial w^{\,n}} {\partial z} \right] \text{        (production de $R_{33}^{\,n}$)}\\
\var{PRODUC(4,IEL)} = & \displaystyle - \left[ R_{12}^{\,n} \frac{\partial u^{\,n}} {\partial x} +R_{22}^{\,n} \frac{\partial u^{\,n}} {\partial y}+R_{23}^{\,n} \frac{\partial u^{\,n}} {\partial z} \right] \\
& \displaystyle - \left[ R_{11}^{\,n} \frac{\partial v^{\,n}} {\partial x} +R_{12}^{\,n} \frac{\partial v^{\,n}} {\partial y}+R_{13}^{\,n} \frac{\partial v^{\,n}} {\partial z} \right] \text{        (production de $R_{12}^{\,n}$)} \\
\var{PRODUC(5,IEL)} = & \displaystyle - \left[ R_{13}^{\,n} \frac{\partial u^{\,n}} {\partial x} +R_{23}^{\,n} \frac{\partial u^{\,n}} {\partial y}+R_{33}^{\,n} \frac{\partial u^{\,n}} {\partial z} \right] \\
& \displaystyle - \left[ R_{11}^{\,n} \frac{\partial w^{\,n}} {\partial x} +R_{12}^{\,n} \frac{\partial w^{\,n}} {\partial y}+R_{13}^{\,n} \frac{\partial w^{\,n}} {\partial z} \right] \text{        (production de $R_{13}^{\,n}$)} \\
\var{PRODUC(6,IEL)} = & \displaystyle - \left[ R_{13}^{\,n} \frac{\partial v^{\,n}} {\partial x} +R_{23}^{\,n} \frac{\partial v^{\,n}} {\partial y}+R_{33}^{\,n} \frac{\partial v^{\,n}} {\partial z} \right] \\
& \displaystyle - \left[ R_{12}^{\,n} \frac{\partial w^{\,n}} {\partial x} +R_{22}^{\,n} \frac{\partial w^{\,n}} {\partial y}+R_{23}^{\,n} \frac{\partial w^{\,n}} {\partial z} \right]  \text{        (production de $R_{23}^{\,n}$)}
\end{array}
$
\end{itemize}

\etape{Calcul du gradient de la masse volumique $\rho^n$ prise au d\'ebut du pas
de temps courant\footnote{{\it i.e.} calcul\'ee \`a partir des
variables du pas de temps pr\'ec\'edent $n$ si n\'ecessaire.} $(n+1)$}
Ce calcul n'a lieu que si les termes de gravit\'e doivent \^etre pris en compte
($\var{IGRARI()} =1$).
\begin{itemize}
\item [$\star$] Appel de \fort{grdcel}  pour calculer le gradient de $\rho^n$
dans les trois directions de l'espace. Les conditions aux limites sur $\rho^n$
sont des conditions de Dirichlet puisque la valeur de $\rho^n$ aux faces de bord
$ik$ (variable \var{IFAC}) est connue et vaut $\rho_{\,b_{\,ik}}$. Pour \'ecrire les conditions aux limites
sous la forme habituelle, $$\rho_{\,b_{\,ik}} = \var{COEFA} + \var{COEFB}
\,\rho^n_{\,I'}$$ on pose alors $\var{COEFA}=
\var{PROPCE(IFAC,IPPROB(IROM(IPHAS)))}$ et $\var{COEFB} = \var{VISCB} = 0$.\\
$\var{PROPCE(1,IPPROB(IROM(IPHAS)))}$ (resp.$\var{VISCB}$) est utilis\'e en lieu
et place de l'habituel \var{COEFA} ($\var{COEFB}$), lors de l'appel \`a \fort{grdcel}.\\
On a donc :\\
$\displaystyle \var{GRAROX}= \frac{\partial \rho^n}{\partial x}\ $,$\displaystyle \ \var{GRAROY}= \frac{\partial
\rho^n}{\partial y}$ et $
\displaystyle \ \var{GRAROZ}= \frac{\partial \rho^n}{\partial z}\ $.

\end{itemize}

Le gradient de $\rho^n$ servira \`a calculer les termes de production par effets de gravit\'e si ces derniers sont pris en compte.

\etape{Boucle \var{ISOU} de $1$ \`a $6$ sur les tensions de Reynolds}
Pour $\var{ISOU} = 1,2,3,4,5,6$, on r\'esout respectivement et dans
l'ordre  les
\'equations de $R_{11}$, $R_{22}$, $R_{33}$, $R_{12}$, $R_{13}$ et $R_{23}$ par
l'appel au sous-programme \fort{resrij}.\\
La r\'esolution se fait par incr\'ement $\delta {R}_{ij}^{\,n+1,k+1}$ , en utilisant la m\^eme m\'ethode que
celle d\'ecrite dans le sous-programme \fort{codits}. On adopte ici les m\^emes notations.
\var{SMBR} est le second membre du syst\`eme \`a inverser, syst\`eme portant sur
les incr\'ements de la variable. \var{ROVSDT} repr\'esente la diagonale de la
matrice, hors convection/diffusion.\\
On va r\'esoudre l'\'equation (\ref{Base_Turrij_Eq_Temp_Rij}) sous forme incr\'ementale en
utilisant \fort{codits}, soit :
\begin{equation}\label{Base_Turrij_Eq_Temp_deltaRij}
\begin{array}{ll}
&\displaystyle \underbrace{\left(\frac {\rho^n_L}{\Delta t^n}
+ \rho^n_L \,C_1\,\frac{\varepsilon^n_L}{k^n_L}(1-\frac{\delta_{ij}}{3})
 - m^n_{\,lm} + \Gamma_L\,+ max(-\alpha^n_{R_{ij}},0)\right)\,|\Omega_l|}
_{\text {$\var{ROVSDT}$ contribuant
\`a la diagonale de la matrice simplifi\'ee de \fort{matrix}}}\,(\delta{R}_{ij}^{\,n+1,p+1})_{\,L}\\\\
&  \underbrace{+\sum\limits_{m\in Vois(l)}\displaystyle \left[
 m^n_{\,lm} \delta R_{ij,\,f_{\,lm}}^{\,n+1,p+1}
- (\mu^n_{\,lm} + \gamma^n_{\,lm})\
\frac{({\delta R}_{ij}^{\,n+1,p+1})_{M}-({\delta R}_{ij}^{\,n+1,p+1})_{L})}{\overline{L'M'}}\,
S_{\,lm} \right]}_{\text { convection upwind pur et diffusion non reconstruite
relatives \`a la matrice simplifi\'ee de \fort{matrix}\footnotemark}} \\
% voir le texte de la footmark plus bas
&= - \displaystyle\frac {\rho^n_L}{\Delta t^n}\,\left(\,(R^{\,n+1,p}_{ij})_L - (R^{\,n}_{ij})_L\,\right)\\
&-\,\underbrace{\displaystyle\int_{\Omega_l} \left(
\dive\,[\,(\rho\,\vect{u})^n\,R^{\,n+1,p}_{ij} - (\mu^n\,+ \gamma^n\,)\,
\grad{R^{\,n+1,p}_{ij}}\,]\right)\,d\Omega}_{\text {convection et diffusion
trait\'ees par \fort{bilsc2}}}\\
&+\displaystyle \int_{\Omega_l} \left[\,\mathcal{P}^{\,n+1,p}_{ij} + \mathcal{G}^{\,n+1,p}_{ij}
- \displaystyle\rho^n \,C_1\,\frac{\varepsilon^n}{k^n}\left[R^{\,n+1,p}_{ij}-
\frac{2}{3}\,k^n\,\delta_{ij}\right] + \phi^{\,n+1,p}_{ij,2} +
\phi^{\,n+1,p}_{ij,w}\,\right]\, d\Omega \\
& + \displaystyle\int_{\Omega_l} \left[- \frac{2}{3} \rho^n \varepsilon^n \delta_{ij}
 + \Gamma\,(\,R^{\,in}_{ij} - R^{\,n+1,p}_{ij}\,) +
\alpha^n_{R_{ij}}\,R^{\,n+1,p}_{ij}+ \beta^n_{R_{ij}}\right]\, d\Omega\\
&+ \sum\limits_{m\in
Vois(l)}\displaystyle \left[\ \tens{E}^n\,\grad{R}^{\,n+1,p}_{ij} \right]_{\,lm}\,.\,\vect{n}_{\,lm}S_{\,lm}\\
&+ \sum\limits_{m\in Vois(l)}\displaystyle \left[\
\tens{D}^n\,\grad{R}^{\,n+1,p}_{ij} \right]_{\,lm}\,.\,\vect{n}_{\,lm}S_{\,lm}\\
&- \sum\limits_{m\in Vois(l)} \gamma^n_{\,lm} \left( \grad{R}^{\,n+1,p}_{ij}\,.\,\vect{n}_{\,lm} \right)  S_{\,lm}\\
&+ \sum\limits_{m\in Vois(l)}  m^n_{\,lm}\,(R^{\,n+1,p}_{ij})_L\\
\end{array}
\end{equation}
% si on ne fait pas comme ca, il n'apparait pas
\footnotetext[\thefootnote]{Si $\var{IRIJNU} = 1$, on remplace  $\mu^n_{\,lm}$ par $(\mu +
\mu_t)^n_{\,lm}$ dans l'expression de la diffusion non reconstruite
associ\'ee \`a la matrice simplifi\'ee de \fort{matrix} ($\mu_t$ d\'esigne la
viscosit\'e turbulente calcul\'ee comme en $k-\varepsilon$).}

o\`u on rappelle :\\
pour $n$ donn\'e entier positif, on d\'efinit la suite
 $({R}_{ij}^{\,n+1,p})_{p \in \grandN}$
 par :
\begin{equation}\notag
\left\{\begin{array}{l}
{R}_{ij}^{\,n+1,0} = {R}_{ij}^{\,n}\\
{R}_{ij}^{\,n+1,p+1} = {R}_{ij}^{\,n+1,p} + \delta{R}_{ij}^{\,n+1,p+1} \\
\end{array}\right.
\end{equation}
$(\delta{R}_{ij}^{\,n+1,p+1})_{\,L}$ d\'esigne la valeur sur l'\'el\'ement
$\Omega_l$ du $\text{$(\,p+1\,)$-i\`eme}$ incr\'ement de ${R}_{ij}^{\,n+1}$,
$ m^n_{\,lm}$ le flux de masse \`a l'instant $n$ \`a travers la face $lm$,
$\delta R_{ij,\,f_{\,lm}}^{\,n+1,p+1}$ vaut $({\delta
R}_{ij}^{\,n+1,p+1})_{L}$  si $ m^n_{\,lm} \geqslant 0$, $({\delta
R}_{ij}^{\,n+1,p+1})_{M}$ sinon,
$\mathcal{P}^{\,n+1,p}_{ij}$, $\phi^{\,n+1,p}_{ij,2}$, $\phi^{\,n+1,p}_{ij,w}$ les valeurs
des quantit\'es associ\'ees correspondant \`a l'incr\'ement
$(\delta{R}_{ij}^{\,n+1,p})$.\\



Tous ces termes sont calcul\'es comme suit :
\begin{itemize}
\item Terme de gauche de l'\'equation (\ref{Base_Turrij_Eq_Temp_deltaRij})\\
Dans \fort{resrij} est calcul\'ee la variable \var{ROVSDT}. Les autres
termes sont compl\'et\'es par \fort{codits}, lors de la construction de la matrice simplifi\'ee , {\it via} un
appel au sous-programme \fort{matrix}. La quantit\'e
 $(\mu^n_{\,lm} + \gamma^n_{\,lm})$ \`a la face $lm$ est calcul\'ee lors de l'appel \`a
\fort{visort}.\\
\item Second membre de l'\'equation (\ref{Base_Turrij_Eq_Temp_deltaRij})\\
Le premier terme non d\'etaill\'e est calcul\'e par le sous-programme
\fort{bilsc2}, qui applique le sch\'ema convectif choisi par l'utilisateur, qui
reconstruit ou non selon le souhait de l'utilisateur les gradients intervenants
dans la convection-diffusion.\\
Les termes sans accolade sont, eux, compl\`etement explicites et ajout\'es au fur et
\`a mesure dans \var{SMBR} pour former
l'expression $f^{\,exp}_s$ de \fort{codits}.
\end{itemize}
On d\'ecrit ci-dessous les \'etapes de \fort{resrij} :
\begin{itemize}

\item DELTIJ = 1, pour $\var{ISOU} \leqslant 3$ et DELTIJ = 0  Si $\var{ISOU} >
3$. Cette valeur repr\'esente le symbole de Kroeneker $\delta_{ij}$.

\item Initialisation \`a z\'ero de \var{SMBR} (tableau contenant le second
membre) et \var{ROVSDT} (tableau contenant la diagonale de la matrice sauf celle
relative \`a la contribution de la
diagonale des op\'erateurs de convection et de diffusion lin\'earis\'es
\footnote{qui correspondent aux sch\'emas convectif upwind pur et diffusif sans
reconstruction.}), tous deux de dimension $\var{NCEL}$.

\item Lecture et prise en compte des termes sources utilisateur pour la variable $R_{ij}$

Appel \`a \fort{ustsri} pour charger les termes sources utilisateurs. Ils sont
stock\'es comme suit. Pour la cellule $\Omega_l$ de centre $L$, repr\'esent\'ee par $\var{IEL}$, on a :\\
\begin{equation}\notag
\left\{\begin{array}{lll}
&\var{ROVSDT(IEL)}&= |\Omega_l| \ \alpha_{R_{ij}}\\
&\var{SMBR(IEL)}&=|\Omega_l| \ \beta_{R_{ij}}\\
\end{array}\right.
\end{equation}
On affecte alors les valeurs ad\'equates au second membre \var{SMBR} et \`a la
diagonale \var{ROVSDT} comme suit :
\begin{equation}\notag
\left\{\begin{array}{lll}
&\var{SMBR(IEL)} &= \var{SMBR(IEL)} +\ |\Omega_l| \ \alpha_{R_{ij}} \ (R^n_{ij})_L \\
&\var{ROVSDT(IEL)}&= \text{max }(-\ |\Omega_l| \ \alpha_{R_{ij}},0)\\
\end{array}\right.
\end{equation}
La valeur de $ \var{ROVSDT}$ est ainsi calcul\'ee pour des raisons de stabilit\'e
num\'erique. En effet, on ne rajoute sur la diagonale que les valeurs positives,
ce qui correspond physiquement \`a impliciter les termes de rappel uniquement.
\item{Calcul du terme source de masse  si $\Gamma_L > 0$}

Appel de \fort{catsma} et incr\'ementation si n\'ecessaire de \var{SMBR} et
\var{ROVSDT} {\it via} :\\
\begin{equation}\notag
\left\{\begin{array}{lll}
\displaystyle \var{SMBR(IEL)} = \var{SMBR(IEL)} + |\Omega_l| \ \Gamma_L \
\left[(R^{\,in}_{ij})_L - (R^{\,n}_{ij})_L \right] \\
\displaystyle \var{ROVSDT(IEL)}=\var{ROVSDT(IEL)} + |\Omega_l| \ \Gamma_L
\end{array}\right.
\end{equation}
\item Calcul du terme d'accumulation de masse et du terme instationnaire

On stocke $\displaystyle \var{W1}= \int_{\Omega_l}\dive\,(\rho\,\vect{u})\,d\Omega$
calcul\'e par \fort{divmas} \`a l'aide des flux de masse aux faces internes
$ m^n_{\,lm}=\sum\limits_{m\in Vois(l)}{(\rho \vect{u})_{\,lm}^n} \text{.}\,
\vect{S}_{\,lm} $ (tableau \var{FLUMAS}) et des flux de masse aux bords  $ m^n_{\,b_{lk}} = \sum\limits_{k\in{\gamma_b(l)}}{(\rho \vect{u})_{\,{b}_{lk}}^n} \text{.}\,
\vect{S}_{\,{b}_{lk}} $ (tableau \var{FLUMAB}).
On incr\'emente ensuite \var{SMBR} et \var{ROVSDT}.
\begin{equation}\notag
\left\{\begin{array}{lll}
&\var{SMBR(IEL)} &= \var{SMBR(IEL)} + \var{ICONV}\  (R^n_{ij})_L\,(\displaystyle
\int_{\Omega_l}\dive\,(\rho\,\vect{u})\ d\Omega) \\
&\var{ROVSDT(IEL)}& = \var{ROVSDT(IEL)} +  \var{ISTAT}\,\displaystyle
\frac{\rho^n_L \ |\Omega_l|}{\Delta t^n} -  \var{ICONV}\ (\displaystyle
\int_{\Omega_l}\dive\,(\rho\,\vect{u})\ d\Omega) \\
\end{array}\right.
\end{equation}
\item Calcul des termes sources de production, des termes $\displaystyle
\phi_{\,ij,1}+\phi_{\,ij,2}$ et de la dissipation~$\displaystyle-\frac{2}{3} \varepsilon\,\delta_{\,ij}$ :

On effectue une boucle d'indice \var{IEL} sur les cellules $\Omega_l$ de centre $L$ :
\begin{itemize}
\item [$\Rightarrow$] $\displaystyle \var{TRPROD}= \frac{1}{2} (\mathcal{P}^n_{ii})_L = \frac{1}{2} \left[ \var{PRODUC(1,IEL)} +  \var{PRODUC(2,IEL)} +  \var{PRODUC(3,IEL)} \right] $
\item [$\Rightarrow$] $\displaystyle \var{TRRIJ }= \frac{1}{2} (R^n_{ii})_L $
\item [$\Rightarrow$] $\displaystyle \var{SMBR(IEL)} =\ \var{SMBR(IEL)}\ +$\\
$\ \displaystyle\rho^n_L |\Omega_l| \left[ \displaystyle
\frac{2}{3}\,\delta_{\,ij} \left( \ \displaystyle \frac{ C_2}{2}\,(\mathcal{P}^n_{ii})_L\ +
(C_1-1)\ \varepsilon^n_L\, \right)\right.$\\
$ + \left.\ (1-C_2) \ \var{PRODUC(ISOU,IEL)} -
\displaystyle C_1\ \frac{2\,\varepsilon^n_L}{(R^n_{ii})_L}\ (R^n_{ij})_L \right]$
\item [$\Rightarrow$] $\displaystyle \var{ROVSDT(IEL)} = \var{ROVSDT(IEL)} +
\rho^n_L \ |\Omega_l| \ (- \displaystyle \frac{1}{3} \ \,\delta_{\,ij} + 1) \ C_1
\ \frac{2\ \varepsilon^n_L}{(R^n_{ii})_L}$
\end{itemize}
\item Appel de \fort{rijech} pour le calcul des termes d'\'echo de paroi
 $\phi^n_{ij,w}$ si $\var{IRIJEC()}=1$ et ajout dans \var{SMBR}.\\
$\var{SMBR} = \var{SMBR} + \phi^n_{ij,w}$\\
Suivant son mode de calcul (\var{ICDPAR}), la distance � la paroi est directement accessible
par \var{RA(IDIPAR+IEL-1)} (\var{|ICDPAR|} = 1) ou bien
est calcul\'ee \`a partir de $\var{IA(IIFAPA(IPHAS)+IEL - 1)}$,
qui donne pour l'\'el\'ement $\var{IEL}$ le num\'ero de la face de bord
paroi la plus  proche (\var{|ICDPAR|} = 2). Ces tableaux ont \'et\'e renseign\'e une fois pour toutes au
d\'ebut de calcul.

\item  Appel de \fort{rijthe} pour le calcul des termes de gravit\'e $\mathcal{G}^n_{ij}$ et ajout dans \var{SMBR}.

Ce calcul n'a lieu que si $\var{IGRARI()} = 1$.
$ \var{SMBR} = \var{SMBR} + \mathcal{G}^n_{ij}$
\item Calcul de la partie explicite du terme de diffusion
 $\dive{\,\left[\tens{A}\,\grad{R}^{\,n}_{ij}\right]}$, plus pr\'ecis\'ement
des contributions du terme extradiagonal pris aux faces purement internes
(remplissage du tableau \var{VISCF}), puis aux faces de bord (remplissage du
tableau \var{VISCB}).
\begin{itemize}
\item [$\star$] Appel de \fort{grdcel} pour le calcul du gradient de
$R^{\,n}_{ij}$ dans chaque direction. Ces gradients sont respectivement
stock\'es dans les tableaux de travail \var{W1}, \var{W2} et \var{W3}.

\item [$\star$] boucle d'indice \var{IEL} sur les cellules $\Omega_l$ de centre
$L$ pour le
calcul de $\tens{E}^n\,\grad{R}^{\,n}_{ij}$ aux cellules dans un premier temps :\\
\begin{itemize}
\item [$\Rightarrow$] $\displaystyle \var{TRRIJ}= \frac{1}{2} (R^{\,n}_{ii})_L $
\item [$\Rightarrow$] $\displaystyle \var{CSTRIJ} = \rho^n_L\ C_S \ \displaystyle\frac{(R^n_{ii})_L}{2\,\varepsilon^n_L}$
\item [$\Rightarrow$] $\displaystyle \var{W4(IEL)} = \rho^n_L\ C_S\
\displaystyle\frac{(R^n_{ii})_L}{2\,\varepsilon^n_L} \left[\,(R^{\,n}_{12})_L \ \var{W2(IEL)} +
(R^{\,n}_{13})_L \ \var{W3(IEL)}\,\right]$
\item [$\Rightarrow$] $\displaystyle \var{W5(IEL)} = \rho^n_L\ C_S\
\displaystyle\frac{(R^n_{ii})_L}{2\,\varepsilon^n_L} \left[\,(R^{\,n}_{12})_L \ \var{W1(IEL)} +
(R^{\,n}_{23})_L \ \var{W3(IEL)}\,\right]$
\item [$\Rightarrow$] $\displaystyle \var{W6(IEL)} = \rho^n_L\ C_S\
\displaystyle\frac{(R^n_{ii})_L}{2\,\varepsilon^n_L} \left[\,(R^{\,n}_{13})_L \ \var{W1(IEL)} + (R^{\,n}_{23})_L \ \var{W2(IEL)}\,\right]$
\end{itemize}



\item [$\star$] Appel de \fort{vectds}\footnote{Le r\'esultat est stock\'e dans
\var{VISCF} et \var{VISCB}. Dans \fort{vectds}, les valeurs aux faces internes
sont interpol\'ees lin\'eairement sans reconstruction et \var{VISCB} est mis \`a
z\'ero.} pour assembler $\displaystyle\left[ \tens{E}^n\,\grad{R}^{\,n}_{ij}
\right]\,.\,\vect{n}_{\,lm}S_{\,lm}$ aux faces $lm$.
\item [$\star$] Appel de \fort{divmas} pour calculer la divergence du flux d\'efini par \var{VISCF} et \var{VISCB}.
Le r\'esultat est stock\'e dans \var{W4}.\\
Ajout au second membre \var{SMBR}.\\
\var{SMBR} = \var{SMBR} + \var{W4}
\end{itemize}

A l'issue de cette \'etape, seule la partie extradiagonale de la diffusion prise
enti\`erement explicite~:
 $$\sum\limits_{m\in
Vois(l)}\left[\ \tens{E}^n\,\grad{R}^{\,n}_{ij} \right]_{\,lm}\,.\,\vect{n}_{\,lm}S_{\,lm}$$ a \'et\'e calcul\'ee.\\

\item Calcul de la partie diagonale du terme de diffusion\footnote{Seule la
composante normale  du  gradient de $R_{ij}$ aux faces sera implicite.} :\\
Comme on l'a d\'eja vu, une partie de cette contribution sera trait\'ee en
implicite (celle relative \`a la composante normale du gradient) et donc
ajout\'ee au second membre par \fort{bilsc2} ; l'autre
partie sera explicite et prise en compte dans $f_s^{\,exp}$.
\begin{itemize}
\item [$\star$] On effectue une boucle d'indice \var{IEL} sur les cellules
$\Omega_l$ de centre $L$ :
\begin{itemize}
\item [$\Rightarrow$] $\displaystyle \var{TRRIJ }= \frac{1}{2} (R^{\,n}_{ii})_L $
\item [$\Rightarrow$] $\displaystyle \var{CSTRIJ} = \rho^n_L \ C_S \ \frac{(R^{\,n}_{ii})_L}{2\,\varepsilon^n_L}$
\item [$\Rightarrow$] $\displaystyle \var{W4(IEL)} = \rho^n_L \ C_S \
\frac{(R^{\,n}_{ii})_L}{2\,\varepsilon^n_L} \ (R^{\,n}_{11})_L$
\item [$\Rightarrow$] $\displaystyle \var{W5(IEL)} = \rho^n_L \ C_S \ \frac{(R^{\,n}_{ii})_L}{2\,\varepsilon^n_L}\ (R^n_{22})_L$
\item [$\Rightarrow$] $\displaystyle \var{W6(IEL)} = \rho^n_L \ C_S \ \frac{(R^{\,n}_{ii})_L}{2\,\varepsilon^n_L} \ (R^n_{33})_L$
\end{itemize}

%\item Traitement du parall\'elisme et de la p\'eriodicit\'e.

\item [$\star$] On effectue une boucle d'indice \var{IFAC} sur les faces
purement internes $lm$ pour remplir le tableau \var{VISCF} :
\begin{itemize}
\item [$\Rightarrow$] Identification des cellules $\Omega_l$ et $\Omega_m$ de
centre respectif $L$ (variable \var{II}) et $M$ (variable \var{JJ}), se trouvant de chaque c\^ot\'e de la face
$lm$\footnote{La normale \`a la face est orient\'ee de L vers M.}.
\item [$\Rightarrow$] Calcul du carr\'e de la surface de la face. La valeur est
stock\'ee dans le tableau \var{SURFN2}.
\item [$\Rightarrow$] Interpolation du gradient de $R^{\,n}_{ij}$ \`a la face
$lm$ (gradient facette $\left[\grad{R}^{\,n}_{ij}\right]_{\,lm}$) :
\begin{equation}\notag
\left\{\begin{array}{ll}
\var{GRDPX} &= \displaystyle \frac{1}{2} \left(\var{W1(II)} + \var{W1(JJ)}
\right) \\
&\\
\var{GRDPY} &= \displaystyle \frac{1}{2} \left(\var{W2(II)} + \var{W2(JJ)} \right) \\
&\\
\var{GRDPZ} &= \displaystyle \frac{1}{2} \left(\var{W3(II)} + \var{W3(JJ)} \right)
\end{array}\right.
\end{equation}
\item [$\Rightarrow$] Calcul du gradient de $R^{\,n}_{ij}$ normal \`a la face
$lm$, $\left[\grad{R}^{\,n}_{ij}\right]_{\,lm}.\vect{n}_{\,lm}\,S_{\,lm}$.\\

$\displaystyle \var{GRDSN} =  \var{GRDPX} \ \var{SURFAC(1,IFAC)} + \var{GRDPY} \ \var{SURFAC(2,IFAC)} +  \var{GRDPZ} \ \var{SURFAC(3,IFAC)}$
$\var{SURFAC}$ est le vecteur surface de la face \var{IFAC}.


\item [$\Rightarrow$] calcul de
 $\left[\grad{R^{\,n}_{ij}} - (\grad
R^{\,n}_{ij}\,.\,\vect{n}_{\,lm})\vect{n}_{\,lm}\right]$, les vecteurs \'etant
calcul\'es \`a la face $lm$ :
\begin{equation}\notag
\left\{\begin{array}{lll}
&\displaystyle \var{GRDPX} &= \var{GRDPX} - \displaystyle\frac{\var{GRDSN}}{\var{SURFN2}} \ \var{SURFAC(1,IFAC)}\\
&&\\
&\displaystyle \var{GRDPY} &= \var{GRDPY} - \displaystyle\frac{\var{GRDSN}}{\var{SURFN2}} \ \var{SURFAC(2,IFAC)} \\
&&\\
&\displaystyle \var{GRDPZ} &= \var{GRDPZ} - \displaystyle\frac{\var{GRDSN}}{\var{SURFN2}} \ \var{SURFAC(3,IFAC)}
\end{array}\right.
\end{equation}
\item [$\Rightarrow$] finalisation du calcul de l'expression totalement
explicite
 $$\left[ \tens{D}^n\,\left( \grad{R^{\,n}_{ij}} - (\grad R^{\,n}_{ij}\,.\,\vect{n}_{\,lm})\,\vect{n}_{\,lm}\right) \right]\,.\,\vect{n}_{\,lm}$$
\begin{equation}\notag
\begin{array} {ll}
\displaystyle \var{VISCF} = &
 \displaystyle\frac{1}{2} (\ \var{W4(II)} +\ \var{W4(JJ)}) \ \var{GRDPX} \
\var{SURFAC(1,IFAC)})\ + \\
&\\
&  \displaystyle\frac{1}{2} (\ \var{W5(II)} +\ \var{W5(JJ)}) \ \var{GRDPY} \
\var{SURFAC(2,IFAC)})\ + \\
&\\
&  \displaystyle\frac{1}{2} (\ \var{W6(II)} +\ \var{W6(JJ)}) \ \var{GRDPZ} \ \var{SURFAC(3,IFAC)})
\end{array}
\end{equation}
\end{itemize}

\item [$\star$] Mise \`a z\'ero du tableau \var{VISCB}.

\item [$\star$] Appel de \fort{divmas} pour calculer la divergence de~:
 $$\tens{D}^{\,n}\,\left( \grad{R^{\,n}_{ij}} - (\grad R^{\,n}_{ij}\,.\,\vect{n}_{\,lm})\vect{n}_{\,lm}\right)$$ d\'efini aux faces dans \var{VISCF} et \var{VISCB}.

Le r\'esultat est stock\'e dans le tableau \var{W1}.\\
Ajout au second membre \var{SMBR}.\\
$\var{SMBR} = \var{SMBR} + \var{W1}$
\end{itemize}
\item Calcul de la viscosit\'e orthotrope $\gamma^n_{\,lm}$ \`a la face $lm$ de la variable principale
$R^{\,n}_{ij}$\\
Ce calcul permet au sous-programme \fort{codits} de compl\'eter le second membre
\var{SMBR} par :
\begin{equation}
\begin{array} {ll}
& \sum\limits_{m\in Vois(l)}
\mu^n_{\,lm}\,\left(\grad{R}^{\,n}_{ij}\,.\,\vect{n}_{\,lm}\right)S_{\,lm}
 + \sum\limits_{m\in Vois(l)} \left(\grad{R}^{\,n}_{ij}
\,.\,\vect{n}_{\,lm}\right)\left[\tens{D}^{\,n}\,\vect{n}_{\,lm}\right]_{\,lm}\,.\,\vect{n}_{\,lm}\
S_{\,lm}\\
& = \sum\limits_{m\in Vois(l)}(\,\mu^n_{\,lm}\, + \,\gamma^n_{\,lm}\,)\,\left(\grad{R}^{\,n}_{ij}\,.\,\vect{n}_{\,lm}\right)S_{\,lm}
\end{array}
\end{equation}
sans pr\'eciser la nature de la face $lm$, {\it via} l'appel \`a \fort{bilsc2}  et de disposer de la quantit\'e
$(\mu^n_{\,lm}\, + \gamma^n_{\,lm})$ pour construire sa
matrice simplifi\'ee.\\
\begin{itemize}
\item [$\star$] On effectue une boucle d'indice \var{IEL} sur les cellules
$\Omega_l$ :
\begin{itemize}
\item [$\Rightarrow$] $\displaystyle \var{TRRIJ }= \frac{1}{2} (R^{\,n}_{ii})_L $
\item [$\Rightarrow$] $\displaystyle \var{RCSTE} = \rho^n_L \ C_S \ \frac{ (R^{\,n}_{ii})_L}{2\,\varepsilon^n_L} $
\item [$\Rightarrow$] $\displaystyle \var{W1(IEL)} = \mu^n +\rho^n_L \ C_S \ \frac{
(R^{\,n}_{ii})_L}{2\,\varepsilon^n_L}\ (R^n_{11})_L$
\item [$\Rightarrow$] $\displaystyle \var{W2(IEL)} = \mu^n + \rho^n_L \ C_S \ \frac{ (R^{\,n}_{ii})_L}{2\,\varepsilon^n_L}\ (R^n_{22})_L$
\item [$\Rightarrow$] $\displaystyle \var{W3(IEL)} = \mu^n + \rho^n_L \ C_S \ \frac{ (R^{\,n}_{ii})_L}{2\,\varepsilon^n_L}\ (R^n_{33})_L$
\end{itemize}

\item [$\star$] Appel de \fort{visort} pour calculer la viscosit\'e orthotrope
\footnote{Comme dans le sous-programme \fort{viscfa}, on multiplie la viscosit\'e par
$\displaystyle \frac{S_{\,lm}}{\overline{L'M'}}$, o\`u $S_{\,lm}$ et
$\overline{L'M'}$ repr\'esentent respectivement la surface de la face $lm$ et la
mesure alg\'ebrique du segment reliant les projections des centres des cellules
voisines sur la normale \`a la face. On garde dans ce sous-programme  la possibilit\'e d'interpoler la viscosit\'e aux cellules lin\'eairement ou d'utiliser une moyenne harmonique. La viscosit\'e au bord est celle de la cellule de bord correspondante.}
$ \gamma^n_{\,lm} = (\tens{D}^{\,n}\,\vect{n}_{\,lm}).\vect{n}_{\,lm}$ aux faces $lm$

Le r\'esultat est stock\'e dans les tableaux \var{VISCF} et \var{VISCB}.
\end{itemize}

\item appel de \fort{codits} pour la r\'esolution de l'\'equation de
convection/diffusion/termes sources de la variable $R_{ij}$. Le terme source est
\var{SMBR}, la viscosit\'e \var{VISCF} aux faces purement internes (
resp. \var{VISCB} aux faces de bord) et \var{FLUMAS} le flux de masse interne
 ( resp. \var{FLUMAB} flux de masse au bord). Le r\'esultat est la variable $R_{ij}$ au temps
$n+1$, donc $R^{\,n+1}_{ij}$. Elle est stock\'ee dans le tableau \var{RTP} des
variables mises \`a jour.

\end{itemize}

\etape{Appel de \fort{reseps} pour la r\'esolution de la variable $\varepsilon$}

On d\'ecrit ci-dessous le sous-programme \fort{reseps}, les commentaires du sous-programme \fort{resrij} ne seront pas r\'ep\'et\'es puisque les deux sous-programmes ne diff\`erent que par leurs termes sources.

\begin{itemize}
\item Initialisation \`a z\'ero de \var{SMBR} et \var{ROVSDT}.

\item{Lecture et prise en compte des termes sources utilisateur pour la variable $\varepsilon$ :}

Appel de \fort{ustsri} pour charger les termes sources utilisateurs. Ils sont
stock\'es dans les tableaux suivants :\\
pour la cellule $\Omega_l$ repr\'esent\'ee par $\var{IEL}$ de centre $L$, on a :
\begin{equation}\notag
\left\{\begin{array}{lll}
&\var{ROVSDT(IEL)}&= |\Omega_l| \ \alpha_{\varepsilon}\\
&\var{SMBR(IEL)}&=|\Omega_l| \ \beta_{\varepsilon}\\
\end{array}\right.
\end{equation}
On affecte alors les valeurs ad\'equates au second membre \var{SMBR} et \`a la
diagonale \var{ROVSDT} comme suit :
\begin{equation}\notag
\left\{\begin{array}{lll}
&\var{SMBR(IEL)} &= \var{SMBR(IEL)} +\ |\Omega_l| \ \alpha_{\,\varepsilon} \
\varepsilon^n_L \\
&\var{ROVSDT(IEL)}&= \text{max }(-\ |\Omega_l| \ \alpha_{\,\varepsilon},0)\\
\end{array}\right.
\end{equation}

\item{Calcul du terme source de masse si $\Gamma_L > 0$ :
\begin{equation}\notag
\left\{\begin{array}{lll}
&\displaystyle \var{SMBR(IEL)} = \var{SMBR(IEL)} + |\Omega_l| \ \Gamma_L \
(\varepsilon^{\,in}_L -\varepsilon^n_L) \\
&\displaystyle \var{ROVSDT(IEL)}= \var{ROVSDT(IEL)} + |\Omega_l| \ \Gamma_L
\end{array}\right.
\end{equation}
\item Calcul du terme d'accumulation de masse et du terme instationnaire \\
On stocke $\displaystyle \var{W1}= \int_{\Omega_l}\dive\,(\rho\,\vect{u})\,d\Omega$
calcul\'e par \fort{divmas} \`a l'aide des flux de masse internes et aux bords.\\
On incr\'emente ensuite \var{SMBR} et \var{ROVSDT}.
\begin{equation}\notag
\left\{\begin{array}{lll}
&\var{SMBR(IEL)} &= \var{SMBR(IEL)} + \var{ICONV}\ \varepsilon^n_L\,(\displaystyle
\int_{\Omega_l}\dive\,(\rho\,\vect{u})\ d\Omega) \\
&\var{ROVSDT(IEL)}& = \var{ROVSDT(IEL)} +  \var{ISTAT}\,\displaystyle
\frac{\rho^n_L \ |\Omega_l|}{\Delta t^n} -  \var{ICONV}\ (\displaystyle
\int_{\Omega_l}\dive\,(\rho\,\vect{u})\ d\Omega) \\
\end{array}\right.
\end{equation}

\item Traitement du terme de production
 $\displaystyle \rho\,C_{\varepsilon_1}\,\frac{\varepsilon}{k}\,\mathcal{P}$
 et du terme de dissipation $-\,\displaystyle \rho\,C_{\varepsilon_2}\,\frac{\varepsilon}{k}\,\varepsilon$ \\
pour cela, on effectue une boucle d'indice \var{IEL} sur les cellules $\Omega_l$
de centre $L$ :
\begin{itemize}
\item [$\Rightarrow$] $\displaystyle \var{TRPROD}= \frac{1}{2} (\mathcal{P}^n_{ii})_L = \frac{1}{2} \left[ \var{PRODUC(1,IEL)} +  \var{PRODUC(2,IEL)} +  \var{PRODUC(3,IEL)} \right] $
\item [$\Rightarrow$] $\displaystyle \var{TRRIJ }= \frac{1}{2} (R^n_{ii})_L $
\item [$\Rightarrow$] $\displaystyle \var{SMBR(IEL)} = \var{SMBR(IEL)} + \rho^n_L
|\Omega_l| \left[ -C_{\varepsilon_2} \ \frac{2\,(\varepsilon^n_L)^2}{(R^n_{ii})_L} + C_{\varepsilon_1} \ \frac{\varepsilon^n_L}{(R^n_{ii})_L}\ (\mathcal{P}^n_{ii})_L \right] $
\item [$\Rightarrow$] $\displaystyle \var{ROVSDT(IEL)} = \var{ROVSDT(IEL)} + C_{\varepsilon_2} \ \rho^n_L \ |\Omega_l| \ \frac{2\,\varepsilon^n_L}{(R^n_{ii})_L}$
\end{itemize}

\item Appel de \fort{rijthe} pour le calcul des termes de gravit\'e $\mathcal{G}^n_{\varepsilon}$ et ajout dans \var{SMBR}.

$ \var{SMBR} = \var{SMBR} + \mathcal{G}^n_{\varepsilon}$\\
Ce calcul n'a lieu que si $\var{IGRARI()} = 1$.

\item Calcul de la diffusion de $\varepsilon$ \\
 Le terme $\dive \left[\mu\, \grad(\varepsilon) + \tens{A'}\,\grad(\varepsilon)
\right]$ est calcul\'e exactement de la m\^eme mani\`ere que pour les tensions
de Reynolds $R_{ij}$ en rempla\c cant $\tens{A}$ par $\tens{A'}$.

\item Appel de \fort{codits} pour la r\'esolution de l'\'equation de
convection/diffusion/termes sources de la variable principale $\varepsilon$. Le
r\'esultat $\varepsilon^{\,n+1}$ est stock\'e dans le tableau \var{RTP} des
variables mises \`a jour.
}
\end{itemize}

\etape{clippings finaux}
On passe enfin dans le sous-programme  \fort{clprij} pour faire un clipping \'eventuel
des variables $R^{\,n+1}_{ij}$ et $\varepsilon^{\,n+1}$. Le sous-programme
\fort{clprij} est appel\'e\footnote{L'option
$\var{ICLIP} = 1$ consiste en un clipping minimal des variables $R_{ii}$ et
$\varepsilon$ en prenant la valeur absolue de ces variables puisqu'elles ne
peuvent \^etre que positives.} avec $\var{ICLIP} = 2$ . Cette option
\footnote{Quand la valeur des grandeurs $R_{ii}$ ou $\varepsilon$ est
n\'egative, on la remplace par le minimum entre sa valeur absolue et (1,1)
fois la valeur obtenue au pas de temps pr\'ec\'edent.} contient l'option $\var{ICLIP} = 1$  et permet de v\'erifier l'in\'egalit\'e de Cauchy-Schwarz sur les grandeurs extra-diagonales du tenseur $\tens{R}$ (pour $i \neq j$, $|R_{ij}|^2 \le R_{ii} R_{jj}$).


%%%%%%%%%%%%%%%%%%%%%%%%%%%%%%%%%%
%%%%%%%%%%%%%%%%%%%%%%%%%%%%%%%%%%
\section{Points \`a traiter}
%%%%%%%%%%%%%%%%%%%%%%%%%%%%%%%%%%
%%%%%%%%%%%%%%%%%%%%%%%%%%%%%%%%%%
Sauf mention explicite, $\phi$ repr\'esentera une tension de Reynolds ou la dissipation turbulente ($\phi = R_{ij} \ \text{ou} \ \varepsilon$).

\begin{itemize}
\item {La vitesse utilis\'ee pour le calcul de la production est explicite. Est-ce qu'une implicitation peut am\'eliorer la pr\'ecision temporelle de $\phi$ \footnote{Cette remarque peut \^etre g\'en\'eralis\'ee. En effet, peut-on envisager d'actualiser les variables d\'ej\`a r\'esolues (sans r\'eactualiser les variables turbulentes apr\`es leur r\'esolution)? Ceci obligerait \`a modifier les sous-programmes tels que \fort{condli} qui sont appel\'es au d\'ebut de la boucle en temps.} ?}
\item {Dans quelle mesure le terme d'\'echo de paroi est-il valide ? En effet, ce terme est remis en question par certains auteurs.}
\item {On peut envisager la r\'esolution d'un syst\`eme hyperbolique pour les
tensions de Reynolds afin d'introduire un couplage avec le champ de vitesse.}
\item {Le flux au bord \var{VISCB} est annul\'e dans le sous-programme
\fort{vectds}. Peut-on envisager de mettre au bord la valeur de la variable
concern\'ee \`a la cellule de bord correspondant? De m\^eme, il faudrait se
pencher sur les hypoth\`eses sous-jacentes \`a l'annulation des contributions
aux bords de \var{VISCB} lors du calcul de : $$\left[ \tens{D}^n\,\left( \grad{R^{\,n}_{ij}} - (\grad R^{\,n}_{ij}\,.\,\vect{n}_{\,lm})\,\vect{n}_{\,lm}\right) \right]\,.\,\vect{n}_{\,lm}.$$}
\item {Un probl\`eme de pond\'eration appara\^\i t plus g\'en\'eralement. Si on prend la partie explicite de $\tens{D}\,\grad(\phi)$, la pond\'eration aux faces internes utilise le coefficient $\displaystyle\frac{1}{2}$ avec pond\'eration s\'epar\'ee de $\tens{D}$ et $\grad(\phi)$, alors que pour $\tens{E}\,\grad(\phi)$, on calcule d'abord ce terme aux cellules pour ensuite l'interpoler lin\'eairement aux faces \footnote{Cette interpolation se fait dans \fort{vectds} avec des coefficients de pond\'eration aux faces.}. Ceci donne donc deux types d'interpolations pour des termes de m\^eme nature.}
\item {On laisse la possibilit\'e dans \fort{visort} d'utiliser une moyenne
harmonique aux faces. Est-ce que ceci est valable puisque les interpolations
utilis\'ees lors du calcul de la partie explicite de $\tens{A}\,\grad{\phi}$
sont des moyennes arithm\'etiques ?}
\item {Les techniques adopt\'ees lors du clipping sont \`a revoir.}
\item {On utilise dans le cadre du mod\`ele $\displaystyle R_{ij}-\varepsilon $ une semi-implicitation de termes comme $\displaystyle \phi_{ij,1}$ ou $\displaystyle -\rho\,C_{\varepsilon_2}\,\frac{\varepsilon}{k}\,\varepsilon$. On peut envisager le m\^eme type d'implicitation dans \fort{turbke} m\^eme en pr\'esence du couplage $\displaystyle k-\varepsilon$.}
\item L'adoption d'une r\'esolution d\'ecoupl\'ee fait perdre l'invariance par rotation.
\item La formulation et l'implantation des conditions aux limites de paroi
devront \^etre v\'erifi\'ees. En effet, il semblerait que, dans certains cas, des ph\'enom\`enes
``oscillatoires'' apparaissent, sans qu'il soit ais\'e d'en d\'eterminer la cause.
\item L'implicitation partielle (du fait de la r\'esolution d\'ecoupl\'ee) des
conditions aux limites conduit souvent \`a des calculs instables. Il
conviendrait d'en conna\^\i tre la raison. L'implicitation partielle avait
\'et\'e mise en \oe uvre afin de tenter d'utiliser un pas de temps plus grand
dans le cas de jets axisym\'etriques en particulier.

\end{itemize}

%                      Code_Saturne version 1.3
%                      ------------------------
%
%     This file is part of the Code_Saturne Kernel, element of the
%     Code_Saturne CFD tool.
%
%     Copyright (C) 1998-2007 EDF S.A., France
%
%     contact: saturne-support@edf.fr
%
%     The Code_Saturne Kernel is free software; you can redistribute it
%     and/or modify it under the terms of the GNU General Public License
%     as published by the Free Software Foundation; either version 2 of
%     the License, or (at your option) any later version.
%
%     The Code_Saturne Kernel is distributed in the hope that it will be
%     useful, but WITHOUT ANY WARRANTY; without even the implied warranty
%     of MERCHANTABILITY or FITNESS FOR A PARTICULAR PURPOSE.  See the
%     GNU General Public License for more details.
%
%     You should have received a copy of the GNU General Public License
%     along with the Code_Saturne Kernel; if not, write to the
%     Free Software Foundation, Inc.,
%     51 Franklin St, Fifth Floor,
%     Boston, MA  02110-1301  USA
%
%-----------------------------------------------------------------------
%
\programme{vortex}
%
\vspace{1cm}
%%%%%%%%%%%%%%%%%%%%%%%%%%%%%%%%%%
%%%%%%%%%%%%%%%%%%%%%%%%%%%%%%%%%%
\section{Fonction}
%%%%%%%%%%%%%%%%%%%%%%%%%%%%%%%%%%
%%%%%%%%%%%%%%%%%%%%%%%%%%%%%%%%%%
Ce sous-programme est d�di� � la g�n�ration des conditions d'entr�e
turbulente utilis�es en LES.


La m�thode des vortex est bas�e sur une approche de tourbillons
ponctuels. L'id�e de la m�thode consiste � injecter des tourbillons 2D dans le
plan d'entr�e du calcul, puis � calculer le champ de vitesse induit par ces
tourbillons au centre des faces d'entr�e.

%                      Code_Saturne version 1.3
%                      ------------------------
%
%     This file is part of the Code_Saturne Kernel, element of the
%     Code_Saturne CFD tool.
% 
%     Copyright (C) 1998-2007 EDF S.A., France
%
%     contact: saturne-support@edf.fr
% 
%     The Code_Saturne Kernel is free software; you can redistribute it
%     and/or modify it under the terms of the GNU General Public License
%     as published by the Free Software Foundation; either version 2 of
%     the License, or (at your option) any later version.
% 
%     The Code_Saturne Kernel is distributed in the hope that it will be
%     useful, but WITHOUT ANY WARRANTY; without even the implied warranty
%     of MERCHANTABILITY or FITNESS FOR A PARTICULAR PURPOSE.  See the
%     GNU General Public License for more details.
% 
%     You should have received a copy of the GNU General Public License
%     along with the Code_Saturne Kernel; if not, write to the
%     Free Software Foundation, Inc.,
%     51 Franklin St, Fifth Floor,
%     Boston, MA  02110-1301  USA
%
%-----------------------------------------------------------------------
%
%%%%%%%%%%%%%%%%%%%%%%%%%%%%%%%%%%
%%%%%%%%%%%%%%%%%%%%%%%%%%%%%%%%%%
\section{Discr\'etisation}
%%%%%%%%%%%%%%%%%%%%%%%%%%%%%%%%%%
%%%%%%%%%%%%%%%%%%%%%%%%%%%%%%%%%%

Le terme convectif en $\dive(\underline{u} \otimes \rho\,\underline{u})$
introduit une non lin\'earit\'e et un couplage des composantes de la vitesse
$\vect{u}$ dans l'�quation (\ref{Base_Preduv_eqqdm}). Une lin\'earisation et un d\'ecouplage
des trois composantes de la 
vitesse sont r\'ealis\'es lors de la discr\'etisation de cette \'etape de
pr\'ediction.\\
En effet, soit :
\begin{equation}
\vect{\widetilde{u}}= \vect{u}^n + \delta \vect{u} 
\end{equation}
La contribution exacte du terme convectif \`a prendre en compte dans cette
\'etape de pr\'ediction serait :\\
\begin{equation}\label{Base_Preduv_Conv_exact}
\begin{array}{ll}
\dive(\vect{\widetilde{u}} \otimes \rho\,\vect{\widetilde{u}}) =
\dive(\vect{u}^{n} \otimes \rho\,\vect{u}^{n}) + \dive(\delta \vect{u} \otimes
\rho\,\vect{u}^{n}) +  \underbrace { \dive(\vect{u}^{n} \otimes
\rho\,\delta \vect{u})}_{\text {terme couplant lin\'eaire}} +  \underbrace { \dive(\delta \vect{u} \otimes
\rho\,\delta \vect{u})}_{\text {terme couplant et non lin\'eaire}}\\
\end{array} 
\end{equation}
Les deux derniers termes de l'expression (\ref{Base_Preduv_Conv_exact}) sont {\em a priori} n�glig�s
de mani�re � obtenir un syst\`eme en vitesse qui soit d\'ecoupl\'e et donc,
�viter l'inversion d'une matrice pouvant \^etre de tr\`es grande taille. Ces
deux termes peuvent n�anmoins �tre pris en compte de mani�re plus ou moins
approch�e par extrapolation explicite du flux de masse en $n+\theta_F$ (pour le
terme couplant lin�aire provenant de la convection de $\vect{u}^{n}$ par $\delta
\vect{u}$) et utilisation d'un point-fixe par sous it�ration sur le sous
programme \fort{navsto} (pour le terme non-lin�aire, en sp�cifiant $\var{NTERUP}>1$).

L'�quation (\ref{Base_Preduv_eqqdm}) est discr�tis�e au temps $n+\theta$ � l'aide d'un
$\theta$-sch�ma, et le tenseur des pertes de charges d�compos� en une partie
diagonale $\tens{K}_{d}$ et une extradiagonale $\tens{K}_{e}$ (soit
 $\tens{K}_{pdc}=\tens{K}_{d}+\tens{K}_{e}$).\\
$\bullet$ La pression est suppos�e connue en $n-1+\theta$ (d�calage temporel
pression-vitesse) et le gradient naturellement calcul� � cet instant.\\ 
$\bullet$ Les termes sources de viscosit� secondaire, de gradient transpos\'e,
ceux provenant du mod�le de turbulence\footnote{except� $\dive (\mu_t\ (\ggrad
\underline {u}))$}, $\rho\,\tens{K}_{\,e}\ \underline{u}$, $(\rho -\rho_0)
\underline {g}$ ainsi que $\underline{T}_{s}^{\,exp}$ et
$\Gamma\,\underline{u}_{\,i}$ sont pris de mani�re explicite au temps $n$, ou
extrapol�s suivant le sch�ma en temps choisi pour les propri�t�s physique et les
termes sources.\\ 
$\bullet$ Les termes sources $\underline{u}\,\,\dive (\rho\,\underline {u})$,
$\Gamma\,\,\underline{u}$, $T_{s}^{\,imp}\,\,\underline{u}$ et
$-\rho\,\tens{K}_{\,d}\,\,\underline{u}$ sont implicit�s est calcul�s �
l'instant $n+\theta$.\\ 
$\bullet$ Le terme de diffusion $\dive (\mu_{\,tot}\,\ggrad \underline{u})$ est
implicit� : la vitesse est prise � l'instant $n+\theta$ et la viscosit�
explicit�e ou extrapol�e.\\ 
$\bullet$ Enfin, le terme de convection en $\dive(\,\underline{u} \otimes
(\rho\underline{u})\,)$ est implicit� : la composante r�solue de la vitesse est
prise en $n+\theta$, et le flux de masse, explicit�, ou extrapol� en
$n+\theta_F$. 

Par souci de clart�, on suppose, en l'absence d'indication, que les propri�tes
physiques $\Phi$ ($\rho,\,\mu_{tot},\,...$) et le flux de masse
$(\rho\underline{u})$ sont pris respectivement aux instants $n+\theta_\Phi$ et
$n+\theta_F$, o� $\theta_\Phi$ et $\theta_F$ d�pendent des sch�mas en temps
sp�cifiquement utilis�s pour ces grandeurs\footnote{cf. \fort{introd}}. 

La discr�tisation temporelle de l'�quation (\ref{Base_Preduv_eqqdm}) s'�crit alors comme suit : 

\begin{equation}\label{Base_Preduv_eq_di1}
 \begin{array}{c}
\displaystyle \rho\,\ \frac{ \underline {\widetilde{u}}^{n+1} -\underline {u}^{n} }
{\Delta t} + \dive(\,\underline{\widetilde{u}}^{n+\theta} \otimes (\rho\underline{u})\,) -\dive
(\mu_{\,tot}\,\ggrad \underline{\widetilde{u}}^{n+\theta}) =
\\
\displaystyle
 - \grad p^{n-1+\theta} + \dive (\rho\,\underline {u})\,\underline{\widetilde{u}}^{n+\theta} +(\Gamma\,\underline{u}_{\,i})^{n+\theta_S}-\Gamma^n\,\,\underline{\widetilde{u}}^{n+\theta}
\\
\begin{array}{c}
\displaystyle
- \rho\,\tens{K}_{\,d}^{n}\,\,\underline{\widetilde{u}}^{n+\theta} - (\rho\,\tens{K}_{\,e}\ \underline{u})^{n+\theta_S} + (\underline{T}_{s}^{\,exp})^{\,n+\theta_S} + T_{s}^{\,imp}\,\,\underline{\widetilde{u}}^{n+\theta}
\\
\displaystyle
+[\dive (\mu_{\,tot}\,^t\ggrad \underline {u})]^{n+\theta_S}-\frac {2} {3}[\,\grad (\mu_{\,tot}\,\dive \underline {u})]^{n+\theta_S} + (\rho -\rho_0) \underline {g}
 - (\underline{turb})^{n+\theta_S}
\end{array}
\end{array}
\end{equation}
o\`u, par souci de simplification, on a pos\'e :
\begin{equation}
\mu_{\,tot}=
\begin{cases}
\mu+\mu_t & \text{pour les mod�les � viscosit� turbulente ou en LES}, \\
\mu & \text{pour les mod�les au second ordre ou en laminaire}
\end{cases} \ 
\end{equation}
\\
et :
\begin{equation}
\underline{turb}^{n}=
\begin{cases}
\displaystyle\frac {2}{3}\grad (\rho^{n}\,k^{n}) & \text{pour les mod�les � viscosit� turbulente}, \\
\dive(\rho^{n}\,\tens{R}^n) & \text{pour les mod�les au second ordre},\\
0 & \text{en laminaire ou en LES}\\
\end{cases}
\end{equation}
Par analogie avec l'�criture du $\theta$-sch�ma pour une variable scalaire, $\,
\underline {\widetilde{u}}^{n+\theta}$ est interpol�e � partir de la vitesse
pr�dite $\underline {\widetilde{u}}^{n+1}$ de la mani\`ere suivante\footnote{si
$\theta=1/2$, ou qu'une extrapolation est utilis�e, l'ordre 2 n'est obtenu que si
le pas de temps $\Delta t$ est uniforme en temps et en espace.}~: 
\begin{equation}
\underline {\widetilde{u}}^{n+\theta}=\theta\, \underline
{\widetilde{u}}^{n+1}+(1-\theta)\, \underline {u}^{n}\\ 
\end{equation}
Avec :
\begin{equation}
\left\{
\begin{array}{ll}
\theta = 1   & \text{Pour un sch\'ema de type Euler implicite d'ordre 1.}\\
\theta = 1/2 & \text{Pour un sch\'ema de type Cranck-Nicolson d'ordre 2.}\\
\end{array}
\right.
\end{equation}

L'�quation (\ref{Base_Preduv_eq_di1}) est alors r��crite sous la forme :

\begin{equation}\label{Base_Preduv_eq_di2}
\begin{array}{c}
\displaystyle \underbrace{\left(\frac{\rho}{\Delta t} -\theta \,\dive (\rho\,\underline {u}) +\theta \,\, \Gamma^n +
\theta \,\, \rho\,\tens{K}_{\,d}^n-\theta \,T_s^{\,imp} \right)}_{\displaystyle f_s^{imp}}\, (\underline {\,\widetilde{u}}^{n+1} -\underline {u}^{n})
\\
 +\, \theta\, \dive(\underline {\widetilde{u}}^{n+1} \otimes (\rho\underline{u}))-\, \theta\,\dive (\mu_{\,tot}\,\ggrad \underline {\widetilde{u}}^{n+1}) =
\\
-\,(1-\theta)\, \dive(\underline {u}^{n} \otimes (\rho\underline{u})) +\,(1-\theta)\,\dive (\mu_{\,tot}\,\ggrad \underline {u}^{n})
\\
f_s^{exp}\left\{
\begin{array}{c}
\displaystyle 
- \grad p^{n-1+\theta} + \dive (\rho\,\underline {u})\,\underline{u}^{n} +\,(\,\Gamma^{n}\,\underline{u}_{\,i}\,)^{n+\theta_S}- \Gamma^n\,\,\underline{u}^{n}
\\
\displaystyle
-(\,\rho\,\tens{K}_{\,e}\ \underline{u}\,)^{n+\theta_S} -\rho\,\tens{K}_{\,d}^n\ \underline{u}^{n}+ (\underline{T}_{s}^{\,exp})^{\,n+\theta_S} + T_s^{\,imp}\,\,\underline {u}^{n} 
\\
\displaystyle
+[\dive (\mu_{\,tot}\,^t\ggrad \underline {u}\,)]^{n+\theta_S}-\frac {2} {3}[\,\grad (\mu_{\,tot}\,\dive \underline {u}\,)]^{n+\theta_S} + (\rho -\rho_0) \underline {g}-(\underline{turb})^{n+\theta_S}
\end{array}
\right.
\end{array}
\end{equation}

d'o� l'�quation r�solue par le sous-programme \fort{codits} :
\begin{equation}\begin{array}{c}
\displaystyle
f_s^{\,imp}(\underline {\widetilde{u}}^{n+1}-\underline {u}^{n}) + \theta\, \dive(\underline{\widetilde{u}}^{n+1} \otimes (\rho
\underline{u})) - \theta\,\dive (\,\mu_{\,tot}\,\ggrad \underline{\widetilde{u}}^{n+1}) = 
\\\\
\displaystyle
-(1-\theta)\,\dive(\underline{u}^{n} \otimes (\rho \underline{u}))+(1-\theta)\,\dive (\,\mu_{\,tot}\,\ggrad \underline{u}^{n})
+ \underline{f}_{\,s}^{\,exp}
\end{array}
\end{equation}
La m\'ethode de discr\'etisation spatiale est d\'evelopp\'ee dans le sous-programme \fort{codits}.\\



\minititre{Remarques :}
{\tiny$\blacksquare$} Dans le cas standard sans extrapolation, le terme
$-\,T_s^{\,imp}$ n'est ajout� � $f_s^{\,imp}$ que s'il est positif afin de ne
pas affaiblir la dominance de la diagonale de la matrice � inverser.\\ 
{\tiny$\blacksquare$} Si une extrapolation est utilis�e, par contre,
$\,T_s^{\,imp}$ est ajout� � $f_s^{\,imp}$ quel que soit son signe. En effet, l'id�e intuitive qui
consiste � prendre~: 
\begin{equation}
\begin{cases}
\displaystyle
(\underline{T}_{s}^{\,exp} + T_{s}^{\,imp}\,\underline {u})^{\,n+\theta_S} &
\text{si } T_{s}^{\,imp} > 0\\ 
\displaystyle
(\underline{T}_{s}^{\,exp})^{\,n+\theta_S} + T_{s}^{\,imp}\,\underline{u}^{n+\theta} &\text{sinon}\\
\end{cases}
\end{equation} 
aboutit � une incoh�rence dans le traitement si $T_s^{imp}$ change de signe
entre deux pas de temps.\\ 
{\tiny$\blacksquare$} la partie diagonale $\tens{K}_{\,d}$ du terme
de perte de charge est utilis�e dans $f_s^{\,imp}$. En fait, pour \^etre rigoureux,
il faudrait ne retenir que les contributions positives (point signal\'e dans le
sous-programme utilisateur associ\'e \fort{uskpdc}). Cette prise en compte sera \`a am\'eliorer.\\
{\tiny$\blacksquare$} Le terme $\theta\,\Gamma^{n}-\theta\,\dive
(\rho\,\underline {u})$ ne pose pas de probl�me pour la 
dominance de la diagonale de la matrice car il est exactement compens� par le
terme de convection (cf. \fort{covofi}). 


%                      Code_Saturne version 1.3
%                      ------------------------
%
%     This file is part of the Code_Saturne Kernel, element of the
%     Code_Saturne CFD tool.
%
%     Copyright (C) 1998-2007 EDF S.A., France
%
%     contact: saturne-support@edf.fr
%
%     The Code_Saturne Kernel is free software; you can redistribute it
%     and/or modify it under the terms of the GNU General Public License
%     as published by the Free Software Foundation; either version 2 of
%     the License, or (at your option) any later version.
%
%     The Code_Saturne Kernel is distributed in the hope that it will be
%     useful, but WITHOUT ANY WARRANTY; without even the implied warranty
%     of MERCHANTABILITY or FITNESS FOR A PARTICULAR PURPOSE.  See the
%     GNU General Public License for more details.
%
%     You should have received a copy of the GNU General Public License
%     along with the Code_Saturne Kernel; if not, write to the
%     Free Software Foundation, Inc.,
%     51 Franklin St, Fifth Floor,
%     Boston, MA  02110-1301  USA
%
%-----------------------------------------------------------------------
%

%%%%%%%%%%%%%%%%%%%%%%%%%%%%%%%%%%
%%%%%%%%%%%%%%%%%%%%%%%%%%%%%%%%%%
\section{Mise en \oe uvre}
%%%%%%%%%%%%%%%%%%%%%%%%%%%%%%%%%%
%%%%%%%%%%%%%%%%%%%%%%%%%%%%%%%%%%
La num\'ero de la phase trait\'ee fait partie des arguments de \fort{turrij}. On
omettra volontairement de le pr\'eciser dans ce qui suit, on indiquera par $(\ )$ la
notion de tableau s'y rattachant.

\etape{Calcul des termes de production $\tens{\mathcal{P}}$}
\begin{itemize}
\item [$\star$] Initialisation \`a z\'ero du tableau \var{PRODUC} dimensionn\'e \`a $\var{NCEL}\times 6$.
\item [$\star$] On appelle trois fois \fort{grdcel} pour calculer les gradients des composantes de la vitesse $u$, $v$ et
$w$ prises au temps $n$.

Au final, on a :\\
$\displaystyle
\begin{array} {ll}
\var{PRODUC(1,IEL)} = & \displaystyle - 2 \left[ R_{11}^{\,n} \frac{\partial u^{\,n}} {\partial x} +R_{12}^{\,n} \frac{\partial u^{\,n}} {\partial y}+R_{13}^{\,n} \frac{\partial u^{\,n}} {\partial z} \right] \text{        (production de $R_{11}^{\,n}$)}\\
\var{PRODUC(2,IEL)} = & \displaystyle - 2 \left[ R_{12}^{\,n} \frac{\partial v^{\,n}} {\partial x} +R_{22}^{\,n} \frac{\partial v^{\,n}} {\partial y}+R_{23}^{\,n} \frac{\partial v^{\,n}} {\partial z} \right] \text{        (production de $R_{22}^{\,n}$)}\\
\var{PRODUC(3,IEL)} = & \displaystyle - 2 \left[ R_{13}^{\,n} \frac{\partial w^{\,n}} {\partial x} +R_{23}^{\,n} \frac{\partial w^{\,n}} {\partial y}+R_{33}^{\,n} \frac{\partial w^{\,n}} {\partial z} \right] \text{        (production de $R_{33}^{\,n}$)}\\
\var{PRODUC(4,IEL)} = & \displaystyle - \left[ R_{12}^{\,n} \frac{\partial u^{\,n}} {\partial x} +R_{22}^{\,n} \frac{\partial u^{\,n}} {\partial y}+R_{23}^{\,n} \frac{\partial u^{\,n}} {\partial z} \right] \\
& \displaystyle - \left[ R_{11}^{\,n} \frac{\partial v^{\,n}} {\partial x} +R_{12}^{\,n} \frac{\partial v^{\,n}} {\partial y}+R_{13}^{\,n} \frac{\partial v^{\,n}} {\partial z} \right] \text{        (production de $R_{12}^{\,n}$)} \\
\var{PRODUC(5,IEL)} = & \displaystyle - \left[ R_{13}^{\,n} \frac{\partial u^{\,n}} {\partial x} +R_{23}^{\,n} \frac{\partial u^{\,n}} {\partial y}+R_{33}^{\,n} \frac{\partial u^{\,n}} {\partial z} \right] \\
& \displaystyle - \left[ R_{11}^{\,n} \frac{\partial w^{\,n}} {\partial x} +R_{12}^{\,n} \frac{\partial w^{\,n}} {\partial y}+R_{13}^{\,n} \frac{\partial w^{\,n}} {\partial z} \right] \text{        (production de $R_{13}^{\,n}$)} \\
\var{PRODUC(6,IEL)} = & \displaystyle - \left[ R_{13}^{\,n} \frac{\partial v^{\,n}} {\partial x} +R_{23}^{\,n} \frac{\partial v^{\,n}} {\partial y}+R_{33}^{\,n} \frac{\partial v^{\,n}} {\partial z} \right] \\
& \displaystyle - \left[ R_{12}^{\,n} \frac{\partial w^{\,n}} {\partial x} +R_{22}^{\,n} \frac{\partial w^{\,n}} {\partial y}+R_{23}^{\,n} \frac{\partial w^{\,n}} {\partial z} \right]  \text{        (production de $R_{23}^{\,n}$)}
\end{array}
$
\end{itemize}

\etape{Calcul du gradient de la masse volumique $\rho^n$ prise au d\'ebut du pas
de temps courant\footnote{{\it i.e.} calcul\'ee \`a partir des
variables du pas de temps pr\'ec\'edent $n$ si n\'ecessaire.} $(n+1)$}
Ce calcul n'a lieu que si les termes de gravit\'e doivent \^etre pris en compte
($\var{IGRARI()} =1$).
\begin{itemize}
\item [$\star$] Appel de \fort{grdcel}  pour calculer le gradient de $\rho^n$
dans les trois directions de l'espace. Les conditions aux limites sur $\rho^n$
sont des conditions de Dirichlet puisque la valeur de $\rho^n$ aux faces de bord
$ik$ (variable \var{IFAC}) est connue et vaut $\rho_{\,b_{\,ik}}$. Pour \'ecrire les conditions aux limites
sous la forme habituelle, $$\rho_{\,b_{\,ik}} = \var{COEFA} + \var{COEFB}
\,\rho^n_{\,I'}$$ on pose alors $\var{COEFA}=
\var{PROPCE(IFAC,IPPROB(IROM(IPHAS)))}$ et $\var{COEFB} = \var{VISCB} = 0$.\\
$\var{PROPCE(1,IPPROB(IROM(IPHAS)))}$ (resp.$\var{VISCB}$) est utilis\'e en lieu
et place de l'habituel \var{COEFA} ($\var{COEFB}$), lors de l'appel \`a \fort{grdcel}.\\
On a donc :\\
$\displaystyle \var{GRAROX}= \frac{\partial \rho^n}{\partial x}\ $,$\displaystyle \ \var{GRAROY}= \frac{\partial
\rho^n}{\partial y}$ et $
\displaystyle \ \var{GRAROZ}= \frac{\partial \rho^n}{\partial z}\ $.

\end{itemize}

Le gradient de $\rho^n$ servira \`a calculer les termes de production par effets de gravit\'e si ces derniers sont pris en compte.

\etape{Boucle \var{ISOU} de $1$ \`a $6$ sur les tensions de Reynolds}
Pour $\var{ISOU} = 1,2,3,4,5,6$, on r\'esout respectivement et dans
l'ordre  les
\'equations de $R_{11}$, $R_{22}$, $R_{33}$, $R_{12}$, $R_{13}$ et $R_{23}$ par
l'appel au sous-programme \fort{resrij}.\\
La r\'esolution se fait par incr\'ement $\delta {R}_{ij}^{\,n+1,k+1}$ , en utilisant la m\^eme m\'ethode que
celle d\'ecrite dans le sous-programme \fort{codits}. On adopte ici les m\^emes notations.
\var{SMBR} est le second membre du syst\`eme \`a inverser, syst\`eme portant sur
les incr\'ements de la variable. \var{ROVSDT} repr\'esente la diagonale de la
matrice, hors convection/diffusion.\\
On va r\'esoudre l'\'equation (\ref{Base_Turrij_Eq_Temp_Rij}) sous forme incr\'ementale en
utilisant \fort{codits}, soit :
\begin{equation}\label{Base_Turrij_Eq_Temp_deltaRij}
\begin{array}{ll}
&\displaystyle \underbrace{\left(\frac {\rho^n_L}{\Delta t^n}
+ \rho^n_L \,C_1\,\frac{\varepsilon^n_L}{k^n_L}(1-\frac{\delta_{ij}}{3})
 - m^n_{\,lm} + \Gamma_L\,+ max(-\alpha^n_{R_{ij}},0)\right)\,|\Omega_l|}
_{\text {$\var{ROVSDT}$ contribuant
\`a la diagonale de la matrice simplifi\'ee de \fort{matrix}}}\,(\delta{R}_{ij}^{\,n+1,p+1})_{\,L}\\\\
&  \underbrace{+\sum\limits_{m\in Vois(l)}\displaystyle \left[
 m^n_{\,lm} \delta R_{ij,\,f_{\,lm}}^{\,n+1,p+1}
- (\mu^n_{\,lm} + \gamma^n_{\,lm})\
\frac{({\delta R}_{ij}^{\,n+1,p+1})_{M}-({\delta R}_{ij}^{\,n+1,p+1})_{L})}{\overline{L'M'}}\,
S_{\,lm} \right]}_{\text { convection upwind pur et diffusion non reconstruite
relatives \`a la matrice simplifi\'ee de \fort{matrix}\footnotemark}} \\
% voir le texte de la footmark plus bas
&= - \displaystyle\frac {\rho^n_L}{\Delta t^n}\,\left(\,(R^{\,n+1,p}_{ij})_L - (R^{\,n}_{ij})_L\,\right)\\
&-\,\underbrace{\displaystyle\int_{\Omega_l} \left(
\dive\,[\,(\rho\,\vect{u})^n\,R^{\,n+1,p}_{ij} - (\mu^n\,+ \gamma^n\,)\,
\grad{R^{\,n+1,p}_{ij}}\,]\right)\,d\Omega}_{\text {convection et diffusion
trait\'ees par \fort{bilsc2}}}\\
&+\displaystyle \int_{\Omega_l} \left[\,\mathcal{P}^{\,n+1,p}_{ij} + \mathcal{G}^{\,n+1,p}_{ij}
- \displaystyle\rho^n \,C_1\,\frac{\varepsilon^n}{k^n}\left[R^{\,n+1,p}_{ij}-
\frac{2}{3}\,k^n\,\delta_{ij}\right] + \phi^{\,n+1,p}_{ij,2} +
\phi^{\,n+1,p}_{ij,w}\,\right]\, d\Omega \\
& + \displaystyle\int_{\Omega_l} \left[- \frac{2}{3} \rho^n \varepsilon^n \delta_{ij}
 + \Gamma\,(\,R^{\,in}_{ij} - R^{\,n+1,p}_{ij}\,) +
\alpha^n_{R_{ij}}\,R^{\,n+1,p}_{ij}+ \beta^n_{R_{ij}}\right]\, d\Omega\\
&+ \sum\limits_{m\in
Vois(l)}\displaystyle \left[\ \tens{E}^n\,\grad{R}^{\,n+1,p}_{ij} \right]_{\,lm}\,.\,\vect{n}_{\,lm}S_{\,lm}\\
&+ \sum\limits_{m\in Vois(l)}\displaystyle \left[\
\tens{D}^n\,\grad{R}^{\,n+1,p}_{ij} \right]_{\,lm}\,.\,\vect{n}_{\,lm}S_{\,lm}\\
&- \sum\limits_{m\in Vois(l)} \gamma^n_{\,lm} \left( \grad{R}^{\,n+1,p}_{ij}\,.\,\vect{n}_{\,lm} \right)  S_{\,lm}\\
&+ \sum\limits_{m\in Vois(l)}  m^n_{\,lm}\,(R^{\,n+1,p}_{ij})_L\\
\end{array}
\end{equation}
% si on ne fait pas comme ca, il n'apparait pas
\footnotetext[\thefootnote]{Si $\var{IRIJNU} = 1$, on remplace  $\mu^n_{\,lm}$ par $(\mu +
\mu_t)^n_{\,lm}$ dans l'expression de la diffusion non reconstruite
associ\'ee \`a la matrice simplifi\'ee de \fort{matrix} ($\mu_t$ d\'esigne la
viscosit\'e turbulente calcul\'ee comme en $k-\varepsilon$).}

o\`u on rappelle :\\
pour $n$ donn\'e entier positif, on d\'efinit la suite
 $({R}_{ij}^{\,n+1,p})_{p \in \grandN}$
 par :
\begin{equation}\notag
\left\{\begin{array}{l}
{R}_{ij}^{\,n+1,0} = {R}_{ij}^{\,n}\\
{R}_{ij}^{\,n+1,p+1} = {R}_{ij}^{\,n+1,p} + \delta{R}_{ij}^{\,n+1,p+1} \\
\end{array}\right.
\end{equation}
$(\delta{R}_{ij}^{\,n+1,p+1})_{\,L}$ d\'esigne la valeur sur l'\'el\'ement
$\Omega_l$ du $\text{$(\,p+1\,)$-i\`eme}$ incr\'ement de ${R}_{ij}^{\,n+1}$,
$ m^n_{\,lm}$ le flux de masse \`a l'instant $n$ \`a travers la face $lm$,
$\delta R_{ij,\,f_{\,lm}}^{\,n+1,p+1}$ vaut $({\delta
R}_{ij}^{\,n+1,p+1})_{L}$  si $ m^n_{\,lm} \geqslant 0$, $({\delta
R}_{ij}^{\,n+1,p+1})_{M}$ sinon,
$\mathcal{P}^{\,n+1,p}_{ij}$, $\phi^{\,n+1,p}_{ij,2}$, $\phi^{\,n+1,p}_{ij,w}$ les valeurs
des quantit\'es associ\'ees correspondant \`a l'incr\'ement
$(\delta{R}_{ij}^{\,n+1,p})$.\\



Tous ces termes sont calcul\'es comme suit :
\begin{itemize}
\item Terme de gauche de l'\'equation (\ref{Base_Turrij_Eq_Temp_deltaRij})\\
Dans \fort{resrij} est calcul\'ee la variable \var{ROVSDT}. Les autres
termes sont compl\'et\'es par \fort{codits}, lors de la construction de la matrice simplifi\'ee , {\it via} un
appel au sous-programme \fort{matrix}. La quantit\'e
 $(\mu^n_{\,lm} + \gamma^n_{\,lm})$ \`a la face $lm$ est calcul\'ee lors de l'appel \`a
\fort{visort}.\\
\item Second membre de l'\'equation (\ref{Base_Turrij_Eq_Temp_deltaRij})\\
Le premier terme non d\'etaill\'e est calcul\'e par le sous-programme
\fort{bilsc2}, qui applique le sch\'ema convectif choisi par l'utilisateur, qui
reconstruit ou non selon le souhait de l'utilisateur les gradients intervenants
dans la convection-diffusion.\\
Les termes sans accolade sont, eux, compl\`etement explicites et ajout\'es au fur et
\`a mesure dans \var{SMBR} pour former
l'expression $f^{\,exp}_s$ de \fort{codits}.
\end{itemize}
On d\'ecrit ci-dessous les \'etapes de \fort{resrij} :
\begin{itemize}

\item DELTIJ = 1, pour $\var{ISOU} \leqslant 3$ et DELTIJ = 0  Si $\var{ISOU} >
3$. Cette valeur repr\'esente le symbole de Kroeneker $\delta_{ij}$.

\item Initialisation \`a z\'ero de \var{SMBR} (tableau contenant le second
membre) et \var{ROVSDT} (tableau contenant la diagonale de la matrice sauf celle
relative \`a la contribution de la
diagonale des op\'erateurs de convection et de diffusion lin\'earis\'es
\footnote{qui correspondent aux sch\'emas convectif upwind pur et diffusif sans
reconstruction.}), tous deux de dimension $\var{NCEL}$.

\item Lecture et prise en compte des termes sources utilisateur pour la variable $R_{ij}$

Appel \`a \fort{ustsri} pour charger les termes sources utilisateurs. Ils sont
stock\'es comme suit. Pour la cellule $\Omega_l$ de centre $L$, repr\'esent\'ee par $\var{IEL}$, on a :\\
\begin{equation}\notag
\left\{\begin{array}{lll}
&\var{ROVSDT(IEL)}&= |\Omega_l| \ \alpha_{R_{ij}}\\
&\var{SMBR(IEL)}&=|\Omega_l| \ \beta_{R_{ij}}\\
\end{array}\right.
\end{equation}
On affecte alors les valeurs ad\'equates au second membre \var{SMBR} et \`a la
diagonale \var{ROVSDT} comme suit :
\begin{equation}\notag
\left\{\begin{array}{lll}
&\var{SMBR(IEL)} &= \var{SMBR(IEL)} +\ |\Omega_l| \ \alpha_{R_{ij}} \ (R^n_{ij})_L \\
&\var{ROVSDT(IEL)}&= \text{max }(-\ |\Omega_l| \ \alpha_{R_{ij}},0)\\
\end{array}\right.
\end{equation}
La valeur de $ \var{ROVSDT}$ est ainsi calcul\'ee pour des raisons de stabilit\'e
num\'erique. En effet, on ne rajoute sur la diagonale que les valeurs positives,
ce qui correspond physiquement \`a impliciter les termes de rappel uniquement.
\item{Calcul du terme source de masse  si $\Gamma_L > 0$}

Appel de \fort{catsma} et incr\'ementation si n\'ecessaire de \var{SMBR} et
\var{ROVSDT} {\it via} :\\
\begin{equation}\notag
\left\{\begin{array}{lll}
\displaystyle \var{SMBR(IEL)} = \var{SMBR(IEL)} + |\Omega_l| \ \Gamma_L \
\left[(R^{\,in}_{ij})_L - (R^{\,n}_{ij})_L \right] \\
\displaystyle \var{ROVSDT(IEL)}=\var{ROVSDT(IEL)} + |\Omega_l| \ \Gamma_L
\end{array}\right.
\end{equation}
\item Calcul du terme d'accumulation de masse et du terme instationnaire

On stocke $\displaystyle \var{W1}= \int_{\Omega_l}\dive\,(\rho\,\vect{u})\,d\Omega$
calcul\'e par \fort{divmas} \`a l'aide des flux de masse aux faces internes
$ m^n_{\,lm}=\sum\limits_{m\in Vois(l)}{(\rho \vect{u})_{\,lm}^n} \text{.}\,
\vect{S}_{\,lm} $ (tableau \var{FLUMAS}) et des flux de masse aux bords  $ m^n_{\,b_{lk}} = \sum\limits_{k\in{\gamma_b(l)}}{(\rho \vect{u})_{\,{b}_{lk}}^n} \text{.}\,
\vect{S}_{\,{b}_{lk}} $ (tableau \var{FLUMAB}).
On incr\'emente ensuite \var{SMBR} et \var{ROVSDT}.
\begin{equation}\notag
\left\{\begin{array}{lll}
&\var{SMBR(IEL)} &= \var{SMBR(IEL)} + \var{ICONV}\  (R^n_{ij})_L\,(\displaystyle
\int_{\Omega_l}\dive\,(\rho\,\vect{u})\ d\Omega) \\
&\var{ROVSDT(IEL)}& = \var{ROVSDT(IEL)} +  \var{ISTAT}\,\displaystyle
\frac{\rho^n_L \ |\Omega_l|}{\Delta t^n} -  \var{ICONV}\ (\displaystyle
\int_{\Omega_l}\dive\,(\rho\,\vect{u})\ d\Omega) \\
\end{array}\right.
\end{equation}
\item Calcul des termes sources de production, des termes $\displaystyle
\phi_{\,ij,1}+\phi_{\,ij,2}$ et de la dissipation~$\displaystyle-\frac{2}{3} \varepsilon\,\delta_{\,ij}$ :

On effectue une boucle d'indice \var{IEL} sur les cellules $\Omega_l$ de centre $L$ :
\begin{itemize}
\item [$\Rightarrow$] $\displaystyle \var{TRPROD}= \frac{1}{2} (\mathcal{P}^n_{ii})_L = \frac{1}{2} \left[ \var{PRODUC(1,IEL)} +  \var{PRODUC(2,IEL)} +  \var{PRODUC(3,IEL)} \right] $
\item [$\Rightarrow$] $\displaystyle \var{TRRIJ }= \frac{1}{2} (R^n_{ii})_L $
\item [$\Rightarrow$] $\displaystyle \var{SMBR(IEL)} =\ \var{SMBR(IEL)}\ +$\\
$\ \displaystyle\rho^n_L |\Omega_l| \left[ \displaystyle
\frac{2}{3}\,\delta_{\,ij} \left( \ \displaystyle \frac{ C_2}{2}\,(\mathcal{P}^n_{ii})_L\ +
(C_1-1)\ \varepsilon^n_L\, \right)\right.$\\
$ + \left.\ (1-C_2) \ \var{PRODUC(ISOU,IEL)} -
\displaystyle C_1\ \frac{2\,\varepsilon^n_L}{(R^n_{ii})_L}\ (R^n_{ij})_L \right]$
\item [$\Rightarrow$] $\displaystyle \var{ROVSDT(IEL)} = \var{ROVSDT(IEL)} +
\rho^n_L \ |\Omega_l| \ (- \displaystyle \frac{1}{3} \ \,\delta_{\,ij} + 1) \ C_1
\ \frac{2\ \varepsilon^n_L}{(R^n_{ii})_L}$
\end{itemize}
\item Appel de \fort{rijech} pour le calcul des termes d'\'echo de paroi
 $\phi^n_{ij,w}$ si $\var{IRIJEC()}=1$ et ajout dans \var{SMBR}.\\
$\var{SMBR} = \var{SMBR} + \phi^n_{ij,w}$\\
Suivant son mode de calcul (\var{ICDPAR}), la distance � la paroi est directement accessible
par \var{RA(IDIPAR+IEL-1)} (\var{|ICDPAR|} = 1) ou bien
est calcul\'ee \`a partir de $\var{IA(IIFAPA(IPHAS)+IEL - 1)}$,
qui donne pour l'\'el\'ement $\var{IEL}$ le num\'ero de la face de bord
paroi la plus  proche (\var{|ICDPAR|} = 2). Ces tableaux ont \'et\'e renseign\'e une fois pour toutes au
d\'ebut de calcul.

\item  Appel de \fort{rijthe} pour le calcul des termes de gravit\'e $\mathcal{G}^n_{ij}$ et ajout dans \var{SMBR}.

Ce calcul n'a lieu que si $\var{IGRARI()} = 1$.
$ \var{SMBR} = \var{SMBR} + \mathcal{G}^n_{ij}$
\item Calcul de la partie explicite du terme de diffusion
 $\dive{\,\left[\tens{A}\,\grad{R}^{\,n}_{ij}\right]}$, plus pr\'ecis\'ement
des contributions du terme extradiagonal pris aux faces purement internes
(remplissage du tableau \var{VISCF}), puis aux faces de bord (remplissage du
tableau \var{VISCB}).
\begin{itemize}
\item [$\star$] Appel de \fort{grdcel} pour le calcul du gradient de
$R^{\,n}_{ij}$ dans chaque direction. Ces gradients sont respectivement
stock\'es dans les tableaux de travail \var{W1}, \var{W2} et \var{W3}.

\item [$\star$] boucle d'indice \var{IEL} sur les cellules $\Omega_l$ de centre
$L$ pour le
calcul de $\tens{E}^n\,\grad{R}^{\,n}_{ij}$ aux cellules dans un premier temps :\\
\begin{itemize}
\item [$\Rightarrow$] $\displaystyle \var{TRRIJ}= \frac{1}{2} (R^{\,n}_{ii})_L $
\item [$\Rightarrow$] $\displaystyle \var{CSTRIJ} = \rho^n_L\ C_S \ \displaystyle\frac{(R^n_{ii})_L}{2\,\varepsilon^n_L}$
\item [$\Rightarrow$] $\displaystyle \var{W4(IEL)} = \rho^n_L\ C_S\
\displaystyle\frac{(R^n_{ii})_L}{2\,\varepsilon^n_L} \left[\,(R^{\,n}_{12})_L \ \var{W2(IEL)} +
(R^{\,n}_{13})_L \ \var{W3(IEL)}\,\right]$
\item [$\Rightarrow$] $\displaystyle \var{W5(IEL)} = \rho^n_L\ C_S\
\displaystyle\frac{(R^n_{ii})_L}{2\,\varepsilon^n_L} \left[\,(R^{\,n}_{12})_L \ \var{W1(IEL)} +
(R^{\,n}_{23})_L \ \var{W3(IEL)}\,\right]$
\item [$\Rightarrow$] $\displaystyle \var{W6(IEL)} = \rho^n_L\ C_S\
\displaystyle\frac{(R^n_{ii})_L}{2\,\varepsilon^n_L} \left[\,(R^{\,n}_{13})_L \ \var{W1(IEL)} + (R^{\,n}_{23})_L \ \var{W2(IEL)}\,\right]$
\end{itemize}



\item [$\star$] Appel de \fort{vectds}\footnote{Le r\'esultat est stock\'e dans
\var{VISCF} et \var{VISCB}. Dans \fort{vectds}, les valeurs aux faces internes
sont interpol\'ees lin\'eairement sans reconstruction et \var{VISCB} est mis \`a
z\'ero.} pour assembler $\displaystyle\left[ \tens{E}^n\,\grad{R}^{\,n}_{ij}
\right]\,.\,\vect{n}_{\,lm}S_{\,lm}$ aux faces $lm$.
\item [$\star$] Appel de \fort{divmas} pour calculer la divergence du flux d\'efini par \var{VISCF} et \var{VISCB}.
Le r\'esultat est stock\'e dans \var{W4}.\\
Ajout au second membre \var{SMBR}.\\
\var{SMBR} = \var{SMBR} + \var{W4}
\end{itemize}

A l'issue de cette \'etape, seule la partie extradiagonale de la diffusion prise
enti\`erement explicite~:
 $$\sum\limits_{m\in
Vois(l)}\left[\ \tens{E}^n\,\grad{R}^{\,n}_{ij} \right]_{\,lm}\,.\,\vect{n}_{\,lm}S_{\,lm}$$ a \'et\'e calcul\'ee.\\

\item Calcul de la partie diagonale du terme de diffusion\footnote{Seule la
composante normale  du  gradient de $R_{ij}$ aux faces sera implicite.} :\\
Comme on l'a d\'eja vu, une partie de cette contribution sera trait\'ee en
implicite (celle relative \`a la composante normale du gradient) et donc
ajout\'ee au second membre par \fort{bilsc2} ; l'autre
partie sera explicite et prise en compte dans $f_s^{\,exp}$.
\begin{itemize}
\item [$\star$] On effectue une boucle d'indice \var{IEL} sur les cellules
$\Omega_l$ de centre $L$ :
\begin{itemize}
\item [$\Rightarrow$] $\displaystyle \var{TRRIJ }= \frac{1}{2} (R^{\,n}_{ii})_L $
\item [$\Rightarrow$] $\displaystyle \var{CSTRIJ} = \rho^n_L \ C_S \ \frac{(R^{\,n}_{ii})_L}{2\,\varepsilon^n_L}$
\item [$\Rightarrow$] $\displaystyle \var{W4(IEL)} = \rho^n_L \ C_S \
\frac{(R^{\,n}_{ii})_L}{2\,\varepsilon^n_L} \ (R^{\,n}_{11})_L$
\item [$\Rightarrow$] $\displaystyle \var{W5(IEL)} = \rho^n_L \ C_S \ \frac{(R^{\,n}_{ii})_L}{2\,\varepsilon^n_L}\ (R^n_{22})_L$
\item [$\Rightarrow$] $\displaystyle \var{W6(IEL)} = \rho^n_L \ C_S \ \frac{(R^{\,n}_{ii})_L}{2\,\varepsilon^n_L} \ (R^n_{33})_L$
\end{itemize}

%\item Traitement du parall\'elisme et de la p\'eriodicit\'e.

\item [$\star$] On effectue une boucle d'indice \var{IFAC} sur les faces
purement internes $lm$ pour remplir le tableau \var{VISCF} :
\begin{itemize}
\item [$\Rightarrow$] Identification des cellules $\Omega_l$ et $\Omega_m$ de
centre respectif $L$ (variable \var{II}) et $M$ (variable \var{JJ}), se trouvant de chaque c\^ot\'e de la face
$lm$\footnote{La normale \`a la face est orient\'ee de L vers M.}.
\item [$\Rightarrow$] Calcul du carr\'e de la surface de la face. La valeur est
stock\'ee dans le tableau \var{SURFN2}.
\item [$\Rightarrow$] Interpolation du gradient de $R^{\,n}_{ij}$ \`a la face
$lm$ (gradient facette $\left[\grad{R}^{\,n}_{ij}\right]_{\,lm}$) :
\begin{equation}\notag
\left\{\begin{array}{ll}
\var{GRDPX} &= \displaystyle \frac{1}{2} \left(\var{W1(II)} + \var{W1(JJ)}
\right) \\
&\\
\var{GRDPY} &= \displaystyle \frac{1}{2} \left(\var{W2(II)} + \var{W2(JJ)} \right) \\
&\\
\var{GRDPZ} &= \displaystyle \frac{1}{2} \left(\var{W3(II)} + \var{W3(JJ)} \right)
\end{array}\right.
\end{equation}
\item [$\Rightarrow$] Calcul du gradient de $R^{\,n}_{ij}$ normal \`a la face
$lm$, $\left[\grad{R}^{\,n}_{ij}\right]_{\,lm}.\vect{n}_{\,lm}\,S_{\,lm}$.\\

$\displaystyle \var{GRDSN} =  \var{GRDPX} \ \var{SURFAC(1,IFAC)} + \var{GRDPY} \ \var{SURFAC(2,IFAC)} +  \var{GRDPZ} \ \var{SURFAC(3,IFAC)}$
$\var{SURFAC}$ est le vecteur surface de la face \var{IFAC}.


\item [$\Rightarrow$] calcul de
 $\left[\grad{R^{\,n}_{ij}} - (\grad
R^{\,n}_{ij}\,.\,\vect{n}_{\,lm})\vect{n}_{\,lm}\right]$, les vecteurs \'etant
calcul\'es \`a la face $lm$ :
\begin{equation}\notag
\left\{\begin{array}{lll}
&\displaystyle \var{GRDPX} &= \var{GRDPX} - \displaystyle\frac{\var{GRDSN}}{\var{SURFN2}} \ \var{SURFAC(1,IFAC)}\\
&&\\
&\displaystyle \var{GRDPY} &= \var{GRDPY} - \displaystyle\frac{\var{GRDSN}}{\var{SURFN2}} \ \var{SURFAC(2,IFAC)} \\
&&\\
&\displaystyle \var{GRDPZ} &= \var{GRDPZ} - \displaystyle\frac{\var{GRDSN}}{\var{SURFN2}} \ \var{SURFAC(3,IFAC)}
\end{array}\right.
\end{equation}
\item [$\Rightarrow$] finalisation du calcul de l'expression totalement
explicite
 $$\left[ \tens{D}^n\,\left( \grad{R^{\,n}_{ij}} - (\grad R^{\,n}_{ij}\,.\,\vect{n}_{\,lm})\,\vect{n}_{\,lm}\right) \right]\,.\,\vect{n}_{\,lm}$$
\begin{equation}\notag
\begin{array} {ll}
\displaystyle \var{VISCF} = &
 \displaystyle\frac{1}{2} (\ \var{W4(II)} +\ \var{W4(JJ)}) \ \var{GRDPX} \
\var{SURFAC(1,IFAC)})\ + \\
&\\
&  \displaystyle\frac{1}{2} (\ \var{W5(II)} +\ \var{W5(JJ)}) \ \var{GRDPY} \
\var{SURFAC(2,IFAC)})\ + \\
&\\
&  \displaystyle\frac{1}{2} (\ \var{W6(II)} +\ \var{W6(JJ)}) \ \var{GRDPZ} \ \var{SURFAC(3,IFAC)})
\end{array}
\end{equation}
\end{itemize}

\item [$\star$] Mise \`a z\'ero du tableau \var{VISCB}.

\item [$\star$] Appel de \fort{divmas} pour calculer la divergence de~:
 $$\tens{D}^{\,n}\,\left( \grad{R^{\,n}_{ij}} - (\grad R^{\,n}_{ij}\,.\,\vect{n}_{\,lm})\vect{n}_{\,lm}\right)$$ d\'efini aux faces dans \var{VISCF} et \var{VISCB}.

Le r\'esultat est stock\'e dans le tableau \var{W1}.\\
Ajout au second membre \var{SMBR}.\\
$\var{SMBR} = \var{SMBR} + \var{W1}$
\end{itemize}
\item Calcul de la viscosit\'e orthotrope $\gamma^n_{\,lm}$ \`a la face $lm$ de la variable principale
$R^{\,n}_{ij}$\\
Ce calcul permet au sous-programme \fort{codits} de compl\'eter le second membre
\var{SMBR} par :
\begin{equation}
\begin{array} {ll}
& \sum\limits_{m\in Vois(l)}
\mu^n_{\,lm}\,\left(\grad{R}^{\,n}_{ij}\,.\,\vect{n}_{\,lm}\right)S_{\,lm}
 + \sum\limits_{m\in Vois(l)} \left(\grad{R}^{\,n}_{ij}
\,.\,\vect{n}_{\,lm}\right)\left[\tens{D}^{\,n}\,\vect{n}_{\,lm}\right]_{\,lm}\,.\,\vect{n}_{\,lm}\
S_{\,lm}\\
& = \sum\limits_{m\in Vois(l)}(\,\mu^n_{\,lm}\, + \,\gamma^n_{\,lm}\,)\,\left(\grad{R}^{\,n}_{ij}\,.\,\vect{n}_{\,lm}\right)S_{\,lm}
\end{array}
\end{equation}
sans pr\'eciser la nature de la face $lm$, {\it via} l'appel \`a \fort{bilsc2}  et de disposer de la quantit\'e
$(\mu^n_{\,lm}\, + \gamma^n_{\,lm})$ pour construire sa
matrice simplifi\'ee.\\
\begin{itemize}
\item [$\star$] On effectue une boucle d'indice \var{IEL} sur les cellules
$\Omega_l$ :
\begin{itemize}
\item [$\Rightarrow$] $\displaystyle \var{TRRIJ }= \frac{1}{2} (R^{\,n}_{ii})_L $
\item [$\Rightarrow$] $\displaystyle \var{RCSTE} = \rho^n_L \ C_S \ \frac{ (R^{\,n}_{ii})_L}{2\,\varepsilon^n_L} $
\item [$\Rightarrow$] $\displaystyle \var{W1(IEL)} = \mu^n +\rho^n_L \ C_S \ \frac{
(R^{\,n}_{ii})_L}{2\,\varepsilon^n_L}\ (R^n_{11})_L$
\item [$\Rightarrow$] $\displaystyle \var{W2(IEL)} = \mu^n + \rho^n_L \ C_S \ \frac{ (R^{\,n}_{ii})_L}{2\,\varepsilon^n_L}\ (R^n_{22})_L$
\item [$\Rightarrow$] $\displaystyle \var{W3(IEL)} = \mu^n + \rho^n_L \ C_S \ \frac{ (R^{\,n}_{ii})_L}{2\,\varepsilon^n_L}\ (R^n_{33})_L$
\end{itemize}

\item [$\star$] Appel de \fort{visort} pour calculer la viscosit\'e orthotrope
\footnote{Comme dans le sous-programme \fort{viscfa}, on multiplie la viscosit\'e par
$\displaystyle \frac{S_{\,lm}}{\overline{L'M'}}$, o\`u $S_{\,lm}$ et
$\overline{L'M'}$ repr\'esentent respectivement la surface de la face $lm$ et la
mesure alg\'ebrique du segment reliant les projections des centres des cellules
voisines sur la normale \`a la face. On garde dans ce sous-programme  la possibilit\'e d'interpoler la viscosit\'e aux cellules lin\'eairement ou d'utiliser une moyenne harmonique. La viscosit\'e au bord est celle de la cellule de bord correspondante.}
$ \gamma^n_{\,lm} = (\tens{D}^{\,n}\,\vect{n}_{\,lm}).\vect{n}_{\,lm}$ aux faces $lm$

Le r\'esultat est stock\'e dans les tableaux \var{VISCF} et \var{VISCB}.
\end{itemize}

\item appel de \fort{codits} pour la r\'esolution de l'\'equation de
convection/diffusion/termes sources de la variable $R_{ij}$. Le terme source est
\var{SMBR}, la viscosit\'e \var{VISCF} aux faces purement internes (
resp. \var{VISCB} aux faces de bord) et \var{FLUMAS} le flux de masse interne
 ( resp. \var{FLUMAB} flux de masse au bord). Le r\'esultat est la variable $R_{ij}$ au temps
$n+1$, donc $R^{\,n+1}_{ij}$. Elle est stock\'ee dans le tableau \var{RTP} des
variables mises \`a jour.

\end{itemize}

\etape{Appel de \fort{reseps} pour la r\'esolution de la variable $\varepsilon$}

On d\'ecrit ci-dessous le sous-programme \fort{reseps}, les commentaires du sous-programme \fort{resrij} ne seront pas r\'ep\'et\'es puisque les deux sous-programmes ne diff\`erent que par leurs termes sources.

\begin{itemize}
\item Initialisation \`a z\'ero de \var{SMBR} et \var{ROVSDT}.

\item{Lecture et prise en compte des termes sources utilisateur pour la variable $\varepsilon$ :}

Appel de \fort{ustsri} pour charger les termes sources utilisateurs. Ils sont
stock\'es dans les tableaux suivants :\\
pour la cellule $\Omega_l$ repr\'esent\'ee par $\var{IEL}$ de centre $L$, on a :
\begin{equation}\notag
\left\{\begin{array}{lll}
&\var{ROVSDT(IEL)}&= |\Omega_l| \ \alpha_{\varepsilon}\\
&\var{SMBR(IEL)}&=|\Omega_l| \ \beta_{\varepsilon}\\
\end{array}\right.
\end{equation}
On affecte alors les valeurs ad\'equates au second membre \var{SMBR} et \`a la
diagonale \var{ROVSDT} comme suit :
\begin{equation}\notag
\left\{\begin{array}{lll}
&\var{SMBR(IEL)} &= \var{SMBR(IEL)} +\ |\Omega_l| \ \alpha_{\,\varepsilon} \
\varepsilon^n_L \\
&\var{ROVSDT(IEL)}&= \text{max }(-\ |\Omega_l| \ \alpha_{\,\varepsilon},0)\\
\end{array}\right.
\end{equation}

\item{Calcul du terme source de masse si $\Gamma_L > 0$ :
\begin{equation}\notag
\left\{\begin{array}{lll}
&\displaystyle \var{SMBR(IEL)} = \var{SMBR(IEL)} + |\Omega_l| \ \Gamma_L \
(\varepsilon^{\,in}_L -\varepsilon^n_L) \\
&\displaystyle \var{ROVSDT(IEL)}= \var{ROVSDT(IEL)} + |\Omega_l| \ \Gamma_L
\end{array}\right.
\end{equation}
\item Calcul du terme d'accumulation de masse et du terme instationnaire \\
On stocke $\displaystyle \var{W1}= \int_{\Omega_l}\dive\,(\rho\,\vect{u})\,d\Omega$
calcul\'e par \fort{divmas} \`a l'aide des flux de masse internes et aux bords.\\
On incr\'emente ensuite \var{SMBR} et \var{ROVSDT}.
\begin{equation}\notag
\left\{\begin{array}{lll}
&\var{SMBR(IEL)} &= \var{SMBR(IEL)} + \var{ICONV}\ \varepsilon^n_L\,(\displaystyle
\int_{\Omega_l}\dive\,(\rho\,\vect{u})\ d\Omega) \\
&\var{ROVSDT(IEL)}& = \var{ROVSDT(IEL)} +  \var{ISTAT}\,\displaystyle
\frac{\rho^n_L \ |\Omega_l|}{\Delta t^n} -  \var{ICONV}\ (\displaystyle
\int_{\Omega_l}\dive\,(\rho\,\vect{u})\ d\Omega) \\
\end{array}\right.
\end{equation}

\item Traitement du terme de production
 $\displaystyle \rho\,C_{\varepsilon_1}\,\frac{\varepsilon}{k}\,\mathcal{P}$
 et du terme de dissipation $-\,\displaystyle \rho\,C_{\varepsilon_2}\,\frac{\varepsilon}{k}\,\varepsilon$ \\
pour cela, on effectue une boucle d'indice \var{IEL} sur les cellules $\Omega_l$
de centre $L$ :
\begin{itemize}
\item [$\Rightarrow$] $\displaystyle \var{TRPROD}= \frac{1}{2} (\mathcal{P}^n_{ii})_L = \frac{1}{2} \left[ \var{PRODUC(1,IEL)} +  \var{PRODUC(2,IEL)} +  \var{PRODUC(3,IEL)} \right] $
\item [$\Rightarrow$] $\displaystyle \var{TRRIJ }= \frac{1}{2} (R^n_{ii})_L $
\item [$\Rightarrow$] $\displaystyle \var{SMBR(IEL)} = \var{SMBR(IEL)} + \rho^n_L
|\Omega_l| \left[ -C_{\varepsilon_2} \ \frac{2\,(\varepsilon^n_L)^2}{(R^n_{ii})_L} + C_{\varepsilon_1} \ \frac{\varepsilon^n_L}{(R^n_{ii})_L}\ (\mathcal{P}^n_{ii})_L \right] $
\item [$\Rightarrow$] $\displaystyle \var{ROVSDT(IEL)} = \var{ROVSDT(IEL)} + C_{\varepsilon_2} \ \rho^n_L \ |\Omega_l| \ \frac{2\,\varepsilon^n_L}{(R^n_{ii})_L}$
\end{itemize}

\item Appel de \fort{rijthe} pour le calcul des termes de gravit\'e $\mathcal{G}^n_{\varepsilon}$ et ajout dans \var{SMBR}.

$ \var{SMBR} = \var{SMBR} + \mathcal{G}^n_{\varepsilon}$\\
Ce calcul n'a lieu que si $\var{IGRARI()} = 1$.

\item Calcul de la diffusion de $\varepsilon$ \\
 Le terme $\dive \left[\mu\, \grad(\varepsilon) + \tens{A'}\,\grad(\varepsilon)
\right]$ est calcul\'e exactement de la m\^eme mani\`ere que pour les tensions
de Reynolds $R_{ij}$ en rempla\c cant $\tens{A}$ par $\tens{A'}$.

\item Appel de \fort{codits} pour la r\'esolution de l'\'equation de
convection/diffusion/termes sources de la variable principale $\varepsilon$. Le
r\'esultat $\varepsilon^{\,n+1}$ est stock\'e dans le tableau \var{RTP} des
variables mises \`a jour.
}
\end{itemize}

\etape{clippings finaux}
On passe enfin dans le sous-programme  \fort{clprij} pour faire un clipping \'eventuel
des variables $R^{\,n+1}_{ij}$ et $\varepsilon^{\,n+1}$. Le sous-programme
\fort{clprij} est appel\'e\footnote{L'option
$\var{ICLIP} = 1$ consiste en un clipping minimal des variables $R_{ii}$ et
$\varepsilon$ en prenant la valeur absolue de ces variables puisqu'elles ne
peuvent \^etre que positives.} avec $\var{ICLIP} = 2$ . Cette option
\footnote{Quand la valeur des grandeurs $R_{ii}$ ou $\varepsilon$ est
n\'egative, on la remplace par le minimum entre sa valeur absolue et (1,1)
fois la valeur obtenue au pas de temps pr\'ec\'edent.} contient l'option $\var{ICLIP} = 1$  et permet de v\'erifier l'in\'egalit\'e de Cauchy-Schwarz sur les grandeurs extra-diagonales du tenseur $\tens{R}$ (pour $i \neq j$, $|R_{ij}|^2 \le R_{ii} R_{jj}$).


%%%%%%%%%%%%%%%%%%%%%%%%%%%%%%%%%%
%%%%%%%%%%%%%%%%%%%%%%%%%%%%%%%%%%
\section{Points \`a traiter}
%%%%%%%%%%%%%%%%%%%%%%%%%%%%%%%%%%
%%%%%%%%%%%%%%%%%%%%%%%%%%%%%%%%%%
Sauf mention explicite, $\phi$ repr\'esentera une tension de Reynolds ou la dissipation turbulente ($\phi = R_{ij} \ \text{ou} \ \varepsilon$).

\begin{itemize}
\item {La vitesse utilis\'ee pour le calcul de la production est explicite. Est-ce qu'une implicitation peut am\'eliorer la pr\'ecision temporelle de $\phi$ \footnote{Cette remarque peut \^etre g\'en\'eralis\'ee. En effet, peut-on envisager d'actualiser les variables d\'ej\`a r\'esolues (sans r\'eactualiser les variables turbulentes apr\`es leur r\'esolution)? Ceci obligerait \`a modifier les sous-programmes tels que \fort{condli} qui sont appel\'es au d\'ebut de la boucle en temps.} ?}
\item {Dans quelle mesure le terme d'\'echo de paroi est-il valide ? En effet, ce terme est remis en question par certains auteurs.}
\item {On peut envisager la r\'esolution d'un syst\`eme hyperbolique pour les
tensions de Reynolds afin d'introduire un couplage avec le champ de vitesse.}
\item {Le flux au bord \var{VISCB} est annul\'e dans le sous-programme
\fort{vectds}. Peut-on envisager de mettre au bord la valeur de la variable
concern\'ee \`a la cellule de bord correspondant? De m\^eme, il faudrait se
pencher sur les hypoth\`eses sous-jacentes \`a l'annulation des contributions
aux bords de \var{VISCB} lors du calcul de : $$\left[ \tens{D}^n\,\left( \grad{R^{\,n}_{ij}} - (\grad R^{\,n}_{ij}\,.\,\vect{n}_{\,lm})\,\vect{n}_{\,lm}\right) \right]\,.\,\vect{n}_{\,lm}.$$}
\item {Un probl\`eme de pond\'eration appara\^\i t plus g\'en\'eralement. Si on prend la partie explicite de $\tens{D}\,\grad(\phi)$, la pond\'eration aux faces internes utilise le coefficient $\displaystyle\frac{1}{2}$ avec pond\'eration s\'epar\'ee de $\tens{D}$ et $\grad(\phi)$, alors que pour $\tens{E}\,\grad(\phi)$, on calcule d'abord ce terme aux cellules pour ensuite l'interpoler lin\'eairement aux faces \footnote{Cette interpolation se fait dans \fort{vectds} avec des coefficients de pond\'eration aux faces.}. Ceci donne donc deux types d'interpolations pour des termes de m\^eme nature.}
\item {On laisse la possibilit\'e dans \fort{visort} d'utiliser une moyenne
harmonique aux faces. Est-ce que ceci est valable puisque les interpolations
utilis\'ees lors du calcul de la partie explicite de $\tens{A}\,\grad{\phi}$
sont des moyennes arithm\'etiques ?}
\item {Les techniques adopt\'ees lors du clipping sont \`a revoir.}
\item {On utilise dans le cadre du mod\`ele $\displaystyle R_{ij}-\varepsilon $ une semi-implicitation de termes comme $\displaystyle \phi_{ij,1}$ ou $\displaystyle -\rho\,C_{\varepsilon_2}\,\frac{\varepsilon}{k}\,\varepsilon$. On peut envisager le m\^eme type d'implicitation dans \fort{turbke} m\^eme en pr\'esence du couplage $\displaystyle k-\varepsilon$.}
\item L'adoption d'une r\'esolution d\'ecoupl\'ee fait perdre l'invariance par rotation.
\item La formulation et l'implantation des conditions aux limites de paroi
devront \^etre v\'erifi\'ees. En effet, il semblerait que, dans certains cas, des ph\'enom\`enes
``oscillatoires'' apparaissent, sans qu'il soit ais\'e d'en d\'eterminer la cause.
\item L'implicitation partielle (du fait de la r\'esolution d\'ecoupl\'ee) des
conditions aux limites conduit souvent \`a des calculs instables. Il
conviendrait d'en conna\^\i tre la raison. L'implicitation partielle avait
\'et\'e mise en \oe uvre afin de tenter d'utiliser un pas de temps plus grand
dans le cas de jets axisym\'etriques en particulier.

\end{itemize}

%                      Code_Saturne version 1.3
%                      ------------------------
%
%     This file is part of the Code_Saturne Kernel, element of the
%     Code_Saturne CFD tool.
%
%     Copyright (C) 1998-2007 EDF S.A., France
%
%     contact: saturne-support@edf.fr
%
%     The Code_Saturne Kernel is free software; you can redistribute it
%     and/or modify it under the terms of the GNU General Public License
%     as published by the Free Software Foundation; either version 2 of
%     the License, or (at your option) any later version.
%
%     The Code_Saturne Kernel is distributed in the hope that it will be
%     useful, but WITHOUT ANY WARRANTY; without even the implied warranty
%     of MERCHANTABILITY or FITNESS FOR A PARTICULAR PURPOSE.  See the
%     GNU General Public License for more details.
%
%     You should have received a copy of the GNU General Public License
%     along with the Code_Saturne Kernel; if not, write to the
%     Free Software Foundation, Inc.,
%     51 Franklin St, Fifth Floor,
%     Boston, MA  02110-1301  USA
%
%-----------------------------------------------------------------------
%
\programme{vortex}
%
\vspace{1cm}
%%%%%%%%%%%%%%%%%%%%%%%%%%%%%%%%%%
%%%%%%%%%%%%%%%%%%%%%%%%%%%%%%%%%%
\section{Fonction}
%%%%%%%%%%%%%%%%%%%%%%%%%%%%%%%%%%
%%%%%%%%%%%%%%%%%%%%%%%%%%%%%%%%%%
Ce sous-programme est d�di� � la g�n�ration des conditions d'entr�e
turbulente utilis�es en LES.


La m�thode des vortex est bas�e sur une approche de tourbillons
ponctuels. L'id�e de la m�thode consiste � injecter des tourbillons 2D dans le
plan d'entr�e du calcul, puis � calculer le champ de vitesse induit par ces
tourbillons au centre des faces d'entr�e.

%                      Code_Saturne version 1.3
%                      ------------------------
%
%     This file is part of the Code_Saturne Kernel, element of the
%     Code_Saturne CFD tool.
% 
%     Copyright (C) 1998-2007 EDF S.A., France
%
%     contact: saturne-support@edf.fr
% 
%     The Code_Saturne Kernel is free software; you can redistribute it
%     and/or modify it under the terms of the GNU General Public License
%     as published by the Free Software Foundation; either version 2 of
%     the License, or (at your option) any later version.
% 
%     The Code_Saturne Kernel is distributed in the hope that it will be
%     useful, but WITHOUT ANY WARRANTY; without even the implied warranty
%     of MERCHANTABILITY or FITNESS FOR A PARTICULAR PURPOSE.  See the
%     GNU General Public License for more details.
% 
%     You should have received a copy of the GNU General Public License
%     along with the Code_Saturne Kernel; if not, write to the
%     Free Software Foundation, Inc.,
%     51 Franklin St, Fifth Floor,
%     Boston, MA  02110-1301  USA
%
%-----------------------------------------------------------------------
%
%%%%%%%%%%%%%%%%%%%%%%%%%%%%%%%%%%
%%%%%%%%%%%%%%%%%%%%%%%%%%%%%%%%%%
\section{Discr\'etisation}
%%%%%%%%%%%%%%%%%%%%%%%%%%%%%%%%%%
%%%%%%%%%%%%%%%%%%%%%%%%%%%%%%%%%%

Le terme convectif en $\dive(\underline{u} \otimes \rho\,\underline{u})$
introduit une non lin\'earit\'e et un couplage des composantes de la vitesse
$\vect{u}$ dans l'�quation (\ref{Base_Preduv_eqqdm}). Une lin\'earisation et un d\'ecouplage
des trois composantes de la 
vitesse sont r\'ealis\'es lors de la discr\'etisation de cette \'etape de
pr\'ediction.\\
En effet, soit :
\begin{equation}
\vect{\widetilde{u}}= \vect{u}^n + \delta \vect{u} 
\end{equation}
La contribution exacte du terme convectif \`a prendre en compte dans cette
\'etape de pr\'ediction serait :\\
\begin{equation}\label{Base_Preduv_Conv_exact}
\begin{array}{ll}
\dive(\vect{\widetilde{u}} \otimes \rho\,\vect{\widetilde{u}}) =
\dive(\vect{u}^{n} \otimes \rho\,\vect{u}^{n}) + \dive(\delta \vect{u} \otimes
\rho\,\vect{u}^{n}) +  \underbrace { \dive(\vect{u}^{n} \otimes
\rho\,\delta \vect{u})}_{\text {terme couplant lin\'eaire}} +  \underbrace { \dive(\delta \vect{u} \otimes
\rho\,\delta \vect{u})}_{\text {terme couplant et non lin\'eaire}}\\
\end{array} 
\end{equation}
Les deux derniers termes de l'expression (\ref{Base_Preduv_Conv_exact}) sont {\em a priori} n�glig�s
de mani�re � obtenir un syst\`eme en vitesse qui soit d\'ecoupl\'e et donc,
�viter l'inversion d'une matrice pouvant \^etre de tr\`es grande taille. Ces
deux termes peuvent n�anmoins �tre pris en compte de mani�re plus ou moins
approch�e par extrapolation explicite du flux de masse en $n+\theta_F$ (pour le
terme couplant lin�aire provenant de la convection de $\vect{u}^{n}$ par $\delta
\vect{u}$) et utilisation d'un point-fixe par sous it�ration sur le sous
programme \fort{navsto} (pour le terme non-lin�aire, en sp�cifiant $\var{NTERUP}>1$).

L'�quation (\ref{Base_Preduv_eqqdm}) est discr�tis�e au temps $n+\theta$ � l'aide d'un
$\theta$-sch�ma, et le tenseur des pertes de charges d�compos� en une partie
diagonale $\tens{K}_{d}$ et une extradiagonale $\tens{K}_{e}$ (soit
 $\tens{K}_{pdc}=\tens{K}_{d}+\tens{K}_{e}$).\\
$\bullet$ La pression est suppos�e connue en $n-1+\theta$ (d�calage temporel
pression-vitesse) et le gradient naturellement calcul� � cet instant.\\ 
$\bullet$ Les termes sources de viscosit� secondaire, de gradient transpos\'e,
ceux provenant du mod�le de turbulence\footnote{except� $\dive (\mu_t\ (\ggrad
\underline {u}))$}, $\rho\,\tens{K}_{\,e}\ \underline{u}$, $(\rho -\rho_0)
\underline {g}$ ainsi que $\underline{T}_{s}^{\,exp}$ et
$\Gamma\,\underline{u}_{\,i}$ sont pris de mani�re explicite au temps $n$, ou
extrapol�s suivant le sch�ma en temps choisi pour les propri�t�s physique et les
termes sources.\\ 
$\bullet$ Les termes sources $\underline{u}\,\,\dive (\rho\,\underline {u})$,
$\Gamma\,\,\underline{u}$, $T_{s}^{\,imp}\,\,\underline{u}$ et
$-\rho\,\tens{K}_{\,d}\,\,\underline{u}$ sont implicit�s est calcul�s �
l'instant $n+\theta$.\\ 
$\bullet$ Le terme de diffusion $\dive (\mu_{\,tot}\,\ggrad \underline{u})$ est
implicit� : la vitesse est prise � l'instant $n+\theta$ et la viscosit�
explicit�e ou extrapol�e.\\ 
$\bullet$ Enfin, le terme de convection en $\dive(\,\underline{u} \otimes
(\rho\underline{u})\,)$ est implicit� : la composante r�solue de la vitesse est
prise en $n+\theta$, et le flux de masse, explicit�, ou extrapol� en
$n+\theta_F$. 

Par souci de clart�, on suppose, en l'absence d'indication, que les propri�tes
physiques $\Phi$ ($\rho,\,\mu_{tot},\,...$) et le flux de masse
$(\rho\underline{u})$ sont pris respectivement aux instants $n+\theta_\Phi$ et
$n+\theta_F$, o� $\theta_\Phi$ et $\theta_F$ d�pendent des sch�mas en temps
sp�cifiquement utilis�s pour ces grandeurs\footnote{cf. \fort{introd}}. 

La discr�tisation temporelle de l'�quation (\ref{Base_Preduv_eqqdm}) s'�crit alors comme suit : 

\begin{equation}\label{Base_Preduv_eq_di1}
 \begin{array}{c}
\displaystyle \rho\,\ \frac{ \underline {\widetilde{u}}^{n+1} -\underline {u}^{n} }
{\Delta t} + \dive(\,\underline{\widetilde{u}}^{n+\theta} \otimes (\rho\underline{u})\,) -\dive
(\mu_{\,tot}\,\ggrad \underline{\widetilde{u}}^{n+\theta}) =
\\
\displaystyle
 - \grad p^{n-1+\theta} + \dive (\rho\,\underline {u})\,\underline{\widetilde{u}}^{n+\theta} +(\Gamma\,\underline{u}_{\,i})^{n+\theta_S}-\Gamma^n\,\,\underline{\widetilde{u}}^{n+\theta}
\\
\begin{array}{c}
\displaystyle
- \rho\,\tens{K}_{\,d}^{n}\,\,\underline{\widetilde{u}}^{n+\theta} - (\rho\,\tens{K}_{\,e}\ \underline{u})^{n+\theta_S} + (\underline{T}_{s}^{\,exp})^{\,n+\theta_S} + T_{s}^{\,imp}\,\,\underline{\widetilde{u}}^{n+\theta}
\\
\displaystyle
+[\dive (\mu_{\,tot}\,^t\ggrad \underline {u})]^{n+\theta_S}-\frac {2} {3}[\,\grad (\mu_{\,tot}\,\dive \underline {u})]^{n+\theta_S} + (\rho -\rho_0) \underline {g}
 - (\underline{turb})^{n+\theta_S}
\end{array}
\end{array}
\end{equation}
o\`u, par souci de simplification, on a pos\'e :
\begin{equation}
\mu_{\,tot}=
\begin{cases}
\mu+\mu_t & \text{pour les mod�les � viscosit� turbulente ou en LES}, \\
\mu & \text{pour les mod�les au second ordre ou en laminaire}
\end{cases} \ 
\end{equation}
\\
et :
\begin{equation}
\underline{turb}^{n}=
\begin{cases}
\displaystyle\frac {2}{3}\grad (\rho^{n}\,k^{n}) & \text{pour les mod�les � viscosit� turbulente}, \\
\dive(\rho^{n}\,\tens{R}^n) & \text{pour les mod�les au second ordre},\\
0 & \text{en laminaire ou en LES}\\
\end{cases}
\end{equation}
Par analogie avec l'�criture du $\theta$-sch�ma pour une variable scalaire, $\,
\underline {\widetilde{u}}^{n+\theta}$ est interpol�e � partir de la vitesse
pr�dite $\underline {\widetilde{u}}^{n+1}$ de la mani\`ere suivante\footnote{si
$\theta=1/2$, ou qu'une extrapolation est utilis�e, l'ordre 2 n'est obtenu que si
le pas de temps $\Delta t$ est uniforme en temps et en espace.}~: 
\begin{equation}
\underline {\widetilde{u}}^{n+\theta}=\theta\, \underline
{\widetilde{u}}^{n+1}+(1-\theta)\, \underline {u}^{n}\\ 
\end{equation}
Avec :
\begin{equation}
\left\{
\begin{array}{ll}
\theta = 1   & \text{Pour un sch\'ema de type Euler implicite d'ordre 1.}\\
\theta = 1/2 & \text{Pour un sch\'ema de type Cranck-Nicolson d'ordre 2.}\\
\end{array}
\right.
\end{equation}

L'�quation (\ref{Base_Preduv_eq_di1}) est alors r��crite sous la forme :

\begin{equation}\label{Base_Preduv_eq_di2}
\begin{array}{c}
\displaystyle \underbrace{\left(\frac{\rho}{\Delta t} -\theta \,\dive (\rho\,\underline {u}) +\theta \,\, \Gamma^n +
\theta \,\, \rho\,\tens{K}_{\,d}^n-\theta \,T_s^{\,imp} \right)}_{\displaystyle f_s^{imp}}\, (\underline {\,\widetilde{u}}^{n+1} -\underline {u}^{n})
\\
 +\, \theta\, \dive(\underline {\widetilde{u}}^{n+1} \otimes (\rho\underline{u}))-\, \theta\,\dive (\mu_{\,tot}\,\ggrad \underline {\widetilde{u}}^{n+1}) =
\\
-\,(1-\theta)\, \dive(\underline {u}^{n} \otimes (\rho\underline{u})) +\,(1-\theta)\,\dive (\mu_{\,tot}\,\ggrad \underline {u}^{n})
\\
f_s^{exp}\left\{
\begin{array}{c}
\displaystyle 
- \grad p^{n-1+\theta} + \dive (\rho\,\underline {u})\,\underline{u}^{n} +\,(\,\Gamma^{n}\,\underline{u}_{\,i}\,)^{n+\theta_S}- \Gamma^n\,\,\underline{u}^{n}
\\
\displaystyle
-(\,\rho\,\tens{K}_{\,e}\ \underline{u}\,)^{n+\theta_S} -\rho\,\tens{K}_{\,d}^n\ \underline{u}^{n}+ (\underline{T}_{s}^{\,exp})^{\,n+\theta_S} + T_s^{\,imp}\,\,\underline {u}^{n} 
\\
\displaystyle
+[\dive (\mu_{\,tot}\,^t\ggrad \underline {u}\,)]^{n+\theta_S}-\frac {2} {3}[\,\grad (\mu_{\,tot}\,\dive \underline {u}\,)]^{n+\theta_S} + (\rho -\rho_0) \underline {g}-(\underline{turb})^{n+\theta_S}
\end{array}
\right.
\end{array}
\end{equation}

d'o� l'�quation r�solue par le sous-programme \fort{codits} :
\begin{equation}\begin{array}{c}
\displaystyle
f_s^{\,imp}(\underline {\widetilde{u}}^{n+1}-\underline {u}^{n}) + \theta\, \dive(\underline{\widetilde{u}}^{n+1} \otimes (\rho
\underline{u})) - \theta\,\dive (\,\mu_{\,tot}\,\ggrad \underline{\widetilde{u}}^{n+1}) = 
\\\\
\displaystyle
-(1-\theta)\,\dive(\underline{u}^{n} \otimes (\rho \underline{u}))+(1-\theta)\,\dive (\,\mu_{\,tot}\,\ggrad \underline{u}^{n})
+ \underline{f}_{\,s}^{\,exp}
\end{array}
\end{equation}
La m\'ethode de discr\'etisation spatiale est d\'evelopp\'ee dans le sous-programme \fort{codits}.\\



\minititre{Remarques :}
{\tiny$\blacksquare$} Dans le cas standard sans extrapolation, le terme
$-\,T_s^{\,imp}$ n'est ajout� � $f_s^{\,imp}$ que s'il est positif afin de ne
pas affaiblir la dominance de la diagonale de la matrice � inverser.\\ 
{\tiny$\blacksquare$} Si une extrapolation est utilis�e, par contre,
$\,T_s^{\,imp}$ est ajout� � $f_s^{\,imp}$ quel que soit son signe. En effet, l'id�e intuitive qui
consiste � prendre~: 
\begin{equation}
\begin{cases}
\displaystyle
(\underline{T}_{s}^{\,exp} + T_{s}^{\,imp}\,\underline {u})^{\,n+\theta_S} &
\text{si } T_{s}^{\,imp} > 0\\ 
\displaystyle
(\underline{T}_{s}^{\,exp})^{\,n+\theta_S} + T_{s}^{\,imp}\,\underline{u}^{n+\theta} &\text{sinon}\\
\end{cases}
\end{equation} 
aboutit � une incoh�rence dans le traitement si $T_s^{imp}$ change de signe
entre deux pas de temps.\\ 
{\tiny$\blacksquare$} la partie diagonale $\tens{K}_{\,d}$ du terme
de perte de charge est utilis�e dans $f_s^{\,imp}$. En fait, pour \^etre rigoureux,
il faudrait ne retenir que les contributions positives (point signal\'e dans le
sous-programme utilisateur associ\'e \fort{uskpdc}). Cette prise en compte sera \`a am\'eliorer.\\
{\tiny$\blacksquare$} Le terme $\theta\,\Gamma^{n}-\theta\,\dive
(\rho\,\underline {u})$ ne pose pas de probl�me pour la 
dominance de la diagonale de la matrice car il est exactement compens� par le
terme de convection (cf. \fort{covofi}). 


%                      Code_Saturne version 1.3
%                      ------------------------
%
%     This file is part of the Code_Saturne Kernel, element of the
%     Code_Saturne CFD tool.
%
%     Copyright (C) 1998-2007 EDF S.A., France
%
%     contact: saturne-support@edf.fr
%
%     The Code_Saturne Kernel is free software; you can redistribute it
%     and/or modify it under the terms of the GNU General Public License
%     as published by the Free Software Foundation; either version 2 of
%     the License, or (at your option) any later version.
%
%     The Code_Saturne Kernel is distributed in the hope that it will be
%     useful, but WITHOUT ANY WARRANTY; without even the implied warranty
%     of MERCHANTABILITY or FITNESS FOR A PARTICULAR PURPOSE.  See the
%     GNU General Public License for more details.
%
%     You should have received a copy of the GNU General Public License
%     along with the Code_Saturne Kernel; if not, write to the
%     Free Software Foundation, Inc.,
%     51 Franklin St, Fifth Floor,
%     Boston, MA  02110-1301  USA
%
%-----------------------------------------------------------------------
%

%%%%%%%%%%%%%%%%%%%%%%%%%%%%%%%%%%
%%%%%%%%%%%%%%%%%%%%%%%%%%%%%%%%%%
\section{Mise en \oe uvre}
%%%%%%%%%%%%%%%%%%%%%%%%%%%%%%%%%%
%%%%%%%%%%%%%%%%%%%%%%%%%%%%%%%%%%
La num\'ero de la phase trait\'ee fait partie des arguments de \fort{turrij}. On
omettra volontairement de le pr\'eciser dans ce qui suit, on indiquera par $(\ )$ la
notion de tableau s'y rattachant.

\etape{Calcul des termes de production $\tens{\mathcal{P}}$}
\begin{itemize}
\item [$\star$] Initialisation \`a z\'ero du tableau \var{PRODUC} dimensionn\'e \`a $\var{NCEL}\times 6$.
\item [$\star$] On appelle trois fois \fort{grdcel} pour calculer les gradients des composantes de la vitesse $u$, $v$ et
$w$ prises au temps $n$.

Au final, on a :\\
$\displaystyle
\begin{array} {ll}
\var{PRODUC(1,IEL)} = & \displaystyle - 2 \left[ R_{11}^{\,n} \frac{\partial u^{\,n}} {\partial x} +R_{12}^{\,n} \frac{\partial u^{\,n}} {\partial y}+R_{13}^{\,n} \frac{\partial u^{\,n}} {\partial z} \right] \text{        (production de $R_{11}^{\,n}$)}\\
\var{PRODUC(2,IEL)} = & \displaystyle - 2 \left[ R_{12}^{\,n} \frac{\partial v^{\,n}} {\partial x} +R_{22}^{\,n} \frac{\partial v^{\,n}} {\partial y}+R_{23}^{\,n} \frac{\partial v^{\,n}} {\partial z} \right] \text{        (production de $R_{22}^{\,n}$)}\\
\var{PRODUC(3,IEL)} = & \displaystyle - 2 \left[ R_{13}^{\,n} \frac{\partial w^{\,n}} {\partial x} +R_{23}^{\,n} \frac{\partial w^{\,n}} {\partial y}+R_{33}^{\,n} \frac{\partial w^{\,n}} {\partial z} \right] \text{        (production de $R_{33}^{\,n}$)}\\
\var{PRODUC(4,IEL)} = & \displaystyle - \left[ R_{12}^{\,n} \frac{\partial u^{\,n}} {\partial x} +R_{22}^{\,n} \frac{\partial u^{\,n}} {\partial y}+R_{23}^{\,n} \frac{\partial u^{\,n}} {\partial z} \right] \\
& \displaystyle - \left[ R_{11}^{\,n} \frac{\partial v^{\,n}} {\partial x} +R_{12}^{\,n} \frac{\partial v^{\,n}} {\partial y}+R_{13}^{\,n} \frac{\partial v^{\,n}} {\partial z} \right] \text{        (production de $R_{12}^{\,n}$)} \\
\var{PRODUC(5,IEL)} = & \displaystyle - \left[ R_{13}^{\,n} \frac{\partial u^{\,n}} {\partial x} +R_{23}^{\,n} \frac{\partial u^{\,n}} {\partial y}+R_{33}^{\,n} \frac{\partial u^{\,n}} {\partial z} \right] \\
& \displaystyle - \left[ R_{11}^{\,n} \frac{\partial w^{\,n}} {\partial x} +R_{12}^{\,n} \frac{\partial w^{\,n}} {\partial y}+R_{13}^{\,n} \frac{\partial w^{\,n}} {\partial z} \right] \text{        (production de $R_{13}^{\,n}$)} \\
\var{PRODUC(6,IEL)} = & \displaystyle - \left[ R_{13}^{\,n} \frac{\partial v^{\,n}} {\partial x} +R_{23}^{\,n} \frac{\partial v^{\,n}} {\partial y}+R_{33}^{\,n} \frac{\partial v^{\,n}} {\partial z} \right] \\
& \displaystyle - \left[ R_{12}^{\,n} \frac{\partial w^{\,n}} {\partial x} +R_{22}^{\,n} \frac{\partial w^{\,n}} {\partial y}+R_{23}^{\,n} \frac{\partial w^{\,n}} {\partial z} \right]  \text{        (production de $R_{23}^{\,n}$)}
\end{array}
$
\end{itemize}

\etape{Calcul du gradient de la masse volumique $\rho^n$ prise au d\'ebut du pas
de temps courant\footnote{{\it i.e.} calcul\'ee \`a partir des
variables du pas de temps pr\'ec\'edent $n$ si n\'ecessaire.} $(n+1)$}
Ce calcul n'a lieu que si les termes de gravit\'e doivent \^etre pris en compte
($\var{IGRARI()} =1$).
\begin{itemize}
\item [$\star$] Appel de \fort{grdcel}  pour calculer le gradient de $\rho^n$
dans les trois directions de l'espace. Les conditions aux limites sur $\rho^n$
sont des conditions de Dirichlet puisque la valeur de $\rho^n$ aux faces de bord
$ik$ (variable \var{IFAC}) est connue et vaut $\rho_{\,b_{\,ik}}$. Pour \'ecrire les conditions aux limites
sous la forme habituelle, $$\rho_{\,b_{\,ik}} = \var{COEFA} + \var{COEFB}
\,\rho^n_{\,I'}$$ on pose alors $\var{COEFA}=
\var{PROPCE(IFAC,IPPROB(IROM(IPHAS)))}$ et $\var{COEFB} = \var{VISCB} = 0$.\\
$\var{PROPCE(1,IPPROB(IROM(IPHAS)))}$ (resp.$\var{VISCB}$) est utilis\'e en lieu
et place de l'habituel \var{COEFA} ($\var{COEFB}$), lors de l'appel \`a \fort{grdcel}.\\
On a donc :\\
$\displaystyle \var{GRAROX}= \frac{\partial \rho^n}{\partial x}\ $,$\displaystyle \ \var{GRAROY}= \frac{\partial
\rho^n}{\partial y}$ et $
\displaystyle \ \var{GRAROZ}= \frac{\partial \rho^n}{\partial z}\ $.

\end{itemize}

Le gradient de $\rho^n$ servira \`a calculer les termes de production par effets de gravit\'e si ces derniers sont pris en compte.

\etape{Boucle \var{ISOU} de $1$ \`a $6$ sur les tensions de Reynolds}
Pour $\var{ISOU} = 1,2,3,4,5,6$, on r\'esout respectivement et dans
l'ordre  les
\'equations de $R_{11}$, $R_{22}$, $R_{33}$, $R_{12}$, $R_{13}$ et $R_{23}$ par
l'appel au sous-programme \fort{resrij}.\\
La r\'esolution se fait par incr\'ement $\delta {R}_{ij}^{\,n+1,k+1}$ , en utilisant la m\^eme m\'ethode que
celle d\'ecrite dans le sous-programme \fort{codits}. On adopte ici les m\^emes notations.
\var{SMBR} est le second membre du syst\`eme \`a inverser, syst\`eme portant sur
les incr\'ements de la variable. \var{ROVSDT} repr\'esente la diagonale de la
matrice, hors convection/diffusion.\\
On va r\'esoudre l'\'equation (\ref{Base_Turrij_Eq_Temp_Rij}) sous forme incr\'ementale en
utilisant \fort{codits}, soit :
\begin{equation}\label{Base_Turrij_Eq_Temp_deltaRij}
\begin{array}{ll}
&\displaystyle \underbrace{\left(\frac {\rho^n_L}{\Delta t^n}
+ \rho^n_L \,C_1\,\frac{\varepsilon^n_L}{k^n_L}(1-\frac{\delta_{ij}}{3})
 - m^n_{\,lm} + \Gamma_L\,+ max(-\alpha^n_{R_{ij}},0)\right)\,|\Omega_l|}
_{\text {$\var{ROVSDT}$ contribuant
\`a la diagonale de la matrice simplifi\'ee de \fort{matrix}}}\,(\delta{R}_{ij}^{\,n+1,p+1})_{\,L}\\\\
&  \underbrace{+\sum\limits_{m\in Vois(l)}\displaystyle \left[
 m^n_{\,lm} \delta R_{ij,\,f_{\,lm}}^{\,n+1,p+1}
- (\mu^n_{\,lm} + \gamma^n_{\,lm})\
\frac{({\delta R}_{ij}^{\,n+1,p+1})_{M}-({\delta R}_{ij}^{\,n+1,p+1})_{L})}{\overline{L'M'}}\,
S_{\,lm} \right]}_{\text { convection upwind pur et diffusion non reconstruite
relatives \`a la matrice simplifi\'ee de \fort{matrix}\footnotemark}} \\
% voir le texte de la footmark plus bas
&= - \displaystyle\frac {\rho^n_L}{\Delta t^n}\,\left(\,(R^{\,n+1,p}_{ij})_L - (R^{\,n}_{ij})_L\,\right)\\
&-\,\underbrace{\displaystyle\int_{\Omega_l} \left(
\dive\,[\,(\rho\,\vect{u})^n\,R^{\,n+1,p}_{ij} - (\mu^n\,+ \gamma^n\,)\,
\grad{R^{\,n+1,p}_{ij}}\,]\right)\,d\Omega}_{\text {convection et diffusion
trait\'ees par \fort{bilsc2}}}\\
&+\displaystyle \int_{\Omega_l} \left[\,\mathcal{P}^{\,n+1,p}_{ij} + \mathcal{G}^{\,n+1,p}_{ij}
- \displaystyle\rho^n \,C_1\,\frac{\varepsilon^n}{k^n}\left[R^{\,n+1,p}_{ij}-
\frac{2}{3}\,k^n\,\delta_{ij}\right] + \phi^{\,n+1,p}_{ij,2} +
\phi^{\,n+1,p}_{ij,w}\,\right]\, d\Omega \\
& + \displaystyle\int_{\Omega_l} \left[- \frac{2}{3} \rho^n \varepsilon^n \delta_{ij}
 + \Gamma\,(\,R^{\,in}_{ij} - R^{\,n+1,p}_{ij}\,) +
\alpha^n_{R_{ij}}\,R^{\,n+1,p}_{ij}+ \beta^n_{R_{ij}}\right]\, d\Omega\\
&+ \sum\limits_{m\in
Vois(l)}\displaystyle \left[\ \tens{E}^n\,\grad{R}^{\,n+1,p}_{ij} \right]_{\,lm}\,.\,\vect{n}_{\,lm}S_{\,lm}\\
&+ \sum\limits_{m\in Vois(l)}\displaystyle \left[\
\tens{D}^n\,\grad{R}^{\,n+1,p}_{ij} \right]_{\,lm}\,.\,\vect{n}_{\,lm}S_{\,lm}\\
&- \sum\limits_{m\in Vois(l)} \gamma^n_{\,lm} \left( \grad{R}^{\,n+1,p}_{ij}\,.\,\vect{n}_{\,lm} \right)  S_{\,lm}\\
&+ \sum\limits_{m\in Vois(l)}  m^n_{\,lm}\,(R^{\,n+1,p}_{ij})_L\\
\end{array}
\end{equation}
% si on ne fait pas comme ca, il n'apparait pas
\footnotetext[\thefootnote]{Si $\var{IRIJNU} = 1$, on remplace  $\mu^n_{\,lm}$ par $(\mu +
\mu_t)^n_{\,lm}$ dans l'expression de la diffusion non reconstruite
associ\'ee \`a la matrice simplifi\'ee de \fort{matrix} ($\mu_t$ d\'esigne la
viscosit\'e turbulente calcul\'ee comme en $k-\varepsilon$).}

o\`u on rappelle :\\
pour $n$ donn\'e entier positif, on d\'efinit la suite
 $({R}_{ij}^{\,n+1,p})_{p \in \grandN}$
 par :
\begin{equation}\notag
\left\{\begin{array}{l}
{R}_{ij}^{\,n+1,0} = {R}_{ij}^{\,n}\\
{R}_{ij}^{\,n+1,p+1} = {R}_{ij}^{\,n+1,p} + \delta{R}_{ij}^{\,n+1,p+1} \\
\end{array}\right.
\end{equation}
$(\delta{R}_{ij}^{\,n+1,p+1})_{\,L}$ d\'esigne la valeur sur l'\'el\'ement
$\Omega_l$ du $\text{$(\,p+1\,)$-i\`eme}$ incr\'ement de ${R}_{ij}^{\,n+1}$,
$ m^n_{\,lm}$ le flux de masse \`a l'instant $n$ \`a travers la face $lm$,
$\delta R_{ij,\,f_{\,lm}}^{\,n+1,p+1}$ vaut $({\delta
R}_{ij}^{\,n+1,p+1})_{L}$  si $ m^n_{\,lm} \geqslant 0$, $({\delta
R}_{ij}^{\,n+1,p+1})_{M}$ sinon,
$\mathcal{P}^{\,n+1,p}_{ij}$, $\phi^{\,n+1,p}_{ij,2}$, $\phi^{\,n+1,p}_{ij,w}$ les valeurs
des quantit\'es associ\'ees correspondant \`a l'incr\'ement
$(\delta{R}_{ij}^{\,n+1,p})$.\\



Tous ces termes sont calcul\'es comme suit :
\begin{itemize}
\item Terme de gauche de l'\'equation (\ref{Base_Turrij_Eq_Temp_deltaRij})\\
Dans \fort{resrij} est calcul\'ee la variable \var{ROVSDT}. Les autres
termes sont compl\'et\'es par \fort{codits}, lors de la construction de la matrice simplifi\'ee , {\it via} un
appel au sous-programme \fort{matrix}. La quantit\'e
 $(\mu^n_{\,lm} + \gamma^n_{\,lm})$ \`a la face $lm$ est calcul\'ee lors de l'appel \`a
\fort{visort}.\\
\item Second membre de l'\'equation (\ref{Base_Turrij_Eq_Temp_deltaRij})\\
Le premier terme non d\'etaill\'e est calcul\'e par le sous-programme
\fort{bilsc2}, qui applique le sch\'ema convectif choisi par l'utilisateur, qui
reconstruit ou non selon le souhait de l'utilisateur les gradients intervenants
dans la convection-diffusion.\\
Les termes sans accolade sont, eux, compl\`etement explicites et ajout\'es au fur et
\`a mesure dans \var{SMBR} pour former
l'expression $f^{\,exp}_s$ de \fort{codits}.
\end{itemize}
On d\'ecrit ci-dessous les \'etapes de \fort{resrij} :
\begin{itemize}

\item DELTIJ = 1, pour $\var{ISOU} \leqslant 3$ et DELTIJ = 0  Si $\var{ISOU} >
3$. Cette valeur repr\'esente le symbole de Kroeneker $\delta_{ij}$.

\item Initialisation \`a z\'ero de \var{SMBR} (tableau contenant le second
membre) et \var{ROVSDT} (tableau contenant la diagonale de la matrice sauf celle
relative \`a la contribution de la
diagonale des op\'erateurs de convection et de diffusion lin\'earis\'es
\footnote{qui correspondent aux sch\'emas convectif upwind pur et diffusif sans
reconstruction.}), tous deux de dimension $\var{NCEL}$.

\item Lecture et prise en compte des termes sources utilisateur pour la variable $R_{ij}$

Appel \`a \fort{ustsri} pour charger les termes sources utilisateurs. Ils sont
stock\'es comme suit. Pour la cellule $\Omega_l$ de centre $L$, repr\'esent\'ee par $\var{IEL}$, on a :\\
\begin{equation}\notag
\left\{\begin{array}{lll}
&\var{ROVSDT(IEL)}&= |\Omega_l| \ \alpha_{R_{ij}}\\
&\var{SMBR(IEL)}&=|\Omega_l| \ \beta_{R_{ij}}\\
\end{array}\right.
\end{equation}
On affecte alors les valeurs ad\'equates au second membre \var{SMBR} et \`a la
diagonale \var{ROVSDT} comme suit :
\begin{equation}\notag
\left\{\begin{array}{lll}
&\var{SMBR(IEL)} &= \var{SMBR(IEL)} +\ |\Omega_l| \ \alpha_{R_{ij}} \ (R^n_{ij})_L \\
&\var{ROVSDT(IEL)}&= \text{max }(-\ |\Omega_l| \ \alpha_{R_{ij}},0)\\
\end{array}\right.
\end{equation}
La valeur de $ \var{ROVSDT}$ est ainsi calcul\'ee pour des raisons de stabilit\'e
num\'erique. En effet, on ne rajoute sur la diagonale que les valeurs positives,
ce qui correspond physiquement \`a impliciter les termes de rappel uniquement.
\item{Calcul du terme source de masse  si $\Gamma_L > 0$}

Appel de \fort{catsma} et incr\'ementation si n\'ecessaire de \var{SMBR} et
\var{ROVSDT} {\it via} :\\
\begin{equation}\notag
\left\{\begin{array}{lll}
\displaystyle \var{SMBR(IEL)} = \var{SMBR(IEL)} + |\Omega_l| \ \Gamma_L \
\left[(R^{\,in}_{ij})_L - (R^{\,n}_{ij})_L \right] \\
\displaystyle \var{ROVSDT(IEL)}=\var{ROVSDT(IEL)} + |\Omega_l| \ \Gamma_L
\end{array}\right.
\end{equation}
\item Calcul du terme d'accumulation de masse et du terme instationnaire

On stocke $\displaystyle \var{W1}= \int_{\Omega_l}\dive\,(\rho\,\vect{u})\,d\Omega$
calcul\'e par \fort{divmas} \`a l'aide des flux de masse aux faces internes
$ m^n_{\,lm}=\sum\limits_{m\in Vois(l)}{(\rho \vect{u})_{\,lm}^n} \text{.}\,
\vect{S}_{\,lm} $ (tableau \var{FLUMAS}) et des flux de masse aux bords  $ m^n_{\,b_{lk}} = \sum\limits_{k\in{\gamma_b(l)}}{(\rho \vect{u})_{\,{b}_{lk}}^n} \text{.}\,
\vect{S}_{\,{b}_{lk}} $ (tableau \var{FLUMAB}).
On incr\'emente ensuite \var{SMBR} et \var{ROVSDT}.
\begin{equation}\notag
\left\{\begin{array}{lll}
&\var{SMBR(IEL)} &= \var{SMBR(IEL)} + \var{ICONV}\  (R^n_{ij})_L\,(\displaystyle
\int_{\Omega_l}\dive\,(\rho\,\vect{u})\ d\Omega) \\
&\var{ROVSDT(IEL)}& = \var{ROVSDT(IEL)} +  \var{ISTAT}\,\displaystyle
\frac{\rho^n_L \ |\Omega_l|}{\Delta t^n} -  \var{ICONV}\ (\displaystyle
\int_{\Omega_l}\dive\,(\rho\,\vect{u})\ d\Omega) \\
\end{array}\right.
\end{equation}
\item Calcul des termes sources de production, des termes $\displaystyle
\phi_{\,ij,1}+\phi_{\,ij,2}$ et de la dissipation~$\displaystyle-\frac{2}{3} \varepsilon\,\delta_{\,ij}$ :

On effectue une boucle d'indice \var{IEL} sur les cellules $\Omega_l$ de centre $L$ :
\begin{itemize}
\item [$\Rightarrow$] $\displaystyle \var{TRPROD}= \frac{1}{2} (\mathcal{P}^n_{ii})_L = \frac{1}{2} \left[ \var{PRODUC(1,IEL)} +  \var{PRODUC(2,IEL)} +  \var{PRODUC(3,IEL)} \right] $
\item [$\Rightarrow$] $\displaystyle \var{TRRIJ }= \frac{1}{2} (R^n_{ii})_L $
\item [$\Rightarrow$] $\displaystyle \var{SMBR(IEL)} =\ \var{SMBR(IEL)}\ +$\\
$\ \displaystyle\rho^n_L |\Omega_l| \left[ \displaystyle
\frac{2}{3}\,\delta_{\,ij} \left( \ \displaystyle \frac{ C_2}{2}\,(\mathcal{P}^n_{ii})_L\ +
(C_1-1)\ \varepsilon^n_L\, \right)\right.$\\
$ + \left.\ (1-C_2) \ \var{PRODUC(ISOU,IEL)} -
\displaystyle C_1\ \frac{2\,\varepsilon^n_L}{(R^n_{ii})_L}\ (R^n_{ij})_L \right]$
\item [$\Rightarrow$] $\displaystyle \var{ROVSDT(IEL)} = \var{ROVSDT(IEL)} +
\rho^n_L \ |\Omega_l| \ (- \displaystyle \frac{1}{3} \ \,\delta_{\,ij} + 1) \ C_1
\ \frac{2\ \varepsilon^n_L}{(R^n_{ii})_L}$
\end{itemize}
\item Appel de \fort{rijech} pour le calcul des termes d'\'echo de paroi
 $\phi^n_{ij,w}$ si $\var{IRIJEC()}=1$ et ajout dans \var{SMBR}.\\
$\var{SMBR} = \var{SMBR} + \phi^n_{ij,w}$\\
Suivant son mode de calcul (\var{ICDPAR}), la distance � la paroi est directement accessible
par \var{RA(IDIPAR+IEL-1)} (\var{|ICDPAR|} = 1) ou bien
est calcul\'ee \`a partir de $\var{IA(IIFAPA(IPHAS)+IEL - 1)}$,
qui donne pour l'\'el\'ement $\var{IEL}$ le num\'ero de la face de bord
paroi la plus  proche (\var{|ICDPAR|} = 2). Ces tableaux ont \'et\'e renseign\'e une fois pour toutes au
d\'ebut de calcul.

\item  Appel de \fort{rijthe} pour le calcul des termes de gravit\'e $\mathcal{G}^n_{ij}$ et ajout dans \var{SMBR}.

Ce calcul n'a lieu que si $\var{IGRARI()} = 1$.
$ \var{SMBR} = \var{SMBR} + \mathcal{G}^n_{ij}$
\item Calcul de la partie explicite du terme de diffusion
 $\dive{\,\left[\tens{A}\,\grad{R}^{\,n}_{ij}\right]}$, plus pr\'ecis\'ement
des contributions du terme extradiagonal pris aux faces purement internes
(remplissage du tableau \var{VISCF}), puis aux faces de bord (remplissage du
tableau \var{VISCB}).
\begin{itemize}
\item [$\star$] Appel de \fort{grdcel} pour le calcul du gradient de
$R^{\,n}_{ij}$ dans chaque direction. Ces gradients sont respectivement
stock\'es dans les tableaux de travail \var{W1}, \var{W2} et \var{W3}.

\item [$\star$] boucle d'indice \var{IEL} sur les cellules $\Omega_l$ de centre
$L$ pour le
calcul de $\tens{E}^n\,\grad{R}^{\,n}_{ij}$ aux cellules dans un premier temps :\\
\begin{itemize}
\item [$\Rightarrow$] $\displaystyle \var{TRRIJ}= \frac{1}{2} (R^{\,n}_{ii})_L $
\item [$\Rightarrow$] $\displaystyle \var{CSTRIJ} = \rho^n_L\ C_S \ \displaystyle\frac{(R^n_{ii})_L}{2\,\varepsilon^n_L}$
\item [$\Rightarrow$] $\displaystyle \var{W4(IEL)} = \rho^n_L\ C_S\
\displaystyle\frac{(R^n_{ii})_L}{2\,\varepsilon^n_L} \left[\,(R^{\,n}_{12})_L \ \var{W2(IEL)} +
(R^{\,n}_{13})_L \ \var{W3(IEL)}\,\right]$
\item [$\Rightarrow$] $\displaystyle \var{W5(IEL)} = \rho^n_L\ C_S\
\displaystyle\frac{(R^n_{ii})_L}{2\,\varepsilon^n_L} \left[\,(R^{\,n}_{12})_L \ \var{W1(IEL)} +
(R^{\,n}_{23})_L \ \var{W3(IEL)}\,\right]$
\item [$\Rightarrow$] $\displaystyle \var{W6(IEL)} = \rho^n_L\ C_S\
\displaystyle\frac{(R^n_{ii})_L}{2\,\varepsilon^n_L} \left[\,(R^{\,n}_{13})_L \ \var{W1(IEL)} + (R^{\,n}_{23})_L \ \var{W2(IEL)}\,\right]$
\end{itemize}



\item [$\star$] Appel de \fort{vectds}\footnote{Le r\'esultat est stock\'e dans
\var{VISCF} et \var{VISCB}. Dans \fort{vectds}, les valeurs aux faces internes
sont interpol\'ees lin\'eairement sans reconstruction et \var{VISCB} est mis \`a
z\'ero.} pour assembler $\displaystyle\left[ \tens{E}^n\,\grad{R}^{\,n}_{ij}
\right]\,.\,\vect{n}_{\,lm}S_{\,lm}$ aux faces $lm$.
\item [$\star$] Appel de \fort{divmas} pour calculer la divergence du flux d\'efini par \var{VISCF} et \var{VISCB}.
Le r\'esultat est stock\'e dans \var{W4}.\\
Ajout au second membre \var{SMBR}.\\
\var{SMBR} = \var{SMBR} + \var{W4}
\end{itemize}

A l'issue de cette \'etape, seule la partie extradiagonale de la diffusion prise
enti\`erement explicite~:
 $$\sum\limits_{m\in
Vois(l)}\left[\ \tens{E}^n\,\grad{R}^{\,n}_{ij} \right]_{\,lm}\,.\,\vect{n}_{\,lm}S_{\,lm}$$ a \'et\'e calcul\'ee.\\

\item Calcul de la partie diagonale du terme de diffusion\footnote{Seule la
composante normale  du  gradient de $R_{ij}$ aux faces sera implicite.} :\\
Comme on l'a d\'eja vu, une partie de cette contribution sera trait\'ee en
implicite (celle relative \`a la composante normale du gradient) et donc
ajout\'ee au second membre par \fort{bilsc2} ; l'autre
partie sera explicite et prise en compte dans $f_s^{\,exp}$.
\begin{itemize}
\item [$\star$] On effectue une boucle d'indice \var{IEL} sur les cellules
$\Omega_l$ de centre $L$ :
\begin{itemize}
\item [$\Rightarrow$] $\displaystyle \var{TRRIJ }= \frac{1}{2} (R^{\,n}_{ii})_L $
\item [$\Rightarrow$] $\displaystyle \var{CSTRIJ} = \rho^n_L \ C_S \ \frac{(R^{\,n}_{ii})_L}{2\,\varepsilon^n_L}$
\item [$\Rightarrow$] $\displaystyle \var{W4(IEL)} = \rho^n_L \ C_S \
\frac{(R^{\,n}_{ii})_L}{2\,\varepsilon^n_L} \ (R^{\,n}_{11})_L$
\item [$\Rightarrow$] $\displaystyle \var{W5(IEL)} = \rho^n_L \ C_S \ \frac{(R^{\,n}_{ii})_L}{2\,\varepsilon^n_L}\ (R^n_{22})_L$
\item [$\Rightarrow$] $\displaystyle \var{W6(IEL)} = \rho^n_L \ C_S \ \frac{(R^{\,n}_{ii})_L}{2\,\varepsilon^n_L} \ (R^n_{33})_L$
\end{itemize}

%\item Traitement du parall\'elisme et de la p\'eriodicit\'e.

\item [$\star$] On effectue une boucle d'indice \var{IFAC} sur les faces
purement internes $lm$ pour remplir le tableau \var{VISCF} :
\begin{itemize}
\item [$\Rightarrow$] Identification des cellules $\Omega_l$ et $\Omega_m$ de
centre respectif $L$ (variable \var{II}) et $M$ (variable \var{JJ}), se trouvant de chaque c\^ot\'e de la face
$lm$\footnote{La normale \`a la face est orient\'ee de L vers M.}.
\item [$\Rightarrow$] Calcul du carr\'e de la surface de la face. La valeur est
stock\'ee dans le tableau \var{SURFN2}.
\item [$\Rightarrow$] Interpolation du gradient de $R^{\,n}_{ij}$ \`a la face
$lm$ (gradient facette $\left[\grad{R}^{\,n}_{ij}\right]_{\,lm}$) :
\begin{equation}\notag
\left\{\begin{array}{ll}
\var{GRDPX} &= \displaystyle \frac{1}{2} \left(\var{W1(II)} + \var{W1(JJ)}
\right) \\
&\\
\var{GRDPY} &= \displaystyle \frac{1}{2} \left(\var{W2(II)} + \var{W2(JJ)} \right) \\
&\\
\var{GRDPZ} &= \displaystyle \frac{1}{2} \left(\var{W3(II)} + \var{W3(JJ)} \right)
\end{array}\right.
\end{equation}
\item [$\Rightarrow$] Calcul du gradient de $R^{\,n}_{ij}$ normal \`a la face
$lm$, $\left[\grad{R}^{\,n}_{ij}\right]_{\,lm}.\vect{n}_{\,lm}\,S_{\,lm}$.\\

$\displaystyle \var{GRDSN} =  \var{GRDPX} \ \var{SURFAC(1,IFAC)} + \var{GRDPY} \ \var{SURFAC(2,IFAC)} +  \var{GRDPZ} \ \var{SURFAC(3,IFAC)}$
$\var{SURFAC}$ est le vecteur surface de la face \var{IFAC}.


\item [$\Rightarrow$] calcul de
 $\left[\grad{R^{\,n}_{ij}} - (\grad
R^{\,n}_{ij}\,.\,\vect{n}_{\,lm})\vect{n}_{\,lm}\right]$, les vecteurs \'etant
calcul\'es \`a la face $lm$ :
\begin{equation}\notag
\left\{\begin{array}{lll}
&\displaystyle \var{GRDPX} &= \var{GRDPX} - \displaystyle\frac{\var{GRDSN}}{\var{SURFN2}} \ \var{SURFAC(1,IFAC)}\\
&&\\
&\displaystyle \var{GRDPY} &= \var{GRDPY} - \displaystyle\frac{\var{GRDSN}}{\var{SURFN2}} \ \var{SURFAC(2,IFAC)} \\
&&\\
&\displaystyle \var{GRDPZ} &= \var{GRDPZ} - \displaystyle\frac{\var{GRDSN}}{\var{SURFN2}} \ \var{SURFAC(3,IFAC)}
\end{array}\right.
\end{equation}
\item [$\Rightarrow$] finalisation du calcul de l'expression totalement
explicite
 $$\left[ \tens{D}^n\,\left( \grad{R^{\,n}_{ij}} - (\grad R^{\,n}_{ij}\,.\,\vect{n}_{\,lm})\,\vect{n}_{\,lm}\right) \right]\,.\,\vect{n}_{\,lm}$$
\begin{equation}\notag
\begin{array} {ll}
\displaystyle \var{VISCF} = &
 \displaystyle\frac{1}{2} (\ \var{W4(II)} +\ \var{W4(JJ)}) \ \var{GRDPX} \
\var{SURFAC(1,IFAC)})\ + \\
&\\
&  \displaystyle\frac{1}{2} (\ \var{W5(II)} +\ \var{W5(JJ)}) \ \var{GRDPY} \
\var{SURFAC(2,IFAC)})\ + \\
&\\
&  \displaystyle\frac{1}{2} (\ \var{W6(II)} +\ \var{W6(JJ)}) \ \var{GRDPZ} \ \var{SURFAC(3,IFAC)})
\end{array}
\end{equation}
\end{itemize}

\item [$\star$] Mise \`a z\'ero du tableau \var{VISCB}.

\item [$\star$] Appel de \fort{divmas} pour calculer la divergence de~:
 $$\tens{D}^{\,n}\,\left( \grad{R^{\,n}_{ij}} - (\grad R^{\,n}_{ij}\,.\,\vect{n}_{\,lm})\vect{n}_{\,lm}\right)$$ d\'efini aux faces dans \var{VISCF} et \var{VISCB}.

Le r\'esultat est stock\'e dans le tableau \var{W1}.\\
Ajout au second membre \var{SMBR}.\\
$\var{SMBR} = \var{SMBR} + \var{W1}$
\end{itemize}
\item Calcul de la viscosit\'e orthotrope $\gamma^n_{\,lm}$ \`a la face $lm$ de la variable principale
$R^{\,n}_{ij}$\\
Ce calcul permet au sous-programme \fort{codits} de compl\'eter le second membre
\var{SMBR} par :
\begin{equation}
\begin{array} {ll}
& \sum\limits_{m\in Vois(l)}
\mu^n_{\,lm}\,\left(\grad{R}^{\,n}_{ij}\,.\,\vect{n}_{\,lm}\right)S_{\,lm}
 + \sum\limits_{m\in Vois(l)} \left(\grad{R}^{\,n}_{ij}
\,.\,\vect{n}_{\,lm}\right)\left[\tens{D}^{\,n}\,\vect{n}_{\,lm}\right]_{\,lm}\,.\,\vect{n}_{\,lm}\
S_{\,lm}\\
& = \sum\limits_{m\in Vois(l)}(\,\mu^n_{\,lm}\, + \,\gamma^n_{\,lm}\,)\,\left(\grad{R}^{\,n}_{ij}\,.\,\vect{n}_{\,lm}\right)S_{\,lm}
\end{array}
\end{equation}
sans pr\'eciser la nature de la face $lm$, {\it via} l'appel \`a \fort{bilsc2}  et de disposer de la quantit\'e
$(\mu^n_{\,lm}\, + \gamma^n_{\,lm})$ pour construire sa
matrice simplifi\'ee.\\
\begin{itemize}
\item [$\star$] On effectue une boucle d'indice \var{IEL} sur les cellules
$\Omega_l$ :
\begin{itemize}
\item [$\Rightarrow$] $\displaystyle \var{TRRIJ }= \frac{1}{2} (R^{\,n}_{ii})_L $
\item [$\Rightarrow$] $\displaystyle \var{RCSTE} = \rho^n_L \ C_S \ \frac{ (R^{\,n}_{ii})_L}{2\,\varepsilon^n_L} $
\item [$\Rightarrow$] $\displaystyle \var{W1(IEL)} = \mu^n +\rho^n_L \ C_S \ \frac{
(R^{\,n}_{ii})_L}{2\,\varepsilon^n_L}\ (R^n_{11})_L$
\item [$\Rightarrow$] $\displaystyle \var{W2(IEL)} = \mu^n + \rho^n_L \ C_S \ \frac{ (R^{\,n}_{ii})_L}{2\,\varepsilon^n_L}\ (R^n_{22})_L$
\item [$\Rightarrow$] $\displaystyle \var{W3(IEL)} = \mu^n + \rho^n_L \ C_S \ \frac{ (R^{\,n}_{ii})_L}{2\,\varepsilon^n_L}\ (R^n_{33})_L$
\end{itemize}

\item [$\star$] Appel de \fort{visort} pour calculer la viscosit\'e orthotrope
\footnote{Comme dans le sous-programme \fort{viscfa}, on multiplie la viscosit\'e par
$\displaystyle \frac{S_{\,lm}}{\overline{L'M'}}$, o\`u $S_{\,lm}$ et
$\overline{L'M'}$ repr\'esentent respectivement la surface de la face $lm$ et la
mesure alg\'ebrique du segment reliant les projections des centres des cellules
voisines sur la normale \`a la face. On garde dans ce sous-programme  la possibilit\'e d'interpoler la viscosit\'e aux cellules lin\'eairement ou d'utiliser une moyenne harmonique. La viscosit\'e au bord est celle de la cellule de bord correspondante.}
$ \gamma^n_{\,lm} = (\tens{D}^{\,n}\,\vect{n}_{\,lm}).\vect{n}_{\,lm}$ aux faces $lm$

Le r\'esultat est stock\'e dans les tableaux \var{VISCF} et \var{VISCB}.
\end{itemize}

\item appel de \fort{codits} pour la r\'esolution de l'\'equation de
convection/diffusion/termes sources de la variable $R_{ij}$. Le terme source est
\var{SMBR}, la viscosit\'e \var{VISCF} aux faces purement internes (
resp. \var{VISCB} aux faces de bord) et \var{FLUMAS} le flux de masse interne
 ( resp. \var{FLUMAB} flux de masse au bord). Le r\'esultat est la variable $R_{ij}$ au temps
$n+1$, donc $R^{\,n+1}_{ij}$. Elle est stock\'ee dans le tableau \var{RTP} des
variables mises \`a jour.

\end{itemize}

\etape{Appel de \fort{reseps} pour la r\'esolution de la variable $\varepsilon$}

On d\'ecrit ci-dessous le sous-programme \fort{reseps}, les commentaires du sous-programme \fort{resrij} ne seront pas r\'ep\'et\'es puisque les deux sous-programmes ne diff\`erent que par leurs termes sources.

\begin{itemize}
\item Initialisation \`a z\'ero de \var{SMBR} et \var{ROVSDT}.

\item{Lecture et prise en compte des termes sources utilisateur pour la variable $\varepsilon$ :}

Appel de \fort{ustsri} pour charger les termes sources utilisateurs. Ils sont
stock\'es dans les tableaux suivants :\\
pour la cellule $\Omega_l$ repr\'esent\'ee par $\var{IEL}$ de centre $L$, on a :
\begin{equation}\notag
\left\{\begin{array}{lll}
&\var{ROVSDT(IEL)}&= |\Omega_l| \ \alpha_{\varepsilon}\\
&\var{SMBR(IEL)}&=|\Omega_l| \ \beta_{\varepsilon}\\
\end{array}\right.
\end{equation}
On affecte alors les valeurs ad\'equates au second membre \var{SMBR} et \`a la
diagonale \var{ROVSDT} comme suit :
\begin{equation}\notag
\left\{\begin{array}{lll}
&\var{SMBR(IEL)} &= \var{SMBR(IEL)} +\ |\Omega_l| \ \alpha_{\,\varepsilon} \
\varepsilon^n_L \\
&\var{ROVSDT(IEL)}&= \text{max }(-\ |\Omega_l| \ \alpha_{\,\varepsilon},0)\\
\end{array}\right.
\end{equation}

\item{Calcul du terme source de masse si $\Gamma_L > 0$ :
\begin{equation}\notag
\left\{\begin{array}{lll}
&\displaystyle \var{SMBR(IEL)} = \var{SMBR(IEL)} + |\Omega_l| \ \Gamma_L \
(\varepsilon^{\,in}_L -\varepsilon^n_L) \\
&\displaystyle \var{ROVSDT(IEL)}= \var{ROVSDT(IEL)} + |\Omega_l| \ \Gamma_L
\end{array}\right.
\end{equation}
\item Calcul du terme d'accumulation de masse et du terme instationnaire \\
On stocke $\displaystyle \var{W1}= \int_{\Omega_l}\dive\,(\rho\,\vect{u})\,d\Omega$
calcul\'e par \fort{divmas} \`a l'aide des flux de masse internes et aux bords.\\
On incr\'emente ensuite \var{SMBR} et \var{ROVSDT}.
\begin{equation}\notag
\left\{\begin{array}{lll}
&\var{SMBR(IEL)} &= \var{SMBR(IEL)} + \var{ICONV}\ \varepsilon^n_L\,(\displaystyle
\int_{\Omega_l}\dive\,(\rho\,\vect{u})\ d\Omega) \\
&\var{ROVSDT(IEL)}& = \var{ROVSDT(IEL)} +  \var{ISTAT}\,\displaystyle
\frac{\rho^n_L \ |\Omega_l|}{\Delta t^n} -  \var{ICONV}\ (\displaystyle
\int_{\Omega_l}\dive\,(\rho\,\vect{u})\ d\Omega) \\
\end{array}\right.
\end{equation}

\item Traitement du terme de production
 $\displaystyle \rho\,C_{\varepsilon_1}\,\frac{\varepsilon}{k}\,\mathcal{P}$
 et du terme de dissipation $-\,\displaystyle \rho\,C_{\varepsilon_2}\,\frac{\varepsilon}{k}\,\varepsilon$ \\
pour cela, on effectue une boucle d'indice \var{IEL} sur les cellules $\Omega_l$
de centre $L$ :
\begin{itemize}
\item [$\Rightarrow$] $\displaystyle \var{TRPROD}= \frac{1}{2} (\mathcal{P}^n_{ii})_L = \frac{1}{2} \left[ \var{PRODUC(1,IEL)} +  \var{PRODUC(2,IEL)} +  \var{PRODUC(3,IEL)} \right] $
\item [$\Rightarrow$] $\displaystyle \var{TRRIJ }= \frac{1}{2} (R^n_{ii})_L $
\item [$\Rightarrow$] $\displaystyle \var{SMBR(IEL)} = \var{SMBR(IEL)} + \rho^n_L
|\Omega_l| \left[ -C_{\varepsilon_2} \ \frac{2\,(\varepsilon^n_L)^2}{(R^n_{ii})_L} + C_{\varepsilon_1} \ \frac{\varepsilon^n_L}{(R^n_{ii})_L}\ (\mathcal{P}^n_{ii})_L \right] $
\item [$\Rightarrow$] $\displaystyle \var{ROVSDT(IEL)} = \var{ROVSDT(IEL)} + C_{\varepsilon_2} \ \rho^n_L \ |\Omega_l| \ \frac{2\,\varepsilon^n_L}{(R^n_{ii})_L}$
\end{itemize}

\item Appel de \fort{rijthe} pour le calcul des termes de gravit\'e $\mathcal{G}^n_{\varepsilon}$ et ajout dans \var{SMBR}.

$ \var{SMBR} = \var{SMBR} + \mathcal{G}^n_{\varepsilon}$\\
Ce calcul n'a lieu que si $\var{IGRARI()} = 1$.

\item Calcul de la diffusion de $\varepsilon$ \\
 Le terme $\dive \left[\mu\, \grad(\varepsilon) + \tens{A'}\,\grad(\varepsilon)
\right]$ est calcul\'e exactement de la m\^eme mani\`ere que pour les tensions
de Reynolds $R_{ij}$ en rempla\c cant $\tens{A}$ par $\tens{A'}$.

\item Appel de \fort{codits} pour la r\'esolution de l'\'equation de
convection/diffusion/termes sources de la variable principale $\varepsilon$. Le
r\'esultat $\varepsilon^{\,n+1}$ est stock\'e dans le tableau \var{RTP} des
variables mises \`a jour.
}
\end{itemize}

\etape{clippings finaux}
On passe enfin dans le sous-programme  \fort{clprij} pour faire un clipping \'eventuel
des variables $R^{\,n+1}_{ij}$ et $\varepsilon^{\,n+1}$. Le sous-programme
\fort{clprij} est appel\'e\footnote{L'option
$\var{ICLIP} = 1$ consiste en un clipping minimal des variables $R_{ii}$ et
$\varepsilon$ en prenant la valeur absolue de ces variables puisqu'elles ne
peuvent \^etre que positives.} avec $\var{ICLIP} = 2$ . Cette option
\footnote{Quand la valeur des grandeurs $R_{ii}$ ou $\varepsilon$ est
n\'egative, on la remplace par le minimum entre sa valeur absolue et (1,1)
fois la valeur obtenue au pas de temps pr\'ec\'edent.} contient l'option $\var{ICLIP} = 1$  et permet de v\'erifier l'in\'egalit\'e de Cauchy-Schwarz sur les grandeurs extra-diagonales du tenseur $\tens{R}$ (pour $i \neq j$, $|R_{ij}|^2 \le R_{ii} R_{jj}$).


%%%%%%%%%%%%%%%%%%%%%%%%%%%%%%%%%%
%%%%%%%%%%%%%%%%%%%%%%%%%%%%%%%%%%
\section{Points \`a traiter}
%%%%%%%%%%%%%%%%%%%%%%%%%%%%%%%%%%
%%%%%%%%%%%%%%%%%%%%%%%%%%%%%%%%%%
Sauf mention explicite, $\phi$ repr\'esentera une tension de Reynolds ou la dissipation turbulente ($\phi = R_{ij} \ \text{ou} \ \varepsilon$).

\begin{itemize}
\item {La vitesse utilis\'ee pour le calcul de la production est explicite. Est-ce qu'une implicitation peut am\'eliorer la pr\'ecision temporelle de $\phi$ \footnote{Cette remarque peut \^etre g\'en\'eralis\'ee. En effet, peut-on envisager d'actualiser les variables d\'ej\`a r\'esolues (sans r\'eactualiser les variables turbulentes apr\`es leur r\'esolution)? Ceci obligerait \`a modifier les sous-programmes tels que \fort{condli} qui sont appel\'es au d\'ebut de la boucle en temps.} ?}
\item {Dans quelle mesure le terme d'\'echo de paroi est-il valide ? En effet, ce terme est remis en question par certains auteurs.}
\item {On peut envisager la r\'esolution d'un syst\`eme hyperbolique pour les
tensions de Reynolds afin d'introduire un couplage avec le champ de vitesse.}
\item {Le flux au bord \var{VISCB} est annul\'e dans le sous-programme
\fort{vectds}. Peut-on envisager de mettre au bord la valeur de la variable
concern\'ee \`a la cellule de bord correspondant? De m\^eme, il faudrait se
pencher sur les hypoth\`eses sous-jacentes \`a l'annulation des contributions
aux bords de \var{VISCB} lors du calcul de : $$\left[ \tens{D}^n\,\left( \grad{R^{\,n}_{ij}} - (\grad R^{\,n}_{ij}\,.\,\vect{n}_{\,lm})\,\vect{n}_{\,lm}\right) \right]\,.\,\vect{n}_{\,lm}.$$}
\item {Un probl\`eme de pond\'eration appara\^\i t plus g\'en\'eralement. Si on prend la partie explicite de $\tens{D}\,\grad(\phi)$, la pond\'eration aux faces internes utilise le coefficient $\displaystyle\frac{1}{2}$ avec pond\'eration s\'epar\'ee de $\tens{D}$ et $\grad(\phi)$, alors que pour $\tens{E}\,\grad(\phi)$, on calcule d'abord ce terme aux cellules pour ensuite l'interpoler lin\'eairement aux faces \footnote{Cette interpolation se fait dans \fort{vectds} avec des coefficients de pond\'eration aux faces.}. Ceci donne donc deux types d'interpolations pour des termes de m\^eme nature.}
\item {On laisse la possibilit\'e dans \fort{visort} d'utiliser une moyenne
harmonique aux faces. Est-ce que ceci est valable puisque les interpolations
utilis\'ees lors du calcul de la partie explicite de $\tens{A}\,\grad{\phi}$
sont des moyennes arithm\'etiques ?}
\item {Les techniques adopt\'ees lors du clipping sont \`a revoir.}
\item {On utilise dans le cadre du mod\`ele $\displaystyle R_{ij}-\varepsilon $ une semi-implicitation de termes comme $\displaystyle \phi_{ij,1}$ ou $\displaystyle -\rho\,C_{\varepsilon_2}\,\frac{\varepsilon}{k}\,\varepsilon$. On peut envisager le m\^eme type d'implicitation dans \fort{turbke} m\^eme en pr\'esence du couplage $\displaystyle k-\varepsilon$.}
\item L'adoption d'une r\'esolution d\'ecoupl\'ee fait perdre l'invariance par rotation.
\item La formulation et l'implantation des conditions aux limites de paroi
devront \^etre v\'erifi\'ees. En effet, il semblerait que, dans certains cas, des ph\'enom\`enes
``oscillatoires'' apparaissent, sans qu'il soit ais\'e d'en d\'eterminer la cause.
\item L'implicitation partielle (du fait de la r\'esolution d\'ecoupl\'ee) des
conditions aux limites conduit souvent \`a des calculs instables. Il
conviendrait d'en conna\^\i tre la raison. L'implicitation partielle avait
\'et\'e mise en \oe uvre afin de tenter d'utiliser un pas de temps plus grand
dans le cas de jets axisym\'etriques en particulier.

\end{itemize}

%                      Code_Saturne version 1.3
%                      ------------------------
%
%     This file is part of the Code_Saturne Kernel, element of the
%     Code_Saturne CFD tool.
%
%     Copyright (C) 1998-2007 EDF S.A., France
%
%     contact: saturne-support@edf.fr
%
%     The Code_Saturne Kernel is free software; you can redistribute it
%     and/or modify it under the terms of the GNU General Public License
%     as published by the Free Software Foundation; either version 2 of
%     the License, or (at your option) any later version.
%
%     The Code_Saturne Kernel is distributed in the hope that it will be
%     useful, but WITHOUT ANY WARRANTY; without even the implied warranty
%     of MERCHANTABILITY or FITNESS FOR A PARTICULAR PURPOSE.  See the
%     GNU General Public License for more details.
%
%     You should have received a copy of the GNU General Public License
%     along with the Code_Saturne Kernel; if not, write to the
%     Free Software Foundation, Inc.,
%     51 Franklin St, Fifth Floor,
%     Boston, MA  02110-1301  USA
%
%-----------------------------------------------------------------------
%
\programme{vortex}
%
\vspace{1cm}
%%%%%%%%%%%%%%%%%%%%%%%%%%%%%%%%%%
%%%%%%%%%%%%%%%%%%%%%%%%%%%%%%%%%%
\section{Fonction}
%%%%%%%%%%%%%%%%%%%%%%%%%%%%%%%%%%
%%%%%%%%%%%%%%%%%%%%%%%%%%%%%%%%%%
Ce sous-programme est d�di� � la g�n�ration des conditions d'entr�e
turbulente utilis�es en LES.


La m�thode des vortex est bas�e sur une approche de tourbillons
ponctuels. L'id�e de la m�thode consiste � injecter des tourbillons 2D dans le
plan d'entr�e du calcul, puis � calculer le champ de vitesse induit par ces
tourbillons au centre des faces d'entr�e.

%                      Code_Saturne version 1.3
%                      ------------------------
%
%     This file is part of the Code_Saturne Kernel, element of the
%     Code_Saturne CFD tool.
% 
%     Copyright (C) 1998-2007 EDF S.A., France
%
%     contact: saturne-support@edf.fr
% 
%     The Code_Saturne Kernel is free software; you can redistribute it
%     and/or modify it under the terms of the GNU General Public License
%     as published by the Free Software Foundation; either version 2 of
%     the License, or (at your option) any later version.
% 
%     The Code_Saturne Kernel is distributed in the hope that it will be
%     useful, but WITHOUT ANY WARRANTY; without even the implied warranty
%     of MERCHANTABILITY or FITNESS FOR A PARTICULAR PURPOSE.  See the
%     GNU General Public License for more details.
% 
%     You should have received a copy of the GNU General Public License
%     along with the Code_Saturne Kernel; if not, write to the
%     Free Software Foundation, Inc.,
%     51 Franklin St, Fifth Floor,
%     Boston, MA  02110-1301  USA
%
%-----------------------------------------------------------------------
%
%%%%%%%%%%%%%%%%%%%%%%%%%%%%%%%%%%
%%%%%%%%%%%%%%%%%%%%%%%%%%%%%%%%%%
\section{Discr\'etisation}
%%%%%%%%%%%%%%%%%%%%%%%%%%%%%%%%%%
%%%%%%%%%%%%%%%%%%%%%%%%%%%%%%%%%%

Le terme convectif en $\dive(\underline{u} \otimes \rho\,\underline{u})$
introduit une non lin\'earit\'e et un couplage des composantes de la vitesse
$\vect{u}$ dans l'�quation (\ref{Base_Preduv_eqqdm}). Une lin\'earisation et un d\'ecouplage
des trois composantes de la 
vitesse sont r\'ealis\'es lors de la discr\'etisation de cette \'etape de
pr\'ediction.\\
En effet, soit :
\begin{equation}
\vect{\widetilde{u}}= \vect{u}^n + \delta \vect{u} 
\end{equation}
La contribution exacte du terme convectif \`a prendre en compte dans cette
\'etape de pr\'ediction serait :\\
\begin{equation}\label{Base_Preduv_Conv_exact}
\begin{array}{ll}
\dive(\vect{\widetilde{u}} \otimes \rho\,\vect{\widetilde{u}}) =
\dive(\vect{u}^{n} \otimes \rho\,\vect{u}^{n}) + \dive(\delta \vect{u} \otimes
\rho\,\vect{u}^{n}) +  \underbrace { \dive(\vect{u}^{n} \otimes
\rho\,\delta \vect{u})}_{\text {terme couplant lin\'eaire}} +  \underbrace { \dive(\delta \vect{u} \otimes
\rho\,\delta \vect{u})}_{\text {terme couplant et non lin\'eaire}}\\
\end{array} 
\end{equation}
Les deux derniers termes de l'expression (\ref{Base_Preduv_Conv_exact}) sont {\em a priori} n�glig�s
de mani�re � obtenir un syst\`eme en vitesse qui soit d\'ecoupl\'e et donc,
�viter l'inversion d'une matrice pouvant \^etre de tr\`es grande taille. Ces
deux termes peuvent n�anmoins �tre pris en compte de mani�re plus ou moins
approch�e par extrapolation explicite du flux de masse en $n+\theta_F$ (pour le
terme couplant lin�aire provenant de la convection de $\vect{u}^{n}$ par $\delta
\vect{u}$) et utilisation d'un point-fixe par sous it�ration sur le sous
programme \fort{navsto} (pour le terme non-lin�aire, en sp�cifiant $\var{NTERUP}>1$).

L'�quation (\ref{Base_Preduv_eqqdm}) est discr�tis�e au temps $n+\theta$ � l'aide d'un
$\theta$-sch�ma, et le tenseur des pertes de charges d�compos� en une partie
diagonale $\tens{K}_{d}$ et une extradiagonale $\tens{K}_{e}$ (soit
 $\tens{K}_{pdc}=\tens{K}_{d}+\tens{K}_{e}$).\\
$\bullet$ La pression est suppos�e connue en $n-1+\theta$ (d�calage temporel
pression-vitesse) et le gradient naturellement calcul� � cet instant.\\ 
$\bullet$ Les termes sources de viscosit� secondaire, de gradient transpos\'e,
ceux provenant du mod�le de turbulence\footnote{except� $\dive (\mu_t\ (\ggrad
\underline {u}))$}, $\rho\,\tens{K}_{\,e}\ \underline{u}$, $(\rho -\rho_0)
\underline {g}$ ainsi que $\underline{T}_{s}^{\,exp}$ et
$\Gamma\,\underline{u}_{\,i}$ sont pris de mani�re explicite au temps $n$, ou
extrapol�s suivant le sch�ma en temps choisi pour les propri�t�s physique et les
termes sources.\\ 
$\bullet$ Les termes sources $\underline{u}\,\,\dive (\rho\,\underline {u})$,
$\Gamma\,\,\underline{u}$, $T_{s}^{\,imp}\,\,\underline{u}$ et
$-\rho\,\tens{K}_{\,d}\,\,\underline{u}$ sont implicit�s est calcul�s �
l'instant $n+\theta$.\\ 
$\bullet$ Le terme de diffusion $\dive (\mu_{\,tot}\,\ggrad \underline{u})$ est
implicit� : la vitesse est prise � l'instant $n+\theta$ et la viscosit�
explicit�e ou extrapol�e.\\ 
$\bullet$ Enfin, le terme de convection en $\dive(\,\underline{u} \otimes
(\rho\underline{u})\,)$ est implicit� : la composante r�solue de la vitesse est
prise en $n+\theta$, et le flux de masse, explicit�, ou extrapol� en
$n+\theta_F$. 

Par souci de clart�, on suppose, en l'absence d'indication, que les propri�tes
physiques $\Phi$ ($\rho,\,\mu_{tot},\,...$) et le flux de masse
$(\rho\underline{u})$ sont pris respectivement aux instants $n+\theta_\Phi$ et
$n+\theta_F$, o� $\theta_\Phi$ et $\theta_F$ d�pendent des sch�mas en temps
sp�cifiquement utilis�s pour ces grandeurs\footnote{cf. \fort{introd}}. 

La discr�tisation temporelle de l'�quation (\ref{Base_Preduv_eqqdm}) s'�crit alors comme suit : 

\begin{equation}\label{Base_Preduv_eq_di1}
 \begin{array}{c}
\displaystyle \rho\,\ \frac{ \underline {\widetilde{u}}^{n+1} -\underline {u}^{n} }
{\Delta t} + \dive(\,\underline{\widetilde{u}}^{n+\theta} \otimes (\rho\underline{u})\,) -\dive
(\mu_{\,tot}\,\ggrad \underline{\widetilde{u}}^{n+\theta}) =
\\
\displaystyle
 - \grad p^{n-1+\theta} + \dive (\rho\,\underline {u})\,\underline{\widetilde{u}}^{n+\theta} +(\Gamma\,\underline{u}_{\,i})^{n+\theta_S}-\Gamma^n\,\,\underline{\widetilde{u}}^{n+\theta}
\\
\begin{array}{c}
\displaystyle
- \rho\,\tens{K}_{\,d}^{n}\,\,\underline{\widetilde{u}}^{n+\theta} - (\rho\,\tens{K}_{\,e}\ \underline{u})^{n+\theta_S} + (\underline{T}_{s}^{\,exp})^{\,n+\theta_S} + T_{s}^{\,imp}\,\,\underline{\widetilde{u}}^{n+\theta}
\\
\displaystyle
+[\dive (\mu_{\,tot}\,^t\ggrad \underline {u})]^{n+\theta_S}-\frac {2} {3}[\,\grad (\mu_{\,tot}\,\dive \underline {u})]^{n+\theta_S} + (\rho -\rho_0) \underline {g}
 - (\underline{turb})^{n+\theta_S}
\end{array}
\end{array}
\end{equation}
o\`u, par souci de simplification, on a pos\'e :
\begin{equation}
\mu_{\,tot}=
\begin{cases}
\mu+\mu_t & \text{pour les mod�les � viscosit� turbulente ou en LES}, \\
\mu & \text{pour les mod�les au second ordre ou en laminaire}
\end{cases} \ 
\end{equation}
\\
et :
\begin{equation}
\underline{turb}^{n}=
\begin{cases}
\displaystyle\frac {2}{3}\grad (\rho^{n}\,k^{n}) & \text{pour les mod�les � viscosit� turbulente}, \\
\dive(\rho^{n}\,\tens{R}^n) & \text{pour les mod�les au second ordre},\\
0 & \text{en laminaire ou en LES}\\
\end{cases}
\end{equation}
Par analogie avec l'�criture du $\theta$-sch�ma pour une variable scalaire, $\,
\underline {\widetilde{u}}^{n+\theta}$ est interpol�e � partir de la vitesse
pr�dite $\underline {\widetilde{u}}^{n+1}$ de la mani\`ere suivante\footnote{si
$\theta=1/2$, ou qu'une extrapolation est utilis�e, l'ordre 2 n'est obtenu que si
le pas de temps $\Delta t$ est uniforme en temps et en espace.}~: 
\begin{equation}
\underline {\widetilde{u}}^{n+\theta}=\theta\, \underline
{\widetilde{u}}^{n+1}+(1-\theta)\, \underline {u}^{n}\\ 
\end{equation}
Avec :
\begin{equation}
\left\{
\begin{array}{ll}
\theta = 1   & \text{Pour un sch\'ema de type Euler implicite d'ordre 1.}\\
\theta = 1/2 & \text{Pour un sch\'ema de type Cranck-Nicolson d'ordre 2.}\\
\end{array}
\right.
\end{equation}

L'�quation (\ref{Base_Preduv_eq_di1}) est alors r��crite sous la forme :

\begin{equation}\label{Base_Preduv_eq_di2}
\begin{array}{c}
\displaystyle \underbrace{\left(\frac{\rho}{\Delta t} -\theta \,\dive (\rho\,\underline {u}) +\theta \,\, \Gamma^n +
\theta \,\, \rho\,\tens{K}_{\,d}^n-\theta \,T_s^{\,imp} \right)}_{\displaystyle f_s^{imp}}\, (\underline {\,\widetilde{u}}^{n+1} -\underline {u}^{n})
\\
 +\, \theta\, \dive(\underline {\widetilde{u}}^{n+1} \otimes (\rho\underline{u}))-\, \theta\,\dive (\mu_{\,tot}\,\ggrad \underline {\widetilde{u}}^{n+1}) =
\\
-\,(1-\theta)\, \dive(\underline {u}^{n} \otimes (\rho\underline{u})) +\,(1-\theta)\,\dive (\mu_{\,tot}\,\ggrad \underline {u}^{n})
\\
f_s^{exp}\left\{
\begin{array}{c}
\displaystyle 
- \grad p^{n-1+\theta} + \dive (\rho\,\underline {u})\,\underline{u}^{n} +\,(\,\Gamma^{n}\,\underline{u}_{\,i}\,)^{n+\theta_S}- \Gamma^n\,\,\underline{u}^{n}
\\
\displaystyle
-(\,\rho\,\tens{K}_{\,e}\ \underline{u}\,)^{n+\theta_S} -\rho\,\tens{K}_{\,d}^n\ \underline{u}^{n}+ (\underline{T}_{s}^{\,exp})^{\,n+\theta_S} + T_s^{\,imp}\,\,\underline {u}^{n} 
\\
\displaystyle
+[\dive (\mu_{\,tot}\,^t\ggrad \underline {u}\,)]^{n+\theta_S}-\frac {2} {3}[\,\grad (\mu_{\,tot}\,\dive \underline {u}\,)]^{n+\theta_S} + (\rho -\rho_0) \underline {g}-(\underline{turb})^{n+\theta_S}
\end{array}
\right.
\end{array}
\end{equation}

d'o� l'�quation r�solue par le sous-programme \fort{codits} :
\begin{equation}\begin{array}{c}
\displaystyle
f_s^{\,imp}(\underline {\widetilde{u}}^{n+1}-\underline {u}^{n}) + \theta\, \dive(\underline{\widetilde{u}}^{n+1} \otimes (\rho
\underline{u})) - \theta\,\dive (\,\mu_{\,tot}\,\ggrad \underline{\widetilde{u}}^{n+1}) = 
\\\\
\displaystyle
-(1-\theta)\,\dive(\underline{u}^{n} \otimes (\rho \underline{u}))+(1-\theta)\,\dive (\,\mu_{\,tot}\,\ggrad \underline{u}^{n})
+ \underline{f}_{\,s}^{\,exp}
\end{array}
\end{equation}
La m\'ethode de discr\'etisation spatiale est d\'evelopp\'ee dans le sous-programme \fort{codits}.\\



\minititre{Remarques :}
{\tiny$\blacksquare$} Dans le cas standard sans extrapolation, le terme
$-\,T_s^{\,imp}$ n'est ajout� � $f_s^{\,imp}$ que s'il est positif afin de ne
pas affaiblir la dominance de la diagonale de la matrice � inverser.\\ 
{\tiny$\blacksquare$} Si une extrapolation est utilis�e, par contre,
$\,T_s^{\,imp}$ est ajout� � $f_s^{\,imp}$ quel que soit son signe. En effet, l'id�e intuitive qui
consiste � prendre~: 
\begin{equation}
\begin{cases}
\displaystyle
(\underline{T}_{s}^{\,exp} + T_{s}^{\,imp}\,\underline {u})^{\,n+\theta_S} &
\text{si } T_{s}^{\,imp} > 0\\ 
\displaystyle
(\underline{T}_{s}^{\,exp})^{\,n+\theta_S} + T_{s}^{\,imp}\,\underline{u}^{n+\theta} &\text{sinon}\\
\end{cases}
\end{equation} 
aboutit � une incoh�rence dans le traitement si $T_s^{imp}$ change de signe
entre deux pas de temps.\\ 
{\tiny$\blacksquare$} la partie diagonale $\tens{K}_{\,d}$ du terme
de perte de charge est utilis�e dans $f_s^{\,imp}$. En fait, pour \^etre rigoureux,
il faudrait ne retenir que les contributions positives (point signal\'e dans le
sous-programme utilisateur associ\'e \fort{uskpdc}). Cette prise en compte sera \`a am\'eliorer.\\
{\tiny$\blacksquare$} Le terme $\theta\,\Gamma^{n}-\theta\,\dive
(\rho\,\underline {u})$ ne pose pas de probl�me pour la 
dominance de la diagonale de la matrice car il est exactement compens� par le
terme de convection (cf. \fort{covofi}). 


%                      Code_Saturne version 1.3
%                      ------------------------
%
%     This file is part of the Code_Saturne Kernel, element of the
%     Code_Saturne CFD tool.
%
%     Copyright (C) 1998-2007 EDF S.A., France
%
%     contact: saturne-support@edf.fr
%
%     The Code_Saturne Kernel is free software; you can redistribute it
%     and/or modify it under the terms of the GNU General Public License
%     as published by the Free Software Foundation; either version 2 of
%     the License, or (at your option) any later version.
%
%     The Code_Saturne Kernel is distributed in the hope that it will be
%     useful, but WITHOUT ANY WARRANTY; without even the implied warranty
%     of MERCHANTABILITY or FITNESS FOR A PARTICULAR PURPOSE.  See the
%     GNU General Public License for more details.
%
%     You should have received a copy of the GNU General Public License
%     along with the Code_Saturne Kernel; if not, write to the
%     Free Software Foundation, Inc.,
%     51 Franklin St, Fifth Floor,
%     Boston, MA  02110-1301  USA
%
%-----------------------------------------------------------------------
%

%%%%%%%%%%%%%%%%%%%%%%%%%%%%%%%%%%
%%%%%%%%%%%%%%%%%%%%%%%%%%%%%%%%%%
\section{Mise en \oe uvre}
%%%%%%%%%%%%%%%%%%%%%%%%%%%%%%%%%%
%%%%%%%%%%%%%%%%%%%%%%%%%%%%%%%%%%
La num\'ero de la phase trait\'ee fait partie des arguments de \fort{turrij}. On
omettra volontairement de le pr\'eciser dans ce qui suit, on indiquera par $(\ )$ la
notion de tableau s'y rattachant.

\etape{Calcul des termes de production $\tens{\mathcal{P}}$}
\begin{itemize}
\item [$\star$] Initialisation \`a z\'ero du tableau \var{PRODUC} dimensionn\'e \`a $\var{NCEL}\times 6$.
\item [$\star$] On appelle trois fois \fort{grdcel} pour calculer les gradients des composantes de la vitesse $u$, $v$ et
$w$ prises au temps $n$.

Au final, on a :\\
$\displaystyle
\begin{array} {ll}
\var{PRODUC(1,IEL)} = & \displaystyle - 2 \left[ R_{11}^{\,n} \frac{\partial u^{\,n}} {\partial x} +R_{12}^{\,n} \frac{\partial u^{\,n}} {\partial y}+R_{13}^{\,n} \frac{\partial u^{\,n}} {\partial z} \right] \text{        (production de $R_{11}^{\,n}$)}\\
\var{PRODUC(2,IEL)} = & \displaystyle - 2 \left[ R_{12}^{\,n} \frac{\partial v^{\,n}} {\partial x} +R_{22}^{\,n} \frac{\partial v^{\,n}} {\partial y}+R_{23}^{\,n} \frac{\partial v^{\,n}} {\partial z} \right] \text{        (production de $R_{22}^{\,n}$)}\\
\var{PRODUC(3,IEL)} = & \displaystyle - 2 \left[ R_{13}^{\,n} \frac{\partial w^{\,n}} {\partial x} +R_{23}^{\,n} \frac{\partial w^{\,n}} {\partial y}+R_{33}^{\,n} \frac{\partial w^{\,n}} {\partial z} \right] \text{        (production de $R_{33}^{\,n}$)}\\
\var{PRODUC(4,IEL)} = & \displaystyle - \left[ R_{12}^{\,n} \frac{\partial u^{\,n}} {\partial x} +R_{22}^{\,n} \frac{\partial u^{\,n}} {\partial y}+R_{23}^{\,n} \frac{\partial u^{\,n}} {\partial z} \right] \\
& \displaystyle - \left[ R_{11}^{\,n} \frac{\partial v^{\,n}} {\partial x} +R_{12}^{\,n} \frac{\partial v^{\,n}} {\partial y}+R_{13}^{\,n} \frac{\partial v^{\,n}} {\partial z} \right] \text{        (production de $R_{12}^{\,n}$)} \\
\var{PRODUC(5,IEL)} = & \displaystyle - \left[ R_{13}^{\,n} \frac{\partial u^{\,n}} {\partial x} +R_{23}^{\,n} \frac{\partial u^{\,n}} {\partial y}+R_{33}^{\,n} \frac{\partial u^{\,n}} {\partial z} \right] \\
& \displaystyle - \left[ R_{11}^{\,n} \frac{\partial w^{\,n}} {\partial x} +R_{12}^{\,n} \frac{\partial w^{\,n}} {\partial y}+R_{13}^{\,n} \frac{\partial w^{\,n}} {\partial z} \right] \text{        (production de $R_{13}^{\,n}$)} \\
\var{PRODUC(6,IEL)} = & \displaystyle - \left[ R_{13}^{\,n} \frac{\partial v^{\,n}} {\partial x} +R_{23}^{\,n} \frac{\partial v^{\,n}} {\partial y}+R_{33}^{\,n} \frac{\partial v^{\,n}} {\partial z} \right] \\
& \displaystyle - \left[ R_{12}^{\,n} \frac{\partial w^{\,n}} {\partial x} +R_{22}^{\,n} \frac{\partial w^{\,n}} {\partial y}+R_{23}^{\,n} \frac{\partial w^{\,n}} {\partial z} \right]  \text{        (production de $R_{23}^{\,n}$)}
\end{array}
$
\end{itemize}

\etape{Calcul du gradient de la masse volumique $\rho^n$ prise au d\'ebut du pas
de temps courant\footnote{{\it i.e.} calcul\'ee \`a partir des
variables du pas de temps pr\'ec\'edent $n$ si n\'ecessaire.} $(n+1)$}
Ce calcul n'a lieu que si les termes de gravit\'e doivent \^etre pris en compte
($\var{IGRARI()} =1$).
\begin{itemize}
\item [$\star$] Appel de \fort{grdcel}  pour calculer le gradient de $\rho^n$
dans les trois directions de l'espace. Les conditions aux limites sur $\rho^n$
sont des conditions de Dirichlet puisque la valeur de $\rho^n$ aux faces de bord
$ik$ (variable \var{IFAC}) est connue et vaut $\rho_{\,b_{\,ik}}$. Pour \'ecrire les conditions aux limites
sous la forme habituelle, $$\rho_{\,b_{\,ik}} = \var{COEFA} + \var{COEFB}
\,\rho^n_{\,I'}$$ on pose alors $\var{COEFA}=
\var{PROPCE(IFAC,IPPROB(IROM(IPHAS)))}$ et $\var{COEFB} = \var{VISCB} = 0$.\\
$\var{PROPCE(1,IPPROB(IROM(IPHAS)))}$ (resp.$\var{VISCB}$) est utilis\'e en lieu
et place de l'habituel \var{COEFA} ($\var{COEFB}$), lors de l'appel \`a \fort{grdcel}.\\
On a donc :\\
$\displaystyle \var{GRAROX}= \frac{\partial \rho^n}{\partial x}\ $,$\displaystyle \ \var{GRAROY}= \frac{\partial
\rho^n}{\partial y}$ et $
\displaystyle \ \var{GRAROZ}= \frac{\partial \rho^n}{\partial z}\ $.

\end{itemize}

Le gradient de $\rho^n$ servira \`a calculer les termes de production par effets de gravit\'e si ces derniers sont pris en compte.

\etape{Boucle \var{ISOU} de $1$ \`a $6$ sur les tensions de Reynolds}
Pour $\var{ISOU} = 1,2,3,4,5,6$, on r\'esout respectivement et dans
l'ordre  les
\'equations de $R_{11}$, $R_{22}$, $R_{33}$, $R_{12}$, $R_{13}$ et $R_{23}$ par
l'appel au sous-programme \fort{resrij}.\\
La r\'esolution se fait par incr\'ement $\delta {R}_{ij}^{\,n+1,k+1}$ , en utilisant la m\^eme m\'ethode que
celle d\'ecrite dans le sous-programme \fort{codits}. On adopte ici les m\^emes notations.
\var{SMBR} est le second membre du syst\`eme \`a inverser, syst\`eme portant sur
les incr\'ements de la variable. \var{ROVSDT} repr\'esente la diagonale de la
matrice, hors convection/diffusion.\\
On va r\'esoudre l'\'equation (\ref{Base_Turrij_Eq_Temp_Rij}) sous forme incr\'ementale en
utilisant \fort{codits}, soit :
\begin{equation}\label{Base_Turrij_Eq_Temp_deltaRij}
\begin{array}{ll}
&\displaystyle \underbrace{\left(\frac {\rho^n_L}{\Delta t^n}
+ \rho^n_L \,C_1\,\frac{\varepsilon^n_L}{k^n_L}(1-\frac{\delta_{ij}}{3})
 - m^n_{\,lm} + \Gamma_L\,+ max(-\alpha^n_{R_{ij}},0)\right)\,|\Omega_l|}
_{\text {$\var{ROVSDT}$ contribuant
\`a la diagonale de la matrice simplifi\'ee de \fort{matrix}}}\,(\delta{R}_{ij}^{\,n+1,p+1})_{\,L}\\\\
&  \underbrace{+\sum\limits_{m\in Vois(l)}\displaystyle \left[
 m^n_{\,lm} \delta R_{ij,\,f_{\,lm}}^{\,n+1,p+1}
- (\mu^n_{\,lm} + \gamma^n_{\,lm})\
\frac{({\delta R}_{ij}^{\,n+1,p+1})_{M}-({\delta R}_{ij}^{\,n+1,p+1})_{L})}{\overline{L'M'}}\,
S_{\,lm} \right]}_{\text { convection upwind pur et diffusion non reconstruite
relatives \`a la matrice simplifi\'ee de \fort{matrix}\footnotemark}} \\
% voir le texte de la footmark plus bas
&= - \displaystyle\frac {\rho^n_L}{\Delta t^n}\,\left(\,(R^{\,n+1,p}_{ij})_L - (R^{\,n}_{ij})_L\,\right)\\
&-\,\underbrace{\displaystyle\int_{\Omega_l} \left(
\dive\,[\,(\rho\,\vect{u})^n\,R^{\,n+1,p}_{ij} - (\mu^n\,+ \gamma^n\,)\,
\grad{R^{\,n+1,p}_{ij}}\,]\right)\,d\Omega}_{\text {convection et diffusion
trait\'ees par \fort{bilsc2}}}\\
&+\displaystyle \int_{\Omega_l} \left[\,\mathcal{P}^{\,n+1,p}_{ij} + \mathcal{G}^{\,n+1,p}_{ij}
- \displaystyle\rho^n \,C_1\,\frac{\varepsilon^n}{k^n}\left[R^{\,n+1,p}_{ij}-
\frac{2}{3}\,k^n\,\delta_{ij}\right] + \phi^{\,n+1,p}_{ij,2} +
\phi^{\,n+1,p}_{ij,w}\,\right]\, d\Omega \\
& + \displaystyle\int_{\Omega_l} \left[- \frac{2}{3} \rho^n \varepsilon^n \delta_{ij}
 + \Gamma\,(\,R^{\,in}_{ij} - R^{\,n+1,p}_{ij}\,) +
\alpha^n_{R_{ij}}\,R^{\,n+1,p}_{ij}+ \beta^n_{R_{ij}}\right]\, d\Omega\\
&+ \sum\limits_{m\in
Vois(l)}\displaystyle \left[\ \tens{E}^n\,\grad{R}^{\,n+1,p}_{ij} \right]_{\,lm}\,.\,\vect{n}_{\,lm}S_{\,lm}\\
&+ \sum\limits_{m\in Vois(l)}\displaystyle \left[\
\tens{D}^n\,\grad{R}^{\,n+1,p}_{ij} \right]_{\,lm}\,.\,\vect{n}_{\,lm}S_{\,lm}\\
&- \sum\limits_{m\in Vois(l)} \gamma^n_{\,lm} \left( \grad{R}^{\,n+1,p}_{ij}\,.\,\vect{n}_{\,lm} \right)  S_{\,lm}\\
&+ \sum\limits_{m\in Vois(l)}  m^n_{\,lm}\,(R^{\,n+1,p}_{ij})_L\\
\end{array}
\end{equation}
% si on ne fait pas comme ca, il n'apparait pas
\footnotetext[\thefootnote]{Si $\var{IRIJNU} = 1$, on remplace  $\mu^n_{\,lm}$ par $(\mu +
\mu_t)^n_{\,lm}$ dans l'expression de la diffusion non reconstruite
associ\'ee \`a la matrice simplifi\'ee de \fort{matrix} ($\mu_t$ d\'esigne la
viscosit\'e turbulente calcul\'ee comme en $k-\varepsilon$).}

o\`u on rappelle :\\
pour $n$ donn\'e entier positif, on d\'efinit la suite
 $({R}_{ij}^{\,n+1,p})_{p \in \grandN}$
 par :
\begin{equation}\notag
\left\{\begin{array}{l}
{R}_{ij}^{\,n+1,0} = {R}_{ij}^{\,n}\\
{R}_{ij}^{\,n+1,p+1} = {R}_{ij}^{\,n+1,p} + \delta{R}_{ij}^{\,n+1,p+1} \\
\end{array}\right.
\end{equation}
$(\delta{R}_{ij}^{\,n+1,p+1})_{\,L}$ d\'esigne la valeur sur l'\'el\'ement
$\Omega_l$ du $\text{$(\,p+1\,)$-i\`eme}$ incr\'ement de ${R}_{ij}^{\,n+1}$,
$ m^n_{\,lm}$ le flux de masse \`a l'instant $n$ \`a travers la face $lm$,
$\delta R_{ij,\,f_{\,lm}}^{\,n+1,p+1}$ vaut $({\delta
R}_{ij}^{\,n+1,p+1})_{L}$  si $ m^n_{\,lm} \geqslant 0$, $({\delta
R}_{ij}^{\,n+1,p+1})_{M}$ sinon,
$\mathcal{P}^{\,n+1,p}_{ij}$, $\phi^{\,n+1,p}_{ij,2}$, $\phi^{\,n+1,p}_{ij,w}$ les valeurs
des quantit\'es associ\'ees correspondant \`a l'incr\'ement
$(\delta{R}_{ij}^{\,n+1,p})$.\\



Tous ces termes sont calcul\'es comme suit :
\begin{itemize}
\item Terme de gauche de l'\'equation (\ref{Base_Turrij_Eq_Temp_deltaRij})\\
Dans \fort{resrij} est calcul\'ee la variable \var{ROVSDT}. Les autres
termes sont compl\'et\'es par \fort{codits}, lors de la construction de la matrice simplifi\'ee , {\it via} un
appel au sous-programme \fort{matrix}. La quantit\'e
 $(\mu^n_{\,lm} + \gamma^n_{\,lm})$ \`a la face $lm$ est calcul\'ee lors de l'appel \`a
\fort{visort}.\\
\item Second membre de l'\'equation (\ref{Base_Turrij_Eq_Temp_deltaRij})\\
Le premier terme non d\'etaill\'e est calcul\'e par le sous-programme
\fort{bilsc2}, qui applique le sch\'ema convectif choisi par l'utilisateur, qui
reconstruit ou non selon le souhait de l'utilisateur les gradients intervenants
dans la convection-diffusion.\\
Les termes sans accolade sont, eux, compl\`etement explicites et ajout\'es au fur et
\`a mesure dans \var{SMBR} pour former
l'expression $f^{\,exp}_s$ de \fort{codits}.
\end{itemize}
On d\'ecrit ci-dessous les \'etapes de \fort{resrij} :
\begin{itemize}

\item DELTIJ = 1, pour $\var{ISOU} \leqslant 3$ et DELTIJ = 0  Si $\var{ISOU} >
3$. Cette valeur repr\'esente le symbole de Kroeneker $\delta_{ij}$.

\item Initialisation \`a z\'ero de \var{SMBR} (tableau contenant le second
membre) et \var{ROVSDT} (tableau contenant la diagonale de la matrice sauf celle
relative \`a la contribution de la
diagonale des op\'erateurs de convection et de diffusion lin\'earis\'es
\footnote{qui correspondent aux sch\'emas convectif upwind pur et diffusif sans
reconstruction.}), tous deux de dimension $\var{NCEL}$.

\item Lecture et prise en compte des termes sources utilisateur pour la variable $R_{ij}$

Appel \`a \fort{ustsri} pour charger les termes sources utilisateurs. Ils sont
stock\'es comme suit. Pour la cellule $\Omega_l$ de centre $L$, repr\'esent\'ee par $\var{IEL}$, on a :\\
\begin{equation}\notag
\left\{\begin{array}{lll}
&\var{ROVSDT(IEL)}&= |\Omega_l| \ \alpha_{R_{ij}}\\
&\var{SMBR(IEL)}&=|\Omega_l| \ \beta_{R_{ij}}\\
\end{array}\right.
\end{equation}
On affecte alors les valeurs ad\'equates au second membre \var{SMBR} et \`a la
diagonale \var{ROVSDT} comme suit :
\begin{equation}\notag
\left\{\begin{array}{lll}
&\var{SMBR(IEL)} &= \var{SMBR(IEL)} +\ |\Omega_l| \ \alpha_{R_{ij}} \ (R^n_{ij})_L \\
&\var{ROVSDT(IEL)}&= \text{max }(-\ |\Omega_l| \ \alpha_{R_{ij}},0)\\
\end{array}\right.
\end{equation}
La valeur de $ \var{ROVSDT}$ est ainsi calcul\'ee pour des raisons de stabilit\'e
num\'erique. En effet, on ne rajoute sur la diagonale que les valeurs positives,
ce qui correspond physiquement \`a impliciter les termes de rappel uniquement.
\item{Calcul du terme source de masse  si $\Gamma_L > 0$}

Appel de \fort{catsma} et incr\'ementation si n\'ecessaire de \var{SMBR} et
\var{ROVSDT} {\it via} :\\
\begin{equation}\notag
\left\{\begin{array}{lll}
\displaystyle \var{SMBR(IEL)} = \var{SMBR(IEL)} + |\Omega_l| \ \Gamma_L \
\left[(R^{\,in}_{ij})_L - (R^{\,n}_{ij})_L \right] \\
\displaystyle \var{ROVSDT(IEL)}=\var{ROVSDT(IEL)} + |\Omega_l| \ \Gamma_L
\end{array}\right.
\end{equation}
\item Calcul du terme d'accumulation de masse et du terme instationnaire

On stocke $\displaystyle \var{W1}= \int_{\Omega_l}\dive\,(\rho\,\vect{u})\,d\Omega$
calcul\'e par \fort{divmas} \`a l'aide des flux de masse aux faces internes
$ m^n_{\,lm}=\sum\limits_{m\in Vois(l)}{(\rho \vect{u})_{\,lm}^n} \text{.}\,
\vect{S}_{\,lm} $ (tableau \var{FLUMAS}) et des flux de masse aux bords  $ m^n_{\,b_{lk}} = \sum\limits_{k\in{\gamma_b(l)}}{(\rho \vect{u})_{\,{b}_{lk}}^n} \text{.}\,
\vect{S}_{\,{b}_{lk}} $ (tableau \var{FLUMAB}).
On incr\'emente ensuite \var{SMBR} et \var{ROVSDT}.
\begin{equation}\notag
\left\{\begin{array}{lll}
&\var{SMBR(IEL)} &= \var{SMBR(IEL)} + \var{ICONV}\  (R^n_{ij})_L\,(\displaystyle
\int_{\Omega_l}\dive\,(\rho\,\vect{u})\ d\Omega) \\
&\var{ROVSDT(IEL)}& = \var{ROVSDT(IEL)} +  \var{ISTAT}\,\displaystyle
\frac{\rho^n_L \ |\Omega_l|}{\Delta t^n} -  \var{ICONV}\ (\displaystyle
\int_{\Omega_l}\dive\,(\rho\,\vect{u})\ d\Omega) \\
\end{array}\right.
\end{equation}
\item Calcul des termes sources de production, des termes $\displaystyle
\phi_{\,ij,1}+\phi_{\,ij,2}$ et de la dissipation~$\displaystyle-\frac{2}{3} \varepsilon\,\delta_{\,ij}$ :

On effectue une boucle d'indice \var{IEL} sur les cellules $\Omega_l$ de centre $L$ :
\begin{itemize}
\item [$\Rightarrow$] $\displaystyle \var{TRPROD}= \frac{1}{2} (\mathcal{P}^n_{ii})_L = \frac{1}{2} \left[ \var{PRODUC(1,IEL)} +  \var{PRODUC(2,IEL)} +  \var{PRODUC(3,IEL)} \right] $
\item [$\Rightarrow$] $\displaystyle \var{TRRIJ }= \frac{1}{2} (R^n_{ii})_L $
\item [$\Rightarrow$] $\displaystyle \var{SMBR(IEL)} =\ \var{SMBR(IEL)}\ +$\\
$\ \displaystyle\rho^n_L |\Omega_l| \left[ \displaystyle
\frac{2}{3}\,\delta_{\,ij} \left( \ \displaystyle \frac{ C_2}{2}\,(\mathcal{P}^n_{ii})_L\ +
(C_1-1)\ \varepsilon^n_L\, \right)\right.$\\
$ + \left.\ (1-C_2) \ \var{PRODUC(ISOU,IEL)} -
\displaystyle C_1\ \frac{2\,\varepsilon^n_L}{(R^n_{ii})_L}\ (R^n_{ij})_L \right]$
\item [$\Rightarrow$] $\displaystyle \var{ROVSDT(IEL)} = \var{ROVSDT(IEL)} +
\rho^n_L \ |\Omega_l| \ (- \displaystyle \frac{1}{3} \ \,\delta_{\,ij} + 1) \ C_1
\ \frac{2\ \varepsilon^n_L}{(R^n_{ii})_L}$
\end{itemize}
\item Appel de \fort{rijech} pour le calcul des termes d'\'echo de paroi
 $\phi^n_{ij,w}$ si $\var{IRIJEC()}=1$ et ajout dans \var{SMBR}.\\
$\var{SMBR} = \var{SMBR} + \phi^n_{ij,w}$\\
Suivant son mode de calcul (\var{ICDPAR}), la distance � la paroi est directement accessible
par \var{RA(IDIPAR+IEL-1)} (\var{|ICDPAR|} = 1) ou bien
est calcul\'ee \`a partir de $\var{IA(IIFAPA(IPHAS)+IEL - 1)}$,
qui donne pour l'\'el\'ement $\var{IEL}$ le num\'ero de la face de bord
paroi la plus  proche (\var{|ICDPAR|} = 2). Ces tableaux ont \'et\'e renseign\'e une fois pour toutes au
d\'ebut de calcul.

\item  Appel de \fort{rijthe} pour le calcul des termes de gravit\'e $\mathcal{G}^n_{ij}$ et ajout dans \var{SMBR}.

Ce calcul n'a lieu que si $\var{IGRARI()} = 1$.
$ \var{SMBR} = \var{SMBR} + \mathcal{G}^n_{ij}$
\item Calcul de la partie explicite du terme de diffusion
 $\dive{\,\left[\tens{A}\,\grad{R}^{\,n}_{ij}\right]}$, plus pr\'ecis\'ement
des contributions du terme extradiagonal pris aux faces purement internes
(remplissage du tableau \var{VISCF}), puis aux faces de bord (remplissage du
tableau \var{VISCB}).
\begin{itemize}
\item [$\star$] Appel de \fort{grdcel} pour le calcul du gradient de
$R^{\,n}_{ij}$ dans chaque direction. Ces gradients sont respectivement
stock\'es dans les tableaux de travail \var{W1}, \var{W2} et \var{W3}.

\item [$\star$] boucle d'indice \var{IEL} sur les cellules $\Omega_l$ de centre
$L$ pour le
calcul de $\tens{E}^n\,\grad{R}^{\,n}_{ij}$ aux cellules dans un premier temps :\\
\begin{itemize}
\item [$\Rightarrow$] $\displaystyle \var{TRRIJ}= \frac{1}{2} (R^{\,n}_{ii})_L $
\item [$\Rightarrow$] $\displaystyle \var{CSTRIJ} = \rho^n_L\ C_S \ \displaystyle\frac{(R^n_{ii})_L}{2\,\varepsilon^n_L}$
\item [$\Rightarrow$] $\displaystyle \var{W4(IEL)} = \rho^n_L\ C_S\
\displaystyle\frac{(R^n_{ii})_L}{2\,\varepsilon^n_L} \left[\,(R^{\,n}_{12})_L \ \var{W2(IEL)} +
(R^{\,n}_{13})_L \ \var{W3(IEL)}\,\right]$
\item [$\Rightarrow$] $\displaystyle \var{W5(IEL)} = \rho^n_L\ C_S\
\displaystyle\frac{(R^n_{ii})_L}{2\,\varepsilon^n_L} \left[\,(R^{\,n}_{12})_L \ \var{W1(IEL)} +
(R^{\,n}_{23})_L \ \var{W3(IEL)}\,\right]$
\item [$\Rightarrow$] $\displaystyle \var{W6(IEL)} = \rho^n_L\ C_S\
\displaystyle\frac{(R^n_{ii})_L}{2\,\varepsilon^n_L} \left[\,(R^{\,n}_{13})_L \ \var{W1(IEL)} + (R^{\,n}_{23})_L \ \var{W2(IEL)}\,\right]$
\end{itemize}



\item [$\star$] Appel de \fort{vectds}\footnote{Le r\'esultat est stock\'e dans
\var{VISCF} et \var{VISCB}. Dans \fort{vectds}, les valeurs aux faces internes
sont interpol\'ees lin\'eairement sans reconstruction et \var{VISCB} est mis \`a
z\'ero.} pour assembler $\displaystyle\left[ \tens{E}^n\,\grad{R}^{\,n}_{ij}
\right]\,.\,\vect{n}_{\,lm}S_{\,lm}$ aux faces $lm$.
\item [$\star$] Appel de \fort{divmas} pour calculer la divergence du flux d\'efini par \var{VISCF} et \var{VISCB}.
Le r\'esultat est stock\'e dans \var{W4}.\\
Ajout au second membre \var{SMBR}.\\
\var{SMBR} = \var{SMBR} + \var{W4}
\end{itemize}

A l'issue de cette \'etape, seule la partie extradiagonale de la diffusion prise
enti\`erement explicite~:
 $$\sum\limits_{m\in
Vois(l)}\left[\ \tens{E}^n\,\grad{R}^{\,n}_{ij} \right]_{\,lm}\,.\,\vect{n}_{\,lm}S_{\,lm}$$ a \'et\'e calcul\'ee.\\

\item Calcul de la partie diagonale du terme de diffusion\footnote{Seule la
composante normale  du  gradient de $R_{ij}$ aux faces sera implicite.} :\\
Comme on l'a d\'eja vu, une partie de cette contribution sera trait\'ee en
implicite (celle relative \`a la composante normale du gradient) et donc
ajout\'ee au second membre par \fort{bilsc2} ; l'autre
partie sera explicite et prise en compte dans $f_s^{\,exp}$.
\begin{itemize}
\item [$\star$] On effectue une boucle d'indice \var{IEL} sur les cellules
$\Omega_l$ de centre $L$ :
\begin{itemize}
\item [$\Rightarrow$] $\displaystyle \var{TRRIJ }= \frac{1}{2} (R^{\,n}_{ii})_L $
\item [$\Rightarrow$] $\displaystyle \var{CSTRIJ} = \rho^n_L \ C_S \ \frac{(R^{\,n}_{ii})_L}{2\,\varepsilon^n_L}$
\item [$\Rightarrow$] $\displaystyle \var{W4(IEL)} = \rho^n_L \ C_S \
\frac{(R^{\,n}_{ii})_L}{2\,\varepsilon^n_L} \ (R^{\,n}_{11})_L$
\item [$\Rightarrow$] $\displaystyle \var{W5(IEL)} = \rho^n_L \ C_S \ \frac{(R^{\,n}_{ii})_L}{2\,\varepsilon^n_L}\ (R^n_{22})_L$
\item [$\Rightarrow$] $\displaystyle \var{W6(IEL)} = \rho^n_L \ C_S \ \frac{(R^{\,n}_{ii})_L}{2\,\varepsilon^n_L} \ (R^n_{33})_L$
\end{itemize}

%\item Traitement du parall\'elisme et de la p\'eriodicit\'e.

\item [$\star$] On effectue une boucle d'indice \var{IFAC} sur les faces
purement internes $lm$ pour remplir le tableau \var{VISCF} :
\begin{itemize}
\item [$\Rightarrow$] Identification des cellules $\Omega_l$ et $\Omega_m$ de
centre respectif $L$ (variable \var{II}) et $M$ (variable \var{JJ}), se trouvant de chaque c\^ot\'e de la face
$lm$\footnote{La normale \`a la face est orient\'ee de L vers M.}.
\item [$\Rightarrow$] Calcul du carr\'e de la surface de la face. La valeur est
stock\'ee dans le tableau \var{SURFN2}.
\item [$\Rightarrow$] Interpolation du gradient de $R^{\,n}_{ij}$ \`a la face
$lm$ (gradient facette $\left[\grad{R}^{\,n}_{ij}\right]_{\,lm}$) :
\begin{equation}\notag
\left\{\begin{array}{ll}
\var{GRDPX} &= \displaystyle \frac{1}{2} \left(\var{W1(II)} + \var{W1(JJ)}
\right) \\
&\\
\var{GRDPY} &= \displaystyle \frac{1}{2} \left(\var{W2(II)} + \var{W2(JJ)} \right) \\
&\\
\var{GRDPZ} &= \displaystyle \frac{1}{2} \left(\var{W3(II)} + \var{W3(JJ)} \right)
\end{array}\right.
\end{equation}
\item [$\Rightarrow$] Calcul du gradient de $R^{\,n}_{ij}$ normal \`a la face
$lm$, $\left[\grad{R}^{\,n}_{ij}\right]_{\,lm}.\vect{n}_{\,lm}\,S_{\,lm}$.\\

$\displaystyle \var{GRDSN} =  \var{GRDPX} \ \var{SURFAC(1,IFAC)} + \var{GRDPY} \ \var{SURFAC(2,IFAC)} +  \var{GRDPZ} \ \var{SURFAC(3,IFAC)}$
$\var{SURFAC}$ est le vecteur surface de la face \var{IFAC}.


\item [$\Rightarrow$] calcul de
 $\left[\grad{R^{\,n}_{ij}} - (\grad
R^{\,n}_{ij}\,.\,\vect{n}_{\,lm})\vect{n}_{\,lm}\right]$, les vecteurs \'etant
calcul\'es \`a la face $lm$ :
\begin{equation}\notag
\left\{\begin{array}{lll}
&\displaystyle \var{GRDPX} &= \var{GRDPX} - \displaystyle\frac{\var{GRDSN}}{\var{SURFN2}} \ \var{SURFAC(1,IFAC)}\\
&&\\
&\displaystyle \var{GRDPY} &= \var{GRDPY} - \displaystyle\frac{\var{GRDSN}}{\var{SURFN2}} \ \var{SURFAC(2,IFAC)} \\
&&\\
&\displaystyle \var{GRDPZ} &= \var{GRDPZ} - \displaystyle\frac{\var{GRDSN}}{\var{SURFN2}} \ \var{SURFAC(3,IFAC)}
\end{array}\right.
\end{equation}
\item [$\Rightarrow$] finalisation du calcul de l'expression totalement
explicite
 $$\left[ \tens{D}^n\,\left( \grad{R^{\,n}_{ij}} - (\grad R^{\,n}_{ij}\,.\,\vect{n}_{\,lm})\,\vect{n}_{\,lm}\right) \right]\,.\,\vect{n}_{\,lm}$$
\begin{equation}\notag
\begin{array} {ll}
\displaystyle \var{VISCF} = &
 \displaystyle\frac{1}{2} (\ \var{W4(II)} +\ \var{W4(JJ)}) \ \var{GRDPX} \
\var{SURFAC(1,IFAC)})\ + \\
&\\
&  \displaystyle\frac{1}{2} (\ \var{W5(II)} +\ \var{W5(JJ)}) \ \var{GRDPY} \
\var{SURFAC(2,IFAC)})\ + \\
&\\
&  \displaystyle\frac{1}{2} (\ \var{W6(II)} +\ \var{W6(JJ)}) \ \var{GRDPZ} \ \var{SURFAC(3,IFAC)})
\end{array}
\end{equation}
\end{itemize}

\item [$\star$] Mise \`a z\'ero du tableau \var{VISCB}.

\item [$\star$] Appel de \fort{divmas} pour calculer la divergence de~:
 $$\tens{D}^{\,n}\,\left( \grad{R^{\,n}_{ij}} - (\grad R^{\,n}_{ij}\,.\,\vect{n}_{\,lm})\vect{n}_{\,lm}\right)$$ d\'efini aux faces dans \var{VISCF} et \var{VISCB}.

Le r\'esultat est stock\'e dans le tableau \var{W1}.\\
Ajout au second membre \var{SMBR}.\\
$\var{SMBR} = \var{SMBR} + \var{W1}$
\end{itemize}
\item Calcul de la viscosit\'e orthotrope $\gamma^n_{\,lm}$ \`a la face $lm$ de la variable principale
$R^{\,n}_{ij}$\\
Ce calcul permet au sous-programme \fort{codits} de compl\'eter le second membre
\var{SMBR} par :
\begin{equation}
\begin{array} {ll}
& \sum\limits_{m\in Vois(l)}
\mu^n_{\,lm}\,\left(\grad{R}^{\,n}_{ij}\,.\,\vect{n}_{\,lm}\right)S_{\,lm}
 + \sum\limits_{m\in Vois(l)} \left(\grad{R}^{\,n}_{ij}
\,.\,\vect{n}_{\,lm}\right)\left[\tens{D}^{\,n}\,\vect{n}_{\,lm}\right]_{\,lm}\,.\,\vect{n}_{\,lm}\
S_{\,lm}\\
& = \sum\limits_{m\in Vois(l)}(\,\mu^n_{\,lm}\, + \,\gamma^n_{\,lm}\,)\,\left(\grad{R}^{\,n}_{ij}\,.\,\vect{n}_{\,lm}\right)S_{\,lm}
\end{array}
\end{equation}
sans pr\'eciser la nature de la face $lm$, {\it via} l'appel \`a \fort{bilsc2}  et de disposer de la quantit\'e
$(\mu^n_{\,lm}\, + \gamma^n_{\,lm})$ pour construire sa
matrice simplifi\'ee.\\
\begin{itemize}
\item [$\star$] On effectue une boucle d'indice \var{IEL} sur les cellules
$\Omega_l$ :
\begin{itemize}
\item [$\Rightarrow$] $\displaystyle \var{TRRIJ }= \frac{1}{2} (R^{\,n}_{ii})_L $
\item [$\Rightarrow$] $\displaystyle \var{RCSTE} = \rho^n_L \ C_S \ \frac{ (R^{\,n}_{ii})_L}{2\,\varepsilon^n_L} $
\item [$\Rightarrow$] $\displaystyle \var{W1(IEL)} = \mu^n +\rho^n_L \ C_S \ \frac{
(R^{\,n}_{ii})_L}{2\,\varepsilon^n_L}\ (R^n_{11})_L$
\item [$\Rightarrow$] $\displaystyle \var{W2(IEL)} = \mu^n + \rho^n_L \ C_S \ \frac{ (R^{\,n}_{ii})_L}{2\,\varepsilon^n_L}\ (R^n_{22})_L$
\item [$\Rightarrow$] $\displaystyle \var{W3(IEL)} = \mu^n + \rho^n_L \ C_S \ \frac{ (R^{\,n}_{ii})_L}{2\,\varepsilon^n_L}\ (R^n_{33})_L$
\end{itemize}

\item [$\star$] Appel de \fort{visort} pour calculer la viscosit\'e orthotrope
\footnote{Comme dans le sous-programme \fort{viscfa}, on multiplie la viscosit\'e par
$\displaystyle \frac{S_{\,lm}}{\overline{L'M'}}$, o\`u $S_{\,lm}$ et
$\overline{L'M'}$ repr\'esentent respectivement la surface de la face $lm$ et la
mesure alg\'ebrique du segment reliant les projections des centres des cellules
voisines sur la normale \`a la face. On garde dans ce sous-programme  la possibilit\'e d'interpoler la viscosit\'e aux cellules lin\'eairement ou d'utiliser une moyenne harmonique. La viscosit\'e au bord est celle de la cellule de bord correspondante.}
$ \gamma^n_{\,lm} = (\tens{D}^{\,n}\,\vect{n}_{\,lm}).\vect{n}_{\,lm}$ aux faces $lm$

Le r\'esultat est stock\'e dans les tableaux \var{VISCF} et \var{VISCB}.
\end{itemize}

\item appel de \fort{codits} pour la r\'esolution de l'\'equation de
convection/diffusion/termes sources de la variable $R_{ij}$. Le terme source est
\var{SMBR}, la viscosit\'e \var{VISCF} aux faces purement internes (
resp. \var{VISCB} aux faces de bord) et \var{FLUMAS} le flux de masse interne
 ( resp. \var{FLUMAB} flux de masse au bord). Le r\'esultat est la variable $R_{ij}$ au temps
$n+1$, donc $R^{\,n+1}_{ij}$. Elle est stock\'ee dans le tableau \var{RTP} des
variables mises \`a jour.

\end{itemize}

\etape{Appel de \fort{reseps} pour la r\'esolution de la variable $\varepsilon$}

On d\'ecrit ci-dessous le sous-programme \fort{reseps}, les commentaires du sous-programme \fort{resrij} ne seront pas r\'ep\'et\'es puisque les deux sous-programmes ne diff\`erent que par leurs termes sources.

\begin{itemize}
\item Initialisation \`a z\'ero de \var{SMBR} et \var{ROVSDT}.

\item{Lecture et prise en compte des termes sources utilisateur pour la variable $\varepsilon$ :}

Appel de \fort{ustsri} pour charger les termes sources utilisateurs. Ils sont
stock\'es dans les tableaux suivants :\\
pour la cellule $\Omega_l$ repr\'esent\'ee par $\var{IEL}$ de centre $L$, on a :
\begin{equation}\notag
\left\{\begin{array}{lll}
&\var{ROVSDT(IEL)}&= |\Omega_l| \ \alpha_{\varepsilon}\\
&\var{SMBR(IEL)}&=|\Omega_l| \ \beta_{\varepsilon}\\
\end{array}\right.
\end{equation}
On affecte alors les valeurs ad\'equates au second membre \var{SMBR} et \`a la
diagonale \var{ROVSDT} comme suit :
\begin{equation}\notag
\left\{\begin{array}{lll}
&\var{SMBR(IEL)} &= \var{SMBR(IEL)} +\ |\Omega_l| \ \alpha_{\,\varepsilon} \
\varepsilon^n_L \\
&\var{ROVSDT(IEL)}&= \text{max }(-\ |\Omega_l| \ \alpha_{\,\varepsilon},0)\\
\end{array}\right.
\end{equation}

\item{Calcul du terme source de masse si $\Gamma_L > 0$ :
\begin{equation}\notag
\left\{\begin{array}{lll}
&\displaystyle \var{SMBR(IEL)} = \var{SMBR(IEL)} + |\Omega_l| \ \Gamma_L \
(\varepsilon^{\,in}_L -\varepsilon^n_L) \\
&\displaystyle \var{ROVSDT(IEL)}= \var{ROVSDT(IEL)} + |\Omega_l| \ \Gamma_L
\end{array}\right.
\end{equation}
\item Calcul du terme d'accumulation de masse et du terme instationnaire \\
On stocke $\displaystyle \var{W1}= \int_{\Omega_l}\dive\,(\rho\,\vect{u})\,d\Omega$
calcul\'e par \fort{divmas} \`a l'aide des flux de masse internes et aux bords.\\
On incr\'emente ensuite \var{SMBR} et \var{ROVSDT}.
\begin{equation}\notag
\left\{\begin{array}{lll}
&\var{SMBR(IEL)} &= \var{SMBR(IEL)} + \var{ICONV}\ \varepsilon^n_L\,(\displaystyle
\int_{\Omega_l}\dive\,(\rho\,\vect{u})\ d\Omega) \\
&\var{ROVSDT(IEL)}& = \var{ROVSDT(IEL)} +  \var{ISTAT}\,\displaystyle
\frac{\rho^n_L \ |\Omega_l|}{\Delta t^n} -  \var{ICONV}\ (\displaystyle
\int_{\Omega_l}\dive\,(\rho\,\vect{u})\ d\Omega) \\
\end{array}\right.
\end{equation}

\item Traitement du terme de production
 $\displaystyle \rho\,C_{\varepsilon_1}\,\frac{\varepsilon}{k}\,\mathcal{P}$
 et du terme de dissipation $-\,\displaystyle \rho\,C_{\varepsilon_2}\,\frac{\varepsilon}{k}\,\varepsilon$ \\
pour cela, on effectue une boucle d'indice \var{IEL} sur les cellules $\Omega_l$
de centre $L$ :
\begin{itemize}
\item [$\Rightarrow$] $\displaystyle \var{TRPROD}= \frac{1}{2} (\mathcal{P}^n_{ii})_L = \frac{1}{2} \left[ \var{PRODUC(1,IEL)} +  \var{PRODUC(2,IEL)} +  \var{PRODUC(3,IEL)} \right] $
\item [$\Rightarrow$] $\displaystyle \var{TRRIJ }= \frac{1}{2} (R^n_{ii})_L $
\item [$\Rightarrow$] $\displaystyle \var{SMBR(IEL)} = \var{SMBR(IEL)} + \rho^n_L
|\Omega_l| \left[ -C_{\varepsilon_2} \ \frac{2\,(\varepsilon^n_L)^2}{(R^n_{ii})_L} + C_{\varepsilon_1} \ \frac{\varepsilon^n_L}{(R^n_{ii})_L}\ (\mathcal{P}^n_{ii})_L \right] $
\item [$\Rightarrow$] $\displaystyle \var{ROVSDT(IEL)} = \var{ROVSDT(IEL)} + C_{\varepsilon_2} \ \rho^n_L \ |\Omega_l| \ \frac{2\,\varepsilon^n_L}{(R^n_{ii})_L}$
\end{itemize}

\item Appel de \fort{rijthe} pour le calcul des termes de gravit\'e $\mathcal{G}^n_{\varepsilon}$ et ajout dans \var{SMBR}.

$ \var{SMBR} = \var{SMBR} + \mathcal{G}^n_{\varepsilon}$\\
Ce calcul n'a lieu que si $\var{IGRARI()} = 1$.

\item Calcul de la diffusion de $\varepsilon$ \\
 Le terme $\dive \left[\mu\, \grad(\varepsilon) + \tens{A'}\,\grad(\varepsilon)
\right]$ est calcul\'e exactement de la m\^eme mani\`ere que pour les tensions
de Reynolds $R_{ij}$ en rempla\c cant $\tens{A}$ par $\tens{A'}$.

\item Appel de \fort{codits} pour la r\'esolution de l'\'equation de
convection/diffusion/termes sources de la variable principale $\varepsilon$. Le
r\'esultat $\varepsilon^{\,n+1}$ est stock\'e dans le tableau \var{RTP} des
variables mises \`a jour.
}
\end{itemize}

\etape{clippings finaux}
On passe enfin dans le sous-programme  \fort{clprij} pour faire un clipping \'eventuel
des variables $R^{\,n+1}_{ij}$ et $\varepsilon^{\,n+1}$. Le sous-programme
\fort{clprij} est appel\'e\footnote{L'option
$\var{ICLIP} = 1$ consiste en un clipping minimal des variables $R_{ii}$ et
$\varepsilon$ en prenant la valeur absolue de ces variables puisqu'elles ne
peuvent \^etre que positives.} avec $\var{ICLIP} = 2$ . Cette option
\footnote{Quand la valeur des grandeurs $R_{ii}$ ou $\varepsilon$ est
n\'egative, on la remplace par le minimum entre sa valeur absolue et (1,1)
fois la valeur obtenue au pas de temps pr\'ec\'edent.} contient l'option $\var{ICLIP} = 1$  et permet de v\'erifier l'in\'egalit\'e de Cauchy-Schwarz sur les grandeurs extra-diagonales du tenseur $\tens{R}$ (pour $i \neq j$, $|R_{ij}|^2 \le R_{ii} R_{jj}$).


%%%%%%%%%%%%%%%%%%%%%%%%%%%%%%%%%%
%%%%%%%%%%%%%%%%%%%%%%%%%%%%%%%%%%
\section{Points \`a traiter}
%%%%%%%%%%%%%%%%%%%%%%%%%%%%%%%%%%
%%%%%%%%%%%%%%%%%%%%%%%%%%%%%%%%%%
Sauf mention explicite, $\phi$ repr\'esentera une tension de Reynolds ou la dissipation turbulente ($\phi = R_{ij} \ \text{ou} \ \varepsilon$).

\begin{itemize}
\item {La vitesse utilis\'ee pour le calcul de la production est explicite. Est-ce qu'une implicitation peut am\'eliorer la pr\'ecision temporelle de $\phi$ \footnote{Cette remarque peut \^etre g\'en\'eralis\'ee. En effet, peut-on envisager d'actualiser les variables d\'ej\`a r\'esolues (sans r\'eactualiser les variables turbulentes apr\`es leur r\'esolution)? Ceci obligerait \`a modifier les sous-programmes tels que \fort{condli} qui sont appel\'es au d\'ebut de la boucle en temps.} ?}
\item {Dans quelle mesure le terme d'\'echo de paroi est-il valide ? En effet, ce terme est remis en question par certains auteurs.}
\item {On peut envisager la r\'esolution d'un syst\`eme hyperbolique pour les
tensions de Reynolds afin d'introduire un couplage avec le champ de vitesse.}
\item {Le flux au bord \var{VISCB} est annul\'e dans le sous-programme
\fort{vectds}. Peut-on envisager de mettre au bord la valeur de la variable
concern\'ee \`a la cellule de bord correspondant? De m\^eme, il faudrait se
pencher sur les hypoth\`eses sous-jacentes \`a l'annulation des contributions
aux bords de \var{VISCB} lors du calcul de : $$\left[ \tens{D}^n\,\left( \grad{R^{\,n}_{ij}} - (\grad R^{\,n}_{ij}\,.\,\vect{n}_{\,lm})\,\vect{n}_{\,lm}\right) \right]\,.\,\vect{n}_{\,lm}.$$}
\item {Un probl\`eme de pond\'eration appara\^\i t plus g\'en\'eralement. Si on prend la partie explicite de $\tens{D}\,\grad(\phi)$, la pond\'eration aux faces internes utilise le coefficient $\displaystyle\frac{1}{2}$ avec pond\'eration s\'epar\'ee de $\tens{D}$ et $\grad(\phi)$, alors que pour $\tens{E}\,\grad(\phi)$, on calcule d'abord ce terme aux cellules pour ensuite l'interpoler lin\'eairement aux faces \footnote{Cette interpolation se fait dans \fort{vectds} avec des coefficients de pond\'eration aux faces.}. Ceci donne donc deux types d'interpolations pour des termes de m\^eme nature.}
\item {On laisse la possibilit\'e dans \fort{visort} d'utiliser une moyenne
harmonique aux faces. Est-ce que ceci est valable puisque les interpolations
utilis\'ees lors du calcul de la partie explicite de $\tens{A}\,\grad{\phi}$
sont des moyennes arithm\'etiques ?}
\item {Les techniques adopt\'ees lors du clipping sont \`a revoir.}
\item {On utilise dans le cadre du mod\`ele $\displaystyle R_{ij}-\varepsilon $ une semi-implicitation de termes comme $\displaystyle \phi_{ij,1}$ ou $\displaystyle -\rho\,C_{\varepsilon_2}\,\frac{\varepsilon}{k}\,\varepsilon$. On peut envisager le m\^eme type d'implicitation dans \fort{turbke} m\^eme en pr\'esence du couplage $\displaystyle k-\varepsilon$.}
\item L'adoption d'une r\'esolution d\'ecoupl\'ee fait perdre l'invariance par rotation.
\item La formulation et l'implantation des conditions aux limites de paroi
devront \^etre v\'erifi\'ees. En effet, il semblerait que, dans certains cas, des ph\'enom\`enes
``oscillatoires'' apparaissent, sans qu'il soit ais\'e d'en d\'eterminer la cause.
\item L'implicitation partielle (du fait de la r\'esolution d\'ecoupl\'ee) des
conditions aux limites conduit souvent \`a des calculs instables. Il
conviendrait d'en conna\^\i tre la raison. L'implicitation partielle avait
\'et\'e mise en \oe uvre afin de tenter d'utiliser un pas de temps plus grand
dans le cas de jets axisym\'etriques en particulier.

\end{itemize}

%                      Code_Saturne version 1.3
%                      ------------------------
%
%     This file is part of the Code_Saturne Kernel, element of the
%     Code_Saturne CFD tool.
%
%     Copyright (C) 1998-2007 EDF S.A., France
%
%     contact: saturne-support@edf.fr
%
%     The Code_Saturne Kernel is free software; you can redistribute it
%     and/or modify it under the terms of the GNU General Public License
%     as published by the Free Software Foundation; either version 2 of
%     the License, or (at your option) any later version.
%
%     The Code_Saturne Kernel is distributed in the hope that it will be
%     useful, but WITHOUT ANY WARRANTY; without even the implied warranty
%     of MERCHANTABILITY or FITNESS FOR A PARTICULAR PURPOSE.  See the
%     GNU General Public License for more details.
%
%     You should have received a copy of the GNU General Public License
%     along with the Code_Saturne Kernel; if not, write to the
%     Free Software Foundation, Inc.,
%     51 Franklin St, Fifth Floor,
%     Boston, MA  02110-1301  USA
%
%-----------------------------------------------------------------------
%
\programme{vortex}
%
\vspace{1cm}
%%%%%%%%%%%%%%%%%%%%%%%%%%%%%%%%%%
%%%%%%%%%%%%%%%%%%%%%%%%%%%%%%%%%%
\section{Fonction}
%%%%%%%%%%%%%%%%%%%%%%%%%%%%%%%%%%
%%%%%%%%%%%%%%%%%%%%%%%%%%%%%%%%%%
Ce sous-programme est d�di� � la g�n�ration des conditions d'entr�e
turbulente utilis�es en LES.


La m�thode des vortex est bas�e sur une approche de tourbillons
ponctuels. L'id�e de la m�thode consiste � injecter des tourbillons 2D dans le
plan d'entr�e du calcul, puis � calculer le champ de vitesse induit par ces
tourbillons au centre des faces d'entr�e.

%                      Code_Saturne version 1.3
%                      ------------------------
%
%     This file is part of the Code_Saturne Kernel, element of the
%     Code_Saturne CFD tool.
% 
%     Copyright (C) 1998-2007 EDF S.A., France
%
%     contact: saturne-support@edf.fr
% 
%     The Code_Saturne Kernel is free software; you can redistribute it
%     and/or modify it under the terms of the GNU General Public License
%     as published by the Free Software Foundation; either version 2 of
%     the License, or (at your option) any later version.
% 
%     The Code_Saturne Kernel is distributed in the hope that it will be
%     useful, but WITHOUT ANY WARRANTY; without even the implied warranty
%     of MERCHANTABILITY or FITNESS FOR A PARTICULAR PURPOSE.  See the
%     GNU General Public License for more details.
% 
%     You should have received a copy of the GNU General Public License
%     along with the Code_Saturne Kernel; if not, write to the
%     Free Software Foundation, Inc.,
%     51 Franklin St, Fifth Floor,
%     Boston, MA  02110-1301  USA
%
%-----------------------------------------------------------------------
%
%%%%%%%%%%%%%%%%%%%%%%%%%%%%%%%%%%
%%%%%%%%%%%%%%%%%%%%%%%%%%%%%%%%%%
\section{Discr\'etisation}
%%%%%%%%%%%%%%%%%%%%%%%%%%%%%%%%%%
%%%%%%%%%%%%%%%%%%%%%%%%%%%%%%%%%%

Le terme convectif en $\dive(\underline{u} \otimes \rho\,\underline{u})$
introduit une non lin\'earit\'e et un couplage des composantes de la vitesse
$\vect{u}$ dans l'�quation (\ref{Base_Preduv_eqqdm}). Une lin\'earisation et un d\'ecouplage
des trois composantes de la 
vitesse sont r\'ealis\'es lors de la discr\'etisation de cette \'etape de
pr\'ediction.\\
En effet, soit :
\begin{equation}
\vect{\widetilde{u}}= \vect{u}^n + \delta \vect{u} 
\end{equation}
La contribution exacte du terme convectif \`a prendre en compte dans cette
\'etape de pr\'ediction serait :\\
\begin{equation}\label{Base_Preduv_Conv_exact}
\begin{array}{ll}
\dive(\vect{\widetilde{u}} \otimes \rho\,\vect{\widetilde{u}}) =
\dive(\vect{u}^{n} \otimes \rho\,\vect{u}^{n}) + \dive(\delta \vect{u} \otimes
\rho\,\vect{u}^{n}) +  \underbrace { \dive(\vect{u}^{n} \otimes
\rho\,\delta \vect{u})}_{\text {terme couplant lin\'eaire}} +  \underbrace { \dive(\delta \vect{u} \otimes
\rho\,\delta \vect{u})}_{\text {terme couplant et non lin\'eaire}}\\
\end{array} 
\end{equation}
Les deux derniers termes de l'expression (\ref{Base_Preduv_Conv_exact}) sont {\em a priori} n�glig�s
de mani�re � obtenir un syst\`eme en vitesse qui soit d\'ecoupl\'e et donc,
�viter l'inversion d'une matrice pouvant \^etre de tr\`es grande taille. Ces
deux termes peuvent n�anmoins �tre pris en compte de mani�re plus ou moins
approch�e par extrapolation explicite du flux de masse en $n+\theta_F$ (pour le
terme couplant lin�aire provenant de la convection de $\vect{u}^{n}$ par $\delta
\vect{u}$) et utilisation d'un point-fixe par sous it�ration sur le sous
programme \fort{navsto} (pour le terme non-lin�aire, en sp�cifiant $\var{NTERUP}>1$).

L'�quation (\ref{Base_Preduv_eqqdm}) est discr�tis�e au temps $n+\theta$ � l'aide d'un
$\theta$-sch�ma, et le tenseur des pertes de charges d�compos� en une partie
diagonale $\tens{K}_{d}$ et une extradiagonale $\tens{K}_{e}$ (soit
 $\tens{K}_{pdc}=\tens{K}_{d}+\tens{K}_{e}$).\\
$\bullet$ La pression est suppos�e connue en $n-1+\theta$ (d�calage temporel
pression-vitesse) et le gradient naturellement calcul� � cet instant.\\ 
$\bullet$ Les termes sources de viscosit� secondaire, de gradient transpos\'e,
ceux provenant du mod�le de turbulence\footnote{except� $\dive (\mu_t\ (\ggrad
\underline {u}))$}, $\rho\,\tens{K}_{\,e}\ \underline{u}$, $(\rho -\rho_0)
\underline {g}$ ainsi que $\underline{T}_{s}^{\,exp}$ et
$\Gamma\,\underline{u}_{\,i}$ sont pris de mani�re explicite au temps $n$, ou
extrapol�s suivant le sch�ma en temps choisi pour les propri�t�s physique et les
termes sources.\\ 
$\bullet$ Les termes sources $\underline{u}\,\,\dive (\rho\,\underline {u})$,
$\Gamma\,\,\underline{u}$, $T_{s}^{\,imp}\,\,\underline{u}$ et
$-\rho\,\tens{K}_{\,d}\,\,\underline{u}$ sont implicit�s est calcul�s �
l'instant $n+\theta$.\\ 
$\bullet$ Le terme de diffusion $\dive (\mu_{\,tot}\,\ggrad \underline{u})$ est
implicit� : la vitesse est prise � l'instant $n+\theta$ et la viscosit�
explicit�e ou extrapol�e.\\ 
$\bullet$ Enfin, le terme de convection en $\dive(\,\underline{u} \otimes
(\rho\underline{u})\,)$ est implicit� : la composante r�solue de la vitesse est
prise en $n+\theta$, et le flux de masse, explicit�, ou extrapol� en
$n+\theta_F$. 

Par souci de clart�, on suppose, en l'absence d'indication, que les propri�tes
physiques $\Phi$ ($\rho,\,\mu_{tot},\,...$) et le flux de masse
$(\rho\underline{u})$ sont pris respectivement aux instants $n+\theta_\Phi$ et
$n+\theta_F$, o� $\theta_\Phi$ et $\theta_F$ d�pendent des sch�mas en temps
sp�cifiquement utilis�s pour ces grandeurs\footnote{cf. \fort{introd}}. 

La discr�tisation temporelle de l'�quation (\ref{Base_Preduv_eqqdm}) s'�crit alors comme suit : 

\begin{equation}\label{Base_Preduv_eq_di1}
 \begin{array}{c}
\displaystyle \rho\,\ \frac{ \underline {\widetilde{u}}^{n+1} -\underline {u}^{n} }
{\Delta t} + \dive(\,\underline{\widetilde{u}}^{n+\theta} \otimes (\rho\underline{u})\,) -\dive
(\mu_{\,tot}\,\ggrad \underline{\widetilde{u}}^{n+\theta}) =
\\
\displaystyle
 - \grad p^{n-1+\theta} + \dive (\rho\,\underline {u})\,\underline{\widetilde{u}}^{n+\theta} +(\Gamma\,\underline{u}_{\,i})^{n+\theta_S}-\Gamma^n\,\,\underline{\widetilde{u}}^{n+\theta}
\\
\begin{array}{c}
\displaystyle
- \rho\,\tens{K}_{\,d}^{n}\,\,\underline{\widetilde{u}}^{n+\theta} - (\rho\,\tens{K}_{\,e}\ \underline{u})^{n+\theta_S} + (\underline{T}_{s}^{\,exp})^{\,n+\theta_S} + T_{s}^{\,imp}\,\,\underline{\widetilde{u}}^{n+\theta}
\\
\displaystyle
+[\dive (\mu_{\,tot}\,^t\ggrad \underline {u})]^{n+\theta_S}-\frac {2} {3}[\,\grad (\mu_{\,tot}\,\dive \underline {u})]^{n+\theta_S} + (\rho -\rho_0) \underline {g}
 - (\underline{turb})^{n+\theta_S}
\end{array}
\end{array}
\end{equation}
o\`u, par souci de simplification, on a pos\'e :
\begin{equation}
\mu_{\,tot}=
\begin{cases}
\mu+\mu_t & \text{pour les mod�les � viscosit� turbulente ou en LES}, \\
\mu & \text{pour les mod�les au second ordre ou en laminaire}
\end{cases} \ 
\end{equation}
\\
et :
\begin{equation}
\underline{turb}^{n}=
\begin{cases}
\displaystyle\frac {2}{3}\grad (\rho^{n}\,k^{n}) & \text{pour les mod�les � viscosit� turbulente}, \\
\dive(\rho^{n}\,\tens{R}^n) & \text{pour les mod�les au second ordre},\\
0 & \text{en laminaire ou en LES}\\
\end{cases}
\end{equation}
Par analogie avec l'�criture du $\theta$-sch�ma pour une variable scalaire, $\,
\underline {\widetilde{u}}^{n+\theta}$ est interpol�e � partir de la vitesse
pr�dite $\underline {\widetilde{u}}^{n+1}$ de la mani\`ere suivante\footnote{si
$\theta=1/2$, ou qu'une extrapolation est utilis�e, l'ordre 2 n'est obtenu que si
le pas de temps $\Delta t$ est uniforme en temps et en espace.}~: 
\begin{equation}
\underline {\widetilde{u}}^{n+\theta}=\theta\, \underline
{\widetilde{u}}^{n+1}+(1-\theta)\, \underline {u}^{n}\\ 
\end{equation}
Avec :
\begin{equation}
\left\{
\begin{array}{ll}
\theta = 1   & \text{Pour un sch\'ema de type Euler implicite d'ordre 1.}\\
\theta = 1/2 & \text{Pour un sch\'ema de type Cranck-Nicolson d'ordre 2.}\\
\end{array}
\right.
\end{equation}

L'�quation (\ref{Base_Preduv_eq_di1}) est alors r��crite sous la forme :

\begin{equation}\label{Base_Preduv_eq_di2}
\begin{array}{c}
\displaystyle \underbrace{\left(\frac{\rho}{\Delta t} -\theta \,\dive (\rho\,\underline {u}) +\theta \,\, \Gamma^n +
\theta \,\, \rho\,\tens{K}_{\,d}^n-\theta \,T_s^{\,imp} \right)}_{\displaystyle f_s^{imp}}\, (\underline {\,\widetilde{u}}^{n+1} -\underline {u}^{n})
\\
 +\, \theta\, \dive(\underline {\widetilde{u}}^{n+1} \otimes (\rho\underline{u}))-\, \theta\,\dive (\mu_{\,tot}\,\ggrad \underline {\widetilde{u}}^{n+1}) =
\\
-\,(1-\theta)\, \dive(\underline {u}^{n} \otimes (\rho\underline{u})) +\,(1-\theta)\,\dive (\mu_{\,tot}\,\ggrad \underline {u}^{n})
\\
f_s^{exp}\left\{
\begin{array}{c}
\displaystyle 
- \grad p^{n-1+\theta} + \dive (\rho\,\underline {u})\,\underline{u}^{n} +\,(\,\Gamma^{n}\,\underline{u}_{\,i}\,)^{n+\theta_S}- \Gamma^n\,\,\underline{u}^{n}
\\
\displaystyle
-(\,\rho\,\tens{K}_{\,e}\ \underline{u}\,)^{n+\theta_S} -\rho\,\tens{K}_{\,d}^n\ \underline{u}^{n}+ (\underline{T}_{s}^{\,exp})^{\,n+\theta_S} + T_s^{\,imp}\,\,\underline {u}^{n} 
\\
\displaystyle
+[\dive (\mu_{\,tot}\,^t\ggrad \underline {u}\,)]^{n+\theta_S}-\frac {2} {3}[\,\grad (\mu_{\,tot}\,\dive \underline {u}\,)]^{n+\theta_S} + (\rho -\rho_0) \underline {g}-(\underline{turb})^{n+\theta_S}
\end{array}
\right.
\end{array}
\end{equation}

d'o� l'�quation r�solue par le sous-programme \fort{codits} :
\begin{equation}\begin{array}{c}
\displaystyle
f_s^{\,imp}(\underline {\widetilde{u}}^{n+1}-\underline {u}^{n}) + \theta\, \dive(\underline{\widetilde{u}}^{n+1} \otimes (\rho
\underline{u})) - \theta\,\dive (\,\mu_{\,tot}\,\ggrad \underline{\widetilde{u}}^{n+1}) = 
\\\\
\displaystyle
-(1-\theta)\,\dive(\underline{u}^{n} \otimes (\rho \underline{u}))+(1-\theta)\,\dive (\,\mu_{\,tot}\,\ggrad \underline{u}^{n})
+ \underline{f}_{\,s}^{\,exp}
\end{array}
\end{equation}
La m\'ethode de discr\'etisation spatiale est d\'evelopp\'ee dans le sous-programme \fort{codits}.\\



\minititre{Remarques :}
{\tiny$\blacksquare$} Dans le cas standard sans extrapolation, le terme
$-\,T_s^{\,imp}$ n'est ajout� � $f_s^{\,imp}$ que s'il est positif afin de ne
pas affaiblir la dominance de la diagonale de la matrice � inverser.\\ 
{\tiny$\blacksquare$} Si une extrapolation est utilis�e, par contre,
$\,T_s^{\,imp}$ est ajout� � $f_s^{\,imp}$ quel que soit son signe. En effet, l'id�e intuitive qui
consiste � prendre~: 
\begin{equation}
\begin{cases}
\displaystyle
(\underline{T}_{s}^{\,exp} + T_{s}^{\,imp}\,\underline {u})^{\,n+\theta_S} &
\text{si } T_{s}^{\,imp} > 0\\ 
\displaystyle
(\underline{T}_{s}^{\,exp})^{\,n+\theta_S} + T_{s}^{\,imp}\,\underline{u}^{n+\theta} &\text{sinon}\\
\end{cases}
\end{equation} 
aboutit � une incoh�rence dans le traitement si $T_s^{imp}$ change de signe
entre deux pas de temps.\\ 
{\tiny$\blacksquare$} la partie diagonale $\tens{K}_{\,d}$ du terme
de perte de charge est utilis�e dans $f_s^{\,imp}$. En fait, pour \^etre rigoureux,
il faudrait ne retenir que les contributions positives (point signal\'e dans le
sous-programme utilisateur associ\'e \fort{uskpdc}). Cette prise en compte sera \`a am\'eliorer.\\
{\tiny$\blacksquare$} Le terme $\theta\,\Gamma^{n}-\theta\,\dive
(\rho\,\underline {u})$ ne pose pas de probl�me pour la 
dominance de la diagonale de la matrice car il est exactement compens� par le
terme de convection (cf. \fort{covofi}). 


%                      Code_Saturne version 1.3
%                      ------------------------
%
%     This file is part of the Code_Saturne Kernel, element of the
%     Code_Saturne CFD tool.
%
%     Copyright (C) 1998-2007 EDF S.A., France
%
%     contact: saturne-support@edf.fr
%
%     The Code_Saturne Kernel is free software; you can redistribute it
%     and/or modify it under the terms of the GNU General Public License
%     as published by the Free Software Foundation; either version 2 of
%     the License, or (at your option) any later version.
%
%     The Code_Saturne Kernel is distributed in the hope that it will be
%     useful, but WITHOUT ANY WARRANTY; without even the implied warranty
%     of MERCHANTABILITY or FITNESS FOR A PARTICULAR PURPOSE.  See the
%     GNU General Public License for more details.
%
%     You should have received a copy of the GNU General Public License
%     along with the Code_Saturne Kernel; if not, write to the
%     Free Software Foundation, Inc.,
%     51 Franklin St, Fifth Floor,
%     Boston, MA  02110-1301  USA
%
%-----------------------------------------------------------------------
%

%%%%%%%%%%%%%%%%%%%%%%%%%%%%%%%%%%
%%%%%%%%%%%%%%%%%%%%%%%%%%%%%%%%%%
\section{Mise en \oe uvre}
%%%%%%%%%%%%%%%%%%%%%%%%%%%%%%%%%%
%%%%%%%%%%%%%%%%%%%%%%%%%%%%%%%%%%
La num\'ero de la phase trait\'ee fait partie des arguments de \fort{turrij}. On
omettra volontairement de le pr\'eciser dans ce qui suit, on indiquera par $(\ )$ la
notion de tableau s'y rattachant.

\etape{Calcul des termes de production $\tens{\mathcal{P}}$}
\begin{itemize}
\item [$\star$] Initialisation \`a z\'ero du tableau \var{PRODUC} dimensionn\'e \`a $\var{NCEL}\times 6$.
\item [$\star$] On appelle trois fois \fort{grdcel} pour calculer les gradients des composantes de la vitesse $u$, $v$ et
$w$ prises au temps $n$.

Au final, on a :\\
$\displaystyle
\begin{array} {ll}
\var{PRODUC(1,IEL)} = & \displaystyle - 2 \left[ R_{11}^{\,n} \frac{\partial u^{\,n}} {\partial x} +R_{12}^{\,n} \frac{\partial u^{\,n}} {\partial y}+R_{13}^{\,n} \frac{\partial u^{\,n}} {\partial z} \right] \text{        (production de $R_{11}^{\,n}$)}\\
\var{PRODUC(2,IEL)} = & \displaystyle - 2 \left[ R_{12}^{\,n} \frac{\partial v^{\,n}} {\partial x} +R_{22}^{\,n} \frac{\partial v^{\,n}} {\partial y}+R_{23}^{\,n} \frac{\partial v^{\,n}} {\partial z} \right] \text{        (production de $R_{22}^{\,n}$)}\\
\var{PRODUC(3,IEL)} = & \displaystyle - 2 \left[ R_{13}^{\,n} \frac{\partial w^{\,n}} {\partial x} +R_{23}^{\,n} \frac{\partial w^{\,n}} {\partial y}+R_{33}^{\,n} \frac{\partial w^{\,n}} {\partial z} \right] \text{        (production de $R_{33}^{\,n}$)}\\
\var{PRODUC(4,IEL)} = & \displaystyle - \left[ R_{12}^{\,n} \frac{\partial u^{\,n}} {\partial x} +R_{22}^{\,n} \frac{\partial u^{\,n}} {\partial y}+R_{23}^{\,n} \frac{\partial u^{\,n}} {\partial z} \right] \\
& \displaystyle - \left[ R_{11}^{\,n} \frac{\partial v^{\,n}} {\partial x} +R_{12}^{\,n} \frac{\partial v^{\,n}} {\partial y}+R_{13}^{\,n} \frac{\partial v^{\,n}} {\partial z} \right] \text{        (production de $R_{12}^{\,n}$)} \\
\var{PRODUC(5,IEL)} = & \displaystyle - \left[ R_{13}^{\,n} \frac{\partial u^{\,n}} {\partial x} +R_{23}^{\,n} \frac{\partial u^{\,n}} {\partial y}+R_{33}^{\,n} \frac{\partial u^{\,n}} {\partial z} \right] \\
& \displaystyle - \left[ R_{11}^{\,n} \frac{\partial w^{\,n}} {\partial x} +R_{12}^{\,n} \frac{\partial w^{\,n}} {\partial y}+R_{13}^{\,n} \frac{\partial w^{\,n}} {\partial z} \right] \text{        (production de $R_{13}^{\,n}$)} \\
\var{PRODUC(6,IEL)} = & \displaystyle - \left[ R_{13}^{\,n} \frac{\partial v^{\,n}} {\partial x} +R_{23}^{\,n} \frac{\partial v^{\,n}} {\partial y}+R_{33}^{\,n} \frac{\partial v^{\,n}} {\partial z} \right] \\
& \displaystyle - \left[ R_{12}^{\,n} \frac{\partial w^{\,n}} {\partial x} +R_{22}^{\,n} \frac{\partial w^{\,n}} {\partial y}+R_{23}^{\,n} \frac{\partial w^{\,n}} {\partial z} \right]  \text{        (production de $R_{23}^{\,n}$)}
\end{array}
$
\end{itemize}

\etape{Calcul du gradient de la masse volumique $\rho^n$ prise au d\'ebut du pas
de temps courant\footnote{{\it i.e.} calcul\'ee \`a partir des
variables du pas de temps pr\'ec\'edent $n$ si n\'ecessaire.} $(n+1)$}
Ce calcul n'a lieu que si les termes de gravit\'e doivent \^etre pris en compte
($\var{IGRARI()} =1$).
\begin{itemize}
\item [$\star$] Appel de \fort{grdcel}  pour calculer le gradient de $\rho^n$
dans les trois directions de l'espace. Les conditions aux limites sur $\rho^n$
sont des conditions de Dirichlet puisque la valeur de $\rho^n$ aux faces de bord
$ik$ (variable \var{IFAC}) est connue et vaut $\rho_{\,b_{\,ik}}$. Pour \'ecrire les conditions aux limites
sous la forme habituelle, $$\rho_{\,b_{\,ik}} = \var{COEFA} + \var{COEFB}
\,\rho^n_{\,I'}$$ on pose alors $\var{COEFA}=
\var{PROPCE(IFAC,IPPROB(IROM(IPHAS)))}$ et $\var{COEFB} = \var{VISCB} = 0$.\\
$\var{PROPCE(1,IPPROB(IROM(IPHAS)))}$ (resp.$\var{VISCB}$) est utilis\'e en lieu
et place de l'habituel \var{COEFA} ($\var{COEFB}$), lors de l'appel \`a \fort{grdcel}.\\
On a donc :\\
$\displaystyle \var{GRAROX}= \frac{\partial \rho^n}{\partial x}\ $,$\displaystyle \ \var{GRAROY}= \frac{\partial
\rho^n}{\partial y}$ et $
\displaystyle \ \var{GRAROZ}= \frac{\partial \rho^n}{\partial z}\ $.

\end{itemize}

Le gradient de $\rho^n$ servira \`a calculer les termes de production par effets de gravit\'e si ces derniers sont pris en compte.

\etape{Boucle \var{ISOU} de $1$ \`a $6$ sur les tensions de Reynolds}
Pour $\var{ISOU} = 1,2,3,4,5,6$, on r\'esout respectivement et dans
l'ordre  les
\'equations de $R_{11}$, $R_{22}$, $R_{33}$, $R_{12}$, $R_{13}$ et $R_{23}$ par
l'appel au sous-programme \fort{resrij}.\\
La r\'esolution se fait par incr\'ement $\delta {R}_{ij}^{\,n+1,k+1}$ , en utilisant la m\^eme m\'ethode que
celle d\'ecrite dans le sous-programme \fort{codits}. On adopte ici les m\^emes notations.
\var{SMBR} est le second membre du syst\`eme \`a inverser, syst\`eme portant sur
les incr\'ements de la variable. \var{ROVSDT} repr\'esente la diagonale de la
matrice, hors convection/diffusion.\\
On va r\'esoudre l'\'equation (\ref{Base_Turrij_Eq_Temp_Rij}) sous forme incr\'ementale en
utilisant \fort{codits}, soit :
\begin{equation}\label{Base_Turrij_Eq_Temp_deltaRij}
\begin{array}{ll}
&\displaystyle \underbrace{\left(\frac {\rho^n_L}{\Delta t^n}
+ \rho^n_L \,C_1\,\frac{\varepsilon^n_L}{k^n_L}(1-\frac{\delta_{ij}}{3})
 - m^n_{\,lm} + \Gamma_L\,+ max(-\alpha^n_{R_{ij}},0)\right)\,|\Omega_l|}
_{\text {$\var{ROVSDT}$ contribuant
\`a la diagonale de la matrice simplifi\'ee de \fort{matrix}}}\,(\delta{R}_{ij}^{\,n+1,p+1})_{\,L}\\\\
&  \underbrace{+\sum\limits_{m\in Vois(l)}\displaystyle \left[
 m^n_{\,lm} \delta R_{ij,\,f_{\,lm}}^{\,n+1,p+1}
- (\mu^n_{\,lm} + \gamma^n_{\,lm})\
\frac{({\delta R}_{ij}^{\,n+1,p+1})_{M}-({\delta R}_{ij}^{\,n+1,p+1})_{L})}{\overline{L'M'}}\,
S_{\,lm} \right]}_{\text { convection upwind pur et diffusion non reconstruite
relatives \`a la matrice simplifi\'ee de \fort{matrix}\footnotemark}} \\
% voir le texte de la footmark plus bas
&= - \displaystyle\frac {\rho^n_L}{\Delta t^n}\,\left(\,(R^{\,n+1,p}_{ij})_L - (R^{\,n}_{ij})_L\,\right)\\
&-\,\underbrace{\displaystyle\int_{\Omega_l} \left(
\dive\,[\,(\rho\,\vect{u})^n\,R^{\,n+1,p}_{ij} - (\mu^n\,+ \gamma^n\,)\,
\grad{R^{\,n+1,p}_{ij}}\,]\right)\,d\Omega}_{\text {convection et diffusion
trait\'ees par \fort{bilsc2}}}\\
&+\displaystyle \int_{\Omega_l} \left[\,\mathcal{P}^{\,n+1,p}_{ij} + \mathcal{G}^{\,n+1,p}_{ij}
- \displaystyle\rho^n \,C_1\,\frac{\varepsilon^n}{k^n}\left[R^{\,n+1,p}_{ij}-
\frac{2}{3}\,k^n\,\delta_{ij}\right] + \phi^{\,n+1,p}_{ij,2} +
\phi^{\,n+1,p}_{ij,w}\,\right]\, d\Omega \\
& + \displaystyle\int_{\Omega_l} \left[- \frac{2}{3} \rho^n \varepsilon^n \delta_{ij}
 + \Gamma\,(\,R^{\,in}_{ij} - R^{\,n+1,p}_{ij}\,) +
\alpha^n_{R_{ij}}\,R^{\,n+1,p}_{ij}+ \beta^n_{R_{ij}}\right]\, d\Omega\\
&+ \sum\limits_{m\in
Vois(l)}\displaystyle \left[\ \tens{E}^n\,\grad{R}^{\,n+1,p}_{ij} \right]_{\,lm}\,.\,\vect{n}_{\,lm}S_{\,lm}\\
&+ \sum\limits_{m\in Vois(l)}\displaystyle \left[\
\tens{D}^n\,\grad{R}^{\,n+1,p}_{ij} \right]_{\,lm}\,.\,\vect{n}_{\,lm}S_{\,lm}\\
&- \sum\limits_{m\in Vois(l)} \gamma^n_{\,lm} \left( \grad{R}^{\,n+1,p}_{ij}\,.\,\vect{n}_{\,lm} \right)  S_{\,lm}\\
&+ \sum\limits_{m\in Vois(l)}  m^n_{\,lm}\,(R^{\,n+1,p}_{ij})_L\\
\end{array}
\end{equation}
% si on ne fait pas comme ca, il n'apparait pas
\footnotetext[\thefootnote]{Si $\var{IRIJNU} = 1$, on remplace  $\mu^n_{\,lm}$ par $(\mu +
\mu_t)^n_{\,lm}$ dans l'expression de la diffusion non reconstruite
associ\'ee \`a la matrice simplifi\'ee de \fort{matrix} ($\mu_t$ d\'esigne la
viscosit\'e turbulente calcul\'ee comme en $k-\varepsilon$).}

o\`u on rappelle :\\
pour $n$ donn\'e entier positif, on d\'efinit la suite
 $({R}_{ij}^{\,n+1,p})_{p \in \grandN}$
 par :
\begin{equation}\notag
\left\{\begin{array}{l}
{R}_{ij}^{\,n+1,0} = {R}_{ij}^{\,n}\\
{R}_{ij}^{\,n+1,p+1} = {R}_{ij}^{\,n+1,p} + \delta{R}_{ij}^{\,n+1,p+1} \\
\end{array}\right.
\end{equation}
$(\delta{R}_{ij}^{\,n+1,p+1})_{\,L}$ d\'esigne la valeur sur l'\'el\'ement
$\Omega_l$ du $\text{$(\,p+1\,)$-i\`eme}$ incr\'ement de ${R}_{ij}^{\,n+1}$,
$ m^n_{\,lm}$ le flux de masse \`a l'instant $n$ \`a travers la face $lm$,
$\delta R_{ij,\,f_{\,lm}}^{\,n+1,p+1}$ vaut $({\delta
R}_{ij}^{\,n+1,p+1})_{L}$  si $ m^n_{\,lm} \geqslant 0$, $({\delta
R}_{ij}^{\,n+1,p+1})_{M}$ sinon,
$\mathcal{P}^{\,n+1,p}_{ij}$, $\phi^{\,n+1,p}_{ij,2}$, $\phi^{\,n+1,p}_{ij,w}$ les valeurs
des quantit\'es associ\'ees correspondant \`a l'incr\'ement
$(\delta{R}_{ij}^{\,n+1,p})$.\\



Tous ces termes sont calcul\'es comme suit :
\begin{itemize}
\item Terme de gauche de l'\'equation (\ref{Base_Turrij_Eq_Temp_deltaRij})\\
Dans \fort{resrij} est calcul\'ee la variable \var{ROVSDT}. Les autres
termes sont compl\'et\'es par \fort{codits}, lors de la construction de la matrice simplifi\'ee , {\it via} un
appel au sous-programme \fort{matrix}. La quantit\'e
 $(\mu^n_{\,lm} + \gamma^n_{\,lm})$ \`a la face $lm$ est calcul\'ee lors de l'appel \`a
\fort{visort}.\\
\item Second membre de l'\'equation (\ref{Base_Turrij_Eq_Temp_deltaRij})\\
Le premier terme non d\'etaill\'e est calcul\'e par le sous-programme
\fort{bilsc2}, qui applique le sch\'ema convectif choisi par l'utilisateur, qui
reconstruit ou non selon le souhait de l'utilisateur les gradients intervenants
dans la convection-diffusion.\\
Les termes sans accolade sont, eux, compl\`etement explicites et ajout\'es au fur et
\`a mesure dans \var{SMBR} pour former
l'expression $f^{\,exp}_s$ de \fort{codits}.
\end{itemize}
On d\'ecrit ci-dessous les \'etapes de \fort{resrij} :
\begin{itemize}

\item DELTIJ = 1, pour $\var{ISOU} \leqslant 3$ et DELTIJ = 0  Si $\var{ISOU} >
3$. Cette valeur repr\'esente le symbole de Kroeneker $\delta_{ij}$.

\item Initialisation \`a z\'ero de \var{SMBR} (tableau contenant le second
membre) et \var{ROVSDT} (tableau contenant la diagonale de la matrice sauf celle
relative \`a la contribution de la
diagonale des op\'erateurs de convection et de diffusion lin\'earis\'es
\footnote{qui correspondent aux sch\'emas convectif upwind pur et diffusif sans
reconstruction.}), tous deux de dimension $\var{NCEL}$.

\item Lecture et prise en compte des termes sources utilisateur pour la variable $R_{ij}$

Appel \`a \fort{ustsri} pour charger les termes sources utilisateurs. Ils sont
stock\'es comme suit. Pour la cellule $\Omega_l$ de centre $L$, repr\'esent\'ee par $\var{IEL}$, on a :\\
\begin{equation}\notag
\left\{\begin{array}{lll}
&\var{ROVSDT(IEL)}&= |\Omega_l| \ \alpha_{R_{ij}}\\
&\var{SMBR(IEL)}&=|\Omega_l| \ \beta_{R_{ij}}\\
\end{array}\right.
\end{equation}
On affecte alors les valeurs ad\'equates au second membre \var{SMBR} et \`a la
diagonale \var{ROVSDT} comme suit :
\begin{equation}\notag
\left\{\begin{array}{lll}
&\var{SMBR(IEL)} &= \var{SMBR(IEL)} +\ |\Omega_l| \ \alpha_{R_{ij}} \ (R^n_{ij})_L \\
&\var{ROVSDT(IEL)}&= \text{max }(-\ |\Omega_l| \ \alpha_{R_{ij}},0)\\
\end{array}\right.
\end{equation}
La valeur de $ \var{ROVSDT}$ est ainsi calcul\'ee pour des raisons de stabilit\'e
num\'erique. En effet, on ne rajoute sur la diagonale que les valeurs positives,
ce qui correspond physiquement \`a impliciter les termes de rappel uniquement.
\item{Calcul du terme source de masse  si $\Gamma_L > 0$}

Appel de \fort{catsma} et incr\'ementation si n\'ecessaire de \var{SMBR} et
\var{ROVSDT} {\it via} :\\
\begin{equation}\notag
\left\{\begin{array}{lll}
\displaystyle \var{SMBR(IEL)} = \var{SMBR(IEL)} + |\Omega_l| \ \Gamma_L \
\left[(R^{\,in}_{ij})_L - (R^{\,n}_{ij})_L \right] \\
\displaystyle \var{ROVSDT(IEL)}=\var{ROVSDT(IEL)} + |\Omega_l| \ \Gamma_L
\end{array}\right.
\end{equation}
\item Calcul du terme d'accumulation de masse et du terme instationnaire

On stocke $\displaystyle \var{W1}= \int_{\Omega_l}\dive\,(\rho\,\vect{u})\,d\Omega$
calcul\'e par \fort{divmas} \`a l'aide des flux de masse aux faces internes
$ m^n_{\,lm}=\sum\limits_{m\in Vois(l)}{(\rho \vect{u})_{\,lm}^n} \text{.}\,
\vect{S}_{\,lm} $ (tableau \var{FLUMAS}) et des flux de masse aux bords  $ m^n_{\,b_{lk}} = \sum\limits_{k\in{\gamma_b(l)}}{(\rho \vect{u})_{\,{b}_{lk}}^n} \text{.}\,
\vect{S}_{\,{b}_{lk}} $ (tableau \var{FLUMAB}).
On incr\'emente ensuite \var{SMBR} et \var{ROVSDT}.
\begin{equation}\notag
\left\{\begin{array}{lll}
&\var{SMBR(IEL)} &= \var{SMBR(IEL)} + \var{ICONV}\  (R^n_{ij})_L\,(\displaystyle
\int_{\Omega_l}\dive\,(\rho\,\vect{u})\ d\Omega) \\
&\var{ROVSDT(IEL)}& = \var{ROVSDT(IEL)} +  \var{ISTAT}\,\displaystyle
\frac{\rho^n_L \ |\Omega_l|}{\Delta t^n} -  \var{ICONV}\ (\displaystyle
\int_{\Omega_l}\dive\,(\rho\,\vect{u})\ d\Omega) \\
\end{array}\right.
\end{equation}
\item Calcul des termes sources de production, des termes $\displaystyle
\phi_{\,ij,1}+\phi_{\,ij,2}$ et de la dissipation~$\displaystyle-\frac{2}{3} \varepsilon\,\delta_{\,ij}$ :

On effectue une boucle d'indice \var{IEL} sur les cellules $\Omega_l$ de centre $L$ :
\begin{itemize}
\item [$\Rightarrow$] $\displaystyle \var{TRPROD}= \frac{1}{2} (\mathcal{P}^n_{ii})_L = \frac{1}{2} \left[ \var{PRODUC(1,IEL)} +  \var{PRODUC(2,IEL)} +  \var{PRODUC(3,IEL)} \right] $
\item [$\Rightarrow$] $\displaystyle \var{TRRIJ }= \frac{1}{2} (R^n_{ii})_L $
\item [$\Rightarrow$] $\displaystyle \var{SMBR(IEL)} =\ \var{SMBR(IEL)}\ +$\\
$\ \displaystyle\rho^n_L |\Omega_l| \left[ \displaystyle
\frac{2}{3}\,\delta_{\,ij} \left( \ \displaystyle \frac{ C_2}{2}\,(\mathcal{P}^n_{ii})_L\ +
(C_1-1)\ \varepsilon^n_L\, \right)\right.$\\
$ + \left.\ (1-C_2) \ \var{PRODUC(ISOU,IEL)} -
\displaystyle C_1\ \frac{2\,\varepsilon^n_L}{(R^n_{ii})_L}\ (R^n_{ij})_L \right]$
\item [$\Rightarrow$] $\displaystyle \var{ROVSDT(IEL)} = \var{ROVSDT(IEL)} +
\rho^n_L \ |\Omega_l| \ (- \displaystyle \frac{1}{3} \ \,\delta_{\,ij} + 1) \ C_1
\ \frac{2\ \varepsilon^n_L}{(R^n_{ii})_L}$
\end{itemize}
\item Appel de \fort{rijech} pour le calcul des termes d'\'echo de paroi
 $\phi^n_{ij,w}$ si $\var{IRIJEC()}=1$ et ajout dans \var{SMBR}.\\
$\var{SMBR} = \var{SMBR} + \phi^n_{ij,w}$\\
Suivant son mode de calcul (\var{ICDPAR}), la distance � la paroi est directement accessible
par \var{RA(IDIPAR+IEL-1)} (\var{|ICDPAR|} = 1) ou bien
est calcul\'ee \`a partir de $\var{IA(IIFAPA(IPHAS)+IEL - 1)}$,
qui donne pour l'\'el\'ement $\var{IEL}$ le num\'ero de la face de bord
paroi la plus  proche (\var{|ICDPAR|} = 2). Ces tableaux ont \'et\'e renseign\'e une fois pour toutes au
d\'ebut de calcul.

\item  Appel de \fort{rijthe} pour le calcul des termes de gravit\'e $\mathcal{G}^n_{ij}$ et ajout dans \var{SMBR}.

Ce calcul n'a lieu que si $\var{IGRARI()} = 1$.
$ \var{SMBR} = \var{SMBR} + \mathcal{G}^n_{ij}$
\item Calcul de la partie explicite du terme de diffusion
 $\dive{\,\left[\tens{A}\,\grad{R}^{\,n}_{ij}\right]}$, plus pr\'ecis\'ement
des contributions du terme extradiagonal pris aux faces purement internes
(remplissage du tableau \var{VISCF}), puis aux faces de bord (remplissage du
tableau \var{VISCB}).
\begin{itemize}
\item [$\star$] Appel de \fort{grdcel} pour le calcul du gradient de
$R^{\,n}_{ij}$ dans chaque direction. Ces gradients sont respectivement
stock\'es dans les tableaux de travail \var{W1}, \var{W2} et \var{W3}.

\item [$\star$] boucle d'indice \var{IEL} sur les cellules $\Omega_l$ de centre
$L$ pour le
calcul de $\tens{E}^n\,\grad{R}^{\,n}_{ij}$ aux cellules dans un premier temps :\\
\begin{itemize}
\item [$\Rightarrow$] $\displaystyle \var{TRRIJ}= \frac{1}{2} (R^{\,n}_{ii})_L $
\item [$\Rightarrow$] $\displaystyle \var{CSTRIJ} = \rho^n_L\ C_S \ \displaystyle\frac{(R^n_{ii})_L}{2\,\varepsilon^n_L}$
\item [$\Rightarrow$] $\displaystyle \var{W4(IEL)} = \rho^n_L\ C_S\
\displaystyle\frac{(R^n_{ii})_L}{2\,\varepsilon^n_L} \left[\,(R^{\,n}_{12})_L \ \var{W2(IEL)} +
(R^{\,n}_{13})_L \ \var{W3(IEL)}\,\right]$
\item [$\Rightarrow$] $\displaystyle \var{W5(IEL)} = \rho^n_L\ C_S\
\displaystyle\frac{(R^n_{ii})_L}{2\,\varepsilon^n_L} \left[\,(R^{\,n}_{12})_L \ \var{W1(IEL)} +
(R^{\,n}_{23})_L \ \var{W3(IEL)}\,\right]$
\item [$\Rightarrow$] $\displaystyle \var{W6(IEL)} = \rho^n_L\ C_S\
\displaystyle\frac{(R^n_{ii})_L}{2\,\varepsilon^n_L} \left[\,(R^{\,n}_{13})_L \ \var{W1(IEL)} + (R^{\,n}_{23})_L \ \var{W2(IEL)}\,\right]$
\end{itemize}



\item [$\star$] Appel de \fort{vectds}\footnote{Le r\'esultat est stock\'e dans
\var{VISCF} et \var{VISCB}. Dans \fort{vectds}, les valeurs aux faces internes
sont interpol\'ees lin\'eairement sans reconstruction et \var{VISCB} est mis \`a
z\'ero.} pour assembler $\displaystyle\left[ \tens{E}^n\,\grad{R}^{\,n}_{ij}
\right]\,.\,\vect{n}_{\,lm}S_{\,lm}$ aux faces $lm$.
\item [$\star$] Appel de \fort{divmas} pour calculer la divergence du flux d\'efini par \var{VISCF} et \var{VISCB}.
Le r\'esultat est stock\'e dans \var{W4}.\\
Ajout au second membre \var{SMBR}.\\
\var{SMBR} = \var{SMBR} + \var{W4}
\end{itemize}

A l'issue de cette \'etape, seule la partie extradiagonale de la diffusion prise
enti\`erement explicite~:
 $$\sum\limits_{m\in
Vois(l)}\left[\ \tens{E}^n\,\grad{R}^{\,n}_{ij} \right]_{\,lm}\,.\,\vect{n}_{\,lm}S_{\,lm}$$ a \'et\'e calcul\'ee.\\

\item Calcul de la partie diagonale du terme de diffusion\footnote{Seule la
composante normale  du  gradient de $R_{ij}$ aux faces sera implicite.} :\\
Comme on l'a d\'eja vu, une partie de cette contribution sera trait\'ee en
implicite (celle relative \`a la composante normale du gradient) et donc
ajout\'ee au second membre par \fort{bilsc2} ; l'autre
partie sera explicite et prise en compte dans $f_s^{\,exp}$.
\begin{itemize}
\item [$\star$] On effectue une boucle d'indice \var{IEL} sur les cellules
$\Omega_l$ de centre $L$ :
\begin{itemize}
\item [$\Rightarrow$] $\displaystyle \var{TRRIJ }= \frac{1}{2} (R^{\,n}_{ii})_L $
\item [$\Rightarrow$] $\displaystyle \var{CSTRIJ} = \rho^n_L \ C_S \ \frac{(R^{\,n}_{ii})_L}{2\,\varepsilon^n_L}$
\item [$\Rightarrow$] $\displaystyle \var{W4(IEL)} = \rho^n_L \ C_S \
\frac{(R^{\,n}_{ii})_L}{2\,\varepsilon^n_L} \ (R^{\,n}_{11})_L$
\item [$\Rightarrow$] $\displaystyle \var{W5(IEL)} = \rho^n_L \ C_S \ \frac{(R^{\,n}_{ii})_L}{2\,\varepsilon^n_L}\ (R^n_{22})_L$
\item [$\Rightarrow$] $\displaystyle \var{W6(IEL)} = \rho^n_L \ C_S \ \frac{(R^{\,n}_{ii})_L}{2\,\varepsilon^n_L} \ (R^n_{33})_L$
\end{itemize}

%\item Traitement du parall\'elisme et de la p\'eriodicit\'e.

\item [$\star$] On effectue une boucle d'indice \var{IFAC} sur les faces
purement internes $lm$ pour remplir le tableau \var{VISCF} :
\begin{itemize}
\item [$\Rightarrow$] Identification des cellules $\Omega_l$ et $\Omega_m$ de
centre respectif $L$ (variable \var{II}) et $M$ (variable \var{JJ}), se trouvant de chaque c\^ot\'e de la face
$lm$\footnote{La normale \`a la face est orient\'ee de L vers M.}.
\item [$\Rightarrow$] Calcul du carr\'e de la surface de la face. La valeur est
stock\'ee dans le tableau \var{SURFN2}.
\item [$\Rightarrow$] Interpolation du gradient de $R^{\,n}_{ij}$ \`a la face
$lm$ (gradient facette $\left[\grad{R}^{\,n}_{ij}\right]_{\,lm}$) :
\begin{equation}\notag
\left\{\begin{array}{ll}
\var{GRDPX} &= \displaystyle \frac{1}{2} \left(\var{W1(II)} + \var{W1(JJ)}
\right) \\
&\\
\var{GRDPY} &= \displaystyle \frac{1}{2} \left(\var{W2(II)} + \var{W2(JJ)} \right) \\
&\\
\var{GRDPZ} &= \displaystyle \frac{1}{2} \left(\var{W3(II)} + \var{W3(JJ)} \right)
\end{array}\right.
\end{equation}
\item [$\Rightarrow$] Calcul du gradient de $R^{\,n}_{ij}$ normal \`a la face
$lm$, $\left[\grad{R}^{\,n}_{ij}\right]_{\,lm}.\vect{n}_{\,lm}\,S_{\,lm}$.\\

$\displaystyle \var{GRDSN} =  \var{GRDPX} \ \var{SURFAC(1,IFAC)} + \var{GRDPY} \ \var{SURFAC(2,IFAC)} +  \var{GRDPZ} \ \var{SURFAC(3,IFAC)}$
$\var{SURFAC}$ est le vecteur surface de la face \var{IFAC}.


\item [$\Rightarrow$] calcul de
 $\left[\grad{R^{\,n}_{ij}} - (\grad
R^{\,n}_{ij}\,.\,\vect{n}_{\,lm})\vect{n}_{\,lm}\right]$, les vecteurs \'etant
calcul\'es \`a la face $lm$ :
\begin{equation}\notag
\left\{\begin{array}{lll}
&\displaystyle \var{GRDPX} &= \var{GRDPX} - \displaystyle\frac{\var{GRDSN}}{\var{SURFN2}} \ \var{SURFAC(1,IFAC)}\\
&&\\
&\displaystyle \var{GRDPY} &= \var{GRDPY} - \displaystyle\frac{\var{GRDSN}}{\var{SURFN2}} \ \var{SURFAC(2,IFAC)} \\
&&\\
&\displaystyle \var{GRDPZ} &= \var{GRDPZ} - \displaystyle\frac{\var{GRDSN}}{\var{SURFN2}} \ \var{SURFAC(3,IFAC)}
\end{array}\right.
\end{equation}
\item [$\Rightarrow$] finalisation du calcul de l'expression totalement
explicite
 $$\left[ \tens{D}^n\,\left( \grad{R^{\,n}_{ij}} - (\grad R^{\,n}_{ij}\,.\,\vect{n}_{\,lm})\,\vect{n}_{\,lm}\right) \right]\,.\,\vect{n}_{\,lm}$$
\begin{equation}\notag
\begin{array} {ll}
\displaystyle \var{VISCF} = &
 \displaystyle\frac{1}{2} (\ \var{W4(II)} +\ \var{W4(JJ)}) \ \var{GRDPX} \
\var{SURFAC(1,IFAC)})\ + \\
&\\
&  \displaystyle\frac{1}{2} (\ \var{W5(II)} +\ \var{W5(JJ)}) \ \var{GRDPY} \
\var{SURFAC(2,IFAC)})\ + \\
&\\
&  \displaystyle\frac{1}{2} (\ \var{W6(II)} +\ \var{W6(JJ)}) \ \var{GRDPZ} \ \var{SURFAC(3,IFAC)})
\end{array}
\end{equation}
\end{itemize}

\item [$\star$] Mise \`a z\'ero du tableau \var{VISCB}.

\item [$\star$] Appel de \fort{divmas} pour calculer la divergence de~:
 $$\tens{D}^{\,n}\,\left( \grad{R^{\,n}_{ij}} - (\grad R^{\,n}_{ij}\,.\,\vect{n}_{\,lm})\vect{n}_{\,lm}\right)$$ d\'efini aux faces dans \var{VISCF} et \var{VISCB}.

Le r\'esultat est stock\'e dans le tableau \var{W1}.\\
Ajout au second membre \var{SMBR}.\\
$\var{SMBR} = \var{SMBR} + \var{W1}$
\end{itemize}
\item Calcul de la viscosit\'e orthotrope $\gamma^n_{\,lm}$ \`a la face $lm$ de la variable principale
$R^{\,n}_{ij}$\\
Ce calcul permet au sous-programme \fort{codits} de compl\'eter le second membre
\var{SMBR} par :
\begin{equation}
\begin{array} {ll}
& \sum\limits_{m\in Vois(l)}
\mu^n_{\,lm}\,\left(\grad{R}^{\,n}_{ij}\,.\,\vect{n}_{\,lm}\right)S_{\,lm}
 + \sum\limits_{m\in Vois(l)} \left(\grad{R}^{\,n}_{ij}
\,.\,\vect{n}_{\,lm}\right)\left[\tens{D}^{\,n}\,\vect{n}_{\,lm}\right]_{\,lm}\,.\,\vect{n}_{\,lm}\
S_{\,lm}\\
& = \sum\limits_{m\in Vois(l)}(\,\mu^n_{\,lm}\, + \,\gamma^n_{\,lm}\,)\,\left(\grad{R}^{\,n}_{ij}\,.\,\vect{n}_{\,lm}\right)S_{\,lm}
\end{array}
\end{equation}
sans pr\'eciser la nature de la face $lm$, {\it via} l'appel \`a \fort{bilsc2}  et de disposer de la quantit\'e
$(\mu^n_{\,lm}\, + \gamma^n_{\,lm})$ pour construire sa
matrice simplifi\'ee.\\
\begin{itemize}
\item [$\star$] On effectue une boucle d'indice \var{IEL} sur les cellules
$\Omega_l$ :
\begin{itemize}
\item [$\Rightarrow$] $\displaystyle \var{TRRIJ }= \frac{1}{2} (R^{\,n}_{ii})_L $
\item [$\Rightarrow$] $\displaystyle \var{RCSTE} = \rho^n_L \ C_S \ \frac{ (R^{\,n}_{ii})_L}{2\,\varepsilon^n_L} $
\item [$\Rightarrow$] $\displaystyle \var{W1(IEL)} = \mu^n +\rho^n_L \ C_S \ \frac{
(R^{\,n}_{ii})_L}{2\,\varepsilon^n_L}\ (R^n_{11})_L$
\item [$\Rightarrow$] $\displaystyle \var{W2(IEL)} = \mu^n + \rho^n_L \ C_S \ \frac{ (R^{\,n}_{ii})_L}{2\,\varepsilon^n_L}\ (R^n_{22})_L$
\item [$\Rightarrow$] $\displaystyle \var{W3(IEL)} = \mu^n + \rho^n_L \ C_S \ \frac{ (R^{\,n}_{ii})_L}{2\,\varepsilon^n_L}\ (R^n_{33})_L$
\end{itemize}

\item [$\star$] Appel de \fort{visort} pour calculer la viscosit\'e orthotrope
\footnote{Comme dans le sous-programme \fort{viscfa}, on multiplie la viscosit\'e par
$\displaystyle \frac{S_{\,lm}}{\overline{L'M'}}$, o\`u $S_{\,lm}$ et
$\overline{L'M'}$ repr\'esentent respectivement la surface de la face $lm$ et la
mesure alg\'ebrique du segment reliant les projections des centres des cellules
voisines sur la normale \`a la face. On garde dans ce sous-programme  la possibilit\'e d'interpoler la viscosit\'e aux cellules lin\'eairement ou d'utiliser une moyenne harmonique. La viscosit\'e au bord est celle de la cellule de bord correspondante.}
$ \gamma^n_{\,lm} = (\tens{D}^{\,n}\,\vect{n}_{\,lm}).\vect{n}_{\,lm}$ aux faces $lm$

Le r\'esultat est stock\'e dans les tableaux \var{VISCF} et \var{VISCB}.
\end{itemize}

\item appel de \fort{codits} pour la r\'esolution de l'\'equation de
convection/diffusion/termes sources de la variable $R_{ij}$. Le terme source est
\var{SMBR}, la viscosit\'e \var{VISCF} aux faces purement internes (
resp. \var{VISCB} aux faces de bord) et \var{FLUMAS} le flux de masse interne
 ( resp. \var{FLUMAB} flux de masse au bord). Le r\'esultat est la variable $R_{ij}$ au temps
$n+1$, donc $R^{\,n+1}_{ij}$. Elle est stock\'ee dans le tableau \var{RTP} des
variables mises \`a jour.

\end{itemize}

\etape{Appel de \fort{reseps} pour la r\'esolution de la variable $\varepsilon$}

On d\'ecrit ci-dessous le sous-programme \fort{reseps}, les commentaires du sous-programme \fort{resrij} ne seront pas r\'ep\'et\'es puisque les deux sous-programmes ne diff\`erent que par leurs termes sources.

\begin{itemize}
\item Initialisation \`a z\'ero de \var{SMBR} et \var{ROVSDT}.

\item{Lecture et prise en compte des termes sources utilisateur pour la variable $\varepsilon$ :}

Appel de \fort{ustsri} pour charger les termes sources utilisateurs. Ils sont
stock\'es dans les tableaux suivants :\\
pour la cellule $\Omega_l$ repr\'esent\'ee par $\var{IEL}$ de centre $L$, on a :
\begin{equation}\notag
\left\{\begin{array}{lll}
&\var{ROVSDT(IEL)}&= |\Omega_l| \ \alpha_{\varepsilon}\\
&\var{SMBR(IEL)}&=|\Omega_l| \ \beta_{\varepsilon}\\
\end{array}\right.
\end{equation}
On affecte alors les valeurs ad\'equates au second membre \var{SMBR} et \`a la
diagonale \var{ROVSDT} comme suit :
\begin{equation}\notag
\left\{\begin{array}{lll}
&\var{SMBR(IEL)} &= \var{SMBR(IEL)} +\ |\Omega_l| \ \alpha_{\,\varepsilon} \
\varepsilon^n_L \\
&\var{ROVSDT(IEL)}&= \text{max }(-\ |\Omega_l| \ \alpha_{\,\varepsilon},0)\\
\end{array}\right.
\end{equation}

\item{Calcul du terme source de masse si $\Gamma_L > 0$ :
\begin{equation}\notag
\left\{\begin{array}{lll}
&\displaystyle \var{SMBR(IEL)} = \var{SMBR(IEL)} + |\Omega_l| \ \Gamma_L \
(\varepsilon^{\,in}_L -\varepsilon^n_L) \\
&\displaystyle \var{ROVSDT(IEL)}= \var{ROVSDT(IEL)} + |\Omega_l| \ \Gamma_L
\end{array}\right.
\end{equation}
\item Calcul du terme d'accumulation de masse et du terme instationnaire \\
On stocke $\displaystyle \var{W1}= \int_{\Omega_l}\dive\,(\rho\,\vect{u})\,d\Omega$
calcul\'e par \fort{divmas} \`a l'aide des flux de masse internes et aux bords.\\
On incr\'emente ensuite \var{SMBR} et \var{ROVSDT}.
\begin{equation}\notag
\left\{\begin{array}{lll}
&\var{SMBR(IEL)} &= \var{SMBR(IEL)} + \var{ICONV}\ \varepsilon^n_L\,(\displaystyle
\int_{\Omega_l}\dive\,(\rho\,\vect{u})\ d\Omega) \\
&\var{ROVSDT(IEL)}& = \var{ROVSDT(IEL)} +  \var{ISTAT}\,\displaystyle
\frac{\rho^n_L \ |\Omega_l|}{\Delta t^n} -  \var{ICONV}\ (\displaystyle
\int_{\Omega_l}\dive\,(\rho\,\vect{u})\ d\Omega) \\
\end{array}\right.
\end{equation}

\item Traitement du terme de production
 $\displaystyle \rho\,C_{\varepsilon_1}\,\frac{\varepsilon}{k}\,\mathcal{P}$
 et du terme de dissipation $-\,\displaystyle \rho\,C_{\varepsilon_2}\,\frac{\varepsilon}{k}\,\varepsilon$ \\
pour cela, on effectue une boucle d'indice \var{IEL} sur les cellules $\Omega_l$
de centre $L$ :
\begin{itemize}
\item [$\Rightarrow$] $\displaystyle \var{TRPROD}= \frac{1}{2} (\mathcal{P}^n_{ii})_L = \frac{1}{2} \left[ \var{PRODUC(1,IEL)} +  \var{PRODUC(2,IEL)} +  \var{PRODUC(3,IEL)} \right] $
\item [$\Rightarrow$] $\displaystyle \var{TRRIJ }= \frac{1}{2} (R^n_{ii})_L $
\item [$\Rightarrow$] $\displaystyle \var{SMBR(IEL)} = \var{SMBR(IEL)} + \rho^n_L
|\Omega_l| \left[ -C_{\varepsilon_2} \ \frac{2\,(\varepsilon^n_L)^2}{(R^n_{ii})_L} + C_{\varepsilon_1} \ \frac{\varepsilon^n_L}{(R^n_{ii})_L}\ (\mathcal{P}^n_{ii})_L \right] $
\item [$\Rightarrow$] $\displaystyle \var{ROVSDT(IEL)} = \var{ROVSDT(IEL)} + C_{\varepsilon_2} \ \rho^n_L \ |\Omega_l| \ \frac{2\,\varepsilon^n_L}{(R^n_{ii})_L}$
\end{itemize}

\item Appel de \fort{rijthe} pour le calcul des termes de gravit\'e $\mathcal{G}^n_{\varepsilon}$ et ajout dans \var{SMBR}.

$ \var{SMBR} = \var{SMBR} + \mathcal{G}^n_{\varepsilon}$\\
Ce calcul n'a lieu que si $\var{IGRARI()} = 1$.

\item Calcul de la diffusion de $\varepsilon$ \\
 Le terme $\dive \left[\mu\, \grad(\varepsilon) + \tens{A'}\,\grad(\varepsilon)
\right]$ est calcul\'e exactement de la m\^eme mani\`ere que pour les tensions
de Reynolds $R_{ij}$ en rempla\c cant $\tens{A}$ par $\tens{A'}$.

\item Appel de \fort{codits} pour la r\'esolution de l'\'equation de
convection/diffusion/termes sources de la variable principale $\varepsilon$. Le
r\'esultat $\varepsilon^{\,n+1}$ est stock\'e dans le tableau \var{RTP} des
variables mises \`a jour.
}
\end{itemize}

\etape{clippings finaux}
On passe enfin dans le sous-programme  \fort{clprij} pour faire un clipping \'eventuel
des variables $R^{\,n+1}_{ij}$ et $\varepsilon^{\,n+1}$. Le sous-programme
\fort{clprij} est appel\'e\footnote{L'option
$\var{ICLIP} = 1$ consiste en un clipping minimal des variables $R_{ii}$ et
$\varepsilon$ en prenant la valeur absolue de ces variables puisqu'elles ne
peuvent \^etre que positives.} avec $\var{ICLIP} = 2$ . Cette option
\footnote{Quand la valeur des grandeurs $R_{ii}$ ou $\varepsilon$ est
n\'egative, on la remplace par le minimum entre sa valeur absolue et (1,1)
fois la valeur obtenue au pas de temps pr\'ec\'edent.} contient l'option $\var{ICLIP} = 1$  et permet de v\'erifier l'in\'egalit\'e de Cauchy-Schwarz sur les grandeurs extra-diagonales du tenseur $\tens{R}$ (pour $i \neq j$, $|R_{ij}|^2 \le R_{ii} R_{jj}$).


%%%%%%%%%%%%%%%%%%%%%%%%%%%%%%%%%%
%%%%%%%%%%%%%%%%%%%%%%%%%%%%%%%%%%
\section{Points \`a traiter}
%%%%%%%%%%%%%%%%%%%%%%%%%%%%%%%%%%
%%%%%%%%%%%%%%%%%%%%%%%%%%%%%%%%%%
Sauf mention explicite, $\phi$ repr\'esentera une tension de Reynolds ou la dissipation turbulente ($\phi = R_{ij} \ \text{ou} \ \varepsilon$).

\begin{itemize}
\item {La vitesse utilis\'ee pour le calcul de la production est explicite. Est-ce qu'une implicitation peut am\'eliorer la pr\'ecision temporelle de $\phi$ \footnote{Cette remarque peut \^etre g\'en\'eralis\'ee. En effet, peut-on envisager d'actualiser les variables d\'ej\`a r\'esolues (sans r\'eactualiser les variables turbulentes apr\`es leur r\'esolution)? Ceci obligerait \`a modifier les sous-programmes tels que \fort{condli} qui sont appel\'es au d\'ebut de la boucle en temps.} ?}
\item {Dans quelle mesure le terme d'\'echo de paroi est-il valide ? En effet, ce terme est remis en question par certains auteurs.}
\item {On peut envisager la r\'esolution d'un syst\`eme hyperbolique pour les
tensions de Reynolds afin d'introduire un couplage avec le champ de vitesse.}
\item {Le flux au bord \var{VISCB} est annul\'e dans le sous-programme
\fort{vectds}. Peut-on envisager de mettre au bord la valeur de la variable
concern\'ee \`a la cellule de bord correspondant? De m\^eme, il faudrait se
pencher sur les hypoth\`eses sous-jacentes \`a l'annulation des contributions
aux bords de \var{VISCB} lors du calcul de : $$\left[ \tens{D}^n\,\left( \grad{R^{\,n}_{ij}} - (\grad R^{\,n}_{ij}\,.\,\vect{n}_{\,lm})\,\vect{n}_{\,lm}\right) \right]\,.\,\vect{n}_{\,lm}.$$}
\item {Un probl\`eme de pond\'eration appara\^\i t plus g\'en\'eralement. Si on prend la partie explicite de $\tens{D}\,\grad(\phi)$, la pond\'eration aux faces internes utilise le coefficient $\displaystyle\frac{1}{2}$ avec pond\'eration s\'epar\'ee de $\tens{D}$ et $\grad(\phi)$, alors que pour $\tens{E}\,\grad(\phi)$, on calcule d'abord ce terme aux cellules pour ensuite l'interpoler lin\'eairement aux faces \footnote{Cette interpolation se fait dans \fort{vectds} avec des coefficients de pond\'eration aux faces.}. Ceci donne donc deux types d'interpolations pour des termes de m\^eme nature.}
\item {On laisse la possibilit\'e dans \fort{visort} d'utiliser une moyenne
harmonique aux faces. Est-ce que ceci est valable puisque les interpolations
utilis\'ees lors du calcul de la partie explicite de $\tens{A}\,\grad{\phi}$
sont des moyennes arithm\'etiques ?}
\item {Les techniques adopt\'ees lors du clipping sont \`a revoir.}
\item {On utilise dans le cadre du mod\`ele $\displaystyle R_{ij}-\varepsilon $ une semi-implicitation de termes comme $\displaystyle \phi_{ij,1}$ ou $\displaystyle -\rho\,C_{\varepsilon_2}\,\frac{\varepsilon}{k}\,\varepsilon$. On peut envisager le m\^eme type d'implicitation dans \fort{turbke} m\^eme en pr\'esence du couplage $\displaystyle k-\varepsilon$.}
\item L'adoption d'une r\'esolution d\'ecoupl\'ee fait perdre l'invariance par rotation.
\item La formulation et l'implantation des conditions aux limites de paroi
devront \^etre v\'erifi\'ees. En effet, il semblerait que, dans certains cas, des ph\'enom\`enes
``oscillatoires'' apparaissent, sans qu'il soit ais\'e d'en d\'eterminer la cause.
\item L'implicitation partielle (du fait de la r\'esolution d\'ecoupl\'ee) des
conditions aux limites conduit souvent \`a des calculs instables. Il
conviendrait d'en conna\^\i tre la raison. L'implicitation partielle avait
\'et\'e mise en \oe uvre afin de tenter d'utiliser un pas de temps plus grand
dans le cas de jets axisym\'etriques en particulier.

\end{itemize}

%                      Code_Saturne version 1.3
%                      ------------------------
%
%     This file is part of the Code_Saturne Kernel, element of the
%     Code_Saturne CFD tool.
%
%     Copyright (C) 1998-2007 EDF S.A., France
%
%     contact: saturne-support@edf.fr
%
%     The Code_Saturne Kernel is free software; you can redistribute it
%     and/or modify it under the terms of the GNU General Public License
%     as published by the Free Software Foundation; either version 2 of
%     the License, or (at your option) any later version.
%
%     The Code_Saturne Kernel is distributed in the hope that it will be
%     useful, but WITHOUT ANY WARRANTY; without even the implied warranty
%     of MERCHANTABILITY or FITNESS FOR A PARTICULAR PURPOSE.  See the
%     GNU General Public License for more details.
%
%     You should have received a copy of the GNU General Public License
%     along with the Code_Saturne Kernel; if not, write to the
%     Free Software Foundation, Inc.,
%     51 Franklin St, Fifth Floor,
%     Boston, MA  02110-1301  USA
%
%-----------------------------------------------------------------------
%
\programme{vortex}
%
\vspace{1cm}
%%%%%%%%%%%%%%%%%%%%%%%%%%%%%%%%%%
%%%%%%%%%%%%%%%%%%%%%%%%%%%%%%%%%%
\section{Fonction}
%%%%%%%%%%%%%%%%%%%%%%%%%%%%%%%%%%
%%%%%%%%%%%%%%%%%%%%%%%%%%%%%%%%%%
Ce sous-programme est d�di� � la g�n�ration des conditions d'entr�e
turbulente utilis�es en LES.


La m�thode des vortex est bas�e sur une approche de tourbillons
ponctuels. L'id�e de la m�thode consiste � injecter des tourbillons 2D dans le
plan d'entr�e du calcul, puis � calculer le champ de vitesse induit par ces
tourbillons au centre des faces d'entr�e.

%                      Code_Saturne version 1.3
%                      ------------------------
%
%     This file is part of the Code_Saturne Kernel, element of the
%     Code_Saturne CFD tool.
% 
%     Copyright (C) 1998-2007 EDF S.A., France
%
%     contact: saturne-support@edf.fr
% 
%     The Code_Saturne Kernel is free software; you can redistribute it
%     and/or modify it under the terms of the GNU General Public License
%     as published by the Free Software Foundation; either version 2 of
%     the License, or (at your option) any later version.
% 
%     The Code_Saturne Kernel is distributed in the hope that it will be
%     useful, but WITHOUT ANY WARRANTY; without even the implied warranty
%     of MERCHANTABILITY or FITNESS FOR A PARTICULAR PURPOSE.  See the
%     GNU General Public License for more details.
% 
%     You should have received a copy of the GNU General Public License
%     along with the Code_Saturne Kernel; if not, write to the
%     Free Software Foundation, Inc.,
%     51 Franklin St, Fifth Floor,
%     Boston, MA  02110-1301  USA
%
%-----------------------------------------------------------------------
%
%%%%%%%%%%%%%%%%%%%%%%%%%%%%%%%%%%
%%%%%%%%%%%%%%%%%%%%%%%%%%%%%%%%%%
\section{Discr\'etisation}
%%%%%%%%%%%%%%%%%%%%%%%%%%%%%%%%%%
%%%%%%%%%%%%%%%%%%%%%%%%%%%%%%%%%%

Le terme convectif en $\dive(\underline{u} \otimes \rho\,\underline{u})$
introduit une non lin\'earit\'e et un couplage des composantes de la vitesse
$\vect{u}$ dans l'�quation (\ref{Base_Preduv_eqqdm}). Une lin\'earisation et un d\'ecouplage
des trois composantes de la 
vitesse sont r\'ealis\'es lors de la discr\'etisation de cette \'etape de
pr\'ediction.\\
En effet, soit :
\begin{equation}
\vect{\widetilde{u}}= \vect{u}^n + \delta \vect{u} 
\end{equation}
La contribution exacte du terme convectif \`a prendre en compte dans cette
\'etape de pr\'ediction serait :\\
\begin{equation}\label{Base_Preduv_Conv_exact}
\begin{array}{ll}
\dive(\vect{\widetilde{u}} \otimes \rho\,\vect{\widetilde{u}}) =
\dive(\vect{u}^{n} \otimes \rho\,\vect{u}^{n}) + \dive(\delta \vect{u} \otimes
\rho\,\vect{u}^{n}) +  \underbrace { \dive(\vect{u}^{n} \otimes
\rho\,\delta \vect{u})}_{\text {terme couplant lin\'eaire}} +  \underbrace { \dive(\delta \vect{u} \otimes
\rho\,\delta \vect{u})}_{\text {terme couplant et non lin\'eaire}}\\
\end{array} 
\end{equation}
Les deux derniers termes de l'expression (\ref{Base_Preduv_Conv_exact}) sont {\em a priori} n�glig�s
de mani�re � obtenir un syst\`eme en vitesse qui soit d\'ecoupl\'e et donc,
�viter l'inversion d'une matrice pouvant \^etre de tr\`es grande taille. Ces
deux termes peuvent n�anmoins �tre pris en compte de mani�re plus ou moins
approch�e par extrapolation explicite du flux de masse en $n+\theta_F$ (pour le
terme couplant lin�aire provenant de la convection de $\vect{u}^{n}$ par $\delta
\vect{u}$) et utilisation d'un point-fixe par sous it�ration sur le sous
programme \fort{navsto} (pour le terme non-lin�aire, en sp�cifiant $\var{NTERUP}>1$).

L'�quation (\ref{Base_Preduv_eqqdm}) est discr�tis�e au temps $n+\theta$ � l'aide d'un
$\theta$-sch�ma, et le tenseur des pertes de charges d�compos� en une partie
diagonale $\tens{K}_{d}$ et une extradiagonale $\tens{K}_{e}$ (soit
 $\tens{K}_{pdc}=\tens{K}_{d}+\tens{K}_{e}$).\\
$\bullet$ La pression est suppos�e connue en $n-1+\theta$ (d�calage temporel
pression-vitesse) et le gradient naturellement calcul� � cet instant.\\ 
$\bullet$ Les termes sources de viscosit� secondaire, de gradient transpos\'e,
ceux provenant du mod�le de turbulence\footnote{except� $\dive (\mu_t\ (\ggrad
\underline {u}))$}, $\rho\,\tens{K}_{\,e}\ \underline{u}$, $(\rho -\rho_0)
\underline {g}$ ainsi que $\underline{T}_{s}^{\,exp}$ et
$\Gamma\,\underline{u}_{\,i}$ sont pris de mani�re explicite au temps $n$, ou
extrapol�s suivant le sch�ma en temps choisi pour les propri�t�s physique et les
termes sources.\\ 
$\bullet$ Les termes sources $\underline{u}\,\,\dive (\rho\,\underline {u})$,
$\Gamma\,\,\underline{u}$, $T_{s}^{\,imp}\,\,\underline{u}$ et
$-\rho\,\tens{K}_{\,d}\,\,\underline{u}$ sont implicit�s est calcul�s �
l'instant $n+\theta$.\\ 
$\bullet$ Le terme de diffusion $\dive (\mu_{\,tot}\,\ggrad \underline{u})$ est
implicit� : la vitesse est prise � l'instant $n+\theta$ et la viscosit�
explicit�e ou extrapol�e.\\ 
$\bullet$ Enfin, le terme de convection en $\dive(\,\underline{u} \otimes
(\rho\underline{u})\,)$ est implicit� : la composante r�solue de la vitesse est
prise en $n+\theta$, et le flux de masse, explicit�, ou extrapol� en
$n+\theta_F$. 

Par souci de clart�, on suppose, en l'absence d'indication, que les propri�tes
physiques $\Phi$ ($\rho,\,\mu_{tot},\,...$) et le flux de masse
$(\rho\underline{u})$ sont pris respectivement aux instants $n+\theta_\Phi$ et
$n+\theta_F$, o� $\theta_\Phi$ et $\theta_F$ d�pendent des sch�mas en temps
sp�cifiquement utilis�s pour ces grandeurs\footnote{cf. \fort{introd}}. 

La discr�tisation temporelle de l'�quation (\ref{Base_Preduv_eqqdm}) s'�crit alors comme suit : 

\begin{equation}\label{Base_Preduv_eq_di1}
 \begin{array}{c}
\displaystyle \rho\,\ \frac{ \underline {\widetilde{u}}^{n+1} -\underline {u}^{n} }
{\Delta t} + \dive(\,\underline{\widetilde{u}}^{n+\theta} \otimes (\rho\underline{u})\,) -\dive
(\mu_{\,tot}\,\ggrad \underline{\widetilde{u}}^{n+\theta}) =
\\
\displaystyle
 - \grad p^{n-1+\theta} + \dive (\rho\,\underline {u})\,\underline{\widetilde{u}}^{n+\theta} +(\Gamma\,\underline{u}_{\,i})^{n+\theta_S}-\Gamma^n\,\,\underline{\widetilde{u}}^{n+\theta}
\\
\begin{array}{c}
\displaystyle
- \rho\,\tens{K}_{\,d}^{n}\,\,\underline{\widetilde{u}}^{n+\theta} - (\rho\,\tens{K}_{\,e}\ \underline{u})^{n+\theta_S} + (\underline{T}_{s}^{\,exp})^{\,n+\theta_S} + T_{s}^{\,imp}\,\,\underline{\widetilde{u}}^{n+\theta}
\\
\displaystyle
+[\dive (\mu_{\,tot}\,^t\ggrad \underline {u})]^{n+\theta_S}-\frac {2} {3}[\,\grad (\mu_{\,tot}\,\dive \underline {u})]^{n+\theta_S} + (\rho -\rho_0) \underline {g}
 - (\underline{turb})^{n+\theta_S}
\end{array}
\end{array}
\end{equation}
o\`u, par souci de simplification, on a pos\'e :
\begin{equation}
\mu_{\,tot}=
\begin{cases}
\mu+\mu_t & \text{pour les mod�les � viscosit� turbulente ou en LES}, \\
\mu & \text{pour les mod�les au second ordre ou en laminaire}
\end{cases} \ 
\end{equation}
\\
et :
\begin{equation}
\underline{turb}^{n}=
\begin{cases}
\displaystyle\frac {2}{3}\grad (\rho^{n}\,k^{n}) & \text{pour les mod�les � viscosit� turbulente}, \\
\dive(\rho^{n}\,\tens{R}^n) & \text{pour les mod�les au second ordre},\\
0 & \text{en laminaire ou en LES}\\
\end{cases}
\end{equation}
Par analogie avec l'�criture du $\theta$-sch�ma pour une variable scalaire, $\,
\underline {\widetilde{u}}^{n+\theta}$ est interpol�e � partir de la vitesse
pr�dite $\underline {\widetilde{u}}^{n+1}$ de la mani\`ere suivante\footnote{si
$\theta=1/2$, ou qu'une extrapolation est utilis�e, l'ordre 2 n'est obtenu que si
le pas de temps $\Delta t$ est uniforme en temps et en espace.}~: 
\begin{equation}
\underline {\widetilde{u}}^{n+\theta}=\theta\, \underline
{\widetilde{u}}^{n+1}+(1-\theta)\, \underline {u}^{n}\\ 
\end{equation}
Avec :
\begin{equation}
\left\{
\begin{array}{ll}
\theta = 1   & \text{Pour un sch\'ema de type Euler implicite d'ordre 1.}\\
\theta = 1/2 & \text{Pour un sch\'ema de type Cranck-Nicolson d'ordre 2.}\\
\end{array}
\right.
\end{equation}

L'�quation (\ref{Base_Preduv_eq_di1}) est alors r��crite sous la forme :

\begin{equation}\label{Base_Preduv_eq_di2}
\begin{array}{c}
\displaystyle \underbrace{\left(\frac{\rho}{\Delta t} -\theta \,\dive (\rho\,\underline {u}) +\theta \,\, \Gamma^n +
\theta \,\, \rho\,\tens{K}_{\,d}^n-\theta \,T_s^{\,imp} \right)}_{\displaystyle f_s^{imp}}\, (\underline {\,\widetilde{u}}^{n+1} -\underline {u}^{n})
\\
 +\, \theta\, \dive(\underline {\widetilde{u}}^{n+1} \otimes (\rho\underline{u}))-\, \theta\,\dive (\mu_{\,tot}\,\ggrad \underline {\widetilde{u}}^{n+1}) =
\\
-\,(1-\theta)\, \dive(\underline {u}^{n} \otimes (\rho\underline{u})) +\,(1-\theta)\,\dive (\mu_{\,tot}\,\ggrad \underline {u}^{n})
\\
f_s^{exp}\left\{
\begin{array}{c}
\displaystyle 
- \grad p^{n-1+\theta} + \dive (\rho\,\underline {u})\,\underline{u}^{n} +\,(\,\Gamma^{n}\,\underline{u}_{\,i}\,)^{n+\theta_S}- \Gamma^n\,\,\underline{u}^{n}
\\
\displaystyle
-(\,\rho\,\tens{K}_{\,e}\ \underline{u}\,)^{n+\theta_S} -\rho\,\tens{K}_{\,d}^n\ \underline{u}^{n}+ (\underline{T}_{s}^{\,exp})^{\,n+\theta_S} + T_s^{\,imp}\,\,\underline {u}^{n} 
\\
\displaystyle
+[\dive (\mu_{\,tot}\,^t\ggrad \underline {u}\,)]^{n+\theta_S}-\frac {2} {3}[\,\grad (\mu_{\,tot}\,\dive \underline {u}\,)]^{n+\theta_S} + (\rho -\rho_0) \underline {g}-(\underline{turb})^{n+\theta_S}
\end{array}
\right.
\end{array}
\end{equation}

d'o� l'�quation r�solue par le sous-programme \fort{codits} :
\begin{equation}\begin{array}{c}
\displaystyle
f_s^{\,imp}(\underline {\widetilde{u}}^{n+1}-\underline {u}^{n}) + \theta\, \dive(\underline{\widetilde{u}}^{n+1} \otimes (\rho
\underline{u})) - \theta\,\dive (\,\mu_{\,tot}\,\ggrad \underline{\widetilde{u}}^{n+1}) = 
\\\\
\displaystyle
-(1-\theta)\,\dive(\underline{u}^{n} \otimes (\rho \underline{u}))+(1-\theta)\,\dive (\,\mu_{\,tot}\,\ggrad \underline{u}^{n})
+ \underline{f}_{\,s}^{\,exp}
\end{array}
\end{equation}
La m\'ethode de discr\'etisation spatiale est d\'evelopp\'ee dans le sous-programme \fort{codits}.\\



\minititre{Remarques :}
{\tiny$\blacksquare$} Dans le cas standard sans extrapolation, le terme
$-\,T_s^{\,imp}$ n'est ajout� � $f_s^{\,imp}$ que s'il est positif afin de ne
pas affaiblir la dominance de la diagonale de la matrice � inverser.\\ 
{\tiny$\blacksquare$} Si une extrapolation est utilis�e, par contre,
$\,T_s^{\,imp}$ est ajout� � $f_s^{\,imp}$ quel que soit son signe. En effet, l'id�e intuitive qui
consiste � prendre~: 
\begin{equation}
\begin{cases}
\displaystyle
(\underline{T}_{s}^{\,exp} + T_{s}^{\,imp}\,\underline {u})^{\,n+\theta_S} &
\text{si } T_{s}^{\,imp} > 0\\ 
\displaystyle
(\underline{T}_{s}^{\,exp})^{\,n+\theta_S} + T_{s}^{\,imp}\,\underline{u}^{n+\theta} &\text{sinon}\\
\end{cases}
\end{equation} 
aboutit � une incoh�rence dans le traitement si $T_s^{imp}$ change de signe
entre deux pas de temps.\\ 
{\tiny$\blacksquare$} la partie diagonale $\tens{K}_{\,d}$ du terme
de perte de charge est utilis�e dans $f_s^{\,imp}$. En fait, pour \^etre rigoureux,
il faudrait ne retenir que les contributions positives (point signal\'e dans le
sous-programme utilisateur associ\'e \fort{uskpdc}). Cette prise en compte sera \`a am\'eliorer.\\
{\tiny$\blacksquare$} Le terme $\theta\,\Gamma^{n}-\theta\,\dive
(\rho\,\underline {u})$ ne pose pas de probl�me pour la 
dominance de la diagonale de la matrice car il est exactement compens� par le
terme de convection (cf. \fort{covofi}). 


%                      Code_Saturne version 1.3
%                      ------------------------
%
%     This file is part of the Code_Saturne Kernel, element of the
%     Code_Saturne CFD tool.
%
%     Copyright (C) 1998-2007 EDF S.A., France
%
%     contact: saturne-support@edf.fr
%
%     The Code_Saturne Kernel is free software; you can redistribute it
%     and/or modify it under the terms of the GNU General Public License
%     as published by the Free Software Foundation; either version 2 of
%     the License, or (at your option) any later version.
%
%     The Code_Saturne Kernel is distributed in the hope that it will be
%     useful, but WITHOUT ANY WARRANTY; without even the implied warranty
%     of MERCHANTABILITY or FITNESS FOR A PARTICULAR PURPOSE.  See the
%     GNU General Public License for more details.
%
%     You should have received a copy of the GNU General Public License
%     along with the Code_Saturne Kernel; if not, write to the
%     Free Software Foundation, Inc.,
%     51 Franklin St, Fifth Floor,
%     Boston, MA  02110-1301  USA
%
%-----------------------------------------------------------------------
%

%%%%%%%%%%%%%%%%%%%%%%%%%%%%%%%%%%
%%%%%%%%%%%%%%%%%%%%%%%%%%%%%%%%%%
\section{Mise en \oe uvre}
%%%%%%%%%%%%%%%%%%%%%%%%%%%%%%%%%%
%%%%%%%%%%%%%%%%%%%%%%%%%%%%%%%%%%
La num\'ero de la phase trait\'ee fait partie des arguments de \fort{turrij}. On
omettra volontairement de le pr\'eciser dans ce qui suit, on indiquera par $(\ )$ la
notion de tableau s'y rattachant.

\etape{Calcul des termes de production $\tens{\mathcal{P}}$}
\begin{itemize}
\item [$\star$] Initialisation \`a z\'ero du tableau \var{PRODUC} dimensionn\'e \`a $\var{NCEL}\times 6$.
\item [$\star$] On appelle trois fois \fort{grdcel} pour calculer les gradients des composantes de la vitesse $u$, $v$ et
$w$ prises au temps $n$.

Au final, on a :\\
$\displaystyle
\begin{array} {ll}
\var{PRODUC(1,IEL)} = & \displaystyle - 2 \left[ R_{11}^{\,n} \frac{\partial u^{\,n}} {\partial x} +R_{12}^{\,n} \frac{\partial u^{\,n}} {\partial y}+R_{13}^{\,n} \frac{\partial u^{\,n}} {\partial z} \right] \text{        (production de $R_{11}^{\,n}$)}\\
\var{PRODUC(2,IEL)} = & \displaystyle - 2 \left[ R_{12}^{\,n} \frac{\partial v^{\,n}} {\partial x} +R_{22}^{\,n} \frac{\partial v^{\,n}} {\partial y}+R_{23}^{\,n} \frac{\partial v^{\,n}} {\partial z} \right] \text{        (production de $R_{22}^{\,n}$)}\\
\var{PRODUC(3,IEL)} = & \displaystyle - 2 \left[ R_{13}^{\,n} \frac{\partial w^{\,n}} {\partial x} +R_{23}^{\,n} \frac{\partial w^{\,n}} {\partial y}+R_{33}^{\,n} \frac{\partial w^{\,n}} {\partial z} \right] \text{        (production de $R_{33}^{\,n}$)}\\
\var{PRODUC(4,IEL)} = & \displaystyle - \left[ R_{12}^{\,n} \frac{\partial u^{\,n}} {\partial x} +R_{22}^{\,n} \frac{\partial u^{\,n}} {\partial y}+R_{23}^{\,n} \frac{\partial u^{\,n}} {\partial z} \right] \\
& \displaystyle - \left[ R_{11}^{\,n} \frac{\partial v^{\,n}} {\partial x} +R_{12}^{\,n} \frac{\partial v^{\,n}} {\partial y}+R_{13}^{\,n} \frac{\partial v^{\,n}} {\partial z} \right] \text{        (production de $R_{12}^{\,n}$)} \\
\var{PRODUC(5,IEL)} = & \displaystyle - \left[ R_{13}^{\,n} \frac{\partial u^{\,n}} {\partial x} +R_{23}^{\,n} \frac{\partial u^{\,n}} {\partial y}+R_{33}^{\,n} \frac{\partial u^{\,n}} {\partial z} \right] \\
& \displaystyle - \left[ R_{11}^{\,n} \frac{\partial w^{\,n}} {\partial x} +R_{12}^{\,n} \frac{\partial w^{\,n}} {\partial y}+R_{13}^{\,n} \frac{\partial w^{\,n}} {\partial z} \right] \text{        (production de $R_{13}^{\,n}$)} \\
\var{PRODUC(6,IEL)} = & \displaystyle - \left[ R_{13}^{\,n} \frac{\partial v^{\,n}} {\partial x} +R_{23}^{\,n} \frac{\partial v^{\,n}} {\partial y}+R_{33}^{\,n} \frac{\partial v^{\,n}} {\partial z} \right] \\
& \displaystyle - \left[ R_{12}^{\,n} \frac{\partial w^{\,n}} {\partial x} +R_{22}^{\,n} \frac{\partial w^{\,n}} {\partial y}+R_{23}^{\,n} \frac{\partial w^{\,n}} {\partial z} \right]  \text{        (production de $R_{23}^{\,n}$)}
\end{array}
$
\end{itemize}

\etape{Calcul du gradient de la masse volumique $\rho^n$ prise au d\'ebut du pas
de temps courant\footnote{{\it i.e.} calcul\'ee \`a partir des
variables du pas de temps pr\'ec\'edent $n$ si n\'ecessaire.} $(n+1)$}
Ce calcul n'a lieu que si les termes de gravit\'e doivent \^etre pris en compte
($\var{IGRARI()} =1$).
\begin{itemize}
\item [$\star$] Appel de \fort{grdcel}  pour calculer le gradient de $\rho^n$
dans les trois directions de l'espace. Les conditions aux limites sur $\rho^n$
sont des conditions de Dirichlet puisque la valeur de $\rho^n$ aux faces de bord
$ik$ (variable \var{IFAC}) est connue et vaut $\rho_{\,b_{\,ik}}$. Pour \'ecrire les conditions aux limites
sous la forme habituelle, $$\rho_{\,b_{\,ik}} = \var{COEFA} + \var{COEFB}
\,\rho^n_{\,I'}$$ on pose alors $\var{COEFA}=
\var{PROPCE(IFAC,IPPROB(IROM(IPHAS)))}$ et $\var{COEFB} = \var{VISCB} = 0$.\\
$\var{PROPCE(1,IPPROB(IROM(IPHAS)))}$ (resp.$\var{VISCB}$) est utilis\'e en lieu
et place de l'habituel \var{COEFA} ($\var{COEFB}$), lors de l'appel \`a \fort{grdcel}.\\
On a donc :\\
$\displaystyle \var{GRAROX}= \frac{\partial \rho^n}{\partial x}\ $,$\displaystyle \ \var{GRAROY}= \frac{\partial
\rho^n}{\partial y}$ et $
\displaystyle \ \var{GRAROZ}= \frac{\partial \rho^n}{\partial z}\ $.

\end{itemize}

Le gradient de $\rho^n$ servira \`a calculer les termes de production par effets de gravit\'e si ces derniers sont pris en compte.

\etape{Boucle \var{ISOU} de $1$ \`a $6$ sur les tensions de Reynolds}
Pour $\var{ISOU} = 1,2,3,4,5,6$, on r\'esout respectivement et dans
l'ordre  les
\'equations de $R_{11}$, $R_{22}$, $R_{33}$, $R_{12}$, $R_{13}$ et $R_{23}$ par
l'appel au sous-programme \fort{resrij}.\\
La r\'esolution se fait par incr\'ement $\delta {R}_{ij}^{\,n+1,k+1}$ , en utilisant la m\^eme m\'ethode que
celle d\'ecrite dans le sous-programme \fort{codits}. On adopte ici les m\^emes notations.
\var{SMBR} est le second membre du syst\`eme \`a inverser, syst\`eme portant sur
les incr\'ements de la variable. \var{ROVSDT} repr\'esente la diagonale de la
matrice, hors convection/diffusion.\\
On va r\'esoudre l'\'equation (\ref{Base_Turrij_Eq_Temp_Rij}) sous forme incr\'ementale en
utilisant \fort{codits}, soit :
\begin{equation}\label{Base_Turrij_Eq_Temp_deltaRij}
\begin{array}{ll}
&\displaystyle \underbrace{\left(\frac {\rho^n_L}{\Delta t^n}
+ \rho^n_L \,C_1\,\frac{\varepsilon^n_L}{k^n_L}(1-\frac{\delta_{ij}}{3})
 - m^n_{\,lm} + \Gamma_L\,+ max(-\alpha^n_{R_{ij}},0)\right)\,|\Omega_l|}
_{\text {$\var{ROVSDT}$ contribuant
\`a la diagonale de la matrice simplifi\'ee de \fort{matrix}}}\,(\delta{R}_{ij}^{\,n+1,p+1})_{\,L}\\\\
&  \underbrace{+\sum\limits_{m\in Vois(l)}\displaystyle \left[
 m^n_{\,lm} \delta R_{ij,\,f_{\,lm}}^{\,n+1,p+1}
- (\mu^n_{\,lm} + \gamma^n_{\,lm})\
\frac{({\delta R}_{ij}^{\,n+1,p+1})_{M}-({\delta R}_{ij}^{\,n+1,p+1})_{L})}{\overline{L'M'}}\,
S_{\,lm} \right]}_{\text { convection upwind pur et diffusion non reconstruite
relatives \`a la matrice simplifi\'ee de \fort{matrix}\footnotemark}} \\
% voir le texte de la footmark plus bas
&= - \displaystyle\frac {\rho^n_L}{\Delta t^n}\,\left(\,(R^{\,n+1,p}_{ij})_L - (R^{\,n}_{ij})_L\,\right)\\
&-\,\underbrace{\displaystyle\int_{\Omega_l} \left(
\dive\,[\,(\rho\,\vect{u})^n\,R^{\,n+1,p}_{ij} - (\mu^n\,+ \gamma^n\,)\,
\grad{R^{\,n+1,p}_{ij}}\,]\right)\,d\Omega}_{\text {convection et diffusion
trait\'ees par \fort{bilsc2}}}\\
&+\displaystyle \int_{\Omega_l} \left[\,\mathcal{P}^{\,n+1,p}_{ij} + \mathcal{G}^{\,n+1,p}_{ij}
- \displaystyle\rho^n \,C_1\,\frac{\varepsilon^n}{k^n}\left[R^{\,n+1,p}_{ij}-
\frac{2}{3}\,k^n\,\delta_{ij}\right] + \phi^{\,n+1,p}_{ij,2} +
\phi^{\,n+1,p}_{ij,w}\,\right]\, d\Omega \\
& + \displaystyle\int_{\Omega_l} \left[- \frac{2}{3} \rho^n \varepsilon^n \delta_{ij}
 + \Gamma\,(\,R^{\,in}_{ij} - R^{\,n+1,p}_{ij}\,) +
\alpha^n_{R_{ij}}\,R^{\,n+1,p}_{ij}+ \beta^n_{R_{ij}}\right]\, d\Omega\\
&+ \sum\limits_{m\in
Vois(l)}\displaystyle \left[\ \tens{E}^n\,\grad{R}^{\,n+1,p}_{ij} \right]_{\,lm}\,.\,\vect{n}_{\,lm}S_{\,lm}\\
&+ \sum\limits_{m\in Vois(l)}\displaystyle \left[\
\tens{D}^n\,\grad{R}^{\,n+1,p}_{ij} \right]_{\,lm}\,.\,\vect{n}_{\,lm}S_{\,lm}\\
&- \sum\limits_{m\in Vois(l)} \gamma^n_{\,lm} \left( \grad{R}^{\,n+1,p}_{ij}\,.\,\vect{n}_{\,lm} \right)  S_{\,lm}\\
&+ \sum\limits_{m\in Vois(l)}  m^n_{\,lm}\,(R^{\,n+1,p}_{ij})_L\\
\end{array}
\end{equation}
% si on ne fait pas comme ca, il n'apparait pas
\footnotetext[\thefootnote]{Si $\var{IRIJNU} = 1$, on remplace  $\mu^n_{\,lm}$ par $(\mu +
\mu_t)^n_{\,lm}$ dans l'expression de la diffusion non reconstruite
associ\'ee \`a la matrice simplifi\'ee de \fort{matrix} ($\mu_t$ d\'esigne la
viscosit\'e turbulente calcul\'ee comme en $k-\varepsilon$).}

o\`u on rappelle :\\
pour $n$ donn\'e entier positif, on d\'efinit la suite
 $({R}_{ij}^{\,n+1,p})_{p \in \grandN}$
 par :
\begin{equation}\notag
\left\{\begin{array}{l}
{R}_{ij}^{\,n+1,0} = {R}_{ij}^{\,n}\\
{R}_{ij}^{\,n+1,p+1} = {R}_{ij}^{\,n+1,p} + \delta{R}_{ij}^{\,n+1,p+1} \\
\end{array}\right.
\end{equation}
$(\delta{R}_{ij}^{\,n+1,p+1})_{\,L}$ d\'esigne la valeur sur l'\'el\'ement
$\Omega_l$ du $\text{$(\,p+1\,)$-i\`eme}$ incr\'ement de ${R}_{ij}^{\,n+1}$,
$ m^n_{\,lm}$ le flux de masse \`a l'instant $n$ \`a travers la face $lm$,
$\delta R_{ij,\,f_{\,lm}}^{\,n+1,p+1}$ vaut $({\delta
R}_{ij}^{\,n+1,p+1})_{L}$  si $ m^n_{\,lm} \geqslant 0$, $({\delta
R}_{ij}^{\,n+1,p+1})_{M}$ sinon,
$\mathcal{P}^{\,n+1,p}_{ij}$, $\phi^{\,n+1,p}_{ij,2}$, $\phi^{\,n+1,p}_{ij,w}$ les valeurs
des quantit\'es associ\'ees correspondant \`a l'incr\'ement
$(\delta{R}_{ij}^{\,n+1,p})$.\\



Tous ces termes sont calcul\'es comme suit :
\begin{itemize}
\item Terme de gauche de l'\'equation (\ref{Base_Turrij_Eq_Temp_deltaRij})\\
Dans \fort{resrij} est calcul\'ee la variable \var{ROVSDT}. Les autres
termes sont compl\'et\'es par \fort{codits}, lors de la construction de la matrice simplifi\'ee , {\it via} un
appel au sous-programme \fort{matrix}. La quantit\'e
 $(\mu^n_{\,lm} + \gamma^n_{\,lm})$ \`a la face $lm$ est calcul\'ee lors de l'appel \`a
\fort{visort}.\\
\item Second membre de l'\'equation (\ref{Base_Turrij_Eq_Temp_deltaRij})\\
Le premier terme non d\'etaill\'e est calcul\'e par le sous-programme
\fort{bilsc2}, qui applique le sch\'ema convectif choisi par l'utilisateur, qui
reconstruit ou non selon le souhait de l'utilisateur les gradients intervenants
dans la convection-diffusion.\\
Les termes sans accolade sont, eux, compl\`etement explicites et ajout\'es au fur et
\`a mesure dans \var{SMBR} pour former
l'expression $f^{\,exp}_s$ de \fort{codits}.
\end{itemize}
On d\'ecrit ci-dessous les \'etapes de \fort{resrij} :
\begin{itemize}

\item DELTIJ = 1, pour $\var{ISOU} \leqslant 3$ et DELTIJ = 0  Si $\var{ISOU} >
3$. Cette valeur repr\'esente le symbole de Kroeneker $\delta_{ij}$.

\item Initialisation \`a z\'ero de \var{SMBR} (tableau contenant le second
membre) et \var{ROVSDT} (tableau contenant la diagonale de la matrice sauf celle
relative \`a la contribution de la
diagonale des op\'erateurs de convection et de diffusion lin\'earis\'es
\footnote{qui correspondent aux sch\'emas convectif upwind pur et diffusif sans
reconstruction.}), tous deux de dimension $\var{NCEL}$.

\item Lecture et prise en compte des termes sources utilisateur pour la variable $R_{ij}$

Appel \`a \fort{ustsri} pour charger les termes sources utilisateurs. Ils sont
stock\'es comme suit. Pour la cellule $\Omega_l$ de centre $L$, repr\'esent\'ee par $\var{IEL}$, on a :\\
\begin{equation}\notag
\left\{\begin{array}{lll}
&\var{ROVSDT(IEL)}&= |\Omega_l| \ \alpha_{R_{ij}}\\
&\var{SMBR(IEL)}&=|\Omega_l| \ \beta_{R_{ij}}\\
\end{array}\right.
\end{equation}
On affecte alors les valeurs ad\'equates au second membre \var{SMBR} et \`a la
diagonale \var{ROVSDT} comme suit :
\begin{equation}\notag
\left\{\begin{array}{lll}
&\var{SMBR(IEL)} &= \var{SMBR(IEL)} +\ |\Omega_l| \ \alpha_{R_{ij}} \ (R^n_{ij})_L \\
&\var{ROVSDT(IEL)}&= \text{max }(-\ |\Omega_l| \ \alpha_{R_{ij}},0)\\
\end{array}\right.
\end{equation}
La valeur de $ \var{ROVSDT}$ est ainsi calcul\'ee pour des raisons de stabilit\'e
num\'erique. En effet, on ne rajoute sur la diagonale que les valeurs positives,
ce qui correspond physiquement \`a impliciter les termes de rappel uniquement.
\item{Calcul du terme source de masse  si $\Gamma_L > 0$}

Appel de \fort{catsma} et incr\'ementation si n\'ecessaire de \var{SMBR} et
\var{ROVSDT} {\it via} :\\
\begin{equation}\notag
\left\{\begin{array}{lll}
\displaystyle \var{SMBR(IEL)} = \var{SMBR(IEL)} + |\Omega_l| \ \Gamma_L \
\left[(R^{\,in}_{ij})_L - (R^{\,n}_{ij})_L \right] \\
\displaystyle \var{ROVSDT(IEL)}=\var{ROVSDT(IEL)} + |\Omega_l| \ \Gamma_L
\end{array}\right.
\end{equation}
\item Calcul du terme d'accumulation de masse et du terme instationnaire

On stocke $\displaystyle \var{W1}= \int_{\Omega_l}\dive\,(\rho\,\vect{u})\,d\Omega$
calcul\'e par \fort{divmas} \`a l'aide des flux de masse aux faces internes
$ m^n_{\,lm}=\sum\limits_{m\in Vois(l)}{(\rho \vect{u})_{\,lm}^n} \text{.}\,
\vect{S}_{\,lm} $ (tableau \var{FLUMAS}) et des flux de masse aux bords  $ m^n_{\,b_{lk}} = \sum\limits_{k\in{\gamma_b(l)}}{(\rho \vect{u})_{\,{b}_{lk}}^n} \text{.}\,
\vect{S}_{\,{b}_{lk}} $ (tableau \var{FLUMAB}).
On incr\'emente ensuite \var{SMBR} et \var{ROVSDT}.
\begin{equation}\notag
\left\{\begin{array}{lll}
&\var{SMBR(IEL)} &= \var{SMBR(IEL)} + \var{ICONV}\  (R^n_{ij})_L\,(\displaystyle
\int_{\Omega_l}\dive\,(\rho\,\vect{u})\ d\Omega) \\
&\var{ROVSDT(IEL)}& = \var{ROVSDT(IEL)} +  \var{ISTAT}\,\displaystyle
\frac{\rho^n_L \ |\Omega_l|}{\Delta t^n} -  \var{ICONV}\ (\displaystyle
\int_{\Omega_l}\dive\,(\rho\,\vect{u})\ d\Omega) \\
\end{array}\right.
\end{equation}
\item Calcul des termes sources de production, des termes $\displaystyle
\phi_{\,ij,1}+\phi_{\,ij,2}$ et de la dissipation~$\displaystyle-\frac{2}{3} \varepsilon\,\delta_{\,ij}$ :

On effectue une boucle d'indice \var{IEL} sur les cellules $\Omega_l$ de centre $L$ :
\begin{itemize}
\item [$\Rightarrow$] $\displaystyle \var{TRPROD}= \frac{1}{2} (\mathcal{P}^n_{ii})_L = \frac{1}{2} \left[ \var{PRODUC(1,IEL)} +  \var{PRODUC(2,IEL)} +  \var{PRODUC(3,IEL)} \right] $
\item [$\Rightarrow$] $\displaystyle \var{TRRIJ }= \frac{1}{2} (R^n_{ii})_L $
\item [$\Rightarrow$] $\displaystyle \var{SMBR(IEL)} =\ \var{SMBR(IEL)}\ +$\\
$\ \displaystyle\rho^n_L |\Omega_l| \left[ \displaystyle
\frac{2}{3}\,\delta_{\,ij} \left( \ \displaystyle \frac{ C_2}{2}\,(\mathcal{P}^n_{ii})_L\ +
(C_1-1)\ \varepsilon^n_L\, \right)\right.$\\
$ + \left.\ (1-C_2) \ \var{PRODUC(ISOU,IEL)} -
\displaystyle C_1\ \frac{2\,\varepsilon^n_L}{(R^n_{ii})_L}\ (R^n_{ij})_L \right]$
\item [$\Rightarrow$] $\displaystyle \var{ROVSDT(IEL)} = \var{ROVSDT(IEL)} +
\rho^n_L \ |\Omega_l| \ (- \displaystyle \frac{1}{3} \ \,\delta_{\,ij} + 1) \ C_1
\ \frac{2\ \varepsilon^n_L}{(R^n_{ii})_L}$
\end{itemize}
\item Appel de \fort{rijech} pour le calcul des termes d'\'echo de paroi
 $\phi^n_{ij,w}$ si $\var{IRIJEC()}=1$ et ajout dans \var{SMBR}.\\
$\var{SMBR} = \var{SMBR} + \phi^n_{ij,w}$\\
Suivant son mode de calcul (\var{ICDPAR}), la distance � la paroi est directement accessible
par \var{RA(IDIPAR+IEL-1)} (\var{|ICDPAR|} = 1) ou bien
est calcul\'ee \`a partir de $\var{IA(IIFAPA(IPHAS)+IEL - 1)}$,
qui donne pour l'\'el\'ement $\var{IEL}$ le num\'ero de la face de bord
paroi la plus  proche (\var{|ICDPAR|} = 2). Ces tableaux ont \'et\'e renseign\'e une fois pour toutes au
d\'ebut de calcul.

\item  Appel de \fort{rijthe} pour le calcul des termes de gravit\'e $\mathcal{G}^n_{ij}$ et ajout dans \var{SMBR}.

Ce calcul n'a lieu que si $\var{IGRARI()} = 1$.
$ \var{SMBR} = \var{SMBR} + \mathcal{G}^n_{ij}$
\item Calcul de la partie explicite du terme de diffusion
 $\dive{\,\left[\tens{A}\,\grad{R}^{\,n}_{ij}\right]}$, plus pr\'ecis\'ement
des contributions du terme extradiagonal pris aux faces purement internes
(remplissage du tableau \var{VISCF}), puis aux faces de bord (remplissage du
tableau \var{VISCB}).
\begin{itemize}
\item [$\star$] Appel de \fort{grdcel} pour le calcul du gradient de
$R^{\,n}_{ij}$ dans chaque direction. Ces gradients sont respectivement
stock\'es dans les tableaux de travail \var{W1}, \var{W2} et \var{W3}.

\item [$\star$] boucle d'indice \var{IEL} sur les cellules $\Omega_l$ de centre
$L$ pour le
calcul de $\tens{E}^n\,\grad{R}^{\,n}_{ij}$ aux cellules dans un premier temps :\\
\begin{itemize}
\item [$\Rightarrow$] $\displaystyle \var{TRRIJ}= \frac{1}{2} (R^{\,n}_{ii})_L $
\item [$\Rightarrow$] $\displaystyle \var{CSTRIJ} = \rho^n_L\ C_S \ \displaystyle\frac{(R^n_{ii})_L}{2\,\varepsilon^n_L}$
\item [$\Rightarrow$] $\displaystyle \var{W4(IEL)} = \rho^n_L\ C_S\
\displaystyle\frac{(R^n_{ii})_L}{2\,\varepsilon^n_L} \left[\,(R^{\,n}_{12})_L \ \var{W2(IEL)} +
(R^{\,n}_{13})_L \ \var{W3(IEL)}\,\right]$
\item [$\Rightarrow$] $\displaystyle \var{W5(IEL)} = \rho^n_L\ C_S\
\displaystyle\frac{(R^n_{ii})_L}{2\,\varepsilon^n_L} \left[\,(R^{\,n}_{12})_L \ \var{W1(IEL)} +
(R^{\,n}_{23})_L \ \var{W3(IEL)}\,\right]$
\item [$\Rightarrow$] $\displaystyle \var{W6(IEL)} = \rho^n_L\ C_S\
\displaystyle\frac{(R^n_{ii})_L}{2\,\varepsilon^n_L} \left[\,(R^{\,n}_{13})_L \ \var{W1(IEL)} + (R^{\,n}_{23})_L \ \var{W2(IEL)}\,\right]$
\end{itemize}



\item [$\star$] Appel de \fort{vectds}\footnote{Le r\'esultat est stock\'e dans
\var{VISCF} et \var{VISCB}. Dans \fort{vectds}, les valeurs aux faces internes
sont interpol\'ees lin\'eairement sans reconstruction et \var{VISCB} est mis \`a
z\'ero.} pour assembler $\displaystyle\left[ \tens{E}^n\,\grad{R}^{\,n}_{ij}
\right]\,.\,\vect{n}_{\,lm}S_{\,lm}$ aux faces $lm$.
\item [$\star$] Appel de \fort{divmas} pour calculer la divergence du flux d\'efini par \var{VISCF} et \var{VISCB}.
Le r\'esultat est stock\'e dans \var{W4}.\\
Ajout au second membre \var{SMBR}.\\
\var{SMBR} = \var{SMBR} + \var{W4}
\end{itemize}

A l'issue de cette \'etape, seule la partie extradiagonale de la diffusion prise
enti\`erement explicite~:
 $$\sum\limits_{m\in
Vois(l)}\left[\ \tens{E}^n\,\grad{R}^{\,n}_{ij} \right]_{\,lm}\,.\,\vect{n}_{\,lm}S_{\,lm}$$ a \'et\'e calcul\'ee.\\

\item Calcul de la partie diagonale du terme de diffusion\footnote{Seule la
composante normale  du  gradient de $R_{ij}$ aux faces sera implicite.} :\\
Comme on l'a d\'eja vu, une partie de cette contribution sera trait\'ee en
implicite (celle relative \`a la composante normale du gradient) et donc
ajout\'ee au second membre par \fort{bilsc2} ; l'autre
partie sera explicite et prise en compte dans $f_s^{\,exp}$.
\begin{itemize}
\item [$\star$] On effectue une boucle d'indice \var{IEL} sur les cellules
$\Omega_l$ de centre $L$ :
\begin{itemize}
\item [$\Rightarrow$] $\displaystyle \var{TRRIJ }= \frac{1}{2} (R^{\,n}_{ii})_L $
\item [$\Rightarrow$] $\displaystyle \var{CSTRIJ} = \rho^n_L \ C_S \ \frac{(R^{\,n}_{ii})_L}{2\,\varepsilon^n_L}$
\item [$\Rightarrow$] $\displaystyle \var{W4(IEL)} = \rho^n_L \ C_S \
\frac{(R^{\,n}_{ii})_L}{2\,\varepsilon^n_L} \ (R^{\,n}_{11})_L$
\item [$\Rightarrow$] $\displaystyle \var{W5(IEL)} = \rho^n_L \ C_S \ \frac{(R^{\,n}_{ii})_L}{2\,\varepsilon^n_L}\ (R^n_{22})_L$
\item [$\Rightarrow$] $\displaystyle \var{W6(IEL)} = \rho^n_L \ C_S \ \frac{(R^{\,n}_{ii})_L}{2\,\varepsilon^n_L} \ (R^n_{33})_L$
\end{itemize}

%\item Traitement du parall\'elisme et de la p\'eriodicit\'e.

\item [$\star$] On effectue une boucle d'indice \var{IFAC} sur les faces
purement internes $lm$ pour remplir le tableau \var{VISCF} :
\begin{itemize}
\item [$\Rightarrow$] Identification des cellules $\Omega_l$ et $\Omega_m$ de
centre respectif $L$ (variable \var{II}) et $M$ (variable \var{JJ}), se trouvant de chaque c\^ot\'e de la face
$lm$\footnote{La normale \`a la face est orient\'ee de L vers M.}.
\item [$\Rightarrow$] Calcul du carr\'e de la surface de la face. La valeur est
stock\'ee dans le tableau \var{SURFN2}.
\item [$\Rightarrow$] Interpolation du gradient de $R^{\,n}_{ij}$ \`a la face
$lm$ (gradient facette $\left[\grad{R}^{\,n}_{ij}\right]_{\,lm}$) :
\begin{equation}\notag
\left\{\begin{array}{ll}
\var{GRDPX} &= \displaystyle \frac{1}{2} \left(\var{W1(II)} + \var{W1(JJ)}
\right) \\
&\\
\var{GRDPY} &= \displaystyle \frac{1}{2} \left(\var{W2(II)} + \var{W2(JJ)} \right) \\
&\\
\var{GRDPZ} &= \displaystyle \frac{1}{2} \left(\var{W3(II)} + \var{W3(JJ)} \right)
\end{array}\right.
\end{equation}
\item [$\Rightarrow$] Calcul du gradient de $R^{\,n}_{ij}$ normal \`a la face
$lm$, $\left[\grad{R}^{\,n}_{ij}\right]_{\,lm}.\vect{n}_{\,lm}\,S_{\,lm}$.\\

$\displaystyle \var{GRDSN} =  \var{GRDPX} \ \var{SURFAC(1,IFAC)} + \var{GRDPY} \ \var{SURFAC(2,IFAC)} +  \var{GRDPZ} \ \var{SURFAC(3,IFAC)}$
$\var{SURFAC}$ est le vecteur surface de la face \var{IFAC}.


\item [$\Rightarrow$] calcul de
 $\left[\grad{R^{\,n}_{ij}} - (\grad
R^{\,n}_{ij}\,.\,\vect{n}_{\,lm})\vect{n}_{\,lm}\right]$, les vecteurs \'etant
calcul\'es \`a la face $lm$ :
\begin{equation}\notag
\left\{\begin{array}{lll}
&\displaystyle \var{GRDPX} &= \var{GRDPX} - \displaystyle\frac{\var{GRDSN}}{\var{SURFN2}} \ \var{SURFAC(1,IFAC)}\\
&&\\
&\displaystyle \var{GRDPY} &= \var{GRDPY} - \displaystyle\frac{\var{GRDSN}}{\var{SURFN2}} \ \var{SURFAC(2,IFAC)} \\
&&\\
&\displaystyle \var{GRDPZ} &= \var{GRDPZ} - \displaystyle\frac{\var{GRDSN}}{\var{SURFN2}} \ \var{SURFAC(3,IFAC)}
\end{array}\right.
\end{equation}
\item [$\Rightarrow$] finalisation du calcul de l'expression totalement
explicite
 $$\left[ \tens{D}^n\,\left( \grad{R^{\,n}_{ij}} - (\grad R^{\,n}_{ij}\,.\,\vect{n}_{\,lm})\,\vect{n}_{\,lm}\right) \right]\,.\,\vect{n}_{\,lm}$$
\begin{equation}\notag
\begin{array} {ll}
\displaystyle \var{VISCF} = &
 \displaystyle\frac{1}{2} (\ \var{W4(II)} +\ \var{W4(JJ)}) \ \var{GRDPX} \
\var{SURFAC(1,IFAC)})\ + \\
&\\
&  \displaystyle\frac{1}{2} (\ \var{W5(II)} +\ \var{W5(JJ)}) \ \var{GRDPY} \
\var{SURFAC(2,IFAC)})\ + \\
&\\
&  \displaystyle\frac{1}{2} (\ \var{W6(II)} +\ \var{W6(JJ)}) \ \var{GRDPZ} \ \var{SURFAC(3,IFAC)})
\end{array}
\end{equation}
\end{itemize}

\item [$\star$] Mise \`a z\'ero du tableau \var{VISCB}.

\item [$\star$] Appel de \fort{divmas} pour calculer la divergence de~:
 $$\tens{D}^{\,n}\,\left( \grad{R^{\,n}_{ij}} - (\grad R^{\,n}_{ij}\,.\,\vect{n}_{\,lm})\vect{n}_{\,lm}\right)$$ d\'efini aux faces dans \var{VISCF} et \var{VISCB}.

Le r\'esultat est stock\'e dans le tableau \var{W1}.\\
Ajout au second membre \var{SMBR}.\\
$\var{SMBR} = \var{SMBR} + \var{W1}$
\end{itemize}
\item Calcul de la viscosit\'e orthotrope $\gamma^n_{\,lm}$ \`a la face $lm$ de la variable principale
$R^{\,n}_{ij}$\\
Ce calcul permet au sous-programme \fort{codits} de compl\'eter le second membre
\var{SMBR} par :
\begin{equation}
\begin{array} {ll}
& \sum\limits_{m\in Vois(l)}
\mu^n_{\,lm}\,\left(\grad{R}^{\,n}_{ij}\,.\,\vect{n}_{\,lm}\right)S_{\,lm}
 + \sum\limits_{m\in Vois(l)} \left(\grad{R}^{\,n}_{ij}
\,.\,\vect{n}_{\,lm}\right)\left[\tens{D}^{\,n}\,\vect{n}_{\,lm}\right]_{\,lm}\,.\,\vect{n}_{\,lm}\
S_{\,lm}\\
& = \sum\limits_{m\in Vois(l)}(\,\mu^n_{\,lm}\, + \,\gamma^n_{\,lm}\,)\,\left(\grad{R}^{\,n}_{ij}\,.\,\vect{n}_{\,lm}\right)S_{\,lm}
\end{array}
\end{equation}
sans pr\'eciser la nature de la face $lm$, {\it via} l'appel \`a \fort{bilsc2}  et de disposer de la quantit\'e
$(\mu^n_{\,lm}\, + \gamma^n_{\,lm})$ pour construire sa
matrice simplifi\'ee.\\
\begin{itemize}
\item [$\star$] On effectue une boucle d'indice \var{IEL} sur les cellules
$\Omega_l$ :
\begin{itemize}
\item [$\Rightarrow$] $\displaystyle \var{TRRIJ }= \frac{1}{2} (R^{\,n}_{ii})_L $
\item [$\Rightarrow$] $\displaystyle \var{RCSTE} = \rho^n_L \ C_S \ \frac{ (R^{\,n}_{ii})_L}{2\,\varepsilon^n_L} $
\item [$\Rightarrow$] $\displaystyle \var{W1(IEL)} = \mu^n +\rho^n_L \ C_S \ \frac{
(R^{\,n}_{ii})_L}{2\,\varepsilon^n_L}\ (R^n_{11})_L$
\item [$\Rightarrow$] $\displaystyle \var{W2(IEL)} = \mu^n + \rho^n_L \ C_S \ \frac{ (R^{\,n}_{ii})_L}{2\,\varepsilon^n_L}\ (R^n_{22})_L$
\item [$\Rightarrow$] $\displaystyle \var{W3(IEL)} = \mu^n + \rho^n_L \ C_S \ \frac{ (R^{\,n}_{ii})_L}{2\,\varepsilon^n_L}\ (R^n_{33})_L$
\end{itemize}

\item [$\star$] Appel de \fort{visort} pour calculer la viscosit\'e orthotrope
\footnote{Comme dans le sous-programme \fort{viscfa}, on multiplie la viscosit\'e par
$\displaystyle \frac{S_{\,lm}}{\overline{L'M'}}$, o\`u $S_{\,lm}$ et
$\overline{L'M'}$ repr\'esentent respectivement la surface de la face $lm$ et la
mesure alg\'ebrique du segment reliant les projections des centres des cellules
voisines sur la normale \`a la face. On garde dans ce sous-programme  la possibilit\'e d'interpoler la viscosit\'e aux cellules lin\'eairement ou d'utiliser une moyenne harmonique. La viscosit\'e au bord est celle de la cellule de bord correspondante.}
$ \gamma^n_{\,lm} = (\tens{D}^{\,n}\,\vect{n}_{\,lm}).\vect{n}_{\,lm}$ aux faces $lm$

Le r\'esultat est stock\'e dans les tableaux \var{VISCF} et \var{VISCB}.
\end{itemize}

\item appel de \fort{codits} pour la r\'esolution de l'\'equation de
convection/diffusion/termes sources de la variable $R_{ij}$. Le terme source est
\var{SMBR}, la viscosit\'e \var{VISCF} aux faces purement internes (
resp. \var{VISCB} aux faces de bord) et \var{FLUMAS} le flux de masse interne
 ( resp. \var{FLUMAB} flux de masse au bord). Le r\'esultat est la variable $R_{ij}$ au temps
$n+1$, donc $R^{\,n+1}_{ij}$. Elle est stock\'ee dans le tableau \var{RTP} des
variables mises \`a jour.

\end{itemize}

\etape{Appel de \fort{reseps} pour la r\'esolution de la variable $\varepsilon$}

On d\'ecrit ci-dessous le sous-programme \fort{reseps}, les commentaires du sous-programme \fort{resrij} ne seront pas r\'ep\'et\'es puisque les deux sous-programmes ne diff\`erent que par leurs termes sources.

\begin{itemize}
\item Initialisation \`a z\'ero de \var{SMBR} et \var{ROVSDT}.

\item{Lecture et prise en compte des termes sources utilisateur pour la variable $\varepsilon$ :}

Appel de \fort{ustsri} pour charger les termes sources utilisateurs. Ils sont
stock\'es dans les tableaux suivants :\\
pour la cellule $\Omega_l$ repr\'esent\'ee par $\var{IEL}$ de centre $L$, on a :
\begin{equation}\notag
\left\{\begin{array}{lll}
&\var{ROVSDT(IEL)}&= |\Omega_l| \ \alpha_{\varepsilon}\\
&\var{SMBR(IEL)}&=|\Omega_l| \ \beta_{\varepsilon}\\
\end{array}\right.
\end{equation}
On affecte alors les valeurs ad\'equates au second membre \var{SMBR} et \`a la
diagonale \var{ROVSDT} comme suit :
\begin{equation}\notag
\left\{\begin{array}{lll}
&\var{SMBR(IEL)} &= \var{SMBR(IEL)} +\ |\Omega_l| \ \alpha_{\,\varepsilon} \
\varepsilon^n_L \\
&\var{ROVSDT(IEL)}&= \text{max }(-\ |\Omega_l| \ \alpha_{\,\varepsilon},0)\\
\end{array}\right.
\end{equation}

\item{Calcul du terme source de masse si $\Gamma_L > 0$ :
\begin{equation}\notag
\left\{\begin{array}{lll}
&\displaystyle \var{SMBR(IEL)} = \var{SMBR(IEL)} + |\Omega_l| \ \Gamma_L \
(\varepsilon^{\,in}_L -\varepsilon^n_L) \\
&\displaystyle \var{ROVSDT(IEL)}= \var{ROVSDT(IEL)} + |\Omega_l| \ \Gamma_L
\end{array}\right.
\end{equation}
\item Calcul du terme d'accumulation de masse et du terme instationnaire \\
On stocke $\displaystyle \var{W1}= \int_{\Omega_l}\dive\,(\rho\,\vect{u})\,d\Omega$
calcul\'e par \fort{divmas} \`a l'aide des flux de masse internes et aux bords.\\
On incr\'emente ensuite \var{SMBR} et \var{ROVSDT}.
\begin{equation}\notag
\left\{\begin{array}{lll}
&\var{SMBR(IEL)} &= \var{SMBR(IEL)} + \var{ICONV}\ \varepsilon^n_L\,(\displaystyle
\int_{\Omega_l}\dive\,(\rho\,\vect{u})\ d\Omega) \\
&\var{ROVSDT(IEL)}& = \var{ROVSDT(IEL)} +  \var{ISTAT}\,\displaystyle
\frac{\rho^n_L \ |\Omega_l|}{\Delta t^n} -  \var{ICONV}\ (\displaystyle
\int_{\Omega_l}\dive\,(\rho\,\vect{u})\ d\Omega) \\
\end{array}\right.
\end{equation}

\item Traitement du terme de production
 $\displaystyle \rho\,C_{\varepsilon_1}\,\frac{\varepsilon}{k}\,\mathcal{P}$
 et du terme de dissipation $-\,\displaystyle \rho\,C_{\varepsilon_2}\,\frac{\varepsilon}{k}\,\varepsilon$ \\
pour cela, on effectue une boucle d'indice \var{IEL} sur les cellules $\Omega_l$
de centre $L$ :
\begin{itemize}
\item [$\Rightarrow$] $\displaystyle \var{TRPROD}= \frac{1}{2} (\mathcal{P}^n_{ii})_L = \frac{1}{2} \left[ \var{PRODUC(1,IEL)} +  \var{PRODUC(2,IEL)} +  \var{PRODUC(3,IEL)} \right] $
\item [$\Rightarrow$] $\displaystyle \var{TRRIJ }= \frac{1}{2} (R^n_{ii})_L $
\item [$\Rightarrow$] $\displaystyle \var{SMBR(IEL)} = \var{SMBR(IEL)} + \rho^n_L
|\Omega_l| \left[ -C_{\varepsilon_2} \ \frac{2\,(\varepsilon^n_L)^2}{(R^n_{ii})_L} + C_{\varepsilon_1} \ \frac{\varepsilon^n_L}{(R^n_{ii})_L}\ (\mathcal{P}^n_{ii})_L \right] $
\item [$\Rightarrow$] $\displaystyle \var{ROVSDT(IEL)} = \var{ROVSDT(IEL)} + C_{\varepsilon_2} \ \rho^n_L \ |\Omega_l| \ \frac{2\,\varepsilon^n_L}{(R^n_{ii})_L}$
\end{itemize}

\item Appel de \fort{rijthe} pour le calcul des termes de gravit\'e $\mathcal{G}^n_{\varepsilon}$ et ajout dans \var{SMBR}.

$ \var{SMBR} = \var{SMBR} + \mathcal{G}^n_{\varepsilon}$\\
Ce calcul n'a lieu que si $\var{IGRARI()} = 1$.

\item Calcul de la diffusion de $\varepsilon$ \\
 Le terme $\dive \left[\mu\, \grad(\varepsilon) + \tens{A'}\,\grad(\varepsilon)
\right]$ est calcul\'e exactement de la m\^eme mani\`ere que pour les tensions
de Reynolds $R_{ij}$ en rempla\c cant $\tens{A}$ par $\tens{A'}$.

\item Appel de \fort{codits} pour la r\'esolution de l'\'equation de
convection/diffusion/termes sources de la variable principale $\varepsilon$. Le
r\'esultat $\varepsilon^{\,n+1}$ est stock\'e dans le tableau \var{RTP} des
variables mises \`a jour.
}
\end{itemize}

\etape{clippings finaux}
On passe enfin dans le sous-programme  \fort{clprij} pour faire un clipping \'eventuel
des variables $R^{\,n+1}_{ij}$ et $\varepsilon^{\,n+1}$. Le sous-programme
\fort{clprij} est appel\'e\footnote{L'option
$\var{ICLIP} = 1$ consiste en un clipping minimal des variables $R_{ii}$ et
$\varepsilon$ en prenant la valeur absolue de ces variables puisqu'elles ne
peuvent \^etre que positives.} avec $\var{ICLIP} = 2$ . Cette option
\footnote{Quand la valeur des grandeurs $R_{ii}$ ou $\varepsilon$ est
n\'egative, on la remplace par le minimum entre sa valeur absolue et (1,1)
fois la valeur obtenue au pas de temps pr\'ec\'edent.} contient l'option $\var{ICLIP} = 1$  et permet de v\'erifier l'in\'egalit\'e de Cauchy-Schwarz sur les grandeurs extra-diagonales du tenseur $\tens{R}$ (pour $i \neq j$, $|R_{ij}|^2 \le R_{ii} R_{jj}$).


%%%%%%%%%%%%%%%%%%%%%%%%%%%%%%%%%%
%%%%%%%%%%%%%%%%%%%%%%%%%%%%%%%%%%
\section{Points \`a traiter}
%%%%%%%%%%%%%%%%%%%%%%%%%%%%%%%%%%
%%%%%%%%%%%%%%%%%%%%%%%%%%%%%%%%%%
Sauf mention explicite, $\phi$ repr\'esentera une tension de Reynolds ou la dissipation turbulente ($\phi = R_{ij} \ \text{ou} \ \varepsilon$).

\begin{itemize}
\item {La vitesse utilis\'ee pour le calcul de la production est explicite. Est-ce qu'une implicitation peut am\'eliorer la pr\'ecision temporelle de $\phi$ \footnote{Cette remarque peut \^etre g\'en\'eralis\'ee. En effet, peut-on envisager d'actualiser les variables d\'ej\`a r\'esolues (sans r\'eactualiser les variables turbulentes apr\`es leur r\'esolution)? Ceci obligerait \`a modifier les sous-programmes tels que \fort{condli} qui sont appel\'es au d\'ebut de la boucle en temps.} ?}
\item {Dans quelle mesure le terme d'\'echo de paroi est-il valide ? En effet, ce terme est remis en question par certains auteurs.}
\item {On peut envisager la r\'esolution d'un syst\`eme hyperbolique pour les
tensions de Reynolds afin d'introduire un couplage avec le champ de vitesse.}
\item {Le flux au bord \var{VISCB} est annul\'e dans le sous-programme
\fort{vectds}. Peut-on envisager de mettre au bord la valeur de la variable
concern\'ee \`a la cellule de bord correspondant? De m\^eme, il faudrait se
pencher sur les hypoth\`eses sous-jacentes \`a l'annulation des contributions
aux bords de \var{VISCB} lors du calcul de : $$\left[ \tens{D}^n\,\left( \grad{R^{\,n}_{ij}} - (\grad R^{\,n}_{ij}\,.\,\vect{n}_{\,lm})\,\vect{n}_{\,lm}\right) \right]\,.\,\vect{n}_{\,lm}.$$}
\item {Un probl\`eme de pond\'eration appara\^\i t plus g\'en\'eralement. Si on prend la partie explicite de $\tens{D}\,\grad(\phi)$, la pond\'eration aux faces internes utilise le coefficient $\displaystyle\frac{1}{2}$ avec pond\'eration s\'epar\'ee de $\tens{D}$ et $\grad(\phi)$, alors que pour $\tens{E}\,\grad(\phi)$, on calcule d'abord ce terme aux cellules pour ensuite l'interpoler lin\'eairement aux faces \footnote{Cette interpolation se fait dans \fort{vectds} avec des coefficients de pond\'eration aux faces.}. Ceci donne donc deux types d'interpolations pour des termes de m\^eme nature.}
\item {On laisse la possibilit\'e dans \fort{visort} d'utiliser une moyenne
harmonique aux faces. Est-ce que ceci est valable puisque les interpolations
utilis\'ees lors du calcul de la partie explicite de $\tens{A}\,\grad{\phi}$
sont des moyennes arithm\'etiques ?}
\item {Les techniques adopt\'ees lors du clipping sont \`a revoir.}
\item {On utilise dans le cadre du mod\`ele $\displaystyle R_{ij}-\varepsilon $ une semi-implicitation de termes comme $\displaystyle \phi_{ij,1}$ ou $\displaystyle -\rho\,C_{\varepsilon_2}\,\frac{\varepsilon}{k}\,\varepsilon$. On peut envisager le m\^eme type d'implicitation dans \fort{turbke} m\^eme en pr\'esence du couplage $\displaystyle k-\varepsilon$.}
\item L'adoption d'une r\'esolution d\'ecoupl\'ee fait perdre l'invariance par rotation.
\item La formulation et l'implantation des conditions aux limites de paroi
devront \^etre v\'erifi\'ees. En effet, il semblerait que, dans certains cas, des ph\'enom\`enes
``oscillatoires'' apparaissent, sans qu'il soit ais\'e d'en d\'eterminer la cause.
\item L'implicitation partielle (du fait de la r\'esolution d\'ecoupl\'ee) des
conditions aux limites conduit souvent \`a des calculs instables. Il
conviendrait d'en conna\^\i tre la raison. L'implicitation partielle avait
\'et\'e mise en \oe uvre afin de tenter d'utiliser un pas de temps plus grand
dans le cas de jets axisym\'etriques en particulier.

\end{itemize}

%                      Code_Saturne version 1.3
%                      ------------------------
%
%     This file is part of the Code_Saturne Kernel, element of the
%     Code_Saturne CFD tool.
%
%     Copyright (C) 1998-2007 EDF S.A., France
%
%     contact: saturne-support@edf.fr
%
%     The Code_Saturne Kernel is free software; you can redistribute it
%     and/or modify it under the terms of the GNU General Public License
%     as published by the Free Software Foundation; either version 2 of
%     the License, or (at your option) any later version.
%
%     The Code_Saturne Kernel is distributed in the hope that it will be
%     useful, but WITHOUT ANY WARRANTY; without even the implied warranty
%     of MERCHANTABILITY or FITNESS FOR A PARTICULAR PURPOSE.  See the
%     GNU General Public License for more details.
%
%     You should have received a copy of the GNU General Public License
%     along with the Code_Saturne Kernel; if not, write to the
%     Free Software Foundation, Inc.,
%     51 Franklin St, Fifth Floor,
%     Boston, MA  02110-1301  USA
%
%-----------------------------------------------------------------------
%
\programme{vortex}
%
\vspace{1cm}
%%%%%%%%%%%%%%%%%%%%%%%%%%%%%%%%%%
%%%%%%%%%%%%%%%%%%%%%%%%%%%%%%%%%%
\section{Fonction}
%%%%%%%%%%%%%%%%%%%%%%%%%%%%%%%%%%
%%%%%%%%%%%%%%%%%%%%%%%%%%%%%%%%%%
Ce sous-programme est d�di� � la g�n�ration des conditions d'entr�e
turbulente utilis�es en LES.


La m�thode des vortex est bas�e sur une approche de tourbillons
ponctuels. L'id�e de la m�thode consiste � injecter des tourbillons 2D dans le
plan d'entr�e du calcul, puis � calculer le champ de vitesse induit par ces
tourbillons au centre des faces d'entr�e.

%                      Code_Saturne version 1.3
%                      ------------------------
%
%     This file is part of the Code_Saturne Kernel, element of the
%     Code_Saturne CFD tool.
% 
%     Copyright (C) 1998-2007 EDF S.A., France
%
%     contact: saturne-support@edf.fr
% 
%     The Code_Saturne Kernel is free software; you can redistribute it
%     and/or modify it under the terms of the GNU General Public License
%     as published by the Free Software Foundation; either version 2 of
%     the License, or (at your option) any later version.
% 
%     The Code_Saturne Kernel is distributed in the hope that it will be
%     useful, but WITHOUT ANY WARRANTY; without even the implied warranty
%     of MERCHANTABILITY or FITNESS FOR A PARTICULAR PURPOSE.  See the
%     GNU General Public License for more details.
% 
%     You should have received a copy of the GNU General Public License
%     along with the Code_Saturne Kernel; if not, write to the
%     Free Software Foundation, Inc.,
%     51 Franklin St, Fifth Floor,
%     Boston, MA  02110-1301  USA
%
%-----------------------------------------------------------------------
%
%%%%%%%%%%%%%%%%%%%%%%%%%%%%%%%%%%
%%%%%%%%%%%%%%%%%%%%%%%%%%%%%%%%%%
\section{Discr\'etisation}
%%%%%%%%%%%%%%%%%%%%%%%%%%%%%%%%%%
%%%%%%%%%%%%%%%%%%%%%%%%%%%%%%%%%%

Le terme convectif en $\dive(\underline{u} \otimes \rho\,\underline{u})$
introduit une non lin\'earit\'e et un couplage des composantes de la vitesse
$\vect{u}$ dans l'�quation (\ref{Base_Preduv_eqqdm}). Une lin\'earisation et un d\'ecouplage
des trois composantes de la 
vitesse sont r\'ealis\'es lors de la discr\'etisation de cette \'etape de
pr\'ediction.\\
En effet, soit :
\begin{equation}
\vect{\widetilde{u}}= \vect{u}^n + \delta \vect{u} 
\end{equation}
La contribution exacte du terme convectif \`a prendre en compte dans cette
\'etape de pr\'ediction serait :\\
\begin{equation}\label{Base_Preduv_Conv_exact}
\begin{array}{ll}
\dive(\vect{\widetilde{u}} \otimes \rho\,\vect{\widetilde{u}}) =
\dive(\vect{u}^{n} \otimes \rho\,\vect{u}^{n}) + \dive(\delta \vect{u} \otimes
\rho\,\vect{u}^{n}) +  \underbrace { \dive(\vect{u}^{n} \otimes
\rho\,\delta \vect{u})}_{\text {terme couplant lin\'eaire}} +  \underbrace { \dive(\delta \vect{u} \otimes
\rho\,\delta \vect{u})}_{\text {terme couplant et non lin\'eaire}}\\
\end{array} 
\end{equation}
Les deux derniers termes de l'expression (\ref{Base_Preduv_Conv_exact}) sont {\em a priori} n�glig�s
de mani�re � obtenir un syst\`eme en vitesse qui soit d\'ecoupl\'e et donc,
�viter l'inversion d'une matrice pouvant \^etre de tr\`es grande taille. Ces
deux termes peuvent n�anmoins �tre pris en compte de mani�re plus ou moins
approch�e par extrapolation explicite du flux de masse en $n+\theta_F$ (pour le
terme couplant lin�aire provenant de la convection de $\vect{u}^{n}$ par $\delta
\vect{u}$) et utilisation d'un point-fixe par sous it�ration sur le sous
programme \fort{navsto} (pour le terme non-lin�aire, en sp�cifiant $\var{NTERUP}>1$).

L'�quation (\ref{Base_Preduv_eqqdm}) est discr�tis�e au temps $n+\theta$ � l'aide d'un
$\theta$-sch�ma, et le tenseur des pertes de charges d�compos� en une partie
diagonale $\tens{K}_{d}$ et une extradiagonale $\tens{K}_{e}$ (soit
 $\tens{K}_{pdc}=\tens{K}_{d}+\tens{K}_{e}$).\\
$\bullet$ La pression est suppos�e connue en $n-1+\theta$ (d�calage temporel
pression-vitesse) et le gradient naturellement calcul� � cet instant.\\ 
$\bullet$ Les termes sources de viscosit� secondaire, de gradient transpos\'e,
ceux provenant du mod�le de turbulence\footnote{except� $\dive (\mu_t\ (\ggrad
\underline {u}))$}, $\rho\,\tens{K}_{\,e}\ \underline{u}$, $(\rho -\rho_0)
\underline {g}$ ainsi que $\underline{T}_{s}^{\,exp}$ et
$\Gamma\,\underline{u}_{\,i}$ sont pris de mani�re explicite au temps $n$, ou
extrapol�s suivant le sch�ma en temps choisi pour les propri�t�s physique et les
termes sources.\\ 
$\bullet$ Les termes sources $\underline{u}\,\,\dive (\rho\,\underline {u})$,
$\Gamma\,\,\underline{u}$, $T_{s}^{\,imp}\,\,\underline{u}$ et
$-\rho\,\tens{K}_{\,d}\,\,\underline{u}$ sont implicit�s est calcul�s �
l'instant $n+\theta$.\\ 
$\bullet$ Le terme de diffusion $\dive (\mu_{\,tot}\,\ggrad \underline{u})$ est
implicit� : la vitesse est prise � l'instant $n+\theta$ et la viscosit�
explicit�e ou extrapol�e.\\ 
$\bullet$ Enfin, le terme de convection en $\dive(\,\underline{u} \otimes
(\rho\underline{u})\,)$ est implicit� : la composante r�solue de la vitesse est
prise en $n+\theta$, et le flux de masse, explicit�, ou extrapol� en
$n+\theta_F$. 

Par souci de clart�, on suppose, en l'absence d'indication, que les propri�tes
physiques $\Phi$ ($\rho,\,\mu_{tot},\,...$) et le flux de masse
$(\rho\underline{u})$ sont pris respectivement aux instants $n+\theta_\Phi$ et
$n+\theta_F$, o� $\theta_\Phi$ et $\theta_F$ d�pendent des sch�mas en temps
sp�cifiquement utilis�s pour ces grandeurs\footnote{cf. \fort{introd}}. 

La discr�tisation temporelle de l'�quation (\ref{Base_Preduv_eqqdm}) s'�crit alors comme suit : 

\begin{equation}\label{Base_Preduv_eq_di1}
 \begin{array}{c}
\displaystyle \rho\,\ \frac{ \underline {\widetilde{u}}^{n+1} -\underline {u}^{n} }
{\Delta t} + \dive(\,\underline{\widetilde{u}}^{n+\theta} \otimes (\rho\underline{u})\,) -\dive
(\mu_{\,tot}\,\ggrad \underline{\widetilde{u}}^{n+\theta}) =
\\
\displaystyle
 - \grad p^{n-1+\theta} + \dive (\rho\,\underline {u})\,\underline{\widetilde{u}}^{n+\theta} +(\Gamma\,\underline{u}_{\,i})^{n+\theta_S}-\Gamma^n\,\,\underline{\widetilde{u}}^{n+\theta}
\\
\begin{array}{c}
\displaystyle
- \rho\,\tens{K}_{\,d}^{n}\,\,\underline{\widetilde{u}}^{n+\theta} - (\rho\,\tens{K}_{\,e}\ \underline{u})^{n+\theta_S} + (\underline{T}_{s}^{\,exp})^{\,n+\theta_S} + T_{s}^{\,imp}\,\,\underline{\widetilde{u}}^{n+\theta}
\\
\displaystyle
+[\dive (\mu_{\,tot}\,^t\ggrad \underline {u})]^{n+\theta_S}-\frac {2} {3}[\,\grad (\mu_{\,tot}\,\dive \underline {u})]^{n+\theta_S} + (\rho -\rho_0) \underline {g}
 - (\underline{turb})^{n+\theta_S}
\end{array}
\end{array}
\end{equation}
o\`u, par souci de simplification, on a pos\'e :
\begin{equation}
\mu_{\,tot}=
\begin{cases}
\mu+\mu_t & \text{pour les mod�les � viscosit� turbulente ou en LES}, \\
\mu & \text{pour les mod�les au second ordre ou en laminaire}
\end{cases} \ 
\end{equation}
\\
et :
\begin{equation}
\underline{turb}^{n}=
\begin{cases}
\displaystyle\frac {2}{3}\grad (\rho^{n}\,k^{n}) & \text{pour les mod�les � viscosit� turbulente}, \\
\dive(\rho^{n}\,\tens{R}^n) & \text{pour les mod�les au second ordre},\\
0 & \text{en laminaire ou en LES}\\
\end{cases}
\end{equation}
Par analogie avec l'�criture du $\theta$-sch�ma pour une variable scalaire, $\,
\underline {\widetilde{u}}^{n+\theta}$ est interpol�e � partir de la vitesse
pr�dite $\underline {\widetilde{u}}^{n+1}$ de la mani\`ere suivante\footnote{si
$\theta=1/2$, ou qu'une extrapolation est utilis�e, l'ordre 2 n'est obtenu que si
le pas de temps $\Delta t$ est uniforme en temps et en espace.}~: 
\begin{equation}
\underline {\widetilde{u}}^{n+\theta}=\theta\, \underline
{\widetilde{u}}^{n+1}+(1-\theta)\, \underline {u}^{n}\\ 
\end{equation}
Avec :
\begin{equation}
\left\{
\begin{array}{ll}
\theta = 1   & \text{Pour un sch\'ema de type Euler implicite d'ordre 1.}\\
\theta = 1/2 & \text{Pour un sch\'ema de type Cranck-Nicolson d'ordre 2.}\\
\end{array}
\right.
\end{equation}

L'�quation (\ref{Base_Preduv_eq_di1}) est alors r��crite sous la forme :

\begin{equation}\label{Base_Preduv_eq_di2}
\begin{array}{c}
\displaystyle \underbrace{\left(\frac{\rho}{\Delta t} -\theta \,\dive (\rho\,\underline {u}) +\theta \,\, \Gamma^n +
\theta \,\, \rho\,\tens{K}_{\,d}^n-\theta \,T_s^{\,imp} \right)}_{\displaystyle f_s^{imp}}\, (\underline {\,\widetilde{u}}^{n+1} -\underline {u}^{n})
\\
 +\, \theta\, \dive(\underline {\widetilde{u}}^{n+1} \otimes (\rho\underline{u}))-\, \theta\,\dive (\mu_{\,tot}\,\ggrad \underline {\widetilde{u}}^{n+1}) =
\\
-\,(1-\theta)\, \dive(\underline {u}^{n} \otimes (\rho\underline{u})) +\,(1-\theta)\,\dive (\mu_{\,tot}\,\ggrad \underline {u}^{n})
\\
f_s^{exp}\left\{
\begin{array}{c}
\displaystyle 
- \grad p^{n-1+\theta} + \dive (\rho\,\underline {u})\,\underline{u}^{n} +\,(\,\Gamma^{n}\,\underline{u}_{\,i}\,)^{n+\theta_S}- \Gamma^n\,\,\underline{u}^{n}
\\
\displaystyle
-(\,\rho\,\tens{K}_{\,e}\ \underline{u}\,)^{n+\theta_S} -\rho\,\tens{K}_{\,d}^n\ \underline{u}^{n}+ (\underline{T}_{s}^{\,exp})^{\,n+\theta_S} + T_s^{\,imp}\,\,\underline {u}^{n} 
\\
\displaystyle
+[\dive (\mu_{\,tot}\,^t\ggrad \underline {u}\,)]^{n+\theta_S}-\frac {2} {3}[\,\grad (\mu_{\,tot}\,\dive \underline {u}\,)]^{n+\theta_S} + (\rho -\rho_0) \underline {g}-(\underline{turb})^{n+\theta_S}
\end{array}
\right.
\end{array}
\end{equation}

d'o� l'�quation r�solue par le sous-programme \fort{codits} :
\begin{equation}\begin{array}{c}
\displaystyle
f_s^{\,imp}(\underline {\widetilde{u}}^{n+1}-\underline {u}^{n}) + \theta\, \dive(\underline{\widetilde{u}}^{n+1} \otimes (\rho
\underline{u})) - \theta\,\dive (\,\mu_{\,tot}\,\ggrad \underline{\widetilde{u}}^{n+1}) = 
\\\\
\displaystyle
-(1-\theta)\,\dive(\underline{u}^{n} \otimes (\rho \underline{u}))+(1-\theta)\,\dive (\,\mu_{\,tot}\,\ggrad \underline{u}^{n})
+ \underline{f}_{\,s}^{\,exp}
\end{array}
\end{equation}
La m\'ethode de discr\'etisation spatiale est d\'evelopp\'ee dans le sous-programme \fort{codits}.\\



\minititre{Remarques :}
{\tiny$\blacksquare$} Dans le cas standard sans extrapolation, le terme
$-\,T_s^{\,imp}$ n'est ajout� � $f_s^{\,imp}$ que s'il est positif afin de ne
pas affaiblir la dominance de la diagonale de la matrice � inverser.\\ 
{\tiny$\blacksquare$} Si une extrapolation est utilis�e, par contre,
$\,T_s^{\,imp}$ est ajout� � $f_s^{\,imp}$ quel que soit son signe. En effet, l'id�e intuitive qui
consiste � prendre~: 
\begin{equation}
\begin{cases}
\displaystyle
(\underline{T}_{s}^{\,exp} + T_{s}^{\,imp}\,\underline {u})^{\,n+\theta_S} &
\text{si } T_{s}^{\,imp} > 0\\ 
\displaystyle
(\underline{T}_{s}^{\,exp})^{\,n+\theta_S} + T_{s}^{\,imp}\,\underline{u}^{n+\theta} &\text{sinon}\\
\end{cases}
\end{equation} 
aboutit � une incoh�rence dans le traitement si $T_s^{imp}$ change de signe
entre deux pas de temps.\\ 
{\tiny$\blacksquare$} la partie diagonale $\tens{K}_{\,d}$ du terme
de perte de charge est utilis�e dans $f_s^{\,imp}$. En fait, pour \^etre rigoureux,
il faudrait ne retenir que les contributions positives (point signal\'e dans le
sous-programme utilisateur associ\'e \fort{uskpdc}). Cette prise en compte sera \`a am\'eliorer.\\
{\tiny$\blacksquare$} Le terme $\theta\,\Gamma^{n}-\theta\,\dive
(\rho\,\underline {u})$ ne pose pas de probl�me pour la 
dominance de la diagonale de la matrice car il est exactement compens� par le
terme de convection (cf. \fort{covofi}). 


%                      Code_Saturne version 1.3
%                      ------------------------
%
%     This file is part of the Code_Saturne Kernel, element of the
%     Code_Saturne CFD tool.
%
%     Copyright (C) 1998-2007 EDF S.A., France
%
%     contact: saturne-support@edf.fr
%
%     The Code_Saturne Kernel is free software; you can redistribute it
%     and/or modify it under the terms of the GNU General Public License
%     as published by the Free Software Foundation; either version 2 of
%     the License, or (at your option) any later version.
%
%     The Code_Saturne Kernel is distributed in the hope that it will be
%     useful, but WITHOUT ANY WARRANTY; without even the implied warranty
%     of MERCHANTABILITY or FITNESS FOR A PARTICULAR PURPOSE.  See the
%     GNU General Public License for more details.
%
%     You should have received a copy of the GNU General Public License
%     along with the Code_Saturne Kernel; if not, write to the
%     Free Software Foundation, Inc.,
%     51 Franklin St, Fifth Floor,
%     Boston, MA  02110-1301  USA
%
%-----------------------------------------------------------------------
%

%%%%%%%%%%%%%%%%%%%%%%%%%%%%%%%%%%
%%%%%%%%%%%%%%%%%%%%%%%%%%%%%%%%%%
\section{Mise en \oe uvre}
%%%%%%%%%%%%%%%%%%%%%%%%%%%%%%%%%%
%%%%%%%%%%%%%%%%%%%%%%%%%%%%%%%%%%
La num\'ero de la phase trait\'ee fait partie des arguments de \fort{turrij}. On
omettra volontairement de le pr\'eciser dans ce qui suit, on indiquera par $(\ )$ la
notion de tableau s'y rattachant.

\etape{Calcul des termes de production $\tens{\mathcal{P}}$}
\begin{itemize}
\item [$\star$] Initialisation \`a z\'ero du tableau \var{PRODUC} dimensionn\'e \`a $\var{NCEL}\times 6$.
\item [$\star$] On appelle trois fois \fort{grdcel} pour calculer les gradients des composantes de la vitesse $u$, $v$ et
$w$ prises au temps $n$.

Au final, on a :\\
$\displaystyle
\begin{array} {ll}
\var{PRODUC(1,IEL)} = & \displaystyle - 2 \left[ R_{11}^{\,n} \frac{\partial u^{\,n}} {\partial x} +R_{12}^{\,n} \frac{\partial u^{\,n}} {\partial y}+R_{13}^{\,n} \frac{\partial u^{\,n}} {\partial z} \right] \text{        (production de $R_{11}^{\,n}$)}\\
\var{PRODUC(2,IEL)} = & \displaystyle - 2 \left[ R_{12}^{\,n} \frac{\partial v^{\,n}} {\partial x} +R_{22}^{\,n} \frac{\partial v^{\,n}} {\partial y}+R_{23}^{\,n} \frac{\partial v^{\,n}} {\partial z} \right] \text{        (production de $R_{22}^{\,n}$)}\\
\var{PRODUC(3,IEL)} = & \displaystyle - 2 \left[ R_{13}^{\,n} \frac{\partial w^{\,n}} {\partial x} +R_{23}^{\,n} \frac{\partial w^{\,n}} {\partial y}+R_{33}^{\,n} \frac{\partial w^{\,n}} {\partial z} \right] \text{        (production de $R_{33}^{\,n}$)}\\
\var{PRODUC(4,IEL)} = & \displaystyle - \left[ R_{12}^{\,n} \frac{\partial u^{\,n}} {\partial x} +R_{22}^{\,n} \frac{\partial u^{\,n}} {\partial y}+R_{23}^{\,n} \frac{\partial u^{\,n}} {\partial z} \right] \\
& \displaystyle - \left[ R_{11}^{\,n} \frac{\partial v^{\,n}} {\partial x} +R_{12}^{\,n} \frac{\partial v^{\,n}} {\partial y}+R_{13}^{\,n} \frac{\partial v^{\,n}} {\partial z} \right] \text{        (production de $R_{12}^{\,n}$)} \\
\var{PRODUC(5,IEL)} = & \displaystyle - \left[ R_{13}^{\,n} \frac{\partial u^{\,n}} {\partial x} +R_{23}^{\,n} \frac{\partial u^{\,n}} {\partial y}+R_{33}^{\,n} \frac{\partial u^{\,n}} {\partial z} \right] \\
& \displaystyle - \left[ R_{11}^{\,n} \frac{\partial w^{\,n}} {\partial x} +R_{12}^{\,n} \frac{\partial w^{\,n}} {\partial y}+R_{13}^{\,n} \frac{\partial w^{\,n}} {\partial z} \right] \text{        (production de $R_{13}^{\,n}$)} \\
\var{PRODUC(6,IEL)} = & \displaystyle - \left[ R_{13}^{\,n} \frac{\partial v^{\,n}} {\partial x} +R_{23}^{\,n} \frac{\partial v^{\,n}} {\partial y}+R_{33}^{\,n} \frac{\partial v^{\,n}} {\partial z} \right] \\
& \displaystyle - \left[ R_{12}^{\,n} \frac{\partial w^{\,n}} {\partial x} +R_{22}^{\,n} \frac{\partial w^{\,n}} {\partial y}+R_{23}^{\,n} \frac{\partial w^{\,n}} {\partial z} \right]  \text{        (production de $R_{23}^{\,n}$)}
\end{array}
$
\end{itemize}

\etape{Calcul du gradient de la masse volumique $\rho^n$ prise au d\'ebut du pas
de temps courant\footnote{{\it i.e.} calcul\'ee \`a partir des
variables du pas de temps pr\'ec\'edent $n$ si n\'ecessaire.} $(n+1)$}
Ce calcul n'a lieu que si les termes de gravit\'e doivent \^etre pris en compte
($\var{IGRARI()} =1$).
\begin{itemize}
\item [$\star$] Appel de \fort{grdcel}  pour calculer le gradient de $\rho^n$
dans les trois directions de l'espace. Les conditions aux limites sur $\rho^n$
sont des conditions de Dirichlet puisque la valeur de $\rho^n$ aux faces de bord
$ik$ (variable \var{IFAC}) est connue et vaut $\rho_{\,b_{\,ik}}$. Pour \'ecrire les conditions aux limites
sous la forme habituelle, $$\rho_{\,b_{\,ik}} = \var{COEFA} + \var{COEFB}
\,\rho^n_{\,I'}$$ on pose alors $\var{COEFA}=
\var{PROPCE(IFAC,IPPROB(IROM(IPHAS)))}$ et $\var{COEFB} = \var{VISCB} = 0$.\\
$\var{PROPCE(1,IPPROB(IROM(IPHAS)))}$ (resp.$\var{VISCB}$) est utilis\'e en lieu
et place de l'habituel \var{COEFA} ($\var{COEFB}$), lors de l'appel \`a \fort{grdcel}.\\
On a donc :\\
$\displaystyle \var{GRAROX}= \frac{\partial \rho^n}{\partial x}\ $,$\displaystyle \ \var{GRAROY}= \frac{\partial
\rho^n}{\partial y}$ et $
\displaystyle \ \var{GRAROZ}= \frac{\partial \rho^n}{\partial z}\ $.

\end{itemize}

Le gradient de $\rho^n$ servira \`a calculer les termes de production par effets de gravit\'e si ces derniers sont pris en compte.

\etape{Boucle \var{ISOU} de $1$ \`a $6$ sur les tensions de Reynolds}
Pour $\var{ISOU} = 1,2,3,4,5,6$, on r\'esout respectivement et dans
l'ordre  les
\'equations de $R_{11}$, $R_{22}$, $R_{33}$, $R_{12}$, $R_{13}$ et $R_{23}$ par
l'appel au sous-programme \fort{resrij}.\\
La r\'esolution se fait par incr\'ement $\delta {R}_{ij}^{\,n+1,k+1}$ , en utilisant la m\^eme m\'ethode que
celle d\'ecrite dans le sous-programme \fort{codits}. On adopte ici les m\^emes notations.
\var{SMBR} est le second membre du syst\`eme \`a inverser, syst\`eme portant sur
les incr\'ements de la variable. \var{ROVSDT} repr\'esente la diagonale de la
matrice, hors convection/diffusion.\\
On va r\'esoudre l'\'equation (\ref{Base_Turrij_Eq_Temp_Rij}) sous forme incr\'ementale en
utilisant \fort{codits}, soit :
\begin{equation}\label{Base_Turrij_Eq_Temp_deltaRij}
\begin{array}{ll}
&\displaystyle \underbrace{\left(\frac {\rho^n_L}{\Delta t^n}
+ \rho^n_L \,C_1\,\frac{\varepsilon^n_L}{k^n_L}(1-\frac{\delta_{ij}}{3})
 - m^n_{\,lm} + \Gamma_L\,+ max(-\alpha^n_{R_{ij}},0)\right)\,|\Omega_l|}
_{\text {$\var{ROVSDT}$ contribuant
\`a la diagonale de la matrice simplifi\'ee de \fort{matrix}}}\,(\delta{R}_{ij}^{\,n+1,p+1})_{\,L}\\\\
&  \underbrace{+\sum\limits_{m\in Vois(l)}\displaystyle \left[
 m^n_{\,lm} \delta R_{ij,\,f_{\,lm}}^{\,n+1,p+1}
- (\mu^n_{\,lm} + \gamma^n_{\,lm})\
\frac{({\delta R}_{ij}^{\,n+1,p+1})_{M}-({\delta R}_{ij}^{\,n+1,p+1})_{L})}{\overline{L'M'}}\,
S_{\,lm} \right]}_{\text { convection upwind pur et diffusion non reconstruite
relatives \`a la matrice simplifi\'ee de \fort{matrix}\footnotemark}} \\
% voir le texte de la footmark plus bas
&= - \displaystyle\frac {\rho^n_L}{\Delta t^n}\,\left(\,(R^{\,n+1,p}_{ij})_L - (R^{\,n}_{ij})_L\,\right)\\
&-\,\underbrace{\displaystyle\int_{\Omega_l} \left(
\dive\,[\,(\rho\,\vect{u})^n\,R^{\,n+1,p}_{ij} - (\mu^n\,+ \gamma^n\,)\,
\grad{R^{\,n+1,p}_{ij}}\,]\right)\,d\Omega}_{\text {convection et diffusion
trait\'ees par \fort{bilsc2}}}\\
&+\displaystyle \int_{\Omega_l} \left[\,\mathcal{P}^{\,n+1,p}_{ij} + \mathcal{G}^{\,n+1,p}_{ij}
- \displaystyle\rho^n \,C_1\,\frac{\varepsilon^n}{k^n}\left[R^{\,n+1,p}_{ij}-
\frac{2}{3}\,k^n\,\delta_{ij}\right] + \phi^{\,n+1,p}_{ij,2} +
\phi^{\,n+1,p}_{ij,w}\,\right]\, d\Omega \\
& + \displaystyle\int_{\Omega_l} \left[- \frac{2}{3} \rho^n \varepsilon^n \delta_{ij}
 + \Gamma\,(\,R^{\,in}_{ij} - R^{\,n+1,p}_{ij}\,) +
\alpha^n_{R_{ij}}\,R^{\,n+1,p}_{ij}+ \beta^n_{R_{ij}}\right]\, d\Omega\\
&+ \sum\limits_{m\in
Vois(l)}\displaystyle \left[\ \tens{E}^n\,\grad{R}^{\,n+1,p}_{ij} \right]_{\,lm}\,.\,\vect{n}_{\,lm}S_{\,lm}\\
&+ \sum\limits_{m\in Vois(l)}\displaystyle \left[\
\tens{D}^n\,\grad{R}^{\,n+1,p}_{ij} \right]_{\,lm}\,.\,\vect{n}_{\,lm}S_{\,lm}\\
&- \sum\limits_{m\in Vois(l)} \gamma^n_{\,lm} \left( \grad{R}^{\,n+1,p}_{ij}\,.\,\vect{n}_{\,lm} \right)  S_{\,lm}\\
&+ \sum\limits_{m\in Vois(l)}  m^n_{\,lm}\,(R^{\,n+1,p}_{ij})_L\\
\end{array}
\end{equation}
% si on ne fait pas comme ca, il n'apparait pas
\footnotetext[\thefootnote]{Si $\var{IRIJNU} = 1$, on remplace  $\mu^n_{\,lm}$ par $(\mu +
\mu_t)^n_{\,lm}$ dans l'expression de la diffusion non reconstruite
associ\'ee \`a la matrice simplifi\'ee de \fort{matrix} ($\mu_t$ d\'esigne la
viscosit\'e turbulente calcul\'ee comme en $k-\varepsilon$).}

o\`u on rappelle :\\
pour $n$ donn\'e entier positif, on d\'efinit la suite
 $({R}_{ij}^{\,n+1,p})_{p \in \grandN}$
 par :
\begin{equation}\notag
\left\{\begin{array}{l}
{R}_{ij}^{\,n+1,0} = {R}_{ij}^{\,n}\\
{R}_{ij}^{\,n+1,p+1} = {R}_{ij}^{\,n+1,p} + \delta{R}_{ij}^{\,n+1,p+1} \\
\end{array}\right.
\end{equation}
$(\delta{R}_{ij}^{\,n+1,p+1})_{\,L}$ d\'esigne la valeur sur l'\'el\'ement
$\Omega_l$ du $\text{$(\,p+1\,)$-i\`eme}$ incr\'ement de ${R}_{ij}^{\,n+1}$,
$ m^n_{\,lm}$ le flux de masse \`a l'instant $n$ \`a travers la face $lm$,
$\delta R_{ij,\,f_{\,lm}}^{\,n+1,p+1}$ vaut $({\delta
R}_{ij}^{\,n+1,p+1})_{L}$  si $ m^n_{\,lm} \geqslant 0$, $({\delta
R}_{ij}^{\,n+1,p+1})_{M}$ sinon,
$\mathcal{P}^{\,n+1,p}_{ij}$, $\phi^{\,n+1,p}_{ij,2}$, $\phi^{\,n+1,p}_{ij,w}$ les valeurs
des quantit\'es associ\'ees correspondant \`a l'incr\'ement
$(\delta{R}_{ij}^{\,n+1,p})$.\\



Tous ces termes sont calcul\'es comme suit :
\begin{itemize}
\item Terme de gauche de l'\'equation (\ref{Base_Turrij_Eq_Temp_deltaRij})\\
Dans \fort{resrij} est calcul\'ee la variable \var{ROVSDT}. Les autres
termes sont compl\'et\'es par \fort{codits}, lors de la construction de la matrice simplifi\'ee , {\it via} un
appel au sous-programme \fort{matrix}. La quantit\'e
 $(\mu^n_{\,lm} + \gamma^n_{\,lm})$ \`a la face $lm$ est calcul\'ee lors de l'appel \`a
\fort{visort}.\\
\item Second membre de l'\'equation (\ref{Base_Turrij_Eq_Temp_deltaRij})\\
Le premier terme non d\'etaill\'e est calcul\'e par le sous-programme
\fort{bilsc2}, qui applique le sch\'ema convectif choisi par l'utilisateur, qui
reconstruit ou non selon le souhait de l'utilisateur les gradients intervenants
dans la convection-diffusion.\\
Les termes sans accolade sont, eux, compl\`etement explicites et ajout\'es au fur et
\`a mesure dans \var{SMBR} pour former
l'expression $f^{\,exp}_s$ de \fort{codits}.
\end{itemize}
On d\'ecrit ci-dessous les \'etapes de \fort{resrij} :
\begin{itemize}

\item DELTIJ = 1, pour $\var{ISOU} \leqslant 3$ et DELTIJ = 0  Si $\var{ISOU} >
3$. Cette valeur repr\'esente le symbole de Kroeneker $\delta_{ij}$.

\item Initialisation \`a z\'ero de \var{SMBR} (tableau contenant le second
membre) et \var{ROVSDT} (tableau contenant la diagonale de la matrice sauf celle
relative \`a la contribution de la
diagonale des op\'erateurs de convection et de diffusion lin\'earis\'es
\footnote{qui correspondent aux sch\'emas convectif upwind pur et diffusif sans
reconstruction.}), tous deux de dimension $\var{NCEL}$.

\item Lecture et prise en compte des termes sources utilisateur pour la variable $R_{ij}$

Appel \`a \fort{ustsri} pour charger les termes sources utilisateurs. Ils sont
stock\'es comme suit. Pour la cellule $\Omega_l$ de centre $L$, repr\'esent\'ee par $\var{IEL}$, on a :\\
\begin{equation}\notag
\left\{\begin{array}{lll}
&\var{ROVSDT(IEL)}&= |\Omega_l| \ \alpha_{R_{ij}}\\
&\var{SMBR(IEL)}&=|\Omega_l| \ \beta_{R_{ij}}\\
\end{array}\right.
\end{equation}
On affecte alors les valeurs ad\'equates au second membre \var{SMBR} et \`a la
diagonale \var{ROVSDT} comme suit :
\begin{equation}\notag
\left\{\begin{array}{lll}
&\var{SMBR(IEL)} &= \var{SMBR(IEL)} +\ |\Omega_l| \ \alpha_{R_{ij}} \ (R^n_{ij})_L \\
&\var{ROVSDT(IEL)}&= \text{max }(-\ |\Omega_l| \ \alpha_{R_{ij}},0)\\
\end{array}\right.
\end{equation}
La valeur de $ \var{ROVSDT}$ est ainsi calcul\'ee pour des raisons de stabilit\'e
num\'erique. En effet, on ne rajoute sur la diagonale que les valeurs positives,
ce qui correspond physiquement \`a impliciter les termes de rappel uniquement.
\item{Calcul du terme source de masse  si $\Gamma_L > 0$}

Appel de \fort{catsma} et incr\'ementation si n\'ecessaire de \var{SMBR} et
\var{ROVSDT} {\it via} :\\
\begin{equation}\notag
\left\{\begin{array}{lll}
\displaystyle \var{SMBR(IEL)} = \var{SMBR(IEL)} + |\Omega_l| \ \Gamma_L \
\left[(R^{\,in}_{ij})_L - (R^{\,n}_{ij})_L \right] \\
\displaystyle \var{ROVSDT(IEL)}=\var{ROVSDT(IEL)} + |\Omega_l| \ \Gamma_L
\end{array}\right.
\end{equation}
\item Calcul du terme d'accumulation de masse et du terme instationnaire

On stocke $\displaystyle \var{W1}= \int_{\Omega_l}\dive\,(\rho\,\vect{u})\,d\Omega$
calcul\'e par \fort{divmas} \`a l'aide des flux de masse aux faces internes
$ m^n_{\,lm}=\sum\limits_{m\in Vois(l)}{(\rho \vect{u})_{\,lm}^n} \text{.}\,
\vect{S}_{\,lm} $ (tableau \var{FLUMAS}) et des flux de masse aux bords  $ m^n_{\,b_{lk}} = \sum\limits_{k\in{\gamma_b(l)}}{(\rho \vect{u})_{\,{b}_{lk}}^n} \text{.}\,
\vect{S}_{\,{b}_{lk}} $ (tableau \var{FLUMAB}).
On incr\'emente ensuite \var{SMBR} et \var{ROVSDT}.
\begin{equation}\notag
\left\{\begin{array}{lll}
&\var{SMBR(IEL)} &= \var{SMBR(IEL)} + \var{ICONV}\  (R^n_{ij})_L\,(\displaystyle
\int_{\Omega_l}\dive\,(\rho\,\vect{u})\ d\Omega) \\
&\var{ROVSDT(IEL)}& = \var{ROVSDT(IEL)} +  \var{ISTAT}\,\displaystyle
\frac{\rho^n_L \ |\Omega_l|}{\Delta t^n} -  \var{ICONV}\ (\displaystyle
\int_{\Omega_l}\dive\,(\rho\,\vect{u})\ d\Omega) \\
\end{array}\right.
\end{equation}
\item Calcul des termes sources de production, des termes $\displaystyle
\phi_{\,ij,1}+\phi_{\,ij,2}$ et de la dissipation~$\displaystyle-\frac{2}{3} \varepsilon\,\delta_{\,ij}$ :

On effectue une boucle d'indice \var{IEL} sur les cellules $\Omega_l$ de centre $L$ :
\begin{itemize}
\item [$\Rightarrow$] $\displaystyle \var{TRPROD}= \frac{1}{2} (\mathcal{P}^n_{ii})_L = \frac{1}{2} \left[ \var{PRODUC(1,IEL)} +  \var{PRODUC(2,IEL)} +  \var{PRODUC(3,IEL)} \right] $
\item [$\Rightarrow$] $\displaystyle \var{TRRIJ }= \frac{1}{2} (R^n_{ii})_L $
\item [$\Rightarrow$] $\displaystyle \var{SMBR(IEL)} =\ \var{SMBR(IEL)}\ +$\\
$\ \displaystyle\rho^n_L |\Omega_l| \left[ \displaystyle
\frac{2}{3}\,\delta_{\,ij} \left( \ \displaystyle \frac{ C_2}{2}\,(\mathcal{P}^n_{ii})_L\ +
(C_1-1)\ \varepsilon^n_L\, \right)\right.$\\
$ + \left.\ (1-C_2) \ \var{PRODUC(ISOU,IEL)} -
\displaystyle C_1\ \frac{2\,\varepsilon^n_L}{(R^n_{ii})_L}\ (R^n_{ij})_L \right]$
\item [$\Rightarrow$] $\displaystyle \var{ROVSDT(IEL)} = \var{ROVSDT(IEL)} +
\rho^n_L \ |\Omega_l| \ (- \displaystyle \frac{1}{3} \ \,\delta_{\,ij} + 1) \ C_1
\ \frac{2\ \varepsilon^n_L}{(R^n_{ii})_L}$
\end{itemize}
\item Appel de \fort{rijech} pour le calcul des termes d'\'echo de paroi
 $\phi^n_{ij,w}$ si $\var{IRIJEC()}=1$ et ajout dans \var{SMBR}.\\
$\var{SMBR} = \var{SMBR} + \phi^n_{ij,w}$\\
Suivant son mode de calcul (\var{ICDPAR}), la distance � la paroi est directement accessible
par \var{RA(IDIPAR+IEL-1)} (\var{|ICDPAR|} = 1) ou bien
est calcul\'ee \`a partir de $\var{IA(IIFAPA(IPHAS)+IEL - 1)}$,
qui donne pour l'\'el\'ement $\var{IEL}$ le num\'ero de la face de bord
paroi la plus  proche (\var{|ICDPAR|} = 2). Ces tableaux ont \'et\'e renseign\'e une fois pour toutes au
d\'ebut de calcul.

\item  Appel de \fort{rijthe} pour le calcul des termes de gravit\'e $\mathcal{G}^n_{ij}$ et ajout dans \var{SMBR}.

Ce calcul n'a lieu que si $\var{IGRARI()} = 1$.
$ \var{SMBR} = \var{SMBR} + \mathcal{G}^n_{ij}$
\item Calcul de la partie explicite du terme de diffusion
 $\dive{\,\left[\tens{A}\,\grad{R}^{\,n}_{ij}\right]}$, plus pr\'ecis\'ement
des contributions du terme extradiagonal pris aux faces purement internes
(remplissage du tableau \var{VISCF}), puis aux faces de bord (remplissage du
tableau \var{VISCB}).
\begin{itemize}
\item [$\star$] Appel de \fort{grdcel} pour le calcul du gradient de
$R^{\,n}_{ij}$ dans chaque direction. Ces gradients sont respectivement
stock\'es dans les tableaux de travail \var{W1}, \var{W2} et \var{W3}.

\item [$\star$] boucle d'indice \var{IEL} sur les cellules $\Omega_l$ de centre
$L$ pour le
calcul de $\tens{E}^n\,\grad{R}^{\,n}_{ij}$ aux cellules dans un premier temps :\\
\begin{itemize}
\item [$\Rightarrow$] $\displaystyle \var{TRRIJ}= \frac{1}{2} (R^{\,n}_{ii})_L $
\item [$\Rightarrow$] $\displaystyle \var{CSTRIJ} = \rho^n_L\ C_S \ \displaystyle\frac{(R^n_{ii})_L}{2\,\varepsilon^n_L}$
\item [$\Rightarrow$] $\displaystyle \var{W4(IEL)} = \rho^n_L\ C_S\
\displaystyle\frac{(R^n_{ii})_L}{2\,\varepsilon^n_L} \left[\,(R^{\,n}_{12})_L \ \var{W2(IEL)} +
(R^{\,n}_{13})_L \ \var{W3(IEL)}\,\right]$
\item [$\Rightarrow$] $\displaystyle \var{W5(IEL)} = \rho^n_L\ C_S\
\displaystyle\frac{(R^n_{ii})_L}{2\,\varepsilon^n_L} \left[\,(R^{\,n}_{12})_L \ \var{W1(IEL)} +
(R^{\,n}_{23})_L \ \var{W3(IEL)}\,\right]$
\item [$\Rightarrow$] $\displaystyle \var{W6(IEL)} = \rho^n_L\ C_S\
\displaystyle\frac{(R^n_{ii})_L}{2\,\varepsilon^n_L} \left[\,(R^{\,n}_{13})_L \ \var{W1(IEL)} + (R^{\,n}_{23})_L \ \var{W2(IEL)}\,\right]$
\end{itemize}



\item [$\star$] Appel de \fort{vectds}\footnote{Le r\'esultat est stock\'e dans
\var{VISCF} et \var{VISCB}. Dans \fort{vectds}, les valeurs aux faces internes
sont interpol\'ees lin\'eairement sans reconstruction et \var{VISCB} est mis \`a
z\'ero.} pour assembler $\displaystyle\left[ \tens{E}^n\,\grad{R}^{\,n}_{ij}
\right]\,.\,\vect{n}_{\,lm}S_{\,lm}$ aux faces $lm$.
\item [$\star$] Appel de \fort{divmas} pour calculer la divergence du flux d\'efini par \var{VISCF} et \var{VISCB}.
Le r\'esultat est stock\'e dans \var{W4}.\\
Ajout au second membre \var{SMBR}.\\
\var{SMBR} = \var{SMBR} + \var{W4}
\end{itemize}

A l'issue de cette \'etape, seule la partie extradiagonale de la diffusion prise
enti\`erement explicite~:
 $$\sum\limits_{m\in
Vois(l)}\left[\ \tens{E}^n\,\grad{R}^{\,n}_{ij} \right]_{\,lm}\,.\,\vect{n}_{\,lm}S_{\,lm}$$ a \'et\'e calcul\'ee.\\

\item Calcul de la partie diagonale du terme de diffusion\footnote{Seule la
composante normale  du  gradient de $R_{ij}$ aux faces sera implicite.} :\\
Comme on l'a d\'eja vu, une partie de cette contribution sera trait\'ee en
implicite (celle relative \`a la composante normale du gradient) et donc
ajout\'ee au second membre par \fort{bilsc2} ; l'autre
partie sera explicite et prise en compte dans $f_s^{\,exp}$.
\begin{itemize}
\item [$\star$] On effectue une boucle d'indice \var{IEL} sur les cellules
$\Omega_l$ de centre $L$ :
\begin{itemize}
\item [$\Rightarrow$] $\displaystyle \var{TRRIJ }= \frac{1}{2} (R^{\,n}_{ii})_L $
\item [$\Rightarrow$] $\displaystyle \var{CSTRIJ} = \rho^n_L \ C_S \ \frac{(R^{\,n}_{ii})_L}{2\,\varepsilon^n_L}$
\item [$\Rightarrow$] $\displaystyle \var{W4(IEL)} = \rho^n_L \ C_S \
\frac{(R^{\,n}_{ii})_L}{2\,\varepsilon^n_L} \ (R^{\,n}_{11})_L$
\item [$\Rightarrow$] $\displaystyle \var{W5(IEL)} = \rho^n_L \ C_S \ \frac{(R^{\,n}_{ii})_L}{2\,\varepsilon^n_L}\ (R^n_{22})_L$
\item [$\Rightarrow$] $\displaystyle \var{W6(IEL)} = \rho^n_L \ C_S \ \frac{(R^{\,n}_{ii})_L}{2\,\varepsilon^n_L} \ (R^n_{33})_L$
\end{itemize}

%\item Traitement du parall\'elisme et de la p\'eriodicit\'e.

\item [$\star$] On effectue une boucle d'indice \var{IFAC} sur les faces
purement internes $lm$ pour remplir le tableau \var{VISCF} :
\begin{itemize}
\item [$\Rightarrow$] Identification des cellules $\Omega_l$ et $\Omega_m$ de
centre respectif $L$ (variable \var{II}) et $M$ (variable \var{JJ}), se trouvant de chaque c\^ot\'e de la face
$lm$\footnote{La normale \`a la face est orient\'ee de L vers M.}.
\item [$\Rightarrow$] Calcul du carr\'e de la surface de la face. La valeur est
stock\'ee dans le tableau \var{SURFN2}.
\item [$\Rightarrow$] Interpolation du gradient de $R^{\,n}_{ij}$ \`a la face
$lm$ (gradient facette $\left[\grad{R}^{\,n}_{ij}\right]_{\,lm}$) :
\begin{equation}\notag
\left\{\begin{array}{ll}
\var{GRDPX} &= \displaystyle \frac{1}{2} \left(\var{W1(II)} + \var{W1(JJ)}
\right) \\
&\\
\var{GRDPY} &= \displaystyle \frac{1}{2} \left(\var{W2(II)} + \var{W2(JJ)} \right) \\
&\\
\var{GRDPZ} &= \displaystyle \frac{1}{2} \left(\var{W3(II)} + \var{W3(JJ)} \right)
\end{array}\right.
\end{equation}
\item [$\Rightarrow$] Calcul du gradient de $R^{\,n}_{ij}$ normal \`a la face
$lm$, $\left[\grad{R}^{\,n}_{ij}\right]_{\,lm}.\vect{n}_{\,lm}\,S_{\,lm}$.\\

$\displaystyle \var{GRDSN} =  \var{GRDPX} \ \var{SURFAC(1,IFAC)} + \var{GRDPY} \ \var{SURFAC(2,IFAC)} +  \var{GRDPZ} \ \var{SURFAC(3,IFAC)}$
$\var{SURFAC}$ est le vecteur surface de la face \var{IFAC}.


\item [$\Rightarrow$] calcul de
 $\left[\grad{R^{\,n}_{ij}} - (\grad
R^{\,n}_{ij}\,.\,\vect{n}_{\,lm})\vect{n}_{\,lm}\right]$, les vecteurs \'etant
calcul\'es \`a la face $lm$ :
\begin{equation}\notag
\left\{\begin{array}{lll}
&\displaystyle \var{GRDPX} &= \var{GRDPX} - \displaystyle\frac{\var{GRDSN}}{\var{SURFN2}} \ \var{SURFAC(1,IFAC)}\\
&&\\
&\displaystyle \var{GRDPY} &= \var{GRDPY} - \displaystyle\frac{\var{GRDSN}}{\var{SURFN2}} \ \var{SURFAC(2,IFAC)} \\
&&\\
&\displaystyle \var{GRDPZ} &= \var{GRDPZ} - \displaystyle\frac{\var{GRDSN}}{\var{SURFN2}} \ \var{SURFAC(3,IFAC)}
\end{array}\right.
\end{equation}
\item [$\Rightarrow$] finalisation du calcul de l'expression totalement
explicite
 $$\left[ \tens{D}^n\,\left( \grad{R^{\,n}_{ij}} - (\grad R^{\,n}_{ij}\,.\,\vect{n}_{\,lm})\,\vect{n}_{\,lm}\right) \right]\,.\,\vect{n}_{\,lm}$$
\begin{equation}\notag
\begin{array} {ll}
\displaystyle \var{VISCF} = &
 \displaystyle\frac{1}{2} (\ \var{W4(II)} +\ \var{W4(JJ)}) \ \var{GRDPX} \
\var{SURFAC(1,IFAC)})\ + \\
&\\
&  \displaystyle\frac{1}{2} (\ \var{W5(II)} +\ \var{W5(JJ)}) \ \var{GRDPY} \
\var{SURFAC(2,IFAC)})\ + \\
&\\
&  \displaystyle\frac{1}{2} (\ \var{W6(II)} +\ \var{W6(JJ)}) \ \var{GRDPZ} \ \var{SURFAC(3,IFAC)})
\end{array}
\end{equation}
\end{itemize}

\item [$\star$] Mise \`a z\'ero du tableau \var{VISCB}.

\item [$\star$] Appel de \fort{divmas} pour calculer la divergence de~:
 $$\tens{D}^{\,n}\,\left( \grad{R^{\,n}_{ij}} - (\grad R^{\,n}_{ij}\,.\,\vect{n}_{\,lm})\vect{n}_{\,lm}\right)$$ d\'efini aux faces dans \var{VISCF} et \var{VISCB}.

Le r\'esultat est stock\'e dans le tableau \var{W1}.\\
Ajout au second membre \var{SMBR}.\\
$\var{SMBR} = \var{SMBR} + \var{W1}$
\end{itemize}
\item Calcul de la viscosit\'e orthotrope $\gamma^n_{\,lm}$ \`a la face $lm$ de la variable principale
$R^{\,n}_{ij}$\\
Ce calcul permet au sous-programme \fort{codits} de compl\'eter le second membre
\var{SMBR} par :
\begin{equation}
\begin{array} {ll}
& \sum\limits_{m\in Vois(l)}
\mu^n_{\,lm}\,\left(\grad{R}^{\,n}_{ij}\,.\,\vect{n}_{\,lm}\right)S_{\,lm}
 + \sum\limits_{m\in Vois(l)} \left(\grad{R}^{\,n}_{ij}
\,.\,\vect{n}_{\,lm}\right)\left[\tens{D}^{\,n}\,\vect{n}_{\,lm}\right]_{\,lm}\,.\,\vect{n}_{\,lm}\
S_{\,lm}\\
& = \sum\limits_{m\in Vois(l)}(\,\mu^n_{\,lm}\, + \,\gamma^n_{\,lm}\,)\,\left(\grad{R}^{\,n}_{ij}\,.\,\vect{n}_{\,lm}\right)S_{\,lm}
\end{array}
\end{equation}
sans pr\'eciser la nature de la face $lm$, {\it via} l'appel \`a \fort{bilsc2}  et de disposer de la quantit\'e
$(\mu^n_{\,lm}\, + \gamma^n_{\,lm})$ pour construire sa
matrice simplifi\'ee.\\
\begin{itemize}
\item [$\star$] On effectue une boucle d'indice \var{IEL} sur les cellules
$\Omega_l$ :
\begin{itemize}
\item [$\Rightarrow$] $\displaystyle \var{TRRIJ }= \frac{1}{2} (R^{\,n}_{ii})_L $
\item [$\Rightarrow$] $\displaystyle \var{RCSTE} = \rho^n_L \ C_S \ \frac{ (R^{\,n}_{ii})_L}{2\,\varepsilon^n_L} $
\item [$\Rightarrow$] $\displaystyle \var{W1(IEL)} = \mu^n +\rho^n_L \ C_S \ \frac{
(R^{\,n}_{ii})_L}{2\,\varepsilon^n_L}\ (R^n_{11})_L$
\item [$\Rightarrow$] $\displaystyle \var{W2(IEL)} = \mu^n + \rho^n_L \ C_S \ \frac{ (R^{\,n}_{ii})_L}{2\,\varepsilon^n_L}\ (R^n_{22})_L$
\item [$\Rightarrow$] $\displaystyle \var{W3(IEL)} = \mu^n + \rho^n_L \ C_S \ \frac{ (R^{\,n}_{ii})_L}{2\,\varepsilon^n_L}\ (R^n_{33})_L$
\end{itemize}

\item [$\star$] Appel de \fort{visort} pour calculer la viscosit\'e orthotrope
\footnote{Comme dans le sous-programme \fort{viscfa}, on multiplie la viscosit\'e par
$\displaystyle \frac{S_{\,lm}}{\overline{L'M'}}$, o\`u $S_{\,lm}$ et
$\overline{L'M'}$ repr\'esentent respectivement la surface de la face $lm$ et la
mesure alg\'ebrique du segment reliant les projections des centres des cellules
voisines sur la normale \`a la face. On garde dans ce sous-programme  la possibilit\'e d'interpoler la viscosit\'e aux cellules lin\'eairement ou d'utiliser une moyenne harmonique. La viscosit\'e au bord est celle de la cellule de bord correspondante.}
$ \gamma^n_{\,lm} = (\tens{D}^{\,n}\,\vect{n}_{\,lm}).\vect{n}_{\,lm}$ aux faces $lm$

Le r\'esultat est stock\'e dans les tableaux \var{VISCF} et \var{VISCB}.
\end{itemize}

\item appel de \fort{codits} pour la r\'esolution de l'\'equation de
convection/diffusion/termes sources de la variable $R_{ij}$. Le terme source est
\var{SMBR}, la viscosit\'e \var{VISCF} aux faces purement internes (
resp. \var{VISCB} aux faces de bord) et \var{FLUMAS} le flux de masse interne
 ( resp. \var{FLUMAB} flux de masse au bord). Le r\'esultat est la variable $R_{ij}$ au temps
$n+1$, donc $R^{\,n+1}_{ij}$. Elle est stock\'ee dans le tableau \var{RTP} des
variables mises \`a jour.

\end{itemize}

\etape{Appel de \fort{reseps} pour la r\'esolution de la variable $\varepsilon$}

On d\'ecrit ci-dessous le sous-programme \fort{reseps}, les commentaires du sous-programme \fort{resrij} ne seront pas r\'ep\'et\'es puisque les deux sous-programmes ne diff\`erent que par leurs termes sources.

\begin{itemize}
\item Initialisation \`a z\'ero de \var{SMBR} et \var{ROVSDT}.

\item{Lecture et prise en compte des termes sources utilisateur pour la variable $\varepsilon$ :}

Appel de \fort{ustsri} pour charger les termes sources utilisateurs. Ils sont
stock\'es dans les tableaux suivants :\\
pour la cellule $\Omega_l$ repr\'esent\'ee par $\var{IEL}$ de centre $L$, on a :
\begin{equation}\notag
\left\{\begin{array}{lll}
&\var{ROVSDT(IEL)}&= |\Omega_l| \ \alpha_{\varepsilon}\\
&\var{SMBR(IEL)}&=|\Omega_l| \ \beta_{\varepsilon}\\
\end{array}\right.
\end{equation}
On affecte alors les valeurs ad\'equates au second membre \var{SMBR} et \`a la
diagonale \var{ROVSDT} comme suit :
\begin{equation}\notag
\left\{\begin{array}{lll}
&\var{SMBR(IEL)} &= \var{SMBR(IEL)} +\ |\Omega_l| \ \alpha_{\,\varepsilon} \
\varepsilon^n_L \\
&\var{ROVSDT(IEL)}&= \text{max }(-\ |\Omega_l| \ \alpha_{\,\varepsilon},0)\\
\end{array}\right.
\end{equation}

\item{Calcul du terme source de masse si $\Gamma_L > 0$ :
\begin{equation}\notag
\left\{\begin{array}{lll}
&\displaystyle \var{SMBR(IEL)} = \var{SMBR(IEL)} + |\Omega_l| \ \Gamma_L \
(\varepsilon^{\,in}_L -\varepsilon^n_L) \\
&\displaystyle \var{ROVSDT(IEL)}= \var{ROVSDT(IEL)} + |\Omega_l| \ \Gamma_L
\end{array}\right.
\end{equation}
\item Calcul du terme d'accumulation de masse et du terme instationnaire \\
On stocke $\displaystyle \var{W1}= \int_{\Omega_l}\dive\,(\rho\,\vect{u})\,d\Omega$
calcul\'e par \fort{divmas} \`a l'aide des flux de masse internes et aux bords.\\
On incr\'emente ensuite \var{SMBR} et \var{ROVSDT}.
\begin{equation}\notag
\left\{\begin{array}{lll}
&\var{SMBR(IEL)} &= \var{SMBR(IEL)} + \var{ICONV}\ \varepsilon^n_L\,(\displaystyle
\int_{\Omega_l}\dive\,(\rho\,\vect{u})\ d\Omega) \\
&\var{ROVSDT(IEL)}& = \var{ROVSDT(IEL)} +  \var{ISTAT}\,\displaystyle
\frac{\rho^n_L \ |\Omega_l|}{\Delta t^n} -  \var{ICONV}\ (\displaystyle
\int_{\Omega_l}\dive\,(\rho\,\vect{u})\ d\Omega) \\
\end{array}\right.
\end{equation}

\item Traitement du terme de production
 $\displaystyle \rho\,C_{\varepsilon_1}\,\frac{\varepsilon}{k}\,\mathcal{P}$
 et du terme de dissipation $-\,\displaystyle \rho\,C_{\varepsilon_2}\,\frac{\varepsilon}{k}\,\varepsilon$ \\
pour cela, on effectue une boucle d'indice \var{IEL} sur les cellules $\Omega_l$
de centre $L$ :
\begin{itemize}
\item [$\Rightarrow$] $\displaystyle \var{TRPROD}= \frac{1}{2} (\mathcal{P}^n_{ii})_L = \frac{1}{2} \left[ \var{PRODUC(1,IEL)} +  \var{PRODUC(2,IEL)} +  \var{PRODUC(3,IEL)} \right] $
\item [$\Rightarrow$] $\displaystyle \var{TRRIJ }= \frac{1}{2} (R^n_{ii})_L $
\item [$\Rightarrow$] $\displaystyle \var{SMBR(IEL)} = \var{SMBR(IEL)} + \rho^n_L
|\Omega_l| \left[ -C_{\varepsilon_2} \ \frac{2\,(\varepsilon^n_L)^2}{(R^n_{ii})_L} + C_{\varepsilon_1} \ \frac{\varepsilon^n_L}{(R^n_{ii})_L}\ (\mathcal{P}^n_{ii})_L \right] $
\item [$\Rightarrow$] $\displaystyle \var{ROVSDT(IEL)} = \var{ROVSDT(IEL)} + C_{\varepsilon_2} \ \rho^n_L \ |\Omega_l| \ \frac{2\,\varepsilon^n_L}{(R^n_{ii})_L}$
\end{itemize}

\item Appel de \fort{rijthe} pour le calcul des termes de gravit\'e $\mathcal{G}^n_{\varepsilon}$ et ajout dans \var{SMBR}.

$ \var{SMBR} = \var{SMBR} + \mathcal{G}^n_{\varepsilon}$\\
Ce calcul n'a lieu que si $\var{IGRARI()} = 1$.

\item Calcul de la diffusion de $\varepsilon$ \\
 Le terme $\dive \left[\mu\, \grad(\varepsilon) + \tens{A'}\,\grad(\varepsilon)
\right]$ est calcul\'e exactement de la m\^eme mani\`ere que pour les tensions
de Reynolds $R_{ij}$ en rempla\c cant $\tens{A}$ par $\tens{A'}$.

\item Appel de \fort{codits} pour la r\'esolution de l'\'equation de
convection/diffusion/termes sources de la variable principale $\varepsilon$. Le
r\'esultat $\varepsilon^{\,n+1}$ est stock\'e dans le tableau \var{RTP} des
variables mises \`a jour.
}
\end{itemize}

\etape{clippings finaux}
On passe enfin dans le sous-programme  \fort{clprij} pour faire un clipping \'eventuel
des variables $R^{\,n+1}_{ij}$ et $\varepsilon^{\,n+1}$. Le sous-programme
\fort{clprij} est appel\'e\footnote{L'option
$\var{ICLIP} = 1$ consiste en un clipping minimal des variables $R_{ii}$ et
$\varepsilon$ en prenant la valeur absolue de ces variables puisqu'elles ne
peuvent \^etre que positives.} avec $\var{ICLIP} = 2$ . Cette option
\footnote{Quand la valeur des grandeurs $R_{ii}$ ou $\varepsilon$ est
n\'egative, on la remplace par le minimum entre sa valeur absolue et (1,1)
fois la valeur obtenue au pas de temps pr\'ec\'edent.} contient l'option $\var{ICLIP} = 1$  et permet de v\'erifier l'in\'egalit\'e de Cauchy-Schwarz sur les grandeurs extra-diagonales du tenseur $\tens{R}$ (pour $i \neq j$, $|R_{ij}|^2 \le R_{ii} R_{jj}$).


%%%%%%%%%%%%%%%%%%%%%%%%%%%%%%%%%%
%%%%%%%%%%%%%%%%%%%%%%%%%%%%%%%%%%
\section{Points \`a traiter}
%%%%%%%%%%%%%%%%%%%%%%%%%%%%%%%%%%
%%%%%%%%%%%%%%%%%%%%%%%%%%%%%%%%%%
Sauf mention explicite, $\phi$ repr\'esentera une tension de Reynolds ou la dissipation turbulente ($\phi = R_{ij} \ \text{ou} \ \varepsilon$).

\begin{itemize}
\item {La vitesse utilis\'ee pour le calcul de la production est explicite. Est-ce qu'une implicitation peut am\'eliorer la pr\'ecision temporelle de $\phi$ \footnote{Cette remarque peut \^etre g\'en\'eralis\'ee. En effet, peut-on envisager d'actualiser les variables d\'ej\`a r\'esolues (sans r\'eactualiser les variables turbulentes apr\`es leur r\'esolution)? Ceci obligerait \`a modifier les sous-programmes tels que \fort{condli} qui sont appel\'es au d\'ebut de la boucle en temps.} ?}
\item {Dans quelle mesure le terme d'\'echo de paroi est-il valide ? En effet, ce terme est remis en question par certains auteurs.}
\item {On peut envisager la r\'esolution d'un syst\`eme hyperbolique pour les
tensions de Reynolds afin d'introduire un couplage avec le champ de vitesse.}
\item {Le flux au bord \var{VISCB} est annul\'e dans le sous-programme
\fort{vectds}. Peut-on envisager de mettre au bord la valeur de la variable
concern\'ee \`a la cellule de bord correspondant? De m\^eme, il faudrait se
pencher sur les hypoth\`eses sous-jacentes \`a l'annulation des contributions
aux bords de \var{VISCB} lors du calcul de : $$\left[ \tens{D}^n\,\left( \grad{R^{\,n}_{ij}} - (\grad R^{\,n}_{ij}\,.\,\vect{n}_{\,lm})\,\vect{n}_{\,lm}\right) \right]\,.\,\vect{n}_{\,lm}.$$}
\item {Un probl\`eme de pond\'eration appara\^\i t plus g\'en\'eralement. Si on prend la partie explicite de $\tens{D}\,\grad(\phi)$, la pond\'eration aux faces internes utilise le coefficient $\displaystyle\frac{1}{2}$ avec pond\'eration s\'epar\'ee de $\tens{D}$ et $\grad(\phi)$, alors que pour $\tens{E}\,\grad(\phi)$, on calcule d'abord ce terme aux cellules pour ensuite l'interpoler lin\'eairement aux faces \footnote{Cette interpolation se fait dans \fort{vectds} avec des coefficients de pond\'eration aux faces.}. Ceci donne donc deux types d'interpolations pour des termes de m\^eme nature.}
\item {On laisse la possibilit\'e dans \fort{visort} d'utiliser une moyenne
harmonique aux faces. Est-ce que ceci est valable puisque les interpolations
utilis\'ees lors du calcul de la partie explicite de $\tens{A}\,\grad{\phi}$
sont des moyennes arithm\'etiques ?}
\item {Les techniques adopt\'ees lors du clipping sont \`a revoir.}
\item {On utilise dans le cadre du mod\`ele $\displaystyle R_{ij}-\varepsilon $ une semi-implicitation de termes comme $\displaystyle \phi_{ij,1}$ ou $\displaystyle -\rho\,C_{\varepsilon_2}\,\frac{\varepsilon}{k}\,\varepsilon$. On peut envisager le m\^eme type d'implicitation dans \fort{turbke} m\^eme en pr\'esence du couplage $\displaystyle k-\varepsilon$.}
\item L'adoption d'une r\'esolution d\'ecoupl\'ee fait perdre l'invariance par rotation.
\item La formulation et l'implantation des conditions aux limites de paroi
devront \^etre v\'erifi\'ees. En effet, il semblerait que, dans certains cas, des ph\'enom\`enes
``oscillatoires'' apparaissent, sans qu'il soit ais\'e d'en d\'eterminer la cause.
\item L'implicitation partielle (du fait de la r\'esolution d\'ecoupl\'ee) des
conditions aux limites conduit souvent \`a des calculs instables. Il
conviendrait d'en conna\^\i tre la raison. L'implicitation partielle avait
\'et\'e mise en \oe uvre afin de tenter d'utiliser un pas de temps plus grand
dans le cas de jets axisym\'etriques en particulier.

\end{itemize}

%                      Code_Saturne version 1.3
%                      ------------------------
%
%     This file is part of the Code_Saturne Kernel, element of the
%     Code_Saturne CFD tool.
%
%     Copyright (C) 1998-2007 EDF S.A., France
%
%     contact: saturne-support@edf.fr
%
%     The Code_Saturne Kernel is free software; you can redistribute it
%     and/or modify it under the terms of the GNU General Public License
%     as published by the Free Software Foundation; either version 2 of
%     the License, or (at your option) any later version.
%
%     The Code_Saturne Kernel is distributed in the hope that it will be
%     useful, but WITHOUT ANY WARRANTY; without even the implied warranty
%     of MERCHANTABILITY or FITNESS FOR A PARTICULAR PURPOSE.  See the
%     GNU General Public License for more details.
%
%     You should have received a copy of the GNU General Public License
%     along with the Code_Saturne Kernel; if not, write to the
%     Free Software Foundation, Inc.,
%     51 Franklin St, Fifth Floor,
%     Boston, MA  02110-1301  USA
%
%-----------------------------------------------------------------------
%
\programme{vortex}
%
\vspace{1cm}
%%%%%%%%%%%%%%%%%%%%%%%%%%%%%%%%%%
%%%%%%%%%%%%%%%%%%%%%%%%%%%%%%%%%%
\section{Fonction}
%%%%%%%%%%%%%%%%%%%%%%%%%%%%%%%%%%
%%%%%%%%%%%%%%%%%%%%%%%%%%%%%%%%%%
Ce sous-programme est d�di� � la g�n�ration des conditions d'entr�e
turbulente utilis�es en LES.


La m�thode des vortex est bas�e sur une approche de tourbillons
ponctuels. L'id�e de la m�thode consiste � injecter des tourbillons 2D dans le
plan d'entr�e du calcul, puis � calculer le champ de vitesse induit par ces
tourbillons au centre des faces d'entr�e.

%                      Code_Saturne version 1.3
%                      ------------------------
%
%     This file is part of the Code_Saturne Kernel, element of the
%     Code_Saturne CFD tool.
% 
%     Copyright (C) 1998-2007 EDF S.A., France
%
%     contact: saturne-support@edf.fr
% 
%     The Code_Saturne Kernel is free software; you can redistribute it
%     and/or modify it under the terms of the GNU General Public License
%     as published by the Free Software Foundation; either version 2 of
%     the License, or (at your option) any later version.
% 
%     The Code_Saturne Kernel is distributed in the hope that it will be
%     useful, but WITHOUT ANY WARRANTY; without even the implied warranty
%     of MERCHANTABILITY or FITNESS FOR A PARTICULAR PURPOSE.  See the
%     GNU General Public License for more details.
% 
%     You should have received a copy of the GNU General Public License
%     along with the Code_Saturne Kernel; if not, write to the
%     Free Software Foundation, Inc.,
%     51 Franklin St, Fifth Floor,
%     Boston, MA  02110-1301  USA
%
%-----------------------------------------------------------------------
%
%%%%%%%%%%%%%%%%%%%%%%%%%%%%%%%%%%
%%%%%%%%%%%%%%%%%%%%%%%%%%%%%%%%%%
\section{Discr\'etisation}
%%%%%%%%%%%%%%%%%%%%%%%%%%%%%%%%%%
%%%%%%%%%%%%%%%%%%%%%%%%%%%%%%%%%%

Le terme convectif en $\dive(\underline{u} \otimes \rho\,\underline{u})$
introduit une non lin\'earit\'e et un couplage des composantes de la vitesse
$\vect{u}$ dans l'�quation (\ref{Base_Preduv_eqqdm}). Une lin\'earisation et un d\'ecouplage
des trois composantes de la 
vitesse sont r\'ealis\'es lors de la discr\'etisation de cette \'etape de
pr\'ediction.\\
En effet, soit :
\begin{equation}
\vect{\widetilde{u}}= \vect{u}^n + \delta \vect{u} 
\end{equation}
La contribution exacte du terme convectif \`a prendre en compte dans cette
\'etape de pr\'ediction serait :\\
\begin{equation}\label{Base_Preduv_Conv_exact}
\begin{array}{ll}
\dive(\vect{\widetilde{u}} \otimes \rho\,\vect{\widetilde{u}}) =
\dive(\vect{u}^{n} \otimes \rho\,\vect{u}^{n}) + \dive(\delta \vect{u} \otimes
\rho\,\vect{u}^{n}) +  \underbrace { \dive(\vect{u}^{n} \otimes
\rho\,\delta \vect{u})}_{\text {terme couplant lin\'eaire}} +  \underbrace { \dive(\delta \vect{u} \otimes
\rho\,\delta \vect{u})}_{\text {terme couplant et non lin\'eaire}}\\
\end{array} 
\end{equation}
Les deux derniers termes de l'expression (\ref{Base_Preduv_Conv_exact}) sont {\em a priori} n�glig�s
de mani�re � obtenir un syst\`eme en vitesse qui soit d\'ecoupl\'e et donc,
�viter l'inversion d'une matrice pouvant \^etre de tr\`es grande taille. Ces
deux termes peuvent n�anmoins �tre pris en compte de mani�re plus ou moins
approch�e par extrapolation explicite du flux de masse en $n+\theta_F$ (pour le
terme couplant lin�aire provenant de la convection de $\vect{u}^{n}$ par $\delta
\vect{u}$) et utilisation d'un point-fixe par sous it�ration sur le sous
programme \fort{navsto} (pour le terme non-lin�aire, en sp�cifiant $\var{NTERUP}>1$).

L'�quation (\ref{Base_Preduv_eqqdm}) est discr�tis�e au temps $n+\theta$ � l'aide d'un
$\theta$-sch�ma, et le tenseur des pertes de charges d�compos� en une partie
diagonale $\tens{K}_{d}$ et une extradiagonale $\tens{K}_{e}$ (soit
 $\tens{K}_{pdc}=\tens{K}_{d}+\tens{K}_{e}$).\\
$\bullet$ La pression est suppos�e connue en $n-1+\theta$ (d�calage temporel
pression-vitesse) et le gradient naturellement calcul� � cet instant.\\ 
$\bullet$ Les termes sources de viscosit� secondaire, de gradient transpos\'e,
ceux provenant du mod�le de turbulence\footnote{except� $\dive (\mu_t\ (\ggrad
\underline {u}))$}, $\rho\,\tens{K}_{\,e}\ \underline{u}$, $(\rho -\rho_0)
\underline {g}$ ainsi que $\underline{T}_{s}^{\,exp}$ et
$\Gamma\,\underline{u}_{\,i}$ sont pris de mani�re explicite au temps $n$, ou
extrapol�s suivant le sch�ma en temps choisi pour les propri�t�s physique et les
termes sources.\\ 
$\bullet$ Les termes sources $\underline{u}\,\,\dive (\rho\,\underline {u})$,
$\Gamma\,\,\underline{u}$, $T_{s}^{\,imp}\,\,\underline{u}$ et
$-\rho\,\tens{K}_{\,d}\,\,\underline{u}$ sont implicit�s est calcul�s �
l'instant $n+\theta$.\\ 
$\bullet$ Le terme de diffusion $\dive (\mu_{\,tot}\,\ggrad \underline{u})$ est
implicit� : la vitesse est prise � l'instant $n+\theta$ et la viscosit�
explicit�e ou extrapol�e.\\ 
$\bullet$ Enfin, le terme de convection en $\dive(\,\underline{u} \otimes
(\rho\underline{u})\,)$ est implicit� : la composante r�solue de la vitesse est
prise en $n+\theta$, et le flux de masse, explicit�, ou extrapol� en
$n+\theta_F$. 

Par souci de clart�, on suppose, en l'absence d'indication, que les propri�tes
physiques $\Phi$ ($\rho,\,\mu_{tot},\,...$) et le flux de masse
$(\rho\underline{u})$ sont pris respectivement aux instants $n+\theta_\Phi$ et
$n+\theta_F$, o� $\theta_\Phi$ et $\theta_F$ d�pendent des sch�mas en temps
sp�cifiquement utilis�s pour ces grandeurs\footnote{cf. \fort{introd}}. 

La discr�tisation temporelle de l'�quation (\ref{Base_Preduv_eqqdm}) s'�crit alors comme suit : 

\begin{equation}\label{Base_Preduv_eq_di1}
 \begin{array}{c}
\displaystyle \rho\,\ \frac{ \underline {\widetilde{u}}^{n+1} -\underline {u}^{n} }
{\Delta t} + \dive(\,\underline{\widetilde{u}}^{n+\theta} \otimes (\rho\underline{u})\,) -\dive
(\mu_{\,tot}\,\ggrad \underline{\widetilde{u}}^{n+\theta}) =
\\
\displaystyle
 - \grad p^{n-1+\theta} + \dive (\rho\,\underline {u})\,\underline{\widetilde{u}}^{n+\theta} +(\Gamma\,\underline{u}_{\,i})^{n+\theta_S}-\Gamma^n\,\,\underline{\widetilde{u}}^{n+\theta}
\\
\begin{array}{c}
\displaystyle
- \rho\,\tens{K}_{\,d}^{n}\,\,\underline{\widetilde{u}}^{n+\theta} - (\rho\,\tens{K}_{\,e}\ \underline{u})^{n+\theta_S} + (\underline{T}_{s}^{\,exp})^{\,n+\theta_S} + T_{s}^{\,imp}\,\,\underline{\widetilde{u}}^{n+\theta}
\\
\displaystyle
+[\dive (\mu_{\,tot}\,^t\ggrad \underline {u})]^{n+\theta_S}-\frac {2} {3}[\,\grad (\mu_{\,tot}\,\dive \underline {u})]^{n+\theta_S} + (\rho -\rho_0) \underline {g}
 - (\underline{turb})^{n+\theta_S}
\end{array}
\end{array}
\end{equation}
o\`u, par souci de simplification, on a pos\'e :
\begin{equation}
\mu_{\,tot}=
\begin{cases}
\mu+\mu_t & \text{pour les mod�les � viscosit� turbulente ou en LES}, \\
\mu & \text{pour les mod�les au second ordre ou en laminaire}
\end{cases} \ 
\end{equation}
\\
et :
\begin{equation}
\underline{turb}^{n}=
\begin{cases}
\displaystyle\frac {2}{3}\grad (\rho^{n}\,k^{n}) & \text{pour les mod�les � viscosit� turbulente}, \\
\dive(\rho^{n}\,\tens{R}^n) & \text{pour les mod�les au second ordre},\\
0 & \text{en laminaire ou en LES}\\
\end{cases}
\end{equation}
Par analogie avec l'�criture du $\theta$-sch�ma pour une variable scalaire, $\,
\underline {\widetilde{u}}^{n+\theta}$ est interpol�e � partir de la vitesse
pr�dite $\underline {\widetilde{u}}^{n+1}$ de la mani\`ere suivante\footnote{si
$\theta=1/2$, ou qu'une extrapolation est utilis�e, l'ordre 2 n'est obtenu que si
le pas de temps $\Delta t$ est uniforme en temps et en espace.}~: 
\begin{equation}
\underline {\widetilde{u}}^{n+\theta}=\theta\, \underline
{\widetilde{u}}^{n+1}+(1-\theta)\, \underline {u}^{n}\\ 
\end{equation}
Avec :
\begin{equation}
\left\{
\begin{array}{ll}
\theta = 1   & \text{Pour un sch\'ema de type Euler implicite d'ordre 1.}\\
\theta = 1/2 & \text{Pour un sch\'ema de type Cranck-Nicolson d'ordre 2.}\\
\end{array}
\right.
\end{equation}

L'�quation (\ref{Base_Preduv_eq_di1}) est alors r��crite sous la forme :

\begin{equation}\label{Base_Preduv_eq_di2}
\begin{array}{c}
\displaystyle \underbrace{\left(\frac{\rho}{\Delta t} -\theta \,\dive (\rho\,\underline {u}) +\theta \,\, \Gamma^n +
\theta \,\, \rho\,\tens{K}_{\,d}^n-\theta \,T_s^{\,imp} \right)}_{\displaystyle f_s^{imp}}\, (\underline {\,\widetilde{u}}^{n+1} -\underline {u}^{n})
\\
 +\, \theta\, \dive(\underline {\widetilde{u}}^{n+1} \otimes (\rho\underline{u}))-\, \theta\,\dive (\mu_{\,tot}\,\ggrad \underline {\widetilde{u}}^{n+1}) =
\\
-\,(1-\theta)\, \dive(\underline {u}^{n} \otimes (\rho\underline{u})) +\,(1-\theta)\,\dive (\mu_{\,tot}\,\ggrad \underline {u}^{n})
\\
f_s^{exp}\left\{
\begin{array}{c}
\displaystyle 
- \grad p^{n-1+\theta} + \dive (\rho\,\underline {u})\,\underline{u}^{n} +\,(\,\Gamma^{n}\,\underline{u}_{\,i}\,)^{n+\theta_S}- \Gamma^n\,\,\underline{u}^{n}
\\
\displaystyle
-(\,\rho\,\tens{K}_{\,e}\ \underline{u}\,)^{n+\theta_S} -\rho\,\tens{K}_{\,d}^n\ \underline{u}^{n}+ (\underline{T}_{s}^{\,exp})^{\,n+\theta_S} + T_s^{\,imp}\,\,\underline {u}^{n} 
\\
\displaystyle
+[\dive (\mu_{\,tot}\,^t\ggrad \underline {u}\,)]^{n+\theta_S}-\frac {2} {3}[\,\grad (\mu_{\,tot}\,\dive \underline {u}\,)]^{n+\theta_S} + (\rho -\rho_0) \underline {g}-(\underline{turb})^{n+\theta_S}
\end{array}
\right.
\end{array}
\end{equation}

d'o� l'�quation r�solue par le sous-programme \fort{codits} :
\begin{equation}\begin{array}{c}
\displaystyle
f_s^{\,imp}(\underline {\widetilde{u}}^{n+1}-\underline {u}^{n}) + \theta\, \dive(\underline{\widetilde{u}}^{n+1} \otimes (\rho
\underline{u})) - \theta\,\dive (\,\mu_{\,tot}\,\ggrad \underline{\widetilde{u}}^{n+1}) = 
\\\\
\displaystyle
-(1-\theta)\,\dive(\underline{u}^{n} \otimes (\rho \underline{u}))+(1-\theta)\,\dive (\,\mu_{\,tot}\,\ggrad \underline{u}^{n})
+ \underline{f}_{\,s}^{\,exp}
\end{array}
\end{equation}
La m\'ethode de discr\'etisation spatiale est d\'evelopp\'ee dans le sous-programme \fort{codits}.\\



\minititre{Remarques :}
{\tiny$\blacksquare$} Dans le cas standard sans extrapolation, le terme
$-\,T_s^{\,imp}$ n'est ajout� � $f_s^{\,imp}$ que s'il est positif afin de ne
pas affaiblir la dominance de la diagonale de la matrice � inverser.\\ 
{\tiny$\blacksquare$} Si une extrapolation est utilis�e, par contre,
$\,T_s^{\,imp}$ est ajout� � $f_s^{\,imp}$ quel que soit son signe. En effet, l'id�e intuitive qui
consiste � prendre~: 
\begin{equation}
\begin{cases}
\displaystyle
(\underline{T}_{s}^{\,exp} + T_{s}^{\,imp}\,\underline {u})^{\,n+\theta_S} &
\text{si } T_{s}^{\,imp} > 0\\ 
\displaystyle
(\underline{T}_{s}^{\,exp})^{\,n+\theta_S} + T_{s}^{\,imp}\,\underline{u}^{n+\theta} &\text{sinon}\\
\end{cases}
\end{equation} 
aboutit � une incoh�rence dans le traitement si $T_s^{imp}$ change de signe
entre deux pas de temps.\\ 
{\tiny$\blacksquare$} la partie diagonale $\tens{K}_{\,d}$ du terme
de perte de charge est utilis�e dans $f_s^{\,imp}$. En fait, pour \^etre rigoureux,
il faudrait ne retenir que les contributions positives (point signal\'e dans le
sous-programme utilisateur associ\'e \fort{uskpdc}). Cette prise en compte sera \`a am\'eliorer.\\
{\tiny$\blacksquare$} Le terme $\theta\,\Gamma^{n}-\theta\,\dive
(\rho\,\underline {u})$ ne pose pas de probl�me pour la 
dominance de la diagonale de la matrice car il est exactement compens� par le
terme de convection (cf. \fort{covofi}). 


%                      Code_Saturne version 1.3
%                      ------------------------
%
%     This file is part of the Code_Saturne Kernel, element of the
%     Code_Saturne CFD tool.
%
%     Copyright (C) 1998-2007 EDF S.A., France
%
%     contact: saturne-support@edf.fr
%
%     The Code_Saturne Kernel is free software; you can redistribute it
%     and/or modify it under the terms of the GNU General Public License
%     as published by the Free Software Foundation; either version 2 of
%     the License, or (at your option) any later version.
%
%     The Code_Saturne Kernel is distributed in the hope that it will be
%     useful, but WITHOUT ANY WARRANTY; without even the implied warranty
%     of MERCHANTABILITY or FITNESS FOR A PARTICULAR PURPOSE.  See the
%     GNU General Public License for more details.
%
%     You should have received a copy of the GNU General Public License
%     along with the Code_Saturne Kernel; if not, write to the
%     Free Software Foundation, Inc.,
%     51 Franklin St, Fifth Floor,
%     Boston, MA  02110-1301  USA
%
%-----------------------------------------------------------------------
%

%%%%%%%%%%%%%%%%%%%%%%%%%%%%%%%%%%
%%%%%%%%%%%%%%%%%%%%%%%%%%%%%%%%%%
\section{Mise en \oe uvre}
%%%%%%%%%%%%%%%%%%%%%%%%%%%%%%%%%%
%%%%%%%%%%%%%%%%%%%%%%%%%%%%%%%%%%
La num\'ero de la phase trait\'ee fait partie des arguments de \fort{turrij}. On
omettra volontairement de le pr\'eciser dans ce qui suit, on indiquera par $(\ )$ la
notion de tableau s'y rattachant.

\etape{Calcul des termes de production $\tens{\mathcal{P}}$}
\begin{itemize}
\item [$\star$] Initialisation \`a z\'ero du tableau \var{PRODUC} dimensionn\'e \`a $\var{NCEL}\times 6$.
\item [$\star$] On appelle trois fois \fort{grdcel} pour calculer les gradients des composantes de la vitesse $u$, $v$ et
$w$ prises au temps $n$.

Au final, on a :\\
$\displaystyle
\begin{array} {ll}
\var{PRODUC(1,IEL)} = & \displaystyle - 2 \left[ R_{11}^{\,n} \frac{\partial u^{\,n}} {\partial x} +R_{12}^{\,n} \frac{\partial u^{\,n}} {\partial y}+R_{13}^{\,n} \frac{\partial u^{\,n}} {\partial z} \right] \text{        (production de $R_{11}^{\,n}$)}\\
\var{PRODUC(2,IEL)} = & \displaystyle - 2 \left[ R_{12}^{\,n} \frac{\partial v^{\,n}} {\partial x} +R_{22}^{\,n} \frac{\partial v^{\,n}} {\partial y}+R_{23}^{\,n} \frac{\partial v^{\,n}} {\partial z} \right] \text{        (production de $R_{22}^{\,n}$)}\\
\var{PRODUC(3,IEL)} = & \displaystyle - 2 \left[ R_{13}^{\,n} \frac{\partial w^{\,n}} {\partial x} +R_{23}^{\,n} \frac{\partial w^{\,n}} {\partial y}+R_{33}^{\,n} \frac{\partial w^{\,n}} {\partial z} \right] \text{        (production de $R_{33}^{\,n}$)}\\
\var{PRODUC(4,IEL)} = & \displaystyle - \left[ R_{12}^{\,n} \frac{\partial u^{\,n}} {\partial x} +R_{22}^{\,n} \frac{\partial u^{\,n}} {\partial y}+R_{23}^{\,n} \frac{\partial u^{\,n}} {\partial z} \right] \\
& \displaystyle - \left[ R_{11}^{\,n} \frac{\partial v^{\,n}} {\partial x} +R_{12}^{\,n} \frac{\partial v^{\,n}} {\partial y}+R_{13}^{\,n} \frac{\partial v^{\,n}} {\partial z} \right] \text{        (production de $R_{12}^{\,n}$)} \\
\var{PRODUC(5,IEL)} = & \displaystyle - \left[ R_{13}^{\,n} \frac{\partial u^{\,n}} {\partial x} +R_{23}^{\,n} \frac{\partial u^{\,n}} {\partial y}+R_{33}^{\,n} \frac{\partial u^{\,n}} {\partial z} \right] \\
& \displaystyle - \left[ R_{11}^{\,n} \frac{\partial w^{\,n}} {\partial x} +R_{12}^{\,n} \frac{\partial w^{\,n}} {\partial y}+R_{13}^{\,n} \frac{\partial w^{\,n}} {\partial z} \right] \text{        (production de $R_{13}^{\,n}$)} \\
\var{PRODUC(6,IEL)} = & \displaystyle - \left[ R_{13}^{\,n} \frac{\partial v^{\,n}} {\partial x} +R_{23}^{\,n} \frac{\partial v^{\,n}} {\partial y}+R_{33}^{\,n} \frac{\partial v^{\,n}} {\partial z} \right] \\
& \displaystyle - \left[ R_{12}^{\,n} \frac{\partial w^{\,n}} {\partial x} +R_{22}^{\,n} \frac{\partial w^{\,n}} {\partial y}+R_{23}^{\,n} \frac{\partial w^{\,n}} {\partial z} \right]  \text{        (production de $R_{23}^{\,n}$)}
\end{array}
$
\end{itemize}

\etape{Calcul du gradient de la masse volumique $\rho^n$ prise au d\'ebut du pas
de temps courant\footnote{{\it i.e.} calcul\'ee \`a partir des
variables du pas de temps pr\'ec\'edent $n$ si n\'ecessaire.} $(n+1)$}
Ce calcul n'a lieu que si les termes de gravit\'e doivent \^etre pris en compte
($\var{IGRARI()} =1$).
\begin{itemize}
\item [$\star$] Appel de \fort{grdcel}  pour calculer le gradient de $\rho^n$
dans les trois directions de l'espace. Les conditions aux limites sur $\rho^n$
sont des conditions de Dirichlet puisque la valeur de $\rho^n$ aux faces de bord
$ik$ (variable \var{IFAC}) est connue et vaut $\rho_{\,b_{\,ik}}$. Pour \'ecrire les conditions aux limites
sous la forme habituelle, $$\rho_{\,b_{\,ik}} = \var{COEFA} + \var{COEFB}
\,\rho^n_{\,I'}$$ on pose alors $\var{COEFA}=
\var{PROPCE(IFAC,IPPROB(IROM(IPHAS)))}$ et $\var{COEFB} = \var{VISCB} = 0$.\\
$\var{PROPCE(1,IPPROB(IROM(IPHAS)))}$ (resp.$\var{VISCB}$) est utilis\'e en lieu
et place de l'habituel \var{COEFA} ($\var{COEFB}$), lors de l'appel \`a \fort{grdcel}.\\
On a donc :\\
$\displaystyle \var{GRAROX}= \frac{\partial \rho^n}{\partial x}\ $,$\displaystyle \ \var{GRAROY}= \frac{\partial
\rho^n}{\partial y}$ et $
\displaystyle \ \var{GRAROZ}= \frac{\partial \rho^n}{\partial z}\ $.

\end{itemize}

Le gradient de $\rho^n$ servira \`a calculer les termes de production par effets de gravit\'e si ces derniers sont pris en compte.

\etape{Boucle \var{ISOU} de $1$ \`a $6$ sur les tensions de Reynolds}
Pour $\var{ISOU} = 1,2,3,4,5,6$, on r\'esout respectivement et dans
l'ordre  les
\'equations de $R_{11}$, $R_{22}$, $R_{33}$, $R_{12}$, $R_{13}$ et $R_{23}$ par
l'appel au sous-programme \fort{resrij}.\\
La r\'esolution se fait par incr\'ement $\delta {R}_{ij}^{\,n+1,k+1}$ , en utilisant la m\^eme m\'ethode que
celle d\'ecrite dans le sous-programme \fort{codits}. On adopte ici les m\^emes notations.
\var{SMBR} est le second membre du syst\`eme \`a inverser, syst\`eme portant sur
les incr\'ements de la variable. \var{ROVSDT} repr\'esente la diagonale de la
matrice, hors convection/diffusion.\\
On va r\'esoudre l'\'equation (\ref{Base_Turrij_Eq_Temp_Rij}) sous forme incr\'ementale en
utilisant \fort{codits}, soit :
\begin{equation}\label{Base_Turrij_Eq_Temp_deltaRij}
\begin{array}{ll}
&\displaystyle \underbrace{\left(\frac {\rho^n_L}{\Delta t^n}
+ \rho^n_L \,C_1\,\frac{\varepsilon^n_L}{k^n_L}(1-\frac{\delta_{ij}}{3})
 - m^n_{\,lm} + \Gamma_L\,+ max(-\alpha^n_{R_{ij}},0)\right)\,|\Omega_l|}
_{\text {$\var{ROVSDT}$ contribuant
\`a la diagonale de la matrice simplifi\'ee de \fort{matrix}}}\,(\delta{R}_{ij}^{\,n+1,p+1})_{\,L}\\\\
&  \underbrace{+\sum\limits_{m\in Vois(l)}\displaystyle \left[
 m^n_{\,lm} \delta R_{ij,\,f_{\,lm}}^{\,n+1,p+1}
- (\mu^n_{\,lm} + \gamma^n_{\,lm})\
\frac{({\delta R}_{ij}^{\,n+1,p+1})_{M}-({\delta R}_{ij}^{\,n+1,p+1})_{L})}{\overline{L'M'}}\,
S_{\,lm} \right]}_{\text { convection upwind pur et diffusion non reconstruite
relatives \`a la matrice simplifi\'ee de \fort{matrix}\footnotemark}} \\
% voir le texte de la footmark plus bas
&= - \displaystyle\frac {\rho^n_L}{\Delta t^n}\,\left(\,(R^{\,n+1,p}_{ij})_L - (R^{\,n}_{ij})_L\,\right)\\
&-\,\underbrace{\displaystyle\int_{\Omega_l} \left(
\dive\,[\,(\rho\,\vect{u})^n\,R^{\,n+1,p}_{ij} - (\mu^n\,+ \gamma^n\,)\,
\grad{R^{\,n+1,p}_{ij}}\,]\right)\,d\Omega}_{\text {convection et diffusion
trait\'ees par \fort{bilsc2}}}\\
&+\displaystyle \int_{\Omega_l} \left[\,\mathcal{P}^{\,n+1,p}_{ij} + \mathcal{G}^{\,n+1,p}_{ij}
- \displaystyle\rho^n \,C_1\,\frac{\varepsilon^n}{k^n}\left[R^{\,n+1,p}_{ij}-
\frac{2}{3}\,k^n\,\delta_{ij}\right] + \phi^{\,n+1,p}_{ij,2} +
\phi^{\,n+1,p}_{ij,w}\,\right]\, d\Omega \\
& + \displaystyle\int_{\Omega_l} \left[- \frac{2}{3} \rho^n \varepsilon^n \delta_{ij}
 + \Gamma\,(\,R^{\,in}_{ij} - R^{\,n+1,p}_{ij}\,) +
\alpha^n_{R_{ij}}\,R^{\,n+1,p}_{ij}+ \beta^n_{R_{ij}}\right]\, d\Omega\\
&+ \sum\limits_{m\in
Vois(l)}\displaystyle \left[\ \tens{E}^n\,\grad{R}^{\,n+1,p}_{ij} \right]_{\,lm}\,.\,\vect{n}_{\,lm}S_{\,lm}\\
&+ \sum\limits_{m\in Vois(l)}\displaystyle \left[\
\tens{D}^n\,\grad{R}^{\,n+1,p}_{ij} \right]_{\,lm}\,.\,\vect{n}_{\,lm}S_{\,lm}\\
&- \sum\limits_{m\in Vois(l)} \gamma^n_{\,lm} \left( \grad{R}^{\,n+1,p}_{ij}\,.\,\vect{n}_{\,lm} \right)  S_{\,lm}\\
&+ \sum\limits_{m\in Vois(l)}  m^n_{\,lm}\,(R^{\,n+1,p}_{ij})_L\\
\end{array}
\end{equation}
% si on ne fait pas comme ca, il n'apparait pas
\footnotetext[\thefootnote]{Si $\var{IRIJNU} = 1$, on remplace  $\mu^n_{\,lm}$ par $(\mu +
\mu_t)^n_{\,lm}$ dans l'expression de la diffusion non reconstruite
associ\'ee \`a la matrice simplifi\'ee de \fort{matrix} ($\mu_t$ d\'esigne la
viscosit\'e turbulente calcul\'ee comme en $k-\varepsilon$).}

o\`u on rappelle :\\
pour $n$ donn\'e entier positif, on d\'efinit la suite
 $({R}_{ij}^{\,n+1,p})_{p \in \grandN}$
 par :
\begin{equation}\notag
\left\{\begin{array}{l}
{R}_{ij}^{\,n+1,0} = {R}_{ij}^{\,n}\\
{R}_{ij}^{\,n+1,p+1} = {R}_{ij}^{\,n+1,p} + \delta{R}_{ij}^{\,n+1,p+1} \\
\end{array}\right.
\end{equation}
$(\delta{R}_{ij}^{\,n+1,p+1})_{\,L}$ d\'esigne la valeur sur l'\'el\'ement
$\Omega_l$ du $\text{$(\,p+1\,)$-i\`eme}$ incr\'ement de ${R}_{ij}^{\,n+1}$,
$ m^n_{\,lm}$ le flux de masse \`a l'instant $n$ \`a travers la face $lm$,
$\delta R_{ij,\,f_{\,lm}}^{\,n+1,p+1}$ vaut $({\delta
R}_{ij}^{\,n+1,p+1})_{L}$  si $ m^n_{\,lm} \geqslant 0$, $({\delta
R}_{ij}^{\,n+1,p+1})_{M}$ sinon,
$\mathcal{P}^{\,n+1,p}_{ij}$, $\phi^{\,n+1,p}_{ij,2}$, $\phi^{\,n+1,p}_{ij,w}$ les valeurs
des quantit\'es associ\'ees correspondant \`a l'incr\'ement
$(\delta{R}_{ij}^{\,n+1,p})$.\\



Tous ces termes sont calcul\'es comme suit :
\begin{itemize}
\item Terme de gauche de l'\'equation (\ref{Base_Turrij_Eq_Temp_deltaRij})\\
Dans \fort{resrij} est calcul\'ee la variable \var{ROVSDT}. Les autres
termes sont compl\'et\'es par \fort{codits}, lors de la construction de la matrice simplifi\'ee , {\it via} un
appel au sous-programme \fort{matrix}. La quantit\'e
 $(\mu^n_{\,lm} + \gamma^n_{\,lm})$ \`a la face $lm$ est calcul\'ee lors de l'appel \`a
\fort{visort}.\\
\item Second membre de l'\'equation (\ref{Base_Turrij_Eq_Temp_deltaRij})\\
Le premier terme non d\'etaill\'e est calcul\'e par le sous-programme
\fort{bilsc2}, qui applique le sch\'ema convectif choisi par l'utilisateur, qui
reconstruit ou non selon le souhait de l'utilisateur les gradients intervenants
dans la convection-diffusion.\\
Les termes sans accolade sont, eux, compl\`etement explicites et ajout\'es au fur et
\`a mesure dans \var{SMBR} pour former
l'expression $f^{\,exp}_s$ de \fort{codits}.
\end{itemize}
On d\'ecrit ci-dessous les \'etapes de \fort{resrij} :
\begin{itemize}

\item DELTIJ = 1, pour $\var{ISOU} \leqslant 3$ et DELTIJ = 0  Si $\var{ISOU} >
3$. Cette valeur repr\'esente le symbole de Kroeneker $\delta_{ij}$.

\item Initialisation \`a z\'ero de \var{SMBR} (tableau contenant le second
membre) et \var{ROVSDT} (tableau contenant la diagonale de la matrice sauf celle
relative \`a la contribution de la
diagonale des op\'erateurs de convection et de diffusion lin\'earis\'es
\footnote{qui correspondent aux sch\'emas convectif upwind pur et diffusif sans
reconstruction.}), tous deux de dimension $\var{NCEL}$.

\item Lecture et prise en compte des termes sources utilisateur pour la variable $R_{ij}$

Appel \`a \fort{ustsri} pour charger les termes sources utilisateurs. Ils sont
stock\'es comme suit. Pour la cellule $\Omega_l$ de centre $L$, repr\'esent\'ee par $\var{IEL}$, on a :\\
\begin{equation}\notag
\left\{\begin{array}{lll}
&\var{ROVSDT(IEL)}&= |\Omega_l| \ \alpha_{R_{ij}}\\
&\var{SMBR(IEL)}&=|\Omega_l| \ \beta_{R_{ij}}\\
\end{array}\right.
\end{equation}
On affecte alors les valeurs ad\'equates au second membre \var{SMBR} et \`a la
diagonale \var{ROVSDT} comme suit :
\begin{equation}\notag
\left\{\begin{array}{lll}
&\var{SMBR(IEL)} &= \var{SMBR(IEL)} +\ |\Omega_l| \ \alpha_{R_{ij}} \ (R^n_{ij})_L \\
&\var{ROVSDT(IEL)}&= \text{max }(-\ |\Omega_l| \ \alpha_{R_{ij}},0)\\
\end{array}\right.
\end{equation}
La valeur de $ \var{ROVSDT}$ est ainsi calcul\'ee pour des raisons de stabilit\'e
num\'erique. En effet, on ne rajoute sur la diagonale que les valeurs positives,
ce qui correspond physiquement \`a impliciter les termes de rappel uniquement.
\item{Calcul du terme source de masse  si $\Gamma_L > 0$}

Appel de \fort{catsma} et incr\'ementation si n\'ecessaire de \var{SMBR} et
\var{ROVSDT} {\it via} :\\
\begin{equation}\notag
\left\{\begin{array}{lll}
\displaystyle \var{SMBR(IEL)} = \var{SMBR(IEL)} + |\Omega_l| \ \Gamma_L \
\left[(R^{\,in}_{ij})_L - (R^{\,n}_{ij})_L \right] \\
\displaystyle \var{ROVSDT(IEL)}=\var{ROVSDT(IEL)} + |\Omega_l| \ \Gamma_L
\end{array}\right.
\end{equation}
\item Calcul du terme d'accumulation de masse et du terme instationnaire

On stocke $\displaystyle \var{W1}= \int_{\Omega_l}\dive\,(\rho\,\vect{u})\,d\Omega$
calcul\'e par \fort{divmas} \`a l'aide des flux de masse aux faces internes
$ m^n_{\,lm}=\sum\limits_{m\in Vois(l)}{(\rho \vect{u})_{\,lm}^n} \text{.}\,
\vect{S}_{\,lm} $ (tableau \var{FLUMAS}) et des flux de masse aux bords  $ m^n_{\,b_{lk}} = \sum\limits_{k\in{\gamma_b(l)}}{(\rho \vect{u})_{\,{b}_{lk}}^n} \text{.}\,
\vect{S}_{\,{b}_{lk}} $ (tableau \var{FLUMAB}).
On incr\'emente ensuite \var{SMBR} et \var{ROVSDT}.
\begin{equation}\notag
\left\{\begin{array}{lll}
&\var{SMBR(IEL)} &= \var{SMBR(IEL)} + \var{ICONV}\  (R^n_{ij})_L\,(\displaystyle
\int_{\Omega_l}\dive\,(\rho\,\vect{u})\ d\Omega) \\
&\var{ROVSDT(IEL)}& = \var{ROVSDT(IEL)} +  \var{ISTAT}\,\displaystyle
\frac{\rho^n_L \ |\Omega_l|}{\Delta t^n} -  \var{ICONV}\ (\displaystyle
\int_{\Omega_l}\dive\,(\rho\,\vect{u})\ d\Omega) \\
\end{array}\right.
\end{equation}
\item Calcul des termes sources de production, des termes $\displaystyle
\phi_{\,ij,1}+\phi_{\,ij,2}$ et de la dissipation~$\displaystyle-\frac{2}{3} \varepsilon\,\delta_{\,ij}$ :

On effectue une boucle d'indice \var{IEL} sur les cellules $\Omega_l$ de centre $L$ :
\begin{itemize}
\item [$\Rightarrow$] $\displaystyle \var{TRPROD}= \frac{1}{2} (\mathcal{P}^n_{ii})_L = \frac{1}{2} \left[ \var{PRODUC(1,IEL)} +  \var{PRODUC(2,IEL)} +  \var{PRODUC(3,IEL)} \right] $
\item [$\Rightarrow$] $\displaystyle \var{TRRIJ }= \frac{1}{2} (R^n_{ii})_L $
\item [$\Rightarrow$] $\displaystyle \var{SMBR(IEL)} =\ \var{SMBR(IEL)}\ +$\\
$\ \displaystyle\rho^n_L |\Omega_l| \left[ \displaystyle
\frac{2}{3}\,\delta_{\,ij} \left( \ \displaystyle \frac{ C_2}{2}\,(\mathcal{P}^n_{ii})_L\ +
(C_1-1)\ \varepsilon^n_L\, \right)\right.$\\
$ + \left.\ (1-C_2) \ \var{PRODUC(ISOU,IEL)} -
\displaystyle C_1\ \frac{2\,\varepsilon^n_L}{(R^n_{ii})_L}\ (R^n_{ij})_L \right]$
\item [$\Rightarrow$] $\displaystyle \var{ROVSDT(IEL)} = \var{ROVSDT(IEL)} +
\rho^n_L \ |\Omega_l| \ (- \displaystyle \frac{1}{3} \ \,\delta_{\,ij} + 1) \ C_1
\ \frac{2\ \varepsilon^n_L}{(R^n_{ii})_L}$
\end{itemize}
\item Appel de \fort{rijech} pour le calcul des termes d'\'echo de paroi
 $\phi^n_{ij,w}$ si $\var{IRIJEC()}=1$ et ajout dans \var{SMBR}.\\
$\var{SMBR} = \var{SMBR} + \phi^n_{ij,w}$\\
Suivant son mode de calcul (\var{ICDPAR}), la distance � la paroi est directement accessible
par \var{RA(IDIPAR+IEL-1)} (\var{|ICDPAR|} = 1) ou bien
est calcul\'ee \`a partir de $\var{IA(IIFAPA(IPHAS)+IEL - 1)}$,
qui donne pour l'\'el\'ement $\var{IEL}$ le num\'ero de la face de bord
paroi la plus  proche (\var{|ICDPAR|} = 2). Ces tableaux ont \'et\'e renseign\'e une fois pour toutes au
d\'ebut de calcul.

\item  Appel de \fort{rijthe} pour le calcul des termes de gravit\'e $\mathcal{G}^n_{ij}$ et ajout dans \var{SMBR}.

Ce calcul n'a lieu que si $\var{IGRARI()} = 1$.
$ \var{SMBR} = \var{SMBR} + \mathcal{G}^n_{ij}$
\item Calcul de la partie explicite du terme de diffusion
 $\dive{\,\left[\tens{A}\,\grad{R}^{\,n}_{ij}\right]}$, plus pr\'ecis\'ement
des contributions du terme extradiagonal pris aux faces purement internes
(remplissage du tableau \var{VISCF}), puis aux faces de bord (remplissage du
tableau \var{VISCB}).
\begin{itemize}
\item [$\star$] Appel de \fort{grdcel} pour le calcul du gradient de
$R^{\,n}_{ij}$ dans chaque direction. Ces gradients sont respectivement
stock\'es dans les tableaux de travail \var{W1}, \var{W2} et \var{W3}.

\item [$\star$] boucle d'indice \var{IEL} sur les cellules $\Omega_l$ de centre
$L$ pour le
calcul de $\tens{E}^n\,\grad{R}^{\,n}_{ij}$ aux cellules dans un premier temps :\\
\begin{itemize}
\item [$\Rightarrow$] $\displaystyle \var{TRRIJ}= \frac{1}{2} (R^{\,n}_{ii})_L $
\item [$\Rightarrow$] $\displaystyle \var{CSTRIJ} = \rho^n_L\ C_S \ \displaystyle\frac{(R^n_{ii})_L}{2\,\varepsilon^n_L}$
\item [$\Rightarrow$] $\displaystyle \var{W4(IEL)} = \rho^n_L\ C_S\
\displaystyle\frac{(R^n_{ii})_L}{2\,\varepsilon^n_L} \left[\,(R^{\,n}_{12})_L \ \var{W2(IEL)} +
(R^{\,n}_{13})_L \ \var{W3(IEL)}\,\right]$
\item [$\Rightarrow$] $\displaystyle \var{W5(IEL)} = \rho^n_L\ C_S\
\displaystyle\frac{(R^n_{ii})_L}{2\,\varepsilon^n_L} \left[\,(R^{\,n}_{12})_L \ \var{W1(IEL)} +
(R^{\,n}_{23})_L \ \var{W3(IEL)}\,\right]$
\item [$\Rightarrow$] $\displaystyle \var{W6(IEL)} = \rho^n_L\ C_S\
\displaystyle\frac{(R^n_{ii})_L}{2\,\varepsilon^n_L} \left[\,(R^{\,n}_{13})_L \ \var{W1(IEL)} + (R^{\,n}_{23})_L \ \var{W2(IEL)}\,\right]$
\end{itemize}



\item [$\star$] Appel de \fort{vectds}\footnote{Le r\'esultat est stock\'e dans
\var{VISCF} et \var{VISCB}. Dans \fort{vectds}, les valeurs aux faces internes
sont interpol\'ees lin\'eairement sans reconstruction et \var{VISCB} est mis \`a
z\'ero.} pour assembler $\displaystyle\left[ \tens{E}^n\,\grad{R}^{\,n}_{ij}
\right]\,.\,\vect{n}_{\,lm}S_{\,lm}$ aux faces $lm$.
\item [$\star$] Appel de \fort{divmas} pour calculer la divergence du flux d\'efini par \var{VISCF} et \var{VISCB}.
Le r\'esultat est stock\'e dans \var{W4}.\\
Ajout au second membre \var{SMBR}.\\
\var{SMBR} = \var{SMBR} + \var{W4}
\end{itemize}

A l'issue de cette \'etape, seule la partie extradiagonale de la diffusion prise
enti\`erement explicite~:
 $$\sum\limits_{m\in
Vois(l)}\left[\ \tens{E}^n\,\grad{R}^{\,n}_{ij} \right]_{\,lm}\,.\,\vect{n}_{\,lm}S_{\,lm}$$ a \'et\'e calcul\'ee.\\

\item Calcul de la partie diagonale du terme de diffusion\footnote{Seule la
composante normale  du  gradient de $R_{ij}$ aux faces sera implicite.} :\\
Comme on l'a d\'eja vu, une partie de cette contribution sera trait\'ee en
implicite (celle relative \`a la composante normale du gradient) et donc
ajout\'ee au second membre par \fort{bilsc2} ; l'autre
partie sera explicite et prise en compte dans $f_s^{\,exp}$.
\begin{itemize}
\item [$\star$] On effectue une boucle d'indice \var{IEL} sur les cellules
$\Omega_l$ de centre $L$ :
\begin{itemize}
\item [$\Rightarrow$] $\displaystyle \var{TRRIJ }= \frac{1}{2} (R^{\,n}_{ii})_L $
\item [$\Rightarrow$] $\displaystyle \var{CSTRIJ} = \rho^n_L \ C_S \ \frac{(R^{\,n}_{ii})_L}{2\,\varepsilon^n_L}$
\item [$\Rightarrow$] $\displaystyle \var{W4(IEL)} = \rho^n_L \ C_S \
\frac{(R^{\,n}_{ii})_L}{2\,\varepsilon^n_L} \ (R^{\,n}_{11})_L$
\item [$\Rightarrow$] $\displaystyle \var{W5(IEL)} = \rho^n_L \ C_S \ \frac{(R^{\,n}_{ii})_L}{2\,\varepsilon^n_L}\ (R^n_{22})_L$
\item [$\Rightarrow$] $\displaystyle \var{W6(IEL)} = \rho^n_L \ C_S \ \frac{(R^{\,n}_{ii})_L}{2\,\varepsilon^n_L} \ (R^n_{33})_L$
\end{itemize}

%\item Traitement du parall\'elisme et de la p\'eriodicit\'e.

\item [$\star$] On effectue une boucle d'indice \var{IFAC} sur les faces
purement internes $lm$ pour remplir le tableau \var{VISCF} :
\begin{itemize}
\item [$\Rightarrow$] Identification des cellules $\Omega_l$ et $\Omega_m$ de
centre respectif $L$ (variable \var{II}) et $M$ (variable \var{JJ}), se trouvant de chaque c\^ot\'e de la face
$lm$\footnote{La normale \`a la face est orient\'ee de L vers M.}.
\item [$\Rightarrow$] Calcul du carr\'e de la surface de la face. La valeur est
stock\'ee dans le tableau \var{SURFN2}.
\item [$\Rightarrow$] Interpolation du gradient de $R^{\,n}_{ij}$ \`a la face
$lm$ (gradient facette $\left[\grad{R}^{\,n}_{ij}\right]_{\,lm}$) :
\begin{equation}\notag
\left\{\begin{array}{ll}
\var{GRDPX} &= \displaystyle \frac{1}{2} \left(\var{W1(II)} + \var{W1(JJ)}
\right) \\
&\\
\var{GRDPY} &= \displaystyle \frac{1}{2} \left(\var{W2(II)} + \var{W2(JJ)} \right) \\
&\\
\var{GRDPZ} &= \displaystyle \frac{1}{2} \left(\var{W3(II)} + \var{W3(JJ)} \right)
\end{array}\right.
\end{equation}
\item [$\Rightarrow$] Calcul du gradient de $R^{\,n}_{ij}$ normal \`a la face
$lm$, $\left[\grad{R}^{\,n}_{ij}\right]_{\,lm}.\vect{n}_{\,lm}\,S_{\,lm}$.\\

$\displaystyle \var{GRDSN} =  \var{GRDPX} \ \var{SURFAC(1,IFAC)} + \var{GRDPY} \ \var{SURFAC(2,IFAC)} +  \var{GRDPZ} \ \var{SURFAC(3,IFAC)}$
$\var{SURFAC}$ est le vecteur surface de la face \var{IFAC}.


\item [$\Rightarrow$] calcul de
 $\left[\grad{R^{\,n}_{ij}} - (\grad
R^{\,n}_{ij}\,.\,\vect{n}_{\,lm})\vect{n}_{\,lm}\right]$, les vecteurs \'etant
calcul\'es \`a la face $lm$ :
\begin{equation}\notag
\left\{\begin{array}{lll}
&\displaystyle \var{GRDPX} &= \var{GRDPX} - \displaystyle\frac{\var{GRDSN}}{\var{SURFN2}} \ \var{SURFAC(1,IFAC)}\\
&&\\
&\displaystyle \var{GRDPY} &= \var{GRDPY} - \displaystyle\frac{\var{GRDSN}}{\var{SURFN2}} \ \var{SURFAC(2,IFAC)} \\
&&\\
&\displaystyle \var{GRDPZ} &= \var{GRDPZ} - \displaystyle\frac{\var{GRDSN}}{\var{SURFN2}} \ \var{SURFAC(3,IFAC)}
\end{array}\right.
\end{equation}
\item [$\Rightarrow$] finalisation du calcul de l'expression totalement
explicite
 $$\left[ \tens{D}^n\,\left( \grad{R^{\,n}_{ij}} - (\grad R^{\,n}_{ij}\,.\,\vect{n}_{\,lm})\,\vect{n}_{\,lm}\right) \right]\,.\,\vect{n}_{\,lm}$$
\begin{equation}\notag
\begin{array} {ll}
\displaystyle \var{VISCF} = &
 \displaystyle\frac{1}{2} (\ \var{W4(II)} +\ \var{W4(JJ)}) \ \var{GRDPX} \
\var{SURFAC(1,IFAC)})\ + \\
&\\
&  \displaystyle\frac{1}{2} (\ \var{W5(II)} +\ \var{W5(JJ)}) \ \var{GRDPY} \
\var{SURFAC(2,IFAC)})\ + \\
&\\
&  \displaystyle\frac{1}{2} (\ \var{W6(II)} +\ \var{W6(JJ)}) \ \var{GRDPZ} \ \var{SURFAC(3,IFAC)})
\end{array}
\end{equation}
\end{itemize}

\item [$\star$] Mise \`a z\'ero du tableau \var{VISCB}.

\item [$\star$] Appel de \fort{divmas} pour calculer la divergence de~:
 $$\tens{D}^{\,n}\,\left( \grad{R^{\,n}_{ij}} - (\grad R^{\,n}_{ij}\,.\,\vect{n}_{\,lm})\vect{n}_{\,lm}\right)$$ d\'efini aux faces dans \var{VISCF} et \var{VISCB}.

Le r\'esultat est stock\'e dans le tableau \var{W1}.\\
Ajout au second membre \var{SMBR}.\\
$\var{SMBR} = \var{SMBR} + \var{W1}$
\end{itemize}
\item Calcul de la viscosit\'e orthotrope $\gamma^n_{\,lm}$ \`a la face $lm$ de la variable principale
$R^{\,n}_{ij}$\\
Ce calcul permet au sous-programme \fort{codits} de compl\'eter le second membre
\var{SMBR} par :
\begin{equation}
\begin{array} {ll}
& \sum\limits_{m\in Vois(l)}
\mu^n_{\,lm}\,\left(\grad{R}^{\,n}_{ij}\,.\,\vect{n}_{\,lm}\right)S_{\,lm}
 + \sum\limits_{m\in Vois(l)} \left(\grad{R}^{\,n}_{ij}
\,.\,\vect{n}_{\,lm}\right)\left[\tens{D}^{\,n}\,\vect{n}_{\,lm}\right]_{\,lm}\,.\,\vect{n}_{\,lm}\
S_{\,lm}\\
& = \sum\limits_{m\in Vois(l)}(\,\mu^n_{\,lm}\, + \,\gamma^n_{\,lm}\,)\,\left(\grad{R}^{\,n}_{ij}\,.\,\vect{n}_{\,lm}\right)S_{\,lm}
\end{array}
\end{equation}
sans pr\'eciser la nature de la face $lm$, {\it via} l'appel \`a \fort{bilsc2}  et de disposer de la quantit\'e
$(\mu^n_{\,lm}\, + \gamma^n_{\,lm})$ pour construire sa
matrice simplifi\'ee.\\
\begin{itemize}
\item [$\star$] On effectue une boucle d'indice \var{IEL} sur les cellules
$\Omega_l$ :
\begin{itemize}
\item [$\Rightarrow$] $\displaystyle \var{TRRIJ }= \frac{1}{2} (R^{\,n}_{ii})_L $
\item [$\Rightarrow$] $\displaystyle \var{RCSTE} = \rho^n_L \ C_S \ \frac{ (R^{\,n}_{ii})_L}{2\,\varepsilon^n_L} $
\item [$\Rightarrow$] $\displaystyle \var{W1(IEL)} = \mu^n +\rho^n_L \ C_S \ \frac{
(R^{\,n}_{ii})_L}{2\,\varepsilon^n_L}\ (R^n_{11})_L$
\item [$\Rightarrow$] $\displaystyle \var{W2(IEL)} = \mu^n + \rho^n_L \ C_S \ \frac{ (R^{\,n}_{ii})_L}{2\,\varepsilon^n_L}\ (R^n_{22})_L$
\item [$\Rightarrow$] $\displaystyle \var{W3(IEL)} = \mu^n + \rho^n_L \ C_S \ \frac{ (R^{\,n}_{ii})_L}{2\,\varepsilon^n_L}\ (R^n_{33})_L$
\end{itemize}

\item [$\star$] Appel de \fort{visort} pour calculer la viscosit\'e orthotrope
\footnote{Comme dans le sous-programme \fort{viscfa}, on multiplie la viscosit\'e par
$\displaystyle \frac{S_{\,lm}}{\overline{L'M'}}$, o\`u $S_{\,lm}$ et
$\overline{L'M'}$ repr\'esentent respectivement la surface de la face $lm$ et la
mesure alg\'ebrique du segment reliant les projections des centres des cellules
voisines sur la normale \`a la face. On garde dans ce sous-programme  la possibilit\'e d'interpoler la viscosit\'e aux cellules lin\'eairement ou d'utiliser une moyenne harmonique. La viscosit\'e au bord est celle de la cellule de bord correspondante.}
$ \gamma^n_{\,lm} = (\tens{D}^{\,n}\,\vect{n}_{\,lm}).\vect{n}_{\,lm}$ aux faces $lm$

Le r\'esultat est stock\'e dans les tableaux \var{VISCF} et \var{VISCB}.
\end{itemize}

\item appel de \fort{codits} pour la r\'esolution de l'\'equation de
convection/diffusion/termes sources de la variable $R_{ij}$. Le terme source est
\var{SMBR}, la viscosit\'e \var{VISCF} aux faces purement internes (
resp. \var{VISCB} aux faces de bord) et \var{FLUMAS} le flux de masse interne
 ( resp. \var{FLUMAB} flux de masse au bord). Le r\'esultat est la variable $R_{ij}$ au temps
$n+1$, donc $R^{\,n+1}_{ij}$. Elle est stock\'ee dans le tableau \var{RTP} des
variables mises \`a jour.

\end{itemize}

\etape{Appel de \fort{reseps} pour la r\'esolution de la variable $\varepsilon$}

On d\'ecrit ci-dessous le sous-programme \fort{reseps}, les commentaires du sous-programme \fort{resrij} ne seront pas r\'ep\'et\'es puisque les deux sous-programmes ne diff\`erent que par leurs termes sources.

\begin{itemize}
\item Initialisation \`a z\'ero de \var{SMBR} et \var{ROVSDT}.

\item{Lecture et prise en compte des termes sources utilisateur pour la variable $\varepsilon$ :}

Appel de \fort{ustsri} pour charger les termes sources utilisateurs. Ils sont
stock\'es dans les tableaux suivants :\\
pour la cellule $\Omega_l$ repr\'esent\'ee par $\var{IEL}$ de centre $L$, on a :
\begin{equation}\notag
\left\{\begin{array}{lll}
&\var{ROVSDT(IEL)}&= |\Omega_l| \ \alpha_{\varepsilon}\\
&\var{SMBR(IEL)}&=|\Omega_l| \ \beta_{\varepsilon}\\
\end{array}\right.
\end{equation}
On affecte alors les valeurs ad\'equates au second membre \var{SMBR} et \`a la
diagonale \var{ROVSDT} comme suit :
\begin{equation}\notag
\left\{\begin{array}{lll}
&\var{SMBR(IEL)} &= \var{SMBR(IEL)} +\ |\Omega_l| \ \alpha_{\,\varepsilon} \
\varepsilon^n_L \\
&\var{ROVSDT(IEL)}&= \text{max }(-\ |\Omega_l| \ \alpha_{\,\varepsilon},0)\\
\end{array}\right.
\end{equation}

\item{Calcul du terme source de masse si $\Gamma_L > 0$ :
\begin{equation}\notag
\left\{\begin{array}{lll}
&\displaystyle \var{SMBR(IEL)} = \var{SMBR(IEL)} + |\Omega_l| \ \Gamma_L \
(\varepsilon^{\,in}_L -\varepsilon^n_L) \\
&\displaystyle \var{ROVSDT(IEL)}= \var{ROVSDT(IEL)} + |\Omega_l| \ \Gamma_L
\end{array}\right.
\end{equation}
\item Calcul du terme d'accumulation de masse et du terme instationnaire \\
On stocke $\displaystyle \var{W1}= \int_{\Omega_l}\dive\,(\rho\,\vect{u})\,d\Omega$
calcul\'e par \fort{divmas} \`a l'aide des flux de masse internes et aux bords.\\
On incr\'emente ensuite \var{SMBR} et \var{ROVSDT}.
\begin{equation}\notag
\left\{\begin{array}{lll}
&\var{SMBR(IEL)} &= \var{SMBR(IEL)} + \var{ICONV}\ \varepsilon^n_L\,(\displaystyle
\int_{\Omega_l}\dive\,(\rho\,\vect{u})\ d\Omega) \\
&\var{ROVSDT(IEL)}& = \var{ROVSDT(IEL)} +  \var{ISTAT}\,\displaystyle
\frac{\rho^n_L \ |\Omega_l|}{\Delta t^n} -  \var{ICONV}\ (\displaystyle
\int_{\Omega_l}\dive\,(\rho\,\vect{u})\ d\Omega) \\
\end{array}\right.
\end{equation}

\item Traitement du terme de production
 $\displaystyle \rho\,C_{\varepsilon_1}\,\frac{\varepsilon}{k}\,\mathcal{P}$
 et du terme de dissipation $-\,\displaystyle \rho\,C_{\varepsilon_2}\,\frac{\varepsilon}{k}\,\varepsilon$ \\
pour cela, on effectue une boucle d'indice \var{IEL} sur les cellules $\Omega_l$
de centre $L$ :
\begin{itemize}
\item [$\Rightarrow$] $\displaystyle \var{TRPROD}= \frac{1}{2} (\mathcal{P}^n_{ii})_L = \frac{1}{2} \left[ \var{PRODUC(1,IEL)} +  \var{PRODUC(2,IEL)} +  \var{PRODUC(3,IEL)} \right] $
\item [$\Rightarrow$] $\displaystyle \var{TRRIJ }= \frac{1}{2} (R^n_{ii})_L $
\item [$\Rightarrow$] $\displaystyle \var{SMBR(IEL)} = \var{SMBR(IEL)} + \rho^n_L
|\Omega_l| \left[ -C_{\varepsilon_2} \ \frac{2\,(\varepsilon^n_L)^2}{(R^n_{ii})_L} + C_{\varepsilon_1} \ \frac{\varepsilon^n_L}{(R^n_{ii})_L}\ (\mathcal{P}^n_{ii})_L \right] $
\item [$\Rightarrow$] $\displaystyle \var{ROVSDT(IEL)} = \var{ROVSDT(IEL)} + C_{\varepsilon_2} \ \rho^n_L \ |\Omega_l| \ \frac{2\,\varepsilon^n_L}{(R^n_{ii})_L}$
\end{itemize}

\item Appel de \fort{rijthe} pour le calcul des termes de gravit\'e $\mathcal{G}^n_{\varepsilon}$ et ajout dans \var{SMBR}.

$ \var{SMBR} = \var{SMBR} + \mathcal{G}^n_{\varepsilon}$\\
Ce calcul n'a lieu que si $\var{IGRARI()} = 1$.

\item Calcul de la diffusion de $\varepsilon$ \\
 Le terme $\dive \left[\mu\, \grad(\varepsilon) + \tens{A'}\,\grad(\varepsilon)
\right]$ est calcul\'e exactement de la m\^eme mani\`ere que pour les tensions
de Reynolds $R_{ij}$ en rempla\c cant $\tens{A}$ par $\tens{A'}$.

\item Appel de \fort{codits} pour la r\'esolution de l'\'equation de
convection/diffusion/termes sources de la variable principale $\varepsilon$. Le
r\'esultat $\varepsilon^{\,n+1}$ est stock\'e dans le tableau \var{RTP} des
variables mises \`a jour.
}
\end{itemize}

\etape{clippings finaux}
On passe enfin dans le sous-programme  \fort{clprij} pour faire un clipping \'eventuel
des variables $R^{\,n+1}_{ij}$ et $\varepsilon^{\,n+1}$. Le sous-programme
\fort{clprij} est appel\'e\footnote{L'option
$\var{ICLIP} = 1$ consiste en un clipping minimal des variables $R_{ii}$ et
$\varepsilon$ en prenant la valeur absolue de ces variables puisqu'elles ne
peuvent \^etre que positives.} avec $\var{ICLIP} = 2$ . Cette option
\footnote{Quand la valeur des grandeurs $R_{ii}$ ou $\varepsilon$ est
n\'egative, on la remplace par le minimum entre sa valeur absolue et (1,1)
fois la valeur obtenue au pas de temps pr\'ec\'edent.} contient l'option $\var{ICLIP} = 1$  et permet de v\'erifier l'in\'egalit\'e de Cauchy-Schwarz sur les grandeurs extra-diagonales du tenseur $\tens{R}$ (pour $i \neq j$, $|R_{ij}|^2 \le R_{ii} R_{jj}$).


%%%%%%%%%%%%%%%%%%%%%%%%%%%%%%%%%%
%%%%%%%%%%%%%%%%%%%%%%%%%%%%%%%%%%
\section{Points \`a traiter}
%%%%%%%%%%%%%%%%%%%%%%%%%%%%%%%%%%
%%%%%%%%%%%%%%%%%%%%%%%%%%%%%%%%%%
Sauf mention explicite, $\phi$ repr\'esentera une tension de Reynolds ou la dissipation turbulente ($\phi = R_{ij} \ \text{ou} \ \varepsilon$).

\begin{itemize}
\item {La vitesse utilis\'ee pour le calcul de la production est explicite. Est-ce qu'une implicitation peut am\'eliorer la pr\'ecision temporelle de $\phi$ \footnote{Cette remarque peut \^etre g\'en\'eralis\'ee. En effet, peut-on envisager d'actualiser les variables d\'ej\`a r\'esolues (sans r\'eactualiser les variables turbulentes apr\`es leur r\'esolution)? Ceci obligerait \`a modifier les sous-programmes tels que \fort{condli} qui sont appel\'es au d\'ebut de la boucle en temps.} ?}
\item {Dans quelle mesure le terme d'\'echo de paroi est-il valide ? En effet, ce terme est remis en question par certains auteurs.}
\item {On peut envisager la r\'esolution d'un syst\`eme hyperbolique pour les
tensions de Reynolds afin d'introduire un couplage avec le champ de vitesse.}
\item {Le flux au bord \var{VISCB} est annul\'e dans le sous-programme
\fort{vectds}. Peut-on envisager de mettre au bord la valeur de la variable
concern\'ee \`a la cellule de bord correspondant? De m\^eme, il faudrait se
pencher sur les hypoth\`eses sous-jacentes \`a l'annulation des contributions
aux bords de \var{VISCB} lors du calcul de : $$\left[ \tens{D}^n\,\left( \grad{R^{\,n}_{ij}} - (\grad R^{\,n}_{ij}\,.\,\vect{n}_{\,lm})\,\vect{n}_{\,lm}\right) \right]\,.\,\vect{n}_{\,lm}.$$}
\item {Un probl\`eme de pond\'eration appara\^\i t plus g\'en\'eralement. Si on prend la partie explicite de $\tens{D}\,\grad(\phi)$, la pond\'eration aux faces internes utilise le coefficient $\displaystyle\frac{1}{2}$ avec pond\'eration s\'epar\'ee de $\tens{D}$ et $\grad(\phi)$, alors que pour $\tens{E}\,\grad(\phi)$, on calcule d'abord ce terme aux cellules pour ensuite l'interpoler lin\'eairement aux faces \footnote{Cette interpolation se fait dans \fort{vectds} avec des coefficients de pond\'eration aux faces.}. Ceci donne donc deux types d'interpolations pour des termes de m\^eme nature.}
\item {On laisse la possibilit\'e dans \fort{visort} d'utiliser une moyenne
harmonique aux faces. Est-ce que ceci est valable puisque les interpolations
utilis\'ees lors du calcul de la partie explicite de $\tens{A}\,\grad{\phi}$
sont des moyennes arithm\'etiques ?}
\item {Les techniques adopt\'ees lors du clipping sont \`a revoir.}
\item {On utilise dans le cadre du mod\`ele $\displaystyle R_{ij}-\varepsilon $ une semi-implicitation de termes comme $\displaystyle \phi_{ij,1}$ ou $\displaystyle -\rho\,C_{\varepsilon_2}\,\frac{\varepsilon}{k}\,\varepsilon$. On peut envisager le m\^eme type d'implicitation dans \fort{turbke} m\^eme en pr\'esence du couplage $\displaystyle k-\varepsilon$.}
\item L'adoption d'une r\'esolution d\'ecoupl\'ee fait perdre l'invariance par rotation.
\item La formulation et l'implantation des conditions aux limites de paroi
devront \^etre v\'erifi\'ees. En effet, il semblerait que, dans certains cas, des ph\'enom\`enes
``oscillatoires'' apparaissent, sans qu'il soit ais\'e d'en d\'eterminer la cause.
\item L'implicitation partielle (du fait de la r\'esolution d\'ecoupl\'ee) des
conditions aux limites conduit souvent \`a des calculs instables. Il
conviendrait d'en conna\^\i tre la raison. L'implicitation partielle avait
\'et\'e mise en \oe uvre afin de tenter d'utiliser un pas de temps plus grand
dans le cas de jets axisym\'etriques en particulier.

\end{itemize}

%                      Code_Saturne version 1.3
%                      ------------------------
%
%     This file is part of the Code_Saturne Kernel, element of the
%     Code_Saturne CFD tool.
%
%     Copyright (C) 1998-2007 EDF S.A., France
%
%     contact: saturne-support@edf.fr
%
%     The Code_Saturne Kernel is free software; you can redistribute it
%     and/or modify it under the terms of the GNU General Public License
%     as published by the Free Software Foundation; either version 2 of
%     the License, or (at your option) any later version.
%
%     The Code_Saturne Kernel is distributed in the hope that it will be
%     useful, but WITHOUT ANY WARRANTY; without even the implied warranty
%     of MERCHANTABILITY or FITNESS FOR A PARTICULAR PURPOSE.  See the
%     GNU General Public License for more details.
%
%     You should have received a copy of the GNU General Public License
%     along with the Code_Saturne Kernel; if not, write to the
%     Free Software Foundation, Inc.,
%     51 Franklin St, Fifth Floor,
%     Boston, MA  02110-1301  USA
%
%-----------------------------------------------------------------------
%
\programme{vortex}
%
\vspace{1cm}
%%%%%%%%%%%%%%%%%%%%%%%%%%%%%%%%%%
%%%%%%%%%%%%%%%%%%%%%%%%%%%%%%%%%%
\section{Fonction}
%%%%%%%%%%%%%%%%%%%%%%%%%%%%%%%%%%
%%%%%%%%%%%%%%%%%%%%%%%%%%%%%%%%%%
Ce sous-programme est d�di� � la g�n�ration des conditions d'entr�e
turbulente utilis�es en LES.


La m�thode des vortex est bas�e sur une approche de tourbillons
ponctuels. L'id�e de la m�thode consiste � injecter des tourbillons 2D dans le
plan d'entr�e du calcul, puis � calculer le champ de vitesse induit par ces
tourbillons au centre des faces d'entr�e.

%                      Code_Saturne version 1.3
%                      ------------------------
%
%     This file is part of the Code_Saturne Kernel, element of the
%     Code_Saturne CFD tool.
% 
%     Copyright (C) 1998-2007 EDF S.A., France
%
%     contact: saturne-support@edf.fr
% 
%     The Code_Saturne Kernel is free software; you can redistribute it
%     and/or modify it under the terms of the GNU General Public License
%     as published by the Free Software Foundation; either version 2 of
%     the License, or (at your option) any later version.
% 
%     The Code_Saturne Kernel is distributed in the hope that it will be
%     useful, but WITHOUT ANY WARRANTY; without even the implied warranty
%     of MERCHANTABILITY or FITNESS FOR A PARTICULAR PURPOSE.  See the
%     GNU General Public License for more details.
% 
%     You should have received a copy of the GNU General Public License
%     along with the Code_Saturne Kernel; if not, write to the
%     Free Software Foundation, Inc.,
%     51 Franklin St, Fifth Floor,
%     Boston, MA  02110-1301  USA
%
%-----------------------------------------------------------------------
%
%%%%%%%%%%%%%%%%%%%%%%%%%%%%%%%%%%
%%%%%%%%%%%%%%%%%%%%%%%%%%%%%%%%%%
\section{Discr\'etisation}
%%%%%%%%%%%%%%%%%%%%%%%%%%%%%%%%%%
%%%%%%%%%%%%%%%%%%%%%%%%%%%%%%%%%%

Le terme convectif en $\dive(\underline{u} \otimes \rho\,\underline{u})$
introduit une non lin\'earit\'e et un couplage des composantes de la vitesse
$\vect{u}$ dans l'�quation (\ref{Base_Preduv_eqqdm}). Une lin\'earisation et un d\'ecouplage
des trois composantes de la 
vitesse sont r\'ealis\'es lors de la discr\'etisation de cette \'etape de
pr\'ediction.\\
En effet, soit :
\begin{equation}
\vect{\widetilde{u}}= \vect{u}^n + \delta \vect{u} 
\end{equation}
La contribution exacte du terme convectif \`a prendre en compte dans cette
\'etape de pr\'ediction serait :\\
\begin{equation}\label{Base_Preduv_Conv_exact}
\begin{array}{ll}
\dive(\vect{\widetilde{u}} \otimes \rho\,\vect{\widetilde{u}}) =
\dive(\vect{u}^{n} \otimes \rho\,\vect{u}^{n}) + \dive(\delta \vect{u} \otimes
\rho\,\vect{u}^{n}) +  \underbrace { \dive(\vect{u}^{n} \otimes
\rho\,\delta \vect{u})}_{\text {terme couplant lin\'eaire}} +  \underbrace { \dive(\delta \vect{u} \otimes
\rho\,\delta \vect{u})}_{\text {terme couplant et non lin\'eaire}}\\
\end{array} 
\end{equation}
Les deux derniers termes de l'expression (\ref{Base_Preduv_Conv_exact}) sont {\em a priori} n�glig�s
de mani�re � obtenir un syst\`eme en vitesse qui soit d\'ecoupl\'e et donc,
�viter l'inversion d'une matrice pouvant \^etre de tr\`es grande taille. Ces
deux termes peuvent n�anmoins �tre pris en compte de mani�re plus ou moins
approch�e par extrapolation explicite du flux de masse en $n+\theta_F$ (pour le
terme couplant lin�aire provenant de la convection de $\vect{u}^{n}$ par $\delta
\vect{u}$) et utilisation d'un point-fixe par sous it�ration sur le sous
programme \fort{navsto} (pour le terme non-lin�aire, en sp�cifiant $\var{NTERUP}>1$).

L'�quation (\ref{Base_Preduv_eqqdm}) est discr�tis�e au temps $n+\theta$ � l'aide d'un
$\theta$-sch�ma, et le tenseur des pertes de charges d�compos� en une partie
diagonale $\tens{K}_{d}$ et une extradiagonale $\tens{K}_{e}$ (soit
 $\tens{K}_{pdc}=\tens{K}_{d}+\tens{K}_{e}$).\\
$\bullet$ La pression est suppos�e connue en $n-1+\theta$ (d�calage temporel
pression-vitesse) et le gradient naturellement calcul� � cet instant.\\ 
$\bullet$ Les termes sources de viscosit� secondaire, de gradient transpos\'e,
ceux provenant du mod�le de turbulence\footnote{except� $\dive (\mu_t\ (\ggrad
\underline {u}))$}, $\rho\,\tens{K}_{\,e}\ \underline{u}$, $(\rho -\rho_0)
\underline {g}$ ainsi que $\underline{T}_{s}^{\,exp}$ et
$\Gamma\,\underline{u}_{\,i}$ sont pris de mani�re explicite au temps $n$, ou
extrapol�s suivant le sch�ma en temps choisi pour les propri�t�s physique et les
termes sources.\\ 
$\bullet$ Les termes sources $\underline{u}\,\,\dive (\rho\,\underline {u})$,
$\Gamma\,\,\underline{u}$, $T_{s}^{\,imp}\,\,\underline{u}$ et
$-\rho\,\tens{K}_{\,d}\,\,\underline{u}$ sont implicit�s est calcul�s �
l'instant $n+\theta$.\\ 
$\bullet$ Le terme de diffusion $\dive (\mu_{\,tot}\,\ggrad \underline{u})$ est
implicit� : la vitesse est prise � l'instant $n+\theta$ et la viscosit�
explicit�e ou extrapol�e.\\ 
$\bullet$ Enfin, le terme de convection en $\dive(\,\underline{u} \otimes
(\rho\underline{u})\,)$ est implicit� : la composante r�solue de la vitesse est
prise en $n+\theta$, et le flux de masse, explicit�, ou extrapol� en
$n+\theta_F$. 

Par souci de clart�, on suppose, en l'absence d'indication, que les propri�tes
physiques $\Phi$ ($\rho,\,\mu_{tot},\,...$) et le flux de masse
$(\rho\underline{u})$ sont pris respectivement aux instants $n+\theta_\Phi$ et
$n+\theta_F$, o� $\theta_\Phi$ et $\theta_F$ d�pendent des sch�mas en temps
sp�cifiquement utilis�s pour ces grandeurs\footnote{cf. \fort{introd}}. 

La discr�tisation temporelle de l'�quation (\ref{Base_Preduv_eqqdm}) s'�crit alors comme suit : 

\begin{equation}\label{Base_Preduv_eq_di1}
 \begin{array}{c}
\displaystyle \rho\,\ \frac{ \underline {\widetilde{u}}^{n+1} -\underline {u}^{n} }
{\Delta t} + \dive(\,\underline{\widetilde{u}}^{n+\theta} \otimes (\rho\underline{u})\,) -\dive
(\mu_{\,tot}\,\ggrad \underline{\widetilde{u}}^{n+\theta}) =
\\
\displaystyle
 - \grad p^{n-1+\theta} + \dive (\rho\,\underline {u})\,\underline{\widetilde{u}}^{n+\theta} +(\Gamma\,\underline{u}_{\,i})^{n+\theta_S}-\Gamma^n\,\,\underline{\widetilde{u}}^{n+\theta}
\\
\begin{array}{c}
\displaystyle
- \rho\,\tens{K}_{\,d}^{n}\,\,\underline{\widetilde{u}}^{n+\theta} - (\rho\,\tens{K}_{\,e}\ \underline{u})^{n+\theta_S} + (\underline{T}_{s}^{\,exp})^{\,n+\theta_S} + T_{s}^{\,imp}\,\,\underline{\widetilde{u}}^{n+\theta}
\\
\displaystyle
+[\dive (\mu_{\,tot}\,^t\ggrad \underline {u})]^{n+\theta_S}-\frac {2} {3}[\,\grad (\mu_{\,tot}\,\dive \underline {u})]^{n+\theta_S} + (\rho -\rho_0) \underline {g}
 - (\underline{turb})^{n+\theta_S}
\end{array}
\end{array}
\end{equation}
o\`u, par souci de simplification, on a pos\'e :
\begin{equation}
\mu_{\,tot}=
\begin{cases}
\mu+\mu_t & \text{pour les mod�les � viscosit� turbulente ou en LES}, \\
\mu & \text{pour les mod�les au second ordre ou en laminaire}
\end{cases} \ 
\end{equation}
\\
et :
\begin{equation}
\underline{turb}^{n}=
\begin{cases}
\displaystyle\frac {2}{3}\grad (\rho^{n}\,k^{n}) & \text{pour les mod�les � viscosit� turbulente}, \\
\dive(\rho^{n}\,\tens{R}^n) & \text{pour les mod�les au second ordre},\\
0 & \text{en laminaire ou en LES}\\
\end{cases}
\end{equation}
Par analogie avec l'�criture du $\theta$-sch�ma pour une variable scalaire, $\,
\underline {\widetilde{u}}^{n+\theta}$ est interpol�e � partir de la vitesse
pr�dite $\underline {\widetilde{u}}^{n+1}$ de la mani\`ere suivante\footnote{si
$\theta=1/2$, ou qu'une extrapolation est utilis�e, l'ordre 2 n'est obtenu que si
le pas de temps $\Delta t$ est uniforme en temps et en espace.}~: 
\begin{equation}
\underline {\widetilde{u}}^{n+\theta}=\theta\, \underline
{\widetilde{u}}^{n+1}+(1-\theta)\, \underline {u}^{n}\\ 
\end{equation}
Avec :
\begin{equation}
\left\{
\begin{array}{ll}
\theta = 1   & \text{Pour un sch\'ema de type Euler implicite d'ordre 1.}\\
\theta = 1/2 & \text{Pour un sch\'ema de type Cranck-Nicolson d'ordre 2.}\\
\end{array}
\right.
\end{equation}

L'�quation (\ref{Base_Preduv_eq_di1}) est alors r��crite sous la forme :

\begin{equation}\label{Base_Preduv_eq_di2}
\begin{array}{c}
\displaystyle \underbrace{\left(\frac{\rho}{\Delta t} -\theta \,\dive (\rho\,\underline {u}) +\theta \,\, \Gamma^n +
\theta \,\, \rho\,\tens{K}_{\,d}^n-\theta \,T_s^{\,imp} \right)}_{\displaystyle f_s^{imp}}\, (\underline {\,\widetilde{u}}^{n+1} -\underline {u}^{n})
\\
 +\, \theta\, \dive(\underline {\widetilde{u}}^{n+1} \otimes (\rho\underline{u}))-\, \theta\,\dive (\mu_{\,tot}\,\ggrad \underline {\widetilde{u}}^{n+1}) =
\\
-\,(1-\theta)\, \dive(\underline {u}^{n} \otimes (\rho\underline{u})) +\,(1-\theta)\,\dive (\mu_{\,tot}\,\ggrad \underline {u}^{n})
\\
f_s^{exp}\left\{
\begin{array}{c}
\displaystyle 
- \grad p^{n-1+\theta} + \dive (\rho\,\underline {u})\,\underline{u}^{n} +\,(\,\Gamma^{n}\,\underline{u}_{\,i}\,)^{n+\theta_S}- \Gamma^n\,\,\underline{u}^{n}
\\
\displaystyle
-(\,\rho\,\tens{K}_{\,e}\ \underline{u}\,)^{n+\theta_S} -\rho\,\tens{K}_{\,d}^n\ \underline{u}^{n}+ (\underline{T}_{s}^{\,exp})^{\,n+\theta_S} + T_s^{\,imp}\,\,\underline {u}^{n} 
\\
\displaystyle
+[\dive (\mu_{\,tot}\,^t\ggrad \underline {u}\,)]^{n+\theta_S}-\frac {2} {3}[\,\grad (\mu_{\,tot}\,\dive \underline {u}\,)]^{n+\theta_S} + (\rho -\rho_0) \underline {g}-(\underline{turb})^{n+\theta_S}
\end{array}
\right.
\end{array}
\end{equation}

d'o� l'�quation r�solue par le sous-programme \fort{codits} :
\begin{equation}\begin{array}{c}
\displaystyle
f_s^{\,imp}(\underline {\widetilde{u}}^{n+1}-\underline {u}^{n}) + \theta\, \dive(\underline{\widetilde{u}}^{n+1} \otimes (\rho
\underline{u})) - \theta\,\dive (\,\mu_{\,tot}\,\ggrad \underline{\widetilde{u}}^{n+1}) = 
\\\\
\displaystyle
-(1-\theta)\,\dive(\underline{u}^{n} \otimes (\rho \underline{u}))+(1-\theta)\,\dive (\,\mu_{\,tot}\,\ggrad \underline{u}^{n})
+ \underline{f}_{\,s}^{\,exp}
\end{array}
\end{equation}
La m\'ethode de discr\'etisation spatiale est d\'evelopp\'ee dans le sous-programme \fort{codits}.\\



\minititre{Remarques :}
{\tiny$\blacksquare$} Dans le cas standard sans extrapolation, le terme
$-\,T_s^{\,imp}$ n'est ajout� � $f_s^{\,imp}$ que s'il est positif afin de ne
pas affaiblir la dominance de la diagonale de la matrice � inverser.\\ 
{\tiny$\blacksquare$} Si une extrapolation est utilis�e, par contre,
$\,T_s^{\,imp}$ est ajout� � $f_s^{\,imp}$ quel que soit son signe. En effet, l'id�e intuitive qui
consiste � prendre~: 
\begin{equation}
\begin{cases}
\displaystyle
(\underline{T}_{s}^{\,exp} + T_{s}^{\,imp}\,\underline {u})^{\,n+\theta_S} &
\text{si } T_{s}^{\,imp} > 0\\ 
\displaystyle
(\underline{T}_{s}^{\,exp})^{\,n+\theta_S} + T_{s}^{\,imp}\,\underline{u}^{n+\theta} &\text{sinon}\\
\end{cases}
\end{equation} 
aboutit � une incoh�rence dans le traitement si $T_s^{imp}$ change de signe
entre deux pas de temps.\\ 
{\tiny$\blacksquare$} la partie diagonale $\tens{K}_{\,d}$ du terme
de perte de charge est utilis�e dans $f_s^{\,imp}$. En fait, pour \^etre rigoureux,
il faudrait ne retenir que les contributions positives (point signal\'e dans le
sous-programme utilisateur associ\'e \fort{uskpdc}). Cette prise en compte sera \`a am\'eliorer.\\
{\tiny$\blacksquare$} Le terme $\theta\,\Gamma^{n}-\theta\,\dive
(\rho\,\underline {u})$ ne pose pas de probl�me pour la 
dominance de la diagonale de la matrice car il est exactement compens� par le
terme de convection (cf. \fort{covofi}). 


%                      Code_Saturne version 1.3
%                      ------------------------
%
%     This file is part of the Code_Saturne Kernel, element of the
%     Code_Saturne CFD tool.
%
%     Copyright (C) 1998-2007 EDF S.A., France
%
%     contact: saturne-support@edf.fr
%
%     The Code_Saturne Kernel is free software; you can redistribute it
%     and/or modify it under the terms of the GNU General Public License
%     as published by the Free Software Foundation; either version 2 of
%     the License, or (at your option) any later version.
%
%     The Code_Saturne Kernel is distributed in the hope that it will be
%     useful, but WITHOUT ANY WARRANTY; without even the implied warranty
%     of MERCHANTABILITY or FITNESS FOR A PARTICULAR PURPOSE.  See the
%     GNU General Public License for more details.
%
%     You should have received a copy of the GNU General Public License
%     along with the Code_Saturne Kernel; if not, write to the
%     Free Software Foundation, Inc.,
%     51 Franklin St, Fifth Floor,
%     Boston, MA  02110-1301  USA
%
%-----------------------------------------------------------------------
%

%%%%%%%%%%%%%%%%%%%%%%%%%%%%%%%%%%
%%%%%%%%%%%%%%%%%%%%%%%%%%%%%%%%%%
\section{Mise en \oe uvre}
%%%%%%%%%%%%%%%%%%%%%%%%%%%%%%%%%%
%%%%%%%%%%%%%%%%%%%%%%%%%%%%%%%%%%
La num\'ero de la phase trait\'ee fait partie des arguments de \fort{turrij}. On
omettra volontairement de le pr\'eciser dans ce qui suit, on indiquera par $(\ )$ la
notion de tableau s'y rattachant.

\etape{Calcul des termes de production $\tens{\mathcal{P}}$}
\begin{itemize}
\item [$\star$] Initialisation \`a z\'ero du tableau \var{PRODUC} dimensionn\'e \`a $\var{NCEL}\times 6$.
\item [$\star$] On appelle trois fois \fort{grdcel} pour calculer les gradients des composantes de la vitesse $u$, $v$ et
$w$ prises au temps $n$.

Au final, on a :\\
$\displaystyle
\begin{array} {ll}
\var{PRODUC(1,IEL)} = & \displaystyle - 2 \left[ R_{11}^{\,n} \frac{\partial u^{\,n}} {\partial x} +R_{12}^{\,n} \frac{\partial u^{\,n}} {\partial y}+R_{13}^{\,n} \frac{\partial u^{\,n}} {\partial z} \right] \text{        (production de $R_{11}^{\,n}$)}\\
\var{PRODUC(2,IEL)} = & \displaystyle - 2 \left[ R_{12}^{\,n} \frac{\partial v^{\,n}} {\partial x} +R_{22}^{\,n} \frac{\partial v^{\,n}} {\partial y}+R_{23}^{\,n} \frac{\partial v^{\,n}} {\partial z} \right] \text{        (production de $R_{22}^{\,n}$)}\\
\var{PRODUC(3,IEL)} = & \displaystyle - 2 \left[ R_{13}^{\,n} \frac{\partial w^{\,n}} {\partial x} +R_{23}^{\,n} \frac{\partial w^{\,n}} {\partial y}+R_{33}^{\,n} \frac{\partial w^{\,n}} {\partial z} \right] \text{        (production de $R_{33}^{\,n}$)}\\
\var{PRODUC(4,IEL)} = & \displaystyle - \left[ R_{12}^{\,n} \frac{\partial u^{\,n}} {\partial x} +R_{22}^{\,n} \frac{\partial u^{\,n}} {\partial y}+R_{23}^{\,n} \frac{\partial u^{\,n}} {\partial z} \right] \\
& \displaystyle - \left[ R_{11}^{\,n} \frac{\partial v^{\,n}} {\partial x} +R_{12}^{\,n} \frac{\partial v^{\,n}} {\partial y}+R_{13}^{\,n} \frac{\partial v^{\,n}} {\partial z} \right] \text{        (production de $R_{12}^{\,n}$)} \\
\var{PRODUC(5,IEL)} = & \displaystyle - \left[ R_{13}^{\,n} \frac{\partial u^{\,n}} {\partial x} +R_{23}^{\,n} \frac{\partial u^{\,n}} {\partial y}+R_{33}^{\,n} \frac{\partial u^{\,n}} {\partial z} \right] \\
& \displaystyle - \left[ R_{11}^{\,n} \frac{\partial w^{\,n}} {\partial x} +R_{12}^{\,n} \frac{\partial w^{\,n}} {\partial y}+R_{13}^{\,n} \frac{\partial w^{\,n}} {\partial z} \right] \text{        (production de $R_{13}^{\,n}$)} \\
\var{PRODUC(6,IEL)} = & \displaystyle - \left[ R_{13}^{\,n} \frac{\partial v^{\,n}} {\partial x} +R_{23}^{\,n} \frac{\partial v^{\,n}} {\partial y}+R_{33}^{\,n} \frac{\partial v^{\,n}} {\partial z} \right] \\
& \displaystyle - \left[ R_{12}^{\,n} \frac{\partial w^{\,n}} {\partial x} +R_{22}^{\,n} \frac{\partial w^{\,n}} {\partial y}+R_{23}^{\,n} \frac{\partial w^{\,n}} {\partial z} \right]  \text{        (production de $R_{23}^{\,n}$)}
\end{array}
$
\end{itemize}

\etape{Calcul du gradient de la masse volumique $\rho^n$ prise au d\'ebut du pas
de temps courant\footnote{{\it i.e.} calcul\'ee \`a partir des
variables du pas de temps pr\'ec\'edent $n$ si n\'ecessaire.} $(n+1)$}
Ce calcul n'a lieu que si les termes de gravit\'e doivent \^etre pris en compte
($\var{IGRARI()} =1$).
\begin{itemize}
\item [$\star$] Appel de \fort{grdcel}  pour calculer le gradient de $\rho^n$
dans les trois directions de l'espace. Les conditions aux limites sur $\rho^n$
sont des conditions de Dirichlet puisque la valeur de $\rho^n$ aux faces de bord
$ik$ (variable \var{IFAC}) est connue et vaut $\rho_{\,b_{\,ik}}$. Pour \'ecrire les conditions aux limites
sous la forme habituelle, $$\rho_{\,b_{\,ik}} = \var{COEFA} + \var{COEFB}
\,\rho^n_{\,I'}$$ on pose alors $\var{COEFA}=
\var{PROPCE(IFAC,IPPROB(IROM(IPHAS)))}$ et $\var{COEFB} = \var{VISCB} = 0$.\\
$\var{PROPCE(1,IPPROB(IROM(IPHAS)))}$ (resp.$\var{VISCB}$) est utilis\'e en lieu
et place de l'habituel \var{COEFA} ($\var{COEFB}$), lors de l'appel \`a \fort{grdcel}.\\
On a donc :\\
$\displaystyle \var{GRAROX}= \frac{\partial \rho^n}{\partial x}\ $,$\displaystyle \ \var{GRAROY}= \frac{\partial
\rho^n}{\partial y}$ et $
\displaystyle \ \var{GRAROZ}= \frac{\partial \rho^n}{\partial z}\ $.

\end{itemize}

Le gradient de $\rho^n$ servira \`a calculer les termes de production par effets de gravit\'e si ces derniers sont pris en compte.

\etape{Boucle \var{ISOU} de $1$ \`a $6$ sur les tensions de Reynolds}
Pour $\var{ISOU} = 1,2,3,4,5,6$, on r\'esout respectivement et dans
l'ordre  les
\'equations de $R_{11}$, $R_{22}$, $R_{33}$, $R_{12}$, $R_{13}$ et $R_{23}$ par
l'appel au sous-programme \fort{resrij}.\\
La r\'esolution se fait par incr\'ement $\delta {R}_{ij}^{\,n+1,k+1}$ , en utilisant la m\^eme m\'ethode que
celle d\'ecrite dans le sous-programme \fort{codits}. On adopte ici les m\^emes notations.
\var{SMBR} est le second membre du syst\`eme \`a inverser, syst\`eme portant sur
les incr\'ements de la variable. \var{ROVSDT} repr\'esente la diagonale de la
matrice, hors convection/diffusion.\\
On va r\'esoudre l'\'equation (\ref{Base_Turrij_Eq_Temp_Rij}) sous forme incr\'ementale en
utilisant \fort{codits}, soit :
\begin{equation}\label{Base_Turrij_Eq_Temp_deltaRij}
\begin{array}{ll}
&\displaystyle \underbrace{\left(\frac {\rho^n_L}{\Delta t^n}
+ \rho^n_L \,C_1\,\frac{\varepsilon^n_L}{k^n_L}(1-\frac{\delta_{ij}}{3})
 - m^n_{\,lm} + \Gamma_L\,+ max(-\alpha^n_{R_{ij}},0)\right)\,|\Omega_l|}
_{\text {$\var{ROVSDT}$ contribuant
\`a la diagonale de la matrice simplifi\'ee de \fort{matrix}}}\,(\delta{R}_{ij}^{\,n+1,p+1})_{\,L}\\\\
&  \underbrace{+\sum\limits_{m\in Vois(l)}\displaystyle \left[
 m^n_{\,lm} \delta R_{ij,\,f_{\,lm}}^{\,n+1,p+1}
- (\mu^n_{\,lm} + \gamma^n_{\,lm})\
\frac{({\delta R}_{ij}^{\,n+1,p+1})_{M}-({\delta R}_{ij}^{\,n+1,p+1})_{L})}{\overline{L'M'}}\,
S_{\,lm} \right]}_{\text { convection upwind pur et diffusion non reconstruite
relatives \`a la matrice simplifi\'ee de \fort{matrix}\footnotemark}} \\
% voir le texte de la footmark plus bas
&= - \displaystyle\frac {\rho^n_L}{\Delta t^n}\,\left(\,(R^{\,n+1,p}_{ij})_L - (R^{\,n}_{ij})_L\,\right)\\
&-\,\underbrace{\displaystyle\int_{\Omega_l} \left(
\dive\,[\,(\rho\,\vect{u})^n\,R^{\,n+1,p}_{ij} - (\mu^n\,+ \gamma^n\,)\,
\grad{R^{\,n+1,p}_{ij}}\,]\right)\,d\Omega}_{\text {convection et diffusion
trait\'ees par \fort{bilsc2}}}\\
&+\displaystyle \int_{\Omega_l} \left[\,\mathcal{P}^{\,n+1,p}_{ij} + \mathcal{G}^{\,n+1,p}_{ij}
- \displaystyle\rho^n \,C_1\,\frac{\varepsilon^n}{k^n}\left[R^{\,n+1,p}_{ij}-
\frac{2}{3}\,k^n\,\delta_{ij}\right] + \phi^{\,n+1,p}_{ij,2} +
\phi^{\,n+1,p}_{ij,w}\,\right]\, d\Omega \\
& + \displaystyle\int_{\Omega_l} \left[- \frac{2}{3} \rho^n \varepsilon^n \delta_{ij}
 + \Gamma\,(\,R^{\,in}_{ij} - R^{\,n+1,p}_{ij}\,) +
\alpha^n_{R_{ij}}\,R^{\,n+1,p}_{ij}+ \beta^n_{R_{ij}}\right]\, d\Omega\\
&+ \sum\limits_{m\in
Vois(l)}\displaystyle \left[\ \tens{E}^n\,\grad{R}^{\,n+1,p}_{ij} \right]_{\,lm}\,.\,\vect{n}_{\,lm}S_{\,lm}\\
&+ \sum\limits_{m\in Vois(l)}\displaystyle \left[\
\tens{D}^n\,\grad{R}^{\,n+1,p}_{ij} \right]_{\,lm}\,.\,\vect{n}_{\,lm}S_{\,lm}\\
&- \sum\limits_{m\in Vois(l)} \gamma^n_{\,lm} \left( \grad{R}^{\,n+1,p}_{ij}\,.\,\vect{n}_{\,lm} \right)  S_{\,lm}\\
&+ \sum\limits_{m\in Vois(l)}  m^n_{\,lm}\,(R^{\,n+1,p}_{ij})_L\\
\end{array}
\end{equation}
% si on ne fait pas comme ca, il n'apparait pas
\footnotetext[\thefootnote]{Si $\var{IRIJNU} = 1$, on remplace  $\mu^n_{\,lm}$ par $(\mu +
\mu_t)^n_{\,lm}$ dans l'expression de la diffusion non reconstruite
associ\'ee \`a la matrice simplifi\'ee de \fort{matrix} ($\mu_t$ d\'esigne la
viscosit\'e turbulente calcul\'ee comme en $k-\varepsilon$).}

o\`u on rappelle :\\
pour $n$ donn\'e entier positif, on d\'efinit la suite
 $({R}_{ij}^{\,n+1,p})_{p \in \grandN}$
 par :
\begin{equation}\notag
\left\{\begin{array}{l}
{R}_{ij}^{\,n+1,0} = {R}_{ij}^{\,n}\\
{R}_{ij}^{\,n+1,p+1} = {R}_{ij}^{\,n+1,p} + \delta{R}_{ij}^{\,n+1,p+1} \\
\end{array}\right.
\end{equation}
$(\delta{R}_{ij}^{\,n+1,p+1})_{\,L}$ d\'esigne la valeur sur l'\'el\'ement
$\Omega_l$ du $\text{$(\,p+1\,)$-i\`eme}$ incr\'ement de ${R}_{ij}^{\,n+1}$,
$ m^n_{\,lm}$ le flux de masse \`a l'instant $n$ \`a travers la face $lm$,
$\delta R_{ij,\,f_{\,lm}}^{\,n+1,p+1}$ vaut $({\delta
R}_{ij}^{\,n+1,p+1})_{L}$  si $ m^n_{\,lm} \geqslant 0$, $({\delta
R}_{ij}^{\,n+1,p+1})_{M}$ sinon,
$\mathcal{P}^{\,n+1,p}_{ij}$, $\phi^{\,n+1,p}_{ij,2}$, $\phi^{\,n+1,p}_{ij,w}$ les valeurs
des quantit\'es associ\'ees correspondant \`a l'incr\'ement
$(\delta{R}_{ij}^{\,n+1,p})$.\\



Tous ces termes sont calcul\'es comme suit :
\begin{itemize}
\item Terme de gauche de l'\'equation (\ref{Base_Turrij_Eq_Temp_deltaRij})\\
Dans \fort{resrij} est calcul\'ee la variable \var{ROVSDT}. Les autres
termes sont compl\'et\'es par \fort{codits}, lors de la construction de la matrice simplifi\'ee , {\it via} un
appel au sous-programme \fort{matrix}. La quantit\'e
 $(\mu^n_{\,lm} + \gamma^n_{\,lm})$ \`a la face $lm$ est calcul\'ee lors de l'appel \`a
\fort{visort}.\\
\item Second membre de l'\'equation (\ref{Base_Turrij_Eq_Temp_deltaRij})\\
Le premier terme non d\'etaill\'e est calcul\'e par le sous-programme
\fort{bilsc2}, qui applique le sch\'ema convectif choisi par l'utilisateur, qui
reconstruit ou non selon le souhait de l'utilisateur les gradients intervenants
dans la convection-diffusion.\\
Les termes sans accolade sont, eux, compl\`etement explicites et ajout\'es au fur et
\`a mesure dans \var{SMBR} pour former
l'expression $f^{\,exp}_s$ de \fort{codits}.
\end{itemize}
On d\'ecrit ci-dessous les \'etapes de \fort{resrij} :
\begin{itemize}

\item DELTIJ = 1, pour $\var{ISOU} \leqslant 3$ et DELTIJ = 0  Si $\var{ISOU} >
3$. Cette valeur repr\'esente le symbole de Kroeneker $\delta_{ij}$.

\item Initialisation \`a z\'ero de \var{SMBR} (tableau contenant le second
membre) et \var{ROVSDT} (tableau contenant la diagonale de la matrice sauf celle
relative \`a la contribution de la
diagonale des op\'erateurs de convection et de diffusion lin\'earis\'es
\footnote{qui correspondent aux sch\'emas convectif upwind pur et diffusif sans
reconstruction.}), tous deux de dimension $\var{NCEL}$.

\item Lecture et prise en compte des termes sources utilisateur pour la variable $R_{ij}$

Appel \`a \fort{ustsri} pour charger les termes sources utilisateurs. Ils sont
stock\'es comme suit. Pour la cellule $\Omega_l$ de centre $L$, repr\'esent\'ee par $\var{IEL}$, on a :\\
\begin{equation}\notag
\left\{\begin{array}{lll}
&\var{ROVSDT(IEL)}&= |\Omega_l| \ \alpha_{R_{ij}}\\
&\var{SMBR(IEL)}&=|\Omega_l| \ \beta_{R_{ij}}\\
\end{array}\right.
\end{equation}
On affecte alors les valeurs ad\'equates au second membre \var{SMBR} et \`a la
diagonale \var{ROVSDT} comme suit :
\begin{equation}\notag
\left\{\begin{array}{lll}
&\var{SMBR(IEL)} &= \var{SMBR(IEL)} +\ |\Omega_l| \ \alpha_{R_{ij}} \ (R^n_{ij})_L \\
&\var{ROVSDT(IEL)}&= \text{max }(-\ |\Omega_l| \ \alpha_{R_{ij}},0)\\
\end{array}\right.
\end{equation}
La valeur de $ \var{ROVSDT}$ est ainsi calcul\'ee pour des raisons de stabilit\'e
num\'erique. En effet, on ne rajoute sur la diagonale que les valeurs positives,
ce qui correspond physiquement \`a impliciter les termes de rappel uniquement.
\item{Calcul du terme source de masse  si $\Gamma_L > 0$}

Appel de \fort{catsma} et incr\'ementation si n\'ecessaire de \var{SMBR} et
\var{ROVSDT} {\it via} :\\
\begin{equation}\notag
\left\{\begin{array}{lll}
\displaystyle \var{SMBR(IEL)} = \var{SMBR(IEL)} + |\Omega_l| \ \Gamma_L \
\left[(R^{\,in}_{ij})_L - (R^{\,n}_{ij})_L \right] \\
\displaystyle \var{ROVSDT(IEL)}=\var{ROVSDT(IEL)} + |\Omega_l| \ \Gamma_L
\end{array}\right.
\end{equation}
\item Calcul du terme d'accumulation de masse et du terme instationnaire

On stocke $\displaystyle \var{W1}= \int_{\Omega_l}\dive\,(\rho\,\vect{u})\,d\Omega$
calcul\'e par \fort{divmas} \`a l'aide des flux de masse aux faces internes
$ m^n_{\,lm}=\sum\limits_{m\in Vois(l)}{(\rho \vect{u})_{\,lm}^n} \text{.}\,
\vect{S}_{\,lm} $ (tableau \var{FLUMAS}) et des flux de masse aux bords  $ m^n_{\,b_{lk}} = \sum\limits_{k\in{\gamma_b(l)}}{(\rho \vect{u})_{\,{b}_{lk}}^n} \text{.}\,
\vect{S}_{\,{b}_{lk}} $ (tableau \var{FLUMAB}).
On incr\'emente ensuite \var{SMBR} et \var{ROVSDT}.
\begin{equation}\notag
\left\{\begin{array}{lll}
&\var{SMBR(IEL)} &= \var{SMBR(IEL)} + \var{ICONV}\  (R^n_{ij})_L\,(\displaystyle
\int_{\Omega_l}\dive\,(\rho\,\vect{u})\ d\Omega) \\
&\var{ROVSDT(IEL)}& = \var{ROVSDT(IEL)} +  \var{ISTAT}\,\displaystyle
\frac{\rho^n_L \ |\Omega_l|}{\Delta t^n} -  \var{ICONV}\ (\displaystyle
\int_{\Omega_l}\dive\,(\rho\,\vect{u})\ d\Omega) \\
\end{array}\right.
\end{equation}
\item Calcul des termes sources de production, des termes $\displaystyle
\phi_{\,ij,1}+\phi_{\,ij,2}$ et de la dissipation~$\displaystyle-\frac{2}{3} \varepsilon\,\delta_{\,ij}$ :

On effectue une boucle d'indice \var{IEL} sur les cellules $\Omega_l$ de centre $L$ :
\begin{itemize}
\item [$\Rightarrow$] $\displaystyle \var{TRPROD}= \frac{1}{2} (\mathcal{P}^n_{ii})_L = \frac{1}{2} \left[ \var{PRODUC(1,IEL)} +  \var{PRODUC(2,IEL)} +  \var{PRODUC(3,IEL)} \right] $
\item [$\Rightarrow$] $\displaystyle \var{TRRIJ }= \frac{1}{2} (R^n_{ii})_L $
\item [$\Rightarrow$] $\displaystyle \var{SMBR(IEL)} =\ \var{SMBR(IEL)}\ +$\\
$\ \displaystyle\rho^n_L |\Omega_l| \left[ \displaystyle
\frac{2}{3}\,\delta_{\,ij} \left( \ \displaystyle \frac{ C_2}{2}\,(\mathcal{P}^n_{ii})_L\ +
(C_1-1)\ \varepsilon^n_L\, \right)\right.$\\
$ + \left.\ (1-C_2) \ \var{PRODUC(ISOU,IEL)} -
\displaystyle C_1\ \frac{2\,\varepsilon^n_L}{(R^n_{ii})_L}\ (R^n_{ij})_L \right]$
\item [$\Rightarrow$] $\displaystyle \var{ROVSDT(IEL)} = \var{ROVSDT(IEL)} +
\rho^n_L \ |\Omega_l| \ (- \displaystyle \frac{1}{3} \ \,\delta_{\,ij} + 1) \ C_1
\ \frac{2\ \varepsilon^n_L}{(R^n_{ii})_L}$
\end{itemize}
\item Appel de \fort{rijech} pour le calcul des termes d'\'echo de paroi
 $\phi^n_{ij,w}$ si $\var{IRIJEC()}=1$ et ajout dans \var{SMBR}.\\
$\var{SMBR} = \var{SMBR} + \phi^n_{ij,w}$\\
Suivant son mode de calcul (\var{ICDPAR}), la distance � la paroi est directement accessible
par \var{RA(IDIPAR+IEL-1)} (\var{|ICDPAR|} = 1) ou bien
est calcul\'ee \`a partir de $\var{IA(IIFAPA(IPHAS)+IEL - 1)}$,
qui donne pour l'\'el\'ement $\var{IEL}$ le num\'ero de la face de bord
paroi la plus  proche (\var{|ICDPAR|} = 2). Ces tableaux ont \'et\'e renseign\'e une fois pour toutes au
d\'ebut de calcul.

\item  Appel de \fort{rijthe} pour le calcul des termes de gravit\'e $\mathcal{G}^n_{ij}$ et ajout dans \var{SMBR}.

Ce calcul n'a lieu que si $\var{IGRARI()} = 1$.
$ \var{SMBR} = \var{SMBR} + \mathcal{G}^n_{ij}$
\item Calcul de la partie explicite du terme de diffusion
 $\dive{\,\left[\tens{A}\,\grad{R}^{\,n}_{ij}\right]}$, plus pr\'ecis\'ement
des contributions du terme extradiagonal pris aux faces purement internes
(remplissage du tableau \var{VISCF}), puis aux faces de bord (remplissage du
tableau \var{VISCB}).
\begin{itemize}
\item [$\star$] Appel de \fort{grdcel} pour le calcul du gradient de
$R^{\,n}_{ij}$ dans chaque direction. Ces gradients sont respectivement
stock\'es dans les tableaux de travail \var{W1}, \var{W2} et \var{W3}.

\item [$\star$] boucle d'indice \var{IEL} sur les cellules $\Omega_l$ de centre
$L$ pour le
calcul de $\tens{E}^n\,\grad{R}^{\,n}_{ij}$ aux cellules dans un premier temps :\\
\begin{itemize}
\item [$\Rightarrow$] $\displaystyle \var{TRRIJ}= \frac{1}{2} (R^{\,n}_{ii})_L $
\item [$\Rightarrow$] $\displaystyle \var{CSTRIJ} = \rho^n_L\ C_S \ \displaystyle\frac{(R^n_{ii})_L}{2\,\varepsilon^n_L}$
\item [$\Rightarrow$] $\displaystyle \var{W4(IEL)} = \rho^n_L\ C_S\
\displaystyle\frac{(R^n_{ii})_L}{2\,\varepsilon^n_L} \left[\,(R^{\,n}_{12})_L \ \var{W2(IEL)} +
(R^{\,n}_{13})_L \ \var{W3(IEL)}\,\right]$
\item [$\Rightarrow$] $\displaystyle \var{W5(IEL)} = \rho^n_L\ C_S\
\displaystyle\frac{(R^n_{ii})_L}{2\,\varepsilon^n_L} \left[\,(R^{\,n}_{12})_L \ \var{W1(IEL)} +
(R^{\,n}_{23})_L \ \var{W3(IEL)}\,\right]$
\item [$\Rightarrow$] $\displaystyle \var{W6(IEL)} = \rho^n_L\ C_S\
\displaystyle\frac{(R^n_{ii})_L}{2\,\varepsilon^n_L} \left[\,(R^{\,n}_{13})_L \ \var{W1(IEL)} + (R^{\,n}_{23})_L \ \var{W2(IEL)}\,\right]$
\end{itemize}



\item [$\star$] Appel de \fort{vectds}\footnote{Le r\'esultat est stock\'e dans
\var{VISCF} et \var{VISCB}. Dans \fort{vectds}, les valeurs aux faces internes
sont interpol\'ees lin\'eairement sans reconstruction et \var{VISCB} est mis \`a
z\'ero.} pour assembler $\displaystyle\left[ \tens{E}^n\,\grad{R}^{\,n}_{ij}
\right]\,.\,\vect{n}_{\,lm}S_{\,lm}$ aux faces $lm$.
\item [$\star$] Appel de \fort{divmas} pour calculer la divergence du flux d\'efini par \var{VISCF} et \var{VISCB}.
Le r\'esultat est stock\'e dans \var{W4}.\\
Ajout au second membre \var{SMBR}.\\
\var{SMBR} = \var{SMBR} + \var{W4}
\end{itemize}

A l'issue de cette \'etape, seule la partie extradiagonale de la diffusion prise
enti\`erement explicite~:
 $$\sum\limits_{m\in
Vois(l)}\left[\ \tens{E}^n\,\grad{R}^{\,n}_{ij} \right]_{\,lm}\,.\,\vect{n}_{\,lm}S_{\,lm}$$ a \'et\'e calcul\'ee.\\

\item Calcul de la partie diagonale du terme de diffusion\footnote{Seule la
composante normale  du  gradient de $R_{ij}$ aux faces sera implicite.} :\\
Comme on l'a d\'eja vu, une partie de cette contribution sera trait\'ee en
implicite (celle relative \`a la composante normale du gradient) et donc
ajout\'ee au second membre par \fort{bilsc2} ; l'autre
partie sera explicite et prise en compte dans $f_s^{\,exp}$.
\begin{itemize}
\item [$\star$] On effectue une boucle d'indice \var{IEL} sur les cellules
$\Omega_l$ de centre $L$ :
\begin{itemize}
\item [$\Rightarrow$] $\displaystyle \var{TRRIJ }= \frac{1}{2} (R^{\,n}_{ii})_L $
\item [$\Rightarrow$] $\displaystyle \var{CSTRIJ} = \rho^n_L \ C_S \ \frac{(R^{\,n}_{ii})_L}{2\,\varepsilon^n_L}$
\item [$\Rightarrow$] $\displaystyle \var{W4(IEL)} = \rho^n_L \ C_S \
\frac{(R^{\,n}_{ii})_L}{2\,\varepsilon^n_L} \ (R^{\,n}_{11})_L$
\item [$\Rightarrow$] $\displaystyle \var{W5(IEL)} = \rho^n_L \ C_S \ \frac{(R^{\,n}_{ii})_L}{2\,\varepsilon^n_L}\ (R^n_{22})_L$
\item [$\Rightarrow$] $\displaystyle \var{W6(IEL)} = \rho^n_L \ C_S \ \frac{(R^{\,n}_{ii})_L}{2\,\varepsilon^n_L} \ (R^n_{33})_L$
\end{itemize}

%\item Traitement du parall\'elisme et de la p\'eriodicit\'e.

\item [$\star$] On effectue une boucle d'indice \var{IFAC} sur les faces
purement internes $lm$ pour remplir le tableau \var{VISCF} :
\begin{itemize}
\item [$\Rightarrow$] Identification des cellules $\Omega_l$ et $\Omega_m$ de
centre respectif $L$ (variable \var{II}) et $M$ (variable \var{JJ}), se trouvant de chaque c\^ot\'e de la face
$lm$\footnote{La normale \`a la face est orient\'ee de L vers M.}.
\item [$\Rightarrow$] Calcul du carr\'e de la surface de la face. La valeur est
stock\'ee dans le tableau \var{SURFN2}.
\item [$\Rightarrow$] Interpolation du gradient de $R^{\,n}_{ij}$ \`a la face
$lm$ (gradient facette $\left[\grad{R}^{\,n}_{ij}\right]_{\,lm}$) :
\begin{equation}\notag
\left\{\begin{array}{ll}
\var{GRDPX} &= \displaystyle \frac{1}{2} \left(\var{W1(II)} + \var{W1(JJ)}
\right) \\
&\\
\var{GRDPY} &= \displaystyle \frac{1}{2} \left(\var{W2(II)} + \var{W2(JJ)} \right) \\
&\\
\var{GRDPZ} &= \displaystyle \frac{1}{2} \left(\var{W3(II)} + \var{W3(JJ)} \right)
\end{array}\right.
\end{equation}
\item [$\Rightarrow$] Calcul du gradient de $R^{\,n}_{ij}$ normal \`a la face
$lm$, $\left[\grad{R}^{\,n}_{ij}\right]_{\,lm}.\vect{n}_{\,lm}\,S_{\,lm}$.\\

$\displaystyle \var{GRDSN} =  \var{GRDPX} \ \var{SURFAC(1,IFAC)} + \var{GRDPY} \ \var{SURFAC(2,IFAC)} +  \var{GRDPZ} \ \var{SURFAC(3,IFAC)}$
$\var{SURFAC}$ est le vecteur surface de la face \var{IFAC}.


\item [$\Rightarrow$] calcul de
 $\left[\grad{R^{\,n}_{ij}} - (\grad
R^{\,n}_{ij}\,.\,\vect{n}_{\,lm})\vect{n}_{\,lm}\right]$, les vecteurs \'etant
calcul\'es \`a la face $lm$ :
\begin{equation}\notag
\left\{\begin{array}{lll}
&\displaystyle \var{GRDPX} &= \var{GRDPX} - \displaystyle\frac{\var{GRDSN}}{\var{SURFN2}} \ \var{SURFAC(1,IFAC)}\\
&&\\
&\displaystyle \var{GRDPY} &= \var{GRDPY} - \displaystyle\frac{\var{GRDSN}}{\var{SURFN2}} \ \var{SURFAC(2,IFAC)} \\
&&\\
&\displaystyle \var{GRDPZ} &= \var{GRDPZ} - \displaystyle\frac{\var{GRDSN}}{\var{SURFN2}} \ \var{SURFAC(3,IFAC)}
\end{array}\right.
\end{equation}
\item [$\Rightarrow$] finalisation du calcul de l'expression totalement
explicite
 $$\left[ \tens{D}^n\,\left( \grad{R^{\,n}_{ij}} - (\grad R^{\,n}_{ij}\,.\,\vect{n}_{\,lm})\,\vect{n}_{\,lm}\right) \right]\,.\,\vect{n}_{\,lm}$$
\begin{equation}\notag
\begin{array} {ll}
\displaystyle \var{VISCF} = &
 \displaystyle\frac{1}{2} (\ \var{W4(II)} +\ \var{W4(JJ)}) \ \var{GRDPX} \
\var{SURFAC(1,IFAC)})\ + \\
&\\
&  \displaystyle\frac{1}{2} (\ \var{W5(II)} +\ \var{W5(JJ)}) \ \var{GRDPY} \
\var{SURFAC(2,IFAC)})\ + \\
&\\
&  \displaystyle\frac{1}{2} (\ \var{W6(II)} +\ \var{W6(JJ)}) \ \var{GRDPZ} \ \var{SURFAC(3,IFAC)})
\end{array}
\end{equation}
\end{itemize}

\item [$\star$] Mise \`a z\'ero du tableau \var{VISCB}.

\item [$\star$] Appel de \fort{divmas} pour calculer la divergence de~:
 $$\tens{D}^{\,n}\,\left( \grad{R^{\,n}_{ij}} - (\grad R^{\,n}_{ij}\,.\,\vect{n}_{\,lm})\vect{n}_{\,lm}\right)$$ d\'efini aux faces dans \var{VISCF} et \var{VISCB}.

Le r\'esultat est stock\'e dans le tableau \var{W1}.\\
Ajout au second membre \var{SMBR}.\\
$\var{SMBR} = \var{SMBR} + \var{W1}$
\end{itemize}
\item Calcul de la viscosit\'e orthotrope $\gamma^n_{\,lm}$ \`a la face $lm$ de la variable principale
$R^{\,n}_{ij}$\\
Ce calcul permet au sous-programme \fort{codits} de compl\'eter le second membre
\var{SMBR} par :
\begin{equation}
\begin{array} {ll}
& \sum\limits_{m\in Vois(l)}
\mu^n_{\,lm}\,\left(\grad{R}^{\,n}_{ij}\,.\,\vect{n}_{\,lm}\right)S_{\,lm}
 + \sum\limits_{m\in Vois(l)} \left(\grad{R}^{\,n}_{ij}
\,.\,\vect{n}_{\,lm}\right)\left[\tens{D}^{\,n}\,\vect{n}_{\,lm}\right]_{\,lm}\,.\,\vect{n}_{\,lm}\
S_{\,lm}\\
& = \sum\limits_{m\in Vois(l)}(\,\mu^n_{\,lm}\, + \,\gamma^n_{\,lm}\,)\,\left(\grad{R}^{\,n}_{ij}\,.\,\vect{n}_{\,lm}\right)S_{\,lm}
\end{array}
\end{equation}
sans pr\'eciser la nature de la face $lm$, {\it via} l'appel \`a \fort{bilsc2}  et de disposer de la quantit\'e
$(\mu^n_{\,lm}\, + \gamma^n_{\,lm})$ pour construire sa
matrice simplifi\'ee.\\
\begin{itemize}
\item [$\star$] On effectue une boucle d'indice \var{IEL} sur les cellules
$\Omega_l$ :
\begin{itemize}
\item [$\Rightarrow$] $\displaystyle \var{TRRIJ }= \frac{1}{2} (R^{\,n}_{ii})_L $
\item [$\Rightarrow$] $\displaystyle \var{RCSTE} = \rho^n_L \ C_S \ \frac{ (R^{\,n}_{ii})_L}{2\,\varepsilon^n_L} $
\item [$\Rightarrow$] $\displaystyle \var{W1(IEL)} = \mu^n +\rho^n_L \ C_S \ \frac{
(R^{\,n}_{ii})_L}{2\,\varepsilon^n_L}\ (R^n_{11})_L$
\item [$\Rightarrow$] $\displaystyle \var{W2(IEL)} = \mu^n + \rho^n_L \ C_S \ \frac{ (R^{\,n}_{ii})_L}{2\,\varepsilon^n_L}\ (R^n_{22})_L$
\item [$\Rightarrow$] $\displaystyle \var{W3(IEL)} = \mu^n + \rho^n_L \ C_S \ \frac{ (R^{\,n}_{ii})_L}{2\,\varepsilon^n_L}\ (R^n_{33})_L$
\end{itemize}

\item [$\star$] Appel de \fort{visort} pour calculer la viscosit\'e orthotrope
\footnote{Comme dans le sous-programme \fort{viscfa}, on multiplie la viscosit\'e par
$\displaystyle \frac{S_{\,lm}}{\overline{L'M'}}$, o\`u $S_{\,lm}$ et
$\overline{L'M'}$ repr\'esentent respectivement la surface de la face $lm$ et la
mesure alg\'ebrique du segment reliant les projections des centres des cellules
voisines sur la normale \`a la face. On garde dans ce sous-programme  la possibilit\'e d'interpoler la viscosit\'e aux cellules lin\'eairement ou d'utiliser une moyenne harmonique. La viscosit\'e au bord est celle de la cellule de bord correspondante.}
$ \gamma^n_{\,lm} = (\tens{D}^{\,n}\,\vect{n}_{\,lm}).\vect{n}_{\,lm}$ aux faces $lm$

Le r\'esultat est stock\'e dans les tableaux \var{VISCF} et \var{VISCB}.
\end{itemize}

\item appel de \fort{codits} pour la r\'esolution de l'\'equation de
convection/diffusion/termes sources de la variable $R_{ij}$. Le terme source est
\var{SMBR}, la viscosit\'e \var{VISCF} aux faces purement internes (
resp. \var{VISCB} aux faces de bord) et \var{FLUMAS} le flux de masse interne
 ( resp. \var{FLUMAB} flux de masse au bord). Le r\'esultat est la variable $R_{ij}$ au temps
$n+1$, donc $R^{\,n+1}_{ij}$. Elle est stock\'ee dans le tableau \var{RTP} des
variables mises \`a jour.

\end{itemize}

\etape{Appel de \fort{reseps} pour la r\'esolution de la variable $\varepsilon$}

On d\'ecrit ci-dessous le sous-programme \fort{reseps}, les commentaires du sous-programme \fort{resrij} ne seront pas r\'ep\'et\'es puisque les deux sous-programmes ne diff\`erent que par leurs termes sources.

\begin{itemize}
\item Initialisation \`a z\'ero de \var{SMBR} et \var{ROVSDT}.

\item{Lecture et prise en compte des termes sources utilisateur pour la variable $\varepsilon$ :}

Appel de \fort{ustsri} pour charger les termes sources utilisateurs. Ils sont
stock\'es dans les tableaux suivants :\\
pour la cellule $\Omega_l$ repr\'esent\'ee par $\var{IEL}$ de centre $L$, on a :
\begin{equation}\notag
\left\{\begin{array}{lll}
&\var{ROVSDT(IEL)}&= |\Omega_l| \ \alpha_{\varepsilon}\\
&\var{SMBR(IEL)}&=|\Omega_l| \ \beta_{\varepsilon}\\
\end{array}\right.
\end{equation}
On affecte alors les valeurs ad\'equates au second membre \var{SMBR} et \`a la
diagonale \var{ROVSDT} comme suit :
\begin{equation}\notag
\left\{\begin{array}{lll}
&\var{SMBR(IEL)} &= \var{SMBR(IEL)} +\ |\Omega_l| \ \alpha_{\,\varepsilon} \
\varepsilon^n_L \\
&\var{ROVSDT(IEL)}&= \text{max }(-\ |\Omega_l| \ \alpha_{\,\varepsilon},0)\\
\end{array}\right.
\end{equation}

\item{Calcul du terme source de masse si $\Gamma_L > 0$ :
\begin{equation}\notag
\left\{\begin{array}{lll}
&\displaystyle \var{SMBR(IEL)} = \var{SMBR(IEL)} + |\Omega_l| \ \Gamma_L \
(\varepsilon^{\,in}_L -\varepsilon^n_L) \\
&\displaystyle \var{ROVSDT(IEL)}= \var{ROVSDT(IEL)} + |\Omega_l| \ \Gamma_L
\end{array}\right.
\end{equation}
\item Calcul du terme d'accumulation de masse et du terme instationnaire \\
On stocke $\displaystyle \var{W1}= \int_{\Omega_l}\dive\,(\rho\,\vect{u})\,d\Omega$
calcul\'e par \fort{divmas} \`a l'aide des flux de masse internes et aux bords.\\
On incr\'emente ensuite \var{SMBR} et \var{ROVSDT}.
\begin{equation}\notag
\left\{\begin{array}{lll}
&\var{SMBR(IEL)} &= \var{SMBR(IEL)} + \var{ICONV}\ \varepsilon^n_L\,(\displaystyle
\int_{\Omega_l}\dive\,(\rho\,\vect{u})\ d\Omega) \\
&\var{ROVSDT(IEL)}& = \var{ROVSDT(IEL)} +  \var{ISTAT}\,\displaystyle
\frac{\rho^n_L \ |\Omega_l|}{\Delta t^n} -  \var{ICONV}\ (\displaystyle
\int_{\Omega_l}\dive\,(\rho\,\vect{u})\ d\Omega) \\
\end{array}\right.
\end{equation}

\item Traitement du terme de production
 $\displaystyle \rho\,C_{\varepsilon_1}\,\frac{\varepsilon}{k}\,\mathcal{P}$
 et du terme de dissipation $-\,\displaystyle \rho\,C_{\varepsilon_2}\,\frac{\varepsilon}{k}\,\varepsilon$ \\
pour cela, on effectue une boucle d'indice \var{IEL} sur les cellules $\Omega_l$
de centre $L$ :
\begin{itemize}
\item [$\Rightarrow$] $\displaystyle \var{TRPROD}= \frac{1}{2} (\mathcal{P}^n_{ii})_L = \frac{1}{2} \left[ \var{PRODUC(1,IEL)} +  \var{PRODUC(2,IEL)} +  \var{PRODUC(3,IEL)} \right] $
\item [$\Rightarrow$] $\displaystyle \var{TRRIJ }= \frac{1}{2} (R^n_{ii})_L $
\item [$\Rightarrow$] $\displaystyle \var{SMBR(IEL)} = \var{SMBR(IEL)} + \rho^n_L
|\Omega_l| \left[ -C_{\varepsilon_2} \ \frac{2\,(\varepsilon^n_L)^2}{(R^n_{ii})_L} + C_{\varepsilon_1} \ \frac{\varepsilon^n_L}{(R^n_{ii})_L}\ (\mathcal{P}^n_{ii})_L \right] $
\item [$\Rightarrow$] $\displaystyle \var{ROVSDT(IEL)} = \var{ROVSDT(IEL)} + C_{\varepsilon_2} \ \rho^n_L \ |\Omega_l| \ \frac{2\,\varepsilon^n_L}{(R^n_{ii})_L}$
\end{itemize}

\item Appel de \fort{rijthe} pour le calcul des termes de gravit\'e $\mathcal{G}^n_{\varepsilon}$ et ajout dans \var{SMBR}.

$ \var{SMBR} = \var{SMBR} + \mathcal{G}^n_{\varepsilon}$\\
Ce calcul n'a lieu que si $\var{IGRARI()} = 1$.

\item Calcul de la diffusion de $\varepsilon$ \\
 Le terme $\dive \left[\mu\, \grad(\varepsilon) + \tens{A'}\,\grad(\varepsilon)
\right]$ est calcul\'e exactement de la m\^eme mani\`ere que pour les tensions
de Reynolds $R_{ij}$ en rempla\c cant $\tens{A}$ par $\tens{A'}$.

\item Appel de \fort{codits} pour la r\'esolution de l'\'equation de
convection/diffusion/termes sources de la variable principale $\varepsilon$. Le
r\'esultat $\varepsilon^{\,n+1}$ est stock\'e dans le tableau \var{RTP} des
variables mises \`a jour.
}
\end{itemize}

\etape{clippings finaux}
On passe enfin dans le sous-programme  \fort{clprij} pour faire un clipping \'eventuel
des variables $R^{\,n+1}_{ij}$ et $\varepsilon^{\,n+1}$. Le sous-programme
\fort{clprij} est appel\'e\footnote{L'option
$\var{ICLIP} = 1$ consiste en un clipping minimal des variables $R_{ii}$ et
$\varepsilon$ en prenant la valeur absolue de ces variables puisqu'elles ne
peuvent \^etre que positives.} avec $\var{ICLIP} = 2$ . Cette option
\footnote{Quand la valeur des grandeurs $R_{ii}$ ou $\varepsilon$ est
n\'egative, on la remplace par le minimum entre sa valeur absolue et (1,1)
fois la valeur obtenue au pas de temps pr\'ec\'edent.} contient l'option $\var{ICLIP} = 1$  et permet de v\'erifier l'in\'egalit\'e de Cauchy-Schwarz sur les grandeurs extra-diagonales du tenseur $\tens{R}$ (pour $i \neq j$, $|R_{ij}|^2 \le R_{ii} R_{jj}$).


%%%%%%%%%%%%%%%%%%%%%%%%%%%%%%%%%%
%%%%%%%%%%%%%%%%%%%%%%%%%%%%%%%%%%
\section{Points \`a traiter}
%%%%%%%%%%%%%%%%%%%%%%%%%%%%%%%%%%
%%%%%%%%%%%%%%%%%%%%%%%%%%%%%%%%%%
Sauf mention explicite, $\phi$ repr\'esentera une tension de Reynolds ou la dissipation turbulente ($\phi = R_{ij} \ \text{ou} \ \varepsilon$).

\begin{itemize}
\item {La vitesse utilis\'ee pour le calcul de la production est explicite. Est-ce qu'une implicitation peut am\'eliorer la pr\'ecision temporelle de $\phi$ \footnote{Cette remarque peut \^etre g\'en\'eralis\'ee. En effet, peut-on envisager d'actualiser les variables d\'ej\`a r\'esolues (sans r\'eactualiser les variables turbulentes apr\`es leur r\'esolution)? Ceci obligerait \`a modifier les sous-programmes tels que \fort{condli} qui sont appel\'es au d\'ebut de la boucle en temps.} ?}
\item {Dans quelle mesure le terme d'\'echo de paroi est-il valide ? En effet, ce terme est remis en question par certains auteurs.}
\item {On peut envisager la r\'esolution d'un syst\`eme hyperbolique pour les
tensions de Reynolds afin d'introduire un couplage avec le champ de vitesse.}
\item {Le flux au bord \var{VISCB} est annul\'e dans le sous-programme
\fort{vectds}. Peut-on envisager de mettre au bord la valeur de la variable
concern\'ee \`a la cellule de bord correspondant? De m\^eme, il faudrait se
pencher sur les hypoth\`eses sous-jacentes \`a l'annulation des contributions
aux bords de \var{VISCB} lors du calcul de : $$\left[ \tens{D}^n\,\left( \grad{R^{\,n}_{ij}} - (\grad R^{\,n}_{ij}\,.\,\vect{n}_{\,lm})\,\vect{n}_{\,lm}\right) \right]\,.\,\vect{n}_{\,lm}.$$}
\item {Un probl\`eme de pond\'eration appara\^\i t plus g\'en\'eralement. Si on prend la partie explicite de $\tens{D}\,\grad(\phi)$, la pond\'eration aux faces internes utilise le coefficient $\displaystyle\frac{1}{2}$ avec pond\'eration s\'epar\'ee de $\tens{D}$ et $\grad(\phi)$, alors que pour $\tens{E}\,\grad(\phi)$, on calcule d'abord ce terme aux cellules pour ensuite l'interpoler lin\'eairement aux faces \footnote{Cette interpolation se fait dans \fort{vectds} avec des coefficients de pond\'eration aux faces.}. Ceci donne donc deux types d'interpolations pour des termes de m\^eme nature.}
\item {On laisse la possibilit\'e dans \fort{visort} d'utiliser une moyenne
harmonique aux faces. Est-ce que ceci est valable puisque les interpolations
utilis\'ees lors du calcul de la partie explicite de $\tens{A}\,\grad{\phi}$
sont des moyennes arithm\'etiques ?}
\item {Les techniques adopt\'ees lors du clipping sont \`a revoir.}
\item {On utilise dans le cadre du mod\`ele $\displaystyle R_{ij}-\varepsilon $ une semi-implicitation de termes comme $\displaystyle \phi_{ij,1}$ ou $\displaystyle -\rho\,C_{\varepsilon_2}\,\frac{\varepsilon}{k}\,\varepsilon$. On peut envisager le m\^eme type d'implicitation dans \fort{turbke} m\^eme en pr\'esence du couplage $\displaystyle k-\varepsilon$.}
\item L'adoption d'une r\'esolution d\'ecoupl\'ee fait perdre l'invariance par rotation.
\item La formulation et l'implantation des conditions aux limites de paroi
devront \^etre v\'erifi\'ees. En effet, il semblerait que, dans certains cas, des ph\'enom\`enes
``oscillatoires'' apparaissent, sans qu'il soit ais\'e d'en d\'eterminer la cause.
\item L'implicitation partielle (du fait de la r\'esolution d\'ecoupl\'ee) des
conditions aux limites conduit souvent \`a des calculs instables. Il
conviendrait d'en conna\^\i tre la raison. L'implicitation partielle avait
\'et\'e mise en \oe uvre afin de tenter d'utiliser un pas de temps plus grand
dans le cas de jets axisym\'etriques en particulier.

\end{itemize}

%                      Code_Saturne version 1.3
%                      ------------------------
%
%     This file is part of the Code_Saturne Kernel, element of the
%     Code_Saturne CFD tool.
%
%     Copyright (C) 1998-2007 EDF S.A., France
%
%     contact: saturne-support@edf.fr
%
%     The Code_Saturne Kernel is free software; you can redistribute it
%     and/or modify it under the terms of the GNU General Public License
%     as published by the Free Software Foundation; either version 2 of
%     the License, or (at your option) any later version.
%
%     The Code_Saturne Kernel is distributed in the hope that it will be
%     useful, but WITHOUT ANY WARRANTY; without even the implied warranty
%     of MERCHANTABILITY or FITNESS FOR A PARTICULAR PURPOSE.  See the
%     GNU General Public License for more details.
%
%     You should have received a copy of the GNU General Public License
%     along with the Code_Saturne Kernel; if not, write to the
%     Free Software Foundation, Inc.,
%     51 Franklin St, Fifth Floor,
%     Boston, MA  02110-1301  USA
%
%-----------------------------------------------------------------------
%
\programme{vortex}
%
\vspace{1cm}
%%%%%%%%%%%%%%%%%%%%%%%%%%%%%%%%%%
%%%%%%%%%%%%%%%%%%%%%%%%%%%%%%%%%%
\section{Fonction}
%%%%%%%%%%%%%%%%%%%%%%%%%%%%%%%%%%
%%%%%%%%%%%%%%%%%%%%%%%%%%%%%%%%%%
Ce sous-programme est d�di� � la g�n�ration des conditions d'entr�e
turbulente utilis�es en LES.


La m�thode des vortex est bas�e sur une approche de tourbillons
ponctuels. L'id�e de la m�thode consiste � injecter des tourbillons 2D dans le
plan d'entr�e du calcul, puis � calculer le champ de vitesse induit par ces
tourbillons au centre des faces d'entr�e.

%                      Code_Saturne version 1.3
%                      ------------------------
%
%     This file is part of the Code_Saturne Kernel, element of the
%     Code_Saturne CFD tool.
% 
%     Copyright (C) 1998-2007 EDF S.A., France
%
%     contact: saturne-support@edf.fr
% 
%     The Code_Saturne Kernel is free software; you can redistribute it
%     and/or modify it under the terms of the GNU General Public License
%     as published by the Free Software Foundation; either version 2 of
%     the License, or (at your option) any later version.
% 
%     The Code_Saturne Kernel is distributed in the hope that it will be
%     useful, but WITHOUT ANY WARRANTY; without even the implied warranty
%     of MERCHANTABILITY or FITNESS FOR A PARTICULAR PURPOSE.  See the
%     GNU General Public License for more details.
% 
%     You should have received a copy of the GNU General Public License
%     along with the Code_Saturne Kernel; if not, write to the
%     Free Software Foundation, Inc.,
%     51 Franklin St, Fifth Floor,
%     Boston, MA  02110-1301  USA
%
%-----------------------------------------------------------------------
%
%%%%%%%%%%%%%%%%%%%%%%%%%%%%%%%%%%
%%%%%%%%%%%%%%%%%%%%%%%%%%%%%%%%%%
\section{Discr\'etisation}
%%%%%%%%%%%%%%%%%%%%%%%%%%%%%%%%%%
%%%%%%%%%%%%%%%%%%%%%%%%%%%%%%%%%%

Le terme convectif en $\dive(\underline{u} \otimes \rho\,\underline{u})$
introduit une non lin\'earit\'e et un couplage des composantes de la vitesse
$\vect{u}$ dans l'�quation (\ref{Base_Preduv_eqqdm}). Une lin\'earisation et un d\'ecouplage
des trois composantes de la 
vitesse sont r\'ealis\'es lors de la discr\'etisation de cette \'etape de
pr\'ediction.\\
En effet, soit :
\begin{equation}
\vect{\widetilde{u}}= \vect{u}^n + \delta \vect{u} 
\end{equation}
La contribution exacte du terme convectif \`a prendre en compte dans cette
\'etape de pr\'ediction serait :\\
\begin{equation}\label{Base_Preduv_Conv_exact}
\begin{array}{ll}
\dive(\vect{\widetilde{u}} \otimes \rho\,\vect{\widetilde{u}}) =
\dive(\vect{u}^{n} \otimes \rho\,\vect{u}^{n}) + \dive(\delta \vect{u} \otimes
\rho\,\vect{u}^{n}) +  \underbrace { \dive(\vect{u}^{n} \otimes
\rho\,\delta \vect{u})}_{\text {terme couplant lin\'eaire}} +  \underbrace { \dive(\delta \vect{u} \otimes
\rho\,\delta \vect{u})}_{\text {terme couplant et non lin\'eaire}}\\
\end{array} 
\end{equation}
Les deux derniers termes de l'expression (\ref{Base_Preduv_Conv_exact}) sont {\em a priori} n�glig�s
de mani�re � obtenir un syst\`eme en vitesse qui soit d\'ecoupl\'e et donc,
�viter l'inversion d'une matrice pouvant \^etre de tr\`es grande taille. Ces
deux termes peuvent n�anmoins �tre pris en compte de mani�re plus ou moins
approch�e par extrapolation explicite du flux de masse en $n+\theta_F$ (pour le
terme couplant lin�aire provenant de la convection de $\vect{u}^{n}$ par $\delta
\vect{u}$) et utilisation d'un point-fixe par sous it�ration sur le sous
programme \fort{navsto} (pour le terme non-lin�aire, en sp�cifiant $\var{NTERUP}>1$).

L'�quation (\ref{Base_Preduv_eqqdm}) est discr�tis�e au temps $n+\theta$ � l'aide d'un
$\theta$-sch�ma, et le tenseur des pertes de charges d�compos� en une partie
diagonale $\tens{K}_{d}$ et une extradiagonale $\tens{K}_{e}$ (soit
 $\tens{K}_{pdc}=\tens{K}_{d}+\tens{K}_{e}$).\\
$\bullet$ La pression est suppos�e connue en $n-1+\theta$ (d�calage temporel
pression-vitesse) et le gradient naturellement calcul� � cet instant.\\ 
$\bullet$ Les termes sources de viscosit� secondaire, de gradient transpos\'e,
ceux provenant du mod�le de turbulence\footnote{except� $\dive (\mu_t\ (\ggrad
\underline {u}))$}, $\rho\,\tens{K}_{\,e}\ \underline{u}$, $(\rho -\rho_0)
\underline {g}$ ainsi que $\underline{T}_{s}^{\,exp}$ et
$\Gamma\,\underline{u}_{\,i}$ sont pris de mani�re explicite au temps $n$, ou
extrapol�s suivant le sch�ma en temps choisi pour les propri�t�s physique et les
termes sources.\\ 
$\bullet$ Les termes sources $\underline{u}\,\,\dive (\rho\,\underline {u})$,
$\Gamma\,\,\underline{u}$, $T_{s}^{\,imp}\,\,\underline{u}$ et
$-\rho\,\tens{K}_{\,d}\,\,\underline{u}$ sont implicit�s est calcul�s �
l'instant $n+\theta$.\\ 
$\bullet$ Le terme de diffusion $\dive (\mu_{\,tot}\,\ggrad \underline{u})$ est
implicit� : la vitesse est prise � l'instant $n+\theta$ et la viscosit�
explicit�e ou extrapol�e.\\ 
$\bullet$ Enfin, le terme de convection en $\dive(\,\underline{u} \otimes
(\rho\underline{u})\,)$ est implicit� : la composante r�solue de la vitesse est
prise en $n+\theta$, et le flux de masse, explicit�, ou extrapol� en
$n+\theta_F$. 

Par souci de clart�, on suppose, en l'absence d'indication, que les propri�tes
physiques $\Phi$ ($\rho,\,\mu_{tot},\,...$) et le flux de masse
$(\rho\underline{u})$ sont pris respectivement aux instants $n+\theta_\Phi$ et
$n+\theta_F$, o� $\theta_\Phi$ et $\theta_F$ d�pendent des sch�mas en temps
sp�cifiquement utilis�s pour ces grandeurs\footnote{cf. \fort{introd}}. 

La discr�tisation temporelle de l'�quation (\ref{Base_Preduv_eqqdm}) s'�crit alors comme suit : 

\begin{equation}\label{Base_Preduv_eq_di1}
 \begin{array}{c}
\displaystyle \rho\,\ \frac{ \underline {\widetilde{u}}^{n+1} -\underline {u}^{n} }
{\Delta t} + \dive(\,\underline{\widetilde{u}}^{n+\theta} \otimes (\rho\underline{u})\,) -\dive
(\mu_{\,tot}\,\ggrad \underline{\widetilde{u}}^{n+\theta}) =
\\
\displaystyle
 - \grad p^{n-1+\theta} + \dive (\rho\,\underline {u})\,\underline{\widetilde{u}}^{n+\theta} +(\Gamma\,\underline{u}_{\,i})^{n+\theta_S}-\Gamma^n\,\,\underline{\widetilde{u}}^{n+\theta}
\\
\begin{array}{c}
\displaystyle
- \rho\,\tens{K}_{\,d}^{n}\,\,\underline{\widetilde{u}}^{n+\theta} - (\rho\,\tens{K}_{\,e}\ \underline{u})^{n+\theta_S} + (\underline{T}_{s}^{\,exp})^{\,n+\theta_S} + T_{s}^{\,imp}\,\,\underline{\widetilde{u}}^{n+\theta}
\\
\displaystyle
+[\dive (\mu_{\,tot}\,^t\ggrad \underline {u})]^{n+\theta_S}-\frac {2} {3}[\,\grad (\mu_{\,tot}\,\dive \underline {u})]^{n+\theta_S} + (\rho -\rho_0) \underline {g}
 - (\underline{turb})^{n+\theta_S}
\end{array}
\end{array}
\end{equation}
o\`u, par souci de simplification, on a pos\'e :
\begin{equation}
\mu_{\,tot}=
\begin{cases}
\mu+\mu_t & \text{pour les mod�les � viscosit� turbulente ou en LES}, \\
\mu & \text{pour les mod�les au second ordre ou en laminaire}
\end{cases} \ 
\end{equation}
\\
et :
\begin{equation}
\underline{turb}^{n}=
\begin{cases}
\displaystyle\frac {2}{3}\grad (\rho^{n}\,k^{n}) & \text{pour les mod�les � viscosit� turbulente}, \\
\dive(\rho^{n}\,\tens{R}^n) & \text{pour les mod�les au second ordre},\\
0 & \text{en laminaire ou en LES}\\
\end{cases}
\end{equation}
Par analogie avec l'�criture du $\theta$-sch�ma pour une variable scalaire, $\,
\underline {\widetilde{u}}^{n+\theta}$ est interpol�e � partir de la vitesse
pr�dite $\underline {\widetilde{u}}^{n+1}$ de la mani\`ere suivante\footnote{si
$\theta=1/2$, ou qu'une extrapolation est utilis�e, l'ordre 2 n'est obtenu que si
le pas de temps $\Delta t$ est uniforme en temps et en espace.}~: 
\begin{equation}
\underline {\widetilde{u}}^{n+\theta}=\theta\, \underline
{\widetilde{u}}^{n+1}+(1-\theta)\, \underline {u}^{n}\\ 
\end{equation}
Avec :
\begin{equation}
\left\{
\begin{array}{ll}
\theta = 1   & \text{Pour un sch\'ema de type Euler implicite d'ordre 1.}\\
\theta = 1/2 & \text{Pour un sch\'ema de type Cranck-Nicolson d'ordre 2.}\\
\end{array}
\right.
\end{equation}

L'�quation (\ref{Base_Preduv_eq_di1}) est alors r��crite sous la forme :

\begin{equation}\label{Base_Preduv_eq_di2}
\begin{array}{c}
\displaystyle \underbrace{\left(\frac{\rho}{\Delta t} -\theta \,\dive (\rho\,\underline {u}) +\theta \,\, \Gamma^n +
\theta \,\, \rho\,\tens{K}_{\,d}^n-\theta \,T_s^{\,imp} \right)}_{\displaystyle f_s^{imp}}\, (\underline {\,\widetilde{u}}^{n+1} -\underline {u}^{n})
\\
 +\, \theta\, \dive(\underline {\widetilde{u}}^{n+1} \otimes (\rho\underline{u}))-\, \theta\,\dive (\mu_{\,tot}\,\ggrad \underline {\widetilde{u}}^{n+1}) =
\\
-\,(1-\theta)\, \dive(\underline {u}^{n} \otimes (\rho\underline{u})) +\,(1-\theta)\,\dive (\mu_{\,tot}\,\ggrad \underline {u}^{n})
\\
f_s^{exp}\left\{
\begin{array}{c}
\displaystyle 
- \grad p^{n-1+\theta} + \dive (\rho\,\underline {u})\,\underline{u}^{n} +\,(\,\Gamma^{n}\,\underline{u}_{\,i}\,)^{n+\theta_S}- \Gamma^n\,\,\underline{u}^{n}
\\
\displaystyle
-(\,\rho\,\tens{K}_{\,e}\ \underline{u}\,)^{n+\theta_S} -\rho\,\tens{K}_{\,d}^n\ \underline{u}^{n}+ (\underline{T}_{s}^{\,exp})^{\,n+\theta_S} + T_s^{\,imp}\,\,\underline {u}^{n} 
\\
\displaystyle
+[\dive (\mu_{\,tot}\,^t\ggrad \underline {u}\,)]^{n+\theta_S}-\frac {2} {3}[\,\grad (\mu_{\,tot}\,\dive \underline {u}\,)]^{n+\theta_S} + (\rho -\rho_0) \underline {g}-(\underline{turb})^{n+\theta_S}
\end{array}
\right.
\end{array}
\end{equation}

d'o� l'�quation r�solue par le sous-programme \fort{codits} :
\begin{equation}\begin{array}{c}
\displaystyle
f_s^{\,imp}(\underline {\widetilde{u}}^{n+1}-\underline {u}^{n}) + \theta\, \dive(\underline{\widetilde{u}}^{n+1} \otimes (\rho
\underline{u})) - \theta\,\dive (\,\mu_{\,tot}\,\ggrad \underline{\widetilde{u}}^{n+1}) = 
\\\\
\displaystyle
-(1-\theta)\,\dive(\underline{u}^{n} \otimes (\rho \underline{u}))+(1-\theta)\,\dive (\,\mu_{\,tot}\,\ggrad \underline{u}^{n})
+ \underline{f}_{\,s}^{\,exp}
\end{array}
\end{equation}
La m\'ethode de discr\'etisation spatiale est d\'evelopp\'ee dans le sous-programme \fort{codits}.\\



\minititre{Remarques :}
{\tiny$\blacksquare$} Dans le cas standard sans extrapolation, le terme
$-\,T_s^{\,imp}$ n'est ajout� � $f_s^{\,imp}$ que s'il est positif afin de ne
pas affaiblir la dominance de la diagonale de la matrice � inverser.\\ 
{\tiny$\blacksquare$} Si une extrapolation est utilis�e, par contre,
$\,T_s^{\,imp}$ est ajout� � $f_s^{\,imp}$ quel que soit son signe. En effet, l'id�e intuitive qui
consiste � prendre~: 
\begin{equation}
\begin{cases}
\displaystyle
(\underline{T}_{s}^{\,exp} + T_{s}^{\,imp}\,\underline {u})^{\,n+\theta_S} &
\text{si } T_{s}^{\,imp} > 0\\ 
\displaystyle
(\underline{T}_{s}^{\,exp})^{\,n+\theta_S} + T_{s}^{\,imp}\,\underline{u}^{n+\theta} &\text{sinon}\\
\end{cases}
\end{equation} 
aboutit � une incoh�rence dans le traitement si $T_s^{imp}$ change de signe
entre deux pas de temps.\\ 
{\tiny$\blacksquare$} la partie diagonale $\tens{K}_{\,d}$ du terme
de perte de charge est utilis�e dans $f_s^{\,imp}$. En fait, pour \^etre rigoureux,
il faudrait ne retenir que les contributions positives (point signal\'e dans le
sous-programme utilisateur associ\'e \fort{uskpdc}). Cette prise en compte sera \`a am\'eliorer.\\
{\tiny$\blacksquare$} Le terme $\theta\,\Gamma^{n}-\theta\,\dive
(\rho\,\underline {u})$ ne pose pas de probl�me pour la 
dominance de la diagonale de la matrice car il est exactement compens� par le
terme de convection (cf. \fort{covofi}). 


%                      Code_Saturne version 1.3
%                      ------------------------
%
%     This file is part of the Code_Saturne Kernel, element of the
%     Code_Saturne CFD tool.
%
%     Copyright (C) 1998-2007 EDF S.A., France
%
%     contact: saturne-support@edf.fr
%
%     The Code_Saturne Kernel is free software; you can redistribute it
%     and/or modify it under the terms of the GNU General Public License
%     as published by the Free Software Foundation; either version 2 of
%     the License, or (at your option) any later version.
%
%     The Code_Saturne Kernel is distributed in the hope that it will be
%     useful, but WITHOUT ANY WARRANTY; without even the implied warranty
%     of MERCHANTABILITY or FITNESS FOR A PARTICULAR PURPOSE.  See the
%     GNU General Public License for more details.
%
%     You should have received a copy of the GNU General Public License
%     along with the Code_Saturne Kernel; if not, write to the
%     Free Software Foundation, Inc.,
%     51 Franklin St, Fifth Floor,
%     Boston, MA  02110-1301  USA
%
%-----------------------------------------------------------------------
%

%%%%%%%%%%%%%%%%%%%%%%%%%%%%%%%%%%
%%%%%%%%%%%%%%%%%%%%%%%%%%%%%%%%%%
\section{Mise en \oe uvre}
%%%%%%%%%%%%%%%%%%%%%%%%%%%%%%%%%%
%%%%%%%%%%%%%%%%%%%%%%%%%%%%%%%%%%
La num\'ero de la phase trait\'ee fait partie des arguments de \fort{turrij}. On
omettra volontairement de le pr\'eciser dans ce qui suit, on indiquera par $(\ )$ la
notion de tableau s'y rattachant.

\etape{Calcul des termes de production $\tens{\mathcal{P}}$}
\begin{itemize}
\item [$\star$] Initialisation \`a z\'ero du tableau \var{PRODUC} dimensionn\'e \`a $\var{NCEL}\times 6$.
\item [$\star$] On appelle trois fois \fort{grdcel} pour calculer les gradients des composantes de la vitesse $u$, $v$ et
$w$ prises au temps $n$.

Au final, on a :\\
$\displaystyle
\begin{array} {ll}
\var{PRODUC(1,IEL)} = & \displaystyle - 2 \left[ R_{11}^{\,n} \frac{\partial u^{\,n}} {\partial x} +R_{12}^{\,n} \frac{\partial u^{\,n}} {\partial y}+R_{13}^{\,n} \frac{\partial u^{\,n}} {\partial z} \right] \text{        (production de $R_{11}^{\,n}$)}\\
\var{PRODUC(2,IEL)} = & \displaystyle - 2 \left[ R_{12}^{\,n} \frac{\partial v^{\,n}} {\partial x} +R_{22}^{\,n} \frac{\partial v^{\,n}} {\partial y}+R_{23}^{\,n} \frac{\partial v^{\,n}} {\partial z} \right] \text{        (production de $R_{22}^{\,n}$)}\\
\var{PRODUC(3,IEL)} = & \displaystyle - 2 \left[ R_{13}^{\,n} \frac{\partial w^{\,n}} {\partial x} +R_{23}^{\,n} \frac{\partial w^{\,n}} {\partial y}+R_{33}^{\,n} \frac{\partial w^{\,n}} {\partial z} \right] \text{        (production de $R_{33}^{\,n}$)}\\
\var{PRODUC(4,IEL)} = & \displaystyle - \left[ R_{12}^{\,n} \frac{\partial u^{\,n}} {\partial x} +R_{22}^{\,n} \frac{\partial u^{\,n}} {\partial y}+R_{23}^{\,n} \frac{\partial u^{\,n}} {\partial z} \right] \\
& \displaystyle - \left[ R_{11}^{\,n} \frac{\partial v^{\,n}} {\partial x} +R_{12}^{\,n} \frac{\partial v^{\,n}} {\partial y}+R_{13}^{\,n} \frac{\partial v^{\,n}} {\partial z} \right] \text{        (production de $R_{12}^{\,n}$)} \\
\var{PRODUC(5,IEL)} = & \displaystyle - \left[ R_{13}^{\,n} \frac{\partial u^{\,n}} {\partial x} +R_{23}^{\,n} \frac{\partial u^{\,n}} {\partial y}+R_{33}^{\,n} \frac{\partial u^{\,n}} {\partial z} \right] \\
& \displaystyle - \left[ R_{11}^{\,n} \frac{\partial w^{\,n}} {\partial x} +R_{12}^{\,n} \frac{\partial w^{\,n}} {\partial y}+R_{13}^{\,n} \frac{\partial w^{\,n}} {\partial z} \right] \text{        (production de $R_{13}^{\,n}$)} \\
\var{PRODUC(6,IEL)} = & \displaystyle - \left[ R_{13}^{\,n} \frac{\partial v^{\,n}} {\partial x} +R_{23}^{\,n} \frac{\partial v^{\,n}} {\partial y}+R_{33}^{\,n} \frac{\partial v^{\,n}} {\partial z} \right] \\
& \displaystyle - \left[ R_{12}^{\,n} \frac{\partial w^{\,n}} {\partial x} +R_{22}^{\,n} \frac{\partial w^{\,n}} {\partial y}+R_{23}^{\,n} \frac{\partial w^{\,n}} {\partial z} \right]  \text{        (production de $R_{23}^{\,n}$)}
\end{array}
$
\end{itemize}

\etape{Calcul du gradient de la masse volumique $\rho^n$ prise au d\'ebut du pas
de temps courant\footnote{{\it i.e.} calcul\'ee \`a partir des
variables du pas de temps pr\'ec\'edent $n$ si n\'ecessaire.} $(n+1)$}
Ce calcul n'a lieu que si les termes de gravit\'e doivent \^etre pris en compte
($\var{IGRARI()} =1$).
\begin{itemize}
\item [$\star$] Appel de \fort{grdcel}  pour calculer le gradient de $\rho^n$
dans les trois directions de l'espace. Les conditions aux limites sur $\rho^n$
sont des conditions de Dirichlet puisque la valeur de $\rho^n$ aux faces de bord
$ik$ (variable \var{IFAC}) est connue et vaut $\rho_{\,b_{\,ik}}$. Pour \'ecrire les conditions aux limites
sous la forme habituelle, $$\rho_{\,b_{\,ik}} = \var{COEFA} + \var{COEFB}
\,\rho^n_{\,I'}$$ on pose alors $\var{COEFA}=
\var{PROPCE(IFAC,IPPROB(IROM(IPHAS)))}$ et $\var{COEFB} = \var{VISCB} = 0$.\\
$\var{PROPCE(1,IPPROB(IROM(IPHAS)))}$ (resp.$\var{VISCB}$) est utilis\'e en lieu
et place de l'habituel \var{COEFA} ($\var{COEFB}$), lors de l'appel \`a \fort{grdcel}.\\
On a donc :\\
$\displaystyle \var{GRAROX}= \frac{\partial \rho^n}{\partial x}\ $,$\displaystyle \ \var{GRAROY}= \frac{\partial
\rho^n}{\partial y}$ et $
\displaystyle \ \var{GRAROZ}= \frac{\partial \rho^n}{\partial z}\ $.

\end{itemize}

Le gradient de $\rho^n$ servira \`a calculer les termes de production par effets de gravit\'e si ces derniers sont pris en compte.

\etape{Boucle \var{ISOU} de $1$ \`a $6$ sur les tensions de Reynolds}
Pour $\var{ISOU} = 1,2,3,4,5,6$, on r\'esout respectivement et dans
l'ordre  les
\'equations de $R_{11}$, $R_{22}$, $R_{33}$, $R_{12}$, $R_{13}$ et $R_{23}$ par
l'appel au sous-programme \fort{resrij}.\\
La r\'esolution se fait par incr\'ement $\delta {R}_{ij}^{\,n+1,k+1}$ , en utilisant la m\^eme m\'ethode que
celle d\'ecrite dans le sous-programme \fort{codits}. On adopte ici les m\^emes notations.
\var{SMBR} est le second membre du syst\`eme \`a inverser, syst\`eme portant sur
les incr\'ements de la variable. \var{ROVSDT} repr\'esente la diagonale de la
matrice, hors convection/diffusion.\\
On va r\'esoudre l'\'equation (\ref{Base_Turrij_Eq_Temp_Rij}) sous forme incr\'ementale en
utilisant \fort{codits}, soit :
\begin{equation}\label{Base_Turrij_Eq_Temp_deltaRij}
\begin{array}{ll}
&\displaystyle \underbrace{\left(\frac {\rho^n_L}{\Delta t^n}
+ \rho^n_L \,C_1\,\frac{\varepsilon^n_L}{k^n_L}(1-\frac{\delta_{ij}}{3})
 - m^n_{\,lm} + \Gamma_L\,+ max(-\alpha^n_{R_{ij}},0)\right)\,|\Omega_l|}
_{\text {$\var{ROVSDT}$ contribuant
\`a la diagonale de la matrice simplifi\'ee de \fort{matrix}}}\,(\delta{R}_{ij}^{\,n+1,p+1})_{\,L}\\\\
&  \underbrace{+\sum\limits_{m\in Vois(l)}\displaystyle \left[
 m^n_{\,lm} \delta R_{ij,\,f_{\,lm}}^{\,n+1,p+1}
- (\mu^n_{\,lm} + \gamma^n_{\,lm})\
\frac{({\delta R}_{ij}^{\,n+1,p+1})_{M}-({\delta R}_{ij}^{\,n+1,p+1})_{L})}{\overline{L'M'}}\,
S_{\,lm} \right]}_{\text { convection upwind pur et diffusion non reconstruite
relatives \`a la matrice simplifi\'ee de \fort{matrix}\footnotemark}} \\
% voir le texte de la footmark plus bas
&= - \displaystyle\frac {\rho^n_L}{\Delta t^n}\,\left(\,(R^{\,n+1,p}_{ij})_L - (R^{\,n}_{ij})_L\,\right)\\
&-\,\underbrace{\displaystyle\int_{\Omega_l} \left(
\dive\,[\,(\rho\,\vect{u})^n\,R^{\,n+1,p}_{ij} - (\mu^n\,+ \gamma^n\,)\,
\grad{R^{\,n+1,p}_{ij}}\,]\right)\,d\Omega}_{\text {convection et diffusion
trait\'ees par \fort{bilsc2}}}\\
&+\displaystyle \int_{\Omega_l} \left[\,\mathcal{P}^{\,n+1,p}_{ij} + \mathcal{G}^{\,n+1,p}_{ij}
- \displaystyle\rho^n \,C_1\,\frac{\varepsilon^n}{k^n}\left[R^{\,n+1,p}_{ij}-
\frac{2}{3}\,k^n\,\delta_{ij}\right] + \phi^{\,n+1,p}_{ij,2} +
\phi^{\,n+1,p}_{ij,w}\,\right]\, d\Omega \\
& + \displaystyle\int_{\Omega_l} \left[- \frac{2}{3} \rho^n \varepsilon^n \delta_{ij}
 + \Gamma\,(\,R^{\,in}_{ij} - R^{\,n+1,p}_{ij}\,) +
\alpha^n_{R_{ij}}\,R^{\,n+1,p}_{ij}+ \beta^n_{R_{ij}}\right]\, d\Omega\\
&+ \sum\limits_{m\in
Vois(l)}\displaystyle \left[\ \tens{E}^n\,\grad{R}^{\,n+1,p}_{ij} \right]_{\,lm}\,.\,\vect{n}_{\,lm}S_{\,lm}\\
&+ \sum\limits_{m\in Vois(l)}\displaystyle \left[\
\tens{D}^n\,\grad{R}^{\,n+1,p}_{ij} \right]_{\,lm}\,.\,\vect{n}_{\,lm}S_{\,lm}\\
&- \sum\limits_{m\in Vois(l)} \gamma^n_{\,lm} \left( \grad{R}^{\,n+1,p}_{ij}\,.\,\vect{n}_{\,lm} \right)  S_{\,lm}\\
&+ \sum\limits_{m\in Vois(l)}  m^n_{\,lm}\,(R^{\,n+1,p}_{ij})_L\\
\end{array}
\end{equation}
% si on ne fait pas comme ca, il n'apparait pas
\footnotetext[\thefootnote]{Si $\var{IRIJNU} = 1$, on remplace  $\mu^n_{\,lm}$ par $(\mu +
\mu_t)^n_{\,lm}$ dans l'expression de la diffusion non reconstruite
associ\'ee \`a la matrice simplifi\'ee de \fort{matrix} ($\mu_t$ d\'esigne la
viscosit\'e turbulente calcul\'ee comme en $k-\varepsilon$).}

o\`u on rappelle :\\
pour $n$ donn\'e entier positif, on d\'efinit la suite
 $({R}_{ij}^{\,n+1,p})_{p \in \grandN}$
 par :
\begin{equation}\notag
\left\{\begin{array}{l}
{R}_{ij}^{\,n+1,0} = {R}_{ij}^{\,n}\\
{R}_{ij}^{\,n+1,p+1} = {R}_{ij}^{\,n+1,p} + \delta{R}_{ij}^{\,n+1,p+1} \\
\end{array}\right.
\end{equation}
$(\delta{R}_{ij}^{\,n+1,p+1})_{\,L}$ d\'esigne la valeur sur l'\'el\'ement
$\Omega_l$ du $\text{$(\,p+1\,)$-i\`eme}$ incr\'ement de ${R}_{ij}^{\,n+1}$,
$ m^n_{\,lm}$ le flux de masse \`a l'instant $n$ \`a travers la face $lm$,
$\delta R_{ij,\,f_{\,lm}}^{\,n+1,p+1}$ vaut $({\delta
R}_{ij}^{\,n+1,p+1})_{L}$  si $ m^n_{\,lm} \geqslant 0$, $({\delta
R}_{ij}^{\,n+1,p+1})_{M}$ sinon,
$\mathcal{P}^{\,n+1,p}_{ij}$, $\phi^{\,n+1,p}_{ij,2}$, $\phi^{\,n+1,p}_{ij,w}$ les valeurs
des quantit\'es associ\'ees correspondant \`a l'incr\'ement
$(\delta{R}_{ij}^{\,n+1,p})$.\\



Tous ces termes sont calcul\'es comme suit :
\begin{itemize}
\item Terme de gauche de l'\'equation (\ref{Base_Turrij_Eq_Temp_deltaRij})\\
Dans \fort{resrij} est calcul\'ee la variable \var{ROVSDT}. Les autres
termes sont compl\'et\'es par \fort{codits}, lors de la construction de la matrice simplifi\'ee , {\it via} un
appel au sous-programme \fort{matrix}. La quantit\'e
 $(\mu^n_{\,lm} + \gamma^n_{\,lm})$ \`a la face $lm$ est calcul\'ee lors de l'appel \`a
\fort{visort}.\\
\item Second membre de l'\'equation (\ref{Base_Turrij_Eq_Temp_deltaRij})\\
Le premier terme non d\'etaill\'e est calcul\'e par le sous-programme
\fort{bilsc2}, qui applique le sch\'ema convectif choisi par l'utilisateur, qui
reconstruit ou non selon le souhait de l'utilisateur les gradients intervenants
dans la convection-diffusion.\\
Les termes sans accolade sont, eux, compl\`etement explicites et ajout\'es au fur et
\`a mesure dans \var{SMBR} pour former
l'expression $f^{\,exp}_s$ de \fort{codits}.
\end{itemize}
On d\'ecrit ci-dessous les \'etapes de \fort{resrij} :
\begin{itemize}

\item DELTIJ = 1, pour $\var{ISOU} \leqslant 3$ et DELTIJ = 0  Si $\var{ISOU} >
3$. Cette valeur repr\'esente le symbole de Kroeneker $\delta_{ij}$.

\item Initialisation \`a z\'ero de \var{SMBR} (tableau contenant le second
membre) et \var{ROVSDT} (tableau contenant la diagonale de la matrice sauf celle
relative \`a la contribution de la
diagonale des op\'erateurs de convection et de diffusion lin\'earis\'es
\footnote{qui correspondent aux sch\'emas convectif upwind pur et diffusif sans
reconstruction.}), tous deux de dimension $\var{NCEL}$.

\item Lecture et prise en compte des termes sources utilisateur pour la variable $R_{ij}$

Appel \`a \fort{ustsri} pour charger les termes sources utilisateurs. Ils sont
stock\'es comme suit. Pour la cellule $\Omega_l$ de centre $L$, repr\'esent\'ee par $\var{IEL}$, on a :\\
\begin{equation}\notag
\left\{\begin{array}{lll}
&\var{ROVSDT(IEL)}&= |\Omega_l| \ \alpha_{R_{ij}}\\
&\var{SMBR(IEL)}&=|\Omega_l| \ \beta_{R_{ij}}\\
\end{array}\right.
\end{equation}
On affecte alors les valeurs ad\'equates au second membre \var{SMBR} et \`a la
diagonale \var{ROVSDT} comme suit :
\begin{equation}\notag
\left\{\begin{array}{lll}
&\var{SMBR(IEL)} &= \var{SMBR(IEL)} +\ |\Omega_l| \ \alpha_{R_{ij}} \ (R^n_{ij})_L \\
&\var{ROVSDT(IEL)}&= \text{max }(-\ |\Omega_l| \ \alpha_{R_{ij}},0)\\
\end{array}\right.
\end{equation}
La valeur de $ \var{ROVSDT}$ est ainsi calcul\'ee pour des raisons de stabilit\'e
num\'erique. En effet, on ne rajoute sur la diagonale que les valeurs positives,
ce qui correspond physiquement \`a impliciter les termes de rappel uniquement.
\item{Calcul du terme source de masse  si $\Gamma_L > 0$}

Appel de \fort{catsma} et incr\'ementation si n\'ecessaire de \var{SMBR} et
\var{ROVSDT} {\it via} :\\
\begin{equation}\notag
\left\{\begin{array}{lll}
\displaystyle \var{SMBR(IEL)} = \var{SMBR(IEL)} + |\Omega_l| \ \Gamma_L \
\left[(R^{\,in}_{ij})_L - (R^{\,n}_{ij})_L \right] \\
\displaystyle \var{ROVSDT(IEL)}=\var{ROVSDT(IEL)} + |\Omega_l| \ \Gamma_L
\end{array}\right.
\end{equation}
\item Calcul du terme d'accumulation de masse et du terme instationnaire

On stocke $\displaystyle \var{W1}= \int_{\Omega_l}\dive\,(\rho\,\vect{u})\,d\Omega$
calcul\'e par \fort{divmas} \`a l'aide des flux de masse aux faces internes
$ m^n_{\,lm}=\sum\limits_{m\in Vois(l)}{(\rho \vect{u})_{\,lm}^n} \text{.}\,
\vect{S}_{\,lm} $ (tableau \var{FLUMAS}) et des flux de masse aux bords  $ m^n_{\,b_{lk}} = \sum\limits_{k\in{\gamma_b(l)}}{(\rho \vect{u})_{\,{b}_{lk}}^n} \text{.}\,
\vect{S}_{\,{b}_{lk}} $ (tableau \var{FLUMAB}).
On incr\'emente ensuite \var{SMBR} et \var{ROVSDT}.
\begin{equation}\notag
\left\{\begin{array}{lll}
&\var{SMBR(IEL)} &= \var{SMBR(IEL)} + \var{ICONV}\  (R^n_{ij})_L\,(\displaystyle
\int_{\Omega_l}\dive\,(\rho\,\vect{u})\ d\Omega) \\
&\var{ROVSDT(IEL)}& = \var{ROVSDT(IEL)} +  \var{ISTAT}\,\displaystyle
\frac{\rho^n_L \ |\Omega_l|}{\Delta t^n} -  \var{ICONV}\ (\displaystyle
\int_{\Omega_l}\dive\,(\rho\,\vect{u})\ d\Omega) \\
\end{array}\right.
\end{equation}
\item Calcul des termes sources de production, des termes $\displaystyle
\phi_{\,ij,1}+\phi_{\,ij,2}$ et de la dissipation~$\displaystyle-\frac{2}{3} \varepsilon\,\delta_{\,ij}$ :

On effectue une boucle d'indice \var{IEL} sur les cellules $\Omega_l$ de centre $L$ :
\begin{itemize}
\item [$\Rightarrow$] $\displaystyle \var{TRPROD}= \frac{1}{2} (\mathcal{P}^n_{ii})_L = \frac{1}{2} \left[ \var{PRODUC(1,IEL)} +  \var{PRODUC(2,IEL)} +  \var{PRODUC(3,IEL)} \right] $
\item [$\Rightarrow$] $\displaystyle \var{TRRIJ }= \frac{1}{2} (R^n_{ii})_L $
\item [$\Rightarrow$] $\displaystyle \var{SMBR(IEL)} =\ \var{SMBR(IEL)}\ +$\\
$\ \displaystyle\rho^n_L |\Omega_l| \left[ \displaystyle
\frac{2}{3}\,\delta_{\,ij} \left( \ \displaystyle \frac{ C_2}{2}\,(\mathcal{P}^n_{ii})_L\ +
(C_1-1)\ \varepsilon^n_L\, \right)\right.$\\
$ + \left.\ (1-C_2) \ \var{PRODUC(ISOU,IEL)} -
\displaystyle C_1\ \frac{2\,\varepsilon^n_L}{(R^n_{ii})_L}\ (R^n_{ij})_L \right]$
\item [$\Rightarrow$] $\displaystyle \var{ROVSDT(IEL)} = \var{ROVSDT(IEL)} +
\rho^n_L \ |\Omega_l| \ (- \displaystyle \frac{1}{3} \ \,\delta_{\,ij} + 1) \ C_1
\ \frac{2\ \varepsilon^n_L}{(R^n_{ii})_L}$
\end{itemize}
\item Appel de \fort{rijech} pour le calcul des termes d'\'echo de paroi
 $\phi^n_{ij,w}$ si $\var{IRIJEC()}=1$ et ajout dans \var{SMBR}.\\
$\var{SMBR} = \var{SMBR} + \phi^n_{ij,w}$\\
Suivant son mode de calcul (\var{ICDPAR}), la distance � la paroi est directement accessible
par \var{RA(IDIPAR+IEL-1)} (\var{|ICDPAR|} = 1) ou bien
est calcul\'ee \`a partir de $\var{IA(IIFAPA(IPHAS)+IEL - 1)}$,
qui donne pour l'\'el\'ement $\var{IEL}$ le num\'ero de la face de bord
paroi la plus  proche (\var{|ICDPAR|} = 2). Ces tableaux ont \'et\'e renseign\'e une fois pour toutes au
d\'ebut de calcul.

\item  Appel de \fort{rijthe} pour le calcul des termes de gravit\'e $\mathcal{G}^n_{ij}$ et ajout dans \var{SMBR}.

Ce calcul n'a lieu que si $\var{IGRARI()} = 1$.
$ \var{SMBR} = \var{SMBR} + \mathcal{G}^n_{ij}$
\item Calcul de la partie explicite du terme de diffusion
 $\dive{\,\left[\tens{A}\,\grad{R}^{\,n}_{ij}\right]}$, plus pr\'ecis\'ement
des contributions du terme extradiagonal pris aux faces purement internes
(remplissage du tableau \var{VISCF}), puis aux faces de bord (remplissage du
tableau \var{VISCB}).
\begin{itemize}
\item [$\star$] Appel de \fort{grdcel} pour le calcul du gradient de
$R^{\,n}_{ij}$ dans chaque direction. Ces gradients sont respectivement
stock\'es dans les tableaux de travail \var{W1}, \var{W2} et \var{W3}.

\item [$\star$] boucle d'indice \var{IEL} sur les cellules $\Omega_l$ de centre
$L$ pour le
calcul de $\tens{E}^n\,\grad{R}^{\,n}_{ij}$ aux cellules dans un premier temps :\\
\begin{itemize}
\item [$\Rightarrow$] $\displaystyle \var{TRRIJ}= \frac{1}{2} (R^{\,n}_{ii})_L $
\item [$\Rightarrow$] $\displaystyle \var{CSTRIJ} = \rho^n_L\ C_S \ \displaystyle\frac{(R^n_{ii})_L}{2\,\varepsilon^n_L}$
\item [$\Rightarrow$] $\displaystyle \var{W4(IEL)} = \rho^n_L\ C_S\
\displaystyle\frac{(R^n_{ii})_L}{2\,\varepsilon^n_L} \left[\,(R^{\,n}_{12})_L \ \var{W2(IEL)} +
(R^{\,n}_{13})_L \ \var{W3(IEL)}\,\right]$
\item [$\Rightarrow$] $\displaystyle \var{W5(IEL)} = \rho^n_L\ C_S\
\displaystyle\frac{(R^n_{ii})_L}{2\,\varepsilon^n_L} \left[\,(R^{\,n}_{12})_L \ \var{W1(IEL)} +
(R^{\,n}_{23})_L \ \var{W3(IEL)}\,\right]$
\item [$\Rightarrow$] $\displaystyle \var{W6(IEL)} = \rho^n_L\ C_S\
\displaystyle\frac{(R^n_{ii})_L}{2\,\varepsilon^n_L} \left[\,(R^{\,n}_{13})_L \ \var{W1(IEL)} + (R^{\,n}_{23})_L \ \var{W2(IEL)}\,\right]$
\end{itemize}



\item [$\star$] Appel de \fort{vectds}\footnote{Le r\'esultat est stock\'e dans
\var{VISCF} et \var{VISCB}. Dans \fort{vectds}, les valeurs aux faces internes
sont interpol\'ees lin\'eairement sans reconstruction et \var{VISCB} est mis \`a
z\'ero.} pour assembler $\displaystyle\left[ \tens{E}^n\,\grad{R}^{\,n}_{ij}
\right]\,.\,\vect{n}_{\,lm}S_{\,lm}$ aux faces $lm$.
\item [$\star$] Appel de \fort{divmas} pour calculer la divergence du flux d\'efini par \var{VISCF} et \var{VISCB}.
Le r\'esultat est stock\'e dans \var{W4}.\\
Ajout au second membre \var{SMBR}.\\
\var{SMBR} = \var{SMBR} + \var{W4}
\end{itemize}

A l'issue de cette \'etape, seule la partie extradiagonale de la diffusion prise
enti\`erement explicite~:
 $$\sum\limits_{m\in
Vois(l)}\left[\ \tens{E}^n\,\grad{R}^{\,n}_{ij} \right]_{\,lm}\,.\,\vect{n}_{\,lm}S_{\,lm}$$ a \'et\'e calcul\'ee.\\

\item Calcul de la partie diagonale du terme de diffusion\footnote{Seule la
composante normale  du  gradient de $R_{ij}$ aux faces sera implicite.} :\\
Comme on l'a d\'eja vu, une partie de cette contribution sera trait\'ee en
implicite (celle relative \`a la composante normale du gradient) et donc
ajout\'ee au second membre par \fort{bilsc2} ; l'autre
partie sera explicite et prise en compte dans $f_s^{\,exp}$.
\begin{itemize}
\item [$\star$] On effectue une boucle d'indice \var{IEL} sur les cellules
$\Omega_l$ de centre $L$ :
\begin{itemize}
\item [$\Rightarrow$] $\displaystyle \var{TRRIJ }= \frac{1}{2} (R^{\,n}_{ii})_L $
\item [$\Rightarrow$] $\displaystyle \var{CSTRIJ} = \rho^n_L \ C_S \ \frac{(R^{\,n}_{ii})_L}{2\,\varepsilon^n_L}$
\item [$\Rightarrow$] $\displaystyle \var{W4(IEL)} = \rho^n_L \ C_S \
\frac{(R^{\,n}_{ii})_L}{2\,\varepsilon^n_L} \ (R^{\,n}_{11})_L$
\item [$\Rightarrow$] $\displaystyle \var{W5(IEL)} = \rho^n_L \ C_S \ \frac{(R^{\,n}_{ii})_L}{2\,\varepsilon^n_L}\ (R^n_{22})_L$
\item [$\Rightarrow$] $\displaystyle \var{W6(IEL)} = \rho^n_L \ C_S \ \frac{(R^{\,n}_{ii})_L}{2\,\varepsilon^n_L} \ (R^n_{33})_L$
\end{itemize}

%\item Traitement du parall\'elisme et de la p\'eriodicit\'e.

\item [$\star$] On effectue une boucle d'indice \var{IFAC} sur les faces
purement internes $lm$ pour remplir le tableau \var{VISCF} :
\begin{itemize}
\item [$\Rightarrow$] Identification des cellules $\Omega_l$ et $\Omega_m$ de
centre respectif $L$ (variable \var{II}) et $M$ (variable \var{JJ}), se trouvant de chaque c\^ot\'e de la face
$lm$\footnote{La normale \`a la face est orient\'ee de L vers M.}.
\item [$\Rightarrow$] Calcul du carr\'e de la surface de la face. La valeur est
stock\'ee dans le tableau \var{SURFN2}.
\item [$\Rightarrow$] Interpolation du gradient de $R^{\,n}_{ij}$ \`a la face
$lm$ (gradient facette $\left[\grad{R}^{\,n}_{ij}\right]_{\,lm}$) :
\begin{equation}\notag
\left\{\begin{array}{ll}
\var{GRDPX} &= \displaystyle \frac{1}{2} \left(\var{W1(II)} + \var{W1(JJ)}
\right) \\
&\\
\var{GRDPY} &= \displaystyle \frac{1}{2} \left(\var{W2(II)} + \var{W2(JJ)} \right) \\
&\\
\var{GRDPZ} &= \displaystyle \frac{1}{2} \left(\var{W3(II)} + \var{W3(JJ)} \right)
\end{array}\right.
\end{equation}
\item [$\Rightarrow$] Calcul du gradient de $R^{\,n}_{ij}$ normal \`a la face
$lm$, $\left[\grad{R}^{\,n}_{ij}\right]_{\,lm}.\vect{n}_{\,lm}\,S_{\,lm}$.\\

$\displaystyle \var{GRDSN} =  \var{GRDPX} \ \var{SURFAC(1,IFAC)} + \var{GRDPY} \ \var{SURFAC(2,IFAC)} +  \var{GRDPZ} \ \var{SURFAC(3,IFAC)}$
$\var{SURFAC}$ est le vecteur surface de la face \var{IFAC}.


\item [$\Rightarrow$] calcul de
 $\left[\grad{R^{\,n}_{ij}} - (\grad
R^{\,n}_{ij}\,.\,\vect{n}_{\,lm})\vect{n}_{\,lm}\right]$, les vecteurs \'etant
calcul\'es \`a la face $lm$ :
\begin{equation}\notag
\left\{\begin{array}{lll}
&\displaystyle \var{GRDPX} &= \var{GRDPX} - \displaystyle\frac{\var{GRDSN}}{\var{SURFN2}} \ \var{SURFAC(1,IFAC)}\\
&&\\
&\displaystyle \var{GRDPY} &= \var{GRDPY} - \displaystyle\frac{\var{GRDSN}}{\var{SURFN2}} \ \var{SURFAC(2,IFAC)} \\
&&\\
&\displaystyle \var{GRDPZ} &= \var{GRDPZ} - \displaystyle\frac{\var{GRDSN}}{\var{SURFN2}} \ \var{SURFAC(3,IFAC)}
\end{array}\right.
\end{equation}
\item [$\Rightarrow$] finalisation du calcul de l'expression totalement
explicite
 $$\left[ \tens{D}^n\,\left( \grad{R^{\,n}_{ij}} - (\grad R^{\,n}_{ij}\,.\,\vect{n}_{\,lm})\,\vect{n}_{\,lm}\right) \right]\,.\,\vect{n}_{\,lm}$$
\begin{equation}\notag
\begin{array} {ll}
\displaystyle \var{VISCF} = &
 \displaystyle\frac{1}{2} (\ \var{W4(II)} +\ \var{W4(JJ)}) \ \var{GRDPX} \
\var{SURFAC(1,IFAC)})\ + \\
&\\
&  \displaystyle\frac{1}{2} (\ \var{W5(II)} +\ \var{W5(JJ)}) \ \var{GRDPY} \
\var{SURFAC(2,IFAC)})\ + \\
&\\
&  \displaystyle\frac{1}{2} (\ \var{W6(II)} +\ \var{W6(JJ)}) \ \var{GRDPZ} \ \var{SURFAC(3,IFAC)})
\end{array}
\end{equation}
\end{itemize}

\item [$\star$] Mise \`a z\'ero du tableau \var{VISCB}.

\item [$\star$] Appel de \fort{divmas} pour calculer la divergence de~:
 $$\tens{D}^{\,n}\,\left( \grad{R^{\,n}_{ij}} - (\grad R^{\,n}_{ij}\,.\,\vect{n}_{\,lm})\vect{n}_{\,lm}\right)$$ d\'efini aux faces dans \var{VISCF} et \var{VISCB}.

Le r\'esultat est stock\'e dans le tableau \var{W1}.\\
Ajout au second membre \var{SMBR}.\\
$\var{SMBR} = \var{SMBR} + \var{W1}$
\end{itemize}
\item Calcul de la viscosit\'e orthotrope $\gamma^n_{\,lm}$ \`a la face $lm$ de la variable principale
$R^{\,n}_{ij}$\\
Ce calcul permet au sous-programme \fort{codits} de compl\'eter le second membre
\var{SMBR} par :
\begin{equation}
\begin{array} {ll}
& \sum\limits_{m\in Vois(l)}
\mu^n_{\,lm}\,\left(\grad{R}^{\,n}_{ij}\,.\,\vect{n}_{\,lm}\right)S_{\,lm}
 + \sum\limits_{m\in Vois(l)} \left(\grad{R}^{\,n}_{ij}
\,.\,\vect{n}_{\,lm}\right)\left[\tens{D}^{\,n}\,\vect{n}_{\,lm}\right]_{\,lm}\,.\,\vect{n}_{\,lm}\
S_{\,lm}\\
& = \sum\limits_{m\in Vois(l)}(\,\mu^n_{\,lm}\, + \,\gamma^n_{\,lm}\,)\,\left(\grad{R}^{\,n}_{ij}\,.\,\vect{n}_{\,lm}\right)S_{\,lm}
\end{array}
\end{equation}
sans pr\'eciser la nature de la face $lm$, {\it via} l'appel \`a \fort{bilsc2}  et de disposer de la quantit\'e
$(\mu^n_{\,lm}\, + \gamma^n_{\,lm})$ pour construire sa
matrice simplifi\'ee.\\
\begin{itemize}
\item [$\star$] On effectue une boucle d'indice \var{IEL} sur les cellules
$\Omega_l$ :
\begin{itemize}
\item [$\Rightarrow$] $\displaystyle \var{TRRIJ }= \frac{1}{2} (R^{\,n}_{ii})_L $
\item [$\Rightarrow$] $\displaystyle \var{RCSTE} = \rho^n_L \ C_S \ \frac{ (R^{\,n}_{ii})_L}{2\,\varepsilon^n_L} $
\item [$\Rightarrow$] $\displaystyle \var{W1(IEL)} = \mu^n +\rho^n_L \ C_S \ \frac{
(R^{\,n}_{ii})_L}{2\,\varepsilon^n_L}\ (R^n_{11})_L$
\item [$\Rightarrow$] $\displaystyle \var{W2(IEL)} = \mu^n + \rho^n_L \ C_S \ \frac{ (R^{\,n}_{ii})_L}{2\,\varepsilon^n_L}\ (R^n_{22})_L$
\item [$\Rightarrow$] $\displaystyle \var{W3(IEL)} = \mu^n + \rho^n_L \ C_S \ \frac{ (R^{\,n}_{ii})_L}{2\,\varepsilon^n_L}\ (R^n_{33})_L$
\end{itemize}

\item [$\star$] Appel de \fort{visort} pour calculer la viscosit\'e orthotrope
\footnote{Comme dans le sous-programme \fort{viscfa}, on multiplie la viscosit\'e par
$\displaystyle \frac{S_{\,lm}}{\overline{L'M'}}$, o\`u $S_{\,lm}$ et
$\overline{L'M'}$ repr\'esentent respectivement la surface de la face $lm$ et la
mesure alg\'ebrique du segment reliant les projections des centres des cellules
voisines sur la normale \`a la face. On garde dans ce sous-programme  la possibilit\'e d'interpoler la viscosit\'e aux cellules lin\'eairement ou d'utiliser une moyenne harmonique. La viscosit\'e au bord est celle de la cellule de bord correspondante.}
$ \gamma^n_{\,lm} = (\tens{D}^{\,n}\,\vect{n}_{\,lm}).\vect{n}_{\,lm}$ aux faces $lm$

Le r\'esultat est stock\'e dans les tableaux \var{VISCF} et \var{VISCB}.
\end{itemize}

\item appel de \fort{codits} pour la r\'esolution de l'\'equation de
convection/diffusion/termes sources de la variable $R_{ij}$. Le terme source est
\var{SMBR}, la viscosit\'e \var{VISCF} aux faces purement internes (
resp. \var{VISCB} aux faces de bord) et \var{FLUMAS} le flux de masse interne
 ( resp. \var{FLUMAB} flux de masse au bord). Le r\'esultat est la variable $R_{ij}$ au temps
$n+1$, donc $R^{\,n+1}_{ij}$. Elle est stock\'ee dans le tableau \var{RTP} des
variables mises \`a jour.

\end{itemize}

\etape{Appel de \fort{reseps} pour la r\'esolution de la variable $\varepsilon$}

On d\'ecrit ci-dessous le sous-programme \fort{reseps}, les commentaires du sous-programme \fort{resrij} ne seront pas r\'ep\'et\'es puisque les deux sous-programmes ne diff\`erent que par leurs termes sources.

\begin{itemize}
\item Initialisation \`a z\'ero de \var{SMBR} et \var{ROVSDT}.

\item{Lecture et prise en compte des termes sources utilisateur pour la variable $\varepsilon$ :}

Appel de \fort{ustsri} pour charger les termes sources utilisateurs. Ils sont
stock\'es dans les tableaux suivants :\\
pour la cellule $\Omega_l$ repr\'esent\'ee par $\var{IEL}$ de centre $L$, on a :
\begin{equation}\notag
\left\{\begin{array}{lll}
&\var{ROVSDT(IEL)}&= |\Omega_l| \ \alpha_{\varepsilon}\\
&\var{SMBR(IEL)}&=|\Omega_l| \ \beta_{\varepsilon}\\
\end{array}\right.
\end{equation}
On affecte alors les valeurs ad\'equates au second membre \var{SMBR} et \`a la
diagonale \var{ROVSDT} comme suit :
\begin{equation}\notag
\left\{\begin{array}{lll}
&\var{SMBR(IEL)} &= \var{SMBR(IEL)} +\ |\Omega_l| \ \alpha_{\,\varepsilon} \
\varepsilon^n_L \\
&\var{ROVSDT(IEL)}&= \text{max }(-\ |\Omega_l| \ \alpha_{\,\varepsilon},0)\\
\end{array}\right.
\end{equation}

\item{Calcul du terme source de masse si $\Gamma_L > 0$ :
\begin{equation}\notag
\left\{\begin{array}{lll}
&\displaystyle \var{SMBR(IEL)} = \var{SMBR(IEL)} + |\Omega_l| \ \Gamma_L \
(\varepsilon^{\,in}_L -\varepsilon^n_L) \\
&\displaystyle \var{ROVSDT(IEL)}= \var{ROVSDT(IEL)} + |\Omega_l| \ \Gamma_L
\end{array}\right.
\end{equation}
\item Calcul du terme d'accumulation de masse et du terme instationnaire \\
On stocke $\displaystyle \var{W1}= \int_{\Omega_l}\dive\,(\rho\,\vect{u})\,d\Omega$
calcul\'e par \fort{divmas} \`a l'aide des flux de masse internes et aux bords.\\
On incr\'emente ensuite \var{SMBR} et \var{ROVSDT}.
\begin{equation}\notag
\left\{\begin{array}{lll}
&\var{SMBR(IEL)} &= \var{SMBR(IEL)} + \var{ICONV}\ \varepsilon^n_L\,(\displaystyle
\int_{\Omega_l}\dive\,(\rho\,\vect{u})\ d\Omega) \\
&\var{ROVSDT(IEL)}& = \var{ROVSDT(IEL)} +  \var{ISTAT}\,\displaystyle
\frac{\rho^n_L \ |\Omega_l|}{\Delta t^n} -  \var{ICONV}\ (\displaystyle
\int_{\Omega_l}\dive\,(\rho\,\vect{u})\ d\Omega) \\
\end{array}\right.
\end{equation}

\item Traitement du terme de production
 $\displaystyle \rho\,C_{\varepsilon_1}\,\frac{\varepsilon}{k}\,\mathcal{P}$
 et du terme de dissipation $-\,\displaystyle \rho\,C_{\varepsilon_2}\,\frac{\varepsilon}{k}\,\varepsilon$ \\
pour cela, on effectue une boucle d'indice \var{IEL} sur les cellules $\Omega_l$
de centre $L$ :
\begin{itemize}
\item [$\Rightarrow$] $\displaystyle \var{TRPROD}= \frac{1}{2} (\mathcal{P}^n_{ii})_L = \frac{1}{2} \left[ \var{PRODUC(1,IEL)} +  \var{PRODUC(2,IEL)} +  \var{PRODUC(3,IEL)} \right] $
\item [$\Rightarrow$] $\displaystyle \var{TRRIJ }= \frac{1}{2} (R^n_{ii})_L $
\item [$\Rightarrow$] $\displaystyle \var{SMBR(IEL)} = \var{SMBR(IEL)} + \rho^n_L
|\Omega_l| \left[ -C_{\varepsilon_2} \ \frac{2\,(\varepsilon^n_L)^2}{(R^n_{ii})_L} + C_{\varepsilon_1} \ \frac{\varepsilon^n_L}{(R^n_{ii})_L}\ (\mathcal{P}^n_{ii})_L \right] $
\item [$\Rightarrow$] $\displaystyle \var{ROVSDT(IEL)} = \var{ROVSDT(IEL)} + C_{\varepsilon_2} \ \rho^n_L \ |\Omega_l| \ \frac{2\,\varepsilon^n_L}{(R^n_{ii})_L}$
\end{itemize}

\item Appel de \fort{rijthe} pour le calcul des termes de gravit\'e $\mathcal{G}^n_{\varepsilon}$ et ajout dans \var{SMBR}.

$ \var{SMBR} = \var{SMBR} + \mathcal{G}^n_{\varepsilon}$\\
Ce calcul n'a lieu que si $\var{IGRARI()} = 1$.

\item Calcul de la diffusion de $\varepsilon$ \\
 Le terme $\dive \left[\mu\, \grad(\varepsilon) + \tens{A'}\,\grad(\varepsilon)
\right]$ est calcul\'e exactement de la m\^eme mani\`ere que pour les tensions
de Reynolds $R_{ij}$ en rempla\c cant $\tens{A}$ par $\tens{A'}$.

\item Appel de \fort{codits} pour la r\'esolution de l'\'equation de
convection/diffusion/termes sources de la variable principale $\varepsilon$. Le
r\'esultat $\varepsilon^{\,n+1}$ est stock\'e dans le tableau \var{RTP} des
variables mises \`a jour.
}
\end{itemize}

\etape{clippings finaux}
On passe enfin dans le sous-programme  \fort{clprij} pour faire un clipping \'eventuel
des variables $R^{\,n+1}_{ij}$ et $\varepsilon^{\,n+1}$. Le sous-programme
\fort{clprij} est appel\'e\footnote{L'option
$\var{ICLIP} = 1$ consiste en un clipping minimal des variables $R_{ii}$ et
$\varepsilon$ en prenant la valeur absolue de ces variables puisqu'elles ne
peuvent \^etre que positives.} avec $\var{ICLIP} = 2$ . Cette option
\footnote{Quand la valeur des grandeurs $R_{ii}$ ou $\varepsilon$ est
n\'egative, on la remplace par le minimum entre sa valeur absolue et (1,1)
fois la valeur obtenue au pas de temps pr\'ec\'edent.} contient l'option $\var{ICLIP} = 1$  et permet de v\'erifier l'in\'egalit\'e de Cauchy-Schwarz sur les grandeurs extra-diagonales du tenseur $\tens{R}$ (pour $i \neq j$, $|R_{ij}|^2 \le R_{ii} R_{jj}$).


%%%%%%%%%%%%%%%%%%%%%%%%%%%%%%%%%%
%%%%%%%%%%%%%%%%%%%%%%%%%%%%%%%%%%
\section{Points \`a traiter}
%%%%%%%%%%%%%%%%%%%%%%%%%%%%%%%%%%
%%%%%%%%%%%%%%%%%%%%%%%%%%%%%%%%%%
Sauf mention explicite, $\phi$ repr\'esentera une tension de Reynolds ou la dissipation turbulente ($\phi = R_{ij} \ \text{ou} \ \varepsilon$).

\begin{itemize}
\item {La vitesse utilis\'ee pour le calcul de la production est explicite. Est-ce qu'une implicitation peut am\'eliorer la pr\'ecision temporelle de $\phi$ \footnote{Cette remarque peut \^etre g\'en\'eralis\'ee. En effet, peut-on envisager d'actualiser les variables d\'ej\`a r\'esolues (sans r\'eactualiser les variables turbulentes apr\`es leur r\'esolution)? Ceci obligerait \`a modifier les sous-programmes tels que \fort{condli} qui sont appel\'es au d\'ebut de la boucle en temps.} ?}
\item {Dans quelle mesure le terme d'\'echo de paroi est-il valide ? En effet, ce terme est remis en question par certains auteurs.}
\item {On peut envisager la r\'esolution d'un syst\`eme hyperbolique pour les
tensions de Reynolds afin d'introduire un couplage avec le champ de vitesse.}
\item {Le flux au bord \var{VISCB} est annul\'e dans le sous-programme
\fort{vectds}. Peut-on envisager de mettre au bord la valeur de la variable
concern\'ee \`a la cellule de bord correspondant? De m\^eme, il faudrait se
pencher sur les hypoth\`eses sous-jacentes \`a l'annulation des contributions
aux bords de \var{VISCB} lors du calcul de : $$\left[ \tens{D}^n\,\left( \grad{R^{\,n}_{ij}} - (\grad R^{\,n}_{ij}\,.\,\vect{n}_{\,lm})\,\vect{n}_{\,lm}\right) \right]\,.\,\vect{n}_{\,lm}.$$}
\item {Un probl\`eme de pond\'eration appara\^\i t plus g\'en\'eralement. Si on prend la partie explicite de $\tens{D}\,\grad(\phi)$, la pond\'eration aux faces internes utilise le coefficient $\displaystyle\frac{1}{2}$ avec pond\'eration s\'epar\'ee de $\tens{D}$ et $\grad(\phi)$, alors que pour $\tens{E}\,\grad(\phi)$, on calcule d'abord ce terme aux cellules pour ensuite l'interpoler lin\'eairement aux faces \footnote{Cette interpolation se fait dans \fort{vectds} avec des coefficients de pond\'eration aux faces.}. Ceci donne donc deux types d'interpolations pour des termes de m\^eme nature.}
\item {On laisse la possibilit\'e dans \fort{visort} d'utiliser une moyenne
harmonique aux faces. Est-ce que ceci est valable puisque les interpolations
utilis\'ees lors du calcul de la partie explicite de $\tens{A}\,\grad{\phi}$
sont des moyennes arithm\'etiques ?}
\item {Les techniques adopt\'ees lors du clipping sont \`a revoir.}
\item {On utilise dans le cadre du mod\`ele $\displaystyle R_{ij}-\varepsilon $ une semi-implicitation de termes comme $\displaystyle \phi_{ij,1}$ ou $\displaystyle -\rho\,C_{\varepsilon_2}\,\frac{\varepsilon}{k}\,\varepsilon$. On peut envisager le m\^eme type d'implicitation dans \fort{turbke} m\^eme en pr\'esence du couplage $\displaystyle k-\varepsilon$.}
\item L'adoption d'une r\'esolution d\'ecoupl\'ee fait perdre l'invariance par rotation.
\item La formulation et l'implantation des conditions aux limites de paroi
devront \^etre v\'erifi\'ees. En effet, il semblerait que, dans certains cas, des ph\'enom\`enes
``oscillatoires'' apparaissent, sans qu'il soit ais\'e d'en d\'eterminer la cause.
\item L'implicitation partielle (du fait de la r\'esolution d\'ecoupl\'ee) des
conditions aux limites conduit souvent \`a des calculs instables. Il
conviendrait d'en conna\^\i tre la raison. L'implicitation partielle avait
\'et\'e mise en \oe uvre afin de tenter d'utiliser un pas de temps plus grand
dans le cas de jets axisym\'etriques en particulier.

\end{itemize}

%                      Code_Saturne version 1.3
%                      ------------------------
%
%     This file is part of the Code_Saturne Kernel, element of the
%     Code_Saturne CFD tool.
%
%     Copyright (C) 1998-2007 EDF S.A., France
%
%     contact: saturne-support@edf.fr
%
%     The Code_Saturne Kernel is free software; you can redistribute it
%     and/or modify it under the terms of the GNU General Public License
%     as published by the Free Software Foundation; either version 2 of
%     the License, or (at your option) any later version.
%
%     The Code_Saturne Kernel is distributed in the hope that it will be
%     useful, but WITHOUT ANY WARRANTY; without even the implied warranty
%     of MERCHANTABILITY or FITNESS FOR A PARTICULAR PURPOSE.  See the
%     GNU General Public License for more details.
%
%     You should have received a copy of the GNU General Public License
%     along with the Code_Saturne Kernel; if not, write to the
%     Free Software Foundation, Inc.,
%     51 Franklin St, Fifth Floor,
%     Boston, MA  02110-1301  USA
%
%-----------------------------------------------------------------------
%
\programme{vortex}
%
\vspace{1cm}
%%%%%%%%%%%%%%%%%%%%%%%%%%%%%%%%%%
%%%%%%%%%%%%%%%%%%%%%%%%%%%%%%%%%%
\section{Fonction}
%%%%%%%%%%%%%%%%%%%%%%%%%%%%%%%%%%
%%%%%%%%%%%%%%%%%%%%%%%%%%%%%%%%%%
Ce sous-programme est d�di� � la g�n�ration des conditions d'entr�e
turbulente utilis�es en LES.


La m�thode des vortex est bas�e sur une approche de tourbillons
ponctuels. L'id�e de la m�thode consiste � injecter des tourbillons 2D dans le
plan d'entr�e du calcul, puis � calculer le champ de vitesse induit par ces
tourbillons au centre des faces d'entr�e.

%                      Code_Saturne version 1.3
%                      ------------------------
%
%     This file is part of the Code_Saturne Kernel, element of the
%     Code_Saturne CFD tool.
% 
%     Copyright (C) 1998-2007 EDF S.A., France
%
%     contact: saturne-support@edf.fr
% 
%     The Code_Saturne Kernel is free software; you can redistribute it
%     and/or modify it under the terms of the GNU General Public License
%     as published by the Free Software Foundation; either version 2 of
%     the License, or (at your option) any later version.
% 
%     The Code_Saturne Kernel is distributed in the hope that it will be
%     useful, but WITHOUT ANY WARRANTY; without even the implied warranty
%     of MERCHANTABILITY or FITNESS FOR A PARTICULAR PURPOSE.  See the
%     GNU General Public License for more details.
% 
%     You should have received a copy of the GNU General Public License
%     along with the Code_Saturne Kernel; if not, write to the
%     Free Software Foundation, Inc.,
%     51 Franklin St, Fifth Floor,
%     Boston, MA  02110-1301  USA
%
%-----------------------------------------------------------------------
%
%%%%%%%%%%%%%%%%%%%%%%%%%%%%%%%%%%
%%%%%%%%%%%%%%%%%%%%%%%%%%%%%%%%%%
\section{Discr\'etisation}
%%%%%%%%%%%%%%%%%%%%%%%%%%%%%%%%%%
%%%%%%%%%%%%%%%%%%%%%%%%%%%%%%%%%%

Le terme convectif en $\dive(\underline{u} \otimes \rho\,\underline{u})$
introduit une non lin\'earit\'e et un couplage des composantes de la vitesse
$\vect{u}$ dans l'�quation (\ref{Base_Preduv_eqqdm}). Une lin\'earisation et un d\'ecouplage
des trois composantes de la 
vitesse sont r\'ealis\'es lors de la discr\'etisation de cette \'etape de
pr\'ediction.\\
En effet, soit :
\begin{equation}
\vect{\widetilde{u}}= \vect{u}^n + \delta \vect{u} 
\end{equation}
La contribution exacte du terme convectif \`a prendre en compte dans cette
\'etape de pr\'ediction serait :\\
\begin{equation}\label{Base_Preduv_Conv_exact}
\begin{array}{ll}
\dive(\vect{\widetilde{u}} \otimes \rho\,\vect{\widetilde{u}}) =
\dive(\vect{u}^{n} \otimes \rho\,\vect{u}^{n}) + \dive(\delta \vect{u} \otimes
\rho\,\vect{u}^{n}) +  \underbrace { \dive(\vect{u}^{n} \otimes
\rho\,\delta \vect{u})}_{\text {terme couplant lin\'eaire}} +  \underbrace { \dive(\delta \vect{u} \otimes
\rho\,\delta \vect{u})}_{\text {terme couplant et non lin\'eaire}}\\
\end{array} 
\end{equation}
Les deux derniers termes de l'expression (\ref{Base_Preduv_Conv_exact}) sont {\em a priori} n�glig�s
de mani�re � obtenir un syst\`eme en vitesse qui soit d\'ecoupl\'e et donc,
�viter l'inversion d'une matrice pouvant \^etre de tr\`es grande taille. Ces
deux termes peuvent n�anmoins �tre pris en compte de mani�re plus ou moins
approch�e par extrapolation explicite du flux de masse en $n+\theta_F$ (pour le
terme couplant lin�aire provenant de la convection de $\vect{u}^{n}$ par $\delta
\vect{u}$) et utilisation d'un point-fixe par sous it�ration sur le sous
programme \fort{navsto} (pour le terme non-lin�aire, en sp�cifiant $\var{NTERUP}>1$).

L'�quation (\ref{Base_Preduv_eqqdm}) est discr�tis�e au temps $n+\theta$ � l'aide d'un
$\theta$-sch�ma, et le tenseur des pertes de charges d�compos� en une partie
diagonale $\tens{K}_{d}$ et une extradiagonale $\tens{K}_{e}$ (soit
 $\tens{K}_{pdc}=\tens{K}_{d}+\tens{K}_{e}$).\\
$\bullet$ La pression est suppos�e connue en $n-1+\theta$ (d�calage temporel
pression-vitesse) et le gradient naturellement calcul� � cet instant.\\ 
$\bullet$ Les termes sources de viscosit� secondaire, de gradient transpos\'e,
ceux provenant du mod�le de turbulence\footnote{except� $\dive (\mu_t\ (\ggrad
\underline {u}))$}, $\rho\,\tens{K}_{\,e}\ \underline{u}$, $(\rho -\rho_0)
\underline {g}$ ainsi que $\underline{T}_{s}^{\,exp}$ et
$\Gamma\,\underline{u}_{\,i}$ sont pris de mani�re explicite au temps $n$, ou
extrapol�s suivant le sch�ma en temps choisi pour les propri�t�s physique et les
termes sources.\\ 
$\bullet$ Les termes sources $\underline{u}\,\,\dive (\rho\,\underline {u})$,
$\Gamma\,\,\underline{u}$, $T_{s}^{\,imp}\,\,\underline{u}$ et
$-\rho\,\tens{K}_{\,d}\,\,\underline{u}$ sont implicit�s est calcul�s �
l'instant $n+\theta$.\\ 
$\bullet$ Le terme de diffusion $\dive (\mu_{\,tot}\,\ggrad \underline{u})$ est
implicit� : la vitesse est prise � l'instant $n+\theta$ et la viscosit�
explicit�e ou extrapol�e.\\ 
$\bullet$ Enfin, le terme de convection en $\dive(\,\underline{u} \otimes
(\rho\underline{u})\,)$ est implicit� : la composante r�solue de la vitesse est
prise en $n+\theta$, et le flux de masse, explicit�, ou extrapol� en
$n+\theta_F$. 

Par souci de clart�, on suppose, en l'absence d'indication, que les propri�tes
physiques $\Phi$ ($\rho,\,\mu_{tot},\,...$) et le flux de masse
$(\rho\underline{u})$ sont pris respectivement aux instants $n+\theta_\Phi$ et
$n+\theta_F$, o� $\theta_\Phi$ et $\theta_F$ d�pendent des sch�mas en temps
sp�cifiquement utilis�s pour ces grandeurs\footnote{cf. \fort{introd}}. 

La discr�tisation temporelle de l'�quation (\ref{Base_Preduv_eqqdm}) s'�crit alors comme suit : 

\begin{equation}\label{Base_Preduv_eq_di1}
 \begin{array}{c}
\displaystyle \rho\,\ \frac{ \underline {\widetilde{u}}^{n+1} -\underline {u}^{n} }
{\Delta t} + \dive(\,\underline{\widetilde{u}}^{n+\theta} \otimes (\rho\underline{u})\,) -\dive
(\mu_{\,tot}\,\ggrad \underline{\widetilde{u}}^{n+\theta}) =
\\
\displaystyle
 - \grad p^{n-1+\theta} + \dive (\rho\,\underline {u})\,\underline{\widetilde{u}}^{n+\theta} +(\Gamma\,\underline{u}_{\,i})^{n+\theta_S}-\Gamma^n\,\,\underline{\widetilde{u}}^{n+\theta}
\\
\begin{array}{c}
\displaystyle
- \rho\,\tens{K}_{\,d}^{n}\,\,\underline{\widetilde{u}}^{n+\theta} - (\rho\,\tens{K}_{\,e}\ \underline{u})^{n+\theta_S} + (\underline{T}_{s}^{\,exp})^{\,n+\theta_S} + T_{s}^{\,imp}\,\,\underline{\widetilde{u}}^{n+\theta}
\\
\displaystyle
+[\dive (\mu_{\,tot}\,^t\ggrad \underline {u})]^{n+\theta_S}-\frac {2} {3}[\,\grad (\mu_{\,tot}\,\dive \underline {u})]^{n+\theta_S} + (\rho -\rho_0) \underline {g}
 - (\underline{turb})^{n+\theta_S}
\end{array}
\end{array}
\end{equation}
o\`u, par souci de simplification, on a pos\'e :
\begin{equation}
\mu_{\,tot}=
\begin{cases}
\mu+\mu_t & \text{pour les mod�les � viscosit� turbulente ou en LES}, \\
\mu & \text{pour les mod�les au second ordre ou en laminaire}
\end{cases} \ 
\end{equation}
\\
et :
\begin{equation}
\underline{turb}^{n}=
\begin{cases}
\displaystyle\frac {2}{3}\grad (\rho^{n}\,k^{n}) & \text{pour les mod�les � viscosit� turbulente}, \\
\dive(\rho^{n}\,\tens{R}^n) & \text{pour les mod�les au second ordre},\\
0 & \text{en laminaire ou en LES}\\
\end{cases}
\end{equation}
Par analogie avec l'�criture du $\theta$-sch�ma pour une variable scalaire, $\,
\underline {\widetilde{u}}^{n+\theta}$ est interpol�e � partir de la vitesse
pr�dite $\underline {\widetilde{u}}^{n+1}$ de la mani\`ere suivante\footnote{si
$\theta=1/2$, ou qu'une extrapolation est utilis�e, l'ordre 2 n'est obtenu que si
le pas de temps $\Delta t$ est uniforme en temps et en espace.}~: 
\begin{equation}
\underline {\widetilde{u}}^{n+\theta}=\theta\, \underline
{\widetilde{u}}^{n+1}+(1-\theta)\, \underline {u}^{n}\\ 
\end{equation}
Avec :
\begin{equation}
\left\{
\begin{array}{ll}
\theta = 1   & \text{Pour un sch\'ema de type Euler implicite d'ordre 1.}\\
\theta = 1/2 & \text{Pour un sch\'ema de type Cranck-Nicolson d'ordre 2.}\\
\end{array}
\right.
\end{equation}

L'�quation (\ref{Base_Preduv_eq_di1}) est alors r��crite sous la forme :

\begin{equation}\label{Base_Preduv_eq_di2}
\begin{array}{c}
\displaystyle \underbrace{\left(\frac{\rho}{\Delta t} -\theta \,\dive (\rho\,\underline {u}) +\theta \,\, \Gamma^n +
\theta \,\, \rho\,\tens{K}_{\,d}^n-\theta \,T_s^{\,imp} \right)}_{\displaystyle f_s^{imp}}\, (\underline {\,\widetilde{u}}^{n+1} -\underline {u}^{n})
\\
 +\, \theta\, \dive(\underline {\widetilde{u}}^{n+1} \otimes (\rho\underline{u}))-\, \theta\,\dive (\mu_{\,tot}\,\ggrad \underline {\widetilde{u}}^{n+1}) =
\\
-\,(1-\theta)\, \dive(\underline {u}^{n} \otimes (\rho\underline{u})) +\,(1-\theta)\,\dive (\mu_{\,tot}\,\ggrad \underline {u}^{n})
\\
f_s^{exp}\left\{
\begin{array}{c}
\displaystyle 
- \grad p^{n-1+\theta} + \dive (\rho\,\underline {u})\,\underline{u}^{n} +\,(\,\Gamma^{n}\,\underline{u}_{\,i}\,)^{n+\theta_S}- \Gamma^n\,\,\underline{u}^{n}
\\
\displaystyle
-(\,\rho\,\tens{K}_{\,e}\ \underline{u}\,)^{n+\theta_S} -\rho\,\tens{K}_{\,d}^n\ \underline{u}^{n}+ (\underline{T}_{s}^{\,exp})^{\,n+\theta_S} + T_s^{\,imp}\,\,\underline {u}^{n} 
\\
\displaystyle
+[\dive (\mu_{\,tot}\,^t\ggrad \underline {u}\,)]^{n+\theta_S}-\frac {2} {3}[\,\grad (\mu_{\,tot}\,\dive \underline {u}\,)]^{n+\theta_S} + (\rho -\rho_0) \underline {g}-(\underline{turb})^{n+\theta_S}
\end{array}
\right.
\end{array}
\end{equation}

d'o� l'�quation r�solue par le sous-programme \fort{codits} :
\begin{equation}\begin{array}{c}
\displaystyle
f_s^{\,imp}(\underline {\widetilde{u}}^{n+1}-\underline {u}^{n}) + \theta\, \dive(\underline{\widetilde{u}}^{n+1} \otimes (\rho
\underline{u})) - \theta\,\dive (\,\mu_{\,tot}\,\ggrad \underline{\widetilde{u}}^{n+1}) = 
\\\\
\displaystyle
-(1-\theta)\,\dive(\underline{u}^{n} \otimes (\rho \underline{u}))+(1-\theta)\,\dive (\,\mu_{\,tot}\,\ggrad \underline{u}^{n})
+ \underline{f}_{\,s}^{\,exp}
\end{array}
\end{equation}
La m\'ethode de discr\'etisation spatiale est d\'evelopp\'ee dans le sous-programme \fort{codits}.\\



\minititre{Remarques :}
{\tiny$\blacksquare$} Dans le cas standard sans extrapolation, le terme
$-\,T_s^{\,imp}$ n'est ajout� � $f_s^{\,imp}$ que s'il est positif afin de ne
pas affaiblir la dominance de la diagonale de la matrice � inverser.\\ 
{\tiny$\blacksquare$} Si une extrapolation est utilis�e, par contre,
$\,T_s^{\,imp}$ est ajout� � $f_s^{\,imp}$ quel que soit son signe. En effet, l'id�e intuitive qui
consiste � prendre~: 
\begin{equation}
\begin{cases}
\displaystyle
(\underline{T}_{s}^{\,exp} + T_{s}^{\,imp}\,\underline {u})^{\,n+\theta_S} &
\text{si } T_{s}^{\,imp} > 0\\ 
\displaystyle
(\underline{T}_{s}^{\,exp})^{\,n+\theta_S} + T_{s}^{\,imp}\,\underline{u}^{n+\theta} &\text{sinon}\\
\end{cases}
\end{equation} 
aboutit � une incoh�rence dans le traitement si $T_s^{imp}$ change de signe
entre deux pas de temps.\\ 
{\tiny$\blacksquare$} la partie diagonale $\tens{K}_{\,d}$ du terme
de perte de charge est utilis�e dans $f_s^{\,imp}$. En fait, pour \^etre rigoureux,
il faudrait ne retenir que les contributions positives (point signal\'e dans le
sous-programme utilisateur associ\'e \fort{uskpdc}). Cette prise en compte sera \`a am\'eliorer.\\
{\tiny$\blacksquare$} Le terme $\theta\,\Gamma^{n}-\theta\,\dive
(\rho\,\underline {u})$ ne pose pas de probl�me pour la 
dominance de la diagonale de la matrice car il est exactement compens� par le
terme de convection (cf. \fort{covofi}). 


%                      Code_Saturne version 1.3
%                      ------------------------
%
%     This file is part of the Code_Saturne Kernel, element of the
%     Code_Saturne CFD tool.
%
%     Copyright (C) 1998-2007 EDF S.A., France
%
%     contact: saturne-support@edf.fr
%
%     The Code_Saturne Kernel is free software; you can redistribute it
%     and/or modify it under the terms of the GNU General Public License
%     as published by the Free Software Foundation; either version 2 of
%     the License, or (at your option) any later version.
%
%     The Code_Saturne Kernel is distributed in the hope that it will be
%     useful, but WITHOUT ANY WARRANTY; without even the implied warranty
%     of MERCHANTABILITY or FITNESS FOR A PARTICULAR PURPOSE.  See the
%     GNU General Public License for more details.
%
%     You should have received a copy of the GNU General Public License
%     along with the Code_Saturne Kernel; if not, write to the
%     Free Software Foundation, Inc.,
%     51 Franklin St, Fifth Floor,
%     Boston, MA  02110-1301  USA
%
%-----------------------------------------------------------------------
%

%%%%%%%%%%%%%%%%%%%%%%%%%%%%%%%%%%
%%%%%%%%%%%%%%%%%%%%%%%%%%%%%%%%%%
\section{Mise en \oe uvre}
%%%%%%%%%%%%%%%%%%%%%%%%%%%%%%%%%%
%%%%%%%%%%%%%%%%%%%%%%%%%%%%%%%%%%
La num\'ero de la phase trait\'ee fait partie des arguments de \fort{turrij}. On
omettra volontairement de le pr\'eciser dans ce qui suit, on indiquera par $(\ )$ la
notion de tableau s'y rattachant.

\etape{Calcul des termes de production $\tens{\mathcal{P}}$}
\begin{itemize}
\item [$\star$] Initialisation \`a z\'ero du tableau \var{PRODUC} dimensionn\'e \`a $\var{NCEL}\times 6$.
\item [$\star$] On appelle trois fois \fort{grdcel} pour calculer les gradients des composantes de la vitesse $u$, $v$ et
$w$ prises au temps $n$.

Au final, on a :\\
$\displaystyle
\begin{array} {ll}
\var{PRODUC(1,IEL)} = & \displaystyle - 2 \left[ R_{11}^{\,n} \frac{\partial u^{\,n}} {\partial x} +R_{12}^{\,n} \frac{\partial u^{\,n}} {\partial y}+R_{13}^{\,n} \frac{\partial u^{\,n}} {\partial z} \right] \text{        (production de $R_{11}^{\,n}$)}\\
\var{PRODUC(2,IEL)} = & \displaystyle - 2 \left[ R_{12}^{\,n} \frac{\partial v^{\,n}} {\partial x} +R_{22}^{\,n} \frac{\partial v^{\,n}} {\partial y}+R_{23}^{\,n} \frac{\partial v^{\,n}} {\partial z} \right] \text{        (production de $R_{22}^{\,n}$)}\\
\var{PRODUC(3,IEL)} = & \displaystyle - 2 \left[ R_{13}^{\,n} \frac{\partial w^{\,n}} {\partial x} +R_{23}^{\,n} \frac{\partial w^{\,n}} {\partial y}+R_{33}^{\,n} \frac{\partial w^{\,n}} {\partial z} \right] \text{        (production de $R_{33}^{\,n}$)}\\
\var{PRODUC(4,IEL)} = & \displaystyle - \left[ R_{12}^{\,n} \frac{\partial u^{\,n}} {\partial x} +R_{22}^{\,n} \frac{\partial u^{\,n}} {\partial y}+R_{23}^{\,n} \frac{\partial u^{\,n}} {\partial z} \right] \\
& \displaystyle - \left[ R_{11}^{\,n} \frac{\partial v^{\,n}} {\partial x} +R_{12}^{\,n} \frac{\partial v^{\,n}} {\partial y}+R_{13}^{\,n} \frac{\partial v^{\,n}} {\partial z} \right] \text{        (production de $R_{12}^{\,n}$)} \\
\var{PRODUC(5,IEL)} = & \displaystyle - \left[ R_{13}^{\,n} \frac{\partial u^{\,n}} {\partial x} +R_{23}^{\,n} \frac{\partial u^{\,n}} {\partial y}+R_{33}^{\,n} \frac{\partial u^{\,n}} {\partial z} \right] \\
& \displaystyle - \left[ R_{11}^{\,n} \frac{\partial w^{\,n}} {\partial x} +R_{12}^{\,n} \frac{\partial w^{\,n}} {\partial y}+R_{13}^{\,n} \frac{\partial w^{\,n}} {\partial z} \right] \text{        (production de $R_{13}^{\,n}$)} \\
\var{PRODUC(6,IEL)} = & \displaystyle - \left[ R_{13}^{\,n} \frac{\partial v^{\,n}} {\partial x} +R_{23}^{\,n} \frac{\partial v^{\,n}} {\partial y}+R_{33}^{\,n} \frac{\partial v^{\,n}} {\partial z} \right] \\
& \displaystyle - \left[ R_{12}^{\,n} \frac{\partial w^{\,n}} {\partial x} +R_{22}^{\,n} \frac{\partial w^{\,n}} {\partial y}+R_{23}^{\,n} \frac{\partial w^{\,n}} {\partial z} \right]  \text{        (production de $R_{23}^{\,n}$)}
\end{array}
$
\end{itemize}

\etape{Calcul du gradient de la masse volumique $\rho^n$ prise au d\'ebut du pas
de temps courant\footnote{{\it i.e.} calcul\'ee \`a partir des
variables du pas de temps pr\'ec\'edent $n$ si n\'ecessaire.} $(n+1)$}
Ce calcul n'a lieu que si les termes de gravit\'e doivent \^etre pris en compte
($\var{IGRARI()} =1$).
\begin{itemize}
\item [$\star$] Appel de \fort{grdcel}  pour calculer le gradient de $\rho^n$
dans les trois directions de l'espace. Les conditions aux limites sur $\rho^n$
sont des conditions de Dirichlet puisque la valeur de $\rho^n$ aux faces de bord
$ik$ (variable \var{IFAC}) est connue et vaut $\rho_{\,b_{\,ik}}$. Pour \'ecrire les conditions aux limites
sous la forme habituelle, $$\rho_{\,b_{\,ik}} = \var{COEFA} + \var{COEFB}
\,\rho^n_{\,I'}$$ on pose alors $\var{COEFA}=
\var{PROPCE(IFAC,IPPROB(IROM(IPHAS)))}$ et $\var{COEFB} = \var{VISCB} = 0$.\\
$\var{PROPCE(1,IPPROB(IROM(IPHAS)))}$ (resp.$\var{VISCB}$) est utilis\'e en lieu
et place de l'habituel \var{COEFA} ($\var{COEFB}$), lors de l'appel \`a \fort{grdcel}.\\
On a donc :\\
$\displaystyle \var{GRAROX}= \frac{\partial \rho^n}{\partial x}\ $,$\displaystyle \ \var{GRAROY}= \frac{\partial
\rho^n}{\partial y}$ et $
\displaystyle \ \var{GRAROZ}= \frac{\partial \rho^n}{\partial z}\ $.

\end{itemize}

Le gradient de $\rho^n$ servira \`a calculer les termes de production par effets de gravit\'e si ces derniers sont pris en compte.

\etape{Boucle \var{ISOU} de $1$ \`a $6$ sur les tensions de Reynolds}
Pour $\var{ISOU} = 1,2,3,4,5,6$, on r\'esout respectivement et dans
l'ordre  les
\'equations de $R_{11}$, $R_{22}$, $R_{33}$, $R_{12}$, $R_{13}$ et $R_{23}$ par
l'appel au sous-programme \fort{resrij}.\\
La r\'esolution se fait par incr\'ement $\delta {R}_{ij}^{\,n+1,k+1}$ , en utilisant la m\^eme m\'ethode que
celle d\'ecrite dans le sous-programme \fort{codits}. On adopte ici les m\^emes notations.
\var{SMBR} est le second membre du syst\`eme \`a inverser, syst\`eme portant sur
les incr\'ements de la variable. \var{ROVSDT} repr\'esente la diagonale de la
matrice, hors convection/diffusion.\\
On va r\'esoudre l'\'equation (\ref{Base_Turrij_Eq_Temp_Rij}) sous forme incr\'ementale en
utilisant \fort{codits}, soit :
\begin{equation}\label{Base_Turrij_Eq_Temp_deltaRij}
\begin{array}{ll}
&\displaystyle \underbrace{\left(\frac {\rho^n_L}{\Delta t^n}
+ \rho^n_L \,C_1\,\frac{\varepsilon^n_L}{k^n_L}(1-\frac{\delta_{ij}}{3})
 - m^n_{\,lm} + \Gamma_L\,+ max(-\alpha^n_{R_{ij}},0)\right)\,|\Omega_l|}
_{\text {$\var{ROVSDT}$ contribuant
\`a la diagonale de la matrice simplifi\'ee de \fort{matrix}}}\,(\delta{R}_{ij}^{\,n+1,p+1})_{\,L}\\\\
&  \underbrace{+\sum\limits_{m\in Vois(l)}\displaystyle \left[
 m^n_{\,lm} \delta R_{ij,\,f_{\,lm}}^{\,n+1,p+1}
- (\mu^n_{\,lm} + \gamma^n_{\,lm})\
\frac{({\delta R}_{ij}^{\,n+1,p+1})_{M}-({\delta R}_{ij}^{\,n+1,p+1})_{L})}{\overline{L'M'}}\,
S_{\,lm} \right]}_{\text { convection upwind pur et diffusion non reconstruite
relatives \`a la matrice simplifi\'ee de \fort{matrix}\footnotemark}} \\
% voir le texte de la footmark plus bas
&= - \displaystyle\frac {\rho^n_L}{\Delta t^n}\,\left(\,(R^{\,n+1,p}_{ij})_L - (R^{\,n}_{ij})_L\,\right)\\
&-\,\underbrace{\displaystyle\int_{\Omega_l} \left(
\dive\,[\,(\rho\,\vect{u})^n\,R^{\,n+1,p}_{ij} - (\mu^n\,+ \gamma^n\,)\,
\grad{R^{\,n+1,p}_{ij}}\,]\right)\,d\Omega}_{\text {convection et diffusion
trait\'ees par \fort{bilsc2}}}\\
&+\displaystyle \int_{\Omega_l} \left[\,\mathcal{P}^{\,n+1,p}_{ij} + \mathcal{G}^{\,n+1,p}_{ij}
- \displaystyle\rho^n \,C_1\,\frac{\varepsilon^n}{k^n}\left[R^{\,n+1,p}_{ij}-
\frac{2}{3}\,k^n\,\delta_{ij}\right] + \phi^{\,n+1,p}_{ij,2} +
\phi^{\,n+1,p}_{ij,w}\,\right]\, d\Omega \\
& + \displaystyle\int_{\Omega_l} \left[- \frac{2}{3} \rho^n \varepsilon^n \delta_{ij}
 + \Gamma\,(\,R^{\,in}_{ij} - R^{\,n+1,p}_{ij}\,) +
\alpha^n_{R_{ij}}\,R^{\,n+1,p}_{ij}+ \beta^n_{R_{ij}}\right]\, d\Omega\\
&+ \sum\limits_{m\in
Vois(l)}\displaystyle \left[\ \tens{E}^n\,\grad{R}^{\,n+1,p}_{ij} \right]_{\,lm}\,.\,\vect{n}_{\,lm}S_{\,lm}\\
&+ \sum\limits_{m\in Vois(l)}\displaystyle \left[\
\tens{D}^n\,\grad{R}^{\,n+1,p}_{ij} \right]_{\,lm}\,.\,\vect{n}_{\,lm}S_{\,lm}\\
&- \sum\limits_{m\in Vois(l)} \gamma^n_{\,lm} \left( \grad{R}^{\,n+1,p}_{ij}\,.\,\vect{n}_{\,lm} \right)  S_{\,lm}\\
&+ \sum\limits_{m\in Vois(l)}  m^n_{\,lm}\,(R^{\,n+1,p}_{ij})_L\\
\end{array}
\end{equation}
% si on ne fait pas comme ca, il n'apparait pas
\footnotetext[\thefootnote]{Si $\var{IRIJNU} = 1$, on remplace  $\mu^n_{\,lm}$ par $(\mu +
\mu_t)^n_{\,lm}$ dans l'expression de la diffusion non reconstruite
associ\'ee \`a la matrice simplifi\'ee de \fort{matrix} ($\mu_t$ d\'esigne la
viscosit\'e turbulente calcul\'ee comme en $k-\varepsilon$).}

o\`u on rappelle :\\
pour $n$ donn\'e entier positif, on d\'efinit la suite
 $({R}_{ij}^{\,n+1,p})_{p \in \grandN}$
 par :
\begin{equation}\notag
\left\{\begin{array}{l}
{R}_{ij}^{\,n+1,0} = {R}_{ij}^{\,n}\\
{R}_{ij}^{\,n+1,p+1} = {R}_{ij}^{\,n+1,p} + \delta{R}_{ij}^{\,n+1,p+1} \\
\end{array}\right.
\end{equation}
$(\delta{R}_{ij}^{\,n+1,p+1})_{\,L}$ d\'esigne la valeur sur l'\'el\'ement
$\Omega_l$ du $\text{$(\,p+1\,)$-i\`eme}$ incr\'ement de ${R}_{ij}^{\,n+1}$,
$ m^n_{\,lm}$ le flux de masse \`a l'instant $n$ \`a travers la face $lm$,
$\delta R_{ij,\,f_{\,lm}}^{\,n+1,p+1}$ vaut $({\delta
R}_{ij}^{\,n+1,p+1})_{L}$  si $ m^n_{\,lm} \geqslant 0$, $({\delta
R}_{ij}^{\,n+1,p+1})_{M}$ sinon,
$\mathcal{P}^{\,n+1,p}_{ij}$, $\phi^{\,n+1,p}_{ij,2}$, $\phi^{\,n+1,p}_{ij,w}$ les valeurs
des quantit\'es associ\'ees correspondant \`a l'incr\'ement
$(\delta{R}_{ij}^{\,n+1,p})$.\\



Tous ces termes sont calcul\'es comme suit :
\begin{itemize}
\item Terme de gauche de l'\'equation (\ref{Base_Turrij_Eq_Temp_deltaRij})\\
Dans \fort{resrij} est calcul\'ee la variable \var{ROVSDT}. Les autres
termes sont compl\'et\'es par \fort{codits}, lors de la construction de la matrice simplifi\'ee , {\it via} un
appel au sous-programme \fort{matrix}. La quantit\'e
 $(\mu^n_{\,lm} + \gamma^n_{\,lm})$ \`a la face $lm$ est calcul\'ee lors de l'appel \`a
\fort{visort}.\\
\item Second membre de l'\'equation (\ref{Base_Turrij_Eq_Temp_deltaRij})\\
Le premier terme non d\'etaill\'e est calcul\'e par le sous-programme
\fort{bilsc2}, qui applique le sch\'ema convectif choisi par l'utilisateur, qui
reconstruit ou non selon le souhait de l'utilisateur les gradients intervenants
dans la convection-diffusion.\\
Les termes sans accolade sont, eux, compl\`etement explicites et ajout\'es au fur et
\`a mesure dans \var{SMBR} pour former
l'expression $f^{\,exp}_s$ de \fort{codits}.
\end{itemize}
On d\'ecrit ci-dessous les \'etapes de \fort{resrij} :
\begin{itemize}

\item DELTIJ = 1, pour $\var{ISOU} \leqslant 3$ et DELTIJ = 0  Si $\var{ISOU} >
3$. Cette valeur repr\'esente le symbole de Kroeneker $\delta_{ij}$.

\item Initialisation \`a z\'ero de \var{SMBR} (tableau contenant le second
membre) et \var{ROVSDT} (tableau contenant la diagonale de la matrice sauf celle
relative \`a la contribution de la
diagonale des op\'erateurs de convection et de diffusion lin\'earis\'es
\footnote{qui correspondent aux sch\'emas convectif upwind pur et diffusif sans
reconstruction.}), tous deux de dimension $\var{NCEL}$.

\item Lecture et prise en compte des termes sources utilisateur pour la variable $R_{ij}$

Appel \`a \fort{ustsri} pour charger les termes sources utilisateurs. Ils sont
stock\'es comme suit. Pour la cellule $\Omega_l$ de centre $L$, repr\'esent\'ee par $\var{IEL}$, on a :\\
\begin{equation}\notag
\left\{\begin{array}{lll}
&\var{ROVSDT(IEL)}&= |\Omega_l| \ \alpha_{R_{ij}}\\
&\var{SMBR(IEL)}&=|\Omega_l| \ \beta_{R_{ij}}\\
\end{array}\right.
\end{equation}
On affecte alors les valeurs ad\'equates au second membre \var{SMBR} et \`a la
diagonale \var{ROVSDT} comme suit :
\begin{equation}\notag
\left\{\begin{array}{lll}
&\var{SMBR(IEL)} &= \var{SMBR(IEL)} +\ |\Omega_l| \ \alpha_{R_{ij}} \ (R^n_{ij})_L \\
&\var{ROVSDT(IEL)}&= \text{max }(-\ |\Omega_l| \ \alpha_{R_{ij}},0)\\
\end{array}\right.
\end{equation}
La valeur de $ \var{ROVSDT}$ est ainsi calcul\'ee pour des raisons de stabilit\'e
num\'erique. En effet, on ne rajoute sur la diagonale que les valeurs positives,
ce qui correspond physiquement \`a impliciter les termes de rappel uniquement.
\item{Calcul du terme source de masse  si $\Gamma_L > 0$}

Appel de \fort{catsma} et incr\'ementation si n\'ecessaire de \var{SMBR} et
\var{ROVSDT} {\it via} :\\
\begin{equation}\notag
\left\{\begin{array}{lll}
\displaystyle \var{SMBR(IEL)} = \var{SMBR(IEL)} + |\Omega_l| \ \Gamma_L \
\left[(R^{\,in}_{ij})_L - (R^{\,n}_{ij})_L \right] \\
\displaystyle \var{ROVSDT(IEL)}=\var{ROVSDT(IEL)} + |\Omega_l| \ \Gamma_L
\end{array}\right.
\end{equation}
\item Calcul du terme d'accumulation de masse et du terme instationnaire

On stocke $\displaystyle \var{W1}= \int_{\Omega_l}\dive\,(\rho\,\vect{u})\,d\Omega$
calcul\'e par \fort{divmas} \`a l'aide des flux de masse aux faces internes
$ m^n_{\,lm}=\sum\limits_{m\in Vois(l)}{(\rho \vect{u})_{\,lm}^n} \text{.}\,
\vect{S}_{\,lm} $ (tableau \var{FLUMAS}) et des flux de masse aux bords  $ m^n_{\,b_{lk}} = \sum\limits_{k\in{\gamma_b(l)}}{(\rho \vect{u})_{\,{b}_{lk}}^n} \text{.}\,
\vect{S}_{\,{b}_{lk}} $ (tableau \var{FLUMAB}).
On incr\'emente ensuite \var{SMBR} et \var{ROVSDT}.
\begin{equation}\notag
\left\{\begin{array}{lll}
&\var{SMBR(IEL)} &= \var{SMBR(IEL)} + \var{ICONV}\  (R^n_{ij})_L\,(\displaystyle
\int_{\Omega_l}\dive\,(\rho\,\vect{u})\ d\Omega) \\
&\var{ROVSDT(IEL)}& = \var{ROVSDT(IEL)} +  \var{ISTAT}\,\displaystyle
\frac{\rho^n_L \ |\Omega_l|}{\Delta t^n} -  \var{ICONV}\ (\displaystyle
\int_{\Omega_l}\dive\,(\rho\,\vect{u})\ d\Omega) \\
\end{array}\right.
\end{equation}
\item Calcul des termes sources de production, des termes $\displaystyle
\phi_{\,ij,1}+\phi_{\,ij,2}$ et de la dissipation~$\displaystyle-\frac{2}{3} \varepsilon\,\delta_{\,ij}$ :

On effectue une boucle d'indice \var{IEL} sur les cellules $\Omega_l$ de centre $L$ :
\begin{itemize}
\item [$\Rightarrow$] $\displaystyle \var{TRPROD}= \frac{1}{2} (\mathcal{P}^n_{ii})_L = \frac{1}{2} \left[ \var{PRODUC(1,IEL)} +  \var{PRODUC(2,IEL)} +  \var{PRODUC(3,IEL)} \right] $
\item [$\Rightarrow$] $\displaystyle \var{TRRIJ }= \frac{1}{2} (R^n_{ii})_L $
\item [$\Rightarrow$] $\displaystyle \var{SMBR(IEL)} =\ \var{SMBR(IEL)}\ +$\\
$\ \displaystyle\rho^n_L |\Omega_l| \left[ \displaystyle
\frac{2}{3}\,\delta_{\,ij} \left( \ \displaystyle \frac{ C_2}{2}\,(\mathcal{P}^n_{ii})_L\ +
(C_1-1)\ \varepsilon^n_L\, \right)\right.$\\
$ + \left.\ (1-C_2) \ \var{PRODUC(ISOU,IEL)} -
\displaystyle C_1\ \frac{2\,\varepsilon^n_L}{(R^n_{ii})_L}\ (R^n_{ij})_L \right]$
\item [$\Rightarrow$] $\displaystyle \var{ROVSDT(IEL)} = \var{ROVSDT(IEL)} +
\rho^n_L \ |\Omega_l| \ (- \displaystyle \frac{1}{3} \ \,\delta_{\,ij} + 1) \ C_1
\ \frac{2\ \varepsilon^n_L}{(R^n_{ii})_L}$
\end{itemize}
\item Appel de \fort{rijech} pour le calcul des termes d'\'echo de paroi
 $\phi^n_{ij,w}$ si $\var{IRIJEC()}=1$ et ajout dans \var{SMBR}.\\
$\var{SMBR} = \var{SMBR} + \phi^n_{ij,w}$\\
Suivant son mode de calcul (\var{ICDPAR}), la distance � la paroi est directement accessible
par \var{RA(IDIPAR+IEL-1)} (\var{|ICDPAR|} = 1) ou bien
est calcul\'ee \`a partir de $\var{IA(IIFAPA(IPHAS)+IEL - 1)}$,
qui donne pour l'\'el\'ement $\var{IEL}$ le num\'ero de la face de bord
paroi la plus  proche (\var{|ICDPAR|} = 2). Ces tableaux ont \'et\'e renseign\'e une fois pour toutes au
d\'ebut de calcul.

\item  Appel de \fort{rijthe} pour le calcul des termes de gravit\'e $\mathcal{G}^n_{ij}$ et ajout dans \var{SMBR}.

Ce calcul n'a lieu que si $\var{IGRARI()} = 1$.
$ \var{SMBR} = \var{SMBR} + \mathcal{G}^n_{ij}$
\item Calcul de la partie explicite du terme de diffusion
 $\dive{\,\left[\tens{A}\,\grad{R}^{\,n}_{ij}\right]}$, plus pr\'ecis\'ement
des contributions du terme extradiagonal pris aux faces purement internes
(remplissage du tableau \var{VISCF}), puis aux faces de bord (remplissage du
tableau \var{VISCB}).
\begin{itemize}
\item [$\star$] Appel de \fort{grdcel} pour le calcul du gradient de
$R^{\,n}_{ij}$ dans chaque direction. Ces gradients sont respectivement
stock\'es dans les tableaux de travail \var{W1}, \var{W2} et \var{W3}.

\item [$\star$] boucle d'indice \var{IEL} sur les cellules $\Omega_l$ de centre
$L$ pour le
calcul de $\tens{E}^n\,\grad{R}^{\,n}_{ij}$ aux cellules dans un premier temps :\\
\begin{itemize}
\item [$\Rightarrow$] $\displaystyle \var{TRRIJ}= \frac{1}{2} (R^{\,n}_{ii})_L $
\item [$\Rightarrow$] $\displaystyle \var{CSTRIJ} = \rho^n_L\ C_S \ \displaystyle\frac{(R^n_{ii})_L}{2\,\varepsilon^n_L}$
\item [$\Rightarrow$] $\displaystyle \var{W4(IEL)} = \rho^n_L\ C_S\
\displaystyle\frac{(R^n_{ii})_L}{2\,\varepsilon^n_L} \left[\,(R^{\,n}_{12})_L \ \var{W2(IEL)} +
(R^{\,n}_{13})_L \ \var{W3(IEL)}\,\right]$
\item [$\Rightarrow$] $\displaystyle \var{W5(IEL)} = \rho^n_L\ C_S\
\displaystyle\frac{(R^n_{ii})_L}{2\,\varepsilon^n_L} \left[\,(R^{\,n}_{12})_L \ \var{W1(IEL)} +
(R^{\,n}_{23})_L \ \var{W3(IEL)}\,\right]$
\item [$\Rightarrow$] $\displaystyle \var{W6(IEL)} = \rho^n_L\ C_S\
\displaystyle\frac{(R^n_{ii})_L}{2\,\varepsilon^n_L} \left[\,(R^{\,n}_{13})_L \ \var{W1(IEL)} + (R^{\,n}_{23})_L \ \var{W2(IEL)}\,\right]$
\end{itemize}



\item [$\star$] Appel de \fort{vectds}\footnote{Le r\'esultat est stock\'e dans
\var{VISCF} et \var{VISCB}. Dans \fort{vectds}, les valeurs aux faces internes
sont interpol\'ees lin\'eairement sans reconstruction et \var{VISCB} est mis \`a
z\'ero.} pour assembler $\displaystyle\left[ \tens{E}^n\,\grad{R}^{\,n}_{ij}
\right]\,.\,\vect{n}_{\,lm}S_{\,lm}$ aux faces $lm$.
\item [$\star$] Appel de \fort{divmas} pour calculer la divergence du flux d\'efini par \var{VISCF} et \var{VISCB}.
Le r\'esultat est stock\'e dans \var{W4}.\\
Ajout au second membre \var{SMBR}.\\
\var{SMBR} = \var{SMBR} + \var{W4}
\end{itemize}

A l'issue de cette \'etape, seule la partie extradiagonale de la diffusion prise
enti\`erement explicite~:
 $$\sum\limits_{m\in
Vois(l)}\left[\ \tens{E}^n\,\grad{R}^{\,n}_{ij} \right]_{\,lm}\,.\,\vect{n}_{\,lm}S_{\,lm}$$ a \'et\'e calcul\'ee.\\

\item Calcul de la partie diagonale du terme de diffusion\footnote{Seule la
composante normale  du  gradient de $R_{ij}$ aux faces sera implicite.} :\\
Comme on l'a d\'eja vu, une partie de cette contribution sera trait\'ee en
implicite (celle relative \`a la composante normale du gradient) et donc
ajout\'ee au second membre par \fort{bilsc2} ; l'autre
partie sera explicite et prise en compte dans $f_s^{\,exp}$.
\begin{itemize}
\item [$\star$] On effectue une boucle d'indice \var{IEL} sur les cellules
$\Omega_l$ de centre $L$ :
\begin{itemize}
\item [$\Rightarrow$] $\displaystyle \var{TRRIJ }= \frac{1}{2} (R^{\,n}_{ii})_L $
\item [$\Rightarrow$] $\displaystyle \var{CSTRIJ} = \rho^n_L \ C_S \ \frac{(R^{\,n}_{ii})_L}{2\,\varepsilon^n_L}$
\item [$\Rightarrow$] $\displaystyle \var{W4(IEL)} = \rho^n_L \ C_S \
\frac{(R^{\,n}_{ii})_L}{2\,\varepsilon^n_L} \ (R^{\,n}_{11})_L$
\item [$\Rightarrow$] $\displaystyle \var{W5(IEL)} = \rho^n_L \ C_S \ \frac{(R^{\,n}_{ii})_L}{2\,\varepsilon^n_L}\ (R^n_{22})_L$
\item [$\Rightarrow$] $\displaystyle \var{W6(IEL)} = \rho^n_L \ C_S \ \frac{(R^{\,n}_{ii})_L}{2\,\varepsilon^n_L} \ (R^n_{33})_L$
\end{itemize}

%\item Traitement du parall\'elisme et de la p\'eriodicit\'e.

\item [$\star$] On effectue une boucle d'indice \var{IFAC} sur les faces
purement internes $lm$ pour remplir le tableau \var{VISCF} :
\begin{itemize}
\item [$\Rightarrow$] Identification des cellules $\Omega_l$ et $\Omega_m$ de
centre respectif $L$ (variable \var{II}) et $M$ (variable \var{JJ}), se trouvant de chaque c\^ot\'e de la face
$lm$\footnote{La normale \`a la face est orient\'ee de L vers M.}.
\item [$\Rightarrow$] Calcul du carr\'e de la surface de la face. La valeur est
stock\'ee dans le tableau \var{SURFN2}.
\item [$\Rightarrow$] Interpolation du gradient de $R^{\,n}_{ij}$ \`a la face
$lm$ (gradient facette $\left[\grad{R}^{\,n}_{ij}\right]_{\,lm}$) :
\begin{equation}\notag
\left\{\begin{array}{ll}
\var{GRDPX} &= \displaystyle \frac{1}{2} \left(\var{W1(II)} + \var{W1(JJ)}
\right) \\
&\\
\var{GRDPY} &= \displaystyle \frac{1}{2} \left(\var{W2(II)} + \var{W2(JJ)} \right) \\
&\\
\var{GRDPZ} &= \displaystyle \frac{1}{2} \left(\var{W3(II)} + \var{W3(JJ)} \right)
\end{array}\right.
\end{equation}
\item [$\Rightarrow$] Calcul du gradient de $R^{\,n}_{ij}$ normal \`a la face
$lm$, $\left[\grad{R}^{\,n}_{ij}\right]_{\,lm}.\vect{n}_{\,lm}\,S_{\,lm}$.\\

$\displaystyle \var{GRDSN} =  \var{GRDPX} \ \var{SURFAC(1,IFAC)} + \var{GRDPY} \ \var{SURFAC(2,IFAC)} +  \var{GRDPZ} \ \var{SURFAC(3,IFAC)}$
$\var{SURFAC}$ est le vecteur surface de la face \var{IFAC}.


\item [$\Rightarrow$] calcul de
 $\left[\grad{R^{\,n}_{ij}} - (\grad
R^{\,n}_{ij}\,.\,\vect{n}_{\,lm})\vect{n}_{\,lm}\right]$, les vecteurs \'etant
calcul\'es \`a la face $lm$ :
\begin{equation}\notag
\left\{\begin{array}{lll}
&\displaystyle \var{GRDPX} &= \var{GRDPX} - \displaystyle\frac{\var{GRDSN}}{\var{SURFN2}} \ \var{SURFAC(1,IFAC)}\\
&&\\
&\displaystyle \var{GRDPY} &= \var{GRDPY} - \displaystyle\frac{\var{GRDSN}}{\var{SURFN2}} \ \var{SURFAC(2,IFAC)} \\
&&\\
&\displaystyle \var{GRDPZ} &= \var{GRDPZ} - \displaystyle\frac{\var{GRDSN}}{\var{SURFN2}} \ \var{SURFAC(3,IFAC)}
\end{array}\right.
\end{equation}
\item [$\Rightarrow$] finalisation du calcul de l'expression totalement
explicite
 $$\left[ \tens{D}^n\,\left( \grad{R^{\,n}_{ij}} - (\grad R^{\,n}_{ij}\,.\,\vect{n}_{\,lm})\,\vect{n}_{\,lm}\right) \right]\,.\,\vect{n}_{\,lm}$$
\begin{equation}\notag
\begin{array} {ll}
\displaystyle \var{VISCF} = &
 \displaystyle\frac{1}{2} (\ \var{W4(II)} +\ \var{W4(JJ)}) \ \var{GRDPX} \
\var{SURFAC(1,IFAC)})\ + \\
&\\
&  \displaystyle\frac{1}{2} (\ \var{W5(II)} +\ \var{W5(JJ)}) \ \var{GRDPY} \
\var{SURFAC(2,IFAC)})\ + \\
&\\
&  \displaystyle\frac{1}{2} (\ \var{W6(II)} +\ \var{W6(JJ)}) \ \var{GRDPZ} \ \var{SURFAC(3,IFAC)})
\end{array}
\end{equation}
\end{itemize}

\item [$\star$] Mise \`a z\'ero du tableau \var{VISCB}.

\item [$\star$] Appel de \fort{divmas} pour calculer la divergence de~:
 $$\tens{D}^{\,n}\,\left( \grad{R^{\,n}_{ij}} - (\grad R^{\,n}_{ij}\,.\,\vect{n}_{\,lm})\vect{n}_{\,lm}\right)$$ d\'efini aux faces dans \var{VISCF} et \var{VISCB}.

Le r\'esultat est stock\'e dans le tableau \var{W1}.\\
Ajout au second membre \var{SMBR}.\\
$\var{SMBR} = \var{SMBR} + \var{W1}$
\end{itemize}
\item Calcul de la viscosit\'e orthotrope $\gamma^n_{\,lm}$ \`a la face $lm$ de la variable principale
$R^{\,n}_{ij}$\\
Ce calcul permet au sous-programme \fort{codits} de compl\'eter le second membre
\var{SMBR} par :
\begin{equation}
\begin{array} {ll}
& \sum\limits_{m\in Vois(l)}
\mu^n_{\,lm}\,\left(\grad{R}^{\,n}_{ij}\,.\,\vect{n}_{\,lm}\right)S_{\,lm}
 + \sum\limits_{m\in Vois(l)} \left(\grad{R}^{\,n}_{ij}
\,.\,\vect{n}_{\,lm}\right)\left[\tens{D}^{\,n}\,\vect{n}_{\,lm}\right]_{\,lm}\,.\,\vect{n}_{\,lm}\
S_{\,lm}\\
& = \sum\limits_{m\in Vois(l)}(\,\mu^n_{\,lm}\, + \,\gamma^n_{\,lm}\,)\,\left(\grad{R}^{\,n}_{ij}\,.\,\vect{n}_{\,lm}\right)S_{\,lm}
\end{array}
\end{equation}
sans pr\'eciser la nature de la face $lm$, {\it via} l'appel \`a \fort{bilsc2}  et de disposer de la quantit\'e
$(\mu^n_{\,lm}\, + \gamma^n_{\,lm})$ pour construire sa
matrice simplifi\'ee.\\
\begin{itemize}
\item [$\star$] On effectue une boucle d'indice \var{IEL} sur les cellules
$\Omega_l$ :
\begin{itemize}
\item [$\Rightarrow$] $\displaystyle \var{TRRIJ }= \frac{1}{2} (R^{\,n}_{ii})_L $
\item [$\Rightarrow$] $\displaystyle \var{RCSTE} = \rho^n_L \ C_S \ \frac{ (R^{\,n}_{ii})_L}{2\,\varepsilon^n_L} $
\item [$\Rightarrow$] $\displaystyle \var{W1(IEL)} = \mu^n +\rho^n_L \ C_S \ \frac{
(R^{\,n}_{ii})_L}{2\,\varepsilon^n_L}\ (R^n_{11})_L$
\item [$\Rightarrow$] $\displaystyle \var{W2(IEL)} = \mu^n + \rho^n_L \ C_S \ \frac{ (R^{\,n}_{ii})_L}{2\,\varepsilon^n_L}\ (R^n_{22})_L$
\item [$\Rightarrow$] $\displaystyle \var{W3(IEL)} = \mu^n + \rho^n_L \ C_S \ \frac{ (R^{\,n}_{ii})_L}{2\,\varepsilon^n_L}\ (R^n_{33})_L$
\end{itemize}

\item [$\star$] Appel de \fort{visort} pour calculer la viscosit\'e orthotrope
\footnote{Comme dans le sous-programme \fort{viscfa}, on multiplie la viscosit\'e par
$\displaystyle \frac{S_{\,lm}}{\overline{L'M'}}$, o\`u $S_{\,lm}$ et
$\overline{L'M'}$ repr\'esentent respectivement la surface de la face $lm$ et la
mesure alg\'ebrique du segment reliant les projections des centres des cellules
voisines sur la normale \`a la face. On garde dans ce sous-programme  la possibilit\'e d'interpoler la viscosit\'e aux cellules lin\'eairement ou d'utiliser une moyenne harmonique. La viscosit\'e au bord est celle de la cellule de bord correspondante.}
$ \gamma^n_{\,lm} = (\tens{D}^{\,n}\,\vect{n}_{\,lm}).\vect{n}_{\,lm}$ aux faces $lm$

Le r\'esultat est stock\'e dans les tableaux \var{VISCF} et \var{VISCB}.
\end{itemize}

\item appel de \fort{codits} pour la r\'esolution de l'\'equation de
convection/diffusion/termes sources de la variable $R_{ij}$. Le terme source est
\var{SMBR}, la viscosit\'e \var{VISCF} aux faces purement internes (
resp. \var{VISCB} aux faces de bord) et \var{FLUMAS} le flux de masse interne
 ( resp. \var{FLUMAB} flux de masse au bord). Le r\'esultat est la variable $R_{ij}$ au temps
$n+1$, donc $R^{\,n+1}_{ij}$. Elle est stock\'ee dans le tableau \var{RTP} des
variables mises \`a jour.

\end{itemize}

\etape{Appel de \fort{reseps} pour la r\'esolution de la variable $\varepsilon$}

On d\'ecrit ci-dessous le sous-programme \fort{reseps}, les commentaires du sous-programme \fort{resrij} ne seront pas r\'ep\'et\'es puisque les deux sous-programmes ne diff\`erent que par leurs termes sources.

\begin{itemize}
\item Initialisation \`a z\'ero de \var{SMBR} et \var{ROVSDT}.

\item{Lecture et prise en compte des termes sources utilisateur pour la variable $\varepsilon$ :}

Appel de \fort{ustsri} pour charger les termes sources utilisateurs. Ils sont
stock\'es dans les tableaux suivants :\\
pour la cellule $\Omega_l$ repr\'esent\'ee par $\var{IEL}$ de centre $L$, on a :
\begin{equation}\notag
\left\{\begin{array}{lll}
&\var{ROVSDT(IEL)}&= |\Omega_l| \ \alpha_{\varepsilon}\\
&\var{SMBR(IEL)}&=|\Omega_l| \ \beta_{\varepsilon}\\
\end{array}\right.
\end{equation}
On affecte alors les valeurs ad\'equates au second membre \var{SMBR} et \`a la
diagonale \var{ROVSDT} comme suit :
\begin{equation}\notag
\left\{\begin{array}{lll}
&\var{SMBR(IEL)} &= \var{SMBR(IEL)} +\ |\Omega_l| \ \alpha_{\,\varepsilon} \
\varepsilon^n_L \\
&\var{ROVSDT(IEL)}&= \text{max }(-\ |\Omega_l| \ \alpha_{\,\varepsilon},0)\\
\end{array}\right.
\end{equation}

\item{Calcul du terme source de masse si $\Gamma_L > 0$ :
\begin{equation}\notag
\left\{\begin{array}{lll}
&\displaystyle \var{SMBR(IEL)} = \var{SMBR(IEL)} + |\Omega_l| \ \Gamma_L \
(\varepsilon^{\,in}_L -\varepsilon^n_L) \\
&\displaystyle \var{ROVSDT(IEL)}= \var{ROVSDT(IEL)} + |\Omega_l| \ \Gamma_L
\end{array}\right.
\end{equation}
\item Calcul du terme d'accumulation de masse et du terme instationnaire \\
On stocke $\displaystyle \var{W1}= \int_{\Omega_l}\dive\,(\rho\,\vect{u})\,d\Omega$
calcul\'e par \fort{divmas} \`a l'aide des flux de masse internes et aux bords.\\
On incr\'emente ensuite \var{SMBR} et \var{ROVSDT}.
\begin{equation}\notag
\left\{\begin{array}{lll}
&\var{SMBR(IEL)} &= \var{SMBR(IEL)} + \var{ICONV}\ \varepsilon^n_L\,(\displaystyle
\int_{\Omega_l}\dive\,(\rho\,\vect{u})\ d\Omega) \\
&\var{ROVSDT(IEL)}& = \var{ROVSDT(IEL)} +  \var{ISTAT}\,\displaystyle
\frac{\rho^n_L \ |\Omega_l|}{\Delta t^n} -  \var{ICONV}\ (\displaystyle
\int_{\Omega_l}\dive\,(\rho\,\vect{u})\ d\Omega) \\
\end{array}\right.
\end{equation}

\item Traitement du terme de production
 $\displaystyle \rho\,C_{\varepsilon_1}\,\frac{\varepsilon}{k}\,\mathcal{P}$
 et du terme de dissipation $-\,\displaystyle \rho\,C_{\varepsilon_2}\,\frac{\varepsilon}{k}\,\varepsilon$ \\
pour cela, on effectue une boucle d'indice \var{IEL} sur les cellules $\Omega_l$
de centre $L$ :
\begin{itemize}
\item [$\Rightarrow$] $\displaystyle \var{TRPROD}= \frac{1}{2} (\mathcal{P}^n_{ii})_L = \frac{1}{2} \left[ \var{PRODUC(1,IEL)} +  \var{PRODUC(2,IEL)} +  \var{PRODUC(3,IEL)} \right] $
\item [$\Rightarrow$] $\displaystyle \var{TRRIJ }= \frac{1}{2} (R^n_{ii})_L $
\item [$\Rightarrow$] $\displaystyle \var{SMBR(IEL)} = \var{SMBR(IEL)} + \rho^n_L
|\Omega_l| \left[ -C_{\varepsilon_2} \ \frac{2\,(\varepsilon^n_L)^2}{(R^n_{ii})_L} + C_{\varepsilon_1} \ \frac{\varepsilon^n_L}{(R^n_{ii})_L}\ (\mathcal{P}^n_{ii})_L \right] $
\item [$\Rightarrow$] $\displaystyle \var{ROVSDT(IEL)} = \var{ROVSDT(IEL)} + C_{\varepsilon_2} \ \rho^n_L \ |\Omega_l| \ \frac{2\,\varepsilon^n_L}{(R^n_{ii})_L}$
\end{itemize}

\item Appel de \fort{rijthe} pour le calcul des termes de gravit\'e $\mathcal{G}^n_{\varepsilon}$ et ajout dans \var{SMBR}.

$ \var{SMBR} = \var{SMBR} + \mathcal{G}^n_{\varepsilon}$\\
Ce calcul n'a lieu que si $\var{IGRARI()} = 1$.

\item Calcul de la diffusion de $\varepsilon$ \\
 Le terme $\dive \left[\mu\, \grad(\varepsilon) + \tens{A'}\,\grad(\varepsilon)
\right]$ est calcul\'e exactement de la m\^eme mani\`ere que pour les tensions
de Reynolds $R_{ij}$ en rempla\c cant $\tens{A}$ par $\tens{A'}$.

\item Appel de \fort{codits} pour la r\'esolution de l'\'equation de
convection/diffusion/termes sources de la variable principale $\varepsilon$. Le
r\'esultat $\varepsilon^{\,n+1}$ est stock\'e dans le tableau \var{RTP} des
variables mises \`a jour.
}
\end{itemize}

\etape{clippings finaux}
On passe enfin dans le sous-programme  \fort{clprij} pour faire un clipping \'eventuel
des variables $R^{\,n+1}_{ij}$ et $\varepsilon^{\,n+1}$. Le sous-programme
\fort{clprij} est appel\'e\footnote{L'option
$\var{ICLIP} = 1$ consiste en un clipping minimal des variables $R_{ii}$ et
$\varepsilon$ en prenant la valeur absolue de ces variables puisqu'elles ne
peuvent \^etre que positives.} avec $\var{ICLIP} = 2$ . Cette option
\footnote{Quand la valeur des grandeurs $R_{ii}$ ou $\varepsilon$ est
n\'egative, on la remplace par le minimum entre sa valeur absolue et (1,1)
fois la valeur obtenue au pas de temps pr\'ec\'edent.} contient l'option $\var{ICLIP} = 1$  et permet de v\'erifier l'in\'egalit\'e de Cauchy-Schwarz sur les grandeurs extra-diagonales du tenseur $\tens{R}$ (pour $i \neq j$, $|R_{ij}|^2 \le R_{ii} R_{jj}$).


%%%%%%%%%%%%%%%%%%%%%%%%%%%%%%%%%%
%%%%%%%%%%%%%%%%%%%%%%%%%%%%%%%%%%
\section{Points \`a traiter}
%%%%%%%%%%%%%%%%%%%%%%%%%%%%%%%%%%
%%%%%%%%%%%%%%%%%%%%%%%%%%%%%%%%%%
Sauf mention explicite, $\phi$ repr\'esentera une tension de Reynolds ou la dissipation turbulente ($\phi = R_{ij} \ \text{ou} \ \varepsilon$).

\begin{itemize}
\item {La vitesse utilis\'ee pour le calcul de la production est explicite. Est-ce qu'une implicitation peut am\'eliorer la pr\'ecision temporelle de $\phi$ \footnote{Cette remarque peut \^etre g\'en\'eralis\'ee. En effet, peut-on envisager d'actualiser les variables d\'ej\`a r\'esolues (sans r\'eactualiser les variables turbulentes apr\`es leur r\'esolution)? Ceci obligerait \`a modifier les sous-programmes tels que \fort{condli} qui sont appel\'es au d\'ebut de la boucle en temps.} ?}
\item {Dans quelle mesure le terme d'\'echo de paroi est-il valide ? En effet, ce terme est remis en question par certains auteurs.}
\item {On peut envisager la r\'esolution d'un syst\`eme hyperbolique pour les
tensions de Reynolds afin d'introduire un couplage avec le champ de vitesse.}
\item {Le flux au bord \var{VISCB} est annul\'e dans le sous-programme
\fort{vectds}. Peut-on envisager de mettre au bord la valeur de la variable
concern\'ee \`a la cellule de bord correspondant? De m\^eme, il faudrait se
pencher sur les hypoth\`eses sous-jacentes \`a l'annulation des contributions
aux bords de \var{VISCB} lors du calcul de : $$\left[ \tens{D}^n\,\left( \grad{R^{\,n}_{ij}} - (\grad R^{\,n}_{ij}\,.\,\vect{n}_{\,lm})\,\vect{n}_{\,lm}\right) \right]\,.\,\vect{n}_{\,lm}.$$}
\item {Un probl\`eme de pond\'eration appara\^\i t plus g\'en\'eralement. Si on prend la partie explicite de $\tens{D}\,\grad(\phi)$, la pond\'eration aux faces internes utilise le coefficient $\displaystyle\frac{1}{2}$ avec pond\'eration s\'epar\'ee de $\tens{D}$ et $\grad(\phi)$, alors que pour $\tens{E}\,\grad(\phi)$, on calcule d'abord ce terme aux cellules pour ensuite l'interpoler lin\'eairement aux faces \footnote{Cette interpolation se fait dans \fort{vectds} avec des coefficients de pond\'eration aux faces.}. Ceci donne donc deux types d'interpolations pour des termes de m\^eme nature.}
\item {On laisse la possibilit\'e dans \fort{visort} d'utiliser une moyenne
harmonique aux faces. Est-ce que ceci est valable puisque les interpolations
utilis\'ees lors du calcul de la partie explicite de $\tens{A}\,\grad{\phi}$
sont des moyennes arithm\'etiques ?}
\item {Les techniques adopt\'ees lors du clipping sont \`a revoir.}
\item {On utilise dans le cadre du mod\`ele $\displaystyle R_{ij}-\varepsilon $ une semi-implicitation de termes comme $\displaystyle \phi_{ij,1}$ ou $\displaystyle -\rho\,C_{\varepsilon_2}\,\frac{\varepsilon}{k}\,\varepsilon$. On peut envisager le m\^eme type d'implicitation dans \fort{turbke} m\^eme en pr\'esence du couplage $\displaystyle k-\varepsilon$.}
\item L'adoption d'une r\'esolution d\'ecoupl\'ee fait perdre l'invariance par rotation.
\item La formulation et l'implantation des conditions aux limites de paroi
devront \^etre v\'erifi\'ees. En effet, il semblerait que, dans certains cas, des ph\'enom\`enes
``oscillatoires'' apparaissent, sans qu'il soit ais\'e d'en d\'eterminer la cause.
\item L'implicitation partielle (du fait de la r\'esolution d\'ecoupl\'ee) des
conditions aux limites conduit souvent \`a des calculs instables. Il
conviendrait d'en conna\^\i tre la raison. L'implicitation partielle avait
\'et\'e mise en \oe uvre afin de tenter d'utiliser un pas de temps plus grand
dans le cas de jets axisym\'etriques en particulier.

\end{itemize}

%                      Code_Saturne version 1.3
%                      ------------------------
%
%     This file is part of the Code_Saturne Kernel, element of the
%     Code_Saturne CFD tool.
%
%     Copyright (C) 1998-2007 EDF S.A., France
%
%     contact: saturne-support@edf.fr
%
%     The Code_Saturne Kernel is free software; you can redistribute it
%     and/or modify it under the terms of the GNU General Public License
%     as published by the Free Software Foundation; either version 2 of
%     the License, or (at your option) any later version.
%
%     The Code_Saturne Kernel is distributed in the hope that it will be
%     useful, but WITHOUT ANY WARRANTY; without even the implied warranty
%     of MERCHANTABILITY or FITNESS FOR A PARTICULAR PURPOSE.  See the
%     GNU General Public License for more details.
%
%     You should have received a copy of the GNU General Public License
%     along with the Code_Saturne Kernel; if not, write to the
%     Free Software Foundation, Inc.,
%     51 Franklin St, Fifth Floor,
%     Boston, MA  02110-1301  USA
%
%-----------------------------------------------------------------------
%
\programme{vortex}
%
\vspace{1cm}
%%%%%%%%%%%%%%%%%%%%%%%%%%%%%%%%%%
%%%%%%%%%%%%%%%%%%%%%%%%%%%%%%%%%%
\section{Fonction}
%%%%%%%%%%%%%%%%%%%%%%%%%%%%%%%%%%
%%%%%%%%%%%%%%%%%%%%%%%%%%%%%%%%%%
Ce sous-programme est d�di� � la g�n�ration des conditions d'entr�e
turbulente utilis�es en LES.


La m�thode des vortex est bas�e sur une approche de tourbillons
ponctuels. L'id�e de la m�thode consiste � injecter des tourbillons 2D dans le
plan d'entr�e du calcul, puis � calculer le champ de vitesse induit par ces
tourbillons au centre des faces d'entr�e.

%                      Code_Saturne version 1.3
%                      ------------------------
%
%     This file is part of the Code_Saturne Kernel, element of the
%     Code_Saturne CFD tool.
% 
%     Copyright (C) 1998-2007 EDF S.A., France
%
%     contact: saturne-support@edf.fr
% 
%     The Code_Saturne Kernel is free software; you can redistribute it
%     and/or modify it under the terms of the GNU General Public License
%     as published by the Free Software Foundation; either version 2 of
%     the License, or (at your option) any later version.
% 
%     The Code_Saturne Kernel is distributed in the hope that it will be
%     useful, but WITHOUT ANY WARRANTY; without even the implied warranty
%     of MERCHANTABILITY or FITNESS FOR A PARTICULAR PURPOSE.  See the
%     GNU General Public License for more details.
% 
%     You should have received a copy of the GNU General Public License
%     along with the Code_Saturne Kernel; if not, write to the
%     Free Software Foundation, Inc.,
%     51 Franklin St, Fifth Floor,
%     Boston, MA  02110-1301  USA
%
%-----------------------------------------------------------------------
%
%%%%%%%%%%%%%%%%%%%%%%%%%%%%%%%%%%
%%%%%%%%%%%%%%%%%%%%%%%%%%%%%%%%%%
\section{Discr\'etisation}
%%%%%%%%%%%%%%%%%%%%%%%%%%%%%%%%%%
%%%%%%%%%%%%%%%%%%%%%%%%%%%%%%%%%%

Le terme convectif en $\dive(\underline{u} \otimes \rho\,\underline{u})$
introduit une non lin\'earit\'e et un couplage des composantes de la vitesse
$\vect{u}$ dans l'�quation (\ref{Base_Preduv_eqqdm}). Une lin\'earisation et un d\'ecouplage
des trois composantes de la 
vitesse sont r\'ealis\'es lors de la discr\'etisation de cette \'etape de
pr\'ediction.\\
En effet, soit :
\begin{equation}
\vect{\widetilde{u}}= \vect{u}^n + \delta \vect{u} 
\end{equation}
La contribution exacte du terme convectif \`a prendre en compte dans cette
\'etape de pr\'ediction serait :\\
\begin{equation}\label{Base_Preduv_Conv_exact}
\begin{array}{ll}
\dive(\vect{\widetilde{u}} \otimes \rho\,\vect{\widetilde{u}}) =
\dive(\vect{u}^{n} \otimes \rho\,\vect{u}^{n}) + \dive(\delta \vect{u} \otimes
\rho\,\vect{u}^{n}) +  \underbrace { \dive(\vect{u}^{n} \otimes
\rho\,\delta \vect{u})}_{\text {terme couplant lin\'eaire}} +  \underbrace { \dive(\delta \vect{u} \otimes
\rho\,\delta \vect{u})}_{\text {terme couplant et non lin\'eaire}}\\
\end{array} 
\end{equation}
Les deux derniers termes de l'expression (\ref{Base_Preduv_Conv_exact}) sont {\em a priori} n�glig�s
de mani�re � obtenir un syst\`eme en vitesse qui soit d\'ecoupl\'e et donc,
�viter l'inversion d'une matrice pouvant \^etre de tr\`es grande taille. Ces
deux termes peuvent n�anmoins �tre pris en compte de mani�re plus ou moins
approch�e par extrapolation explicite du flux de masse en $n+\theta_F$ (pour le
terme couplant lin�aire provenant de la convection de $\vect{u}^{n}$ par $\delta
\vect{u}$) et utilisation d'un point-fixe par sous it�ration sur le sous
programme \fort{navsto} (pour le terme non-lin�aire, en sp�cifiant $\var{NTERUP}>1$).

L'�quation (\ref{Base_Preduv_eqqdm}) est discr�tis�e au temps $n+\theta$ � l'aide d'un
$\theta$-sch�ma, et le tenseur des pertes de charges d�compos� en une partie
diagonale $\tens{K}_{d}$ et une extradiagonale $\tens{K}_{e}$ (soit
 $\tens{K}_{pdc}=\tens{K}_{d}+\tens{K}_{e}$).\\
$\bullet$ La pression est suppos�e connue en $n-1+\theta$ (d�calage temporel
pression-vitesse) et le gradient naturellement calcul� � cet instant.\\ 
$\bullet$ Les termes sources de viscosit� secondaire, de gradient transpos\'e,
ceux provenant du mod�le de turbulence\footnote{except� $\dive (\mu_t\ (\ggrad
\underline {u}))$}, $\rho\,\tens{K}_{\,e}\ \underline{u}$, $(\rho -\rho_0)
\underline {g}$ ainsi que $\underline{T}_{s}^{\,exp}$ et
$\Gamma\,\underline{u}_{\,i}$ sont pris de mani�re explicite au temps $n$, ou
extrapol�s suivant le sch�ma en temps choisi pour les propri�t�s physique et les
termes sources.\\ 
$\bullet$ Les termes sources $\underline{u}\,\,\dive (\rho\,\underline {u})$,
$\Gamma\,\,\underline{u}$, $T_{s}^{\,imp}\,\,\underline{u}$ et
$-\rho\,\tens{K}_{\,d}\,\,\underline{u}$ sont implicit�s est calcul�s �
l'instant $n+\theta$.\\ 
$\bullet$ Le terme de diffusion $\dive (\mu_{\,tot}\,\ggrad \underline{u})$ est
implicit� : la vitesse est prise � l'instant $n+\theta$ et la viscosit�
explicit�e ou extrapol�e.\\ 
$\bullet$ Enfin, le terme de convection en $\dive(\,\underline{u} \otimes
(\rho\underline{u})\,)$ est implicit� : la composante r�solue de la vitesse est
prise en $n+\theta$, et le flux de masse, explicit�, ou extrapol� en
$n+\theta_F$. 

Par souci de clart�, on suppose, en l'absence d'indication, que les propri�tes
physiques $\Phi$ ($\rho,\,\mu_{tot},\,...$) et le flux de masse
$(\rho\underline{u})$ sont pris respectivement aux instants $n+\theta_\Phi$ et
$n+\theta_F$, o� $\theta_\Phi$ et $\theta_F$ d�pendent des sch�mas en temps
sp�cifiquement utilis�s pour ces grandeurs\footnote{cf. \fort{introd}}. 

La discr�tisation temporelle de l'�quation (\ref{Base_Preduv_eqqdm}) s'�crit alors comme suit : 

\begin{equation}\label{Base_Preduv_eq_di1}
 \begin{array}{c}
\displaystyle \rho\,\ \frac{ \underline {\widetilde{u}}^{n+1} -\underline {u}^{n} }
{\Delta t} + \dive(\,\underline{\widetilde{u}}^{n+\theta} \otimes (\rho\underline{u})\,) -\dive
(\mu_{\,tot}\,\ggrad \underline{\widetilde{u}}^{n+\theta}) =
\\
\displaystyle
 - \grad p^{n-1+\theta} + \dive (\rho\,\underline {u})\,\underline{\widetilde{u}}^{n+\theta} +(\Gamma\,\underline{u}_{\,i})^{n+\theta_S}-\Gamma^n\,\,\underline{\widetilde{u}}^{n+\theta}
\\
\begin{array}{c}
\displaystyle
- \rho\,\tens{K}_{\,d}^{n}\,\,\underline{\widetilde{u}}^{n+\theta} - (\rho\,\tens{K}_{\,e}\ \underline{u})^{n+\theta_S} + (\underline{T}_{s}^{\,exp})^{\,n+\theta_S} + T_{s}^{\,imp}\,\,\underline{\widetilde{u}}^{n+\theta}
\\
\displaystyle
+[\dive (\mu_{\,tot}\,^t\ggrad \underline {u})]^{n+\theta_S}-\frac {2} {3}[\,\grad (\mu_{\,tot}\,\dive \underline {u})]^{n+\theta_S} + (\rho -\rho_0) \underline {g}
 - (\underline{turb})^{n+\theta_S}
\end{array}
\end{array}
\end{equation}
o\`u, par souci de simplification, on a pos\'e :
\begin{equation}
\mu_{\,tot}=
\begin{cases}
\mu+\mu_t & \text{pour les mod�les � viscosit� turbulente ou en LES}, \\
\mu & \text{pour les mod�les au second ordre ou en laminaire}
\end{cases} \ 
\end{equation}
\\
et :
\begin{equation}
\underline{turb}^{n}=
\begin{cases}
\displaystyle\frac {2}{3}\grad (\rho^{n}\,k^{n}) & \text{pour les mod�les � viscosit� turbulente}, \\
\dive(\rho^{n}\,\tens{R}^n) & \text{pour les mod�les au second ordre},\\
0 & \text{en laminaire ou en LES}\\
\end{cases}
\end{equation}
Par analogie avec l'�criture du $\theta$-sch�ma pour une variable scalaire, $\,
\underline {\widetilde{u}}^{n+\theta}$ est interpol�e � partir de la vitesse
pr�dite $\underline {\widetilde{u}}^{n+1}$ de la mani\`ere suivante\footnote{si
$\theta=1/2$, ou qu'une extrapolation est utilis�e, l'ordre 2 n'est obtenu que si
le pas de temps $\Delta t$ est uniforme en temps et en espace.}~: 
\begin{equation}
\underline {\widetilde{u}}^{n+\theta}=\theta\, \underline
{\widetilde{u}}^{n+1}+(1-\theta)\, \underline {u}^{n}\\ 
\end{equation}
Avec :
\begin{equation}
\left\{
\begin{array}{ll}
\theta = 1   & \text{Pour un sch\'ema de type Euler implicite d'ordre 1.}\\
\theta = 1/2 & \text{Pour un sch\'ema de type Cranck-Nicolson d'ordre 2.}\\
\end{array}
\right.
\end{equation}

L'�quation (\ref{Base_Preduv_eq_di1}) est alors r��crite sous la forme :

\begin{equation}\label{Base_Preduv_eq_di2}
\begin{array}{c}
\displaystyle \underbrace{\left(\frac{\rho}{\Delta t} -\theta \,\dive (\rho\,\underline {u}) +\theta \,\, \Gamma^n +
\theta \,\, \rho\,\tens{K}_{\,d}^n-\theta \,T_s^{\,imp} \right)}_{\displaystyle f_s^{imp}}\, (\underline {\,\widetilde{u}}^{n+1} -\underline {u}^{n})
\\
 +\, \theta\, \dive(\underline {\widetilde{u}}^{n+1} \otimes (\rho\underline{u}))-\, \theta\,\dive (\mu_{\,tot}\,\ggrad \underline {\widetilde{u}}^{n+1}) =
\\
-\,(1-\theta)\, \dive(\underline {u}^{n} \otimes (\rho\underline{u})) +\,(1-\theta)\,\dive (\mu_{\,tot}\,\ggrad \underline {u}^{n})
\\
f_s^{exp}\left\{
\begin{array}{c}
\displaystyle 
- \grad p^{n-1+\theta} + \dive (\rho\,\underline {u})\,\underline{u}^{n} +\,(\,\Gamma^{n}\,\underline{u}_{\,i}\,)^{n+\theta_S}- \Gamma^n\,\,\underline{u}^{n}
\\
\displaystyle
-(\,\rho\,\tens{K}_{\,e}\ \underline{u}\,)^{n+\theta_S} -\rho\,\tens{K}_{\,d}^n\ \underline{u}^{n}+ (\underline{T}_{s}^{\,exp})^{\,n+\theta_S} + T_s^{\,imp}\,\,\underline {u}^{n} 
\\
\displaystyle
+[\dive (\mu_{\,tot}\,^t\ggrad \underline {u}\,)]^{n+\theta_S}-\frac {2} {3}[\,\grad (\mu_{\,tot}\,\dive \underline {u}\,)]^{n+\theta_S} + (\rho -\rho_0) \underline {g}-(\underline{turb})^{n+\theta_S}
\end{array}
\right.
\end{array}
\end{equation}

d'o� l'�quation r�solue par le sous-programme \fort{codits} :
\begin{equation}\begin{array}{c}
\displaystyle
f_s^{\,imp}(\underline {\widetilde{u}}^{n+1}-\underline {u}^{n}) + \theta\, \dive(\underline{\widetilde{u}}^{n+1} \otimes (\rho
\underline{u})) - \theta\,\dive (\,\mu_{\,tot}\,\ggrad \underline{\widetilde{u}}^{n+1}) = 
\\\\
\displaystyle
-(1-\theta)\,\dive(\underline{u}^{n} \otimes (\rho \underline{u}))+(1-\theta)\,\dive (\,\mu_{\,tot}\,\ggrad \underline{u}^{n})
+ \underline{f}_{\,s}^{\,exp}
\end{array}
\end{equation}
La m\'ethode de discr\'etisation spatiale est d\'evelopp\'ee dans le sous-programme \fort{codits}.\\



\minititre{Remarques :}
{\tiny$\blacksquare$} Dans le cas standard sans extrapolation, le terme
$-\,T_s^{\,imp}$ n'est ajout� � $f_s^{\,imp}$ que s'il est positif afin de ne
pas affaiblir la dominance de la diagonale de la matrice � inverser.\\ 
{\tiny$\blacksquare$} Si une extrapolation est utilis�e, par contre,
$\,T_s^{\,imp}$ est ajout� � $f_s^{\,imp}$ quel que soit son signe. En effet, l'id�e intuitive qui
consiste � prendre~: 
\begin{equation}
\begin{cases}
\displaystyle
(\underline{T}_{s}^{\,exp} + T_{s}^{\,imp}\,\underline {u})^{\,n+\theta_S} &
\text{si } T_{s}^{\,imp} > 0\\ 
\displaystyle
(\underline{T}_{s}^{\,exp})^{\,n+\theta_S} + T_{s}^{\,imp}\,\underline{u}^{n+\theta} &\text{sinon}\\
\end{cases}
\end{equation} 
aboutit � une incoh�rence dans le traitement si $T_s^{imp}$ change de signe
entre deux pas de temps.\\ 
{\tiny$\blacksquare$} la partie diagonale $\tens{K}_{\,d}$ du terme
de perte de charge est utilis�e dans $f_s^{\,imp}$. En fait, pour \^etre rigoureux,
il faudrait ne retenir que les contributions positives (point signal\'e dans le
sous-programme utilisateur associ\'e \fort{uskpdc}). Cette prise en compte sera \`a am\'eliorer.\\
{\tiny$\blacksquare$} Le terme $\theta\,\Gamma^{n}-\theta\,\dive
(\rho\,\underline {u})$ ne pose pas de probl�me pour la 
dominance de la diagonale de la matrice car il est exactement compens� par le
terme de convection (cf. \fort{covofi}). 


%                      Code_Saturne version 1.3
%                      ------------------------
%
%     This file is part of the Code_Saturne Kernel, element of the
%     Code_Saturne CFD tool.
%
%     Copyright (C) 1998-2007 EDF S.A., France
%
%     contact: saturne-support@edf.fr
%
%     The Code_Saturne Kernel is free software; you can redistribute it
%     and/or modify it under the terms of the GNU General Public License
%     as published by the Free Software Foundation; either version 2 of
%     the License, or (at your option) any later version.
%
%     The Code_Saturne Kernel is distributed in the hope that it will be
%     useful, but WITHOUT ANY WARRANTY; without even the implied warranty
%     of MERCHANTABILITY or FITNESS FOR A PARTICULAR PURPOSE.  See the
%     GNU General Public License for more details.
%
%     You should have received a copy of the GNU General Public License
%     along with the Code_Saturne Kernel; if not, write to the
%     Free Software Foundation, Inc.,
%     51 Franklin St, Fifth Floor,
%     Boston, MA  02110-1301  USA
%
%-----------------------------------------------------------------------
%

%%%%%%%%%%%%%%%%%%%%%%%%%%%%%%%%%%
%%%%%%%%%%%%%%%%%%%%%%%%%%%%%%%%%%
\section{Mise en \oe uvre}
%%%%%%%%%%%%%%%%%%%%%%%%%%%%%%%%%%
%%%%%%%%%%%%%%%%%%%%%%%%%%%%%%%%%%
La num\'ero de la phase trait\'ee fait partie des arguments de \fort{turrij}. On
omettra volontairement de le pr\'eciser dans ce qui suit, on indiquera par $(\ )$ la
notion de tableau s'y rattachant.

\etape{Calcul des termes de production $\tens{\mathcal{P}}$}
\begin{itemize}
\item [$\star$] Initialisation \`a z\'ero du tableau \var{PRODUC} dimensionn\'e \`a $\var{NCEL}\times 6$.
\item [$\star$] On appelle trois fois \fort{grdcel} pour calculer les gradients des composantes de la vitesse $u$, $v$ et
$w$ prises au temps $n$.

Au final, on a :\\
$\displaystyle
\begin{array} {ll}
\var{PRODUC(1,IEL)} = & \displaystyle - 2 \left[ R_{11}^{\,n} \frac{\partial u^{\,n}} {\partial x} +R_{12}^{\,n} \frac{\partial u^{\,n}} {\partial y}+R_{13}^{\,n} \frac{\partial u^{\,n}} {\partial z} \right] \text{        (production de $R_{11}^{\,n}$)}\\
\var{PRODUC(2,IEL)} = & \displaystyle - 2 \left[ R_{12}^{\,n} \frac{\partial v^{\,n}} {\partial x} +R_{22}^{\,n} \frac{\partial v^{\,n}} {\partial y}+R_{23}^{\,n} \frac{\partial v^{\,n}} {\partial z} \right] \text{        (production de $R_{22}^{\,n}$)}\\
\var{PRODUC(3,IEL)} = & \displaystyle - 2 \left[ R_{13}^{\,n} \frac{\partial w^{\,n}} {\partial x} +R_{23}^{\,n} \frac{\partial w^{\,n}} {\partial y}+R_{33}^{\,n} \frac{\partial w^{\,n}} {\partial z} \right] \text{        (production de $R_{33}^{\,n}$)}\\
\var{PRODUC(4,IEL)} = & \displaystyle - \left[ R_{12}^{\,n} \frac{\partial u^{\,n}} {\partial x} +R_{22}^{\,n} \frac{\partial u^{\,n}} {\partial y}+R_{23}^{\,n} \frac{\partial u^{\,n}} {\partial z} \right] \\
& \displaystyle - \left[ R_{11}^{\,n} \frac{\partial v^{\,n}} {\partial x} +R_{12}^{\,n} \frac{\partial v^{\,n}} {\partial y}+R_{13}^{\,n} \frac{\partial v^{\,n}} {\partial z} \right] \text{        (production de $R_{12}^{\,n}$)} \\
\var{PRODUC(5,IEL)} = & \displaystyle - \left[ R_{13}^{\,n} \frac{\partial u^{\,n}} {\partial x} +R_{23}^{\,n} \frac{\partial u^{\,n}} {\partial y}+R_{33}^{\,n} \frac{\partial u^{\,n}} {\partial z} \right] \\
& \displaystyle - \left[ R_{11}^{\,n} \frac{\partial w^{\,n}} {\partial x} +R_{12}^{\,n} \frac{\partial w^{\,n}} {\partial y}+R_{13}^{\,n} \frac{\partial w^{\,n}} {\partial z} \right] \text{        (production de $R_{13}^{\,n}$)} \\
\var{PRODUC(6,IEL)} = & \displaystyle - \left[ R_{13}^{\,n} \frac{\partial v^{\,n}} {\partial x} +R_{23}^{\,n} \frac{\partial v^{\,n}} {\partial y}+R_{33}^{\,n} \frac{\partial v^{\,n}} {\partial z} \right] \\
& \displaystyle - \left[ R_{12}^{\,n} \frac{\partial w^{\,n}} {\partial x} +R_{22}^{\,n} \frac{\partial w^{\,n}} {\partial y}+R_{23}^{\,n} \frac{\partial w^{\,n}} {\partial z} \right]  \text{        (production de $R_{23}^{\,n}$)}
\end{array}
$
\end{itemize}

\etape{Calcul du gradient de la masse volumique $\rho^n$ prise au d\'ebut du pas
de temps courant\footnote{{\it i.e.} calcul\'ee \`a partir des
variables du pas de temps pr\'ec\'edent $n$ si n\'ecessaire.} $(n+1)$}
Ce calcul n'a lieu que si les termes de gravit\'e doivent \^etre pris en compte
($\var{IGRARI()} =1$).
\begin{itemize}
\item [$\star$] Appel de \fort{grdcel}  pour calculer le gradient de $\rho^n$
dans les trois directions de l'espace. Les conditions aux limites sur $\rho^n$
sont des conditions de Dirichlet puisque la valeur de $\rho^n$ aux faces de bord
$ik$ (variable \var{IFAC}) est connue et vaut $\rho_{\,b_{\,ik}}$. Pour \'ecrire les conditions aux limites
sous la forme habituelle, $$\rho_{\,b_{\,ik}} = \var{COEFA} + \var{COEFB}
\,\rho^n_{\,I'}$$ on pose alors $\var{COEFA}=
\var{PROPCE(IFAC,IPPROB(IROM(IPHAS)))}$ et $\var{COEFB} = \var{VISCB} = 0$.\\
$\var{PROPCE(1,IPPROB(IROM(IPHAS)))}$ (resp.$\var{VISCB}$) est utilis\'e en lieu
et place de l'habituel \var{COEFA} ($\var{COEFB}$), lors de l'appel \`a \fort{grdcel}.\\
On a donc :\\
$\displaystyle \var{GRAROX}= \frac{\partial \rho^n}{\partial x}\ $,$\displaystyle \ \var{GRAROY}= \frac{\partial
\rho^n}{\partial y}$ et $
\displaystyle \ \var{GRAROZ}= \frac{\partial \rho^n}{\partial z}\ $.

\end{itemize}

Le gradient de $\rho^n$ servira \`a calculer les termes de production par effets de gravit\'e si ces derniers sont pris en compte.

\etape{Boucle \var{ISOU} de $1$ \`a $6$ sur les tensions de Reynolds}
Pour $\var{ISOU} = 1,2,3,4,5,6$, on r\'esout respectivement et dans
l'ordre  les
\'equations de $R_{11}$, $R_{22}$, $R_{33}$, $R_{12}$, $R_{13}$ et $R_{23}$ par
l'appel au sous-programme \fort{resrij}.\\
La r\'esolution se fait par incr\'ement $\delta {R}_{ij}^{\,n+1,k+1}$ , en utilisant la m\^eme m\'ethode que
celle d\'ecrite dans le sous-programme \fort{codits}. On adopte ici les m\^emes notations.
\var{SMBR} est le second membre du syst\`eme \`a inverser, syst\`eme portant sur
les incr\'ements de la variable. \var{ROVSDT} repr\'esente la diagonale de la
matrice, hors convection/diffusion.\\
On va r\'esoudre l'\'equation (\ref{Base_Turrij_Eq_Temp_Rij}) sous forme incr\'ementale en
utilisant \fort{codits}, soit :
\begin{equation}\label{Base_Turrij_Eq_Temp_deltaRij}
\begin{array}{ll}
&\displaystyle \underbrace{\left(\frac {\rho^n_L}{\Delta t^n}
+ \rho^n_L \,C_1\,\frac{\varepsilon^n_L}{k^n_L}(1-\frac{\delta_{ij}}{3})
 - m^n_{\,lm} + \Gamma_L\,+ max(-\alpha^n_{R_{ij}},0)\right)\,|\Omega_l|}
_{\text {$\var{ROVSDT}$ contribuant
\`a la diagonale de la matrice simplifi\'ee de \fort{matrix}}}\,(\delta{R}_{ij}^{\,n+1,p+1})_{\,L}\\\\
&  \underbrace{+\sum\limits_{m\in Vois(l)}\displaystyle \left[
 m^n_{\,lm} \delta R_{ij,\,f_{\,lm}}^{\,n+1,p+1}
- (\mu^n_{\,lm} + \gamma^n_{\,lm})\
\frac{({\delta R}_{ij}^{\,n+1,p+1})_{M}-({\delta R}_{ij}^{\,n+1,p+1})_{L})}{\overline{L'M'}}\,
S_{\,lm} \right]}_{\text { convection upwind pur et diffusion non reconstruite
relatives \`a la matrice simplifi\'ee de \fort{matrix}\footnotemark}} \\
% voir le texte de la footmark plus bas
&= - \displaystyle\frac {\rho^n_L}{\Delta t^n}\,\left(\,(R^{\,n+1,p}_{ij})_L - (R^{\,n}_{ij})_L\,\right)\\
&-\,\underbrace{\displaystyle\int_{\Omega_l} \left(
\dive\,[\,(\rho\,\vect{u})^n\,R^{\,n+1,p}_{ij} - (\mu^n\,+ \gamma^n\,)\,
\grad{R^{\,n+1,p}_{ij}}\,]\right)\,d\Omega}_{\text {convection et diffusion
trait\'ees par \fort{bilsc2}}}\\
&+\displaystyle \int_{\Omega_l} \left[\,\mathcal{P}^{\,n+1,p}_{ij} + \mathcal{G}^{\,n+1,p}_{ij}
- \displaystyle\rho^n \,C_1\,\frac{\varepsilon^n}{k^n}\left[R^{\,n+1,p}_{ij}-
\frac{2}{3}\,k^n\,\delta_{ij}\right] + \phi^{\,n+1,p}_{ij,2} +
\phi^{\,n+1,p}_{ij,w}\,\right]\, d\Omega \\
& + \displaystyle\int_{\Omega_l} \left[- \frac{2}{3} \rho^n \varepsilon^n \delta_{ij}
 + \Gamma\,(\,R^{\,in}_{ij} - R^{\,n+1,p}_{ij}\,) +
\alpha^n_{R_{ij}}\,R^{\,n+1,p}_{ij}+ \beta^n_{R_{ij}}\right]\, d\Omega\\
&+ \sum\limits_{m\in
Vois(l)}\displaystyle \left[\ \tens{E}^n\,\grad{R}^{\,n+1,p}_{ij} \right]_{\,lm}\,.\,\vect{n}_{\,lm}S_{\,lm}\\
&+ \sum\limits_{m\in Vois(l)}\displaystyle \left[\
\tens{D}^n\,\grad{R}^{\,n+1,p}_{ij} \right]_{\,lm}\,.\,\vect{n}_{\,lm}S_{\,lm}\\
&- \sum\limits_{m\in Vois(l)} \gamma^n_{\,lm} \left( \grad{R}^{\,n+1,p}_{ij}\,.\,\vect{n}_{\,lm} \right)  S_{\,lm}\\
&+ \sum\limits_{m\in Vois(l)}  m^n_{\,lm}\,(R^{\,n+1,p}_{ij})_L\\
\end{array}
\end{equation}
% si on ne fait pas comme ca, il n'apparait pas
\footnotetext[\thefootnote]{Si $\var{IRIJNU} = 1$, on remplace  $\mu^n_{\,lm}$ par $(\mu +
\mu_t)^n_{\,lm}$ dans l'expression de la diffusion non reconstruite
associ\'ee \`a la matrice simplifi\'ee de \fort{matrix} ($\mu_t$ d\'esigne la
viscosit\'e turbulente calcul\'ee comme en $k-\varepsilon$).}

o\`u on rappelle :\\
pour $n$ donn\'e entier positif, on d\'efinit la suite
 $({R}_{ij}^{\,n+1,p})_{p \in \grandN}$
 par :
\begin{equation}\notag
\left\{\begin{array}{l}
{R}_{ij}^{\,n+1,0} = {R}_{ij}^{\,n}\\
{R}_{ij}^{\,n+1,p+1} = {R}_{ij}^{\,n+1,p} + \delta{R}_{ij}^{\,n+1,p+1} \\
\end{array}\right.
\end{equation}
$(\delta{R}_{ij}^{\,n+1,p+1})_{\,L}$ d\'esigne la valeur sur l'\'el\'ement
$\Omega_l$ du $\text{$(\,p+1\,)$-i\`eme}$ incr\'ement de ${R}_{ij}^{\,n+1}$,
$ m^n_{\,lm}$ le flux de masse \`a l'instant $n$ \`a travers la face $lm$,
$\delta R_{ij,\,f_{\,lm}}^{\,n+1,p+1}$ vaut $({\delta
R}_{ij}^{\,n+1,p+1})_{L}$  si $ m^n_{\,lm} \geqslant 0$, $({\delta
R}_{ij}^{\,n+1,p+1})_{M}$ sinon,
$\mathcal{P}^{\,n+1,p}_{ij}$, $\phi^{\,n+1,p}_{ij,2}$, $\phi^{\,n+1,p}_{ij,w}$ les valeurs
des quantit\'es associ\'ees correspondant \`a l'incr\'ement
$(\delta{R}_{ij}^{\,n+1,p})$.\\



Tous ces termes sont calcul\'es comme suit :
\begin{itemize}
\item Terme de gauche de l'\'equation (\ref{Base_Turrij_Eq_Temp_deltaRij})\\
Dans \fort{resrij} est calcul\'ee la variable \var{ROVSDT}. Les autres
termes sont compl\'et\'es par \fort{codits}, lors de la construction de la matrice simplifi\'ee , {\it via} un
appel au sous-programme \fort{matrix}. La quantit\'e
 $(\mu^n_{\,lm} + \gamma^n_{\,lm})$ \`a la face $lm$ est calcul\'ee lors de l'appel \`a
\fort{visort}.\\
\item Second membre de l'\'equation (\ref{Base_Turrij_Eq_Temp_deltaRij})\\
Le premier terme non d\'etaill\'e est calcul\'e par le sous-programme
\fort{bilsc2}, qui applique le sch\'ema convectif choisi par l'utilisateur, qui
reconstruit ou non selon le souhait de l'utilisateur les gradients intervenants
dans la convection-diffusion.\\
Les termes sans accolade sont, eux, compl\`etement explicites et ajout\'es au fur et
\`a mesure dans \var{SMBR} pour former
l'expression $f^{\,exp}_s$ de \fort{codits}.
\end{itemize}
On d\'ecrit ci-dessous les \'etapes de \fort{resrij} :
\begin{itemize}

\item DELTIJ = 1, pour $\var{ISOU} \leqslant 3$ et DELTIJ = 0  Si $\var{ISOU} >
3$. Cette valeur repr\'esente le symbole de Kroeneker $\delta_{ij}$.

\item Initialisation \`a z\'ero de \var{SMBR} (tableau contenant le second
membre) et \var{ROVSDT} (tableau contenant la diagonale de la matrice sauf celle
relative \`a la contribution de la
diagonale des op\'erateurs de convection et de diffusion lin\'earis\'es
\footnote{qui correspondent aux sch\'emas convectif upwind pur et diffusif sans
reconstruction.}), tous deux de dimension $\var{NCEL}$.

\item Lecture et prise en compte des termes sources utilisateur pour la variable $R_{ij}$

Appel \`a \fort{ustsri} pour charger les termes sources utilisateurs. Ils sont
stock\'es comme suit. Pour la cellule $\Omega_l$ de centre $L$, repr\'esent\'ee par $\var{IEL}$, on a :\\
\begin{equation}\notag
\left\{\begin{array}{lll}
&\var{ROVSDT(IEL)}&= |\Omega_l| \ \alpha_{R_{ij}}\\
&\var{SMBR(IEL)}&=|\Omega_l| \ \beta_{R_{ij}}\\
\end{array}\right.
\end{equation}
On affecte alors les valeurs ad\'equates au second membre \var{SMBR} et \`a la
diagonale \var{ROVSDT} comme suit :
\begin{equation}\notag
\left\{\begin{array}{lll}
&\var{SMBR(IEL)} &= \var{SMBR(IEL)} +\ |\Omega_l| \ \alpha_{R_{ij}} \ (R^n_{ij})_L \\
&\var{ROVSDT(IEL)}&= \text{max }(-\ |\Omega_l| \ \alpha_{R_{ij}},0)\\
\end{array}\right.
\end{equation}
La valeur de $ \var{ROVSDT}$ est ainsi calcul\'ee pour des raisons de stabilit\'e
num\'erique. En effet, on ne rajoute sur la diagonale que les valeurs positives,
ce qui correspond physiquement \`a impliciter les termes de rappel uniquement.
\item{Calcul du terme source de masse  si $\Gamma_L > 0$}

Appel de \fort{catsma} et incr\'ementation si n\'ecessaire de \var{SMBR} et
\var{ROVSDT} {\it via} :\\
\begin{equation}\notag
\left\{\begin{array}{lll}
\displaystyle \var{SMBR(IEL)} = \var{SMBR(IEL)} + |\Omega_l| \ \Gamma_L \
\left[(R^{\,in}_{ij})_L - (R^{\,n}_{ij})_L \right] \\
\displaystyle \var{ROVSDT(IEL)}=\var{ROVSDT(IEL)} + |\Omega_l| \ \Gamma_L
\end{array}\right.
\end{equation}
\item Calcul du terme d'accumulation de masse et du terme instationnaire

On stocke $\displaystyle \var{W1}= \int_{\Omega_l}\dive\,(\rho\,\vect{u})\,d\Omega$
calcul\'e par \fort{divmas} \`a l'aide des flux de masse aux faces internes
$ m^n_{\,lm}=\sum\limits_{m\in Vois(l)}{(\rho \vect{u})_{\,lm}^n} \text{.}\,
\vect{S}_{\,lm} $ (tableau \var{FLUMAS}) et des flux de masse aux bords  $ m^n_{\,b_{lk}} = \sum\limits_{k\in{\gamma_b(l)}}{(\rho \vect{u})_{\,{b}_{lk}}^n} \text{.}\,
\vect{S}_{\,{b}_{lk}} $ (tableau \var{FLUMAB}).
On incr\'emente ensuite \var{SMBR} et \var{ROVSDT}.
\begin{equation}\notag
\left\{\begin{array}{lll}
&\var{SMBR(IEL)} &= \var{SMBR(IEL)} + \var{ICONV}\  (R^n_{ij})_L\,(\displaystyle
\int_{\Omega_l}\dive\,(\rho\,\vect{u})\ d\Omega) \\
&\var{ROVSDT(IEL)}& = \var{ROVSDT(IEL)} +  \var{ISTAT}\,\displaystyle
\frac{\rho^n_L \ |\Omega_l|}{\Delta t^n} -  \var{ICONV}\ (\displaystyle
\int_{\Omega_l}\dive\,(\rho\,\vect{u})\ d\Omega) \\
\end{array}\right.
\end{equation}
\item Calcul des termes sources de production, des termes $\displaystyle
\phi_{\,ij,1}+\phi_{\,ij,2}$ et de la dissipation~$\displaystyle-\frac{2}{3} \varepsilon\,\delta_{\,ij}$ :

On effectue une boucle d'indice \var{IEL} sur les cellules $\Omega_l$ de centre $L$ :
\begin{itemize}
\item [$\Rightarrow$] $\displaystyle \var{TRPROD}= \frac{1}{2} (\mathcal{P}^n_{ii})_L = \frac{1}{2} \left[ \var{PRODUC(1,IEL)} +  \var{PRODUC(2,IEL)} +  \var{PRODUC(3,IEL)} \right] $
\item [$\Rightarrow$] $\displaystyle \var{TRRIJ }= \frac{1}{2} (R^n_{ii})_L $
\item [$\Rightarrow$] $\displaystyle \var{SMBR(IEL)} =\ \var{SMBR(IEL)}\ +$\\
$\ \displaystyle\rho^n_L |\Omega_l| \left[ \displaystyle
\frac{2}{3}\,\delta_{\,ij} \left( \ \displaystyle \frac{ C_2}{2}\,(\mathcal{P}^n_{ii})_L\ +
(C_1-1)\ \varepsilon^n_L\, \right)\right.$\\
$ + \left.\ (1-C_2) \ \var{PRODUC(ISOU,IEL)} -
\displaystyle C_1\ \frac{2\,\varepsilon^n_L}{(R^n_{ii})_L}\ (R^n_{ij})_L \right]$
\item [$\Rightarrow$] $\displaystyle \var{ROVSDT(IEL)} = \var{ROVSDT(IEL)} +
\rho^n_L \ |\Omega_l| \ (- \displaystyle \frac{1}{3} \ \,\delta_{\,ij} + 1) \ C_1
\ \frac{2\ \varepsilon^n_L}{(R^n_{ii})_L}$
\end{itemize}
\item Appel de \fort{rijech} pour le calcul des termes d'\'echo de paroi
 $\phi^n_{ij,w}$ si $\var{IRIJEC()}=1$ et ajout dans \var{SMBR}.\\
$\var{SMBR} = \var{SMBR} + \phi^n_{ij,w}$\\
Suivant son mode de calcul (\var{ICDPAR}), la distance � la paroi est directement accessible
par \var{RA(IDIPAR+IEL-1)} (\var{|ICDPAR|} = 1) ou bien
est calcul\'ee \`a partir de $\var{IA(IIFAPA(IPHAS)+IEL - 1)}$,
qui donne pour l'\'el\'ement $\var{IEL}$ le num\'ero de la face de bord
paroi la plus  proche (\var{|ICDPAR|} = 2). Ces tableaux ont \'et\'e renseign\'e une fois pour toutes au
d\'ebut de calcul.

\item  Appel de \fort{rijthe} pour le calcul des termes de gravit\'e $\mathcal{G}^n_{ij}$ et ajout dans \var{SMBR}.

Ce calcul n'a lieu que si $\var{IGRARI()} = 1$.
$ \var{SMBR} = \var{SMBR} + \mathcal{G}^n_{ij}$
\item Calcul de la partie explicite du terme de diffusion
 $\dive{\,\left[\tens{A}\,\grad{R}^{\,n}_{ij}\right]}$, plus pr\'ecis\'ement
des contributions du terme extradiagonal pris aux faces purement internes
(remplissage du tableau \var{VISCF}), puis aux faces de bord (remplissage du
tableau \var{VISCB}).
\begin{itemize}
\item [$\star$] Appel de \fort{grdcel} pour le calcul du gradient de
$R^{\,n}_{ij}$ dans chaque direction. Ces gradients sont respectivement
stock\'es dans les tableaux de travail \var{W1}, \var{W2} et \var{W3}.

\item [$\star$] boucle d'indice \var{IEL} sur les cellules $\Omega_l$ de centre
$L$ pour le
calcul de $\tens{E}^n\,\grad{R}^{\,n}_{ij}$ aux cellules dans un premier temps :\\
\begin{itemize}
\item [$\Rightarrow$] $\displaystyle \var{TRRIJ}= \frac{1}{2} (R^{\,n}_{ii})_L $
\item [$\Rightarrow$] $\displaystyle \var{CSTRIJ} = \rho^n_L\ C_S \ \displaystyle\frac{(R^n_{ii})_L}{2\,\varepsilon^n_L}$
\item [$\Rightarrow$] $\displaystyle \var{W4(IEL)} = \rho^n_L\ C_S\
\displaystyle\frac{(R^n_{ii})_L}{2\,\varepsilon^n_L} \left[\,(R^{\,n}_{12})_L \ \var{W2(IEL)} +
(R^{\,n}_{13})_L \ \var{W3(IEL)}\,\right]$
\item [$\Rightarrow$] $\displaystyle \var{W5(IEL)} = \rho^n_L\ C_S\
\displaystyle\frac{(R^n_{ii})_L}{2\,\varepsilon^n_L} \left[\,(R^{\,n}_{12})_L \ \var{W1(IEL)} +
(R^{\,n}_{23})_L \ \var{W3(IEL)}\,\right]$
\item [$\Rightarrow$] $\displaystyle \var{W6(IEL)} = \rho^n_L\ C_S\
\displaystyle\frac{(R^n_{ii})_L}{2\,\varepsilon^n_L} \left[\,(R^{\,n}_{13})_L \ \var{W1(IEL)} + (R^{\,n}_{23})_L \ \var{W2(IEL)}\,\right]$
\end{itemize}



\item [$\star$] Appel de \fort{vectds}\footnote{Le r\'esultat est stock\'e dans
\var{VISCF} et \var{VISCB}. Dans \fort{vectds}, les valeurs aux faces internes
sont interpol\'ees lin\'eairement sans reconstruction et \var{VISCB} est mis \`a
z\'ero.} pour assembler $\displaystyle\left[ \tens{E}^n\,\grad{R}^{\,n}_{ij}
\right]\,.\,\vect{n}_{\,lm}S_{\,lm}$ aux faces $lm$.
\item [$\star$] Appel de \fort{divmas} pour calculer la divergence du flux d\'efini par \var{VISCF} et \var{VISCB}.
Le r\'esultat est stock\'e dans \var{W4}.\\
Ajout au second membre \var{SMBR}.\\
\var{SMBR} = \var{SMBR} + \var{W4}
\end{itemize}

A l'issue de cette \'etape, seule la partie extradiagonale de la diffusion prise
enti\`erement explicite~:
 $$\sum\limits_{m\in
Vois(l)}\left[\ \tens{E}^n\,\grad{R}^{\,n}_{ij} \right]_{\,lm}\,.\,\vect{n}_{\,lm}S_{\,lm}$$ a \'et\'e calcul\'ee.\\

\item Calcul de la partie diagonale du terme de diffusion\footnote{Seule la
composante normale  du  gradient de $R_{ij}$ aux faces sera implicite.} :\\
Comme on l'a d\'eja vu, une partie de cette contribution sera trait\'ee en
implicite (celle relative \`a la composante normale du gradient) et donc
ajout\'ee au second membre par \fort{bilsc2} ; l'autre
partie sera explicite et prise en compte dans $f_s^{\,exp}$.
\begin{itemize}
\item [$\star$] On effectue une boucle d'indice \var{IEL} sur les cellules
$\Omega_l$ de centre $L$ :
\begin{itemize}
\item [$\Rightarrow$] $\displaystyle \var{TRRIJ }= \frac{1}{2} (R^{\,n}_{ii})_L $
\item [$\Rightarrow$] $\displaystyle \var{CSTRIJ} = \rho^n_L \ C_S \ \frac{(R^{\,n}_{ii})_L}{2\,\varepsilon^n_L}$
\item [$\Rightarrow$] $\displaystyle \var{W4(IEL)} = \rho^n_L \ C_S \
\frac{(R^{\,n}_{ii})_L}{2\,\varepsilon^n_L} \ (R^{\,n}_{11})_L$
\item [$\Rightarrow$] $\displaystyle \var{W5(IEL)} = \rho^n_L \ C_S \ \frac{(R^{\,n}_{ii})_L}{2\,\varepsilon^n_L}\ (R^n_{22})_L$
\item [$\Rightarrow$] $\displaystyle \var{W6(IEL)} = \rho^n_L \ C_S \ \frac{(R^{\,n}_{ii})_L}{2\,\varepsilon^n_L} \ (R^n_{33})_L$
\end{itemize}

%\item Traitement du parall\'elisme et de la p\'eriodicit\'e.

\item [$\star$] On effectue une boucle d'indice \var{IFAC} sur les faces
purement internes $lm$ pour remplir le tableau \var{VISCF} :
\begin{itemize}
\item [$\Rightarrow$] Identification des cellules $\Omega_l$ et $\Omega_m$ de
centre respectif $L$ (variable \var{II}) et $M$ (variable \var{JJ}), se trouvant de chaque c\^ot\'e de la face
$lm$\footnote{La normale \`a la face est orient\'ee de L vers M.}.
\item [$\Rightarrow$] Calcul du carr\'e de la surface de la face. La valeur est
stock\'ee dans le tableau \var{SURFN2}.
\item [$\Rightarrow$] Interpolation du gradient de $R^{\,n}_{ij}$ \`a la face
$lm$ (gradient facette $\left[\grad{R}^{\,n}_{ij}\right]_{\,lm}$) :
\begin{equation}\notag
\left\{\begin{array}{ll}
\var{GRDPX} &= \displaystyle \frac{1}{2} \left(\var{W1(II)} + \var{W1(JJ)}
\right) \\
&\\
\var{GRDPY} &= \displaystyle \frac{1}{2} \left(\var{W2(II)} + \var{W2(JJ)} \right) \\
&\\
\var{GRDPZ} &= \displaystyle \frac{1}{2} \left(\var{W3(II)} + \var{W3(JJ)} \right)
\end{array}\right.
\end{equation}
\item [$\Rightarrow$] Calcul du gradient de $R^{\,n}_{ij}$ normal \`a la face
$lm$, $\left[\grad{R}^{\,n}_{ij}\right]_{\,lm}.\vect{n}_{\,lm}\,S_{\,lm}$.\\

$\displaystyle \var{GRDSN} =  \var{GRDPX} \ \var{SURFAC(1,IFAC)} + \var{GRDPY} \ \var{SURFAC(2,IFAC)} +  \var{GRDPZ} \ \var{SURFAC(3,IFAC)}$
$\var{SURFAC}$ est le vecteur surface de la face \var{IFAC}.


\item [$\Rightarrow$] calcul de
 $\left[\grad{R^{\,n}_{ij}} - (\grad
R^{\,n}_{ij}\,.\,\vect{n}_{\,lm})\vect{n}_{\,lm}\right]$, les vecteurs \'etant
calcul\'es \`a la face $lm$ :
\begin{equation}\notag
\left\{\begin{array}{lll}
&\displaystyle \var{GRDPX} &= \var{GRDPX} - \displaystyle\frac{\var{GRDSN}}{\var{SURFN2}} \ \var{SURFAC(1,IFAC)}\\
&&\\
&\displaystyle \var{GRDPY} &= \var{GRDPY} - \displaystyle\frac{\var{GRDSN}}{\var{SURFN2}} \ \var{SURFAC(2,IFAC)} \\
&&\\
&\displaystyle \var{GRDPZ} &= \var{GRDPZ} - \displaystyle\frac{\var{GRDSN}}{\var{SURFN2}} \ \var{SURFAC(3,IFAC)}
\end{array}\right.
\end{equation}
\item [$\Rightarrow$] finalisation du calcul de l'expression totalement
explicite
 $$\left[ \tens{D}^n\,\left( \grad{R^{\,n}_{ij}} - (\grad R^{\,n}_{ij}\,.\,\vect{n}_{\,lm})\,\vect{n}_{\,lm}\right) \right]\,.\,\vect{n}_{\,lm}$$
\begin{equation}\notag
\begin{array} {ll}
\displaystyle \var{VISCF} = &
 \displaystyle\frac{1}{2} (\ \var{W4(II)} +\ \var{W4(JJ)}) \ \var{GRDPX} \
\var{SURFAC(1,IFAC)})\ + \\
&\\
&  \displaystyle\frac{1}{2} (\ \var{W5(II)} +\ \var{W5(JJ)}) \ \var{GRDPY} \
\var{SURFAC(2,IFAC)})\ + \\
&\\
&  \displaystyle\frac{1}{2} (\ \var{W6(II)} +\ \var{W6(JJ)}) \ \var{GRDPZ} \ \var{SURFAC(3,IFAC)})
\end{array}
\end{equation}
\end{itemize}

\item [$\star$] Mise \`a z\'ero du tableau \var{VISCB}.

\item [$\star$] Appel de \fort{divmas} pour calculer la divergence de~:
 $$\tens{D}^{\,n}\,\left( \grad{R^{\,n}_{ij}} - (\grad R^{\,n}_{ij}\,.\,\vect{n}_{\,lm})\vect{n}_{\,lm}\right)$$ d\'efini aux faces dans \var{VISCF} et \var{VISCB}.

Le r\'esultat est stock\'e dans le tableau \var{W1}.\\
Ajout au second membre \var{SMBR}.\\
$\var{SMBR} = \var{SMBR} + \var{W1}$
\end{itemize}
\item Calcul de la viscosit\'e orthotrope $\gamma^n_{\,lm}$ \`a la face $lm$ de la variable principale
$R^{\,n}_{ij}$\\
Ce calcul permet au sous-programme \fort{codits} de compl\'eter le second membre
\var{SMBR} par :
\begin{equation}
\begin{array} {ll}
& \sum\limits_{m\in Vois(l)}
\mu^n_{\,lm}\,\left(\grad{R}^{\,n}_{ij}\,.\,\vect{n}_{\,lm}\right)S_{\,lm}
 + \sum\limits_{m\in Vois(l)} \left(\grad{R}^{\,n}_{ij}
\,.\,\vect{n}_{\,lm}\right)\left[\tens{D}^{\,n}\,\vect{n}_{\,lm}\right]_{\,lm}\,.\,\vect{n}_{\,lm}\
S_{\,lm}\\
& = \sum\limits_{m\in Vois(l)}(\,\mu^n_{\,lm}\, + \,\gamma^n_{\,lm}\,)\,\left(\grad{R}^{\,n}_{ij}\,.\,\vect{n}_{\,lm}\right)S_{\,lm}
\end{array}
\end{equation}
sans pr\'eciser la nature de la face $lm$, {\it via} l'appel \`a \fort{bilsc2}  et de disposer de la quantit\'e
$(\mu^n_{\,lm}\, + \gamma^n_{\,lm})$ pour construire sa
matrice simplifi\'ee.\\
\begin{itemize}
\item [$\star$] On effectue une boucle d'indice \var{IEL} sur les cellules
$\Omega_l$ :
\begin{itemize}
\item [$\Rightarrow$] $\displaystyle \var{TRRIJ }= \frac{1}{2} (R^{\,n}_{ii})_L $
\item [$\Rightarrow$] $\displaystyle \var{RCSTE} = \rho^n_L \ C_S \ \frac{ (R^{\,n}_{ii})_L}{2\,\varepsilon^n_L} $
\item [$\Rightarrow$] $\displaystyle \var{W1(IEL)} = \mu^n +\rho^n_L \ C_S \ \frac{
(R^{\,n}_{ii})_L}{2\,\varepsilon^n_L}\ (R^n_{11})_L$
\item [$\Rightarrow$] $\displaystyle \var{W2(IEL)} = \mu^n + \rho^n_L \ C_S \ \frac{ (R^{\,n}_{ii})_L}{2\,\varepsilon^n_L}\ (R^n_{22})_L$
\item [$\Rightarrow$] $\displaystyle \var{W3(IEL)} = \mu^n + \rho^n_L \ C_S \ \frac{ (R^{\,n}_{ii})_L}{2\,\varepsilon^n_L}\ (R^n_{33})_L$
\end{itemize}

\item [$\star$] Appel de \fort{visort} pour calculer la viscosit\'e orthotrope
\footnote{Comme dans le sous-programme \fort{viscfa}, on multiplie la viscosit\'e par
$\displaystyle \frac{S_{\,lm}}{\overline{L'M'}}$, o\`u $S_{\,lm}$ et
$\overline{L'M'}$ repr\'esentent respectivement la surface de la face $lm$ et la
mesure alg\'ebrique du segment reliant les projections des centres des cellules
voisines sur la normale \`a la face. On garde dans ce sous-programme  la possibilit\'e d'interpoler la viscosit\'e aux cellules lin\'eairement ou d'utiliser une moyenne harmonique. La viscosit\'e au bord est celle de la cellule de bord correspondante.}
$ \gamma^n_{\,lm} = (\tens{D}^{\,n}\,\vect{n}_{\,lm}).\vect{n}_{\,lm}$ aux faces $lm$

Le r\'esultat est stock\'e dans les tableaux \var{VISCF} et \var{VISCB}.
\end{itemize}

\item appel de \fort{codits} pour la r\'esolution de l'\'equation de
convection/diffusion/termes sources de la variable $R_{ij}$. Le terme source est
\var{SMBR}, la viscosit\'e \var{VISCF} aux faces purement internes (
resp. \var{VISCB} aux faces de bord) et \var{FLUMAS} le flux de masse interne
 ( resp. \var{FLUMAB} flux de masse au bord). Le r\'esultat est la variable $R_{ij}$ au temps
$n+1$, donc $R^{\,n+1}_{ij}$. Elle est stock\'ee dans le tableau \var{RTP} des
variables mises \`a jour.

\end{itemize}

\etape{Appel de \fort{reseps} pour la r\'esolution de la variable $\varepsilon$}

On d\'ecrit ci-dessous le sous-programme \fort{reseps}, les commentaires du sous-programme \fort{resrij} ne seront pas r\'ep\'et\'es puisque les deux sous-programmes ne diff\`erent que par leurs termes sources.

\begin{itemize}
\item Initialisation \`a z\'ero de \var{SMBR} et \var{ROVSDT}.

\item{Lecture et prise en compte des termes sources utilisateur pour la variable $\varepsilon$ :}

Appel de \fort{ustsri} pour charger les termes sources utilisateurs. Ils sont
stock\'es dans les tableaux suivants :\\
pour la cellule $\Omega_l$ repr\'esent\'ee par $\var{IEL}$ de centre $L$, on a :
\begin{equation}\notag
\left\{\begin{array}{lll}
&\var{ROVSDT(IEL)}&= |\Omega_l| \ \alpha_{\varepsilon}\\
&\var{SMBR(IEL)}&=|\Omega_l| \ \beta_{\varepsilon}\\
\end{array}\right.
\end{equation}
On affecte alors les valeurs ad\'equates au second membre \var{SMBR} et \`a la
diagonale \var{ROVSDT} comme suit :
\begin{equation}\notag
\left\{\begin{array}{lll}
&\var{SMBR(IEL)} &= \var{SMBR(IEL)} +\ |\Omega_l| \ \alpha_{\,\varepsilon} \
\varepsilon^n_L \\
&\var{ROVSDT(IEL)}&= \text{max }(-\ |\Omega_l| \ \alpha_{\,\varepsilon},0)\\
\end{array}\right.
\end{equation}

\item{Calcul du terme source de masse si $\Gamma_L > 0$ :
\begin{equation}\notag
\left\{\begin{array}{lll}
&\displaystyle \var{SMBR(IEL)} = \var{SMBR(IEL)} + |\Omega_l| \ \Gamma_L \
(\varepsilon^{\,in}_L -\varepsilon^n_L) \\
&\displaystyle \var{ROVSDT(IEL)}= \var{ROVSDT(IEL)} + |\Omega_l| \ \Gamma_L
\end{array}\right.
\end{equation}
\item Calcul du terme d'accumulation de masse et du terme instationnaire \\
On stocke $\displaystyle \var{W1}= \int_{\Omega_l}\dive\,(\rho\,\vect{u})\,d\Omega$
calcul\'e par \fort{divmas} \`a l'aide des flux de masse internes et aux bords.\\
On incr\'emente ensuite \var{SMBR} et \var{ROVSDT}.
\begin{equation}\notag
\left\{\begin{array}{lll}
&\var{SMBR(IEL)} &= \var{SMBR(IEL)} + \var{ICONV}\ \varepsilon^n_L\,(\displaystyle
\int_{\Omega_l}\dive\,(\rho\,\vect{u})\ d\Omega) \\
&\var{ROVSDT(IEL)}& = \var{ROVSDT(IEL)} +  \var{ISTAT}\,\displaystyle
\frac{\rho^n_L \ |\Omega_l|}{\Delta t^n} -  \var{ICONV}\ (\displaystyle
\int_{\Omega_l}\dive\,(\rho\,\vect{u})\ d\Omega) \\
\end{array}\right.
\end{equation}

\item Traitement du terme de production
 $\displaystyle \rho\,C_{\varepsilon_1}\,\frac{\varepsilon}{k}\,\mathcal{P}$
 et du terme de dissipation $-\,\displaystyle \rho\,C_{\varepsilon_2}\,\frac{\varepsilon}{k}\,\varepsilon$ \\
pour cela, on effectue une boucle d'indice \var{IEL} sur les cellules $\Omega_l$
de centre $L$ :
\begin{itemize}
\item [$\Rightarrow$] $\displaystyle \var{TRPROD}= \frac{1}{2} (\mathcal{P}^n_{ii})_L = \frac{1}{2} \left[ \var{PRODUC(1,IEL)} +  \var{PRODUC(2,IEL)} +  \var{PRODUC(3,IEL)} \right] $
\item [$\Rightarrow$] $\displaystyle \var{TRRIJ }= \frac{1}{2} (R^n_{ii})_L $
\item [$\Rightarrow$] $\displaystyle \var{SMBR(IEL)} = \var{SMBR(IEL)} + \rho^n_L
|\Omega_l| \left[ -C_{\varepsilon_2} \ \frac{2\,(\varepsilon^n_L)^2}{(R^n_{ii})_L} + C_{\varepsilon_1} \ \frac{\varepsilon^n_L}{(R^n_{ii})_L}\ (\mathcal{P}^n_{ii})_L \right] $
\item [$\Rightarrow$] $\displaystyle \var{ROVSDT(IEL)} = \var{ROVSDT(IEL)} + C_{\varepsilon_2} \ \rho^n_L \ |\Omega_l| \ \frac{2\,\varepsilon^n_L}{(R^n_{ii})_L}$
\end{itemize}

\item Appel de \fort{rijthe} pour le calcul des termes de gravit\'e $\mathcal{G}^n_{\varepsilon}$ et ajout dans \var{SMBR}.

$ \var{SMBR} = \var{SMBR} + \mathcal{G}^n_{\varepsilon}$\\
Ce calcul n'a lieu que si $\var{IGRARI()} = 1$.

\item Calcul de la diffusion de $\varepsilon$ \\
 Le terme $\dive \left[\mu\, \grad(\varepsilon) + \tens{A'}\,\grad(\varepsilon)
\right]$ est calcul\'e exactement de la m\^eme mani\`ere que pour les tensions
de Reynolds $R_{ij}$ en rempla\c cant $\tens{A}$ par $\tens{A'}$.

\item Appel de \fort{codits} pour la r\'esolution de l'\'equation de
convection/diffusion/termes sources de la variable principale $\varepsilon$. Le
r\'esultat $\varepsilon^{\,n+1}$ est stock\'e dans le tableau \var{RTP} des
variables mises \`a jour.
}
\end{itemize}

\etape{clippings finaux}
On passe enfin dans le sous-programme  \fort{clprij} pour faire un clipping \'eventuel
des variables $R^{\,n+1}_{ij}$ et $\varepsilon^{\,n+1}$. Le sous-programme
\fort{clprij} est appel\'e\footnote{L'option
$\var{ICLIP} = 1$ consiste en un clipping minimal des variables $R_{ii}$ et
$\varepsilon$ en prenant la valeur absolue de ces variables puisqu'elles ne
peuvent \^etre que positives.} avec $\var{ICLIP} = 2$ . Cette option
\footnote{Quand la valeur des grandeurs $R_{ii}$ ou $\varepsilon$ est
n\'egative, on la remplace par le minimum entre sa valeur absolue et (1,1)
fois la valeur obtenue au pas de temps pr\'ec\'edent.} contient l'option $\var{ICLIP} = 1$  et permet de v\'erifier l'in\'egalit\'e de Cauchy-Schwarz sur les grandeurs extra-diagonales du tenseur $\tens{R}$ (pour $i \neq j$, $|R_{ij}|^2 \le R_{ii} R_{jj}$).


%%%%%%%%%%%%%%%%%%%%%%%%%%%%%%%%%%
%%%%%%%%%%%%%%%%%%%%%%%%%%%%%%%%%%
\section{Points \`a traiter}
%%%%%%%%%%%%%%%%%%%%%%%%%%%%%%%%%%
%%%%%%%%%%%%%%%%%%%%%%%%%%%%%%%%%%
Sauf mention explicite, $\phi$ repr\'esentera une tension de Reynolds ou la dissipation turbulente ($\phi = R_{ij} \ \text{ou} \ \varepsilon$).

\begin{itemize}
\item {La vitesse utilis\'ee pour le calcul de la production est explicite. Est-ce qu'une implicitation peut am\'eliorer la pr\'ecision temporelle de $\phi$ \footnote{Cette remarque peut \^etre g\'en\'eralis\'ee. En effet, peut-on envisager d'actualiser les variables d\'ej\`a r\'esolues (sans r\'eactualiser les variables turbulentes apr\`es leur r\'esolution)? Ceci obligerait \`a modifier les sous-programmes tels que \fort{condli} qui sont appel\'es au d\'ebut de la boucle en temps.} ?}
\item {Dans quelle mesure le terme d'\'echo de paroi est-il valide ? En effet, ce terme est remis en question par certains auteurs.}
\item {On peut envisager la r\'esolution d'un syst\`eme hyperbolique pour les
tensions de Reynolds afin d'introduire un couplage avec le champ de vitesse.}
\item {Le flux au bord \var{VISCB} est annul\'e dans le sous-programme
\fort{vectds}. Peut-on envisager de mettre au bord la valeur de la variable
concern\'ee \`a la cellule de bord correspondant? De m\^eme, il faudrait se
pencher sur les hypoth\`eses sous-jacentes \`a l'annulation des contributions
aux bords de \var{VISCB} lors du calcul de : $$\left[ \tens{D}^n\,\left( \grad{R^{\,n}_{ij}} - (\grad R^{\,n}_{ij}\,.\,\vect{n}_{\,lm})\,\vect{n}_{\,lm}\right) \right]\,.\,\vect{n}_{\,lm}.$$}
\item {Un probl\`eme de pond\'eration appara\^\i t plus g\'en\'eralement. Si on prend la partie explicite de $\tens{D}\,\grad(\phi)$, la pond\'eration aux faces internes utilise le coefficient $\displaystyle\frac{1}{2}$ avec pond\'eration s\'epar\'ee de $\tens{D}$ et $\grad(\phi)$, alors que pour $\tens{E}\,\grad(\phi)$, on calcule d'abord ce terme aux cellules pour ensuite l'interpoler lin\'eairement aux faces \footnote{Cette interpolation se fait dans \fort{vectds} avec des coefficients de pond\'eration aux faces.}. Ceci donne donc deux types d'interpolations pour des termes de m\^eme nature.}
\item {On laisse la possibilit\'e dans \fort{visort} d'utiliser une moyenne
harmonique aux faces. Est-ce que ceci est valable puisque les interpolations
utilis\'ees lors du calcul de la partie explicite de $\tens{A}\,\grad{\phi}$
sont des moyennes arithm\'etiques ?}
\item {Les techniques adopt\'ees lors du clipping sont \`a revoir.}
\item {On utilise dans le cadre du mod\`ele $\displaystyle R_{ij}-\varepsilon $ une semi-implicitation de termes comme $\displaystyle \phi_{ij,1}$ ou $\displaystyle -\rho\,C_{\varepsilon_2}\,\frac{\varepsilon}{k}\,\varepsilon$. On peut envisager le m\^eme type d'implicitation dans \fort{turbke} m\^eme en pr\'esence du couplage $\displaystyle k-\varepsilon$.}
\item L'adoption d'une r\'esolution d\'ecoupl\'ee fait perdre l'invariance par rotation.
\item La formulation et l'implantation des conditions aux limites de paroi
devront \^etre v\'erifi\'ees. En effet, il semblerait que, dans certains cas, des ph\'enom\`enes
``oscillatoires'' apparaissent, sans qu'il soit ais\'e d'en d\'eterminer la cause.
\item L'implicitation partielle (du fait de la r\'esolution d\'ecoupl\'ee) des
conditions aux limites conduit souvent \`a des calculs instables. Il
conviendrait d'en conna\^\i tre la raison. L'implicitation partielle avait
\'et\'e mise en \oe uvre afin de tenter d'utiliser un pas de temps plus grand
dans le cas de jets axisym\'etriques en particulier.

\end{itemize}

%                      Code_Saturne version 1.3
%                      ------------------------
%
%     This file is part of the Code_Saturne Kernel, element of the
%     Code_Saturne CFD tool.
%
%     Copyright (C) 1998-2007 EDF S.A., France
%
%     contact: saturne-support@edf.fr
%
%     The Code_Saturne Kernel is free software; you can redistribute it
%     and/or modify it under the terms of the GNU General Public License
%     as published by the Free Software Foundation; either version 2 of
%     the License, or (at your option) any later version.
%
%     The Code_Saturne Kernel is distributed in the hope that it will be
%     useful, but WITHOUT ANY WARRANTY; without even the implied warranty
%     of MERCHANTABILITY or FITNESS FOR A PARTICULAR PURPOSE.  See the
%     GNU General Public License for more details.
%
%     You should have received a copy of the GNU General Public License
%     along with the Code_Saturne Kernel; if not, write to the
%     Free Software Foundation, Inc.,
%     51 Franklin St, Fifth Floor,
%     Boston, MA  02110-1301  USA
%
%-----------------------------------------------------------------------
%
\programme{vortex}
%
\vspace{1cm}
%%%%%%%%%%%%%%%%%%%%%%%%%%%%%%%%%%
%%%%%%%%%%%%%%%%%%%%%%%%%%%%%%%%%%
\section{Fonction}
%%%%%%%%%%%%%%%%%%%%%%%%%%%%%%%%%%
%%%%%%%%%%%%%%%%%%%%%%%%%%%%%%%%%%
Ce sous-programme est d�di� � la g�n�ration des conditions d'entr�e
turbulente utilis�es en LES.


La m�thode des vortex est bas�e sur une approche de tourbillons
ponctuels. L'id�e de la m�thode consiste � injecter des tourbillons 2D dans le
plan d'entr�e du calcul, puis � calculer le champ de vitesse induit par ces
tourbillons au centre des faces d'entr�e.

%                      Code_Saturne version 1.3
%                      ------------------------
%
%     This file is part of the Code_Saturne Kernel, element of the
%     Code_Saturne CFD tool.
% 
%     Copyright (C) 1998-2007 EDF S.A., France
%
%     contact: saturne-support@edf.fr
% 
%     The Code_Saturne Kernel is free software; you can redistribute it
%     and/or modify it under the terms of the GNU General Public License
%     as published by the Free Software Foundation; either version 2 of
%     the License, or (at your option) any later version.
% 
%     The Code_Saturne Kernel is distributed in the hope that it will be
%     useful, but WITHOUT ANY WARRANTY; without even the implied warranty
%     of MERCHANTABILITY or FITNESS FOR A PARTICULAR PURPOSE.  See the
%     GNU General Public License for more details.
% 
%     You should have received a copy of the GNU General Public License
%     along with the Code_Saturne Kernel; if not, write to the
%     Free Software Foundation, Inc.,
%     51 Franklin St, Fifth Floor,
%     Boston, MA  02110-1301  USA
%
%-----------------------------------------------------------------------
%
%%%%%%%%%%%%%%%%%%%%%%%%%%%%%%%%%%
%%%%%%%%%%%%%%%%%%%%%%%%%%%%%%%%%%
\section{Discr\'etisation}
%%%%%%%%%%%%%%%%%%%%%%%%%%%%%%%%%%
%%%%%%%%%%%%%%%%%%%%%%%%%%%%%%%%%%

Le terme convectif en $\dive(\underline{u} \otimes \rho\,\underline{u})$
introduit une non lin\'earit\'e et un couplage des composantes de la vitesse
$\vect{u}$ dans l'�quation (\ref{Base_Preduv_eqqdm}). Une lin\'earisation et un d\'ecouplage
des trois composantes de la 
vitesse sont r\'ealis\'es lors de la discr\'etisation de cette \'etape de
pr\'ediction.\\
En effet, soit :
\begin{equation}
\vect{\widetilde{u}}= \vect{u}^n + \delta \vect{u} 
\end{equation}
La contribution exacte du terme convectif \`a prendre en compte dans cette
\'etape de pr\'ediction serait :\\
\begin{equation}\label{Base_Preduv_Conv_exact}
\begin{array}{ll}
\dive(\vect{\widetilde{u}} \otimes \rho\,\vect{\widetilde{u}}) =
\dive(\vect{u}^{n} \otimes \rho\,\vect{u}^{n}) + \dive(\delta \vect{u} \otimes
\rho\,\vect{u}^{n}) +  \underbrace { \dive(\vect{u}^{n} \otimes
\rho\,\delta \vect{u})}_{\text {terme couplant lin\'eaire}} +  \underbrace { \dive(\delta \vect{u} \otimes
\rho\,\delta \vect{u})}_{\text {terme couplant et non lin\'eaire}}\\
\end{array} 
\end{equation}
Les deux derniers termes de l'expression (\ref{Base_Preduv_Conv_exact}) sont {\em a priori} n�glig�s
de mani�re � obtenir un syst\`eme en vitesse qui soit d\'ecoupl\'e et donc,
�viter l'inversion d'une matrice pouvant \^etre de tr\`es grande taille. Ces
deux termes peuvent n�anmoins �tre pris en compte de mani�re plus ou moins
approch�e par extrapolation explicite du flux de masse en $n+\theta_F$ (pour le
terme couplant lin�aire provenant de la convection de $\vect{u}^{n}$ par $\delta
\vect{u}$) et utilisation d'un point-fixe par sous it�ration sur le sous
programme \fort{navsto} (pour le terme non-lin�aire, en sp�cifiant $\var{NTERUP}>1$).

L'�quation (\ref{Base_Preduv_eqqdm}) est discr�tis�e au temps $n+\theta$ � l'aide d'un
$\theta$-sch�ma, et le tenseur des pertes de charges d�compos� en une partie
diagonale $\tens{K}_{d}$ et une extradiagonale $\tens{K}_{e}$ (soit
 $\tens{K}_{pdc}=\tens{K}_{d}+\tens{K}_{e}$).\\
$\bullet$ La pression est suppos�e connue en $n-1+\theta$ (d�calage temporel
pression-vitesse) et le gradient naturellement calcul� � cet instant.\\ 
$\bullet$ Les termes sources de viscosit� secondaire, de gradient transpos\'e,
ceux provenant du mod�le de turbulence\footnote{except� $\dive (\mu_t\ (\ggrad
\underline {u}))$}, $\rho\,\tens{K}_{\,e}\ \underline{u}$, $(\rho -\rho_0)
\underline {g}$ ainsi que $\underline{T}_{s}^{\,exp}$ et
$\Gamma\,\underline{u}_{\,i}$ sont pris de mani�re explicite au temps $n$, ou
extrapol�s suivant le sch�ma en temps choisi pour les propri�t�s physique et les
termes sources.\\ 
$\bullet$ Les termes sources $\underline{u}\,\,\dive (\rho\,\underline {u})$,
$\Gamma\,\,\underline{u}$, $T_{s}^{\,imp}\,\,\underline{u}$ et
$-\rho\,\tens{K}_{\,d}\,\,\underline{u}$ sont implicit�s est calcul�s �
l'instant $n+\theta$.\\ 
$\bullet$ Le terme de diffusion $\dive (\mu_{\,tot}\,\ggrad \underline{u})$ est
implicit� : la vitesse est prise � l'instant $n+\theta$ et la viscosit�
explicit�e ou extrapol�e.\\ 
$\bullet$ Enfin, le terme de convection en $\dive(\,\underline{u} \otimes
(\rho\underline{u})\,)$ est implicit� : la composante r�solue de la vitesse est
prise en $n+\theta$, et le flux de masse, explicit�, ou extrapol� en
$n+\theta_F$. 

Par souci de clart�, on suppose, en l'absence d'indication, que les propri�tes
physiques $\Phi$ ($\rho,\,\mu_{tot},\,...$) et le flux de masse
$(\rho\underline{u})$ sont pris respectivement aux instants $n+\theta_\Phi$ et
$n+\theta_F$, o� $\theta_\Phi$ et $\theta_F$ d�pendent des sch�mas en temps
sp�cifiquement utilis�s pour ces grandeurs\footnote{cf. \fort{introd}}. 

La discr�tisation temporelle de l'�quation (\ref{Base_Preduv_eqqdm}) s'�crit alors comme suit : 

\begin{equation}\label{Base_Preduv_eq_di1}
 \begin{array}{c}
\displaystyle \rho\,\ \frac{ \underline {\widetilde{u}}^{n+1} -\underline {u}^{n} }
{\Delta t} + \dive(\,\underline{\widetilde{u}}^{n+\theta} \otimes (\rho\underline{u})\,) -\dive
(\mu_{\,tot}\,\ggrad \underline{\widetilde{u}}^{n+\theta}) =
\\
\displaystyle
 - \grad p^{n-1+\theta} + \dive (\rho\,\underline {u})\,\underline{\widetilde{u}}^{n+\theta} +(\Gamma\,\underline{u}_{\,i})^{n+\theta_S}-\Gamma^n\,\,\underline{\widetilde{u}}^{n+\theta}
\\
\begin{array}{c}
\displaystyle
- \rho\,\tens{K}_{\,d}^{n}\,\,\underline{\widetilde{u}}^{n+\theta} - (\rho\,\tens{K}_{\,e}\ \underline{u})^{n+\theta_S} + (\underline{T}_{s}^{\,exp})^{\,n+\theta_S} + T_{s}^{\,imp}\,\,\underline{\widetilde{u}}^{n+\theta}
\\
\displaystyle
+[\dive (\mu_{\,tot}\,^t\ggrad \underline {u})]^{n+\theta_S}-\frac {2} {3}[\,\grad (\mu_{\,tot}\,\dive \underline {u})]^{n+\theta_S} + (\rho -\rho_0) \underline {g}
 - (\underline{turb})^{n+\theta_S}
\end{array}
\end{array}
\end{equation}
o\`u, par souci de simplification, on a pos\'e :
\begin{equation}
\mu_{\,tot}=
\begin{cases}
\mu+\mu_t & \text{pour les mod�les � viscosit� turbulente ou en LES}, \\
\mu & \text{pour les mod�les au second ordre ou en laminaire}
\end{cases} \ 
\end{equation}
\\
et :
\begin{equation}
\underline{turb}^{n}=
\begin{cases}
\displaystyle\frac {2}{3}\grad (\rho^{n}\,k^{n}) & \text{pour les mod�les � viscosit� turbulente}, \\
\dive(\rho^{n}\,\tens{R}^n) & \text{pour les mod�les au second ordre},\\
0 & \text{en laminaire ou en LES}\\
\end{cases}
\end{equation}
Par analogie avec l'�criture du $\theta$-sch�ma pour une variable scalaire, $\,
\underline {\widetilde{u}}^{n+\theta}$ est interpol�e � partir de la vitesse
pr�dite $\underline {\widetilde{u}}^{n+1}$ de la mani\`ere suivante\footnote{si
$\theta=1/2$, ou qu'une extrapolation est utilis�e, l'ordre 2 n'est obtenu que si
le pas de temps $\Delta t$ est uniforme en temps et en espace.}~: 
\begin{equation}
\underline {\widetilde{u}}^{n+\theta}=\theta\, \underline
{\widetilde{u}}^{n+1}+(1-\theta)\, \underline {u}^{n}\\ 
\end{equation}
Avec :
\begin{equation}
\left\{
\begin{array}{ll}
\theta = 1   & \text{Pour un sch\'ema de type Euler implicite d'ordre 1.}\\
\theta = 1/2 & \text{Pour un sch\'ema de type Cranck-Nicolson d'ordre 2.}\\
\end{array}
\right.
\end{equation}

L'�quation (\ref{Base_Preduv_eq_di1}) est alors r��crite sous la forme :

\begin{equation}\label{Base_Preduv_eq_di2}
\begin{array}{c}
\displaystyle \underbrace{\left(\frac{\rho}{\Delta t} -\theta \,\dive (\rho\,\underline {u}) +\theta \,\, \Gamma^n +
\theta \,\, \rho\,\tens{K}_{\,d}^n-\theta \,T_s^{\,imp} \right)}_{\displaystyle f_s^{imp}}\, (\underline {\,\widetilde{u}}^{n+1} -\underline {u}^{n})
\\
 +\, \theta\, \dive(\underline {\widetilde{u}}^{n+1} \otimes (\rho\underline{u}))-\, \theta\,\dive (\mu_{\,tot}\,\ggrad \underline {\widetilde{u}}^{n+1}) =
\\
-\,(1-\theta)\, \dive(\underline {u}^{n} \otimes (\rho\underline{u})) +\,(1-\theta)\,\dive (\mu_{\,tot}\,\ggrad \underline {u}^{n})
\\
f_s^{exp}\left\{
\begin{array}{c}
\displaystyle 
- \grad p^{n-1+\theta} + \dive (\rho\,\underline {u})\,\underline{u}^{n} +\,(\,\Gamma^{n}\,\underline{u}_{\,i}\,)^{n+\theta_S}- \Gamma^n\,\,\underline{u}^{n}
\\
\displaystyle
-(\,\rho\,\tens{K}_{\,e}\ \underline{u}\,)^{n+\theta_S} -\rho\,\tens{K}_{\,d}^n\ \underline{u}^{n}+ (\underline{T}_{s}^{\,exp})^{\,n+\theta_S} + T_s^{\,imp}\,\,\underline {u}^{n} 
\\
\displaystyle
+[\dive (\mu_{\,tot}\,^t\ggrad \underline {u}\,)]^{n+\theta_S}-\frac {2} {3}[\,\grad (\mu_{\,tot}\,\dive \underline {u}\,)]^{n+\theta_S} + (\rho -\rho_0) \underline {g}-(\underline{turb})^{n+\theta_S}
\end{array}
\right.
\end{array}
\end{equation}

d'o� l'�quation r�solue par le sous-programme \fort{codits} :
\begin{equation}\begin{array}{c}
\displaystyle
f_s^{\,imp}(\underline {\widetilde{u}}^{n+1}-\underline {u}^{n}) + \theta\, \dive(\underline{\widetilde{u}}^{n+1} \otimes (\rho
\underline{u})) - \theta\,\dive (\,\mu_{\,tot}\,\ggrad \underline{\widetilde{u}}^{n+1}) = 
\\\\
\displaystyle
-(1-\theta)\,\dive(\underline{u}^{n} \otimes (\rho \underline{u}))+(1-\theta)\,\dive (\,\mu_{\,tot}\,\ggrad \underline{u}^{n})
+ \underline{f}_{\,s}^{\,exp}
\end{array}
\end{equation}
La m\'ethode de discr\'etisation spatiale est d\'evelopp\'ee dans le sous-programme \fort{codits}.\\



\minititre{Remarques :}
{\tiny$\blacksquare$} Dans le cas standard sans extrapolation, le terme
$-\,T_s^{\,imp}$ n'est ajout� � $f_s^{\,imp}$ que s'il est positif afin de ne
pas affaiblir la dominance de la diagonale de la matrice � inverser.\\ 
{\tiny$\blacksquare$} Si une extrapolation est utilis�e, par contre,
$\,T_s^{\,imp}$ est ajout� � $f_s^{\,imp}$ quel que soit son signe. En effet, l'id�e intuitive qui
consiste � prendre~: 
\begin{equation}
\begin{cases}
\displaystyle
(\underline{T}_{s}^{\,exp} + T_{s}^{\,imp}\,\underline {u})^{\,n+\theta_S} &
\text{si } T_{s}^{\,imp} > 0\\ 
\displaystyle
(\underline{T}_{s}^{\,exp})^{\,n+\theta_S} + T_{s}^{\,imp}\,\underline{u}^{n+\theta} &\text{sinon}\\
\end{cases}
\end{equation} 
aboutit � une incoh�rence dans le traitement si $T_s^{imp}$ change de signe
entre deux pas de temps.\\ 
{\tiny$\blacksquare$} la partie diagonale $\tens{K}_{\,d}$ du terme
de perte de charge est utilis�e dans $f_s^{\,imp}$. En fait, pour \^etre rigoureux,
il faudrait ne retenir que les contributions positives (point signal\'e dans le
sous-programme utilisateur associ\'e \fort{uskpdc}). Cette prise en compte sera \`a am\'eliorer.\\
{\tiny$\blacksquare$} Le terme $\theta\,\Gamma^{n}-\theta\,\dive
(\rho\,\underline {u})$ ne pose pas de probl�me pour la 
dominance de la diagonale de la matrice car il est exactement compens� par le
terme de convection (cf. \fort{covofi}). 


%                      Code_Saturne version 1.3
%                      ------------------------
%
%     This file is part of the Code_Saturne Kernel, element of the
%     Code_Saturne CFD tool.
%
%     Copyright (C) 1998-2007 EDF S.A., France
%
%     contact: saturne-support@edf.fr
%
%     The Code_Saturne Kernel is free software; you can redistribute it
%     and/or modify it under the terms of the GNU General Public License
%     as published by the Free Software Foundation; either version 2 of
%     the License, or (at your option) any later version.
%
%     The Code_Saturne Kernel is distributed in the hope that it will be
%     useful, but WITHOUT ANY WARRANTY; without even the implied warranty
%     of MERCHANTABILITY or FITNESS FOR A PARTICULAR PURPOSE.  See the
%     GNU General Public License for more details.
%
%     You should have received a copy of the GNU General Public License
%     along with the Code_Saturne Kernel; if not, write to the
%     Free Software Foundation, Inc.,
%     51 Franklin St, Fifth Floor,
%     Boston, MA  02110-1301  USA
%
%-----------------------------------------------------------------------
%

%%%%%%%%%%%%%%%%%%%%%%%%%%%%%%%%%%
%%%%%%%%%%%%%%%%%%%%%%%%%%%%%%%%%%
\section{Mise en \oe uvre}
%%%%%%%%%%%%%%%%%%%%%%%%%%%%%%%%%%
%%%%%%%%%%%%%%%%%%%%%%%%%%%%%%%%%%
La num\'ero de la phase trait\'ee fait partie des arguments de \fort{turrij}. On
omettra volontairement de le pr\'eciser dans ce qui suit, on indiquera par $(\ )$ la
notion de tableau s'y rattachant.

\etape{Calcul des termes de production $\tens{\mathcal{P}}$}
\begin{itemize}
\item [$\star$] Initialisation \`a z\'ero du tableau \var{PRODUC} dimensionn\'e \`a $\var{NCEL}\times 6$.
\item [$\star$] On appelle trois fois \fort{grdcel} pour calculer les gradients des composantes de la vitesse $u$, $v$ et
$w$ prises au temps $n$.

Au final, on a :\\
$\displaystyle
\begin{array} {ll}
\var{PRODUC(1,IEL)} = & \displaystyle - 2 \left[ R_{11}^{\,n} \frac{\partial u^{\,n}} {\partial x} +R_{12}^{\,n} \frac{\partial u^{\,n}} {\partial y}+R_{13}^{\,n} \frac{\partial u^{\,n}} {\partial z} \right] \text{        (production de $R_{11}^{\,n}$)}\\
\var{PRODUC(2,IEL)} = & \displaystyle - 2 \left[ R_{12}^{\,n} \frac{\partial v^{\,n}} {\partial x} +R_{22}^{\,n} \frac{\partial v^{\,n}} {\partial y}+R_{23}^{\,n} \frac{\partial v^{\,n}} {\partial z} \right] \text{        (production de $R_{22}^{\,n}$)}\\
\var{PRODUC(3,IEL)} = & \displaystyle - 2 \left[ R_{13}^{\,n} \frac{\partial w^{\,n}} {\partial x} +R_{23}^{\,n} \frac{\partial w^{\,n}} {\partial y}+R_{33}^{\,n} \frac{\partial w^{\,n}} {\partial z} \right] \text{        (production de $R_{33}^{\,n}$)}\\
\var{PRODUC(4,IEL)} = & \displaystyle - \left[ R_{12}^{\,n} \frac{\partial u^{\,n}} {\partial x} +R_{22}^{\,n} \frac{\partial u^{\,n}} {\partial y}+R_{23}^{\,n} \frac{\partial u^{\,n}} {\partial z} \right] \\
& \displaystyle - \left[ R_{11}^{\,n} \frac{\partial v^{\,n}} {\partial x} +R_{12}^{\,n} \frac{\partial v^{\,n}} {\partial y}+R_{13}^{\,n} \frac{\partial v^{\,n}} {\partial z} \right] \text{        (production de $R_{12}^{\,n}$)} \\
\var{PRODUC(5,IEL)} = & \displaystyle - \left[ R_{13}^{\,n} \frac{\partial u^{\,n}} {\partial x} +R_{23}^{\,n} \frac{\partial u^{\,n}} {\partial y}+R_{33}^{\,n} \frac{\partial u^{\,n}} {\partial z} \right] \\
& \displaystyle - \left[ R_{11}^{\,n} \frac{\partial w^{\,n}} {\partial x} +R_{12}^{\,n} \frac{\partial w^{\,n}} {\partial y}+R_{13}^{\,n} \frac{\partial w^{\,n}} {\partial z} \right] \text{        (production de $R_{13}^{\,n}$)} \\
\var{PRODUC(6,IEL)} = & \displaystyle - \left[ R_{13}^{\,n} \frac{\partial v^{\,n}} {\partial x} +R_{23}^{\,n} \frac{\partial v^{\,n}} {\partial y}+R_{33}^{\,n} \frac{\partial v^{\,n}} {\partial z} \right] \\
& \displaystyle - \left[ R_{12}^{\,n} \frac{\partial w^{\,n}} {\partial x} +R_{22}^{\,n} \frac{\partial w^{\,n}} {\partial y}+R_{23}^{\,n} \frac{\partial w^{\,n}} {\partial z} \right]  \text{        (production de $R_{23}^{\,n}$)}
\end{array}
$
\end{itemize}

\etape{Calcul du gradient de la masse volumique $\rho^n$ prise au d\'ebut du pas
de temps courant\footnote{{\it i.e.} calcul\'ee \`a partir des
variables du pas de temps pr\'ec\'edent $n$ si n\'ecessaire.} $(n+1)$}
Ce calcul n'a lieu que si les termes de gravit\'e doivent \^etre pris en compte
($\var{IGRARI()} =1$).
\begin{itemize}
\item [$\star$] Appel de \fort{grdcel}  pour calculer le gradient de $\rho^n$
dans les trois directions de l'espace. Les conditions aux limites sur $\rho^n$
sont des conditions de Dirichlet puisque la valeur de $\rho^n$ aux faces de bord
$ik$ (variable \var{IFAC}) est connue et vaut $\rho_{\,b_{\,ik}}$. Pour \'ecrire les conditions aux limites
sous la forme habituelle, $$\rho_{\,b_{\,ik}} = \var{COEFA} + \var{COEFB}
\,\rho^n_{\,I'}$$ on pose alors $\var{COEFA}=
\var{PROPCE(IFAC,IPPROB(IROM(IPHAS)))}$ et $\var{COEFB} = \var{VISCB} = 0$.\\
$\var{PROPCE(1,IPPROB(IROM(IPHAS)))}$ (resp.$\var{VISCB}$) est utilis\'e en lieu
et place de l'habituel \var{COEFA} ($\var{COEFB}$), lors de l'appel \`a \fort{grdcel}.\\
On a donc :\\
$\displaystyle \var{GRAROX}= \frac{\partial \rho^n}{\partial x}\ $,$\displaystyle \ \var{GRAROY}= \frac{\partial
\rho^n}{\partial y}$ et $
\displaystyle \ \var{GRAROZ}= \frac{\partial \rho^n}{\partial z}\ $.

\end{itemize}

Le gradient de $\rho^n$ servira \`a calculer les termes de production par effets de gravit\'e si ces derniers sont pris en compte.

\etape{Boucle \var{ISOU} de $1$ \`a $6$ sur les tensions de Reynolds}
Pour $\var{ISOU} = 1,2,3,4,5,6$, on r\'esout respectivement et dans
l'ordre  les
\'equations de $R_{11}$, $R_{22}$, $R_{33}$, $R_{12}$, $R_{13}$ et $R_{23}$ par
l'appel au sous-programme \fort{resrij}.\\
La r\'esolution se fait par incr\'ement $\delta {R}_{ij}^{\,n+1,k+1}$ , en utilisant la m\^eme m\'ethode que
celle d\'ecrite dans le sous-programme \fort{codits}. On adopte ici les m\^emes notations.
\var{SMBR} est le second membre du syst\`eme \`a inverser, syst\`eme portant sur
les incr\'ements de la variable. \var{ROVSDT} repr\'esente la diagonale de la
matrice, hors convection/diffusion.\\
On va r\'esoudre l'\'equation (\ref{Base_Turrij_Eq_Temp_Rij}) sous forme incr\'ementale en
utilisant \fort{codits}, soit :
\begin{equation}\label{Base_Turrij_Eq_Temp_deltaRij}
\begin{array}{ll}
&\displaystyle \underbrace{\left(\frac {\rho^n_L}{\Delta t^n}
+ \rho^n_L \,C_1\,\frac{\varepsilon^n_L}{k^n_L}(1-\frac{\delta_{ij}}{3})
 - m^n_{\,lm} + \Gamma_L\,+ max(-\alpha^n_{R_{ij}},0)\right)\,|\Omega_l|}
_{\text {$\var{ROVSDT}$ contribuant
\`a la diagonale de la matrice simplifi\'ee de \fort{matrix}}}\,(\delta{R}_{ij}^{\,n+1,p+1})_{\,L}\\\\
&  \underbrace{+\sum\limits_{m\in Vois(l)}\displaystyle \left[
 m^n_{\,lm} \delta R_{ij,\,f_{\,lm}}^{\,n+1,p+1}
- (\mu^n_{\,lm} + \gamma^n_{\,lm})\
\frac{({\delta R}_{ij}^{\,n+1,p+1})_{M}-({\delta R}_{ij}^{\,n+1,p+1})_{L})}{\overline{L'M'}}\,
S_{\,lm} \right]}_{\text { convection upwind pur et diffusion non reconstruite
relatives \`a la matrice simplifi\'ee de \fort{matrix}\footnotemark}} \\
% voir le texte de la footmark plus bas
&= - \displaystyle\frac {\rho^n_L}{\Delta t^n}\,\left(\,(R^{\,n+1,p}_{ij})_L - (R^{\,n}_{ij})_L\,\right)\\
&-\,\underbrace{\displaystyle\int_{\Omega_l} \left(
\dive\,[\,(\rho\,\vect{u})^n\,R^{\,n+1,p}_{ij} - (\mu^n\,+ \gamma^n\,)\,
\grad{R^{\,n+1,p}_{ij}}\,]\right)\,d\Omega}_{\text {convection et diffusion
trait\'ees par \fort{bilsc2}}}\\
&+\displaystyle \int_{\Omega_l} \left[\,\mathcal{P}^{\,n+1,p}_{ij} + \mathcal{G}^{\,n+1,p}_{ij}
- \displaystyle\rho^n \,C_1\,\frac{\varepsilon^n}{k^n}\left[R^{\,n+1,p}_{ij}-
\frac{2}{3}\,k^n\,\delta_{ij}\right] + \phi^{\,n+1,p}_{ij,2} +
\phi^{\,n+1,p}_{ij,w}\,\right]\, d\Omega \\
& + \displaystyle\int_{\Omega_l} \left[- \frac{2}{3} \rho^n \varepsilon^n \delta_{ij}
 + \Gamma\,(\,R^{\,in}_{ij} - R^{\,n+1,p}_{ij}\,) +
\alpha^n_{R_{ij}}\,R^{\,n+1,p}_{ij}+ \beta^n_{R_{ij}}\right]\, d\Omega\\
&+ \sum\limits_{m\in
Vois(l)}\displaystyle \left[\ \tens{E}^n\,\grad{R}^{\,n+1,p}_{ij} \right]_{\,lm}\,.\,\vect{n}_{\,lm}S_{\,lm}\\
&+ \sum\limits_{m\in Vois(l)}\displaystyle \left[\
\tens{D}^n\,\grad{R}^{\,n+1,p}_{ij} \right]_{\,lm}\,.\,\vect{n}_{\,lm}S_{\,lm}\\
&- \sum\limits_{m\in Vois(l)} \gamma^n_{\,lm} \left( \grad{R}^{\,n+1,p}_{ij}\,.\,\vect{n}_{\,lm} \right)  S_{\,lm}\\
&+ \sum\limits_{m\in Vois(l)}  m^n_{\,lm}\,(R^{\,n+1,p}_{ij})_L\\
\end{array}
\end{equation}
% si on ne fait pas comme ca, il n'apparait pas
\footnotetext[\thefootnote]{Si $\var{IRIJNU} = 1$, on remplace  $\mu^n_{\,lm}$ par $(\mu +
\mu_t)^n_{\,lm}$ dans l'expression de la diffusion non reconstruite
associ\'ee \`a la matrice simplifi\'ee de \fort{matrix} ($\mu_t$ d\'esigne la
viscosit\'e turbulente calcul\'ee comme en $k-\varepsilon$).}

o\`u on rappelle :\\
pour $n$ donn\'e entier positif, on d\'efinit la suite
 $({R}_{ij}^{\,n+1,p})_{p \in \grandN}$
 par :
\begin{equation}\notag
\left\{\begin{array}{l}
{R}_{ij}^{\,n+1,0} = {R}_{ij}^{\,n}\\
{R}_{ij}^{\,n+1,p+1} = {R}_{ij}^{\,n+1,p} + \delta{R}_{ij}^{\,n+1,p+1} \\
\end{array}\right.
\end{equation}
$(\delta{R}_{ij}^{\,n+1,p+1})_{\,L}$ d\'esigne la valeur sur l'\'el\'ement
$\Omega_l$ du $\text{$(\,p+1\,)$-i\`eme}$ incr\'ement de ${R}_{ij}^{\,n+1}$,
$ m^n_{\,lm}$ le flux de masse \`a l'instant $n$ \`a travers la face $lm$,
$\delta R_{ij,\,f_{\,lm}}^{\,n+1,p+1}$ vaut $({\delta
R}_{ij}^{\,n+1,p+1})_{L}$  si $ m^n_{\,lm} \geqslant 0$, $({\delta
R}_{ij}^{\,n+1,p+1})_{M}$ sinon,
$\mathcal{P}^{\,n+1,p}_{ij}$, $\phi^{\,n+1,p}_{ij,2}$, $\phi^{\,n+1,p}_{ij,w}$ les valeurs
des quantit\'es associ\'ees correspondant \`a l'incr\'ement
$(\delta{R}_{ij}^{\,n+1,p})$.\\



Tous ces termes sont calcul\'es comme suit :
\begin{itemize}
\item Terme de gauche de l'\'equation (\ref{Base_Turrij_Eq_Temp_deltaRij})\\
Dans \fort{resrij} est calcul\'ee la variable \var{ROVSDT}. Les autres
termes sont compl\'et\'es par \fort{codits}, lors de la construction de la matrice simplifi\'ee , {\it via} un
appel au sous-programme \fort{matrix}. La quantit\'e
 $(\mu^n_{\,lm} + \gamma^n_{\,lm})$ \`a la face $lm$ est calcul\'ee lors de l'appel \`a
\fort{visort}.\\
\item Second membre de l'\'equation (\ref{Base_Turrij_Eq_Temp_deltaRij})\\
Le premier terme non d\'etaill\'e est calcul\'e par le sous-programme
\fort{bilsc2}, qui applique le sch\'ema convectif choisi par l'utilisateur, qui
reconstruit ou non selon le souhait de l'utilisateur les gradients intervenants
dans la convection-diffusion.\\
Les termes sans accolade sont, eux, compl\`etement explicites et ajout\'es au fur et
\`a mesure dans \var{SMBR} pour former
l'expression $f^{\,exp}_s$ de \fort{codits}.
\end{itemize}
On d\'ecrit ci-dessous les \'etapes de \fort{resrij} :
\begin{itemize}

\item DELTIJ = 1, pour $\var{ISOU} \leqslant 3$ et DELTIJ = 0  Si $\var{ISOU} >
3$. Cette valeur repr\'esente le symbole de Kroeneker $\delta_{ij}$.

\item Initialisation \`a z\'ero de \var{SMBR} (tableau contenant le second
membre) et \var{ROVSDT} (tableau contenant la diagonale de la matrice sauf celle
relative \`a la contribution de la
diagonale des op\'erateurs de convection et de diffusion lin\'earis\'es
\footnote{qui correspondent aux sch\'emas convectif upwind pur et diffusif sans
reconstruction.}), tous deux de dimension $\var{NCEL}$.

\item Lecture et prise en compte des termes sources utilisateur pour la variable $R_{ij}$

Appel \`a \fort{ustsri} pour charger les termes sources utilisateurs. Ils sont
stock\'es comme suit. Pour la cellule $\Omega_l$ de centre $L$, repr\'esent\'ee par $\var{IEL}$, on a :\\
\begin{equation}\notag
\left\{\begin{array}{lll}
&\var{ROVSDT(IEL)}&= |\Omega_l| \ \alpha_{R_{ij}}\\
&\var{SMBR(IEL)}&=|\Omega_l| \ \beta_{R_{ij}}\\
\end{array}\right.
\end{equation}
On affecte alors les valeurs ad\'equates au second membre \var{SMBR} et \`a la
diagonale \var{ROVSDT} comme suit :
\begin{equation}\notag
\left\{\begin{array}{lll}
&\var{SMBR(IEL)} &= \var{SMBR(IEL)} +\ |\Omega_l| \ \alpha_{R_{ij}} \ (R^n_{ij})_L \\
&\var{ROVSDT(IEL)}&= \text{max }(-\ |\Omega_l| \ \alpha_{R_{ij}},0)\\
\end{array}\right.
\end{equation}
La valeur de $ \var{ROVSDT}$ est ainsi calcul\'ee pour des raisons de stabilit\'e
num\'erique. En effet, on ne rajoute sur la diagonale que les valeurs positives,
ce qui correspond physiquement \`a impliciter les termes de rappel uniquement.
\item{Calcul du terme source de masse  si $\Gamma_L > 0$}

Appel de \fort{catsma} et incr\'ementation si n\'ecessaire de \var{SMBR} et
\var{ROVSDT} {\it via} :\\
\begin{equation}\notag
\left\{\begin{array}{lll}
\displaystyle \var{SMBR(IEL)} = \var{SMBR(IEL)} + |\Omega_l| \ \Gamma_L \
\left[(R^{\,in}_{ij})_L - (R^{\,n}_{ij})_L \right] \\
\displaystyle \var{ROVSDT(IEL)}=\var{ROVSDT(IEL)} + |\Omega_l| \ \Gamma_L
\end{array}\right.
\end{equation}
\item Calcul du terme d'accumulation de masse et du terme instationnaire

On stocke $\displaystyle \var{W1}= \int_{\Omega_l}\dive\,(\rho\,\vect{u})\,d\Omega$
calcul\'e par \fort{divmas} \`a l'aide des flux de masse aux faces internes
$ m^n_{\,lm}=\sum\limits_{m\in Vois(l)}{(\rho \vect{u})_{\,lm}^n} \text{.}\,
\vect{S}_{\,lm} $ (tableau \var{FLUMAS}) et des flux de masse aux bords  $ m^n_{\,b_{lk}} = \sum\limits_{k\in{\gamma_b(l)}}{(\rho \vect{u})_{\,{b}_{lk}}^n} \text{.}\,
\vect{S}_{\,{b}_{lk}} $ (tableau \var{FLUMAB}).
On incr\'emente ensuite \var{SMBR} et \var{ROVSDT}.
\begin{equation}\notag
\left\{\begin{array}{lll}
&\var{SMBR(IEL)} &= \var{SMBR(IEL)} + \var{ICONV}\  (R^n_{ij})_L\,(\displaystyle
\int_{\Omega_l}\dive\,(\rho\,\vect{u})\ d\Omega) \\
&\var{ROVSDT(IEL)}& = \var{ROVSDT(IEL)} +  \var{ISTAT}\,\displaystyle
\frac{\rho^n_L \ |\Omega_l|}{\Delta t^n} -  \var{ICONV}\ (\displaystyle
\int_{\Omega_l}\dive\,(\rho\,\vect{u})\ d\Omega) \\
\end{array}\right.
\end{equation}
\item Calcul des termes sources de production, des termes $\displaystyle
\phi_{\,ij,1}+\phi_{\,ij,2}$ et de la dissipation~$\displaystyle-\frac{2}{3} \varepsilon\,\delta_{\,ij}$ :

On effectue une boucle d'indice \var{IEL} sur les cellules $\Omega_l$ de centre $L$ :
\begin{itemize}
\item [$\Rightarrow$] $\displaystyle \var{TRPROD}= \frac{1}{2} (\mathcal{P}^n_{ii})_L = \frac{1}{2} \left[ \var{PRODUC(1,IEL)} +  \var{PRODUC(2,IEL)} +  \var{PRODUC(3,IEL)} \right] $
\item [$\Rightarrow$] $\displaystyle \var{TRRIJ }= \frac{1}{2} (R^n_{ii})_L $
\item [$\Rightarrow$] $\displaystyle \var{SMBR(IEL)} =\ \var{SMBR(IEL)}\ +$\\
$\ \displaystyle\rho^n_L |\Omega_l| \left[ \displaystyle
\frac{2}{3}\,\delta_{\,ij} \left( \ \displaystyle \frac{ C_2}{2}\,(\mathcal{P}^n_{ii})_L\ +
(C_1-1)\ \varepsilon^n_L\, \right)\right.$\\
$ + \left.\ (1-C_2) \ \var{PRODUC(ISOU,IEL)} -
\displaystyle C_1\ \frac{2\,\varepsilon^n_L}{(R^n_{ii})_L}\ (R^n_{ij})_L \right]$
\item [$\Rightarrow$] $\displaystyle \var{ROVSDT(IEL)} = \var{ROVSDT(IEL)} +
\rho^n_L \ |\Omega_l| \ (- \displaystyle \frac{1}{3} \ \,\delta_{\,ij} + 1) \ C_1
\ \frac{2\ \varepsilon^n_L}{(R^n_{ii})_L}$
\end{itemize}
\item Appel de \fort{rijech} pour le calcul des termes d'\'echo de paroi
 $\phi^n_{ij,w}$ si $\var{IRIJEC()}=1$ et ajout dans \var{SMBR}.\\
$\var{SMBR} = \var{SMBR} + \phi^n_{ij,w}$\\
Suivant son mode de calcul (\var{ICDPAR}), la distance � la paroi est directement accessible
par \var{RA(IDIPAR+IEL-1)} (\var{|ICDPAR|} = 1) ou bien
est calcul\'ee \`a partir de $\var{IA(IIFAPA(IPHAS)+IEL - 1)}$,
qui donne pour l'\'el\'ement $\var{IEL}$ le num\'ero de la face de bord
paroi la plus  proche (\var{|ICDPAR|} = 2). Ces tableaux ont \'et\'e renseign\'e une fois pour toutes au
d\'ebut de calcul.

\item  Appel de \fort{rijthe} pour le calcul des termes de gravit\'e $\mathcal{G}^n_{ij}$ et ajout dans \var{SMBR}.

Ce calcul n'a lieu que si $\var{IGRARI()} = 1$.
$ \var{SMBR} = \var{SMBR} + \mathcal{G}^n_{ij}$
\item Calcul de la partie explicite du terme de diffusion
 $\dive{\,\left[\tens{A}\,\grad{R}^{\,n}_{ij}\right]}$, plus pr\'ecis\'ement
des contributions du terme extradiagonal pris aux faces purement internes
(remplissage du tableau \var{VISCF}), puis aux faces de bord (remplissage du
tableau \var{VISCB}).
\begin{itemize}
\item [$\star$] Appel de \fort{grdcel} pour le calcul du gradient de
$R^{\,n}_{ij}$ dans chaque direction. Ces gradients sont respectivement
stock\'es dans les tableaux de travail \var{W1}, \var{W2} et \var{W3}.

\item [$\star$] boucle d'indice \var{IEL} sur les cellules $\Omega_l$ de centre
$L$ pour le
calcul de $\tens{E}^n\,\grad{R}^{\,n}_{ij}$ aux cellules dans un premier temps :\\
\begin{itemize}
\item [$\Rightarrow$] $\displaystyle \var{TRRIJ}= \frac{1}{2} (R^{\,n}_{ii})_L $
\item [$\Rightarrow$] $\displaystyle \var{CSTRIJ} = \rho^n_L\ C_S \ \displaystyle\frac{(R^n_{ii})_L}{2\,\varepsilon^n_L}$
\item [$\Rightarrow$] $\displaystyle \var{W4(IEL)} = \rho^n_L\ C_S\
\displaystyle\frac{(R^n_{ii})_L}{2\,\varepsilon^n_L} \left[\,(R^{\,n}_{12})_L \ \var{W2(IEL)} +
(R^{\,n}_{13})_L \ \var{W3(IEL)}\,\right]$
\item [$\Rightarrow$] $\displaystyle \var{W5(IEL)} = \rho^n_L\ C_S\
\displaystyle\frac{(R^n_{ii})_L}{2\,\varepsilon^n_L} \left[\,(R^{\,n}_{12})_L \ \var{W1(IEL)} +
(R^{\,n}_{23})_L \ \var{W3(IEL)}\,\right]$
\item [$\Rightarrow$] $\displaystyle \var{W6(IEL)} = \rho^n_L\ C_S\
\displaystyle\frac{(R^n_{ii})_L}{2\,\varepsilon^n_L} \left[\,(R^{\,n}_{13})_L \ \var{W1(IEL)} + (R^{\,n}_{23})_L \ \var{W2(IEL)}\,\right]$
\end{itemize}



\item [$\star$] Appel de \fort{vectds}\footnote{Le r\'esultat est stock\'e dans
\var{VISCF} et \var{VISCB}. Dans \fort{vectds}, les valeurs aux faces internes
sont interpol\'ees lin\'eairement sans reconstruction et \var{VISCB} est mis \`a
z\'ero.} pour assembler $\displaystyle\left[ \tens{E}^n\,\grad{R}^{\,n}_{ij}
\right]\,.\,\vect{n}_{\,lm}S_{\,lm}$ aux faces $lm$.
\item [$\star$] Appel de \fort{divmas} pour calculer la divergence du flux d\'efini par \var{VISCF} et \var{VISCB}.
Le r\'esultat est stock\'e dans \var{W4}.\\
Ajout au second membre \var{SMBR}.\\
\var{SMBR} = \var{SMBR} + \var{W4}
\end{itemize}

A l'issue de cette \'etape, seule la partie extradiagonale de la diffusion prise
enti\`erement explicite~:
 $$\sum\limits_{m\in
Vois(l)}\left[\ \tens{E}^n\,\grad{R}^{\,n}_{ij} \right]_{\,lm}\,.\,\vect{n}_{\,lm}S_{\,lm}$$ a \'et\'e calcul\'ee.\\

\item Calcul de la partie diagonale du terme de diffusion\footnote{Seule la
composante normale  du  gradient de $R_{ij}$ aux faces sera implicite.} :\\
Comme on l'a d\'eja vu, une partie de cette contribution sera trait\'ee en
implicite (celle relative \`a la composante normale du gradient) et donc
ajout\'ee au second membre par \fort{bilsc2} ; l'autre
partie sera explicite et prise en compte dans $f_s^{\,exp}$.
\begin{itemize}
\item [$\star$] On effectue une boucle d'indice \var{IEL} sur les cellules
$\Omega_l$ de centre $L$ :
\begin{itemize}
\item [$\Rightarrow$] $\displaystyle \var{TRRIJ }= \frac{1}{2} (R^{\,n}_{ii})_L $
\item [$\Rightarrow$] $\displaystyle \var{CSTRIJ} = \rho^n_L \ C_S \ \frac{(R^{\,n}_{ii})_L}{2\,\varepsilon^n_L}$
\item [$\Rightarrow$] $\displaystyle \var{W4(IEL)} = \rho^n_L \ C_S \
\frac{(R^{\,n}_{ii})_L}{2\,\varepsilon^n_L} \ (R^{\,n}_{11})_L$
\item [$\Rightarrow$] $\displaystyle \var{W5(IEL)} = \rho^n_L \ C_S \ \frac{(R^{\,n}_{ii})_L}{2\,\varepsilon^n_L}\ (R^n_{22})_L$
\item [$\Rightarrow$] $\displaystyle \var{W6(IEL)} = \rho^n_L \ C_S \ \frac{(R^{\,n}_{ii})_L}{2\,\varepsilon^n_L} \ (R^n_{33})_L$
\end{itemize}

%\item Traitement du parall\'elisme et de la p\'eriodicit\'e.

\item [$\star$] On effectue une boucle d'indice \var{IFAC} sur les faces
purement internes $lm$ pour remplir le tableau \var{VISCF} :
\begin{itemize}
\item [$\Rightarrow$] Identification des cellules $\Omega_l$ et $\Omega_m$ de
centre respectif $L$ (variable \var{II}) et $M$ (variable \var{JJ}), se trouvant de chaque c\^ot\'e de la face
$lm$\footnote{La normale \`a la face est orient\'ee de L vers M.}.
\item [$\Rightarrow$] Calcul du carr\'e de la surface de la face. La valeur est
stock\'ee dans le tableau \var{SURFN2}.
\item [$\Rightarrow$] Interpolation du gradient de $R^{\,n}_{ij}$ \`a la face
$lm$ (gradient facette $\left[\grad{R}^{\,n}_{ij}\right]_{\,lm}$) :
\begin{equation}\notag
\left\{\begin{array}{ll}
\var{GRDPX} &= \displaystyle \frac{1}{2} \left(\var{W1(II)} + \var{W1(JJ)}
\right) \\
&\\
\var{GRDPY} &= \displaystyle \frac{1}{2} \left(\var{W2(II)} + \var{W2(JJ)} \right) \\
&\\
\var{GRDPZ} &= \displaystyle \frac{1}{2} \left(\var{W3(II)} + \var{W3(JJ)} \right)
\end{array}\right.
\end{equation}
\item [$\Rightarrow$] Calcul du gradient de $R^{\,n}_{ij}$ normal \`a la face
$lm$, $\left[\grad{R}^{\,n}_{ij}\right]_{\,lm}.\vect{n}_{\,lm}\,S_{\,lm}$.\\

$\displaystyle \var{GRDSN} =  \var{GRDPX} \ \var{SURFAC(1,IFAC)} + \var{GRDPY} \ \var{SURFAC(2,IFAC)} +  \var{GRDPZ} \ \var{SURFAC(3,IFAC)}$
$\var{SURFAC}$ est le vecteur surface de la face \var{IFAC}.


\item [$\Rightarrow$] calcul de
 $\left[\grad{R^{\,n}_{ij}} - (\grad
R^{\,n}_{ij}\,.\,\vect{n}_{\,lm})\vect{n}_{\,lm}\right]$, les vecteurs \'etant
calcul\'es \`a la face $lm$ :
\begin{equation}\notag
\left\{\begin{array}{lll}
&\displaystyle \var{GRDPX} &= \var{GRDPX} - \displaystyle\frac{\var{GRDSN}}{\var{SURFN2}} \ \var{SURFAC(1,IFAC)}\\
&&\\
&\displaystyle \var{GRDPY} &= \var{GRDPY} - \displaystyle\frac{\var{GRDSN}}{\var{SURFN2}} \ \var{SURFAC(2,IFAC)} \\
&&\\
&\displaystyle \var{GRDPZ} &= \var{GRDPZ} - \displaystyle\frac{\var{GRDSN}}{\var{SURFN2}} \ \var{SURFAC(3,IFAC)}
\end{array}\right.
\end{equation}
\item [$\Rightarrow$] finalisation du calcul de l'expression totalement
explicite
 $$\left[ \tens{D}^n\,\left( \grad{R^{\,n}_{ij}} - (\grad R^{\,n}_{ij}\,.\,\vect{n}_{\,lm})\,\vect{n}_{\,lm}\right) \right]\,.\,\vect{n}_{\,lm}$$
\begin{equation}\notag
\begin{array} {ll}
\displaystyle \var{VISCF} = &
 \displaystyle\frac{1}{2} (\ \var{W4(II)} +\ \var{W4(JJ)}) \ \var{GRDPX} \
\var{SURFAC(1,IFAC)})\ + \\
&\\
&  \displaystyle\frac{1}{2} (\ \var{W5(II)} +\ \var{W5(JJ)}) \ \var{GRDPY} \
\var{SURFAC(2,IFAC)})\ + \\
&\\
&  \displaystyle\frac{1}{2} (\ \var{W6(II)} +\ \var{W6(JJ)}) \ \var{GRDPZ} \ \var{SURFAC(3,IFAC)})
\end{array}
\end{equation}
\end{itemize}

\item [$\star$] Mise \`a z\'ero du tableau \var{VISCB}.

\item [$\star$] Appel de \fort{divmas} pour calculer la divergence de~:
 $$\tens{D}^{\,n}\,\left( \grad{R^{\,n}_{ij}} - (\grad R^{\,n}_{ij}\,.\,\vect{n}_{\,lm})\vect{n}_{\,lm}\right)$$ d\'efini aux faces dans \var{VISCF} et \var{VISCB}.

Le r\'esultat est stock\'e dans le tableau \var{W1}.\\
Ajout au second membre \var{SMBR}.\\
$\var{SMBR} = \var{SMBR} + \var{W1}$
\end{itemize}
\item Calcul de la viscosit\'e orthotrope $\gamma^n_{\,lm}$ \`a la face $lm$ de la variable principale
$R^{\,n}_{ij}$\\
Ce calcul permet au sous-programme \fort{codits} de compl\'eter le second membre
\var{SMBR} par :
\begin{equation}
\begin{array} {ll}
& \sum\limits_{m\in Vois(l)}
\mu^n_{\,lm}\,\left(\grad{R}^{\,n}_{ij}\,.\,\vect{n}_{\,lm}\right)S_{\,lm}
 + \sum\limits_{m\in Vois(l)} \left(\grad{R}^{\,n}_{ij}
\,.\,\vect{n}_{\,lm}\right)\left[\tens{D}^{\,n}\,\vect{n}_{\,lm}\right]_{\,lm}\,.\,\vect{n}_{\,lm}\
S_{\,lm}\\
& = \sum\limits_{m\in Vois(l)}(\,\mu^n_{\,lm}\, + \,\gamma^n_{\,lm}\,)\,\left(\grad{R}^{\,n}_{ij}\,.\,\vect{n}_{\,lm}\right)S_{\,lm}
\end{array}
\end{equation}
sans pr\'eciser la nature de la face $lm$, {\it via} l'appel \`a \fort{bilsc2}  et de disposer de la quantit\'e
$(\mu^n_{\,lm}\, + \gamma^n_{\,lm})$ pour construire sa
matrice simplifi\'ee.\\
\begin{itemize}
\item [$\star$] On effectue une boucle d'indice \var{IEL} sur les cellules
$\Omega_l$ :
\begin{itemize}
\item [$\Rightarrow$] $\displaystyle \var{TRRIJ }= \frac{1}{2} (R^{\,n}_{ii})_L $
\item [$\Rightarrow$] $\displaystyle \var{RCSTE} = \rho^n_L \ C_S \ \frac{ (R^{\,n}_{ii})_L}{2\,\varepsilon^n_L} $
\item [$\Rightarrow$] $\displaystyle \var{W1(IEL)} = \mu^n +\rho^n_L \ C_S \ \frac{
(R^{\,n}_{ii})_L}{2\,\varepsilon^n_L}\ (R^n_{11})_L$
\item [$\Rightarrow$] $\displaystyle \var{W2(IEL)} = \mu^n + \rho^n_L \ C_S \ \frac{ (R^{\,n}_{ii})_L}{2\,\varepsilon^n_L}\ (R^n_{22})_L$
\item [$\Rightarrow$] $\displaystyle \var{W3(IEL)} = \mu^n + \rho^n_L \ C_S \ \frac{ (R^{\,n}_{ii})_L}{2\,\varepsilon^n_L}\ (R^n_{33})_L$
\end{itemize}

\item [$\star$] Appel de \fort{visort} pour calculer la viscosit\'e orthotrope
\footnote{Comme dans le sous-programme \fort{viscfa}, on multiplie la viscosit\'e par
$\displaystyle \frac{S_{\,lm}}{\overline{L'M'}}$, o\`u $S_{\,lm}$ et
$\overline{L'M'}$ repr\'esentent respectivement la surface de la face $lm$ et la
mesure alg\'ebrique du segment reliant les projections des centres des cellules
voisines sur la normale \`a la face. On garde dans ce sous-programme  la possibilit\'e d'interpoler la viscosit\'e aux cellules lin\'eairement ou d'utiliser une moyenne harmonique. La viscosit\'e au bord est celle de la cellule de bord correspondante.}
$ \gamma^n_{\,lm} = (\tens{D}^{\,n}\,\vect{n}_{\,lm}).\vect{n}_{\,lm}$ aux faces $lm$

Le r\'esultat est stock\'e dans les tableaux \var{VISCF} et \var{VISCB}.
\end{itemize}

\item appel de \fort{codits} pour la r\'esolution de l'\'equation de
convection/diffusion/termes sources de la variable $R_{ij}$. Le terme source est
\var{SMBR}, la viscosit\'e \var{VISCF} aux faces purement internes (
resp. \var{VISCB} aux faces de bord) et \var{FLUMAS} le flux de masse interne
 ( resp. \var{FLUMAB} flux de masse au bord). Le r\'esultat est la variable $R_{ij}$ au temps
$n+1$, donc $R^{\,n+1}_{ij}$. Elle est stock\'ee dans le tableau \var{RTP} des
variables mises \`a jour.

\end{itemize}

\etape{Appel de \fort{reseps} pour la r\'esolution de la variable $\varepsilon$}

On d\'ecrit ci-dessous le sous-programme \fort{reseps}, les commentaires du sous-programme \fort{resrij} ne seront pas r\'ep\'et\'es puisque les deux sous-programmes ne diff\`erent que par leurs termes sources.

\begin{itemize}
\item Initialisation \`a z\'ero de \var{SMBR} et \var{ROVSDT}.

\item{Lecture et prise en compte des termes sources utilisateur pour la variable $\varepsilon$ :}

Appel de \fort{ustsri} pour charger les termes sources utilisateurs. Ils sont
stock\'es dans les tableaux suivants :\\
pour la cellule $\Omega_l$ repr\'esent\'ee par $\var{IEL}$ de centre $L$, on a :
\begin{equation}\notag
\left\{\begin{array}{lll}
&\var{ROVSDT(IEL)}&= |\Omega_l| \ \alpha_{\varepsilon}\\
&\var{SMBR(IEL)}&=|\Omega_l| \ \beta_{\varepsilon}\\
\end{array}\right.
\end{equation}
On affecte alors les valeurs ad\'equates au second membre \var{SMBR} et \`a la
diagonale \var{ROVSDT} comme suit :
\begin{equation}\notag
\left\{\begin{array}{lll}
&\var{SMBR(IEL)} &= \var{SMBR(IEL)} +\ |\Omega_l| \ \alpha_{\,\varepsilon} \
\varepsilon^n_L \\
&\var{ROVSDT(IEL)}&= \text{max }(-\ |\Omega_l| \ \alpha_{\,\varepsilon},0)\\
\end{array}\right.
\end{equation}

\item{Calcul du terme source de masse si $\Gamma_L > 0$ :
\begin{equation}\notag
\left\{\begin{array}{lll}
&\displaystyle \var{SMBR(IEL)} = \var{SMBR(IEL)} + |\Omega_l| \ \Gamma_L \
(\varepsilon^{\,in}_L -\varepsilon^n_L) \\
&\displaystyle \var{ROVSDT(IEL)}= \var{ROVSDT(IEL)} + |\Omega_l| \ \Gamma_L
\end{array}\right.
\end{equation}
\item Calcul du terme d'accumulation de masse et du terme instationnaire \\
On stocke $\displaystyle \var{W1}= \int_{\Omega_l}\dive\,(\rho\,\vect{u})\,d\Omega$
calcul\'e par \fort{divmas} \`a l'aide des flux de masse internes et aux bords.\\
On incr\'emente ensuite \var{SMBR} et \var{ROVSDT}.
\begin{equation}\notag
\left\{\begin{array}{lll}
&\var{SMBR(IEL)} &= \var{SMBR(IEL)} + \var{ICONV}\ \varepsilon^n_L\,(\displaystyle
\int_{\Omega_l}\dive\,(\rho\,\vect{u})\ d\Omega) \\
&\var{ROVSDT(IEL)}& = \var{ROVSDT(IEL)} +  \var{ISTAT}\,\displaystyle
\frac{\rho^n_L \ |\Omega_l|}{\Delta t^n} -  \var{ICONV}\ (\displaystyle
\int_{\Omega_l}\dive\,(\rho\,\vect{u})\ d\Omega) \\
\end{array}\right.
\end{equation}

\item Traitement du terme de production
 $\displaystyle \rho\,C_{\varepsilon_1}\,\frac{\varepsilon}{k}\,\mathcal{P}$
 et du terme de dissipation $-\,\displaystyle \rho\,C_{\varepsilon_2}\,\frac{\varepsilon}{k}\,\varepsilon$ \\
pour cela, on effectue une boucle d'indice \var{IEL} sur les cellules $\Omega_l$
de centre $L$ :
\begin{itemize}
\item [$\Rightarrow$] $\displaystyle \var{TRPROD}= \frac{1}{2} (\mathcal{P}^n_{ii})_L = \frac{1}{2} \left[ \var{PRODUC(1,IEL)} +  \var{PRODUC(2,IEL)} +  \var{PRODUC(3,IEL)} \right] $
\item [$\Rightarrow$] $\displaystyle \var{TRRIJ }= \frac{1}{2} (R^n_{ii})_L $
\item [$\Rightarrow$] $\displaystyle \var{SMBR(IEL)} = \var{SMBR(IEL)} + \rho^n_L
|\Omega_l| \left[ -C_{\varepsilon_2} \ \frac{2\,(\varepsilon^n_L)^2}{(R^n_{ii})_L} + C_{\varepsilon_1} \ \frac{\varepsilon^n_L}{(R^n_{ii})_L}\ (\mathcal{P}^n_{ii})_L \right] $
\item [$\Rightarrow$] $\displaystyle \var{ROVSDT(IEL)} = \var{ROVSDT(IEL)} + C_{\varepsilon_2} \ \rho^n_L \ |\Omega_l| \ \frac{2\,\varepsilon^n_L}{(R^n_{ii})_L}$
\end{itemize}

\item Appel de \fort{rijthe} pour le calcul des termes de gravit\'e $\mathcal{G}^n_{\varepsilon}$ et ajout dans \var{SMBR}.

$ \var{SMBR} = \var{SMBR} + \mathcal{G}^n_{\varepsilon}$\\
Ce calcul n'a lieu que si $\var{IGRARI()} = 1$.

\item Calcul de la diffusion de $\varepsilon$ \\
 Le terme $\dive \left[\mu\, \grad(\varepsilon) + \tens{A'}\,\grad(\varepsilon)
\right]$ est calcul\'e exactement de la m\^eme mani\`ere que pour les tensions
de Reynolds $R_{ij}$ en rempla\c cant $\tens{A}$ par $\tens{A'}$.

\item Appel de \fort{codits} pour la r\'esolution de l'\'equation de
convection/diffusion/termes sources de la variable principale $\varepsilon$. Le
r\'esultat $\varepsilon^{\,n+1}$ est stock\'e dans le tableau \var{RTP} des
variables mises \`a jour.
}
\end{itemize}

\etape{clippings finaux}
On passe enfin dans le sous-programme  \fort{clprij} pour faire un clipping \'eventuel
des variables $R^{\,n+1}_{ij}$ et $\varepsilon^{\,n+1}$. Le sous-programme
\fort{clprij} est appel\'e\footnote{L'option
$\var{ICLIP} = 1$ consiste en un clipping minimal des variables $R_{ii}$ et
$\varepsilon$ en prenant la valeur absolue de ces variables puisqu'elles ne
peuvent \^etre que positives.} avec $\var{ICLIP} = 2$ . Cette option
\footnote{Quand la valeur des grandeurs $R_{ii}$ ou $\varepsilon$ est
n\'egative, on la remplace par le minimum entre sa valeur absolue et (1,1)
fois la valeur obtenue au pas de temps pr\'ec\'edent.} contient l'option $\var{ICLIP} = 1$  et permet de v\'erifier l'in\'egalit\'e de Cauchy-Schwarz sur les grandeurs extra-diagonales du tenseur $\tens{R}$ (pour $i \neq j$, $|R_{ij}|^2 \le R_{ii} R_{jj}$).


%%%%%%%%%%%%%%%%%%%%%%%%%%%%%%%%%%
%%%%%%%%%%%%%%%%%%%%%%%%%%%%%%%%%%
\section{Points \`a traiter}
%%%%%%%%%%%%%%%%%%%%%%%%%%%%%%%%%%
%%%%%%%%%%%%%%%%%%%%%%%%%%%%%%%%%%
Sauf mention explicite, $\phi$ repr\'esentera une tension de Reynolds ou la dissipation turbulente ($\phi = R_{ij} \ \text{ou} \ \varepsilon$).

\begin{itemize}
\item {La vitesse utilis\'ee pour le calcul de la production est explicite. Est-ce qu'une implicitation peut am\'eliorer la pr\'ecision temporelle de $\phi$ \footnote{Cette remarque peut \^etre g\'en\'eralis\'ee. En effet, peut-on envisager d'actualiser les variables d\'ej\`a r\'esolues (sans r\'eactualiser les variables turbulentes apr\`es leur r\'esolution)? Ceci obligerait \`a modifier les sous-programmes tels que \fort{condli} qui sont appel\'es au d\'ebut de la boucle en temps.} ?}
\item {Dans quelle mesure le terme d'\'echo de paroi est-il valide ? En effet, ce terme est remis en question par certains auteurs.}
\item {On peut envisager la r\'esolution d'un syst\`eme hyperbolique pour les
tensions de Reynolds afin d'introduire un couplage avec le champ de vitesse.}
\item {Le flux au bord \var{VISCB} est annul\'e dans le sous-programme
\fort{vectds}. Peut-on envisager de mettre au bord la valeur de la variable
concern\'ee \`a la cellule de bord correspondant? De m\^eme, il faudrait se
pencher sur les hypoth\`eses sous-jacentes \`a l'annulation des contributions
aux bords de \var{VISCB} lors du calcul de : $$\left[ \tens{D}^n\,\left( \grad{R^{\,n}_{ij}} - (\grad R^{\,n}_{ij}\,.\,\vect{n}_{\,lm})\,\vect{n}_{\,lm}\right) \right]\,.\,\vect{n}_{\,lm}.$$}
\item {Un probl\`eme de pond\'eration appara\^\i t plus g\'en\'eralement. Si on prend la partie explicite de $\tens{D}\,\grad(\phi)$, la pond\'eration aux faces internes utilise le coefficient $\displaystyle\frac{1}{2}$ avec pond\'eration s\'epar\'ee de $\tens{D}$ et $\grad(\phi)$, alors que pour $\tens{E}\,\grad(\phi)$, on calcule d'abord ce terme aux cellules pour ensuite l'interpoler lin\'eairement aux faces \footnote{Cette interpolation se fait dans \fort{vectds} avec des coefficients de pond\'eration aux faces.}. Ceci donne donc deux types d'interpolations pour des termes de m\^eme nature.}
\item {On laisse la possibilit\'e dans \fort{visort} d'utiliser une moyenne
harmonique aux faces. Est-ce que ceci est valable puisque les interpolations
utilis\'ees lors du calcul de la partie explicite de $\tens{A}\,\grad{\phi}$
sont des moyennes arithm\'etiques ?}
\item {Les techniques adopt\'ees lors du clipping sont \`a revoir.}
\item {On utilise dans le cadre du mod\`ele $\displaystyle R_{ij}-\varepsilon $ une semi-implicitation de termes comme $\displaystyle \phi_{ij,1}$ ou $\displaystyle -\rho\,C_{\varepsilon_2}\,\frac{\varepsilon}{k}\,\varepsilon$. On peut envisager le m\^eme type d'implicitation dans \fort{turbke} m\^eme en pr\'esence du couplage $\displaystyle k-\varepsilon$.}
\item L'adoption d'une r\'esolution d\'ecoupl\'ee fait perdre l'invariance par rotation.
\item La formulation et l'implantation des conditions aux limites de paroi
devront \^etre v\'erifi\'ees. En effet, il semblerait que, dans certains cas, des ph\'enom\`enes
``oscillatoires'' apparaissent, sans qu'il soit ais\'e d'en d\'eterminer la cause.
\item L'implicitation partielle (du fait de la r\'esolution d\'ecoupl\'ee) des
conditions aux limites conduit souvent \`a des calculs instables. Il
conviendrait d'en conna\^\i tre la raison. L'implicitation partielle avait
\'et\'e mise en \oe uvre afin de tenter d'utiliser un pas de temps plus grand
dans le cas de jets axisym\'etriques en particulier.

\end{itemize}

%                      Code_Saturne version 1.3
%                      ------------------------
%
%     This file is part of the Code_Saturne Kernel, element of the
%     Code_Saturne CFD tool.
%
%     Copyright (C) 1998-2007 EDF S.A., France
%
%     contact: saturne-support@edf.fr
%
%     The Code_Saturne Kernel is free software; you can redistribute it
%     and/or modify it under the terms of the GNU General Public License
%     as published by the Free Software Foundation; either version 2 of
%     the License, or (at your option) any later version.
%
%     The Code_Saturne Kernel is distributed in the hope that it will be
%     useful, but WITHOUT ANY WARRANTY; without even the implied warranty
%     of MERCHANTABILITY or FITNESS FOR A PARTICULAR PURPOSE.  See the
%     GNU General Public License for more details.
%
%     You should have received a copy of the GNU General Public License
%     along with the Code_Saturne Kernel; if not, write to the
%     Free Software Foundation, Inc.,
%     51 Franklin St, Fifth Floor,
%     Boston, MA  02110-1301  USA
%
%-----------------------------------------------------------------------
%
\programme{vortex}
%
\vspace{1cm}
%%%%%%%%%%%%%%%%%%%%%%%%%%%%%%%%%%
%%%%%%%%%%%%%%%%%%%%%%%%%%%%%%%%%%
\section{Fonction}
%%%%%%%%%%%%%%%%%%%%%%%%%%%%%%%%%%
%%%%%%%%%%%%%%%%%%%%%%%%%%%%%%%%%%
Ce sous-programme est d�di� � la g�n�ration des conditions d'entr�e
turbulente utilis�es en LES.


La m�thode des vortex est bas�e sur une approche de tourbillons
ponctuels. L'id�e de la m�thode consiste � injecter des tourbillons 2D dans le
plan d'entr�e du calcul, puis � calculer le champ de vitesse induit par ces
tourbillons au centre des faces d'entr�e.

%                      Code_Saturne version 1.3
%                      ------------------------
%
%     This file is part of the Code_Saturne Kernel, element of the
%     Code_Saturne CFD tool.
% 
%     Copyright (C) 1998-2007 EDF S.A., France
%
%     contact: saturne-support@edf.fr
% 
%     The Code_Saturne Kernel is free software; you can redistribute it
%     and/or modify it under the terms of the GNU General Public License
%     as published by the Free Software Foundation; either version 2 of
%     the License, or (at your option) any later version.
% 
%     The Code_Saturne Kernel is distributed in the hope that it will be
%     useful, but WITHOUT ANY WARRANTY; without even the implied warranty
%     of MERCHANTABILITY or FITNESS FOR A PARTICULAR PURPOSE.  See the
%     GNU General Public License for more details.
% 
%     You should have received a copy of the GNU General Public License
%     along with the Code_Saturne Kernel; if not, write to the
%     Free Software Foundation, Inc.,
%     51 Franklin St, Fifth Floor,
%     Boston, MA  02110-1301  USA
%
%-----------------------------------------------------------------------
%
%%%%%%%%%%%%%%%%%%%%%%%%%%%%%%%%%%
%%%%%%%%%%%%%%%%%%%%%%%%%%%%%%%%%%
\section{Discr\'etisation}
%%%%%%%%%%%%%%%%%%%%%%%%%%%%%%%%%%
%%%%%%%%%%%%%%%%%%%%%%%%%%%%%%%%%%

Le terme convectif en $\dive(\underline{u} \otimes \rho\,\underline{u})$
introduit une non lin\'earit\'e et un couplage des composantes de la vitesse
$\vect{u}$ dans l'�quation (\ref{Base_Preduv_eqqdm}). Une lin\'earisation et un d\'ecouplage
des trois composantes de la 
vitesse sont r\'ealis\'es lors de la discr\'etisation de cette \'etape de
pr\'ediction.\\
En effet, soit :
\begin{equation}
\vect{\widetilde{u}}= \vect{u}^n + \delta \vect{u} 
\end{equation}
La contribution exacte du terme convectif \`a prendre en compte dans cette
\'etape de pr\'ediction serait :\\
\begin{equation}\label{Base_Preduv_Conv_exact}
\begin{array}{ll}
\dive(\vect{\widetilde{u}} \otimes \rho\,\vect{\widetilde{u}}) =
\dive(\vect{u}^{n} \otimes \rho\,\vect{u}^{n}) + \dive(\delta \vect{u} \otimes
\rho\,\vect{u}^{n}) +  \underbrace { \dive(\vect{u}^{n} \otimes
\rho\,\delta \vect{u})}_{\text {terme couplant lin\'eaire}} +  \underbrace { \dive(\delta \vect{u} \otimes
\rho\,\delta \vect{u})}_{\text {terme couplant et non lin\'eaire}}\\
\end{array} 
\end{equation}
Les deux derniers termes de l'expression (\ref{Base_Preduv_Conv_exact}) sont {\em a priori} n�glig�s
de mani�re � obtenir un syst\`eme en vitesse qui soit d\'ecoupl\'e et donc,
�viter l'inversion d'une matrice pouvant \^etre de tr\`es grande taille. Ces
deux termes peuvent n�anmoins �tre pris en compte de mani�re plus ou moins
approch�e par extrapolation explicite du flux de masse en $n+\theta_F$ (pour le
terme couplant lin�aire provenant de la convection de $\vect{u}^{n}$ par $\delta
\vect{u}$) et utilisation d'un point-fixe par sous it�ration sur le sous
programme \fort{navsto} (pour le terme non-lin�aire, en sp�cifiant $\var{NTERUP}>1$).

L'�quation (\ref{Base_Preduv_eqqdm}) est discr�tis�e au temps $n+\theta$ � l'aide d'un
$\theta$-sch�ma, et le tenseur des pertes de charges d�compos� en une partie
diagonale $\tens{K}_{d}$ et une extradiagonale $\tens{K}_{e}$ (soit
 $\tens{K}_{pdc}=\tens{K}_{d}+\tens{K}_{e}$).\\
$\bullet$ La pression est suppos�e connue en $n-1+\theta$ (d�calage temporel
pression-vitesse) et le gradient naturellement calcul� � cet instant.\\ 
$\bullet$ Les termes sources de viscosit� secondaire, de gradient transpos\'e,
ceux provenant du mod�le de turbulence\footnote{except� $\dive (\mu_t\ (\ggrad
\underline {u}))$}, $\rho\,\tens{K}_{\,e}\ \underline{u}$, $(\rho -\rho_0)
\underline {g}$ ainsi que $\underline{T}_{s}^{\,exp}$ et
$\Gamma\,\underline{u}_{\,i}$ sont pris de mani�re explicite au temps $n$, ou
extrapol�s suivant le sch�ma en temps choisi pour les propri�t�s physique et les
termes sources.\\ 
$\bullet$ Les termes sources $\underline{u}\,\,\dive (\rho\,\underline {u})$,
$\Gamma\,\,\underline{u}$, $T_{s}^{\,imp}\,\,\underline{u}$ et
$-\rho\,\tens{K}_{\,d}\,\,\underline{u}$ sont implicit�s est calcul�s �
l'instant $n+\theta$.\\ 
$\bullet$ Le terme de diffusion $\dive (\mu_{\,tot}\,\ggrad \underline{u})$ est
implicit� : la vitesse est prise � l'instant $n+\theta$ et la viscosit�
explicit�e ou extrapol�e.\\ 
$\bullet$ Enfin, le terme de convection en $\dive(\,\underline{u} \otimes
(\rho\underline{u})\,)$ est implicit� : la composante r�solue de la vitesse est
prise en $n+\theta$, et le flux de masse, explicit�, ou extrapol� en
$n+\theta_F$. 

Par souci de clart�, on suppose, en l'absence d'indication, que les propri�tes
physiques $\Phi$ ($\rho,\,\mu_{tot},\,...$) et le flux de masse
$(\rho\underline{u})$ sont pris respectivement aux instants $n+\theta_\Phi$ et
$n+\theta_F$, o� $\theta_\Phi$ et $\theta_F$ d�pendent des sch�mas en temps
sp�cifiquement utilis�s pour ces grandeurs\footnote{cf. \fort{introd}}. 

La discr�tisation temporelle de l'�quation (\ref{Base_Preduv_eqqdm}) s'�crit alors comme suit : 

\begin{equation}\label{Base_Preduv_eq_di1}
 \begin{array}{c}
\displaystyle \rho\,\ \frac{ \underline {\widetilde{u}}^{n+1} -\underline {u}^{n} }
{\Delta t} + \dive(\,\underline{\widetilde{u}}^{n+\theta} \otimes (\rho\underline{u})\,) -\dive
(\mu_{\,tot}\,\ggrad \underline{\widetilde{u}}^{n+\theta}) =
\\
\displaystyle
 - \grad p^{n-1+\theta} + \dive (\rho\,\underline {u})\,\underline{\widetilde{u}}^{n+\theta} +(\Gamma\,\underline{u}_{\,i})^{n+\theta_S}-\Gamma^n\,\,\underline{\widetilde{u}}^{n+\theta}
\\
\begin{array}{c}
\displaystyle
- \rho\,\tens{K}_{\,d}^{n}\,\,\underline{\widetilde{u}}^{n+\theta} - (\rho\,\tens{K}_{\,e}\ \underline{u})^{n+\theta_S} + (\underline{T}_{s}^{\,exp})^{\,n+\theta_S} + T_{s}^{\,imp}\,\,\underline{\widetilde{u}}^{n+\theta}
\\
\displaystyle
+[\dive (\mu_{\,tot}\,^t\ggrad \underline {u})]^{n+\theta_S}-\frac {2} {3}[\,\grad (\mu_{\,tot}\,\dive \underline {u})]^{n+\theta_S} + (\rho -\rho_0) \underline {g}
 - (\underline{turb})^{n+\theta_S}
\end{array}
\end{array}
\end{equation}
o\`u, par souci de simplification, on a pos\'e :
\begin{equation}
\mu_{\,tot}=
\begin{cases}
\mu+\mu_t & \text{pour les mod�les � viscosit� turbulente ou en LES}, \\
\mu & \text{pour les mod�les au second ordre ou en laminaire}
\end{cases} \ 
\end{equation}
\\
et :
\begin{equation}
\underline{turb}^{n}=
\begin{cases}
\displaystyle\frac {2}{3}\grad (\rho^{n}\,k^{n}) & \text{pour les mod�les � viscosit� turbulente}, \\
\dive(\rho^{n}\,\tens{R}^n) & \text{pour les mod�les au second ordre},\\
0 & \text{en laminaire ou en LES}\\
\end{cases}
\end{equation}
Par analogie avec l'�criture du $\theta$-sch�ma pour une variable scalaire, $\,
\underline {\widetilde{u}}^{n+\theta}$ est interpol�e � partir de la vitesse
pr�dite $\underline {\widetilde{u}}^{n+1}$ de la mani\`ere suivante\footnote{si
$\theta=1/2$, ou qu'une extrapolation est utilis�e, l'ordre 2 n'est obtenu que si
le pas de temps $\Delta t$ est uniforme en temps et en espace.}~: 
\begin{equation}
\underline {\widetilde{u}}^{n+\theta}=\theta\, \underline
{\widetilde{u}}^{n+1}+(1-\theta)\, \underline {u}^{n}\\ 
\end{equation}
Avec :
\begin{equation}
\left\{
\begin{array}{ll}
\theta = 1   & \text{Pour un sch\'ema de type Euler implicite d'ordre 1.}\\
\theta = 1/2 & \text{Pour un sch\'ema de type Cranck-Nicolson d'ordre 2.}\\
\end{array}
\right.
\end{equation}

L'�quation (\ref{Base_Preduv_eq_di1}) est alors r��crite sous la forme :

\begin{equation}\label{Base_Preduv_eq_di2}
\begin{array}{c}
\displaystyle \underbrace{\left(\frac{\rho}{\Delta t} -\theta \,\dive (\rho\,\underline {u}) +\theta \,\, \Gamma^n +
\theta \,\, \rho\,\tens{K}_{\,d}^n-\theta \,T_s^{\,imp} \right)}_{\displaystyle f_s^{imp}}\, (\underline {\,\widetilde{u}}^{n+1} -\underline {u}^{n})
\\
 +\, \theta\, \dive(\underline {\widetilde{u}}^{n+1} \otimes (\rho\underline{u}))-\, \theta\,\dive (\mu_{\,tot}\,\ggrad \underline {\widetilde{u}}^{n+1}) =
\\
-\,(1-\theta)\, \dive(\underline {u}^{n} \otimes (\rho\underline{u})) +\,(1-\theta)\,\dive (\mu_{\,tot}\,\ggrad \underline {u}^{n})
\\
f_s^{exp}\left\{
\begin{array}{c}
\displaystyle 
- \grad p^{n-1+\theta} + \dive (\rho\,\underline {u})\,\underline{u}^{n} +\,(\,\Gamma^{n}\,\underline{u}_{\,i}\,)^{n+\theta_S}- \Gamma^n\,\,\underline{u}^{n}
\\
\displaystyle
-(\,\rho\,\tens{K}_{\,e}\ \underline{u}\,)^{n+\theta_S} -\rho\,\tens{K}_{\,d}^n\ \underline{u}^{n}+ (\underline{T}_{s}^{\,exp})^{\,n+\theta_S} + T_s^{\,imp}\,\,\underline {u}^{n} 
\\
\displaystyle
+[\dive (\mu_{\,tot}\,^t\ggrad \underline {u}\,)]^{n+\theta_S}-\frac {2} {3}[\,\grad (\mu_{\,tot}\,\dive \underline {u}\,)]^{n+\theta_S} + (\rho -\rho_0) \underline {g}-(\underline{turb})^{n+\theta_S}
\end{array}
\right.
\end{array}
\end{equation}

d'o� l'�quation r�solue par le sous-programme \fort{codits} :
\begin{equation}\begin{array}{c}
\displaystyle
f_s^{\,imp}(\underline {\widetilde{u}}^{n+1}-\underline {u}^{n}) + \theta\, \dive(\underline{\widetilde{u}}^{n+1} \otimes (\rho
\underline{u})) - \theta\,\dive (\,\mu_{\,tot}\,\ggrad \underline{\widetilde{u}}^{n+1}) = 
\\\\
\displaystyle
-(1-\theta)\,\dive(\underline{u}^{n} \otimes (\rho \underline{u}))+(1-\theta)\,\dive (\,\mu_{\,tot}\,\ggrad \underline{u}^{n})
+ \underline{f}_{\,s}^{\,exp}
\end{array}
\end{equation}
La m\'ethode de discr\'etisation spatiale est d\'evelopp\'ee dans le sous-programme \fort{codits}.\\



\minititre{Remarques :}
{\tiny$\blacksquare$} Dans le cas standard sans extrapolation, le terme
$-\,T_s^{\,imp}$ n'est ajout� � $f_s^{\,imp}$ que s'il est positif afin de ne
pas affaiblir la dominance de la diagonale de la matrice � inverser.\\ 
{\tiny$\blacksquare$} Si une extrapolation est utilis�e, par contre,
$\,T_s^{\,imp}$ est ajout� � $f_s^{\,imp}$ quel que soit son signe. En effet, l'id�e intuitive qui
consiste � prendre~: 
\begin{equation}
\begin{cases}
\displaystyle
(\underline{T}_{s}^{\,exp} + T_{s}^{\,imp}\,\underline {u})^{\,n+\theta_S} &
\text{si } T_{s}^{\,imp} > 0\\ 
\displaystyle
(\underline{T}_{s}^{\,exp})^{\,n+\theta_S} + T_{s}^{\,imp}\,\underline{u}^{n+\theta} &\text{sinon}\\
\end{cases}
\end{equation} 
aboutit � une incoh�rence dans le traitement si $T_s^{imp}$ change de signe
entre deux pas de temps.\\ 
{\tiny$\blacksquare$} la partie diagonale $\tens{K}_{\,d}$ du terme
de perte de charge est utilis�e dans $f_s^{\,imp}$. En fait, pour \^etre rigoureux,
il faudrait ne retenir que les contributions positives (point signal\'e dans le
sous-programme utilisateur associ\'e \fort{uskpdc}). Cette prise en compte sera \`a am\'eliorer.\\
{\tiny$\blacksquare$} Le terme $\theta\,\Gamma^{n}-\theta\,\dive
(\rho\,\underline {u})$ ne pose pas de probl�me pour la 
dominance de la diagonale de la matrice car il est exactement compens� par le
terme de convection (cf. \fort{covofi}). 


%                      Code_Saturne version 1.3
%                      ------------------------
%
%     This file is part of the Code_Saturne Kernel, element of the
%     Code_Saturne CFD tool.
%
%     Copyright (C) 1998-2007 EDF S.A., France
%
%     contact: saturne-support@edf.fr
%
%     The Code_Saturne Kernel is free software; you can redistribute it
%     and/or modify it under the terms of the GNU General Public License
%     as published by the Free Software Foundation; either version 2 of
%     the License, or (at your option) any later version.
%
%     The Code_Saturne Kernel is distributed in the hope that it will be
%     useful, but WITHOUT ANY WARRANTY; without even the implied warranty
%     of MERCHANTABILITY or FITNESS FOR A PARTICULAR PURPOSE.  See the
%     GNU General Public License for more details.
%
%     You should have received a copy of the GNU General Public License
%     along with the Code_Saturne Kernel; if not, write to the
%     Free Software Foundation, Inc.,
%     51 Franklin St, Fifth Floor,
%     Boston, MA  02110-1301  USA
%
%-----------------------------------------------------------------------
%

%%%%%%%%%%%%%%%%%%%%%%%%%%%%%%%%%%
%%%%%%%%%%%%%%%%%%%%%%%%%%%%%%%%%%
\section{Mise en \oe uvre}
%%%%%%%%%%%%%%%%%%%%%%%%%%%%%%%%%%
%%%%%%%%%%%%%%%%%%%%%%%%%%%%%%%%%%
La num\'ero de la phase trait\'ee fait partie des arguments de \fort{turrij}. On
omettra volontairement de le pr\'eciser dans ce qui suit, on indiquera par $(\ )$ la
notion de tableau s'y rattachant.

\etape{Calcul des termes de production $\tens{\mathcal{P}}$}
\begin{itemize}
\item [$\star$] Initialisation \`a z\'ero du tableau \var{PRODUC} dimensionn\'e \`a $\var{NCEL}\times 6$.
\item [$\star$] On appelle trois fois \fort{grdcel} pour calculer les gradients des composantes de la vitesse $u$, $v$ et
$w$ prises au temps $n$.

Au final, on a :\\
$\displaystyle
\begin{array} {ll}
\var{PRODUC(1,IEL)} = & \displaystyle - 2 \left[ R_{11}^{\,n} \frac{\partial u^{\,n}} {\partial x} +R_{12}^{\,n} \frac{\partial u^{\,n}} {\partial y}+R_{13}^{\,n} \frac{\partial u^{\,n}} {\partial z} \right] \text{        (production de $R_{11}^{\,n}$)}\\
\var{PRODUC(2,IEL)} = & \displaystyle - 2 \left[ R_{12}^{\,n} \frac{\partial v^{\,n}} {\partial x} +R_{22}^{\,n} \frac{\partial v^{\,n}} {\partial y}+R_{23}^{\,n} \frac{\partial v^{\,n}} {\partial z} \right] \text{        (production de $R_{22}^{\,n}$)}\\
\var{PRODUC(3,IEL)} = & \displaystyle - 2 \left[ R_{13}^{\,n} \frac{\partial w^{\,n}} {\partial x} +R_{23}^{\,n} \frac{\partial w^{\,n}} {\partial y}+R_{33}^{\,n} \frac{\partial w^{\,n}} {\partial z} \right] \text{        (production de $R_{33}^{\,n}$)}\\
\var{PRODUC(4,IEL)} = & \displaystyle - \left[ R_{12}^{\,n} \frac{\partial u^{\,n}} {\partial x} +R_{22}^{\,n} \frac{\partial u^{\,n}} {\partial y}+R_{23}^{\,n} \frac{\partial u^{\,n}} {\partial z} \right] \\
& \displaystyle - \left[ R_{11}^{\,n} \frac{\partial v^{\,n}} {\partial x} +R_{12}^{\,n} \frac{\partial v^{\,n}} {\partial y}+R_{13}^{\,n} \frac{\partial v^{\,n}} {\partial z} \right] \text{        (production de $R_{12}^{\,n}$)} \\
\var{PRODUC(5,IEL)} = & \displaystyle - \left[ R_{13}^{\,n} \frac{\partial u^{\,n}} {\partial x} +R_{23}^{\,n} \frac{\partial u^{\,n}} {\partial y}+R_{33}^{\,n} \frac{\partial u^{\,n}} {\partial z} \right] \\
& \displaystyle - \left[ R_{11}^{\,n} \frac{\partial w^{\,n}} {\partial x} +R_{12}^{\,n} \frac{\partial w^{\,n}} {\partial y}+R_{13}^{\,n} \frac{\partial w^{\,n}} {\partial z} \right] \text{        (production de $R_{13}^{\,n}$)} \\
\var{PRODUC(6,IEL)} = & \displaystyle - \left[ R_{13}^{\,n} \frac{\partial v^{\,n}} {\partial x} +R_{23}^{\,n} \frac{\partial v^{\,n}} {\partial y}+R_{33}^{\,n} \frac{\partial v^{\,n}} {\partial z} \right] \\
& \displaystyle - \left[ R_{12}^{\,n} \frac{\partial w^{\,n}} {\partial x} +R_{22}^{\,n} \frac{\partial w^{\,n}} {\partial y}+R_{23}^{\,n} \frac{\partial w^{\,n}} {\partial z} \right]  \text{        (production de $R_{23}^{\,n}$)}
\end{array}
$
\end{itemize}

\etape{Calcul du gradient de la masse volumique $\rho^n$ prise au d\'ebut du pas
de temps courant\footnote{{\it i.e.} calcul\'ee \`a partir des
variables du pas de temps pr\'ec\'edent $n$ si n\'ecessaire.} $(n+1)$}
Ce calcul n'a lieu que si les termes de gravit\'e doivent \^etre pris en compte
($\var{IGRARI()} =1$).
\begin{itemize}
\item [$\star$] Appel de \fort{grdcel}  pour calculer le gradient de $\rho^n$
dans les trois directions de l'espace. Les conditions aux limites sur $\rho^n$
sont des conditions de Dirichlet puisque la valeur de $\rho^n$ aux faces de bord
$ik$ (variable \var{IFAC}) est connue et vaut $\rho_{\,b_{\,ik}}$. Pour \'ecrire les conditions aux limites
sous la forme habituelle, $$\rho_{\,b_{\,ik}} = \var{COEFA} + \var{COEFB}
\,\rho^n_{\,I'}$$ on pose alors $\var{COEFA}=
\var{PROPCE(IFAC,IPPROB(IROM(IPHAS)))}$ et $\var{COEFB} = \var{VISCB} = 0$.\\
$\var{PROPCE(1,IPPROB(IROM(IPHAS)))}$ (resp.$\var{VISCB}$) est utilis\'e en lieu
et place de l'habituel \var{COEFA} ($\var{COEFB}$), lors de l'appel \`a \fort{grdcel}.\\
On a donc :\\
$\displaystyle \var{GRAROX}= \frac{\partial \rho^n}{\partial x}\ $,$\displaystyle \ \var{GRAROY}= \frac{\partial
\rho^n}{\partial y}$ et $
\displaystyle \ \var{GRAROZ}= \frac{\partial \rho^n}{\partial z}\ $.

\end{itemize}

Le gradient de $\rho^n$ servira \`a calculer les termes de production par effets de gravit\'e si ces derniers sont pris en compte.

\etape{Boucle \var{ISOU} de $1$ \`a $6$ sur les tensions de Reynolds}
Pour $\var{ISOU} = 1,2,3,4,5,6$, on r\'esout respectivement et dans
l'ordre  les
\'equations de $R_{11}$, $R_{22}$, $R_{33}$, $R_{12}$, $R_{13}$ et $R_{23}$ par
l'appel au sous-programme \fort{resrij}.\\
La r\'esolution se fait par incr\'ement $\delta {R}_{ij}^{\,n+1,k+1}$ , en utilisant la m\^eme m\'ethode que
celle d\'ecrite dans le sous-programme \fort{codits}. On adopte ici les m\^emes notations.
\var{SMBR} est le second membre du syst\`eme \`a inverser, syst\`eme portant sur
les incr\'ements de la variable. \var{ROVSDT} repr\'esente la diagonale de la
matrice, hors convection/diffusion.\\
On va r\'esoudre l'\'equation (\ref{Base_Turrij_Eq_Temp_Rij}) sous forme incr\'ementale en
utilisant \fort{codits}, soit :
\begin{equation}\label{Base_Turrij_Eq_Temp_deltaRij}
\begin{array}{ll}
&\displaystyle \underbrace{\left(\frac {\rho^n_L}{\Delta t^n}
+ \rho^n_L \,C_1\,\frac{\varepsilon^n_L}{k^n_L}(1-\frac{\delta_{ij}}{3})
 - m^n_{\,lm} + \Gamma_L\,+ max(-\alpha^n_{R_{ij}},0)\right)\,|\Omega_l|}
_{\text {$\var{ROVSDT}$ contribuant
\`a la diagonale de la matrice simplifi\'ee de \fort{matrix}}}\,(\delta{R}_{ij}^{\,n+1,p+1})_{\,L}\\\\
&  \underbrace{+\sum\limits_{m\in Vois(l)}\displaystyle \left[
 m^n_{\,lm} \delta R_{ij,\,f_{\,lm}}^{\,n+1,p+1}
- (\mu^n_{\,lm} + \gamma^n_{\,lm})\
\frac{({\delta R}_{ij}^{\,n+1,p+1})_{M}-({\delta R}_{ij}^{\,n+1,p+1})_{L})}{\overline{L'M'}}\,
S_{\,lm} \right]}_{\text { convection upwind pur et diffusion non reconstruite
relatives \`a la matrice simplifi\'ee de \fort{matrix}\footnotemark}} \\
% voir le texte de la footmark plus bas
&= - \displaystyle\frac {\rho^n_L}{\Delta t^n}\,\left(\,(R^{\,n+1,p}_{ij})_L - (R^{\,n}_{ij})_L\,\right)\\
&-\,\underbrace{\displaystyle\int_{\Omega_l} \left(
\dive\,[\,(\rho\,\vect{u})^n\,R^{\,n+1,p}_{ij} - (\mu^n\,+ \gamma^n\,)\,
\grad{R^{\,n+1,p}_{ij}}\,]\right)\,d\Omega}_{\text {convection et diffusion
trait\'ees par \fort{bilsc2}}}\\
&+\displaystyle \int_{\Omega_l} \left[\,\mathcal{P}^{\,n+1,p}_{ij} + \mathcal{G}^{\,n+1,p}_{ij}
- \displaystyle\rho^n \,C_1\,\frac{\varepsilon^n}{k^n}\left[R^{\,n+1,p}_{ij}-
\frac{2}{3}\,k^n\,\delta_{ij}\right] + \phi^{\,n+1,p}_{ij,2} +
\phi^{\,n+1,p}_{ij,w}\,\right]\, d\Omega \\
& + \displaystyle\int_{\Omega_l} \left[- \frac{2}{3} \rho^n \varepsilon^n \delta_{ij}
 + \Gamma\,(\,R^{\,in}_{ij} - R^{\,n+1,p}_{ij}\,) +
\alpha^n_{R_{ij}}\,R^{\,n+1,p}_{ij}+ \beta^n_{R_{ij}}\right]\, d\Omega\\
&+ \sum\limits_{m\in
Vois(l)}\displaystyle \left[\ \tens{E}^n\,\grad{R}^{\,n+1,p}_{ij} \right]_{\,lm}\,.\,\vect{n}_{\,lm}S_{\,lm}\\
&+ \sum\limits_{m\in Vois(l)}\displaystyle \left[\
\tens{D}^n\,\grad{R}^{\,n+1,p}_{ij} \right]_{\,lm}\,.\,\vect{n}_{\,lm}S_{\,lm}\\
&- \sum\limits_{m\in Vois(l)} \gamma^n_{\,lm} \left( \grad{R}^{\,n+1,p}_{ij}\,.\,\vect{n}_{\,lm} \right)  S_{\,lm}\\
&+ \sum\limits_{m\in Vois(l)}  m^n_{\,lm}\,(R^{\,n+1,p}_{ij})_L\\
\end{array}
\end{equation}
% si on ne fait pas comme ca, il n'apparait pas
\footnotetext[\thefootnote]{Si $\var{IRIJNU} = 1$, on remplace  $\mu^n_{\,lm}$ par $(\mu +
\mu_t)^n_{\,lm}$ dans l'expression de la diffusion non reconstruite
associ\'ee \`a la matrice simplifi\'ee de \fort{matrix} ($\mu_t$ d\'esigne la
viscosit\'e turbulente calcul\'ee comme en $k-\varepsilon$).}

o\`u on rappelle :\\
pour $n$ donn\'e entier positif, on d\'efinit la suite
 $({R}_{ij}^{\,n+1,p})_{p \in \grandN}$
 par :
\begin{equation}\notag
\left\{\begin{array}{l}
{R}_{ij}^{\,n+1,0} = {R}_{ij}^{\,n}\\
{R}_{ij}^{\,n+1,p+1} = {R}_{ij}^{\,n+1,p} + \delta{R}_{ij}^{\,n+1,p+1} \\
\end{array}\right.
\end{equation}
$(\delta{R}_{ij}^{\,n+1,p+1})_{\,L}$ d\'esigne la valeur sur l'\'el\'ement
$\Omega_l$ du $\text{$(\,p+1\,)$-i\`eme}$ incr\'ement de ${R}_{ij}^{\,n+1}$,
$ m^n_{\,lm}$ le flux de masse \`a l'instant $n$ \`a travers la face $lm$,
$\delta R_{ij,\,f_{\,lm}}^{\,n+1,p+1}$ vaut $({\delta
R}_{ij}^{\,n+1,p+1})_{L}$  si $ m^n_{\,lm} \geqslant 0$, $({\delta
R}_{ij}^{\,n+1,p+1})_{M}$ sinon,
$\mathcal{P}^{\,n+1,p}_{ij}$, $\phi^{\,n+1,p}_{ij,2}$, $\phi^{\,n+1,p}_{ij,w}$ les valeurs
des quantit\'es associ\'ees correspondant \`a l'incr\'ement
$(\delta{R}_{ij}^{\,n+1,p})$.\\



Tous ces termes sont calcul\'es comme suit :
\begin{itemize}
\item Terme de gauche de l'\'equation (\ref{Base_Turrij_Eq_Temp_deltaRij})\\
Dans \fort{resrij} est calcul\'ee la variable \var{ROVSDT}. Les autres
termes sont compl\'et\'es par \fort{codits}, lors de la construction de la matrice simplifi\'ee , {\it via} un
appel au sous-programme \fort{matrix}. La quantit\'e
 $(\mu^n_{\,lm} + \gamma^n_{\,lm})$ \`a la face $lm$ est calcul\'ee lors de l'appel \`a
\fort{visort}.\\
\item Second membre de l'\'equation (\ref{Base_Turrij_Eq_Temp_deltaRij})\\
Le premier terme non d\'etaill\'e est calcul\'e par le sous-programme
\fort{bilsc2}, qui applique le sch\'ema convectif choisi par l'utilisateur, qui
reconstruit ou non selon le souhait de l'utilisateur les gradients intervenants
dans la convection-diffusion.\\
Les termes sans accolade sont, eux, compl\`etement explicites et ajout\'es au fur et
\`a mesure dans \var{SMBR} pour former
l'expression $f^{\,exp}_s$ de \fort{codits}.
\end{itemize}
On d\'ecrit ci-dessous les \'etapes de \fort{resrij} :
\begin{itemize}

\item DELTIJ = 1, pour $\var{ISOU} \leqslant 3$ et DELTIJ = 0  Si $\var{ISOU} >
3$. Cette valeur repr\'esente le symbole de Kroeneker $\delta_{ij}$.

\item Initialisation \`a z\'ero de \var{SMBR} (tableau contenant le second
membre) et \var{ROVSDT} (tableau contenant la diagonale de la matrice sauf celle
relative \`a la contribution de la
diagonale des op\'erateurs de convection et de diffusion lin\'earis\'es
\footnote{qui correspondent aux sch\'emas convectif upwind pur et diffusif sans
reconstruction.}), tous deux de dimension $\var{NCEL}$.

\item Lecture et prise en compte des termes sources utilisateur pour la variable $R_{ij}$

Appel \`a \fort{ustsri} pour charger les termes sources utilisateurs. Ils sont
stock\'es comme suit. Pour la cellule $\Omega_l$ de centre $L$, repr\'esent\'ee par $\var{IEL}$, on a :\\
\begin{equation}\notag
\left\{\begin{array}{lll}
&\var{ROVSDT(IEL)}&= |\Omega_l| \ \alpha_{R_{ij}}\\
&\var{SMBR(IEL)}&=|\Omega_l| \ \beta_{R_{ij}}\\
\end{array}\right.
\end{equation}
On affecte alors les valeurs ad\'equates au second membre \var{SMBR} et \`a la
diagonale \var{ROVSDT} comme suit :
\begin{equation}\notag
\left\{\begin{array}{lll}
&\var{SMBR(IEL)} &= \var{SMBR(IEL)} +\ |\Omega_l| \ \alpha_{R_{ij}} \ (R^n_{ij})_L \\
&\var{ROVSDT(IEL)}&= \text{max }(-\ |\Omega_l| \ \alpha_{R_{ij}},0)\\
\end{array}\right.
\end{equation}
La valeur de $ \var{ROVSDT}$ est ainsi calcul\'ee pour des raisons de stabilit\'e
num\'erique. En effet, on ne rajoute sur la diagonale que les valeurs positives,
ce qui correspond physiquement \`a impliciter les termes de rappel uniquement.
\item{Calcul du terme source de masse  si $\Gamma_L > 0$}

Appel de \fort{catsma} et incr\'ementation si n\'ecessaire de \var{SMBR} et
\var{ROVSDT} {\it via} :\\
\begin{equation}\notag
\left\{\begin{array}{lll}
\displaystyle \var{SMBR(IEL)} = \var{SMBR(IEL)} + |\Omega_l| \ \Gamma_L \
\left[(R^{\,in}_{ij})_L - (R^{\,n}_{ij})_L \right] \\
\displaystyle \var{ROVSDT(IEL)}=\var{ROVSDT(IEL)} + |\Omega_l| \ \Gamma_L
\end{array}\right.
\end{equation}
\item Calcul du terme d'accumulation de masse et du terme instationnaire

On stocke $\displaystyle \var{W1}= \int_{\Omega_l}\dive\,(\rho\,\vect{u})\,d\Omega$
calcul\'e par \fort{divmas} \`a l'aide des flux de masse aux faces internes
$ m^n_{\,lm}=\sum\limits_{m\in Vois(l)}{(\rho \vect{u})_{\,lm}^n} \text{.}\,
\vect{S}_{\,lm} $ (tableau \var{FLUMAS}) et des flux de masse aux bords  $ m^n_{\,b_{lk}} = \sum\limits_{k\in{\gamma_b(l)}}{(\rho \vect{u})_{\,{b}_{lk}}^n} \text{.}\,
\vect{S}_{\,{b}_{lk}} $ (tableau \var{FLUMAB}).
On incr\'emente ensuite \var{SMBR} et \var{ROVSDT}.
\begin{equation}\notag
\left\{\begin{array}{lll}
&\var{SMBR(IEL)} &= \var{SMBR(IEL)} + \var{ICONV}\  (R^n_{ij})_L\,(\displaystyle
\int_{\Omega_l}\dive\,(\rho\,\vect{u})\ d\Omega) \\
&\var{ROVSDT(IEL)}& = \var{ROVSDT(IEL)} +  \var{ISTAT}\,\displaystyle
\frac{\rho^n_L \ |\Omega_l|}{\Delta t^n} -  \var{ICONV}\ (\displaystyle
\int_{\Omega_l}\dive\,(\rho\,\vect{u})\ d\Omega) \\
\end{array}\right.
\end{equation}
\item Calcul des termes sources de production, des termes $\displaystyle
\phi_{\,ij,1}+\phi_{\,ij,2}$ et de la dissipation~$\displaystyle-\frac{2}{3} \varepsilon\,\delta_{\,ij}$ :

On effectue une boucle d'indice \var{IEL} sur les cellules $\Omega_l$ de centre $L$ :
\begin{itemize}
\item [$\Rightarrow$] $\displaystyle \var{TRPROD}= \frac{1}{2} (\mathcal{P}^n_{ii})_L = \frac{1}{2} \left[ \var{PRODUC(1,IEL)} +  \var{PRODUC(2,IEL)} +  \var{PRODUC(3,IEL)} \right] $
\item [$\Rightarrow$] $\displaystyle \var{TRRIJ }= \frac{1}{2} (R^n_{ii})_L $
\item [$\Rightarrow$] $\displaystyle \var{SMBR(IEL)} =\ \var{SMBR(IEL)}\ +$\\
$\ \displaystyle\rho^n_L |\Omega_l| \left[ \displaystyle
\frac{2}{3}\,\delta_{\,ij} \left( \ \displaystyle \frac{ C_2}{2}\,(\mathcal{P}^n_{ii})_L\ +
(C_1-1)\ \varepsilon^n_L\, \right)\right.$\\
$ + \left.\ (1-C_2) \ \var{PRODUC(ISOU,IEL)} -
\displaystyle C_1\ \frac{2\,\varepsilon^n_L}{(R^n_{ii})_L}\ (R^n_{ij})_L \right]$
\item [$\Rightarrow$] $\displaystyle \var{ROVSDT(IEL)} = \var{ROVSDT(IEL)} +
\rho^n_L \ |\Omega_l| \ (- \displaystyle \frac{1}{3} \ \,\delta_{\,ij} + 1) \ C_1
\ \frac{2\ \varepsilon^n_L}{(R^n_{ii})_L}$
\end{itemize}
\item Appel de \fort{rijech} pour le calcul des termes d'\'echo de paroi
 $\phi^n_{ij,w}$ si $\var{IRIJEC()}=1$ et ajout dans \var{SMBR}.\\
$\var{SMBR} = \var{SMBR} + \phi^n_{ij,w}$\\
Suivant son mode de calcul (\var{ICDPAR}), la distance � la paroi est directement accessible
par \var{RA(IDIPAR+IEL-1)} (\var{|ICDPAR|} = 1) ou bien
est calcul\'ee \`a partir de $\var{IA(IIFAPA(IPHAS)+IEL - 1)}$,
qui donne pour l'\'el\'ement $\var{IEL}$ le num\'ero de la face de bord
paroi la plus  proche (\var{|ICDPAR|} = 2). Ces tableaux ont \'et\'e renseign\'e une fois pour toutes au
d\'ebut de calcul.

\item  Appel de \fort{rijthe} pour le calcul des termes de gravit\'e $\mathcal{G}^n_{ij}$ et ajout dans \var{SMBR}.

Ce calcul n'a lieu que si $\var{IGRARI()} = 1$.
$ \var{SMBR} = \var{SMBR} + \mathcal{G}^n_{ij}$
\item Calcul de la partie explicite du terme de diffusion
 $\dive{\,\left[\tens{A}\,\grad{R}^{\,n}_{ij}\right]}$, plus pr\'ecis\'ement
des contributions du terme extradiagonal pris aux faces purement internes
(remplissage du tableau \var{VISCF}), puis aux faces de bord (remplissage du
tableau \var{VISCB}).
\begin{itemize}
\item [$\star$] Appel de \fort{grdcel} pour le calcul du gradient de
$R^{\,n}_{ij}$ dans chaque direction. Ces gradients sont respectivement
stock\'es dans les tableaux de travail \var{W1}, \var{W2} et \var{W3}.

\item [$\star$] boucle d'indice \var{IEL} sur les cellules $\Omega_l$ de centre
$L$ pour le
calcul de $\tens{E}^n\,\grad{R}^{\,n}_{ij}$ aux cellules dans un premier temps :\\
\begin{itemize}
\item [$\Rightarrow$] $\displaystyle \var{TRRIJ}= \frac{1}{2} (R^{\,n}_{ii})_L $
\item [$\Rightarrow$] $\displaystyle \var{CSTRIJ} = \rho^n_L\ C_S \ \displaystyle\frac{(R^n_{ii})_L}{2\,\varepsilon^n_L}$
\item [$\Rightarrow$] $\displaystyle \var{W4(IEL)} = \rho^n_L\ C_S\
\displaystyle\frac{(R^n_{ii})_L}{2\,\varepsilon^n_L} \left[\,(R^{\,n}_{12})_L \ \var{W2(IEL)} +
(R^{\,n}_{13})_L \ \var{W3(IEL)}\,\right]$
\item [$\Rightarrow$] $\displaystyle \var{W5(IEL)} = \rho^n_L\ C_S\
\displaystyle\frac{(R^n_{ii})_L}{2\,\varepsilon^n_L} \left[\,(R^{\,n}_{12})_L \ \var{W1(IEL)} +
(R^{\,n}_{23})_L \ \var{W3(IEL)}\,\right]$
\item [$\Rightarrow$] $\displaystyle \var{W6(IEL)} = \rho^n_L\ C_S\
\displaystyle\frac{(R^n_{ii})_L}{2\,\varepsilon^n_L} \left[\,(R^{\,n}_{13})_L \ \var{W1(IEL)} + (R^{\,n}_{23})_L \ \var{W2(IEL)}\,\right]$
\end{itemize}



\item [$\star$] Appel de \fort{vectds}\footnote{Le r\'esultat est stock\'e dans
\var{VISCF} et \var{VISCB}. Dans \fort{vectds}, les valeurs aux faces internes
sont interpol\'ees lin\'eairement sans reconstruction et \var{VISCB} est mis \`a
z\'ero.} pour assembler $\displaystyle\left[ \tens{E}^n\,\grad{R}^{\,n}_{ij}
\right]\,.\,\vect{n}_{\,lm}S_{\,lm}$ aux faces $lm$.
\item [$\star$] Appel de \fort{divmas} pour calculer la divergence du flux d\'efini par \var{VISCF} et \var{VISCB}.
Le r\'esultat est stock\'e dans \var{W4}.\\
Ajout au second membre \var{SMBR}.\\
\var{SMBR} = \var{SMBR} + \var{W4}
\end{itemize}

A l'issue de cette \'etape, seule la partie extradiagonale de la diffusion prise
enti\`erement explicite~:
 $$\sum\limits_{m\in
Vois(l)}\left[\ \tens{E}^n\,\grad{R}^{\,n}_{ij} \right]_{\,lm}\,.\,\vect{n}_{\,lm}S_{\,lm}$$ a \'et\'e calcul\'ee.\\

\item Calcul de la partie diagonale du terme de diffusion\footnote{Seule la
composante normale  du  gradient de $R_{ij}$ aux faces sera implicite.} :\\
Comme on l'a d\'eja vu, une partie de cette contribution sera trait\'ee en
implicite (celle relative \`a la composante normale du gradient) et donc
ajout\'ee au second membre par \fort{bilsc2} ; l'autre
partie sera explicite et prise en compte dans $f_s^{\,exp}$.
\begin{itemize}
\item [$\star$] On effectue une boucle d'indice \var{IEL} sur les cellules
$\Omega_l$ de centre $L$ :
\begin{itemize}
\item [$\Rightarrow$] $\displaystyle \var{TRRIJ }= \frac{1}{2} (R^{\,n}_{ii})_L $
\item [$\Rightarrow$] $\displaystyle \var{CSTRIJ} = \rho^n_L \ C_S \ \frac{(R^{\,n}_{ii})_L}{2\,\varepsilon^n_L}$
\item [$\Rightarrow$] $\displaystyle \var{W4(IEL)} = \rho^n_L \ C_S \
\frac{(R^{\,n}_{ii})_L}{2\,\varepsilon^n_L} \ (R^{\,n}_{11})_L$
\item [$\Rightarrow$] $\displaystyle \var{W5(IEL)} = \rho^n_L \ C_S \ \frac{(R^{\,n}_{ii})_L}{2\,\varepsilon^n_L}\ (R^n_{22})_L$
\item [$\Rightarrow$] $\displaystyle \var{W6(IEL)} = \rho^n_L \ C_S \ \frac{(R^{\,n}_{ii})_L}{2\,\varepsilon^n_L} \ (R^n_{33})_L$
\end{itemize}

%\item Traitement du parall\'elisme et de la p\'eriodicit\'e.

\item [$\star$] On effectue une boucle d'indice \var{IFAC} sur les faces
purement internes $lm$ pour remplir le tableau \var{VISCF} :
\begin{itemize}
\item [$\Rightarrow$] Identification des cellules $\Omega_l$ et $\Omega_m$ de
centre respectif $L$ (variable \var{II}) et $M$ (variable \var{JJ}), se trouvant de chaque c\^ot\'e de la face
$lm$\footnote{La normale \`a la face est orient\'ee de L vers M.}.
\item [$\Rightarrow$] Calcul du carr\'e de la surface de la face. La valeur est
stock\'ee dans le tableau \var{SURFN2}.
\item [$\Rightarrow$] Interpolation du gradient de $R^{\,n}_{ij}$ \`a la face
$lm$ (gradient facette $\left[\grad{R}^{\,n}_{ij}\right]_{\,lm}$) :
\begin{equation}\notag
\left\{\begin{array}{ll}
\var{GRDPX} &= \displaystyle \frac{1}{2} \left(\var{W1(II)} + \var{W1(JJ)}
\right) \\
&\\
\var{GRDPY} &= \displaystyle \frac{1}{2} \left(\var{W2(II)} + \var{W2(JJ)} \right) \\
&\\
\var{GRDPZ} &= \displaystyle \frac{1}{2} \left(\var{W3(II)} + \var{W3(JJ)} \right)
\end{array}\right.
\end{equation}
\item [$\Rightarrow$] Calcul du gradient de $R^{\,n}_{ij}$ normal \`a la face
$lm$, $\left[\grad{R}^{\,n}_{ij}\right]_{\,lm}.\vect{n}_{\,lm}\,S_{\,lm}$.\\

$\displaystyle \var{GRDSN} =  \var{GRDPX} \ \var{SURFAC(1,IFAC)} + \var{GRDPY} \ \var{SURFAC(2,IFAC)} +  \var{GRDPZ} \ \var{SURFAC(3,IFAC)}$
$\var{SURFAC}$ est le vecteur surface de la face \var{IFAC}.


\item [$\Rightarrow$] calcul de
 $\left[\grad{R^{\,n}_{ij}} - (\grad
R^{\,n}_{ij}\,.\,\vect{n}_{\,lm})\vect{n}_{\,lm}\right]$, les vecteurs \'etant
calcul\'es \`a la face $lm$ :
\begin{equation}\notag
\left\{\begin{array}{lll}
&\displaystyle \var{GRDPX} &= \var{GRDPX} - \displaystyle\frac{\var{GRDSN}}{\var{SURFN2}} \ \var{SURFAC(1,IFAC)}\\
&&\\
&\displaystyle \var{GRDPY} &= \var{GRDPY} - \displaystyle\frac{\var{GRDSN}}{\var{SURFN2}} \ \var{SURFAC(2,IFAC)} \\
&&\\
&\displaystyle \var{GRDPZ} &= \var{GRDPZ} - \displaystyle\frac{\var{GRDSN}}{\var{SURFN2}} \ \var{SURFAC(3,IFAC)}
\end{array}\right.
\end{equation}
\item [$\Rightarrow$] finalisation du calcul de l'expression totalement
explicite
 $$\left[ \tens{D}^n\,\left( \grad{R^{\,n}_{ij}} - (\grad R^{\,n}_{ij}\,.\,\vect{n}_{\,lm})\,\vect{n}_{\,lm}\right) \right]\,.\,\vect{n}_{\,lm}$$
\begin{equation}\notag
\begin{array} {ll}
\displaystyle \var{VISCF} = &
 \displaystyle\frac{1}{2} (\ \var{W4(II)} +\ \var{W4(JJ)}) \ \var{GRDPX} \
\var{SURFAC(1,IFAC)})\ + \\
&\\
&  \displaystyle\frac{1}{2} (\ \var{W5(II)} +\ \var{W5(JJ)}) \ \var{GRDPY} \
\var{SURFAC(2,IFAC)})\ + \\
&\\
&  \displaystyle\frac{1}{2} (\ \var{W6(II)} +\ \var{W6(JJ)}) \ \var{GRDPZ} \ \var{SURFAC(3,IFAC)})
\end{array}
\end{equation}
\end{itemize}

\item [$\star$] Mise \`a z\'ero du tableau \var{VISCB}.

\item [$\star$] Appel de \fort{divmas} pour calculer la divergence de~:
 $$\tens{D}^{\,n}\,\left( \grad{R^{\,n}_{ij}} - (\grad R^{\,n}_{ij}\,.\,\vect{n}_{\,lm})\vect{n}_{\,lm}\right)$$ d\'efini aux faces dans \var{VISCF} et \var{VISCB}.

Le r\'esultat est stock\'e dans le tableau \var{W1}.\\
Ajout au second membre \var{SMBR}.\\
$\var{SMBR} = \var{SMBR} + \var{W1}$
\end{itemize}
\item Calcul de la viscosit\'e orthotrope $\gamma^n_{\,lm}$ \`a la face $lm$ de la variable principale
$R^{\,n}_{ij}$\\
Ce calcul permet au sous-programme \fort{codits} de compl\'eter le second membre
\var{SMBR} par :
\begin{equation}
\begin{array} {ll}
& \sum\limits_{m\in Vois(l)}
\mu^n_{\,lm}\,\left(\grad{R}^{\,n}_{ij}\,.\,\vect{n}_{\,lm}\right)S_{\,lm}
 + \sum\limits_{m\in Vois(l)} \left(\grad{R}^{\,n}_{ij}
\,.\,\vect{n}_{\,lm}\right)\left[\tens{D}^{\,n}\,\vect{n}_{\,lm}\right]_{\,lm}\,.\,\vect{n}_{\,lm}\
S_{\,lm}\\
& = \sum\limits_{m\in Vois(l)}(\,\mu^n_{\,lm}\, + \,\gamma^n_{\,lm}\,)\,\left(\grad{R}^{\,n}_{ij}\,.\,\vect{n}_{\,lm}\right)S_{\,lm}
\end{array}
\end{equation}
sans pr\'eciser la nature de la face $lm$, {\it via} l'appel \`a \fort{bilsc2}  et de disposer de la quantit\'e
$(\mu^n_{\,lm}\, + \gamma^n_{\,lm})$ pour construire sa
matrice simplifi\'ee.\\
\begin{itemize}
\item [$\star$] On effectue une boucle d'indice \var{IEL} sur les cellules
$\Omega_l$ :
\begin{itemize}
\item [$\Rightarrow$] $\displaystyle \var{TRRIJ }= \frac{1}{2} (R^{\,n}_{ii})_L $
\item [$\Rightarrow$] $\displaystyle \var{RCSTE} = \rho^n_L \ C_S \ \frac{ (R^{\,n}_{ii})_L}{2\,\varepsilon^n_L} $
\item [$\Rightarrow$] $\displaystyle \var{W1(IEL)} = \mu^n +\rho^n_L \ C_S \ \frac{
(R^{\,n}_{ii})_L}{2\,\varepsilon^n_L}\ (R^n_{11})_L$
\item [$\Rightarrow$] $\displaystyle \var{W2(IEL)} = \mu^n + \rho^n_L \ C_S \ \frac{ (R^{\,n}_{ii})_L}{2\,\varepsilon^n_L}\ (R^n_{22})_L$
\item [$\Rightarrow$] $\displaystyle \var{W3(IEL)} = \mu^n + \rho^n_L \ C_S \ \frac{ (R^{\,n}_{ii})_L}{2\,\varepsilon^n_L}\ (R^n_{33})_L$
\end{itemize}

\item [$\star$] Appel de \fort{visort} pour calculer la viscosit\'e orthotrope
\footnote{Comme dans le sous-programme \fort{viscfa}, on multiplie la viscosit\'e par
$\displaystyle \frac{S_{\,lm}}{\overline{L'M'}}$, o\`u $S_{\,lm}$ et
$\overline{L'M'}$ repr\'esentent respectivement la surface de la face $lm$ et la
mesure alg\'ebrique du segment reliant les projections des centres des cellules
voisines sur la normale \`a la face. On garde dans ce sous-programme  la possibilit\'e d'interpoler la viscosit\'e aux cellules lin\'eairement ou d'utiliser une moyenne harmonique. La viscosit\'e au bord est celle de la cellule de bord correspondante.}
$ \gamma^n_{\,lm} = (\tens{D}^{\,n}\,\vect{n}_{\,lm}).\vect{n}_{\,lm}$ aux faces $lm$

Le r\'esultat est stock\'e dans les tableaux \var{VISCF} et \var{VISCB}.
\end{itemize}

\item appel de \fort{codits} pour la r\'esolution de l'\'equation de
convection/diffusion/termes sources de la variable $R_{ij}$. Le terme source est
\var{SMBR}, la viscosit\'e \var{VISCF} aux faces purement internes (
resp. \var{VISCB} aux faces de bord) et \var{FLUMAS} le flux de masse interne
 ( resp. \var{FLUMAB} flux de masse au bord). Le r\'esultat est la variable $R_{ij}$ au temps
$n+1$, donc $R^{\,n+1}_{ij}$. Elle est stock\'ee dans le tableau \var{RTP} des
variables mises \`a jour.

\end{itemize}

\etape{Appel de \fort{reseps} pour la r\'esolution de la variable $\varepsilon$}

On d\'ecrit ci-dessous le sous-programme \fort{reseps}, les commentaires du sous-programme \fort{resrij} ne seront pas r\'ep\'et\'es puisque les deux sous-programmes ne diff\`erent que par leurs termes sources.

\begin{itemize}
\item Initialisation \`a z\'ero de \var{SMBR} et \var{ROVSDT}.

\item{Lecture et prise en compte des termes sources utilisateur pour la variable $\varepsilon$ :}

Appel de \fort{ustsri} pour charger les termes sources utilisateurs. Ils sont
stock\'es dans les tableaux suivants :\\
pour la cellule $\Omega_l$ repr\'esent\'ee par $\var{IEL}$ de centre $L$, on a :
\begin{equation}\notag
\left\{\begin{array}{lll}
&\var{ROVSDT(IEL)}&= |\Omega_l| \ \alpha_{\varepsilon}\\
&\var{SMBR(IEL)}&=|\Omega_l| \ \beta_{\varepsilon}\\
\end{array}\right.
\end{equation}
On affecte alors les valeurs ad\'equates au second membre \var{SMBR} et \`a la
diagonale \var{ROVSDT} comme suit :
\begin{equation}\notag
\left\{\begin{array}{lll}
&\var{SMBR(IEL)} &= \var{SMBR(IEL)} +\ |\Omega_l| \ \alpha_{\,\varepsilon} \
\varepsilon^n_L \\
&\var{ROVSDT(IEL)}&= \text{max }(-\ |\Omega_l| \ \alpha_{\,\varepsilon},0)\\
\end{array}\right.
\end{equation}

\item{Calcul du terme source de masse si $\Gamma_L > 0$ :
\begin{equation}\notag
\left\{\begin{array}{lll}
&\displaystyle \var{SMBR(IEL)} = \var{SMBR(IEL)} + |\Omega_l| \ \Gamma_L \
(\varepsilon^{\,in}_L -\varepsilon^n_L) \\
&\displaystyle \var{ROVSDT(IEL)}= \var{ROVSDT(IEL)} + |\Omega_l| \ \Gamma_L
\end{array}\right.
\end{equation}
\item Calcul du terme d'accumulation de masse et du terme instationnaire \\
On stocke $\displaystyle \var{W1}= \int_{\Omega_l}\dive\,(\rho\,\vect{u})\,d\Omega$
calcul\'e par \fort{divmas} \`a l'aide des flux de masse internes et aux bords.\\
On incr\'emente ensuite \var{SMBR} et \var{ROVSDT}.
\begin{equation}\notag
\left\{\begin{array}{lll}
&\var{SMBR(IEL)} &= \var{SMBR(IEL)} + \var{ICONV}\ \varepsilon^n_L\,(\displaystyle
\int_{\Omega_l}\dive\,(\rho\,\vect{u})\ d\Omega) \\
&\var{ROVSDT(IEL)}& = \var{ROVSDT(IEL)} +  \var{ISTAT}\,\displaystyle
\frac{\rho^n_L \ |\Omega_l|}{\Delta t^n} -  \var{ICONV}\ (\displaystyle
\int_{\Omega_l}\dive\,(\rho\,\vect{u})\ d\Omega) \\
\end{array}\right.
\end{equation}

\item Traitement du terme de production
 $\displaystyle \rho\,C_{\varepsilon_1}\,\frac{\varepsilon}{k}\,\mathcal{P}$
 et du terme de dissipation $-\,\displaystyle \rho\,C_{\varepsilon_2}\,\frac{\varepsilon}{k}\,\varepsilon$ \\
pour cela, on effectue une boucle d'indice \var{IEL} sur les cellules $\Omega_l$
de centre $L$ :
\begin{itemize}
\item [$\Rightarrow$] $\displaystyle \var{TRPROD}= \frac{1}{2} (\mathcal{P}^n_{ii})_L = \frac{1}{2} \left[ \var{PRODUC(1,IEL)} +  \var{PRODUC(2,IEL)} +  \var{PRODUC(3,IEL)} \right] $
\item [$\Rightarrow$] $\displaystyle \var{TRRIJ }= \frac{1}{2} (R^n_{ii})_L $
\item [$\Rightarrow$] $\displaystyle \var{SMBR(IEL)} = \var{SMBR(IEL)} + \rho^n_L
|\Omega_l| \left[ -C_{\varepsilon_2} \ \frac{2\,(\varepsilon^n_L)^2}{(R^n_{ii})_L} + C_{\varepsilon_1} \ \frac{\varepsilon^n_L}{(R^n_{ii})_L}\ (\mathcal{P}^n_{ii})_L \right] $
\item [$\Rightarrow$] $\displaystyle \var{ROVSDT(IEL)} = \var{ROVSDT(IEL)} + C_{\varepsilon_2} \ \rho^n_L \ |\Omega_l| \ \frac{2\,\varepsilon^n_L}{(R^n_{ii})_L}$
\end{itemize}

\item Appel de \fort{rijthe} pour le calcul des termes de gravit\'e $\mathcal{G}^n_{\varepsilon}$ et ajout dans \var{SMBR}.

$ \var{SMBR} = \var{SMBR} + \mathcal{G}^n_{\varepsilon}$\\
Ce calcul n'a lieu que si $\var{IGRARI()} = 1$.

\item Calcul de la diffusion de $\varepsilon$ \\
 Le terme $\dive \left[\mu\, \grad(\varepsilon) + \tens{A'}\,\grad(\varepsilon)
\right]$ est calcul\'e exactement de la m\^eme mani\`ere que pour les tensions
de Reynolds $R_{ij}$ en rempla\c cant $\tens{A}$ par $\tens{A'}$.

\item Appel de \fort{codits} pour la r\'esolution de l'\'equation de
convection/diffusion/termes sources de la variable principale $\varepsilon$. Le
r\'esultat $\varepsilon^{\,n+1}$ est stock\'e dans le tableau \var{RTP} des
variables mises \`a jour.
}
\end{itemize}

\etape{clippings finaux}
On passe enfin dans le sous-programme  \fort{clprij} pour faire un clipping \'eventuel
des variables $R^{\,n+1}_{ij}$ et $\varepsilon^{\,n+1}$. Le sous-programme
\fort{clprij} est appel\'e\footnote{L'option
$\var{ICLIP} = 1$ consiste en un clipping minimal des variables $R_{ii}$ et
$\varepsilon$ en prenant la valeur absolue de ces variables puisqu'elles ne
peuvent \^etre que positives.} avec $\var{ICLIP} = 2$ . Cette option
\footnote{Quand la valeur des grandeurs $R_{ii}$ ou $\varepsilon$ est
n\'egative, on la remplace par le minimum entre sa valeur absolue et (1,1)
fois la valeur obtenue au pas de temps pr\'ec\'edent.} contient l'option $\var{ICLIP} = 1$  et permet de v\'erifier l'in\'egalit\'e de Cauchy-Schwarz sur les grandeurs extra-diagonales du tenseur $\tens{R}$ (pour $i \neq j$, $|R_{ij}|^2 \le R_{ii} R_{jj}$).


%%%%%%%%%%%%%%%%%%%%%%%%%%%%%%%%%%
%%%%%%%%%%%%%%%%%%%%%%%%%%%%%%%%%%
\section{Points \`a traiter}
%%%%%%%%%%%%%%%%%%%%%%%%%%%%%%%%%%
%%%%%%%%%%%%%%%%%%%%%%%%%%%%%%%%%%
Sauf mention explicite, $\phi$ repr\'esentera une tension de Reynolds ou la dissipation turbulente ($\phi = R_{ij} \ \text{ou} \ \varepsilon$).

\begin{itemize}
\item {La vitesse utilis\'ee pour le calcul de la production est explicite. Est-ce qu'une implicitation peut am\'eliorer la pr\'ecision temporelle de $\phi$ \footnote{Cette remarque peut \^etre g\'en\'eralis\'ee. En effet, peut-on envisager d'actualiser les variables d\'ej\`a r\'esolues (sans r\'eactualiser les variables turbulentes apr\`es leur r\'esolution)? Ceci obligerait \`a modifier les sous-programmes tels que \fort{condli} qui sont appel\'es au d\'ebut de la boucle en temps.} ?}
\item {Dans quelle mesure le terme d'\'echo de paroi est-il valide ? En effet, ce terme est remis en question par certains auteurs.}
\item {On peut envisager la r\'esolution d'un syst\`eme hyperbolique pour les
tensions de Reynolds afin d'introduire un couplage avec le champ de vitesse.}
\item {Le flux au bord \var{VISCB} est annul\'e dans le sous-programme
\fort{vectds}. Peut-on envisager de mettre au bord la valeur de la variable
concern\'ee \`a la cellule de bord correspondant? De m\^eme, il faudrait se
pencher sur les hypoth\`eses sous-jacentes \`a l'annulation des contributions
aux bords de \var{VISCB} lors du calcul de : $$\left[ \tens{D}^n\,\left( \grad{R^{\,n}_{ij}} - (\grad R^{\,n}_{ij}\,.\,\vect{n}_{\,lm})\,\vect{n}_{\,lm}\right) \right]\,.\,\vect{n}_{\,lm}.$$}
\item {Un probl\`eme de pond\'eration appara\^\i t plus g\'en\'eralement. Si on prend la partie explicite de $\tens{D}\,\grad(\phi)$, la pond\'eration aux faces internes utilise le coefficient $\displaystyle\frac{1}{2}$ avec pond\'eration s\'epar\'ee de $\tens{D}$ et $\grad(\phi)$, alors que pour $\tens{E}\,\grad(\phi)$, on calcule d'abord ce terme aux cellules pour ensuite l'interpoler lin\'eairement aux faces \footnote{Cette interpolation se fait dans \fort{vectds} avec des coefficients de pond\'eration aux faces.}. Ceci donne donc deux types d'interpolations pour des termes de m\^eme nature.}
\item {On laisse la possibilit\'e dans \fort{visort} d'utiliser une moyenne
harmonique aux faces. Est-ce que ceci est valable puisque les interpolations
utilis\'ees lors du calcul de la partie explicite de $\tens{A}\,\grad{\phi}$
sont des moyennes arithm\'etiques ?}
\item {Les techniques adopt\'ees lors du clipping sont \`a revoir.}
\item {On utilise dans le cadre du mod\`ele $\displaystyle R_{ij}-\varepsilon $ une semi-implicitation de termes comme $\displaystyle \phi_{ij,1}$ ou $\displaystyle -\rho\,C_{\varepsilon_2}\,\frac{\varepsilon}{k}\,\varepsilon$. On peut envisager le m\^eme type d'implicitation dans \fort{turbke} m\^eme en pr\'esence du couplage $\displaystyle k-\varepsilon$.}
\item L'adoption d'une r\'esolution d\'ecoupl\'ee fait perdre l'invariance par rotation.
\item La formulation et l'implantation des conditions aux limites de paroi
devront \^etre v\'erifi\'ees. En effet, il semblerait que, dans certains cas, des ph\'enom\`enes
``oscillatoires'' apparaissent, sans qu'il soit ais\'e d'en d\'eterminer la cause.
\item L'implicitation partielle (du fait de la r\'esolution d\'ecoupl\'ee) des
conditions aux limites conduit souvent \`a des calculs instables. Il
conviendrait d'en conna\^\i tre la raison. L'implicitation partielle avait
\'et\'e mise en \oe uvre afin de tenter d'utiliser un pas de temps plus grand
dans le cas de jets axisym\'etriques en particulier.

\end{itemize}

%                      Code_Saturne version 1.3
%                      ------------------------
%
%     This file is part of the Code_Saturne Kernel, element of the
%     Code_Saturne CFD tool.
%
%     Copyright (C) 1998-2007 EDF S.A., France
%
%     contact: saturne-support@edf.fr
%
%     The Code_Saturne Kernel is free software; you can redistribute it
%     and/or modify it under the terms of the GNU General Public License
%     as published by the Free Software Foundation; either version 2 of
%     the License, or (at your option) any later version.
%
%     The Code_Saturne Kernel is distributed in the hope that it will be
%     useful, but WITHOUT ANY WARRANTY; without even the implied warranty
%     of MERCHANTABILITY or FITNESS FOR A PARTICULAR PURPOSE.  See the
%     GNU General Public License for more details.
%
%     You should have received a copy of the GNU General Public License
%     along with the Code_Saturne Kernel; if not, write to the
%     Free Software Foundation, Inc.,
%     51 Franklin St, Fifth Floor,
%     Boston, MA  02110-1301  USA
%
%-----------------------------------------------------------------------
%
\programme{vortex}
%
\vspace{1cm}
%%%%%%%%%%%%%%%%%%%%%%%%%%%%%%%%%%
%%%%%%%%%%%%%%%%%%%%%%%%%%%%%%%%%%
\section{Fonction}
%%%%%%%%%%%%%%%%%%%%%%%%%%%%%%%%%%
%%%%%%%%%%%%%%%%%%%%%%%%%%%%%%%%%%
Ce sous-programme est d�di� � la g�n�ration des conditions d'entr�e
turbulente utilis�es en LES.


La m�thode des vortex est bas�e sur une approche de tourbillons
ponctuels. L'id�e de la m�thode consiste � injecter des tourbillons 2D dans le
plan d'entr�e du calcul, puis � calculer le champ de vitesse induit par ces
tourbillons au centre des faces d'entr�e.

%                      Code_Saturne version 1.3
%                      ------------------------
%
%     This file is part of the Code_Saturne Kernel, element of the
%     Code_Saturne CFD tool.
% 
%     Copyright (C) 1998-2007 EDF S.A., France
%
%     contact: saturne-support@edf.fr
% 
%     The Code_Saturne Kernel is free software; you can redistribute it
%     and/or modify it under the terms of the GNU General Public License
%     as published by the Free Software Foundation; either version 2 of
%     the License, or (at your option) any later version.
% 
%     The Code_Saturne Kernel is distributed in the hope that it will be
%     useful, but WITHOUT ANY WARRANTY; without even the implied warranty
%     of MERCHANTABILITY or FITNESS FOR A PARTICULAR PURPOSE.  See the
%     GNU General Public License for more details.
% 
%     You should have received a copy of the GNU General Public License
%     along with the Code_Saturne Kernel; if not, write to the
%     Free Software Foundation, Inc.,
%     51 Franklin St, Fifth Floor,
%     Boston, MA  02110-1301  USA
%
%-----------------------------------------------------------------------
%
%%%%%%%%%%%%%%%%%%%%%%%%%%%%%%%%%%
%%%%%%%%%%%%%%%%%%%%%%%%%%%%%%%%%%
\section{Discr\'etisation}
%%%%%%%%%%%%%%%%%%%%%%%%%%%%%%%%%%
%%%%%%%%%%%%%%%%%%%%%%%%%%%%%%%%%%

Le terme convectif en $\dive(\underline{u} \otimes \rho\,\underline{u})$
introduit une non lin\'earit\'e et un couplage des composantes de la vitesse
$\vect{u}$ dans l'�quation (\ref{Base_Preduv_eqqdm}). Une lin\'earisation et un d\'ecouplage
des trois composantes de la 
vitesse sont r\'ealis\'es lors de la discr\'etisation de cette \'etape de
pr\'ediction.\\
En effet, soit :
\begin{equation}
\vect{\widetilde{u}}= \vect{u}^n + \delta \vect{u} 
\end{equation}
La contribution exacte du terme convectif \`a prendre en compte dans cette
\'etape de pr\'ediction serait :\\
\begin{equation}\label{Base_Preduv_Conv_exact}
\begin{array}{ll}
\dive(\vect{\widetilde{u}} \otimes \rho\,\vect{\widetilde{u}}) =
\dive(\vect{u}^{n} \otimes \rho\,\vect{u}^{n}) + \dive(\delta \vect{u} \otimes
\rho\,\vect{u}^{n}) +  \underbrace { \dive(\vect{u}^{n} \otimes
\rho\,\delta \vect{u})}_{\text {terme couplant lin\'eaire}} +  \underbrace { \dive(\delta \vect{u} \otimes
\rho\,\delta \vect{u})}_{\text {terme couplant et non lin\'eaire}}\\
\end{array} 
\end{equation}
Les deux derniers termes de l'expression (\ref{Base_Preduv_Conv_exact}) sont {\em a priori} n�glig�s
de mani�re � obtenir un syst\`eme en vitesse qui soit d\'ecoupl\'e et donc,
�viter l'inversion d'une matrice pouvant \^etre de tr\`es grande taille. Ces
deux termes peuvent n�anmoins �tre pris en compte de mani�re plus ou moins
approch�e par extrapolation explicite du flux de masse en $n+\theta_F$ (pour le
terme couplant lin�aire provenant de la convection de $\vect{u}^{n}$ par $\delta
\vect{u}$) et utilisation d'un point-fixe par sous it�ration sur le sous
programme \fort{navsto} (pour le terme non-lin�aire, en sp�cifiant $\var{NTERUP}>1$).

L'�quation (\ref{Base_Preduv_eqqdm}) est discr�tis�e au temps $n+\theta$ � l'aide d'un
$\theta$-sch�ma, et le tenseur des pertes de charges d�compos� en une partie
diagonale $\tens{K}_{d}$ et une extradiagonale $\tens{K}_{e}$ (soit
 $\tens{K}_{pdc}=\tens{K}_{d}+\tens{K}_{e}$).\\
$\bullet$ La pression est suppos�e connue en $n-1+\theta$ (d�calage temporel
pression-vitesse) et le gradient naturellement calcul� � cet instant.\\ 
$\bullet$ Les termes sources de viscosit� secondaire, de gradient transpos\'e,
ceux provenant du mod�le de turbulence\footnote{except� $\dive (\mu_t\ (\ggrad
\underline {u}))$}, $\rho\,\tens{K}_{\,e}\ \underline{u}$, $(\rho -\rho_0)
\underline {g}$ ainsi que $\underline{T}_{s}^{\,exp}$ et
$\Gamma\,\underline{u}_{\,i}$ sont pris de mani�re explicite au temps $n$, ou
extrapol�s suivant le sch�ma en temps choisi pour les propri�t�s physique et les
termes sources.\\ 
$\bullet$ Les termes sources $\underline{u}\,\,\dive (\rho\,\underline {u})$,
$\Gamma\,\,\underline{u}$, $T_{s}^{\,imp}\,\,\underline{u}$ et
$-\rho\,\tens{K}_{\,d}\,\,\underline{u}$ sont implicit�s est calcul�s �
l'instant $n+\theta$.\\ 
$\bullet$ Le terme de diffusion $\dive (\mu_{\,tot}\,\ggrad \underline{u})$ est
implicit� : la vitesse est prise � l'instant $n+\theta$ et la viscosit�
explicit�e ou extrapol�e.\\ 
$\bullet$ Enfin, le terme de convection en $\dive(\,\underline{u} \otimes
(\rho\underline{u})\,)$ est implicit� : la composante r�solue de la vitesse est
prise en $n+\theta$, et le flux de masse, explicit�, ou extrapol� en
$n+\theta_F$. 

Par souci de clart�, on suppose, en l'absence d'indication, que les propri�tes
physiques $\Phi$ ($\rho,\,\mu_{tot},\,...$) et le flux de masse
$(\rho\underline{u})$ sont pris respectivement aux instants $n+\theta_\Phi$ et
$n+\theta_F$, o� $\theta_\Phi$ et $\theta_F$ d�pendent des sch�mas en temps
sp�cifiquement utilis�s pour ces grandeurs\footnote{cf. \fort{introd}}. 

La discr�tisation temporelle de l'�quation (\ref{Base_Preduv_eqqdm}) s'�crit alors comme suit : 

\begin{equation}\label{Base_Preduv_eq_di1}
 \begin{array}{c}
\displaystyle \rho\,\ \frac{ \underline {\widetilde{u}}^{n+1} -\underline {u}^{n} }
{\Delta t} + \dive(\,\underline{\widetilde{u}}^{n+\theta} \otimes (\rho\underline{u})\,) -\dive
(\mu_{\,tot}\,\ggrad \underline{\widetilde{u}}^{n+\theta}) =
\\
\displaystyle
 - \grad p^{n-1+\theta} + \dive (\rho\,\underline {u})\,\underline{\widetilde{u}}^{n+\theta} +(\Gamma\,\underline{u}_{\,i})^{n+\theta_S}-\Gamma^n\,\,\underline{\widetilde{u}}^{n+\theta}
\\
\begin{array}{c}
\displaystyle
- \rho\,\tens{K}_{\,d}^{n}\,\,\underline{\widetilde{u}}^{n+\theta} - (\rho\,\tens{K}_{\,e}\ \underline{u})^{n+\theta_S} + (\underline{T}_{s}^{\,exp})^{\,n+\theta_S} + T_{s}^{\,imp}\,\,\underline{\widetilde{u}}^{n+\theta}
\\
\displaystyle
+[\dive (\mu_{\,tot}\,^t\ggrad \underline {u})]^{n+\theta_S}-\frac {2} {3}[\,\grad (\mu_{\,tot}\,\dive \underline {u})]^{n+\theta_S} + (\rho -\rho_0) \underline {g}
 - (\underline{turb})^{n+\theta_S}
\end{array}
\end{array}
\end{equation}
o\`u, par souci de simplification, on a pos\'e :
\begin{equation}
\mu_{\,tot}=
\begin{cases}
\mu+\mu_t & \text{pour les mod�les � viscosit� turbulente ou en LES}, \\
\mu & \text{pour les mod�les au second ordre ou en laminaire}
\end{cases} \ 
\end{equation}
\\
et :
\begin{equation}
\underline{turb}^{n}=
\begin{cases}
\displaystyle\frac {2}{3}\grad (\rho^{n}\,k^{n}) & \text{pour les mod�les � viscosit� turbulente}, \\
\dive(\rho^{n}\,\tens{R}^n) & \text{pour les mod�les au second ordre},\\
0 & \text{en laminaire ou en LES}\\
\end{cases}
\end{equation}
Par analogie avec l'�criture du $\theta$-sch�ma pour une variable scalaire, $\,
\underline {\widetilde{u}}^{n+\theta}$ est interpol�e � partir de la vitesse
pr�dite $\underline {\widetilde{u}}^{n+1}$ de la mani\`ere suivante\footnote{si
$\theta=1/2$, ou qu'une extrapolation est utilis�e, l'ordre 2 n'est obtenu que si
le pas de temps $\Delta t$ est uniforme en temps et en espace.}~: 
\begin{equation}
\underline {\widetilde{u}}^{n+\theta}=\theta\, \underline
{\widetilde{u}}^{n+1}+(1-\theta)\, \underline {u}^{n}\\ 
\end{equation}
Avec :
\begin{equation}
\left\{
\begin{array}{ll}
\theta = 1   & \text{Pour un sch\'ema de type Euler implicite d'ordre 1.}\\
\theta = 1/2 & \text{Pour un sch\'ema de type Cranck-Nicolson d'ordre 2.}\\
\end{array}
\right.
\end{equation}

L'�quation (\ref{Base_Preduv_eq_di1}) est alors r��crite sous la forme :

\begin{equation}\label{Base_Preduv_eq_di2}
\begin{array}{c}
\displaystyle \underbrace{\left(\frac{\rho}{\Delta t} -\theta \,\dive (\rho\,\underline {u}) +\theta \,\, \Gamma^n +
\theta \,\, \rho\,\tens{K}_{\,d}^n-\theta \,T_s^{\,imp} \right)}_{\displaystyle f_s^{imp}}\, (\underline {\,\widetilde{u}}^{n+1} -\underline {u}^{n})
\\
 +\, \theta\, \dive(\underline {\widetilde{u}}^{n+1} \otimes (\rho\underline{u}))-\, \theta\,\dive (\mu_{\,tot}\,\ggrad \underline {\widetilde{u}}^{n+1}) =
\\
-\,(1-\theta)\, \dive(\underline {u}^{n} \otimes (\rho\underline{u})) +\,(1-\theta)\,\dive (\mu_{\,tot}\,\ggrad \underline {u}^{n})
\\
f_s^{exp}\left\{
\begin{array}{c}
\displaystyle 
- \grad p^{n-1+\theta} + \dive (\rho\,\underline {u})\,\underline{u}^{n} +\,(\,\Gamma^{n}\,\underline{u}_{\,i}\,)^{n+\theta_S}- \Gamma^n\,\,\underline{u}^{n}
\\
\displaystyle
-(\,\rho\,\tens{K}_{\,e}\ \underline{u}\,)^{n+\theta_S} -\rho\,\tens{K}_{\,d}^n\ \underline{u}^{n}+ (\underline{T}_{s}^{\,exp})^{\,n+\theta_S} + T_s^{\,imp}\,\,\underline {u}^{n} 
\\
\displaystyle
+[\dive (\mu_{\,tot}\,^t\ggrad \underline {u}\,)]^{n+\theta_S}-\frac {2} {3}[\,\grad (\mu_{\,tot}\,\dive \underline {u}\,)]^{n+\theta_S} + (\rho -\rho_0) \underline {g}-(\underline{turb})^{n+\theta_S}
\end{array}
\right.
\end{array}
\end{equation}

d'o� l'�quation r�solue par le sous-programme \fort{codits} :
\begin{equation}\begin{array}{c}
\displaystyle
f_s^{\,imp}(\underline {\widetilde{u}}^{n+1}-\underline {u}^{n}) + \theta\, \dive(\underline{\widetilde{u}}^{n+1} \otimes (\rho
\underline{u})) - \theta\,\dive (\,\mu_{\,tot}\,\ggrad \underline{\widetilde{u}}^{n+1}) = 
\\\\
\displaystyle
-(1-\theta)\,\dive(\underline{u}^{n} \otimes (\rho \underline{u}))+(1-\theta)\,\dive (\,\mu_{\,tot}\,\ggrad \underline{u}^{n})
+ \underline{f}_{\,s}^{\,exp}
\end{array}
\end{equation}
La m\'ethode de discr\'etisation spatiale est d\'evelopp\'ee dans le sous-programme \fort{codits}.\\



\minititre{Remarques :}
{\tiny$\blacksquare$} Dans le cas standard sans extrapolation, le terme
$-\,T_s^{\,imp}$ n'est ajout� � $f_s^{\,imp}$ que s'il est positif afin de ne
pas affaiblir la dominance de la diagonale de la matrice � inverser.\\ 
{\tiny$\blacksquare$} Si une extrapolation est utilis�e, par contre,
$\,T_s^{\,imp}$ est ajout� � $f_s^{\,imp}$ quel que soit son signe. En effet, l'id�e intuitive qui
consiste � prendre~: 
\begin{equation}
\begin{cases}
\displaystyle
(\underline{T}_{s}^{\,exp} + T_{s}^{\,imp}\,\underline {u})^{\,n+\theta_S} &
\text{si } T_{s}^{\,imp} > 0\\ 
\displaystyle
(\underline{T}_{s}^{\,exp})^{\,n+\theta_S} + T_{s}^{\,imp}\,\underline{u}^{n+\theta} &\text{sinon}\\
\end{cases}
\end{equation} 
aboutit � une incoh�rence dans le traitement si $T_s^{imp}$ change de signe
entre deux pas de temps.\\ 
{\tiny$\blacksquare$} la partie diagonale $\tens{K}_{\,d}$ du terme
de perte de charge est utilis�e dans $f_s^{\,imp}$. En fait, pour \^etre rigoureux,
il faudrait ne retenir que les contributions positives (point signal\'e dans le
sous-programme utilisateur associ\'e \fort{uskpdc}). Cette prise en compte sera \`a am\'eliorer.\\
{\tiny$\blacksquare$} Le terme $\theta\,\Gamma^{n}-\theta\,\dive
(\rho\,\underline {u})$ ne pose pas de probl�me pour la 
dominance de la diagonale de la matrice car il est exactement compens� par le
terme de convection (cf. \fort{covofi}). 


%                      Code_Saturne version 1.3
%                      ------------------------
%
%     This file is part of the Code_Saturne Kernel, element of the
%     Code_Saturne CFD tool.
%
%     Copyright (C) 1998-2007 EDF S.A., France
%
%     contact: saturne-support@edf.fr
%
%     The Code_Saturne Kernel is free software; you can redistribute it
%     and/or modify it under the terms of the GNU General Public License
%     as published by the Free Software Foundation; either version 2 of
%     the License, or (at your option) any later version.
%
%     The Code_Saturne Kernel is distributed in the hope that it will be
%     useful, but WITHOUT ANY WARRANTY; without even the implied warranty
%     of MERCHANTABILITY or FITNESS FOR A PARTICULAR PURPOSE.  See the
%     GNU General Public License for more details.
%
%     You should have received a copy of the GNU General Public License
%     along with the Code_Saturne Kernel; if not, write to the
%     Free Software Foundation, Inc.,
%     51 Franklin St, Fifth Floor,
%     Boston, MA  02110-1301  USA
%
%-----------------------------------------------------------------------
%

%%%%%%%%%%%%%%%%%%%%%%%%%%%%%%%%%%
%%%%%%%%%%%%%%%%%%%%%%%%%%%%%%%%%%
\section{Mise en \oe uvre}
%%%%%%%%%%%%%%%%%%%%%%%%%%%%%%%%%%
%%%%%%%%%%%%%%%%%%%%%%%%%%%%%%%%%%
La num\'ero de la phase trait\'ee fait partie des arguments de \fort{turrij}. On
omettra volontairement de le pr\'eciser dans ce qui suit, on indiquera par $(\ )$ la
notion de tableau s'y rattachant.

\etape{Calcul des termes de production $\tens{\mathcal{P}}$}
\begin{itemize}
\item [$\star$] Initialisation \`a z\'ero du tableau \var{PRODUC} dimensionn\'e \`a $\var{NCEL}\times 6$.
\item [$\star$] On appelle trois fois \fort{grdcel} pour calculer les gradients des composantes de la vitesse $u$, $v$ et
$w$ prises au temps $n$.

Au final, on a :\\
$\displaystyle
\begin{array} {ll}
\var{PRODUC(1,IEL)} = & \displaystyle - 2 \left[ R_{11}^{\,n} \frac{\partial u^{\,n}} {\partial x} +R_{12}^{\,n} \frac{\partial u^{\,n}} {\partial y}+R_{13}^{\,n} \frac{\partial u^{\,n}} {\partial z} \right] \text{        (production de $R_{11}^{\,n}$)}\\
\var{PRODUC(2,IEL)} = & \displaystyle - 2 \left[ R_{12}^{\,n} \frac{\partial v^{\,n}} {\partial x} +R_{22}^{\,n} \frac{\partial v^{\,n}} {\partial y}+R_{23}^{\,n} \frac{\partial v^{\,n}} {\partial z} \right] \text{        (production de $R_{22}^{\,n}$)}\\
\var{PRODUC(3,IEL)} = & \displaystyle - 2 \left[ R_{13}^{\,n} \frac{\partial w^{\,n}} {\partial x} +R_{23}^{\,n} \frac{\partial w^{\,n}} {\partial y}+R_{33}^{\,n} \frac{\partial w^{\,n}} {\partial z} \right] \text{        (production de $R_{33}^{\,n}$)}\\
\var{PRODUC(4,IEL)} = & \displaystyle - \left[ R_{12}^{\,n} \frac{\partial u^{\,n}} {\partial x} +R_{22}^{\,n} \frac{\partial u^{\,n}} {\partial y}+R_{23}^{\,n} \frac{\partial u^{\,n}} {\partial z} \right] \\
& \displaystyle - \left[ R_{11}^{\,n} \frac{\partial v^{\,n}} {\partial x} +R_{12}^{\,n} \frac{\partial v^{\,n}} {\partial y}+R_{13}^{\,n} \frac{\partial v^{\,n}} {\partial z} \right] \text{        (production de $R_{12}^{\,n}$)} \\
\var{PRODUC(5,IEL)} = & \displaystyle - \left[ R_{13}^{\,n} \frac{\partial u^{\,n}} {\partial x} +R_{23}^{\,n} \frac{\partial u^{\,n}} {\partial y}+R_{33}^{\,n} \frac{\partial u^{\,n}} {\partial z} \right] \\
& \displaystyle - \left[ R_{11}^{\,n} \frac{\partial w^{\,n}} {\partial x} +R_{12}^{\,n} \frac{\partial w^{\,n}} {\partial y}+R_{13}^{\,n} \frac{\partial w^{\,n}} {\partial z} \right] \text{        (production de $R_{13}^{\,n}$)} \\
\var{PRODUC(6,IEL)} = & \displaystyle - \left[ R_{13}^{\,n} \frac{\partial v^{\,n}} {\partial x} +R_{23}^{\,n} \frac{\partial v^{\,n}} {\partial y}+R_{33}^{\,n} \frac{\partial v^{\,n}} {\partial z} \right] \\
& \displaystyle - \left[ R_{12}^{\,n} \frac{\partial w^{\,n}} {\partial x} +R_{22}^{\,n} \frac{\partial w^{\,n}} {\partial y}+R_{23}^{\,n} \frac{\partial w^{\,n}} {\partial z} \right]  \text{        (production de $R_{23}^{\,n}$)}
\end{array}
$
\end{itemize}

\etape{Calcul du gradient de la masse volumique $\rho^n$ prise au d\'ebut du pas
de temps courant\footnote{{\it i.e.} calcul\'ee \`a partir des
variables du pas de temps pr\'ec\'edent $n$ si n\'ecessaire.} $(n+1)$}
Ce calcul n'a lieu que si les termes de gravit\'e doivent \^etre pris en compte
($\var{IGRARI()} =1$).
\begin{itemize}
\item [$\star$] Appel de \fort{grdcel}  pour calculer le gradient de $\rho^n$
dans les trois directions de l'espace. Les conditions aux limites sur $\rho^n$
sont des conditions de Dirichlet puisque la valeur de $\rho^n$ aux faces de bord
$ik$ (variable \var{IFAC}) est connue et vaut $\rho_{\,b_{\,ik}}$. Pour \'ecrire les conditions aux limites
sous la forme habituelle, $$\rho_{\,b_{\,ik}} = \var{COEFA} + \var{COEFB}
\,\rho^n_{\,I'}$$ on pose alors $\var{COEFA}=
\var{PROPCE(IFAC,IPPROB(IROM(IPHAS)))}$ et $\var{COEFB} = \var{VISCB} = 0$.\\
$\var{PROPCE(1,IPPROB(IROM(IPHAS)))}$ (resp.$\var{VISCB}$) est utilis\'e en lieu
et place de l'habituel \var{COEFA} ($\var{COEFB}$), lors de l'appel \`a \fort{grdcel}.\\
On a donc :\\
$\displaystyle \var{GRAROX}= \frac{\partial \rho^n}{\partial x}\ $,$\displaystyle \ \var{GRAROY}= \frac{\partial
\rho^n}{\partial y}$ et $
\displaystyle \ \var{GRAROZ}= \frac{\partial \rho^n}{\partial z}\ $.

\end{itemize}

Le gradient de $\rho^n$ servira \`a calculer les termes de production par effets de gravit\'e si ces derniers sont pris en compte.

\etape{Boucle \var{ISOU} de $1$ \`a $6$ sur les tensions de Reynolds}
Pour $\var{ISOU} = 1,2,3,4,5,6$, on r\'esout respectivement et dans
l'ordre  les
\'equations de $R_{11}$, $R_{22}$, $R_{33}$, $R_{12}$, $R_{13}$ et $R_{23}$ par
l'appel au sous-programme \fort{resrij}.\\
La r\'esolution se fait par incr\'ement $\delta {R}_{ij}^{\,n+1,k+1}$ , en utilisant la m\^eme m\'ethode que
celle d\'ecrite dans le sous-programme \fort{codits}. On adopte ici les m\^emes notations.
\var{SMBR} est le second membre du syst\`eme \`a inverser, syst\`eme portant sur
les incr\'ements de la variable. \var{ROVSDT} repr\'esente la diagonale de la
matrice, hors convection/diffusion.\\
On va r\'esoudre l'\'equation (\ref{Base_Turrij_Eq_Temp_Rij}) sous forme incr\'ementale en
utilisant \fort{codits}, soit :
\begin{equation}\label{Base_Turrij_Eq_Temp_deltaRij}
\begin{array}{ll}
&\displaystyle \underbrace{\left(\frac {\rho^n_L}{\Delta t^n}
+ \rho^n_L \,C_1\,\frac{\varepsilon^n_L}{k^n_L}(1-\frac{\delta_{ij}}{3})
 - m^n_{\,lm} + \Gamma_L\,+ max(-\alpha^n_{R_{ij}},0)\right)\,|\Omega_l|}
_{\text {$\var{ROVSDT}$ contribuant
\`a la diagonale de la matrice simplifi\'ee de \fort{matrix}}}\,(\delta{R}_{ij}^{\,n+1,p+1})_{\,L}\\\\
&  \underbrace{+\sum\limits_{m\in Vois(l)}\displaystyle \left[
 m^n_{\,lm} \delta R_{ij,\,f_{\,lm}}^{\,n+1,p+1}
- (\mu^n_{\,lm} + \gamma^n_{\,lm})\
\frac{({\delta R}_{ij}^{\,n+1,p+1})_{M}-({\delta R}_{ij}^{\,n+1,p+1})_{L})}{\overline{L'M'}}\,
S_{\,lm} \right]}_{\text { convection upwind pur et diffusion non reconstruite
relatives \`a la matrice simplifi\'ee de \fort{matrix}\footnotemark}} \\
% voir le texte de la footmark plus bas
&= - \displaystyle\frac {\rho^n_L}{\Delta t^n}\,\left(\,(R^{\,n+1,p}_{ij})_L - (R^{\,n}_{ij})_L\,\right)\\
&-\,\underbrace{\displaystyle\int_{\Omega_l} \left(
\dive\,[\,(\rho\,\vect{u})^n\,R^{\,n+1,p}_{ij} - (\mu^n\,+ \gamma^n\,)\,
\grad{R^{\,n+1,p}_{ij}}\,]\right)\,d\Omega}_{\text {convection et diffusion
trait\'ees par \fort{bilsc2}}}\\
&+\displaystyle \int_{\Omega_l} \left[\,\mathcal{P}^{\,n+1,p}_{ij} + \mathcal{G}^{\,n+1,p}_{ij}
- \displaystyle\rho^n \,C_1\,\frac{\varepsilon^n}{k^n}\left[R^{\,n+1,p}_{ij}-
\frac{2}{3}\,k^n\,\delta_{ij}\right] + \phi^{\,n+1,p}_{ij,2} +
\phi^{\,n+1,p}_{ij,w}\,\right]\, d\Omega \\
& + \displaystyle\int_{\Omega_l} \left[- \frac{2}{3} \rho^n \varepsilon^n \delta_{ij}
 + \Gamma\,(\,R^{\,in}_{ij} - R^{\,n+1,p}_{ij}\,) +
\alpha^n_{R_{ij}}\,R^{\,n+1,p}_{ij}+ \beta^n_{R_{ij}}\right]\, d\Omega\\
&+ \sum\limits_{m\in
Vois(l)}\displaystyle \left[\ \tens{E}^n\,\grad{R}^{\,n+1,p}_{ij} \right]_{\,lm}\,.\,\vect{n}_{\,lm}S_{\,lm}\\
&+ \sum\limits_{m\in Vois(l)}\displaystyle \left[\
\tens{D}^n\,\grad{R}^{\,n+1,p}_{ij} \right]_{\,lm}\,.\,\vect{n}_{\,lm}S_{\,lm}\\
&- \sum\limits_{m\in Vois(l)} \gamma^n_{\,lm} \left( \grad{R}^{\,n+1,p}_{ij}\,.\,\vect{n}_{\,lm} \right)  S_{\,lm}\\
&+ \sum\limits_{m\in Vois(l)}  m^n_{\,lm}\,(R^{\,n+1,p}_{ij})_L\\
\end{array}
\end{equation}
% si on ne fait pas comme ca, il n'apparait pas
\footnotetext[\thefootnote]{Si $\var{IRIJNU} = 1$, on remplace  $\mu^n_{\,lm}$ par $(\mu +
\mu_t)^n_{\,lm}$ dans l'expression de la diffusion non reconstruite
associ\'ee \`a la matrice simplifi\'ee de \fort{matrix} ($\mu_t$ d\'esigne la
viscosit\'e turbulente calcul\'ee comme en $k-\varepsilon$).}

o\`u on rappelle :\\
pour $n$ donn\'e entier positif, on d\'efinit la suite
 $({R}_{ij}^{\,n+1,p})_{p \in \grandN}$
 par :
\begin{equation}\notag
\left\{\begin{array}{l}
{R}_{ij}^{\,n+1,0} = {R}_{ij}^{\,n}\\
{R}_{ij}^{\,n+1,p+1} = {R}_{ij}^{\,n+1,p} + \delta{R}_{ij}^{\,n+1,p+1} \\
\end{array}\right.
\end{equation}
$(\delta{R}_{ij}^{\,n+1,p+1})_{\,L}$ d\'esigne la valeur sur l'\'el\'ement
$\Omega_l$ du $\text{$(\,p+1\,)$-i\`eme}$ incr\'ement de ${R}_{ij}^{\,n+1}$,
$ m^n_{\,lm}$ le flux de masse \`a l'instant $n$ \`a travers la face $lm$,
$\delta R_{ij,\,f_{\,lm}}^{\,n+1,p+1}$ vaut $({\delta
R}_{ij}^{\,n+1,p+1})_{L}$  si $ m^n_{\,lm} \geqslant 0$, $({\delta
R}_{ij}^{\,n+1,p+1})_{M}$ sinon,
$\mathcal{P}^{\,n+1,p}_{ij}$, $\phi^{\,n+1,p}_{ij,2}$, $\phi^{\,n+1,p}_{ij,w}$ les valeurs
des quantit\'es associ\'ees correspondant \`a l'incr\'ement
$(\delta{R}_{ij}^{\,n+1,p})$.\\



Tous ces termes sont calcul\'es comme suit :
\begin{itemize}
\item Terme de gauche de l'\'equation (\ref{Base_Turrij_Eq_Temp_deltaRij})\\
Dans \fort{resrij} est calcul\'ee la variable \var{ROVSDT}. Les autres
termes sont compl\'et\'es par \fort{codits}, lors de la construction de la matrice simplifi\'ee , {\it via} un
appel au sous-programme \fort{matrix}. La quantit\'e
 $(\mu^n_{\,lm} + \gamma^n_{\,lm})$ \`a la face $lm$ est calcul\'ee lors de l'appel \`a
\fort{visort}.\\
\item Second membre de l'\'equation (\ref{Base_Turrij_Eq_Temp_deltaRij})\\
Le premier terme non d\'etaill\'e est calcul\'e par le sous-programme
\fort{bilsc2}, qui applique le sch\'ema convectif choisi par l'utilisateur, qui
reconstruit ou non selon le souhait de l'utilisateur les gradients intervenants
dans la convection-diffusion.\\
Les termes sans accolade sont, eux, compl\`etement explicites et ajout\'es au fur et
\`a mesure dans \var{SMBR} pour former
l'expression $f^{\,exp}_s$ de \fort{codits}.
\end{itemize}
On d\'ecrit ci-dessous les \'etapes de \fort{resrij} :
\begin{itemize}

\item DELTIJ = 1, pour $\var{ISOU} \leqslant 3$ et DELTIJ = 0  Si $\var{ISOU} >
3$. Cette valeur repr\'esente le symbole de Kroeneker $\delta_{ij}$.

\item Initialisation \`a z\'ero de \var{SMBR} (tableau contenant le second
membre) et \var{ROVSDT} (tableau contenant la diagonale de la matrice sauf celle
relative \`a la contribution de la
diagonale des op\'erateurs de convection et de diffusion lin\'earis\'es
\footnote{qui correspondent aux sch\'emas convectif upwind pur et diffusif sans
reconstruction.}), tous deux de dimension $\var{NCEL}$.

\item Lecture et prise en compte des termes sources utilisateur pour la variable $R_{ij}$

Appel \`a \fort{ustsri} pour charger les termes sources utilisateurs. Ils sont
stock\'es comme suit. Pour la cellule $\Omega_l$ de centre $L$, repr\'esent\'ee par $\var{IEL}$, on a :\\
\begin{equation}\notag
\left\{\begin{array}{lll}
&\var{ROVSDT(IEL)}&= |\Omega_l| \ \alpha_{R_{ij}}\\
&\var{SMBR(IEL)}&=|\Omega_l| \ \beta_{R_{ij}}\\
\end{array}\right.
\end{equation}
On affecte alors les valeurs ad\'equates au second membre \var{SMBR} et \`a la
diagonale \var{ROVSDT} comme suit :
\begin{equation}\notag
\left\{\begin{array}{lll}
&\var{SMBR(IEL)} &= \var{SMBR(IEL)} +\ |\Omega_l| \ \alpha_{R_{ij}} \ (R^n_{ij})_L \\
&\var{ROVSDT(IEL)}&= \text{max }(-\ |\Omega_l| \ \alpha_{R_{ij}},0)\\
\end{array}\right.
\end{equation}
La valeur de $ \var{ROVSDT}$ est ainsi calcul\'ee pour des raisons de stabilit\'e
num\'erique. En effet, on ne rajoute sur la diagonale que les valeurs positives,
ce qui correspond physiquement \`a impliciter les termes de rappel uniquement.
\item{Calcul du terme source de masse  si $\Gamma_L > 0$}

Appel de \fort{catsma} et incr\'ementation si n\'ecessaire de \var{SMBR} et
\var{ROVSDT} {\it via} :\\
\begin{equation}\notag
\left\{\begin{array}{lll}
\displaystyle \var{SMBR(IEL)} = \var{SMBR(IEL)} + |\Omega_l| \ \Gamma_L \
\left[(R^{\,in}_{ij})_L - (R^{\,n}_{ij})_L \right] \\
\displaystyle \var{ROVSDT(IEL)}=\var{ROVSDT(IEL)} + |\Omega_l| \ \Gamma_L
\end{array}\right.
\end{equation}
\item Calcul du terme d'accumulation de masse et du terme instationnaire

On stocke $\displaystyle \var{W1}= \int_{\Omega_l}\dive\,(\rho\,\vect{u})\,d\Omega$
calcul\'e par \fort{divmas} \`a l'aide des flux de masse aux faces internes
$ m^n_{\,lm}=\sum\limits_{m\in Vois(l)}{(\rho \vect{u})_{\,lm}^n} \text{.}\,
\vect{S}_{\,lm} $ (tableau \var{FLUMAS}) et des flux de masse aux bords  $ m^n_{\,b_{lk}} = \sum\limits_{k\in{\gamma_b(l)}}{(\rho \vect{u})_{\,{b}_{lk}}^n} \text{.}\,
\vect{S}_{\,{b}_{lk}} $ (tableau \var{FLUMAB}).
On incr\'emente ensuite \var{SMBR} et \var{ROVSDT}.
\begin{equation}\notag
\left\{\begin{array}{lll}
&\var{SMBR(IEL)} &= \var{SMBR(IEL)} + \var{ICONV}\  (R^n_{ij})_L\,(\displaystyle
\int_{\Omega_l}\dive\,(\rho\,\vect{u})\ d\Omega) \\
&\var{ROVSDT(IEL)}& = \var{ROVSDT(IEL)} +  \var{ISTAT}\,\displaystyle
\frac{\rho^n_L \ |\Omega_l|}{\Delta t^n} -  \var{ICONV}\ (\displaystyle
\int_{\Omega_l}\dive\,(\rho\,\vect{u})\ d\Omega) \\
\end{array}\right.
\end{equation}
\item Calcul des termes sources de production, des termes $\displaystyle
\phi_{\,ij,1}+\phi_{\,ij,2}$ et de la dissipation~$\displaystyle-\frac{2}{3} \varepsilon\,\delta_{\,ij}$ :

On effectue une boucle d'indice \var{IEL} sur les cellules $\Omega_l$ de centre $L$ :
\begin{itemize}
\item [$\Rightarrow$] $\displaystyle \var{TRPROD}= \frac{1}{2} (\mathcal{P}^n_{ii})_L = \frac{1}{2} \left[ \var{PRODUC(1,IEL)} +  \var{PRODUC(2,IEL)} +  \var{PRODUC(3,IEL)} \right] $
\item [$\Rightarrow$] $\displaystyle \var{TRRIJ }= \frac{1}{2} (R^n_{ii})_L $
\item [$\Rightarrow$] $\displaystyle \var{SMBR(IEL)} =\ \var{SMBR(IEL)}\ +$\\
$\ \displaystyle\rho^n_L |\Omega_l| \left[ \displaystyle
\frac{2}{3}\,\delta_{\,ij} \left( \ \displaystyle \frac{ C_2}{2}\,(\mathcal{P}^n_{ii})_L\ +
(C_1-1)\ \varepsilon^n_L\, \right)\right.$\\
$ + \left.\ (1-C_2) \ \var{PRODUC(ISOU,IEL)} -
\displaystyle C_1\ \frac{2\,\varepsilon^n_L}{(R^n_{ii})_L}\ (R^n_{ij})_L \right]$
\item [$\Rightarrow$] $\displaystyle \var{ROVSDT(IEL)} = \var{ROVSDT(IEL)} +
\rho^n_L \ |\Omega_l| \ (- \displaystyle \frac{1}{3} \ \,\delta_{\,ij} + 1) \ C_1
\ \frac{2\ \varepsilon^n_L}{(R^n_{ii})_L}$
\end{itemize}
\item Appel de \fort{rijech} pour le calcul des termes d'\'echo de paroi
 $\phi^n_{ij,w}$ si $\var{IRIJEC()}=1$ et ajout dans \var{SMBR}.\\
$\var{SMBR} = \var{SMBR} + \phi^n_{ij,w}$\\
Suivant son mode de calcul (\var{ICDPAR}), la distance � la paroi est directement accessible
par \var{RA(IDIPAR+IEL-1)} (\var{|ICDPAR|} = 1) ou bien
est calcul\'ee \`a partir de $\var{IA(IIFAPA(IPHAS)+IEL - 1)}$,
qui donne pour l'\'el\'ement $\var{IEL}$ le num\'ero de la face de bord
paroi la plus  proche (\var{|ICDPAR|} = 2). Ces tableaux ont \'et\'e renseign\'e une fois pour toutes au
d\'ebut de calcul.

\item  Appel de \fort{rijthe} pour le calcul des termes de gravit\'e $\mathcal{G}^n_{ij}$ et ajout dans \var{SMBR}.

Ce calcul n'a lieu que si $\var{IGRARI()} = 1$.
$ \var{SMBR} = \var{SMBR} + \mathcal{G}^n_{ij}$
\item Calcul de la partie explicite du terme de diffusion
 $\dive{\,\left[\tens{A}\,\grad{R}^{\,n}_{ij}\right]}$, plus pr\'ecis\'ement
des contributions du terme extradiagonal pris aux faces purement internes
(remplissage du tableau \var{VISCF}), puis aux faces de bord (remplissage du
tableau \var{VISCB}).
\begin{itemize}
\item [$\star$] Appel de \fort{grdcel} pour le calcul du gradient de
$R^{\,n}_{ij}$ dans chaque direction. Ces gradients sont respectivement
stock\'es dans les tableaux de travail \var{W1}, \var{W2} et \var{W3}.

\item [$\star$] boucle d'indice \var{IEL} sur les cellules $\Omega_l$ de centre
$L$ pour le
calcul de $\tens{E}^n\,\grad{R}^{\,n}_{ij}$ aux cellules dans un premier temps :\\
\begin{itemize}
\item [$\Rightarrow$] $\displaystyle \var{TRRIJ}= \frac{1}{2} (R^{\,n}_{ii})_L $
\item [$\Rightarrow$] $\displaystyle \var{CSTRIJ} = \rho^n_L\ C_S \ \displaystyle\frac{(R^n_{ii})_L}{2\,\varepsilon^n_L}$
\item [$\Rightarrow$] $\displaystyle \var{W4(IEL)} = \rho^n_L\ C_S\
\displaystyle\frac{(R^n_{ii})_L}{2\,\varepsilon^n_L} \left[\,(R^{\,n}_{12})_L \ \var{W2(IEL)} +
(R^{\,n}_{13})_L \ \var{W3(IEL)}\,\right]$
\item [$\Rightarrow$] $\displaystyle \var{W5(IEL)} = \rho^n_L\ C_S\
\displaystyle\frac{(R^n_{ii})_L}{2\,\varepsilon^n_L} \left[\,(R^{\,n}_{12})_L \ \var{W1(IEL)} +
(R^{\,n}_{23})_L \ \var{W3(IEL)}\,\right]$
\item [$\Rightarrow$] $\displaystyle \var{W6(IEL)} = \rho^n_L\ C_S\
\displaystyle\frac{(R^n_{ii})_L}{2\,\varepsilon^n_L} \left[\,(R^{\,n}_{13})_L \ \var{W1(IEL)} + (R^{\,n}_{23})_L \ \var{W2(IEL)}\,\right]$
\end{itemize}



\item [$\star$] Appel de \fort{vectds}\footnote{Le r\'esultat est stock\'e dans
\var{VISCF} et \var{VISCB}. Dans \fort{vectds}, les valeurs aux faces internes
sont interpol\'ees lin\'eairement sans reconstruction et \var{VISCB} est mis \`a
z\'ero.} pour assembler $\displaystyle\left[ \tens{E}^n\,\grad{R}^{\,n}_{ij}
\right]\,.\,\vect{n}_{\,lm}S_{\,lm}$ aux faces $lm$.
\item [$\star$] Appel de \fort{divmas} pour calculer la divergence du flux d\'efini par \var{VISCF} et \var{VISCB}.
Le r\'esultat est stock\'e dans \var{W4}.\\
Ajout au second membre \var{SMBR}.\\
\var{SMBR} = \var{SMBR} + \var{W4}
\end{itemize}

A l'issue de cette \'etape, seule la partie extradiagonale de la diffusion prise
enti\`erement explicite~:
 $$\sum\limits_{m\in
Vois(l)}\left[\ \tens{E}^n\,\grad{R}^{\,n}_{ij} \right]_{\,lm}\,.\,\vect{n}_{\,lm}S_{\,lm}$$ a \'et\'e calcul\'ee.\\

\item Calcul de la partie diagonale du terme de diffusion\footnote{Seule la
composante normale  du  gradient de $R_{ij}$ aux faces sera implicite.} :\\
Comme on l'a d\'eja vu, une partie de cette contribution sera trait\'ee en
implicite (celle relative \`a la composante normale du gradient) et donc
ajout\'ee au second membre par \fort{bilsc2} ; l'autre
partie sera explicite et prise en compte dans $f_s^{\,exp}$.
\begin{itemize}
\item [$\star$] On effectue une boucle d'indice \var{IEL} sur les cellules
$\Omega_l$ de centre $L$ :
\begin{itemize}
\item [$\Rightarrow$] $\displaystyle \var{TRRIJ }= \frac{1}{2} (R^{\,n}_{ii})_L $
\item [$\Rightarrow$] $\displaystyle \var{CSTRIJ} = \rho^n_L \ C_S \ \frac{(R^{\,n}_{ii})_L}{2\,\varepsilon^n_L}$
\item [$\Rightarrow$] $\displaystyle \var{W4(IEL)} = \rho^n_L \ C_S \
\frac{(R^{\,n}_{ii})_L}{2\,\varepsilon^n_L} \ (R^{\,n}_{11})_L$
\item [$\Rightarrow$] $\displaystyle \var{W5(IEL)} = \rho^n_L \ C_S \ \frac{(R^{\,n}_{ii})_L}{2\,\varepsilon^n_L}\ (R^n_{22})_L$
\item [$\Rightarrow$] $\displaystyle \var{W6(IEL)} = \rho^n_L \ C_S \ \frac{(R^{\,n}_{ii})_L}{2\,\varepsilon^n_L} \ (R^n_{33})_L$
\end{itemize}

%\item Traitement du parall\'elisme et de la p\'eriodicit\'e.

\item [$\star$] On effectue une boucle d'indice \var{IFAC} sur les faces
purement internes $lm$ pour remplir le tableau \var{VISCF} :
\begin{itemize}
\item [$\Rightarrow$] Identification des cellules $\Omega_l$ et $\Omega_m$ de
centre respectif $L$ (variable \var{II}) et $M$ (variable \var{JJ}), se trouvant de chaque c\^ot\'e de la face
$lm$\footnote{La normale \`a la face est orient\'ee de L vers M.}.
\item [$\Rightarrow$] Calcul du carr\'e de la surface de la face. La valeur est
stock\'ee dans le tableau \var{SURFN2}.
\item [$\Rightarrow$] Interpolation du gradient de $R^{\,n}_{ij}$ \`a la face
$lm$ (gradient facette $\left[\grad{R}^{\,n}_{ij}\right]_{\,lm}$) :
\begin{equation}\notag
\left\{\begin{array}{ll}
\var{GRDPX} &= \displaystyle \frac{1}{2} \left(\var{W1(II)} + \var{W1(JJ)}
\right) \\
&\\
\var{GRDPY} &= \displaystyle \frac{1}{2} \left(\var{W2(II)} + \var{W2(JJ)} \right) \\
&\\
\var{GRDPZ} &= \displaystyle \frac{1}{2} \left(\var{W3(II)} + \var{W3(JJ)} \right)
\end{array}\right.
\end{equation}
\item [$\Rightarrow$] Calcul du gradient de $R^{\,n}_{ij}$ normal \`a la face
$lm$, $\left[\grad{R}^{\,n}_{ij}\right]_{\,lm}.\vect{n}_{\,lm}\,S_{\,lm}$.\\

$\displaystyle \var{GRDSN} =  \var{GRDPX} \ \var{SURFAC(1,IFAC)} + \var{GRDPY} \ \var{SURFAC(2,IFAC)} +  \var{GRDPZ} \ \var{SURFAC(3,IFAC)}$
$\var{SURFAC}$ est le vecteur surface de la face \var{IFAC}.


\item [$\Rightarrow$] calcul de
 $\left[\grad{R^{\,n}_{ij}} - (\grad
R^{\,n}_{ij}\,.\,\vect{n}_{\,lm})\vect{n}_{\,lm}\right]$, les vecteurs \'etant
calcul\'es \`a la face $lm$ :
\begin{equation}\notag
\left\{\begin{array}{lll}
&\displaystyle \var{GRDPX} &= \var{GRDPX} - \displaystyle\frac{\var{GRDSN}}{\var{SURFN2}} \ \var{SURFAC(1,IFAC)}\\
&&\\
&\displaystyle \var{GRDPY} &= \var{GRDPY} - \displaystyle\frac{\var{GRDSN}}{\var{SURFN2}} \ \var{SURFAC(2,IFAC)} \\
&&\\
&\displaystyle \var{GRDPZ} &= \var{GRDPZ} - \displaystyle\frac{\var{GRDSN}}{\var{SURFN2}} \ \var{SURFAC(3,IFAC)}
\end{array}\right.
\end{equation}
\item [$\Rightarrow$] finalisation du calcul de l'expression totalement
explicite
 $$\left[ \tens{D}^n\,\left( \grad{R^{\,n}_{ij}} - (\grad R^{\,n}_{ij}\,.\,\vect{n}_{\,lm})\,\vect{n}_{\,lm}\right) \right]\,.\,\vect{n}_{\,lm}$$
\begin{equation}\notag
\begin{array} {ll}
\displaystyle \var{VISCF} = &
 \displaystyle\frac{1}{2} (\ \var{W4(II)} +\ \var{W4(JJ)}) \ \var{GRDPX} \
\var{SURFAC(1,IFAC)})\ + \\
&\\
&  \displaystyle\frac{1}{2} (\ \var{W5(II)} +\ \var{W5(JJ)}) \ \var{GRDPY} \
\var{SURFAC(2,IFAC)})\ + \\
&\\
&  \displaystyle\frac{1}{2} (\ \var{W6(II)} +\ \var{W6(JJ)}) \ \var{GRDPZ} \ \var{SURFAC(3,IFAC)})
\end{array}
\end{equation}
\end{itemize}

\item [$\star$] Mise \`a z\'ero du tableau \var{VISCB}.

\item [$\star$] Appel de \fort{divmas} pour calculer la divergence de~:
 $$\tens{D}^{\,n}\,\left( \grad{R^{\,n}_{ij}} - (\grad R^{\,n}_{ij}\,.\,\vect{n}_{\,lm})\vect{n}_{\,lm}\right)$$ d\'efini aux faces dans \var{VISCF} et \var{VISCB}.

Le r\'esultat est stock\'e dans le tableau \var{W1}.\\
Ajout au second membre \var{SMBR}.\\
$\var{SMBR} = \var{SMBR} + \var{W1}$
\end{itemize}
\item Calcul de la viscosit\'e orthotrope $\gamma^n_{\,lm}$ \`a la face $lm$ de la variable principale
$R^{\,n}_{ij}$\\
Ce calcul permet au sous-programme \fort{codits} de compl\'eter le second membre
\var{SMBR} par :
\begin{equation}
\begin{array} {ll}
& \sum\limits_{m\in Vois(l)}
\mu^n_{\,lm}\,\left(\grad{R}^{\,n}_{ij}\,.\,\vect{n}_{\,lm}\right)S_{\,lm}
 + \sum\limits_{m\in Vois(l)} \left(\grad{R}^{\,n}_{ij}
\,.\,\vect{n}_{\,lm}\right)\left[\tens{D}^{\,n}\,\vect{n}_{\,lm}\right]_{\,lm}\,.\,\vect{n}_{\,lm}\
S_{\,lm}\\
& = \sum\limits_{m\in Vois(l)}(\,\mu^n_{\,lm}\, + \,\gamma^n_{\,lm}\,)\,\left(\grad{R}^{\,n}_{ij}\,.\,\vect{n}_{\,lm}\right)S_{\,lm}
\end{array}
\end{equation}
sans pr\'eciser la nature de la face $lm$, {\it via} l'appel \`a \fort{bilsc2}  et de disposer de la quantit\'e
$(\mu^n_{\,lm}\, + \gamma^n_{\,lm})$ pour construire sa
matrice simplifi\'ee.\\
\begin{itemize}
\item [$\star$] On effectue une boucle d'indice \var{IEL} sur les cellules
$\Omega_l$ :
\begin{itemize}
\item [$\Rightarrow$] $\displaystyle \var{TRRIJ }= \frac{1}{2} (R^{\,n}_{ii})_L $
\item [$\Rightarrow$] $\displaystyle \var{RCSTE} = \rho^n_L \ C_S \ \frac{ (R^{\,n}_{ii})_L}{2\,\varepsilon^n_L} $
\item [$\Rightarrow$] $\displaystyle \var{W1(IEL)} = \mu^n +\rho^n_L \ C_S \ \frac{
(R^{\,n}_{ii})_L}{2\,\varepsilon^n_L}\ (R^n_{11})_L$
\item [$\Rightarrow$] $\displaystyle \var{W2(IEL)} = \mu^n + \rho^n_L \ C_S \ \frac{ (R^{\,n}_{ii})_L}{2\,\varepsilon^n_L}\ (R^n_{22})_L$
\item [$\Rightarrow$] $\displaystyle \var{W3(IEL)} = \mu^n + \rho^n_L \ C_S \ \frac{ (R^{\,n}_{ii})_L}{2\,\varepsilon^n_L}\ (R^n_{33})_L$
\end{itemize}

\item [$\star$] Appel de \fort{visort} pour calculer la viscosit\'e orthotrope
\footnote{Comme dans le sous-programme \fort{viscfa}, on multiplie la viscosit\'e par
$\displaystyle \frac{S_{\,lm}}{\overline{L'M'}}$, o\`u $S_{\,lm}$ et
$\overline{L'M'}$ repr\'esentent respectivement la surface de la face $lm$ et la
mesure alg\'ebrique du segment reliant les projections des centres des cellules
voisines sur la normale \`a la face. On garde dans ce sous-programme  la possibilit\'e d'interpoler la viscosit\'e aux cellules lin\'eairement ou d'utiliser une moyenne harmonique. La viscosit\'e au bord est celle de la cellule de bord correspondante.}
$ \gamma^n_{\,lm} = (\tens{D}^{\,n}\,\vect{n}_{\,lm}).\vect{n}_{\,lm}$ aux faces $lm$

Le r\'esultat est stock\'e dans les tableaux \var{VISCF} et \var{VISCB}.
\end{itemize}

\item appel de \fort{codits} pour la r\'esolution de l'\'equation de
convection/diffusion/termes sources de la variable $R_{ij}$. Le terme source est
\var{SMBR}, la viscosit\'e \var{VISCF} aux faces purement internes (
resp. \var{VISCB} aux faces de bord) et \var{FLUMAS} le flux de masse interne
 ( resp. \var{FLUMAB} flux de masse au bord). Le r\'esultat est la variable $R_{ij}$ au temps
$n+1$, donc $R^{\,n+1}_{ij}$. Elle est stock\'ee dans le tableau \var{RTP} des
variables mises \`a jour.

\end{itemize}

\etape{Appel de \fort{reseps} pour la r\'esolution de la variable $\varepsilon$}

On d\'ecrit ci-dessous le sous-programme \fort{reseps}, les commentaires du sous-programme \fort{resrij} ne seront pas r\'ep\'et\'es puisque les deux sous-programmes ne diff\`erent que par leurs termes sources.

\begin{itemize}
\item Initialisation \`a z\'ero de \var{SMBR} et \var{ROVSDT}.

\item{Lecture et prise en compte des termes sources utilisateur pour la variable $\varepsilon$ :}

Appel de \fort{ustsri} pour charger les termes sources utilisateurs. Ils sont
stock\'es dans les tableaux suivants :\\
pour la cellule $\Omega_l$ repr\'esent\'ee par $\var{IEL}$ de centre $L$, on a :
\begin{equation}\notag
\left\{\begin{array}{lll}
&\var{ROVSDT(IEL)}&= |\Omega_l| \ \alpha_{\varepsilon}\\
&\var{SMBR(IEL)}&=|\Omega_l| \ \beta_{\varepsilon}\\
\end{array}\right.
\end{equation}
On affecte alors les valeurs ad\'equates au second membre \var{SMBR} et \`a la
diagonale \var{ROVSDT} comme suit :
\begin{equation}\notag
\left\{\begin{array}{lll}
&\var{SMBR(IEL)} &= \var{SMBR(IEL)} +\ |\Omega_l| \ \alpha_{\,\varepsilon} \
\varepsilon^n_L \\
&\var{ROVSDT(IEL)}&= \text{max }(-\ |\Omega_l| \ \alpha_{\,\varepsilon},0)\\
\end{array}\right.
\end{equation}

\item{Calcul du terme source de masse si $\Gamma_L > 0$ :
\begin{equation}\notag
\left\{\begin{array}{lll}
&\displaystyle \var{SMBR(IEL)} = \var{SMBR(IEL)} + |\Omega_l| \ \Gamma_L \
(\varepsilon^{\,in}_L -\varepsilon^n_L) \\
&\displaystyle \var{ROVSDT(IEL)}= \var{ROVSDT(IEL)} + |\Omega_l| \ \Gamma_L
\end{array}\right.
\end{equation}
\item Calcul du terme d'accumulation de masse et du terme instationnaire \\
On stocke $\displaystyle \var{W1}= \int_{\Omega_l}\dive\,(\rho\,\vect{u})\,d\Omega$
calcul\'e par \fort{divmas} \`a l'aide des flux de masse internes et aux bords.\\
On incr\'emente ensuite \var{SMBR} et \var{ROVSDT}.
\begin{equation}\notag
\left\{\begin{array}{lll}
&\var{SMBR(IEL)} &= \var{SMBR(IEL)} + \var{ICONV}\ \varepsilon^n_L\,(\displaystyle
\int_{\Omega_l}\dive\,(\rho\,\vect{u})\ d\Omega) \\
&\var{ROVSDT(IEL)}& = \var{ROVSDT(IEL)} +  \var{ISTAT}\,\displaystyle
\frac{\rho^n_L \ |\Omega_l|}{\Delta t^n} -  \var{ICONV}\ (\displaystyle
\int_{\Omega_l}\dive\,(\rho\,\vect{u})\ d\Omega) \\
\end{array}\right.
\end{equation}

\item Traitement du terme de production
 $\displaystyle \rho\,C_{\varepsilon_1}\,\frac{\varepsilon}{k}\,\mathcal{P}$
 et du terme de dissipation $-\,\displaystyle \rho\,C_{\varepsilon_2}\,\frac{\varepsilon}{k}\,\varepsilon$ \\
pour cela, on effectue une boucle d'indice \var{IEL} sur les cellules $\Omega_l$
de centre $L$ :
\begin{itemize}
\item [$\Rightarrow$] $\displaystyle \var{TRPROD}= \frac{1}{2} (\mathcal{P}^n_{ii})_L = \frac{1}{2} \left[ \var{PRODUC(1,IEL)} +  \var{PRODUC(2,IEL)} +  \var{PRODUC(3,IEL)} \right] $
\item [$\Rightarrow$] $\displaystyle \var{TRRIJ }= \frac{1}{2} (R^n_{ii})_L $
\item [$\Rightarrow$] $\displaystyle \var{SMBR(IEL)} = \var{SMBR(IEL)} + \rho^n_L
|\Omega_l| \left[ -C_{\varepsilon_2} \ \frac{2\,(\varepsilon^n_L)^2}{(R^n_{ii})_L} + C_{\varepsilon_1} \ \frac{\varepsilon^n_L}{(R^n_{ii})_L}\ (\mathcal{P}^n_{ii})_L \right] $
\item [$\Rightarrow$] $\displaystyle \var{ROVSDT(IEL)} = \var{ROVSDT(IEL)} + C_{\varepsilon_2} \ \rho^n_L \ |\Omega_l| \ \frac{2\,\varepsilon^n_L}{(R^n_{ii})_L}$
\end{itemize}

\item Appel de \fort{rijthe} pour le calcul des termes de gravit\'e $\mathcal{G}^n_{\varepsilon}$ et ajout dans \var{SMBR}.

$ \var{SMBR} = \var{SMBR} + \mathcal{G}^n_{\varepsilon}$\\
Ce calcul n'a lieu que si $\var{IGRARI()} = 1$.

\item Calcul de la diffusion de $\varepsilon$ \\
 Le terme $\dive \left[\mu\, \grad(\varepsilon) + \tens{A'}\,\grad(\varepsilon)
\right]$ est calcul\'e exactement de la m\^eme mani\`ere que pour les tensions
de Reynolds $R_{ij}$ en rempla\c cant $\tens{A}$ par $\tens{A'}$.

\item Appel de \fort{codits} pour la r\'esolution de l'\'equation de
convection/diffusion/termes sources de la variable principale $\varepsilon$. Le
r\'esultat $\varepsilon^{\,n+1}$ est stock\'e dans le tableau \var{RTP} des
variables mises \`a jour.
}
\end{itemize}

\etape{clippings finaux}
On passe enfin dans le sous-programme  \fort{clprij} pour faire un clipping \'eventuel
des variables $R^{\,n+1}_{ij}$ et $\varepsilon^{\,n+1}$. Le sous-programme
\fort{clprij} est appel\'e\footnote{L'option
$\var{ICLIP} = 1$ consiste en un clipping minimal des variables $R_{ii}$ et
$\varepsilon$ en prenant la valeur absolue de ces variables puisqu'elles ne
peuvent \^etre que positives.} avec $\var{ICLIP} = 2$ . Cette option
\footnote{Quand la valeur des grandeurs $R_{ii}$ ou $\varepsilon$ est
n\'egative, on la remplace par le minimum entre sa valeur absolue et (1,1)
fois la valeur obtenue au pas de temps pr\'ec\'edent.} contient l'option $\var{ICLIP} = 1$  et permet de v\'erifier l'in\'egalit\'e de Cauchy-Schwarz sur les grandeurs extra-diagonales du tenseur $\tens{R}$ (pour $i \neq j$, $|R_{ij}|^2 \le R_{ii} R_{jj}$).


%%%%%%%%%%%%%%%%%%%%%%%%%%%%%%%%%%
%%%%%%%%%%%%%%%%%%%%%%%%%%%%%%%%%%
\section{Points \`a traiter}
%%%%%%%%%%%%%%%%%%%%%%%%%%%%%%%%%%
%%%%%%%%%%%%%%%%%%%%%%%%%%%%%%%%%%
Sauf mention explicite, $\phi$ repr\'esentera une tension de Reynolds ou la dissipation turbulente ($\phi = R_{ij} \ \text{ou} \ \varepsilon$).

\begin{itemize}
\item {La vitesse utilis\'ee pour le calcul de la production est explicite. Est-ce qu'une implicitation peut am\'eliorer la pr\'ecision temporelle de $\phi$ \footnote{Cette remarque peut \^etre g\'en\'eralis\'ee. En effet, peut-on envisager d'actualiser les variables d\'ej\`a r\'esolues (sans r\'eactualiser les variables turbulentes apr\`es leur r\'esolution)? Ceci obligerait \`a modifier les sous-programmes tels que \fort{condli} qui sont appel\'es au d\'ebut de la boucle en temps.} ?}
\item {Dans quelle mesure le terme d'\'echo de paroi est-il valide ? En effet, ce terme est remis en question par certains auteurs.}
\item {On peut envisager la r\'esolution d'un syst\`eme hyperbolique pour les
tensions de Reynolds afin d'introduire un couplage avec le champ de vitesse.}
\item {Le flux au bord \var{VISCB} est annul\'e dans le sous-programme
\fort{vectds}. Peut-on envisager de mettre au bord la valeur de la variable
concern\'ee \`a la cellule de bord correspondant? De m\^eme, il faudrait se
pencher sur les hypoth\`eses sous-jacentes \`a l'annulation des contributions
aux bords de \var{VISCB} lors du calcul de : $$\left[ \tens{D}^n\,\left( \grad{R^{\,n}_{ij}} - (\grad R^{\,n}_{ij}\,.\,\vect{n}_{\,lm})\,\vect{n}_{\,lm}\right) \right]\,.\,\vect{n}_{\,lm}.$$}
\item {Un probl\`eme de pond\'eration appara\^\i t plus g\'en\'eralement. Si on prend la partie explicite de $\tens{D}\,\grad(\phi)$, la pond\'eration aux faces internes utilise le coefficient $\displaystyle\frac{1}{2}$ avec pond\'eration s\'epar\'ee de $\tens{D}$ et $\grad(\phi)$, alors que pour $\tens{E}\,\grad(\phi)$, on calcule d'abord ce terme aux cellules pour ensuite l'interpoler lin\'eairement aux faces \footnote{Cette interpolation se fait dans \fort{vectds} avec des coefficients de pond\'eration aux faces.}. Ceci donne donc deux types d'interpolations pour des termes de m\^eme nature.}
\item {On laisse la possibilit\'e dans \fort{visort} d'utiliser une moyenne
harmonique aux faces. Est-ce que ceci est valable puisque les interpolations
utilis\'ees lors du calcul de la partie explicite de $\tens{A}\,\grad{\phi}$
sont des moyennes arithm\'etiques ?}
\item {Les techniques adopt\'ees lors du clipping sont \`a revoir.}
\item {On utilise dans le cadre du mod\`ele $\displaystyle R_{ij}-\varepsilon $ une semi-implicitation de termes comme $\displaystyle \phi_{ij,1}$ ou $\displaystyle -\rho\,C_{\varepsilon_2}\,\frac{\varepsilon}{k}\,\varepsilon$. On peut envisager le m\^eme type d'implicitation dans \fort{turbke} m\^eme en pr\'esence du couplage $\displaystyle k-\varepsilon$.}
\item L'adoption d'une r\'esolution d\'ecoupl\'ee fait perdre l'invariance par rotation.
\item La formulation et l'implantation des conditions aux limites de paroi
devront \^etre v\'erifi\'ees. En effet, il semblerait que, dans certains cas, des ph\'enom\`enes
``oscillatoires'' apparaissent, sans qu'il soit ais\'e d'en d\'eterminer la cause.
\item L'implicitation partielle (du fait de la r\'esolution d\'ecoupl\'ee) des
conditions aux limites conduit souvent \`a des calculs instables. Il
conviendrait d'en conna\^\i tre la raison. L'implicitation partielle avait
\'et\'e mise en \oe uvre afin de tenter d'utiliser un pas de temps plus grand
dans le cas de jets axisym\'etriques en particulier.

\end{itemize}

%                      Code_Saturne version 1.3
%                      ------------------------
%
%     This file is part of the Code_Saturne Kernel, element of the
%     Code_Saturne CFD tool.
%
%     Copyright (C) 1998-2007 EDF S.A., France
%
%     contact: saturne-support@edf.fr
%
%     The Code_Saturne Kernel is free software; you can redistribute it
%     and/or modify it under the terms of the GNU General Public License
%     as published by the Free Software Foundation; either version 2 of
%     the License, or (at your option) any later version.
%
%     The Code_Saturne Kernel is distributed in the hope that it will be
%     useful, but WITHOUT ANY WARRANTY; without even the implied warranty
%     of MERCHANTABILITY or FITNESS FOR A PARTICULAR PURPOSE.  See the
%     GNU General Public License for more details.
%
%     You should have received a copy of the GNU General Public License
%     along with the Code_Saturne Kernel; if not, write to the
%     Free Software Foundation, Inc.,
%     51 Franklin St, Fifth Floor,
%     Boston, MA  02110-1301  USA
%
%-----------------------------------------------------------------------
%
\programme{vortex}
%
\vspace{1cm}
%%%%%%%%%%%%%%%%%%%%%%%%%%%%%%%%%%
%%%%%%%%%%%%%%%%%%%%%%%%%%%%%%%%%%
\section{Fonction}
%%%%%%%%%%%%%%%%%%%%%%%%%%%%%%%%%%
%%%%%%%%%%%%%%%%%%%%%%%%%%%%%%%%%%
Ce sous-programme est d�di� � la g�n�ration des conditions d'entr�e
turbulente utilis�es en LES.


La m�thode des vortex est bas�e sur une approche de tourbillons
ponctuels. L'id�e de la m�thode consiste � injecter des tourbillons 2D dans le
plan d'entr�e du calcul, puis � calculer le champ de vitesse induit par ces
tourbillons au centre des faces d'entr�e.

%                      Code_Saturne version 1.3
%                      ------------------------
%
%     This file is part of the Code_Saturne Kernel, element of the
%     Code_Saturne CFD tool.
% 
%     Copyright (C) 1998-2007 EDF S.A., France
%
%     contact: saturne-support@edf.fr
% 
%     The Code_Saturne Kernel is free software; you can redistribute it
%     and/or modify it under the terms of the GNU General Public License
%     as published by the Free Software Foundation; either version 2 of
%     the License, or (at your option) any later version.
% 
%     The Code_Saturne Kernel is distributed in the hope that it will be
%     useful, but WITHOUT ANY WARRANTY; without even the implied warranty
%     of MERCHANTABILITY or FITNESS FOR A PARTICULAR PURPOSE.  See the
%     GNU General Public License for more details.
% 
%     You should have received a copy of the GNU General Public License
%     along with the Code_Saturne Kernel; if not, write to the
%     Free Software Foundation, Inc.,
%     51 Franklin St, Fifth Floor,
%     Boston, MA  02110-1301  USA
%
%-----------------------------------------------------------------------
%
%%%%%%%%%%%%%%%%%%%%%%%%%%%%%%%%%%
%%%%%%%%%%%%%%%%%%%%%%%%%%%%%%%%%%
\section{Discr\'etisation}
%%%%%%%%%%%%%%%%%%%%%%%%%%%%%%%%%%
%%%%%%%%%%%%%%%%%%%%%%%%%%%%%%%%%%

Le terme convectif en $\dive(\underline{u} \otimes \rho\,\underline{u})$
introduit une non lin\'earit\'e et un couplage des composantes de la vitesse
$\vect{u}$ dans l'�quation (\ref{Base_Preduv_eqqdm}). Une lin\'earisation et un d\'ecouplage
des trois composantes de la 
vitesse sont r\'ealis\'es lors de la discr\'etisation de cette \'etape de
pr\'ediction.\\
En effet, soit :
\begin{equation}
\vect{\widetilde{u}}= \vect{u}^n + \delta \vect{u} 
\end{equation}
La contribution exacte du terme convectif \`a prendre en compte dans cette
\'etape de pr\'ediction serait :\\
\begin{equation}\label{Base_Preduv_Conv_exact}
\begin{array}{ll}
\dive(\vect{\widetilde{u}} \otimes \rho\,\vect{\widetilde{u}}) =
\dive(\vect{u}^{n} \otimes \rho\,\vect{u}^{n}) + \dive(\delta \vect{u} \otimes
\rho\,\vect{u}^{n}) +  \underbrace { \dive(\vect{u}^{n} \otimes
\rho\,\delta \vect{u})}_{\text {terme couplant lin\'eaire}} +  \underbrace { \dive(\delta \vect{u} \otimes
\rho\,\delta \vect{u})}_{\text {terme couplant et non lin\'eaire}}\\
\end{array} 
\end{equation}
Les deux derniers termes de l'expression (\ref{Base_Preduv_Conv_exact}) sont {\em a priori} n�glig�s
de mani�re � obtenir un syst\`eme en vitesse qui soit d\'ecoupl\'e et donc,
�viter l'inversion d'une matrice pouvant \^etre de tr\`es grande taille. Ces
deux termes peuvent n�anmoins �tre pris en compte de mani�re plus ou moins
approch�e par extrapolation explicite du flux de masse en $n+\theta_F$ (pour le
terme couplant lin�aire provenant de la convection de $\vect{u}^{n}$ par $\delta
\vect{u}$) et utilisation d'un point-fixe par sous it�ration sur le sous
programme \fort{navsto} (pour le terme non-lin�aire, en sp�cifiant $\var{NTERUP}>1$).

L'�quation (\ref{Base_Preduv_eqqdm}) est discr�tis�e au temps $n+\theta$ � l'aide d'un
$\theta$-sch�ma, et le tenseur des pertes de charges d�compos� en une partie
diagonale $\tens{K}_{d}$ et une extradiagonale $\tens{K}_{e}$ (soit
 $\tens{K}_{pdc}=\tens{K}_{d}+\tens{K}_{e}$).\\
$\bullet$ La pression est suppos�e connue en $n-1+\theta$ (d�calage temporel
pression-vitesse) et le gradient naturellement calcul� � cet instant.\\ 
$\bullet$ Les termes sources de viscosit� secondaire, de gradient transpos\'e,
ceux provenant du mod�le de turbulence\footnote{except� $\dive (\mu_t\ (\ggrad
\underline {u}))$}, $\rho\,\tens{K}_{\,e}\ \underline{u}$, $(\rho -\rho_0)
\underline {g}$ ainsi que $\underline{T}_{s}^{\,exp}$ et
$\Gamma\,\underline{u}_{\,i}$ sont pris de mani�re explicite au temps $n$, ou
extrapol�s suivant le sch�ma en temps choisi pour les propri�t�s physique et les
termes sources.\\ 
$\bullet$ Les termes sources $\underline{u}\,\,\dive (\rho\,\underline {u})$,
$\Gamma\,\,\underline{u}$, $T_{s}^{\,imp}\,\,\underline{u}$ et
$-\rho\,\tens{K}_{\,d}\,\,\underline{u}$ sont implicit�s est calcul�s �
l'instant $n+\theta$.\\ 
$\bullet$ Le terme de diffusion $\dive (\mu_{\,tot}\,\ggrad \underline{u})$ est
implicit� : la vitesse est prise � l'instant $n+\theta$ et la viscosit�
explicit�e ou extrapol�e.\\ 
$\bullet$ Enfin, le terme de convection en $\dive(\,\underline{u} \otimes
(\rho\underline{u})\,)$ est implicit� : la composante r�solue de la vitesse est
prise en $n+\theta$, et le flux de masse, explicit�, ou extrapol� en
$n+\theta_F$. 

Par souci de clart�, on suppose, en l'absence d'indication, que les propri�tes
physiques $\Phi$ ($\rho,\,\mu_{tot},\,...$) et le flux de masse
$(\rho\underline{u})$ sont pris respectivement aux instants $n+\theta_\Phi$ et
$n+\theta_F$, o� $\theta_\Phi$ et $\theta_F$ d�pendent des sch�mas en temps
sp�cifiquement utilis�s pour ces grandeurs\footnote{cf. \fort{introd}}. 

La discr�tisation temporelle de l'�quation (\ref{Base_Preduv_eqqdm}) s'�crit alors comme suit : 

\begin{equation}\label{Base_Preduv_eq_di1}
 \begin{array}{c}
\displaystyle \rho\,\ \frac{ \underline {\widetilde{u}}^{n+1} -\underline {u}^{n} }
{\Delta t} + \dive(\,\underline{\widetilde{u}}^{n+\theta} \otimes (\rho\underline{u})\,) -\dive
(\mu_{\,tot}\,\ggrad \underline{\widetilde{u}}^{n+\theta}) =
\\
\displaystyle
 - \grad p^{n-1+\theta} + \dive (\rho\,\underline {u})\,\underline{\widetilde{u}}^{n+\theta} +(\Gamma\,\underline{u}_{\,i})^{n+\theta_S}-\Gamma^n\,\,\underline{\widetilde{u}}^{n+\theta}
\\
\begin{array}{c}
\displaystyle
- \rho\,\tens{K}_{\,d}^{n}\,\,\underline{\widetilde{u}}^{n+\theta} - (\rho\,\tens{K}_{\,e}\ \underline{u})^{n+\theta_S} + (\underline{T}_{s}^{\,exp})^{\,n+\theta_S} + T_{s}^{\,imp}\,\,\underline{\widetilde{u}}^{n+\theta}
\\
\displaystyle
+[\dive (\mu_{\,tot}\,^t\ggrad \underline {u})]^{n+\theta_S}-\frac {2} {3}[\,\grad (\mu_{\,tot}\,\dive \underline {u})]^{n+\theta_S} + (\rho -\rho_0) \underline {g}
 - (\underline{turb})^{n+\theta_S}
\end{array}
\end{array}
\end{equation}
o\`u, par souci de simplification, on a pos\'e :
\begin{equation}
\mu_{\,tot}=
\begin{cases}
\mu+\mu_t & \text{pour les mod�les � viscosit� turbulente ou en LES}, \\
\mu & \text{pour les mod�les au second ordre ou en laminaire}
\end{cases} \ 
\end{equation}
\\
et :
\begin{equation}
\underline{turb}^{n}=
\begin{cases}
\displaystyle\frac {2}{3}\grad (\rho^{n}\,k^{n}) & \text{pour les mod�les � viscosit� turbulente}, \\
\dive(\rho^{n}\,\tens{R}^n) & \text{pour les mod�les au second ordre},\\
0 & \text{en laminaire ou en LES}\\
\end{cases}
\end{equation}
Par analogie avec l'�criture du $\theta$-sch�ma pour une variable scalaire, $\,
\underline {\widetilde{u}}^{n+\theta}$ est interpol�e � partir de la vitesse
pr�dite $\underline {\widetilde{u}}^{n+1}$ de la mani\`ere suivante\footnote{si
$\theta=1/2$, ou qu'une extrapolation est utilis�e, l'ordre 2 n'est obtenu que si
le pas de temps $\Delta t$ est uniforme en temps et en espace.}~: 
\begin{equation}
\underline {\widetilde{u}}^{n+\theta}=\theta\, \underline
{\widetilde{u}}^{n+1}+(1-\theta)\, \underline {u}^{n}\\ 
\end{equation}
Avec :
\begin{equation}
\left\{
\begin{array}{ll}
\theta = 1   & \text{Pour un sch\'ema de type Euler implicite d'ordre 1.}\\
\theta = 1/2 & \text{Pour un sch\'ema de type Cranck-Nicolson d'ordre 2.}\\
\end{array}
\right.
\end{equation}

L'�quation (\ref{Base_Preduv_eq_di1}) est alors r��crite sous la forme :

\begin{equation}\label{Base_Preduv_eq_di2}
\begin{array}{c}
\displaystyle \underbrace{\left(\frac{\rho}{\Delta t} -\theta \,\dive (\rho\,\underline {u}) +\theta \,\, \Gamma^n +
\theta \,\, \rho\,\tens{K}_{\,d}^n-\theta \,T_s^{\,imp} \right)}_{\displaystyle f_s^{imp}}\, (\underline {\,\widetilde{u}}^{n+1} -\underline {u}^{n})
\\
 +\, \theta\, \dive(\underline {\widetilde{u}}^{n+1} \otimes (\rho\underline{u}))-\, \theta\,\dive (\mu_{\,tot}\,\ggrad \underline {\widetilde{u}}^{n+1}) =
\\
-\,(1-\theta)\, \dive(\underline {u}^{n} \otimes (\rho\underline{u})) +\,(1-\theta)\,\dive (\mu_{\,tot}\,\ggrad \underline {u}^{n})
\\
f_s^{exp}\left\{
\begin{array}{c}
\displaystyle 
- \grad p^{n-1+\theta} + \dive (\rho\,\underline {u})\,\underline{u}^{n} +\,(\,\Gamma^{n}\,\underline{u}_{\,i}\,)^{n+\theta_S}- \Gamma^n\,\,\underline{u}^{n}
\\
\displaystyle
-(\,\rho\,\tens{K}_{\,e}\ \underline{u}\,)^{n+\theta_S} -\rho\,\tens{K}_{\,d}^n\ \underline{u}^{n}+ (\underline{T}_{s}^{\,exp})^{\,n+\theta_S} + T_s^{\,imp}\,\,\underline {u}^{n} 
\\
\displaystyle
+[\dive (\mu_{\,tot}\,^t\ggrad \underline {u}\,)]^{n+\theta_S}-\frac {2} {3}[\,\grad (\mu_{\,tot}\,\dive \underline {u}\,)]^{n+\theta_S} + (\rho -\rho_0) \underline {g}-(\underline{turb})^{n+\theta_S}
\end{array}
\right.
\end{array}
\end{equation}

d'o� l'�quation r�solue par le sous-programme \fort{codits} :
\begin{equation}\begin{array}{c}
\displaystyle
f_s^{\,imp}(\underline {\widetilde{u}}^{n+1}-\underline {u}^{n}) + \theta\, \dive(\underline{\widetilde{u}}^{n+1} \otimes (\rho
\underline{u})) - \theta\,\dive (\,\mu_{\,tot}\,\ggrad \underline{\widetilde{u}}^{n+1}) = 
\\\\
\displaystyle
-(1-\theta)\,\dive(\underline{u}^{n} \otimes (\rho \underline{u}))+(1-\theta)\,\dive (\,\mu_{\,tot}\,\ggrad \underline{u}^{n})
+ \underline{f}_{\,s}^{\,exp}
\end{array}
\end{equation}
La m\'ethode de discr\'etisation spatiale est d\'evelopp\'ee dans le sous-programme \fort{codits}.\\



\minititre{Remarques :}
{\tiny$\blacksquare$} Dans le cas standard sans extrapolation, le terme
$-\,T_s^{\,imp}$ n'est ajout� � $f_s^{\,imp}$ que s'il est positif afin de ne
pas affaiblir la dominance de la diagonale de la matrice � inverser.\\ 
{\tiny$\blacksquare$} Si une extrapolation est utilis�e, par contre,
$\,T_s^{\,imp}$ est ajout� � $f_s^{\,imp}$ quel que soit son signe. En effet, l'id�e intuitive qui
consiste � prendre~: 
\begin{equation}
\begin{cases}
\displaystyle
(\underline{T}_{s}^{\,exp} + T_{s}^{\,imp}\,\underline {u})^{\,n+\theta_S} &
\text{si } T_{s}^{\,imp} > 0\\ 
\displaystyle
(\underline{T}_{s}^{\,exp})^{\,n+\theta_S} + T_{s}^{\,imp}\,\underline{u}^{n+\theta} &\text{sinon}\\
\end{cases}
\end{equation} 
aboutit � une incoh�rence dans le traitement si $T_s^{imp}$ change de signe
entre deux pas de temps.\\ 
{\tiny$\blacksquare$} la partie diagonale $\tens{K}_{\,d}$ du terme
de perte de charge est utilis�e dans $f_s^{\,imp}$. En fait, pour \^etre rigoureux,
il faudrait ne retenir que les contributions positives (point signal\'e dans le
sous-programme utilisateur associ\'e \fort{uskpdc}). Cette prise en compte sera \`a am\'eliorer.\\
{\tiny$\blacksquare$} Le terme $\theta\,\Gamma^{n}-\theta\,\dive
(\rho\,\underline {u})$ ne pose pas de probl�me pour la 
dominance de la diagonale de la matrice car il est exactement compens� par le
terme de convection (cf. \fort{covofi}). 


%                      Code_Saturne version 1.3
%                      ------------------------
%
%     This file is part of the Code_Saturne Kernel, element of the
%     Code_Saturne CFD tool.
%
%     Copyright (C) 1998-2007 EDF S.A., France
%
%     contact: saturne-support@edf.fr
%
%     The Code_Saturne Kernel is free software; you can redistribute it
%     and/or modify it under the terms of the GNU General Public License
%     as published by the Free Software Foundation; either version 2 of
%     the License, or (at your option) any later version.
%
%     The Code_Saturne Kernel is distributed in the hope that it will be
%     useful, but WITHOUT ANY WARRANTY; without even the implied warranty
%     of MERCHANTABILITY or FITNESS FOR A PARTICULAR PURPOSE.  See the
%     GNU General Public License for more details.
%
%     You should have received a copy of the GNU General Public License
%     along with the Code_Saturne Kernel; if not, write to the
%     Free Software Foundation, Inc.,
%     51 Franklin St, Fifth Floor,
%     Boston, MA  02110-1301  USA
%
%-----------------------------------------------------------------------
%

%%%%%%%%%%%%%%%%%%%%%%%%%%%%%%%%%%
%%%%%%%%%%%%%%%%%%%%%%%%%%%%%%%%%%
\section{Mise en \oe uvre}
%%%%%%%%%%%%%%%%%%%%%%%%%%%%%%%%%%
%%%%%%%%%%%%%%%%%%%%%%%%%%%%%%%%%%
La num\'ero de la phase trait\'ee fait partie des arguments de \fort{turrij}. On
omettra volontairement de le pr\'eciser dans ce qui suit, on indiquera par $(\ )$ la
notion de tableau s'y rattachant.

\etape{Calcul des termes de production $\tens{\mathcal{P}}$}
\begin{itemize}
\item [$\star$] Initialisation \`a z\'ero du tableau \var{PRODUC} dimensionn\'e \`a $\var{NCEL}\times 6$.
\item [$\star$] On appelle trois fois \fort{grdcel} pour calculer les gradients des composantes de la vitesse $u$, $v$ et
$w$ prises au temps $n$.

Au final, on a :\\
$\displaystyle
\begin{array} {ll}
\var{PRODUC(1,IEL)} = & \displaystyle - 2 \left[ R_{11}^{\,n} \frac{\partial u^{\,n}} {\partial x} +R_{12}^{\,n} \frac{\partial u^{\,n}} {\partial y}+R_{13}^{\,n} \frac{\partial u^{\,n}} {\partial z} \right] \text{        (production de $R_{11}^{\,n}$)}\\
\var{PRODUC(2,IEL)} = & \displaystyle - 2 \left[ R_{12}^{\,n} \frac{\partial v^{\,n}} {\partial x} +R_{22}^{\,n} \frac{\partial v^{\,n}} {\partial y}+R_{23}^{\,n} \frac{\partial v^{\,n}} {\partial z} \right] \text{        (production de $R_{22}^{\,n}$)}\\
\var{PRODUC(3,IEL)} = & \displaystyle - 2 \left[ R_{13}^{\,n} \frac{\partial w^{\,n}} {\partial x} +R_{23}^{\,n} \frac{\partial w^{\,n}} {\partial y}+R_{33}^{\,n} \frac{\partial w^{\,n}} {\partial z} \right] \text{        (production de $R_{33}^{\,n}$)}\\
\var{PRODUC(4,IEL)} = & \displaystyle - \left[ R_{12}^{\,n} \frac{\partial u^{\,n}} {\partial x} +R_{22}^{\,n} \frac{\partial u^{\,n}} {\partial y}+R_{23}^{\,n} \frac{\partial u^{\,n}} {\partial z} \right] \\
& \displaystyle - \left[ R_{11}^{\,n} \frac{\partial v^{\,n}} {\partial x} +R_{12}^{\,n} \frac{\partial v^{\,n}} {\partial y}+R_{13}^{\,n} \frac{\partial v^{\,n}} {\partial z} \right] \text{        (production de $R_{12}^{\,n}$)} \\
\var{PRODUC(5,IEL)} = & \displaystyle - \left[ R_{13}^{\,n} \frac{\partial u^{\,n}} {\partial x} +R_{23}^{\,n} \frac{\partial u^{\,n}} {\partial y}+R_{33}^{\,n} \frac{\partial u^{\,n}} {\partial z} \right] \\
& \displaystyle - \left[ R_{11}^{\,n} \frac{\partial w^{\,n}} {\partial x} +R_{12}^{\,n} \frac{\partial w^{\,n}} {\partial y}+R_{13}^{\,n} \frac{\partial w^{\,n}} {\partial z} \right] \text{        (production de $R_{13}^{\,n}$)} \\
\var{PRODUC(6,IEL)} = & \displaystyle - \left[ R_{13}^{\,n} \frac{\partial v^{\,n}} {\partial x} +R_{23}^{\,n} \frac{\partial v^{\,n}} {\partial y}+R_{33}^{\,n} \frac{\partial v^{\,n}} {\partial z} \right] \\
& \displaystyle - \left[ R_{12}^{\,n} \frac{\partial w^{\,n}} {\partial x} +R_{22}^{\,n} \frac{\partial w^{\,n}} {\partial y}+R_{23}^{\,n} \frac{\partial w^{\,n}} {\partial z} \right]  \text{        (production de $R_{23}^{\,n}$)}
\end{array}
$
\end{itemize}

\etape{Calcul du gradient de la masse volumique $\rho^n$ prise au d\'ebut du pas
de temps courant\footnote{{\it i.e.} calcul\'ee \`a partir des
variables du pas de temps pr\'ec\'edent $n$ si n\'ecessaire.} $(n+1)$}
Ce calcul n'a lieu que si les termes de gravit\'e doivent \^etre pris en compte
($\var{IGRARI()} =1$).
\begin{itemize}
\item [$\star$] Appel de \fort{grdcel}  pour calculer le gradient de $\rho^n$
dans les trois directions de l'espace. Les conditions aux limites sur $\rho^n$
sont des conditions de Dirichlet puisque la valeur de $\rho^n$ aux faces de bord
$ik$ (variable \var{IFAC}) est connue et vaut $\rho_{\,b_{\,ik}}$. Pour \'ecrire les conditions aux limites
sous la forme habituelle, $$\rho_{\,b_{\,ik}} = \var{COEFA} + \var{COEFB}
\,\rho^n_{\,I'}$$ on pose alors $\var{COEFA}=
\var{PROPCE(IFAC,IPPROB(IROM(IPHAS)))}$ et $\var{COEFB} = \var{VISCB} = 0$.\\
$\var{PROPCE(1,IPPROB(IROM(IPHAS)))}$ (resp.$\var{VISCB}$) est utilis\'e en lieu
et place de l'habituel \var{COEFA} ($\var{COEFB}$), lors de l'appel \`a \fort{grdcel}.\\
On a donc :\\
$\displaystyle \var{GRAROX}= \frac{\partial \rho^n}{\partial x}\ $,$\displaystyle \ \var{GRAROY}= \frac{\partial
\rho^n}{\partial y}$ et $
\displaystyle \ \var{GRAROZ}= \frac{\partial \rho^n}{\partial z}\ $.

\end{itemize}

Le gradient de $\rho^n$ servira \`a calculer les termes de production par effets de gravit\'e si ces derniers sont pris en compte.

\etape{Boucle \var{ISOU} de $1$ \`a $6$ sur les tensions de Reynolds}
Pour $\var{ISOU} = 1,2,3,4,5,6$, on r\'esout respectivement et dans
l'ordre  les
\'equations de $R_{11}$, $R_{22}$, $R_{33}$, $R_{12}$, $R_{13}$ et $R_{23}$ par
l'appel au sous-programme \fort{resrij}.\\
La r\'esolution se fait par incr\'ement $\delta {R}_{ij}^{\,n+1,k+1}$ , en utilisant la m\^eme m\'ethode que
celle d\'ecrite dans le sous-programme \fort{codits}. On adopte ici les m\^emes notations.
\var{SMBR} est le second membre du syst\`eme \`a inverser, syst\`eme portant sur
les incr\'ements de la variable. \var{ROVSDT} repr\'esente la diagonale de la
matrice, hors convection/diffusion.\\
On va r\'esoudre l'\'equation (\ref{Base_Turrij_Eq_Temp_Rij}) sous forme incr\'ementale en
utilisant \fort{codits}, soit :
\begin{equation}\label{Base_Turrij_Eq_Temp_deltaRij}
\begin{array}{ll}
&\displaystyle \underbrace{\left(\frac {\rho^n_L}{\Delta t^n}
+ \rho^n_L \,C_1\,\frac{\varepsilon^n_L}{k^n_L}(1-\frac{\delta_{ij}}{3})
 - m^n_{\,lm} + \Gamma_L\,+ max(-\alpha^n_{R_{ij}},0)\right)\,|\Omega_l|}
_{\text {$\var{ROVSDT}$ contribuant
\`a la diagonale de la matrice simplifi\'ee de \fort{matrix}}}\,(\delta{R}_{ij}^{\,n+1,p+1})_{\,L}\\\\
&  \underbrace{+\sum\limits_{m\in Vois(l)}\displaystyle \left[
 m^n_{\,lm} \delta R_{ij,\,f_{\,lm}}^{\,n+1,p+1}
- (\mu^n_{\,lm} + \gamma^n_{\,lm})\
\frac{({\delta R}_{ij}^{\,n+1,p+1})_{M}-({\delta R}_{ij}^{\,n+1,p+1})_{L})}{\overline{L'M'}}\,
S_{\,lm} \right]}_{\text { convection upwind pur et diffusion non reconstruite
relatives \`a la matrice simplifi\'ee de \fort{matrix}\footnotemark}} \\
% voir le texte de la footmark plus bas
&= - \displaystyle\frac {\rho^n_L}{\Delta t^n}\,\left(\,(R^{\,n+1,p}_{ij})_L - (R^{\,n}_{ij})_L\,\right)\\
&-\,\underbrace{\displaystyle\int_{\Omega_l} \left(
\dive\,[\,(\rho\,\vect{u})^n\,R^{\,n+1,p}_{ij} - (\mu^n\,+ \gamma^n\,)\,
\grad{R^{\,n+1,p}_{ij}}\,]\right)\,d\Omega}_{\text {convection et diffusion
trait\'ees par \fort{bilsc2}}}\\
&+\displaystyle \int_{\Omega_l} \left[\,\mathcal{P}^{\,n+1,p}_{ij} + \mathcal{G}^{\,n+1,p}_{ij}
- \displaystyle\rho^n \,C_1\,\frac{\varepsilon^n}{k^n}\left[R^{\,n+1,p}_{ij}-
\frac{2}{3}\,k^n\,\delta_{ij}\right] + \phi^{\,n+1,p}_{ij,2} +
\phi^{\,n+1,p}_{ij,w}\,\right]\, d\Omega \\
& + \displaystyle\int_{\Omega_l} \left[- \frac{2}{3} \rho^n \varepsilon^n \delta_{ij}
 + \Gamma\,(\,R^{\,in}_{ij} - R^{\,n+1,p}_{ij}\,) +
\alpha^n_{R_{ij}}\,R^{\,n+1,p}_{ij}+ \beta^n_{R_{ij}}\right]\, d\Omega\\
&+ \sum\limits_{m\in
Vois(l)}\displaystyle \left[\ \tens{E}^n\,\grad{R}^{\,n+1,p}_{ij} \right]_{\,lm}\,.\,\vect{n}_{\,lm}S_{\,lm}\\
&+ \sum\limits_{m\in Vois(l)}\displaystyle \left[\
\tens{D}^n\,\grad{R}^{\,n+1,p}_{ij} \right]_{\,lm}\,.\,\vect{n}_{\,lm}S_{\,lm}\\
&- \sum\limits_{m\in Vois(l)} \gamma^n_{\,lm} \left( \grad{R}^{\,n+1,p}_{ij}\,.\,\vect{n}_{\,lm} \right)  S_{\,lm}\\
&+ \sum\limits_{m\in Vois(l)}  m^n_{\,lm}\,(R^{\,n+1,p}_{ij})_L\\
\end{array}
\end{equation}
% si on ne fait pas comme ca, il n'apparait pas
\footnotetext[\thefootnote]{Si $\var{IRIJNU} = 1$, on remplace  $\mu^n_{\,lm}$ par $(\mu +
\mu_t)^n_{\,lm}$ dans l'expression de la diffusion non reconstruite
associ\'ee \`a la matrice simplifi\'ee de \fort{matrix} ($\mu_t$ d\'esigne la
viscosit\'e turbulente calcul\'ee comme en $k-\varepsilon$).}

o\`u on rappelle :\\
pour $n$ donn\'e entier positif, on d\'efinit la suite
 $({R}_{ij}^{\,n+1,p})_{p \in \grandN}$
 par :
\begin{equation}\notag
\left\{\begin{array}{l}
{R}_{ij}^{\,n+1,0} = {R}_{ij}^{\,n}\\
{R}_{ij}^{\,n+1,p+1} = {R}_{ij}^{\,n+1,p} + \delta{R}_{ij}^{\,n+1,p+1} \\
\end{array}\right.
\end{equation}
$(\delta{R}_{ij}^{\,n+1,p+1})_{\,L}$ d\'esigne la valeur sur l'\'el\'ement
$\Omega_l$ du $\text{$(\,p+1\,)$-i\`eme}$ incr\'ement de ${R}_{ij}^{\,n+1}$,
$ m^n_{\,lm}$ le flux de masse \`a l'instant $n$ \`a travers la face $lm$,
$\delta R_{ij,\,f_{\,lm}}^{\,n+1,p+1}$ vaut $({\delta
R}_{ij}^{\,n+1,p+1})_{L}$  si $ m^n_{\,lm} \geqslant 0$, $({\delta
R}_{ij}^{\,n+1,p+1})_{M}$ sinon,
$\mathcal{P}^{\,n+1,p}_{ij}$, $\phi^{\,n+1,p}_{ij,2}$, $\phi^{\,n+1,p}_{ij,w}$ les valeurs
des quantit\'es associ\'ees correspondant \`a l'incr\'ement
$(\delta{R}_{ij}^{\,n+1,p})$.\\



Tous ces termes sont calcul\'es comme suit :
\begin{itemize}
\item Terme de gauche de l'\'equation (\ref{Base_Turrij_Eq_Temp_deltaRij})\\
Dans \fort{resrij} est calcul\'ee la variable \var{ROVSDT}. Les autres
termes sont compl\'et\'es par \fort{codits}, lors de la construction de la matrice simplifi\'ee , {\it via} un
appel au sous-programme \fort{matrix}. La quantit\'e
 $(\mu^n_{\,lm} + \gamma^n_{\,lm})$ \`a la face $lm$ est calcul\'ee lors de l'appel \`a
\fort{visort}.\\
\item Second membre de l'\'equation (\ref{Base_Turrij_Eq_Temp_deltaRij})\\
Le premier terme non d\'etaill\'e est calcul\'e par le sous-programme
\fort{bilsc2}, qui applique le sch\'ema convectif choisi par l'utilisateur, qui
reconstruit ou non selon le souhait de l'utilisateur les gradients intervenants
dans la convection-diffusion.\\
Les termes sans accolade sont, eux, compl\`etement explicites et ajout\'es au fur et
\`a mesure dans \var{SMBR} pour former
l'expression $f^{\,exp}_s$ de \fort{codits}.
\end{itemize}
On d\'ecrit ci-dessous les \'etapes de \fort{resrij} :
\begin{itemize}

\item DELTIJ = 1, pour $\var{ISOU} \leqslant 3$ et DELTIJ = 0  Si $\var{ISOU} >
3$. Cette valeur repr\'esente le symbole de Kroeneker $\delta_{ij}$.

\item Initialisation \`a z\'ero de \var{SMBR} (tableau contenant le second
membre) et \var{ROVSDT} (tableau contenant la diagonale de la matrice sauf celle
relative \`a la contribution de la
diagonale des op\'erateurs de convection et de diffusion lin\'earis\'es
\footnote{qui correspondent aux sch\'emas convectif upwind pur et diffusif sans
reconstruction.}), tous deux de dimension $\var{NCEL}$.

\item Lecture et prise en compte des termes sources utilisateur pour la variable $R_{ij}$

Appel \`a \fort{ustsri} pour charger les termes sources utilisateurs. Ils sont
stock\'es comme suit. Pour la cellule $\Omega_l$ de centre $L$, repr\'esent\'ee par $\var{IEL}$, on a :\\
\begin{equation}\notag
\left\{\begin{array}{lll}
&\var{ROVSDT(IEL)}&= |\Omega_l| \ \alpha_{R_{ij}}\\
&\var{SMBR(IEL)}&=|\Omega_l| \ \beta_{R_{ij}}\\
\end{array}\right.
\end{equation}
On affecte alors les valeurs ad\'equates au second membre \var{SMBR} et \`a la
diagonale \var{ROVSDT} comme suit :
\begin{equation}\notag
\left\{\begin{array}{lll}
&\var{SMBR(IEL)} &= \var{SMBR(IEL)} +\ |\Omega_l| \ \alpha_{R_{ij}} \ (R^n_{ij})_L \\
&\var{ROVSDT(IEL)}&= \text{max }(-\ |\Omega_l| \ \alpha_{R_{ij}},0)\\
\end{array}\right.
\end{equation}
La valeur de $ \var{ROVSDT}$ est ainsi calcul\'ee pour des raisons de stabilit\'e
num\'erique. En effet, on ne rajoute sur la diagonale que les valeurs positives,
ce qui correspond physiquement \`a impliciter les termes de rappel uniquement.
\item{Calcul du terme source de masse  si $\Gamma_L > 0$}

Appel de \fort{catsma} et incr\'ementation si n\'ecessaire de \var{SMBR} et
\var{ROVSDT} {\it via} :\\
\begin{equation}\notag
\left\{\begin{array}{lll}
\displaystyle \var{SMBR(IEL)} = \var{SMBR(IEL)} + |\Omega_l| \ \Gamma_L \
\left[(R^{\,in}_{ij})_L - (R^{\,n}_{ij})_L \right] \\
\displaystyle \var{ROVSDT(IEL)}=\var{ROVSDT(IEL)} + |\Omega_l| \ \Gamma_L
\end{array}\right.
\end{equation}
\item Calcul du terme d'accumulation de masse et du terme instationnaire

On stocke $\displaystyle \var{W1}= \int_{\Omega_l}\dive\,(\rho\,\vect{u})\,d\Omega$
calcul\'e par \fort{divmas} \`a l'aide des flux de masse aux faces internes
$ m^n_{\,lm}=\sum\limits_{m\in Vois(l)}{(\rho \vect{u})_{\,lm}^n} \text{.}\,
\vect{S}_{\,lm} $ (tableau \var{FLUMAS}) et des flux de masse aux bords  $ m^n_{\,b_{lk}} = \sum\limits_{k\in{\gamma_b(l)}}{(\rho \vect{u})_{\,{b}_{lk}}^n} \text{.}\,
\vect{S}_{\,{b}_{lk}} $ (tableau \var{FLUMAB}).
On incr\'emente ensuite \var{SMBR} et \var{ROVSDT}.
\begin{equation}\notag
\left\{\begin{array}{lll}
&\var{SMBR(IEL)} &= \var{SMBR(IEL)} + \var{ICONV}\  (R^n_{ij})_L\,(\displaystyle
\int_{\Omega_l}\dive\,(\rho\,\vect{u})\ d\Omega) \\
&\var{ROVSDT(IEL)}& = \var{ROVSDT(IEL)} +  \var{ISTAT}\,\displaystyle
\frac{\rho^n_L \ |\Omega_l|}{\Delta t^n} -  \var{ICONV}\ (\displaystyle
\int_{\Omega_l}\dive\,(\rho\,\vect{u})\ d\Omega) \\
\end{array}\right.
\end{equation}
\item Calcul des termes sources de production, des termes $\displaystyle
\phi_{\,ij,1}+\phi_{\,ij,2}$ et de la dissipation~$\displaystyle-\frac{2}{3} \varepsilon\,\delta_{\,ij}$ :

On effectue une boucle d'indice \var{IEL} sur les cellules $\Omega_l$ de centre $L$ :
\begin{itemize}
\item [$\Rightarrow$] $\displaystyle \var{TRPROD}= \frac{1}{2} (\mathcal{P}^n_{ii})_L = \frac{1}{2} \left[ \var{PRODUC(1,IEL)} +  \var{PRODUC(2,IEL)} +  \var{PRODUC(3,IEL)} \right] $
\item [$\Rightarrow$] $\displaystyle \var{TRRIJ }= \frac{1}{2} (R^n_{ii})_L $
\item [$\Rightarrow$] $\displaystyle \var{SMBR(IEL)} =\ \var{SMBR(IEL)}\ +$\\
$\ \displaystyle\rho^n_L |\Omega_l| \left[ \displaystyle
\frac{2}{3}\,\delta_{\,ij} \left( \ \displaystyle \frac{ C_2}{2}\,(\mathcal{P}^n_{ii})_L\ +
(C_1-1)\ \varepsilon^n_L\, \right)\right.$\\
$ + \left.\ (1-C_2) \ \var{PRODUC(ISOU,IEL)} -
\displaystyle C_1\ \frac{2\,\varepsilon^n_L}{(R^n_{ii})_L}\ (R^n_{ij})_L \right]$
\item [$\Rightarrow$] $\displaystyle \var{ROVSDT(IEL)} = \var{ROVSDT(IEL)} +
\rho^n_L \ |\Omega_l| \ (- \displaystyle \frac{1}{3} \ \,\delta_{\,ij} + 1) \ C_1
\ \frac{2\ \varepsilon^n_L}{(R^n_{ii})_L}$
\end{itemize}
\item Appel de \fort{rijech} pour le calcul des termes d'\'echo de paroi
 $\phi^n_{ij,w}$ si $\var{IRIJEC()}=1$ et ajout dans \var{SMBR}.\\
$\var{SMBR} = \var{SMBR} + \phi^n_{ij,w}$\\
Suivant son mode de calcul (\var{ICDPAR}), la distance � la paroi est directement accessible
par \var{RA(IDIPAR+IEL-1)} (\var{|ICDPAR|} = 1) ou bien
est calcul\'ee \`a partir de $\var{IA(IIFAPA(IPHAS)+IEL - 1)}$,
qui donne pour l'\'el\'ement $\var{IEL}$ le num\'ero de la face de bord
paroi la plus  proche (\var{|ICDPAR|} = 2). Ces tableaux ont \'et\'e renseign\'e une fois pour toutes au
d\'ebut de calcul.

\item  Appel de \fort{rijthe} pour le calcul des termes de gravit\'e $\mathcal{G}^n_{ij}$ et ajout dans \var{SMBR}.

Ce calcul n'a lieu que si $\var{IGRARI()} = 1$.
$ \var{SMBR} = \var{SMBR} + \mathcal{G}^n_{ij}$
\item Calcul de la partie explicite du terme de diffusion
 $\dive{\,\left[\tens{A}\,\grad{R}^{\,n}_{ij}\right]}$, plus pr\'ecis\'ement
des contributions du terme extradiagonal pris aux faces purement internes
(remplissage du tableau \var{VISCF}), puis aux faces de bord (remplissage du
tableau \var{VISCB}).
\begin{itemize}
\item [$\star$] Appel de \fort{grdcel} pour le calcul du gradient de
$R^{\,n}_{ij}$ dans chaque direction. Ces gradients sont respectivement
stock\'es dans les tableaux de travail \var{W1}, \var{W2} et \var{W3}.

\item [$\star$] boucle d'indice \var{IEL} sur les cellules $\Omega_l$ de centre
$L$ pour le
calcul de $\tens{E}^n\,\grad{R}^{\,n}_{ij}$ aux cellules dans un premier temps :\\
\begin{itemize}
\item [$\Rightarrow$] $\displaystyle \var{TRRIJ}= \frac{1}{2} (R^{\,n}_{ii})_L $
\item [$\Rightarrow$] $\displaystyle \var{CSTRIJ} = \rho^n_L\ C_S \ \displaystyle\frac{(R^n_{ii})_L}{2\,\varepsilon^n_L}$
\item [$\Rightarrow$] $\displaystyle \var{W4(IEL)} = \rho^n_L\ C_S\
\displaystyle\frac{(R^n_{ii})_L}{2\,\varepsilon^n_L} \left[\,(R^{\,n}_{12})_L \ \var{W2(IEL)} +
(R^{\,n}_{13})_L \ \var{W3(IEL)}\,\right]$
\item [$\Rightarrow$] $\displaystyle \var{W5(IEL)} = \rho^n_L\ C_S\
\displaystyle\frac{(R^n_{ii})_L}{2\,\varepsilon^n_L} \left[\,(R^{\,n}_{12})_L \ \var{W1(IEL)} +
(R^{\,n}_{23})_L \ \var{W3(IEL)}\,\right]$
\item [$\Rightarrow$] $\displaystyle \var{W6(IEL)} = \rho^n_L\ C_S\
\displaystyle\frac{(R^n_{ii})_L}{2\,\varepsilon^n_L} \left[\,(R^{\,n}_{13})_L \ \var{W1(IEL)} + (R^{\,n}_{23})_L \ \var{W2(IEL)}\,\right]$
\end{itemize}



\item [$\star$] Appel de \fort{vectds}\footnote{Le r\'esultat est stock\'e dans
\var{VISCF} et \var{VISCB}. Dans \fort{vectds}, les valeurs aux faces internes
sont interpol\'ees lin\'eairement sans reconstruction et \var{VISCB} est mis \`a
z\'ero.} pour assembler $\displaystyle\left[ \tens{E}^n\,\grad{R}^{\,n}_{ij}
\right]\,.\,\vect{n}_{\,lm}S_{\,lm}$ aux faces $lm$.
\item [$\star$] Appel de \fort{divmas} pour calculer la divergence du flux d\'efini par \var{VISCF} et \var{VISCB}.
Le r\'esultat est stock\'e dans \var{W4}.\\
Ajout au second membre \var{SMBR}.\\
\var{SMBR} = \var{SMBR} + \var{W4}
\end{itemize}

A l'issue de cette \'etape, seule la partie extradiagonale de la diffusion prise
enti\`erement explicite~:
 $$\sum\limits_{m\in
Vois(l)}\left[\ \tens{E}^n\,\grad{R}^{\,n}_{ij} \right]_{\,lm}\,.\,\vect{n}_{\,lm}S_{\,lm}$$ a \'et\'e calcul\'ee.\\

\item Calcul de la partie diagonale du terme de diffusion\footnote{Seule la
composante normale  du  gradient de $R_{ij}$ aux faces sera implicite.} :\\
Comme on l'a d\'eja vu, une partie de cette contribution sera trait\'ee en
implicite (celle relative \`a la composante normale du gradient) et donc
ajout\'ee au second membre par \fort{bilsc2} ; l'autre
partie sera explicite et prise en compte dans $f_s^{\,exp}$.
\begin{itemize}
\item [$\star$] On effectue une boucle d'indice \var{IEL} sur les cellules
$\Omega_l$ de centre $L$ :
\begin{itemize}
\item [$\Rightarrow$] $\displaystyle \var{TRRIJ }= \frac{1}{2} (R^{\,n}_{ii})_L $
\item [$\Rightarrow$] $\displaystyle \var{CSTRIJ} = \rho^n_L \ C_S \ \frac{(R^{\,n}_{ii})_L}{2\,\varepsilon^n_L}$
\item [$\Rightarrow$] $\displaystyle \var{W4(IEL)} = \rho^n_L \ C_S \
\frac{(R^{\,n}_{ii})_L}{2\,\varepsilon^n_L} \ (R^{\,n}_{11})_L$
\item [$\Rightarrow$] $\displaystyle \var{W5(IEL)} = \rho^n_L \ C_S \ \frac{(R^{\,n}_{ii})_L}{2\,\varepsilon^n_L}\ (R^n_{22})_L$
\item [$\Rightarrow$] $\displaystyle \var{W6(IEL)} = \rho^n_L \ C_S \ \frac{(R^{\,n}_{ii})_L}{2\,\varepsilon^n_L} \ (R^n_{33})_L$
\end{itemize}

%\item Traitement du parall\'elisme et de la p\'eriodicit\'e.

\item [$\star$] On effectue une boucle d'indice \var{IFAC} sur les faces
purement internes $lm$ pour remplir le tableau \var{VISCF} :
\begin{itemize}
\item [$\Rightarrow$] Identification des cellules $\Omega_l$ et $\Omega_m$ de
centre respectif $L$ (variable \var{II}) et $M$ (variable \var{JJ}), se trouvant de chaque c\^ot\'e de la face
$lm$\footnote{La normale \`a la face est orient\'ee de L vers M.}.
\item [$\Rightarrow$] Calcul du carr\'e de la surface de la face. La valeur est
stock\'ee dans le tableau \var{SURFN2}.
\item [$\Rightarrow$] Interpolation du gradient de $R^{\,n}_{ij}$ \`a la face
$lm$ (gradient facette $\left[\grad{R}^{\,n}_{ij}\right]_{\,lm}$) :
\begin{equation}\notag
\left\{\begin{array}{ll}
\var{GRDPX} &= \displaystyle \frac{1}{2} \left(\var{W1(II)} + \var{W1(JJ)}
\right) \\
&\\
\var{GRDPY} &= \displaystyle \frac{1}{2} \left(\var{W2(II)} + \var{W2(JJ)} \right) \\
&\\
\var{GRDPZ} &= \displaystyle \frac{1}{2} \left(\var{W3(II)} + \var{W3(JJ)} \right)
\end{array}\right.
\end{equation}
\item [$\Rightarrow$] Calcul du gradient de $R^{\,n}_{ij}$ normal \`a la face
$lm$, $\left[\grad{R}^{\,n}_{ij}\right]_{\,lm}.\vect{n}_{\,lm}\,S_{\,lm}$.\\

$\displaystyle \var{GRDSN} =  \var{GRDPX} \ \var{SURFAC(1,IFAC)} + \var{GRDPY} \ \var{SURFAC(2,IFAC)} +  \var{GRDPZ} \ \var{SURFAC(3,IFAC)}$
$\var{SURFAC}$ est le vecteur surface de la face \var{IFAC}.


\item [$\Rightarrow$] calcul de
 $\left[\grad{R^{\,n}_{ij}} - (\grad
R^{\,n}_{ij}\,.\,\vect{n}_{\,lm})\vect{n}_{\,lm}\right]$, les vecteurs \'etant
calcul\'es \`a la face $lm$ :
\begin{equation}\notag
\left\{\begin{array}{lll}
&\displaystyle \var{GRDPX} &= \var{GRDPX} - \displaystyle\frac{\var{GRDSN}}{\var{SURFN2}} \ \var{SURFAC(1,IFAC)}\\
&&\\
&\displaystyle \var{GRDPY} &= \var{GRDPY} - \displaystyle\frac{\var{GRDSN}}{\var{SURFN2}} \ \var{SURFAC(2,IFAC)} \\
&&\\
&\displaystyle \var{GRDPZ} &= \var{GRDPZ} - \displaystyle\frac{\var{GRDSN}}{\var{SURFN2}} \ \var{SURFAC(3,IFAC)}
\end{array}\right.
\end{equation}
\item [$\Rightarrow$] finalisation du calcul de l'expression totalement
explicite
 $$\left[ \tens{D}^n\,\left( \grad{R^{\,n}_{ij}} - (\grad R^{\,n}_{ij}\,.\,\vect{n}_{\,lm})\,\vect{n}_{\,lm}\right) \right]\,.\,\vect{n}_{\,lm}$$
\begin{equation}\notag
\begin{array} {ll}
\displaystyle \var{VISCF} = &
 \displaystyle\frac{1}{2} (\ \var{W4(II)} +\ \var{W4(JJ)}) \ \var{GRDPX} \
\var{SURFAC(1,IFAC)})\ + \\
&\\
&  \displaystyle\frac{1}{2} (\ \var{W5(II)} +\ \var{W5(JJ)}) \ \var{GRDPY} \
\var{SURFAC(2,IFAC)})\ + \\
&\\
&  \displaystyle\frac{1}{2} (\ \var{W6(II)} +\ \var{W6(JJ)}) \ \var{GRDPZ} \ \var{SURFAC(3,IFAC)})
\end{array}
\end{equation}
\end{itemize}

\item [$\star$] Mise \`a z\'ero du tableau \var{VISCB}.

\item [$\star$] Appel de \fort{divmas} pour calculer la divergence de~:
 $$\tens{D}^{\,n}\,\left( \grad{R^{\,n}_{ij}} - (\grad R^{\,n}_{ij}\,.\,\vect{n}_{\,lm})\vect{n}_{\,lm}\right)$$ d\'efini aux faces dans \var{VISCF} et \var{VISCB}.

Le r\'esultat est stock\'e dans le tableau \var{W1}.\\
Ajout au second membre \var{SMBR}.\\
$\var{SMBR} = \var{SMBR} + \var{W1}$
\end{itemize}
\item Calcul de la viscosit\'e orthotrope $\gamma^n_{\,lm}$ \`a la face $lm$ de la variable principale
$R^{\,n}_{ij}$\\
Ce calcul permet au sous-programme \fort{codits} de compl\'eter le second membre
\var{SMBR} par :
\begin{equation}
\begin{array} {ll}
& \sum\limits_{m\in Vois(l)}
\mu^n_{\,lm}\,\left(\grad{R}^{\,n}_{ij}\,.\,\vect{n}_{\,lm}\right)S_{\,lm}
 + \sum\limits_{m\in Vois(l)} \left(\grad{R}^{\,n}_{ij}
\,.\,\vect{n}_{\,lm}\right)\left[\tens{D}^{\,n}\,\vect{n}_{\,lm}\right]_{\,lm}\,.\,\vect{n}_{\,lm}\
S_{\,lm}\\
& = \sum\limits_{m\in Vois(l)}(\,\mu^n_{\,lm}\, + \,\gamma^n_{\,lm}\,)\,\left(\grad{R}^{\,n}_{ij}\,.\,\vect{n}_{\,lm}\right)S_{\,lm}
\end{array}
\end{equation}
sans pr\'eciser la nature de la face $lm$, {\it via} l'appel \`a \fort{bilsc2}  et de disposer de la quantit\'e
$(\mu^n_{\,lm}\, + \gamma^n_{\,lm})$ pour construire sa
matrice simplifi\'ee.\\
\begin{itemize}
\item [$\star$] On effectue une boucle d'indice \var{IEL} sur les cellules
$\Omega_l$ :
\begin{itemize}
\item [$\Rightarrow$] $\displaystyle \var{TRRIJ }= \frac{1}{2} (R^{\,n}_{ii})_L $
\item [$\Rightarrow$] $\displaystyle \var{RCSTE} = \rho^n_L \ C_S \ \frac{ (R^{\,n}_{ii})_L}{2\,\varepsilon^n_L} $
\item [$\Rightarrow$] $\displaystyle \var{W1(IEL)} = \mu^n +\rho^n_L \ C_S \ \frac{
(R^{\,n}_{ii})_L}{2\,\varepsilon^n_L}\ (R^n_{11})_L$
\item [$\Rightarrow$] $\displaystyle \var{W2(IEL)} = \mu^n + \rho^n_L \ C_S \ \frac{ (R^{\,n}_{ii})_L}{2\,\varepsilon^n_L}\ (R^n_{22})_L$
\item [$\Rightarrow$] $\displaystyle \var{W3(IEL)} = \mu^n + \rho^n_L \ C_S \ \frac{ (R^{\,n}_{ii})_L}{2\,\varepsilon^n_L}\ (R^n_{33})_L$
\end{itemize}

\item [$\star$] Appel de \fort{visort} pour calculer la viscosit\'e orthotrope
\footnote{Comme dans le sous-programme \fort{viscfa}, on multiplie la viscosit\'e par
$\displaystyle \frac{S_{\,lm}}{\overline{L'M'}}$, o\`u $S_{\,lm}$ et
$\overline{L'M'}$ repr\'esentent respectivement la surface de la face $lm$ et la
mesure alg\'ebrique du segment reliant les projections des centres des cellules
voisines sur la normale \`a la face. On garde dans ce sous-programme  la possibilit\'e d'interpoler la viscosit\'e aux cellules lin\'eairement ou d'utiliser une moyenne harmonique. La viscosit\'e au bord est celle de la cellule de bord correspondante.}
$ \gamma^n_{\,lm} = (\tens{D}^{\,n}\,\vect{n}_{\,lm}).\vect{n}_{\,lm}$ aux faces $lm$

Le r\'esultat est stock\'e dans les tableaux \var{VISCF} et \var{VISCB}.
\end{itemize}

\item appel de \fort{codits} pour la r\'esolution de l'\'equation de
convection/diffusion/termes sources de la variable $R_{ij}$. Le terme source est
\var{SMBR}, la viscosit\'e \var{VISCF} aux faces purement internes (
resp. \var{VISCB} aux faces de bord) et \var{FLUMAS} le flux de masse interne
 ( resp. \var{FLUMAB} flux de masse au bord). Le r\'esultat est la variable $R_{ij}$ au temps
$n+1$, donc $R^{\,n+1}_{ij}$. Elle est stock\'ee dans le tableau \var{RTP} des
variables mises \`a jour.

\end{itemize}

\etape{Appel de \fort{reseps} pour la r\'esolution de la variable $\varepsilon$}

On d\'ecrit ci-dessous le sous-programme \fort{reseps}, les commentaires du sous-programme \fort{resrij} ne seront pas r\'ep\'et\'es puisque les deux sous-programmes ne diff\`erent que par leurs termes sources.

\begin{itemize}
\item Initialisation \`a z\'ero de \var{SMBR} et \var{ROVSDT}.

\item{Lecture et prise en compte des termes sources utilisateur pour la variable $\varepsilon$ :}

Appel de \fort{ustsri} pour charger les termes sources utilisateurs. Ils sont
stock\'es dans les tableaux suivants :\\
pour la cellule $\Omega_l$ repr\'esent\'ee par $\var{IEL}$ de centre $L$, on a :
\begin{equation}\notag
\left\{\begin{array}{lll}
&\var{ROVSDT(IEL)}&= |\Omega_l| \ \alpha_{\varepsilon}\\
&\var{SMBR(IEL)}&=|\Omega_l| \ \beta_{\varepsilon}\\
\end{array}\right.
\end{equation}
On affecte alors les valeurs ad\'equates au second membre \var{SMBR} et \`a la
diagonale \var{ROVSDT} comme suit :
\begin{equation}\notag
\left\{\begin{array}{lll}
&\var{SMBR(IEL)} &= \var{SMBR(IEL)} +\ |\Omega_l| \ \alpha_{\,\varepsilon} \
\varepsilon^n_L \\
&\var{ROVSDT(IEL)}&= \text{max }(-\ |\Omega_l| \ \alpha_{\,\varepsilon},0)\\
\end{array}\right.
\end{equation}

\item{Calcul du terme source de masse si $\Gamma_L > 0$ :
\begin{equation}\notag
\left\{\begin{array}{lll}
&\displaystyle \var{SMBR(IEL)} = \var{SMBR(IEL)} + |\Omega_l| \ \Gamma_L \
(\varepsilon^{\,in}_L -\varepsilon^n_L) \\
&\displaystyle \var{ROVSDT(IEL)}= \var{ROVSDT(IEL)} + |\Omega_l| \ \Gamma_L
\end{array}\right.
\end{equation}
\item Calcul du terme d'accumulation de masse et du terme instationnaire \\
On stocke $\displaystyle \var{W1}= \int_{\Omega_l}\dive\,(\rho\,\vect{u})\,d\Omega$
calcul\'e par \fort{divmas} \`a l'aide des flux de masse internes et aux bords.\\
On incr\'emente ensuite \var{SMBR} et \var{ROVSDT}.
\begin{equation}\notag
\left\{\begin{array}{lll}
&\var{SMBR(IEL)} &= \var{SMBR(IEL)} + \var{ICONV}\ \varepsilon^n_L\,(\displaystyle
\int_{\Omega_l}\dive\,(\rho\,\vect{u})\ d\Omega) \\
&\var{ROVSDT(IEL)}& = \var{ROVSDT(IEL)} +  \var{ISTAT}\,\displaystyle
\frac{\rho^n_L \ |\Omega_l|}{\Delta t^n} -  \var{ICONV}\ (\displaystyle
\int_{\Omega_l}\dive\,(\rho\,\vect{u})\ d\Omega) \\
\end{array}\right.
\end{equation}

\item Traitement du terme de production
 $\displaystyle \rho\,C_{\varepsilon_1}\,\frac{\varepsilon}{k}\,\mathcal{P}$
 et du terme de dissipation $-\,\displaystyle \rho\,C_{\varepsilon_2}\,\frac{\varepsilon}{k}\,\varepsilon$ \\
pour cela, on effectue une boucle d'indice \var{IEL} sur les cellules $\Omega_l$
de centre $L$ :
\begin{itemize}
\item [$\Rightarrow$] $\displaystyle \var{TRPROD}= \frac{1}{2} (\mathcal{P}^n_{ii})_L = \frac{1}{2} \left[ \var{PRODUC(1,IEL)} +  \var{PRODUC(2,IEL)} +  \var{PRODUC(3,IEL)} \right] $
\item [$\Rightarrow$] $\displaystyle \var{TRRIJ }= \frac{1}{2} (R^n_{ii})_L $
\item [$\Rightarrow$] $\displaystyle \var{SMBR(IEL)} = \var{SMBR(IEL)} + \rho^n_L
|\Omega_l| \left[ -C_{\varepsilon_2} \ \frac{2\,(\varepsilon^n_L)^2}{(R^n_{ii})_L} + C_{\varepsilon_1} \ \frac{\varepsilon^n_L}{(R^n_{ii})_L}\ (\mathcal{P}^n_{ii})_L \right] $
\item [$\Rightarrow$] $\displaystyle \var{ROVSDT(IEL)} = \var{ROVSDT(IEL)} + C_{\varepsilon_2} \ \rho^n_L \ |\Omega_l| \ \frac{2\,\varepsilon^n_L}{(R^n_{ii})_L}$
\end{itemize}

\item Appel de \fort{rijthe} pour le calcul des termes de gravit\'e $\mathcal{G}^n_{\varepsilon}$ et ajout dans \var{SMBR}.

$ \var{SMBR} = \var{SMBR} + \mathcal{G}^n_{\varepsilon}$\\
Ce calcul n'a lieu que si $\var{IGRARI()} = 1$.

\item Calcul de la diffusion de $\varepsilon$ \\
 Le terme $\dive \left[\mu\, \grad(\varepsilon) + \tens{A'}\,\grad(\varepsilon)
\right]$ est calcul\'e exactement de la m\^eme mani\`ere que pour les tensions
de Reynolds $R_{ij}$ en rempla\c cant $\tens{A}$ par $\tens{A'}$.

\item Appel de \fort{codits} pour la r\'esolution de l'\'equation de
convection/diffusion/termes sources de la variable principale $\varepsilon$. Le
r\'esultat $\varepsilon^{\,n+1}$ est stock\'e dans le tableau \var{RTP} des
variables mises \`a jour.
}
\end{itemize}

\etape{clippings finaux}
On passe enfin dans le sous-programme  \fort{clprij} pour faire un clipping \'eventuel
des variables $R^{\,n+1}_{ij}$ et $\varepsilon^{\,n+1}$. Le sous-programme
\fort{clprij} est appel\'e\footnote{L'option
$\var{ICLIP} = 1$ consiste en un clipping minimal des variables $R_{ii}$ et
$\varepsilon$ en prenant la valeur absolue de ces variables puisqu'elles ne
peuvent \^etre que positives.} avec $\var{ICLIP} = 2$ . Cette option
\footnote{Quand la valeur des grandeurs $R_{ii}$ ou $\varepsilon$ est
n\'egative, on la remplace par le minimum entre sa valeur absolue et (1,1)
fois la valeur obtenue au pas de temps pr\'ec\'edent.} contient l'option $\var{ICLIP} = 1$  et permet de v\'erifier l'in\'egalit\'e de Cauchy-Schwarz sur les grandeurs extra-diagonales du tenseur $\tens{R}$ (pour $i \neq j$, $|R_{ij}|^2 \le R_{ii} R_{jj}$).


%%%%%%%%%%%%%%%%%%%%%%%%%%%%%%%%%%
%%%%%%%%%%%%%%%%%%%%%%%%%%%%%%%%%%
\section{Points \`a traiter}
%%%%%%%%%%%%%%%%%%%%%%%%%%%%%%%%%%
%%%%%%%%%%%%%%%%%%%%%%%%%%%%%%%%%%
Sauf mention explicite, $\phi$ repr\'esentera une tension de Reynolds ou la dissipation turbulente ($\phi = R_{ij} \ \text{ou} \ \varepsilon$).

\begin{itemize}
\item {La vitesse utilis\'ee pour le calcul de la production est explicite. Est-ce qu'une implicitation peut am\'eliorer la pr\'ecision temporelle de $\phi$ \footnote{Cette remarque peut \^etre g\'en\'eralis\'ee. En effet, peut-on envisager d'actualiser les variables d\'ej\`a r\'esolues (sans r\'eactualiser les variables turbulentes apr\`es leur r\'esolution)? Ceci obligerait \`a modifier les sous-programmes tels que \fort{condli} qui sont appel\'es au d\'ebut de la boucle en temps.} ?}
\item {Dans quelle mesure le terme d'\'echo de paroi est-il valide ? En effet, ce terme est remis en question par certains auteurs.}
\item {On peut envisager la r\'esolution d'un syst\`eme hyperbolique pour les
tensions de Reynolds afin d'introduire un couplage avec le champ de vitesse.}
\item {Le flux au bord \var{VISCB} est annul\'e dans le sous-programme
\fort{vectds}. Peut-on envisager de mettre au bord la valeur de la variable
concern\'ee \`a la cellule de bord correspondant? De m\^eme, il faudrait se
pencher sur les hypoth\`eses sous-jacentes \`a l'annulation des contributions
aux bords de \var{VISCB} lors du calcul de : $$\left[ \tens{D}^n\,\left( \grad{R^{\,n}_{ij}} - (\grad R^{\,n}_{ij}\,.\,\vect{n}_{\,lm})\,\vect{n}_{\,lm}\right) \right]\,.\,\vect{n}_{\,lm}.$$}
\item {Un probl\`eme de pond\'eration appara\^\i t plus g\'en\'eralement. Si on prend la partie explicite de $\tens{D}\,\grad(\phi)$, la pond\'eration aux faces internes utilise le coefficient $\displaystyle\frac{1}{2}$ avec pond\'eration s\'epar\'ee de $\tens{D}$ et $\grad(\phi)$, alors que pour $\tens{E}\,\grad(\phi)$, on calcule d'abord ce terme aux cellules pour ensuite l'interpoler lin\'eairement aux faces \footnote{Cette interpolation se fait dans \fort{vectds} avec des coefficients de pond\'eration aux faces.}. Ceci donne donc deux types d'interpolations pour des termes de m\^eme nature.}
\item {On laisse la possibilit\'e dans \fort{visort} d'utiliser une moyenne
harmonique aux faces. Est-ce que ceci est valable puisque les interpolations
utilis\'ees lors du calcul de la partie explicite de $\tens{A}\,\grad{\phi}$
sont des moyennes arithm\'etiques ?}
\item {Les techniques adopt\'ees lors du clipping sont \`a revoir.}
\item {On utilise dans le cadre du mod\`ele $\displaystyle R_{ij}-\varepsilon $ une semi-implicitation de termes comme $\displaystyle \phi_{ij,1}$ ou $\displaystyle -\rho\,C_{\varepsilon_2}\,\frac{\varepsilon}{k}\,\varepsilon$. On peut envisager le m\^eme type d'implicitation dans \fort{turbke} m\^eme en pr\'esence du couplage $\displaystyle k-\varepsilon$.}
\item L'adoption d'une r\'esolution d\'ecoupl\'ee fait perdre l'invariance par rotation.
\item La formulation et l'implantation des conditions aux limites de paroi
devront \^etre v\'erifi\'ees. En effet, il semblerait que, dans certains cas, des ph\'enom\`enes
``oscillatoires'' apparaissent, sans qu'il soit ais\'e d'en d\'eterminer la cause.
\item L'implicitation partielle (du fait de la r\'esolution d\'ecoupl\'ee) des
conditions aux limites conduit souvent \`a des calculs instables. Il
conviendrait d'en conna\^\i tre la raison. L'implicitation partielle avait
\'et\'e mise en \oe uvre afin de tenter d'utiliser un pas de temps plus grand
dans le cas de jets axisym\'etriques en particulier.

\end{itemize}

%                      Code_Saturne version 1.3
%                      ------------------------
%
%     This file is part of the Code_Saturne Kernel, element of the
%     Code_Saturne CFD tool.
%
%     Copyright (C) 1998-2007 EDF S.A., France
%
%     contact: saturne-support@edf.fr
%
%     The Code_Saturne Kernel is free software; you can redistribute it
%     and/or modify it under the terms of the GNU General Public License
%     as published by the Free Software Foundation; either version 2 of
%     the License, or (at your option) any later version.
%
%     The Code_Saturne Kernel is distributed in the hope that it will be
%     useful, but WITHOUT ANY WARRANTY; without even the implied warranty
%     of MERCHANTABILITY or FITNESS FOR A PARTICULAR PURPOSE.  See the
%     GNU General Public License for more details.
%
%     You should have received a copy of the GNU General Public License
%     along with the Code_Saturne Kernel; if not, write to the
%     Free Software Foundation, Inc.,
%     51 Franklin St, Fifth Floor,
%     Boston, MA  02110-1301  USA
%
%-----------------------------------------------------------------------
%
\programme{vortex}
%
\vspace{1cm}
%%%%%%%%%%%%%%%%%%%%%%%%%%%%%%%%%%
%%%%%%%%%%%%%%%%%%%%%%%%%%%%%%%%%%
\section{Fonction}
%%%%%%%%%%%%%%%%%%%%%%%%%%%%%%%%%%
%%%%%%%%%%%%%%%%%%%%%%%%%%%%%%%%%%
Ce sous-programme est d�di� � la g�n�ration des conditions d'entr�e
turbulente utilis�es en LES.


La m�thode des vortex est bas�e sur une approche de tourbillons
ponctuels. L'id�e de la m�thode consiste � injecter des tourbillons 2D dans le
plan d'entr�e du calcul, puis � calculer le champ de vitesse induit par ces
tourbillons au centre des faces d'entr�e.

%                      Code_Saturne version 1.3
%                      ------------------------
%
%     This file is part of the Code_Saturne Kernel, element of the
%     Code_Saturne CFD tool.
% 
%     Copyright (C) 1998-2007 EDF S.A., France
%
%     contact: saturne-support@edf.fr
% 
%     The Code_Saturne Kernel is free software; you can redistribute it
%     and/or modify it under the terms of the GNU General Public License
%     as published by the Free Software Foundation; either version 2 of
%     the License, or (at your option) any later version.
% 
%     The Code_Saturne Kernel is distributed in the hope that it will be
%     useful, but WITHOUT ANY WARRANTY; without even the implied warranty
%     of MERCHANTABILITY or FITNESS FOR A PARTICULAR PURPOSE.  See the
%     GNU General Public License for more details.
% 
%     You should have received a copy of the GNU General Public License
%     along with the Code_Saturne Kernel; if not, write to the
%     Free Software Foundation, Inc.,
%     51 Franklin St, Fifth Floor,
%     Boston, MA  02110-1301  USA
%
%-----------------------------------------------------------------------
%
%%%%%%%%%%%%%%%%%%%%%%%%%%%%%%%%%%
%%%%%%%%%%%%%%%%%%%%%%%%%%%%%%%%%%
\section{Discr\'etisation}
%%%%%%%%%%%%%%%%%%%%%%%%%%%%%%%%%%
%%%%%%%%%%%%%%%%%%%%%%%%%%%%%%%%%%

Le terme convectif en $\dive(\underline{u} \otimes \rho\,\underline{u})$
introduit une non lin\'earit\'e et un couplage des composantes de la vitesse
$\vect{u}$ dans l'�quation (\ref{Base_Preduv_eqqdm}). Une lin\'earisation et un d\'ecouplage
des trois composantes de la 
vitesse sont r\'ealis\'es lors de la discr\'etisation de cette \'etape de
pr\'ediction.\\
En effet, soit :
\begin{equation}
\vect{\widetilde{u}}= \vect{u}^n + \delta \vect{u} 
\end{equation}
La contribution exacte du terme convectif \`a prendre en compte dans cette
\'etape de pr\'ediction serait :\\
\begin{equation}\label{Base_Preduv_Conv_exact}
\begin{array}{ll}
\dive(\vect{\widetilde{u}} \otimes \rho\,\vect{\widetilde{u}}) =
\dive(\vect{u}^{n} \otimes \rho\,\vect{u}^{n}) + \dive(\delta \vect{u} \otimes
\rho\,\vect{u}^{n}) +  \underbrace { \dive(\vect{u}^{n} \otimes
\rho\,\delta \vect{u})}_{\text {terme couplant lin\'eaire}} +  \underbrace { \dive(\delta \vect{u} \otimes
\rho\,\delta \vect{u})}_{\text {terme couplant et non lin\'eaire}}\\
\end{array} 
\end{equation}
Les deux derniers termes de l'expression (\ref{Base_Preduv_Conv_exact}) sont {\em a priori} n�glig�s
de mani�re � obtenir un syst\`eme en vitesse qui soit d\'ecoupl\'e et donc,
�viter l'inversion d'une matrice pouvant \^etre de tr\`es grande taille. Ces
deux termes peuvent n�anmoins �tre pris en compte de mani�re plus ou moins
approch�e par extrapolation explicite du flux de masse en $n+\theta_F$ (pour le
terme couplant lin�aire provenant de la convection de $\vect{u}^{n}$ par $\delta
\vect{u}$) et utilisation d'un point-fixe par sous it�ration sur le sous
programme \fort{navsto} (pour le terme non-lin�aire, en sp�cifiant $\var{NTERUP}>1$).

L'�quation (\ref{Base_Preduv_eqqdm}) est discr�tis�e au temps $n+\theta$ � l'aide d'un
$\theta$-sch�ma, et le tenseur des pertes de charges d�compos� en une partie
diagonale $\tens{K}_{d}$ et une extradiagonale $\tens{K}_{e}$ (soit
 $\tens{K}_{pdc}=\tens{K}_{d}+\tens{K}_{e}$).\\
$\bullet$ La pression est suppos�e connue en $n-1+\theta$ (d�calage temporel
pression-vitesse) et le gradient naturellement calcul� � cet instant.\\ 
$\bullet$ Les termes sources de viscosit� secondaire, de gradient transpos\'e,
ceux provenant du mod�le de turbulence\footnote{except� $\dive (\mu_t\ (\ggrad
\underline {u}))$}, $\rho\,\tens{K}_{\,e}\ \underline{u}$, $(\rho -\rho_0)
\underline {g}$ ainsi que $\underline{T}_{s}^{\,exp}$ et
$\Gamma\,\underline{u}_{\,i}$ sont pris de mani�re explicite au temps $n$, ou
extrapol�s suivant le sch�ma en temps choisi pour les propri�t�s physique et les
termes sources.\\ 
$\bullet$ Les termes sources $\underline{u}\,\,\dive (\rho\,\underline {u})$,
$\Gamma\,\,\underline{u}$, $T_{s}^{\,imp}\,\,\underline{u}$ et
$-\rho\,\tens{K}_{\,d}\,\,\underline{u}$ sont implicit�s est calcul�s �
l'instant $n+\theta$.\\ 
$\bullet$ Le terme de diffusion $\dive (\mu_{\,tot}\,\ggrad \underline{u})$ est
implicit� : la vitesse est prise � l'instant $n+\theta$ et la viscosit�
explicit�e ou extrapol�e.\\ 
$\bullet$ Enfin, le terme de convection en $\dive(\,\underline{u} \otimes
(\rho\underline{u})\,)$ est implicit� : la composante r�solue de la vitesse est
prise en $n+\theta$, et le flux de masse, explicit�, ou extrapol� en
$n+\theta_F$. 

Par souci de clart�, on suppose, en l'absence d'indication, que les propri�tes
physiques $\Phi$ ($\rho,\,\mu_{tot},\,...$) et le flux de masse
$(\rho\underline{u})$ sont pris respectivement aux instants $n+\theta_\Phi$ et
$n+\theta_F$, o� $\theta_\Phi$ et $\theta_F$ d�pendent des sch�mas en temps
sp�cifiquement utilis�s pour ces grandeurs\footnote{cf. \fort{introd}}. 

La discr�tisation temporelle de l'�quation (\ref{Base_Preduv_eqqdm}) s'�crit alors comme suit : 

\begin{equation}\label{Base_Preduv_eq_di1}
 \begin{array}{c}
\displaystyle \rho\,\ \frac{ \underline {\widetilde{u}}^{n+1} -\underline {u}^{n} }
{\Delta t} + \dive(\,\underline{\widetilde{u}}^{n+\theta} \otimes (\rho\underline{u})\,) -\dive
(\mu_{\,tot}\,\ggrad \underline{\widetilde{u}}^{n+\theta}) =
\\
\displaystyle
 - \grad p^{n-1+\theta} + \dive (\rho\,\underline {u})\,\underline{\widetilde{u}}^{n+\theta} +(\Gamma\,\underline{u}_{\,i})^{n+\theta_S}-\Gamma^n\,\,\underline{\widetilde{u}}^{n+\theta}
\\
\begin{array}{c}
\displaystyle
- \rho\,\tens{K}_{\,d}^{n}\,\,\underline{\widetilde{u}}^{n+\theta} - (\rho\,\tens{K}_{\,e}\ \underline{u})^{n+\theta_S} + (\underline{T}_{s}^{\,exp})^{\,n+\theta_S} + T_{s}^{\,imp}\,\,\underline{\widetilde{u}}^{n+\theta}
\\
\displaystyle
+[\dive (\mu_{\,tot}\,^t\ggrad \underline {u})]^{n+\theta_S}-\frac {2} {3}[\,\grad (\mu_{\,tot}\,\dive \underline {u})]^{n+\theta_S} + (\rho -\rho_0) \underline {g}
 - (\underline{turb})^{n+\theta_S}
\end{array}
\end{array}
\end{equation}
o\`u, par souci de simplification, on a pos\'e :
\begin{equation}
\mu_{\,tot}=
\begin{cases}
\mu+\mu_t & \text{pour les mod�les � viscosit� turbulente ou en LES}, \\
\mu & \text{pour les mod�les au second ordre ou en laminaire}
\end{cases} \ 
\end{equation}
\\
et :
\begin{equation}
\underline{turb}^{n}=
\begin{cases}
\displaystyle\frac {2}{3}\grad (\rho^{n}\,k^{n}) & \text{pour les mod�les � viscosit� turbulente}, \\
\dive(\rho^{n}\,\tens{R}^n) & \text{pour les mod�les au second ordre},\\
0 & \text{en laminaire ou en LES}\\
\end{cases}
\end{equation}
Par analogie avec l'�criture du $\theta$-sch�ma pour une variable scalaire, $\,
\underline {\widetilde{u}}^{n+\theta}$ est interpol�e � partir de la vitesse
pr�dite $\underline {\widetilde{u}}^{n+1}$ de la mani\`ere suivante\footnote{si
$\theta=1/2$, ou qu'une extrapolation est utilis�e, l'ordre 2 n'est obtenu que si
le pas de temps $\Delta t$ est uniforme en temps et en espace.}~: 
\begin{equation}
\underline {\widetilde{u}}^{n+\theta}=\theta\, \underline
{\widetilde{u}}^{n+1}+(1-\theta)\, \underline {u}^{n}\\ 
\end{equation}
Avec :
\begin{equation}
\left\{
\begin{array}{ll}
\theta = 1   & \text{Pour un sch\'ema de type Euler implicite d'ordre 1.}\\
\theta = 1/2 & \text{Pour un sch\'ema de type Cranck-Nicolson d'ordre 2.}\\
\end{array}
\right.
\end{equation}

L'�quation (\ref{Base_Preduv_eq_di1}) est alors r��crite sous la forme :

\begin{equation}\label{Base_Preduv_eq_di2}
\begin{array}{c}
\displaystyle \underbrace{\left(\frac{\rho}{\Delta t} -\theta \,\dive (\rho\,\underline {u}) +\theta \,\, \Gamma^n +
\theta \,\, \rho\,\tens{K}_{\,d}^n-\theta \,T_s^{\,imp} \right)}_{\displaystyle f_s^{imp}}\, (\underline {\,\widetilde{u}}^{n+1} -\underline {u}^{n})
\\
 +\, \theta\, \dive(\underline {\widetilde{u}}^{n+1} \otimes (\rho\underline{u}))-\, \theta\,\dive (\mu_{\,tot}\,\ggrad \underline {\widetilde{u}}^{n+1}) =
\\
-\,(1-\theta)\, \dive(\underline {u}^{n} \otimes (\rho\underline{u})) +\,(1-\theta)\,\dive (\mu_{\,tot}\,\ggrad \underline {u}^{n})
\\
f_s^{exp}\left\{
\begin{array}{c}
\displaystyle 
- \grad p^{n-1+\theta} + \dive (\rho\,\underline {u})\,\underline{u}^{n} +\,(\,\Gamma^{n}\,\underline{u}_{\,i}\,)^{n+\theta_S}- \Gamma^n\,\,\underline{u}^{n}
\\
\displaystyle
-(\,\rho\,\tens{K}_{\,e}\ \underline{u}\,)^{n+\theta_S} -\rho\,\tens{K}_{\,d}^n\ \underline{u}^{n}+ (\underline{T}_{s}^{\,exp})^{\,n+\theta_S} + T_s^{\,imp}\,\,\underline {u}^{n} 
\\
\displaystyle
+[\dive (\mu_{\,tot}\,^t\ggrad \underline {u}\,)]^{n+\theta_S}-\frac {2} {3}[\,\grad (\mu_{\,tot}\,\dive \underline {u}\,)]^{n+\theta_S} + (\rho -\rho_0) \underline {g}-(\underline{turb})^{n+\theta_S}
\end{array}
\right.
\end{array}
\end{equation}

d'o� l'�quation r�solue par le sous-programme \fort{codits} :
\begin{equation}\begin{array}{c}
\displaystyle
f_s^{\,imp}(\underline {\widetilde{u}}^{n+1}-\underline {u}^{n}) + \theta\, \dive(\underline{\widetilde{u}}^{n+1} \otimes (\rho
\underline{u})) - \theta\,\dive (\,\mu_{\,tot}\,\ggrad \underline{\widetilde{u}}^{n+1}) = 
\\\\
\displaystyle
-(1-\theta)\,\dive(\underline{u}^{n} \otimes (\rho \underline{u}))+(1-\theta)\,\dive (\,\mu_{\,tot}\,\ggrad \underline{u}^{n})
+ \underline{f}_{\,s}^{\,exp}
\end{array}
\end{equation}
La m\'ethode de discr\'etisation spatiale est d\'evelopp\'ee dans le sous-programme \fort{codits}.\\



\minititre{Remarques :}
{\tiny$\blacksquare$} Dans le cas standard sans extrapolation, le terme
$-\,T_s^{\,imp}$ n'est ajout� � $f_s^{\,imp}$ que s'il est positif afin de ne
pas affaiblir la dominance de la diagonale de la matrice � inverser.\\ 
{\tiny$\blacksquare$} Si une extrapolation est utilis�e, par contre,
$\,T_s^{\,imp}$ est ajout� � $f_s^{\,imp}$ quel que soit son signe. En effet, l'id�e intuitive qui
consiste � prendre~: 
\begin{equation}
\begin{cases}
\displaystyle
(\underline{T}_{s}^{\,exp} + T_{s}^{\,imp}\,\underline {u})^{\,n+\theta_S} &
\text{si } T_{s}^{\,imp} > 0\\ 
\displaystyle
(\underline{T}_{s}^{\,exp})^{\,n+\theta_S} + T_{s}^{\,imp}\,\underline{u}^{n+\theta} &\text{sinon}\\
\end{cases}
\end{equation} 
aboutit � une incoh�rence dans le traitement si $T_s^{imp}$ change de signe
entre deux pas de temps.\\ 
{\tiny$\blacksquare$} la partie diagonale $\tens{K}_{\,d}$ du terme
de perte de charge est utilis�e dans $f_s^{\,imp}$. En fait, pour \^etre rigoureux,
il faudrait ne retenir que les contributions positives (point signal\'e dans le
sous-programme utilisateur associ\'e \fort{uskpdc}). Cette prise en compte sera \`a am\'eliorer.\\
{\tiny$\blacksquare$} Le terme $\theta\,\Gamma^{n}-\theta\,\dive
(\rho\,\underline {u})$ ne pose pas de probl�me pour la 
dominance de la diagonale de la matrice car il est exactement compens� par le
terme de convection (cf. \fort{covofi}). 


%                      Code_Saturne version 1.3
%                      ------------------------
%
%     This file is part of the Code_Saturne Kernel, element of the
%     Code_Saturne CFD tool.
%
%     Copyright (C) 1998-2007 EDF S.A., France
%
%     contact: saturne-support@edf.fr
%
%     The Code_Saturne Kernel is free software; you can redistribute it
%     and/or modify it under the terms of the GNU General Public License
%     as published by the Free Software Foundation; either version 2 of
%     the License, or (at your option) any later version.
%
%     The Code_Saturne Kernel is distributed in the hope that it will be
%     useful, but WITHOUT ANY WARRANTY; without even the implied warranty
%     of MERCHANTABILITY or FITNESS FOR A PARTICULAR PURPOSE.  See the
%     GNU General Public License for more details.
%
%     You should have received a copy of the GNU General Public License
%     along with the Code_Saturne Kernel; if not, write to the
%     Free Software Foundation, Inc.,
%     51 Franklin St, Fifth Floor,
%     Boston, MA  02110-1301  USA
%
%-----------------------------------------------------------------------
%

%%%%%%%%%%%%%%%%%%%%%%%%%%%%%%%%%%
%%%%%%%%%%%%%%%%%%%%%%%%%%%%%%%%%%
\section{Mise en \oe uvre}
%%%%%%%%%%%%%%%%%%%%%%%%%%%%%%%%%%
%%%%%%%%%%%%%%%%%%%%%%%%%%%%%%%%%%
La num\'ero de la phase trait\'ee fait partie des arguments de \fort{turrij}. On
omettra volontairement de le pr\'eciser dans ce qui suit, on indiquera par $(\ )$ la
notion de tableau s'y rattachant.

\etape{Calcul des termes de production $\tens{\mathcal{P}}$}
\begin{itemize}
\item [$\star$] Initialisation \`a z\'ero du tableau \var{PRODUC} dimensionn\'e \`a $\var{NCEL}\times 6$.
\item [$\star$] On appelle trois fois \fort{grdcel} pour calculer les gradients des composantes de la vitesse $u$, $v$ et
$w$ prises au temps $n$.

Au final, on a :\\
$\displaystyle
\begin{array} {ll}
\var{PRODUC(1,IEL)} = & \displaystyle - 2 \left[ R_{11}^{\,n} \frac{\partial u^{\,n}} {\partial x} +R_{12}^{\,n} \frac{\partial u^{\,n}} {\partial y}+R_{13}^{\,n} \frac{\partial u^{\,n}} {\partial z} \right] \text{        (production de $R_{11}^{\,n}$)}\\
\var{PRODUC(2,IEL)} = & \displaystyle - 2 \left[ R_{12}^{\,n} \frac{\partial v^{\,n}} {\partial x} +R_{22}^{\,n} \frac{\partial v^{\,n}} {\partial y}+R_{23}^{\,n} \frac{\partial v^{\,n}} {\partial z} \right] \text{        (production de $R_{22}^{\,n}$)}\\
\var{PRODUC(3,IEL)} = & \displaystyle - 2 \left[ R_{13}^{\,n} \frac{\partial w^{\,n}} {\partial x} +R_{23}^{\,n} \frac{\partial w^{\,n}} {\partial y}+R_{33}^{\,n} \frac{\partial w^{\,n}} {\partial z} \right] \text{        (production de $R_{33}^{\,n}$)}\\
\var{PRODUC(4,IEL)} = & \displaystyle - \left[ R_{12}^{\,n} \frac{\partial u^{\,n}} {\partial x} +R_{22}^{\,n} \frac{\partial u^{\,n}} {\partial y}+R_{23}^{\,n} \frac{\partial u^{\,n}} {\partial z} \right] \\
& \displaystyle - \left[ R_{11}^{\,n} \frac{\partial v^{\,n}} {\partial x} +R_{12}^{\,n} \frac{\partial v^{\,n}} {\partial y}+R_{13}^{\,n} \frac{\partial v^{\,n}} {\partial z} \right] \text{        (production de $R_{12}^{\,n}$)} \\
\var{PRODUC(5,IEL)} = & \displaystyle - \left[ R_{13}^{\,n} \frac{\partial u^{\,n}} {\partial x} +R_{23}^{\,n} \frac{\partial u^{\,n}} {\partial y}+R_{33}^{\,n} \frac{\partial u^{\,n}} {\partial z} \right] \\
& \displaystyle - \left[ R_{11}^{\,n} \frac{\partial w^{\,n}} {\partial x} +R_{12}^{\,n} \frac{\partial w^{\,n}} {\partial y}+R_{13}^{\,n} \frac{\partial w^{\,n}} {\partial z} \right] \text{        (production de $R_{13}^{\,n}$)} \\
\var{PRODUC(6,IEL)} = & \displaystyle - \left[ R_{13}^{\,n} \frac{\partial v^{\,n}} {\partial x} +R_{23}^{\,n} \frac{\partial v^{\,n}} {\partial y}+R_{33}^{\,n} \frac{\partial v^{\,n}} {\partial z} \right] \\
& \displaystyle - \left[ R_{12}^{\,n} \frac{\partial w^{\,n}} {\partial x} +R_{22}^{\,n} \frac{\partial w^{\,n}} {\partial y}+R_{23}^{\,n} \frac{\partial w^{\,n}} {\partial z} \right]  \text{        (production de $R_{23}^{\,n}$)}
\end{array}
$
\end{itemize}

\etape{Calcul du gradient de la masse volumique $\rho^n$ prise au d\'ebut du pas
de temps courant\footnote{{\it i.e.} calcul\'ee \`a partir des
variables du pas de temps pr\'ec\'edent $n$ si n\'ecessaire.} $(n+1)$}
Ce calcul n'a lieu que si les termes de gravit\'e doivent \^etre pris en compte
($\var{IGRARI()} =1$).
\begin{itemize}
\item [$\star$] Appel de \fort{grdcel}  pour calculer le gradient de $\rho^n$
dans les trois directions de l'espace. Les conditions aux limites sur $\rho^n$
sont des conditions de Dirichlet puisque la valeur de $\rho^n$ aux faces de bord
$ik$ (variable \var{IFAC}) est connue et vaut $\rho_{\,b_{\,ik}}$. Pour \'ecrire les conditions aux limites
sous la forme habituelle, $$\rho_{\,b_{\,ik}} = \var{COEFA} + \var{COEFB}
\,\rho^n_{\,I'}$$ on pose alors $\var{COEFA}=
\var{PROPCE(IFAC,IPPROB(IROM(IPHAS)))}$ et $\var{COEFB} = \var{VISCB} = 0$.\\
$\var{PROPCE(1,IPPROB(IROM(IPHAS)))}$ (resp.$\var{VISCB}$) est utilis\'e en lieu
et place de l'habituel \var{COEFA} ($\var{COEFB}$), lors de l'appel \`a \fort{grdcel}.\\
On a donc :\\
$\displaystyle \var{GRAROX}= \frac{\partial \rho^n}{\partial x}\ $,$\displaystyle \ \var{GRAROY}= \frac{\partial
\rho^n}{\partial y}$ et $
\displaystyle \ \var{GRAROZ}= \frac{\partial \rho^n}{\partial z}\ $.

\end{itemize}

Le gradient de $\rho^n$ servira \`a calculer les termes de production par effets de gravit\'e si ces derniers sont pris en compte.

\etape{Boucle \var{ISOU} de $1$ \`a $6$ sur les tensions de Reynolds}
Pour $\var{ISOU} = 1,2,3,4,5,6$, on r\'esout respectivement et dans
l'ordre  les
\'equations de $R_{11}$, $R_{22}$, $R_{33}$, $R_{12}$, $R_{13}$ et $R_{23}$ par
l'appel au sous-programme \fort{resrij}.\\
La r\'esolution se fait par incr\'ement $\delta {R}_{ij}^{\,n+1,k+1}$ , en utilisant la m\^eme m\'ethode que
celle d\'ecrite dans le sous-programme \fort{codits}. On adopte ici les m\^emes notations.
\var{SMBR} est le second membre du syst\`eme \`a inverser, syst\`eme portant sur
les incr\'ements de la variable. \var{ROVSDT} repr\'esente la diagonale de la
matrice, hors convection/diffusion.\\
On va r\'esoudre l'\'equation (\ref{Base_Turrij_Eq_Temp_Rij}) sous forme incr\'ementale en
utilisant \fort{codits}, soit :
\begin{equation}\label{Base_Turrij_Eq_Temp_deltaRij}
\begin{array}{ll}
&\displaystyle \underbrace{\left(\frac {\rho^n_L}{\Delta t^n}
+ \rho^n_L \,C_1\,\frac{\varepsilon^n_L}{k^n_L}(1-\frac{\delta_{ij}}{3})
 - m^n_{\,lm} + \Gamma_L\,+ max(-\alpha^n_{R_{ij}},0)\right)\,|\Omega_l|}
_{\text {$\var{ROVSDT}$ contribuant
\`a la diagonale de la matrice simplifi\'ee de \fort{matrix}}}\,(\delta{R}_{ij}^{\,n+1,p+1})_{\,L}\\\\
&  \underbrace{+\sum\limits_{m\in Vois(l)}\displaystyle \left[
 m^n_{\,lm} \delta R_{ij,\,f_{\,lm}}^{\,n+1,p+1}
- (\mu^n_{\,lm} + \gamma^n_{\,lm})\
\frac{({\delta R}_{ij}^{\,n+1,p+1})_{M}-({\delta R}_{ij}^{\,n+1,p+1})_{L})}{\overline{L'M'}}\,
S_{\,lm} \right]}_{\text { convection upwind pur et diffusion non reconstruite
relatives \`a la matrice simplifi\'ee de \fort{matrix}\footnotemark}} \\
% voir le texte de la footmark plus bas
&= - \displaystyle\frac {\rho^n_L}{\Delta t^n}\,\left(\,(R^{\,n+1,p}_{ij})_L - (R^{\,n}_{ij})_L\,\right)\\
&-\,\underbrace{\displaystyle\int_{\Omega_l} \left(
\dive\,[\,(\rho\,\vect{u})^n\,R^{\,n+1,p}_{ij} - (\mu^n\,+ \gamma^n\,)\,
\grad{R^{\,n+1,p}_{ij}}\,]\right)\,d\Omega}_{\text {convection et diffusion
trait\'ees par \fort{bilsc2}}}\\
&+\displaystyle \int_{\Omega_l} \left[\,\mathcal{P}^{\,n+1,p}_{ij} + \mathcal{G}^{\,n+1,p}_{ij}
- \displaystyle\rho^n \,C_1\,\frac{\varepsilon^n}{k^n}\left[R^{\,n+1,p}_{ij}-
\frac{2}{3}\,k^n\,\delta_{ij}\right] + \phi^{\,n+1,p}_{ij,2} +
\phi^{\,n+1,p}_{ij,w}\,\right]\, d\Omega \\
& + \displaystyle\int_{\Omega_l} \left[- \frac{2}{3} \rho^n \varepsilon^n \delta_{ij}
 + \Gamma\,(\,R^{\,in}_{ij} - R^{\,n+1,p}_{ij}\,) +
\alpha^n_{R_{ij}}\,R^{\,n+1,p}_{ij}+ \beta^n_{R_{ij}}\right]\, d\Omega\\
&+ \sum\limits_{m\in
Vois(l)}\displaystyle \left[\ \tens{E}^n\,\grad{R}^{\,n+1,p}_{ij} \right]_{\,lm}\,.\,\vect{n}_{\,lm}S_{\,lm}\\
&+ \sum\limits_{m\in Vois(l)}\displaystyle \left[\
\tens{D}^n\,\grad{R}^{\,n+1,p}_{ij} \right]_{\,lm}\,.\,\vect{n}_{\,lm}S_{\,lm}\\
&- \sum\limits_{m\in Vois(l)} \gamma^n_{\,lm} \left( \grad{R}^{\,n+1,p}_{ij}\,.\,\vect{n}_{\,lm} \right)  S_{\,lm}\\
&+ \sum\limits_{m\in Vois(l)}  m^n_{\,lm}\,(R^{\,n+1,p}_{ij})_L\\
\end{array}
\end{equation}
% si on ne fait pas comme ca, il n'apparait pas
\footnotetext[\thefootnote]{Si $\var{IRIJNU} = 1$, on remplace  $\mu^n_{\,lm}$ par $(\mu +
\mu_t)^n_{\,lm}$ dans l'expression de la diffusion non reconstruite
associ\'ee \`a la matrice simplifi\'ee de \fort{matrix} ($\mu_t$ d\'esigne la
viscosit\'e turbulente calcul\'ee comme en $k-\varepsilon$).}

o\`u on rappelle :\\
pour $n$ donn\'e entier positif, on d\'efinit la suite
 $({R}_{ij}^{\,n+1,p})_{p \in \grandN}$
 par :
\begin{equation}\notag
\left\{\begin{array}{l}
{R}_{ij}^{\,n+1,0} = {R}_{ij}^{\,n}\\
{R}_{ij}^{\,n+1,p+1} = {R}_{ij}^{\,n+1,p} + \delta{R}_{ij}^{\,n+1,p+1} \\
\end{array}\right.
\end{equation}
$(\delta{R}_{ij}^{\,n+1,p+1})_{\,L}$ d\'esigne la valeur sur l'\'el\'ement
$\Omega_l$ du $\text{$(\,p+1\,)$-i\`eme}$ incr\'ement de ${R}_{ij}^{\,n+1}$,
$ m^n_{\,lm}$ le flux de masse \`a l'instant $n$ \`a travers la face $lm$,
$\delta R_{ij,\,f_{\,lm}}^{\,n+1,p+1}$ vaut $({\delta
R}_{ij}^{\,n+1,p+1})_{L}$  si $ m^n_{\,lm} \geqslant 0$, $({\delta
R}_{ij}^{\,n+1,p+1})_{M}$ sinon,
$\mathcal{P}^{\,n+1,p}_{ij}$, $\phi^{\,n+1,p}_{ij,2}$, $\phi^{\,n+1,p}_{ij,w}$ les valeurs
des quantit\'es associ\'ees correspondant \`a l'incr\'ement
$(\delta{R}_{ij}^{\,n+1,p})$.\\



Tous ces termes sont calcul\'es comme suit :
\begin{itemize}
\item Terme de gauche de l'\'equation (\ref{Base_Turrij_Eq_Temp_deltaRij})\\
Dans \fort{resrij} est calcul\'ee la variable \var{ROVSDT}. Les autres
termes sont compl\'et\'es par \fort{codits}, lors de la construction de la matrice simplifi\'ee , {\it via} un
appel au sous-programme \fort{matrix}. La quantit\'e
 $(\mu^n_{\,lm} + \gamma^n_{\,lm})$ \`a la face $lm$ est calcul\'ee lors de l'appel \`a
\fort{visort}.\\
\item Second membre de l'\'equation (\ref{Base_Turrij_Eq_Temp_deltaRij})\\
Le premier terme non d\'etaill\'e est calcul\'e par le sous-programme
\fort{bilsc2}, qui applique le sch\'ema convectif choisi par l'utilisateur, qui
reconstruit ou non selon le souhait de l'utilisateur les gradients intervenants
dans la convection-diffusion.\\
Les termes sans accolade sont, eux, compl\`etement explicites et ajout\'es au fur et
\`a mesure dans \var{SMBR} pour former
l'expression $f^{\,exp}_s$ de \fort{codits}.
\end{itemize}
On d\'ecrit ci-dessous les \'etapes de \fort{resrij} :
\begin{itemize}

\item DELTIJ = 1, pour $\var{ISOU} \leqslant 3$ et DELTIJ = 0  Si $\var{ISOU} >
3$. Cette valeur repr\'esente le symbole de Kroeneker $\delta_{ij}$.

\item Initialisation \`a z\'ero de \var{SMBR} (tableau contenant le second
membre) et \var{ROVSDT} (tableau contenant la diagonale de la matrice sauf celle
relative \`a la contribution de la
diagonale des op\'erateurs de convection et de diffusion lin\'earis\'es
\footnote{qui correspondent aux sch\'emas convectif upwind pur et diffusif sans
reconstruction.}), tous deux de dimension $\var{NCEL}$.

\item Lecture et prise en compte des termes sources utilisateur pour la variable $R_{ij}$

Appel \`a \fort{ustsri} pour charger les termes sources utilisateurs. Ils sont
stock\'es comme suit. Pour la cellule $\Omega_l$ de centre $L$, repr\'esent\'ee par $\var{IEL}$, on a :\\
\begin{equation}\notag
\left\{\begin{array}{lll}
&\var{ROVSDT(IEL)}&= |\Omega_l| \ \alpha_{R_{ij}}\\
&\var{SMBR(IEL)}&=|\Omega_l| \ \beta_{R_{ij}}\\
\end{array}\right.
\end{equation}
On affecte alors les valeurs ad\'equates au second membre \var{SMBR} et \`a la
diagonale \var{ROVSDT} comme suit :
\begin{equation}\notag
\left\{\begin{array}{lll}
&\var{SMBR(IEL)} &= \var{SMBR(IEL)} +\ |\Omega_l| \ \alpha_{R_{ij}} \ (R^n_{ij})_L \\
&\var{ROVSDT(IEL)}&= \text{max }(-\ |\Omega_l| \ \alpha_{R_{ij}},0)\\
\end{array}\right.
\end{equation}
La valeur de $ \var{ROVSDT}$ est ainsi calcul\'ee pour des raisons de stabilit\'e
num\'erique. En effet, on ne rajoute sur la diagonale que les valeurs positives,
ce qui correspond physiquement \`a impliciter les termes de rappel uniquement.
\item{Calcul du terme source de masse  si $\Gamma_L > 0$}

Appel de \fort{catsma} et incr\'ementation si n\'ecessaire de \var{SMBR} et
\var{ROVSDT} {\it via} :\\
\begin{equation}\notag
\left\{\begin{array}{lll}
\displaystyle \var{SMBR(IEL)} = \var{SMBR(IEL)} + |\Omega_l| \ \Gamma_L \
\left[(R^{\,in}_{ij})_L - (R^{\,n}_{ij})_L \right] \\
\displaystyle \var{ROVSDT(IEL)}=\var{ROVSDT(IEL)} + |\Omega_l| \ \Gamma_L
\end{array}\right.
\end{equation}
\item Calcul du terme d'accumulation de masse et du terme instationnaire

On stocke $\displaystyle \var{W1}= \int_{\Omega_l}\dive\,(\rho\,\vect{u})\,d\Omega$
calcul\'e par \fort{divmas} \`a l'aide des flux de masse aux faces internes
$ m^n_{\,lm}=\sum\limits_{m\in Vois(l)}{(\rho \vect{u})_{\,lm}^n} \text{.}\,
\vect{S}_{\,lm} $ (tableau \var{FLUMAS}) et des flux de masse aux bords  $ m^n_{\,b_{lk}} = \sum\limits_{k\in{\gamma_b(l)}}{(\rho \vect{u})_{\,{b}_{lk}}^n} \text{.}\,
\vect{S}_{\,{b}_{lk}} $ (tableau \var{FLUMAB}).
On incr\'emente ensuite \var{SMBR} et \var{ROVSDT}.
\begin{equation}\notag
\left\{\begin{array}{lll}
&\var{SMBR(IEL)} &= \var{SMBR(IEL)} + \var{ICONV}\  (R^n_{ij})_L\,(\displaystyle
\int_{\Omega_l}\dive\,(\rho\,\vect{u})\ d\Omega) \\
&\var{ROVSDT(IEL)}& = \var{ROVSDT(IEL)} +  \var{ISTAT}\,\displaystyle
\frac{\rho^n_L \ |\Omega_l|}{\Delta t^n} -  \var{ICONV}\ (\displaystyle
\int_{\Omega_l}\dive\,(\rho\,\vect{u})\ d\Omega) \\
\end{array}\right.
\end{equation}
\item Calcul des termes sources de production, des termes $\displaystyle
\phi_{\,ij,1}+\phi_{\,ij,2}$ et de la dissipation~$\displaystyle-\frac{2}{3} \varepsilon\,\delta_{\,ij}$ :

On effectue une boucle d'indice \var{IEL} sur les cellules $\Omega_l$ de centre $L$ :
\begin{itemize}
\item [$\Rightarrow$] $\displaystyle \var{TRPROD}= \frac{1}{2} (\mathcal{P}^n_{ii})_L = \frac{1}{2} \left[ \var{PRODUC(1,IEL)} +  \var{PRODUC(2,IEL)} +  \var{PRODUC(3,IEL)} \right] $
\item [$\Rightarrow$] $\displaystyle \var{TRRIJ }= \frac{1}{2} (R^n_{ii})_L $
\item [$\Rightarrow$] $\displaystyle \var{SMBR(IEL)} =\ \var{SMBR(IEL)}\ +$\\
$\ \displaystyle\rho^n_L |\Omega_l| \left[ \displaystyle
\frac{2}{3}\,\delta_{\,ij} \left( \ \displaystyle \frac{ C_2}{2}\,(\mathcal{P}^n_{ii})_L\ +
(C_1-1)\ \varepsilon^n_L\, \right)\right.$\\
$ + \left.\ (1-C_2) \ \var{PRODUC(ISOU,IEL)} -
\displaystyle C_1\ \frac{2\,\varepsilon^n_L}{(R^n_{ii})_L}\ (R^n_{ij})_L \right]$
\item [$\Rightarrow$] $\displaystyle \var{ROVSDT(IEL)} = \var{ROVSDT(IEL)} +
\rho^n_L \ |\Omega_l| \ (- \displaystyle \frac{1}{3} \ \,\delta_{\,ij} + 1) \ C_1
\ \frac{2\ \varepsilon^n_L}{(R^n_{ii})_L}$
\end{itemize}
\item Appel de \fort{rijech} pour le calcul des termes d'\'echo de paroi
 $\phi^n_{ij,w}$ si $\var{IRIJEC()}=1$ et ajout dans \var{SMBR}.\\
$\var{SMBR} = \var{SMBR} + \phi^n_{ij,w}$\\
Suivant son mode de calcul (\var{ICDPAR}), la distance � la paroi est directement accessible
par \var{RA(IDIPAR+IEL-1)} (\var{|ICDPAR|} = 1) ou bien
est calcul\'ee \`a partir de $\var{IA(IIFAPA(IPHAS)+IEL - 1)}$,
qui donne pour l'\'el\'ement $\var{IEL}$ le num\'ero de la face de bord
paroi la plus  proche (\var{|ICDPAR|} = 2). Ces tableaux ont \'et\'e renseign\'e une fois pour toutes au
d\'ebut de calcul.

\item  Appel de \fort{rijthe} pour le calcul des termes de gravit\'e $\mathcal{G}^n_{ij}$ et ajout dans \var{SMBR}.

Ce calcul n'a lieu que si $\var{IGRARI()} = 1$.
$ \var{SMBR} = \var{SMBR} + \mathcal{G}^n_{ij}$
\item Calcul de la partie explicite du terme de diffusion
 $\dive{\,\left[\tens{A}\,\grad{R}^{\,n}_{ij}\right]}$, plus pr\'ecis\'ement
des contributions du terme extradiagonal pris aux faces purement internes
(remplissage du tableau \var{VISCF}), puis aux faces de bord (remplissage du
tableau \var{VISCB}).
\begin{itemize}
\item [$\star$] Appel de \fort{grdcel} pour le calcul du gradient de
$R^{\,n}_{ij}$ dans chaque direction. Ces gradients sont respectivement
stock\'es dans les tableaux de travail \var{W1}, \var{W2} et \var{W3}.

\item [$\star$] boucle d'indice \var{IEL} sur les cellules $\Omega_l$ de centre
$L$ pour le
calcul de $\tens{E}^n\,\grad{R}^{\,n}_{ij}$ aux cellules dans un premier temps :\\
\begin{itemize}
\item [$\Rightarrow$] $\displaystyle \var{TRRIJ}= \frac{1}{2} (R^{\,n}_{ii})_L $
\item [$\Rightarrow$] $\displaystyle \var{CSTRIJ} = \rho^n_L\ C_S \ \displaystyle\frac{(R^n_{ii})_L}{2\,\varepsilon^n_L}$
\item [$\Rightarrow$] $\displaystyle \var{W4(IEL)} = \rho^n_L\ C_S\
\displaystyle\frac{(R^n_{ii})_L}{2\,\varepsilon^n_L} \left[\,(R^{\,n}_{12})_L \ \var{W2(IEL)} +
(R^{\,n}_{13})_L \ \var{W3(IEL)}\,\right]$
\item [$\Rightarrow$] $\displaystyle \var{W5(IEL)} = \rho^n_L\ C_S\
\displaystyle\frac{(R^n_{ii})_L}{2\,\varepsilon^n_L} \left[\,(R^{\,n}_{12})_L \ \var{W1(IEL)} +
(R^{\,n}_{23})_L \ \var{W3(IEL)}\,\right]$
\item [$\Rightarrow$] $\displaystyle \var{W6(IEL)} = \rho^n_L\ C_S\
\displaystyle\frac{(R^n_{ii})_L}{2\,\varepsilon^n_L} \left[\,(R^{\,n}_{13})_L \ \var{W1(IEL)} + (R^{\,n}_{23})_L \ \var{W2(IEL)}\,\right]$
\end{itemize}



\item [$\star$] Appel de \fort{vectds}\footnote{Le r\'esultat est stock\'e dans
\var{VISCF} et \var{VISCB}. Dans \fort{vectds}, les valeurs aux faces internes
sont interpol\'ees lin\'eairement sans reconstruction et \var{VISCB} est mis \`a
z\'ero.} pour assembler $\displaystyle\left[ \tens{E}^n\,\grad{R}^{\,n}_{ij}
\right]\,.\,\vect{n}_{\,lm}S_{\,lm}$ aux faces $lm$.
\item [$\star$] Appel de \fort{divmas} pour calculer la divergence du flux d\'efini par \var{VISCF} et \var{VISCB}.
Le r\'esultat est stock\'e dans \var{W4}.\\
Ajout au second membre \var{SMBR}.\\
\var{SMBR} = \var{SMBR} + \var{W4}
\end{itemize}

A l'issue de cette \'etape, seule la partie extradiagonale de la diffusion prise
enti\`erement explicite~:
 $$\sum\limits_{m\in
Vois(l)}\left[\ \tens{E}^n\,\grad{R}^{\,n}_{ij} \right]_{\,lm}\,.\,\vect{n}_{\,lm}S_{\,lm}$$ a \'et\'e calcul\'ee.\\

\item Calcul de la partie diagonale du terme de diffusion\footnote{Seule la
composante normale  du  gradient de $R_{ij}$ aux faces sera implicite.} :\\
Comme on l'a d\'eja vu, une partie de cette contribution sera trait\'ee en
implicite (celle relative \`a la composante normale du gradient) et donc
ajout\'ee au second membre par \fort{bilsc2} ; l'autre
partie sera explicite et prise en compte dans $f_s^{\,exp}$.
\begin{itemize}
\item [$\star$] On effectue une boucle d'indice \var{IEL} sur les cellules
$\Omega_l$ de centre $L$ :
\begin{itemize}
\item [$\Rightarrow$] $\displaystyle \var{TRRIJ }= \frac{1}{2} (R^{\,n}_{ii})_L $
\item [$\Rightarrow$] $\displaystyle \var{CSTRIJ} = \rho^n_L \ C_S \ \frac{(R^{\,n}_{ii})_L}{2\,\varepsilon^n_L}$
\item [$\Rightarrow$] $\displaystyle \var{W4(IEL)} = \rho^n_L \ C_S \
\frac{(R^{\,n}_{ii})_L}{2\,\varepsilon^n_L} \ (R^{\,n}_{11})_L$
\item [$\Rightarrow$] $\displaystyle \var{W5(IEL)} = \rho^n_L \ C_S \ \frac{(R^{\,n}_{ii})_L}{2\,\varepsilon^n_L}\ (R^n_{22})_L$
\item [$\Rightarrow$] $\displaystyle \var{W6(IEL)} = \rho^n_L \ C_S \ \frac{(R^{\,n}_{ii})_L}{2\,\varepsilon^n_L} \ (R^n_{33})_L$
\end{itemize}

%\item Traitement du parall\'elisme et de la p\'eriodicit\'e.

\item [$\star$] On effectue une boucle d'indice \var{IFAC} sur les faces
purement internes $lm$ pour remplir le tableau \var{VISCF} :
\begin{itemize}
\item [$\Rightarrow$] Identification des cellules $\Omega_l$ et $\Omega_m$ de
centre respectif $L$ (variable \var{II}) et $M$ (variable \var{JJ}), se trouvant de chaque c\^ot\'e de la face
$lm$\footnote{La normale \`a la face est orient\'ee de L vers M.}.
\item [$\Rightarrow$] Calcul du carr\'e de la surface de la face. La valeur est
stock\'ee dans le tableau \var{SURFN2}.
\item [$\Rightarrow$] Interpolation du gradient de $R^{\,n}_{ij}$ \`a la face
$lm$ (gradient facette $\left[\grad{R}^{\,n}_{ij}\right]_{\,lm}$) :
\begin{equation}\notag
\left\{\begin{array}{ll}
\var{GRDPX} &= \displaystyle \frac{1}{2} \left(\var{W1(II)} + \var{W1(JJ)}
\right) \\
&\\
\var{GRDPY} &= \displaystyle \frac{1}{2} \left(\var{W2(II)} + \var{W2(JJ)} \right) \\
&\\
\var{GRDPZ} &= \displaystyle \frac{1}{2} \left(\var{W3(II)} + \var{W3(JJ)} \right)
\end{array}\right.
\end{equation}
\item [$\Rightarrow$] Calcul du gradient de $R^{\,n}_{ij}$ normal \`a la face
$lm$, $\left[\grad{R}^{\,n}_{ij}\right]_{\,lm}.\vect{n}_{\,lm}\,S_{\,lm}$.\\

$\displaystyle \var{GRDSN} =  \var{GRDPX} \ \var{SURFAC(1,IFAC)} + \var{GRDPY} \ \var{SURFAC(2,IFAC)} +  \var{GRDPZ} \ \var{SURFAC(3,IFAC)}$
$\var{SURFAC}$ est le vecteur surface de la face \var{IFAC}.


\item [$\Rightarrow$] calcul de
 $\left[\grad{R^{\,n}_{ij}} - (\grad
R^{\,n}_{ij}\,.\,\vect{n}_{\,lm})\vect{n}_{\,lm}\right]$, les vecteurs \'etant
calcul\'es \`a la face $lm$ :
\begin{equation}\notag
\left\{\begin{array}{lll}
&\displaystyle \var{GRDPX} &= \var{GRDPX} - \displaystyle\frac{\var{GRDSN}}{\var{SURFN2}} \ \var{SURFAC(1,IFAC)}\\
&&\\
&\displaystyle \var{GRDPY} &= \var{GRDPY} - \displaystyle\frac{\var{GRDSN}}{\var{SURFN2}} \ \var{SURFAC(2,IFAC)} \\
&&\\
&\displaystyle \var{GRDPZ} &= \var{GRDPZ} - \displaystyle\frac{\var{GRDSN}}{\var{SURFN2}} \ \var{SURFAC(3,IFAC)}
\end{array}\right.
\end{equation}
\item [$\Rightarrow$] finalisation du calcul de l'expression totalement
explicite
 $$\left[ \tens{D}^n\,\left( \grad{R^{\,n}_{ij}} - (\grad R^{\,n}_{ij}\,.\,\vect{n}_{\,lm})\,\vect{n}_{\,lm}\right) \right]\,.\,\vect{n}_{\,lm}$$
\begin{equation}\notag
\begin{array} {ll}
\displaystyle \var{VISCF} = &
 \displaystyle\frac{1}{2} (\ \var{W4(II)} +\ \var{W4(JJ)}) \ \var{GRDPX} \
\var{SURFAC(1,IFAC)})\ + \\
&\\
&  \displaystyle\frac{1}{2} (\ \var{W5(II)} +\ \var{W5(JJ)}) \ \var{GRDPY} \
\var{SURFAC(2,IFAC)})\ + \\
&\\
&  \displaystyle\frac{1}{2} (\ \var{W6(II)} +\ \var{W6(JJ)}) \ \var{GRDPZ} \ \var{SURFAC(3,IFAC)})
\end{array}
\end{equation}
\end{itemize}

\item [$\star$] Mise \`a z\'ero du tableau \var{VISCB}.

\item [$\star$] Appel de \fort{divmas} pour calculer la divergence de~:
 $$\tens{D}^{\,n}\,\left( \grad{R^{\,n}_{ij}} - (\grad R^{\,n}_{ij}\,.\,\vect{n}_{\,lm})\vect{n}_{\,lm}\right)$$ d\'efini aux faces dans \var{VISCF} et \var{VISCB}.

Le r\'esultat est stock\'e dans le tableau \var{W1}.\\
Ajout au second membre \var{SMBR}.\\
$\var{SMBR} = \var{SMBR} + \var{W1}$
\end{itemize}
\item Calcul de la viscosit\'e orthotrope $\gamma^n_{\,lm}$ \`a la face $lm$ de la variable principale
$R^{\,n}_{ij}$\\
Ce calcul permet au sous-programme \fort{codits} de compl\'eter le second membre
\var{SMBR} par :
\begin{equation}
\begin{array} {ll}
& \sum\limits_{m\in Vois(l)}
\mu^n_{\,lm}\,\left(\grad{R}^{\,n}_{ij}\,.\,\vect{n}_{\,lm}\right)S_{\,lm}
 + \sum\limits_{m\in Vois(l)} \left(\grad{R}^{\,n}_{ij}
\,.\,\vect{n}_{\,lm}\right)\left[\tens{D}^{\,n}\,\vect{n}_{\,lm}\right]_{\,lm}\,.\,\vect{n}_{\,lm}\
S_{\,lm}\\
& = \sum\limits_{m\in Vois(l)}(\,\mu^n_{\,lm}\, + \,\gamma^n_{\,lm}\,)\,\left(\grad{R}^{\,n}_{ij}\,.\,\vect{n}_{\,lm}\right)S_{\,lm}
\end{array}
\end{equation}
sans pr\'eciser la nature de la face $lm$, {\it via} l'appel \`a \fort{bilsc2}  et de disposer de la quantit\'e
$(\mu^n_{\,lm}\, + \gamma^n_{\,lm})$ pour construire sa
matrice simplifi\'ee.\\
\begin{itemize}
\item [$\star$] On effectue une boucle d'indice \var{IEL} sur les cellules
$\Omega_l$ :
\begin{itemize}
\item [$\Rightarrow$] $\displaystyle \var{TRRIJ }= \frac{1}{2} (R^{\,n}_{ii})_L $
\item [$\Rightarrow$] $\displaystyle \var{RCSTE} = \rho^n_L \ C_S \ \frac{ (R^{\,n}_{ii})_L}{2\,\varepsilon^n_L} $
\item [$\Rightarrow$] $\displaystyle \var{W1(IEL)} = \mu^n +\rho^n_L \ C_S \ \frac{
(R^{\,n}_{ii})_L}{2\,\varepsilon^n_L}\ (R^n_{11})_L$
\item [$\Rightarrow$] $\displaystyle \var{W2(IEL)} = \mu^n + \rho^n_L \ C_S \ \frac{ (R^{\,n}_{ii})_L}{2\,\varepsilon^n_L}\ (R^n_{22})_L$
\item [$\Rightarrow$] $\displaystyle \var{W3(IEL)} = \mu^n + \rho^n_L \ C_S \ \frac{ (R^{\,n}_{ii})_L}{2\,\varepsilon^n_L}\ (R^n_{33})_L$
\end{itemize}

\item [$\star$] Appel de \fort{visort} pour calculer la viscosit\'e orthotrope
\footnote{Comme dans le sous-programme \fort{viscfa}, on multiplie la viscosit\'e par
$\displaystyle \frac{S_{\,lm}}{\overline{L'M'}}$, o\`u $S_{\,lm}$ et
$\overline{L'M'}$ repr\'esentent respectivement la surface de la face $lm$ et la
mesure alg\'ebrique du segment reliant les projections des centres des cellules
voisines sur la normale \`a la face. On garde dans ce sous-programme  la possibilit\'e d'interpoler la viscosit\'e aux cellules lin\'eairement ou d'utiliser une moyenne harmonique. La viscosit\'e au bord est celle de la cellule de bord correspondante.}
$ \gamma^n_{\,lm} = (\tens{D}^{\,n}\,\vect{n}_{\,lm}).\vect{n}_{\,lm}$ aux faces $lm$

Le r\'esultat est stock\'e dans les tableaux \var{VISCF} et \var{VISCB}.
\end{itemize}

\item appel de \fort{codits} pour la r\'esolution de l'\'equation de
convection/diffusion/termes sources de la variable $R_{ij}$. Le terme source est
\var{SMBR}, la viscosit\'e \var{VISCF} aux faces purement internes (
resp. \var{VISCB} aux faces de bord) et \var{FLUMAS} le flux de masse interne
 ( resp. \var{FLUMAB} flux de masse au bord). Le r\'esultat est la variable $R_{ij}$ au temps
$n+1$, donc $R^{\,n+1}_{ij}$. Elle est stock\'ee dans le tableau \var{RTP} des
variables mises \`a jour.

\end{itemize}

\etape{Appel de \fort{reseps} pour la r\'esolution de la variable $\varepsilon$}

On d\'ecrit ci-dessous le sous-programme \fort{reseps}, les commentaires du sous-programme \fort{resrij} ne seront pas r\'ep\'et\'es puisque les deux sous-programmes ne diff\`erent que par leurs termes sources.

\begin{itemize}
\item Initialisation \`a z\'ero de \var{SMBR} et \var{ROVSDT}.

\item{Lecture et prise en compte des termes sources utilisateur pour la variable $\varepsilon$ :}

Appel de \fort{ustsri} pour charger les termes sources utilisateurs. Ils sont
stock\'es dans les tableaux suivants :\\
pour la cellule $\Omega_l$ repr\'esent\'ee par $\var{IEL}$ de centre $L$, on a :
\begin{equation}\notag
\left\{\begin{array}{lll}
&\var{ROVSDT(IEL)}&= |\Omega_l| \ \alpha_{\varepsilon}\\
&\var{SMBR(IEL)}&=|\Omega_l| \ \beta_{\varepsilon}\\
\end{array}\right.
\end{equation}
On affecte alors les valeurs ad\'equates au second membre \var{SMBR} et \`a la
diagonale \var{ROVSDT} comme suit :
\begin{equation}\notag
\left\{\begin{array}{lll}
&\var{SMBR(IEL)} &= \var{SMBR(IEL)} +\ |\Omega_l| \ \alpha_{\,\varepsilon} \
\varepsilon^n_L \\
&\var{ROVSDT(IEL)}&= \text{max }(-\ |\Omega_l| \ \alpha_{\,\varepsilon},0)\\
\end{array}\right.
\end{equation}

\item{Calcul du terme source de masse si $\Gamma_L > 0$ :
\begin{equation}\notag
\left\{\begin{array}{lll}
&\displaystyle \var{SMBR(IEL)} = \var{SMBR(IEL)} + |\Omega_l| \ \Gamma_L \
(\varepsilon^{\,in}_L -\varepsilon^n_L) \\
&\displaystyle \var{ROVSDT(IEL)}= \var{ROVSDT(IEL)} + |\Omega_l| \ \Gamma_L
\end{array}\right.
\end{equation}
\item Calcul du terme d'accumulation de masse et du terme instationnaire \\
On stocke $\displaystyle \var{W1}= \int_{\Omega_l}\dive\,(\rho\,\vect{u})\,d\Omega$
calcul\'e par \fort{divmas} \`a l'aide des flux de masse internes et aux bords.\\
On incr\'emente ensuite \var{SMBR} et \var{ROVSDT}.
\begin{equation}\notag
\left\{\begin{array}{lll}
&\var{SMBR(IEL)} &= \var{SMBR(IEL)} + \var{ICONV}\ \varepsilon^n_L\,(\displaystyle
\int_{\Omega_l}\dive\,(\rho\,\vect{u})\ d\Omega) \\
&\var{ROVSDT(IEL)}& = \var{ROVSDT(IEL)} +  \var{ISTAT}\,\displaystyle
\frac{\rho^n_L \ |\Omega_l|}{\Delta t^n} -  \var{ICONV}\ (\displaystyle
\int_{\Omega_l}\dive\,(\rho\,\vect{u})\ d\Omega) \\
\end{array}\right.
\end{equation}

\item Traitement du terme de production
 $\displaystyle \rho\,C_{\varepsilon_1}\,\frac{\varepsilon}{k}\,\mathcal{P}$
 et du terme de dissipation $-\,\displaystyle \rho\,C_{\varepsilon_2}\,\frac{\varepsilon}{k}\,\varepsilon$ \\
pour cela, on effectue une boucle d'indice \var{IEL} sur les cellules $\Omega_l$
de centre $L$ :
\begin{itemize}
\item [$\Rightarrow$] $\displaystyle \var{TRPROD}= \frac{1}{2} (\mathcal{P}^n_{ii})_L = \frac{1}{2} \left[ \var{PRODUC(1,IEL)} +  \var{PRODUC(2,IEL)} +  \var{PRODUC(3,IEL)} \right] $
\item [$\Rightarrow$] $\displaystyle \var{TRRIJ }= \frac{1}{2} (R^n_{ii})_L $
\item [$\Rightarrow$] $\displaystyle \var{SMBR(IEL)} = \var{SMBR(IEL)} + \rho^n_L
|\Omega_l| \left[ -C_{\varepsilon_2} \ \frac{2\,(\varepsilon^n_L)^2}{(R^n_{ii})_L} + C_{\varepsilon_1} \ \frac{\varepsilon^n_L}{(R^n_{ii})_L}\ (\mathcal{P}^n_{ii})_L \right] $
\item [$\Rightarrow$] $\displaystyle \var{ROVSDT(IEL)} = \var{ROVSDT(IEL)} + C_{\varepsilon_2} \ \rho^n_L \ |\Omega_l| \ \frac{2\,\varepsilon^n_L}{(R^n_{ii})_L}$
\end{itemize}

\item Appel de \fort{rijthe} pour le calcul des termes de gravit\'e $\mathcal{G}^n_{\varepsilon}$ et ajout dans \var{SMBR}.

$ \var{SMBR} = \var{SMBR} + \mathcal{G}^n_{\varepsilon}$\\
Ce calcul n'a lieu que si $\var{IGRARI()} = 1$.

\item Calcul de la diffusion de $\varepsilon$ \\
 Le terme $\dive \left[\mu\, \grad(\varepsilon) + \tens{A'}\,\grad(\varepsilon)
\right]$ est calcul\'e exactement de la m\^eme mani\`ere que pour les tensions
de Reynolds $R_{ij}$ en rempla\c cant $\tens{A}$ par $\tens{A'}$.

\item Appel de \fort{codits} pour la r\'esolution de l'\'equation de
convection/diffusion/termes sources de la variable principale $\varepsilon$. Le
r\'esultat $\varepsilon^{\,n+1}$ est stock\'e dans le tableau \var{RTP} des
variables mises \`a jour.
}
\end{itemize}

\etape{clippings finaux}
On passe enfin dans le sous-programme  \fort{clprij} pour faire un clipping \'eventuel
des variables $R^{\,n+1}_{ij}$ et $\varepsilon^{\,n+1}$. Le sous-programme
\fort{clprij} est appel\'e\footnote{L'option
$\var{ICLIP} = 1$ consiste en un clipping minimal des variables $R_{ii}$ et
$\varepsilon$ en prenant la valeur absolue de ces variables puisqu'elles ne
peuvent \^etre que positives.} avec $\var{ICLIP} = 2$ . Cette option
\footnote{Quand la valeur des grandeurs $R_{ii}$ ou $\varepsilon$ est
n\'egative, on la remplace par le minimum entre sa valeur absolue et (1,1)
fois la valeur obtenue au pas de temps pr\'ec\'edent.} contient l'option $\var{ICLIP} = 1$  et permet de v\'erifier l'in\'egalit\'e de Cauchy-Schwarz sur les grandeurs extra-diagonales du tenseur $\tens{R}$ (pour $i \neq j$, $|R_{ij}|^2 \le R_{ii} R_{jj}$).


%%%%%%%%%%%%%%%%%%%%%%%%%%%%%%%%%%
%%%%%%%%%%%%%%%%%%%%%%%%%%%%%%%%%%
\section{Points \`a traiter}
%%%%%%%%%%%%%%%%%%%%%%%%%%%%%%%%%%
%%%%%%%%%%%%%%%%%%%%%%%%%%%%%%%%%%
Sauf mention explicite, $\phi$ repr\'esentera une tension de Reynolds ou la dissipation turbulente ($\phi = R_{ij} \ \text{ou} \ \varepsilon$).

\begin{itemize}
\item {La vitesse utilis\'ee pour le calcul de la production est explicite. Est-ce qu'une implicitation peut am\'eliorer la pr\'ecision temporelle de $\phi$ \footnote{Cette remarque peut \^etre g\'en\'eralis\'ee. En effet, peut-on envisager d'actualiser les variables d\'ej\`a r\'esolues (sans r\'eactualiser les variables turbulentes apr\`es leur r\'esolution)? Ceci obligerait \`a modifier les sous-programmes tels que \fort{condli} qui sont appel\'es au d\'ebut de la boucle en temps.} ?}
\item {Dans quelle mesure le terme d'\'echo de paroi est-il valide ? En effet, ce terme est remis en question par certains auteurs.}
\item {On peut envisager la r\'esolution d'un syst\`eme hyperbolique pour les
tensions de Reynolds afin d'introduire un couplage avec le champ de vitesse.}
\item {Le flux au bord \var{VISCB} est annul\'e dans le sous-programme
\fort{vectds}. Peut-on envisager de mettre au bord la valeur de la variable
concern\'ee \`a la cellule de bord correspondant? De m\^eme, il faudrait se
pencher sur les hypoth\`eses sous-jacentes \`a l'annulation des contributions
aux bords de \var{VISCB} lors du calcul de : $$\left[ \tens{D}^n\,\left( \grad{R^{\,n}_{ij}} - (\grad R^{\,n}_{ij}\,.\,\vect{n}_{\,lm})\,\vect{n}_{\,lm}\right) \right]\,.\,\vect{n}_{\,lm}.$$}
\item {Un probl\`eme de pond\'eration appara\^\i t plus g\'en\'eralement. Si on prend la partie explicite de $\tens{D}\,\grad(\phi)$, la pond\'eration aux faces internes utilise le coefficient $\displaystyle\frac{1}{2}$ avec pond\'eration s\'epar\'ee de $\tens{D}$ et $\grad(\phi)$, alors que pour $\tens{E}\,\grad(\phi)$, on calcule d'abord ce terme aux cellules pour ensuite l'interpoler lin\'eairement aux faces \footnote{Cette interpolation se fait dans \fort{vectds} avec des coefficients de pond\'eration aux faces.}. Ceci donne donc deux types d'interpolations pour des termes de m\^eme nature.}
\item {On laisse la possibilit\'e dans \fort{visort} d'utiliser une moyenne
harmonique aux faces. Est-ce que ceci est valable puisque les interpolations
utilis\'ees lors du calcul de la partie explicite de $\tens{A}\,\grad{\phi}$
sont des moyennes arithm\'etiques ?}
\item {Les techniques adopt\'ees lors du clipping sont \`a revoir.}
\item {On utilise dans le cadre du mod\`ele $\displaystyle R_{ij}-\varepsilon $ une semi-implicitation de termes comme $\displaystyle \phi_{ij,1}$ ou $\displaystyle -\rho\,C_{\varepsilon_2}\,\frac{\varepsilon}{k}\,\varepsilon$. On peut envisager le m\^eme type d'implicitation dans \fort{turbke} m\^eme en pr\'esence du couplage $\displaystyle k-\varepsilon$.}
\item L'adoption d'une r\'esolution d\'ecoupl\'ee fait perdre l'invariance par rotation.
\item La formulation et l'implantation des conditions aux limites de paroi
devront \^etre v\'erifi\'ees. En effet, il semblerait que, dans certains cas, des ph\'enom\`enes
``oscillatoires'' apparaissent, sans qu'il soit ais\'e d'en d\'eterminer la cause.
\item L'implicitation partielle (du fait de la r\'esolution d\'ecoupl\'ee) des
conditions aux limites conduit souvent \`a des calculs instables. Il
conviendrait d'en conna\^\i tre la raison. L'implicitation partielle avait
\'et\'e mise en \oe uvre afin de tenter d'utiliser un pas de temps plus grand
dans le cas de jets axisym\'etriques en particulier.

\end{itemize}

%                      Code_Saturne version 1.3
%                      ------------------------
%
%     This file is part of the Code_Saturne Kernel, element of the
%     Code_Saturne CFD tool.
%
%     Copyright (C) 1998-2007 EDF S.A., France
%
%     contact: saturne-support@edf.fr
%
%     The Code_Saturne Kernel is free software; you can redistribute it
%     and/or modify it under the terms of the GNU General Public License
%     as published by the Free Software Foundation; either version 2 of
%     the License, or (at your option) any later version.
%
%     The Code_Saturne Kernel is distributed in the hope that it will be
%     useful, but WITHOUT ANY WARRANTY; without even the implied warranty
%     of MERCHANTABILITY or FITNESS FOR A PARTICULAR PURPOSE.  See the
%     GNU General Public License for more details.
%
%     You should have received a copy of the GNU General Public License
%     along with the Code_Saturne Kernel; if not, write to the
%     Free Software Foundation, Inc.,
%     51 Franklin St, Fifth Floor,
%     Boston, MA  02110-1301  USA
%
%-----------------------------------------------------------------------
%
\programme{vortex}
%
\vspace{1cm}
%%%%%%%%%%%%%%%%%%%%%%%%%%%%%%%%%%
%%%%%%%%%%%%%%%%%%%%%%%%%%%%%%%%%%
\section{Fonction}
%%%%%%%%%%%%%%%%%%%%%%%%%%%%%%%%%%
%%%%%%%%%%%%%%%%%%%%%%%%%%%%%%%%%%
Ce sous-programme est d�di� � la g�n�ration des conditions d'entr�e
turbulente utilis�es en LES.


La m�thode des vortex est bas�e sur une approche de tourbillons
ponctuels. L'id�e de la m�thode consiste � injecter des tourbillons 2D dans le
plan d'entr�e du calcul, puis � calculer le champ de vitesse induit par ces
tourbillons au centre des faces d'entr�e.

%                      Code_Saturne version 1.3
%                      ------------------------
%
%     This file is part of the Code_Saturne Kernel, element of the
%     Code_Saturne CFD tool.
% 
%     Copyright (C) 1998-2007 EDF S.A., France
%
%     contact: saturne-support@edf.fr
% 
%     The Code_Saturne Kernel is free software; you can redistribute it
%     and/or modify it under the terms of the GNU General Public License
%     as published by the Free Software Foundation; either version 2 of
%     the License, or (at your option) any later version.
% 
%     The Code_Saturne Kernel is distributed in the hope that it will be
%     useful, but WITHOUT ANY WARRANTY; without even the implied warranty
%     of MERCHANTABILITY or FITNESS FOR A PARTICULAR PURPOSE.  See the
%     GNU General Public License for more details.
% 
%     You should have received a copy of the GNU General Public License
%     along with the Code_Saturne Kernel; if not, write to the
%     Free Software Foundation, Inc.,
%     51 Franklin St, Fifth Floor,
%     Boston, MA  02110-1301  USA
%
%-----------------------------------------------------------------------
%
%%%%%%%%%%%%%%%%%%%%%%%%%%%%%%%%%%
%%%%%%%%%%%%%%%%%%%%%%%%%%%%%%%%%%
\section{Discr\'etisation}
%%%%%%%%%%%%%%%%%%%%%%%%%%%%%%%%%%
%%%%%%%%%%%%%%%%%%%%%%%%%%%%%%%%%%

Le terme convectif en $\dive(\underline{u} \otimes \rho\,\underline{u})$
introduit une non lin\'earit\'e et un couplage des composantes de la vitesse
$\vect{u}$ dans l'�quation (\ref{Base_Preduv_eqqdm}). Une lin\'earisation et un d\'ecouplage
des trois composantes de la 
vitesse sont r\'ealis\'es lors de la discr\'etisation de cette \'etape de
pr\'ediction.\\
En effet, soit :
\begin{equation}
\vect{\widetilde{u}}= \vect{u}^n + \delta \vect{u} 
\end{equation}
La contribution exacte du terme convectif \`a prendre en compte dans cette
\'etape de pr\'ediction serait :\\
\begin{equation}\label{Base_Preduv_Conv_exact}
\begin{array}{ll}
\dive(\vect{\widetilde{u}} \otimes \rho\,\vect{\widetilde{u}}) =
\dive(\vect{u}^{n} \otimes \rho\,\vect{u}^{n}) + \dive(\delta \vect{u} \otimes
\rho\,\vect{u}^{n}) +  \underbrace { \dive(\vect{u}^{n} \otimes
\rho\,\delta \vect{u})}_{\text {terme couplant lin\'eaire}} +  \underbrace { \dive(\delta \vect{u} \otimes
\rho\,\delta \vect{u})}_{\text {terme couplant et non lin\'eaire}}\\
\end{array} 
\end{equation}
Les deux derniers termes de l'expression (\ref{Base_Preduv_Conv_exact}) sont {\em a priori} n�glig�s
de mani�re � obtenir un syst\`eme en vitesse qui soit d\'ecoupl\'e et donc,
�viter l'inversion d'une matrice pouvant \^etre de tr\`es grande taille. Ces
deux termes peuvent n�anmoins �tre pris en compte de mani�re plus ou moins
approch�e par extrapolation explicite du flux de masse en $n+\theta_F$ (pour le
terme couplant lin�aire provenant de la convection de $\vect{u}^{n}$ par $\delta
\vect{u}$) et utilisation d'un point-fixe par sous it�ration sur le sous
programme \fort{navsto} (pour le terme non-lin�aire, en sp�cifiant $\var{NTERUP}>1$).

L'�quation (\ref{Base_Preduv_eqqdm}) est discr�tis�e au temps $n+\theta$ � l'aide d'un
$\theta$-sch�ma, et le tenseur des pertes de charges d�compos� en une partie
diagonale $\tens{K}_{d}$ et une extradiagonale $\tens{K}_{e}$ (soit
 $\tens{K}_{pdc}=\tens{K}_{d}+\tens{K}_{e}$).\\
$\bullet$ La pression est suppos�e connue en $n-1+\theta$ (d�calage temporel
pression-vitesse) et le gradient naturellement calcul� � cet instant.\\ 
$\bullet$ Les termes sources de viscosit� secondaire, de gradient transpos\'e,
ceux provenant du mod�le de turbulence\footnote{except� $\dive (\mu_t\ (\ggrad
\underline {u}))$}, $\rho\,\tens{K}_{\,e}\ \underline{u}$, $(\rho -\rho_0)
\underline {g}$ ainsi que $\underline{T}_{s}^{\,exp}$ et
$\Gamma\,\underline{u}_{\,i}$ sont pris de mani�re explicite au temps $n$, ou
extrapol�s suivant le sch�ma en temps choisi pour les propri�t�s physique et les
termes sources.\\ 
$\bullet$ Les termes sources $\underline{u}\,\,\dive (\rho\,\underline {u})$,
$\Gamma\,\,\underline{u}$, $T_{s}^{\,imp}\,\,\underline{u}$ et
$-\rho\,\tens{K}_{\,d}\,\,\underline{u}$ sont implicit�s est calcul�s �
l'instant $n+\theta$.\\ 
$\bullet$ Le terme de diffusion $\dive (\mu_{\,tot}\,\ggrad \underline{u})$ est
implicit� : la vitesse est prise � l'instant $n+\theta$ et la viscosit�
explicit�e ou extrapol�e.\\ 
$\bullet$ Enfin, le terme de convection en $\dive(\,\underline{u} \otimes
(\rho\underline{u})\,)$ est implicit� : la composante r�solue de la vitesse est
prise en $n+\theta$, et le flux de masse, explicit�, ou extrapol� en
$n+\theta_F$. 

Par souci de clart�, on suppose, en l'absence d'indication, que les propri�tes
physiques $\Phi$ ($\rho,\,\mu_{tot},\,...$) et le flux de masse
$(\rho\underline{u})$ sont pris respectivement aux instants $n+\theta_\Phi$ et
$n+\theta_F$, o� $\theta_\Phi$ et $\theta_F$ d�pendent des sch�mas en temps
sp�cifiquement utilis�s pour ces grandeurs\footnote{cf. \fort{introd}}. 

La discr�tisation temporelle de l'�quation (\ref{Base_Preduv_eqqdm}) s'�crit alors comme suit : 

\begin{equation}\label{Base_Preduv_eq_di1}
 \begin{array}{c}
\displaystyle \rho\,\ \frac{ \underline {\widetilde{u}}^{n+1} -\underline {u}^{n} }
{\Delta t} + \dive(\,\underline{\widetilde{u}}^{n+\theta} \otimes (\rho\underline{u})\,) -\dive
(\mu_{\,tot}\,\ggrad \underline{\widetilde{u}}^{n+\theta}) =
\\
\displaystyle
 - \grad p^{n-1+\theta} + \dive (\rho\,\underline {u})\,\underline{\widetilde{u}}^{n+\theta} +(\Gamma\,\underline{u}_{\,i})^{n+\theta_S}-\Gamma^n\,\,\underline{\widetilde{u}}^{n+\theta}
\\
\begin{array}{c}
\displaystyle
- \rho\,\tens{K}_{\,d}^{n}\,\,\underline{\widetilde{u}}^{n+\theta} - (\rho\,\tens{K}_{\,e}\ \underline{u})^{n+\theta_S} + (\underline{T}_{s}^{\,exp})^{\,n+\theta_S} + T_{s}^{\,imp}\,\,\underline{\widetilde{u}}^{n+\theta}
\\
\displaystyle
+[\dive (\mu_{\,tot}\,^t\ggrad \underline {u})]^{n+\theta_S}-\frac {2} {3}[\,\grad (\mu_{\,tot}\,\dive \underline {u})]^{n+\theta_S} + (\rho -\rho_0) \underline {g}
 - (\underline{turb})^{n+\theta_S}
\end{array}
\end{array}
\end{equation}
o\`u, par souci de simplification, on a pos\'e :
\begin{equation}
\mu_{\,tot}=
\begin{cases}
\mu+\mu_t & \text{pour les mod�les � viscosit� turbulente ou en LES}, \\
\mu & \text{pour les mod�les au second ordre ou en laminaire}
\end{cases} \ 
\end{equation}
\\
et :
\begin{equation}
\underline{turb}^{n}=
\begin{cases}
\displaystyle\frac {2}{3}\grad (\rho^{n}\,k^{n}) & \text{pour les mod�les � viscosit� turbulente}, \\
\dive(\rho^{n}\,\tens{R}^n) & \text{pour les mod�les au second ordre},\\
0 & \text{en laminaire ou en LES}\\
\end{cases}
\end{equation}
Par analogie avec l'�criture du $\theta$-sch�ma pour une variable scalaire, $\,
\underline {\widetilde{u}}^{n+\theta}$ est interpol�e � partir de la vitesse
pr�dite $\underline {\widetilde{u}}^{n+1}$ de la mani\`ere suivante\footnote{si
$\theta=1/2$, ou qu'une extrapolation est utilis�e, l'ordre 2 n'est obtenu que si
le pas de temps $\Delta t$ est uniforme en temps et en espace.}~: 
\begin{equation}
\underline {\widetilde{u}}^{n+\theta}=\theta\, \underline
{\widetilde{u}}^{n+1}+(1-\theta)\, \underline {u}^{n}\\ 
\end{equation}
Avec :
\begin{equation}
\left\{
\begin{array}{ll}
\theta = 1   & \text{Pour un sch\'ema de type Euler implicite d'ordre 1.}\\
\theta = 1/2 & \text{Pour un sch\'ema de type Cranck-Nicolson d'ordre 2.}\\
\end{array}
\right.
\end{equation}

L'�quation (\ref{Base_Preduv_eq_di1}) est alors r��crite sous la forme :

\begin{equation}\label{Base_Preduv_eq_di2}
\begin{array}{c}
\displaystyle \underbrace{\left(\frac{\rho}{\Delta t} -\theta \,\dive (\rho\,\underline {u}) +\theta \,\, \Gamma^n +
\theta \,\, \rho\,\tens{K}_{\,d}^n-\theta \,T_s^{\,imp} \right)}_{\displaystyle f_s^{imp}}\, (\underline {\,\widetilde{u}}^{n+1} -\underline {u}^{n})
\\
 +\, \theta\, \dive(\underline {\widetilde{u}}^{n+1} \otimes (\rho\underline{u}))-\, \theta\,\dive (\mu_{\,tot}\,\ggrad \underline {\widetilde{u}}^{n+1}) =
\\
-\,(1-\theta)\, \dive(\underline {u}^{n} \otimes (\rho\underline{u})) +\,(1-\theta)\,\dive (\mu_{\,tot}\,\ggrad \underline {u}^{n})
\\
f_s^{exp}\left\{
\begin{array}{c}
\displaystyle 
- \grad p^{n-1+\theta} + \dive (\rho\,\underline {u})\,\underline{u}^{n} +\,(\,\Gamma^{n}\,\underline{u}_{\,i}\,)^{n+\theta_S}- \Gamma^n\,\,\underline{u}^{n}
\\
\displaystyle
-(\,\rho\,\tens{K}_{\,e}\ \underline{u}\,)^{n+\theta_S} -\rho\,\tens{K}_{\,d}^n\ \underline{u}^{n}+ (\underline{T}_{s}^{\,exp})^{\,n+\theta_S} + T_s^{\,imp}\,\,\underline {u}^{n} 
\\
\displaystyle
+[\dive (\mu_{\,tot}\,^t\ggrad \underline {u}\,)]^{n+\theta_S}-\frac {2} {3}[\,\grad (\mu_{\,tot}\,\dive \underline {u}\,)]^{n+\theta_S} + (\rho -\rho_0) \underline {g}-(\underline{turb})^{n+\theta_S}
\end{array}
\right.
\end{array}
\end{equation}

d'o� l'�quation r�solue par le sous-programme \fort{codits} :
\begin{equation}\begin{array}{c}
\displaystyle
f_s^{\,imp}(\underline {\widetilde{u}}^{n+1}-\underline {u}^{n}) + \theta\, \dive(\underline{\widetilde{u}}^{n+1} \otimes (\rho
\underline{u})) - \theta\,\dive (\,\mu_{\,tot}\,\ggrad \underline{\widetilde{u}}^{n+1}) = 
\\\\
\displaystyle
-(1-\theta)\,\dive(\underline{u}^{n} \otimes (\rho \underline{u}))+(1-\theta)\,\dive (\,\mu_{\,tot}\,\ggrad \underline{u}^{n})
+ \underline{f}_{\,s}^{\,exp}
\end{array}
\end{equation}
La m\'ethode de discr\'etisation spatiale est d\'evelopp\'ee dans le sous-programme \fort{codits}.\\



\minititre{Remarques :}
{\tiny$\blacksquare$} Dans le cas standard sans extrapolation, le terme
$-\,T_s^{\,imp}$ n'est ajout� � $f_s^{\,imp}$ que s'il est positif afin de ne
pas affaiblir la dominance de la diagonale de la matrice � inverser.\\ 
{\tiny$\blacksquare$} Si une extrapolation est utilis�e, par contre,
$\,T_s^{\,imp}$ est ajout� � $f_s^{\,imp}$ quel que soit son signe. En effet, l'id�e intuitive qui
consiste � prendre~: 
\begin{equation}
\begin{cases}
\displaystyle
(\underline{T}_{s}^{\,exp} + T_{s}^{\,imp}\,\underline {u})^{\,n+\theta_S} &
\text{si } T_{s}^{\,imp} > 0\\ 
\displaystyle
(\underline{T}_{s}^{\,exp})^{\,n+\theta_S} + T_{s}^{\,imp}\,\underline{u}^{n+\theta} &\text{sinon}\\
\end{cases}
\end{equation} 
aboutit � une incoh�rence dans le traitement si $T_s^{imp}$ change de signe
entre deux pas de temps.\\ 
{\tiny$\blacksquare$} la partie diagonale $\tens{K}_{\,d}$ du terme
de perte de charge est utilis�e dans $f_s^{\,imp}$. En fait, pour \^etre rigoureux,
il faudrait ne retenir que les contributions positives (point signal\'e dans le
sous-programme utilisateur associ\'e \fort{uskpdc}). Cette prise en compte sera \`a am\'eliorer.\\
{\tiny$\blacksquare$} Le terme $\theta\,\Gamma^{n}-\theta\,\dive
(\rho\,\underline {u})$ ne pose pas de probl�me pour la 
dominance de la diagonale de la matrice car il est exactement compens� par le
terme de convection (cf. \fort{covofi}). 


%                      Code_Saturne version 1.3
%                      ------------------------
%
%     This file is part of the Code_Saturne Kernel, element of the
%     Code_Saturne CFD tool.
%
%     Copyright (C) 1998-2007 EDF S.A., France
%
%     contact: saturne-support@edf.fr
%
%     The Code_Saturne Kernel is free software; you can redistribute it
%     and/or modify it under the terms of the GNU General Public License
%     as published by the Free Software Foundation; either version 2 of
%     the License, or (at your option) any later version.
%
%     The Code_Saturne Kernel is distributed in the hope that it will be
%     useful, but WITHOUT ANY WARRANTY; without even the implied warranty
%     of MERCHANTABILITY or FITNESS FOR A PARTICULAR PURPOSE.  See the
%     GNU General Public License for more details.
%
%     You should have received a copy of the GNU General Public License
%     along with the Code_Saturne Kernel; if not, write to the
%     Free Software Foundation, Inc.,
%     51 Franklin St, Fifth Floor,
%     Boston, MA  02110-1301  USA
%
%-----------------------------------------------------------------------
%

%%%%%%%%%%%%%%%%%%%%%%%%%%%%%%%%%%
%%%%%%%%%%%%%%%%%%%%%%%%%%%%%%%%%%
\section{Mise en \oe uvre}
%%%%%%%%%%%%%%%%%%%%%%%%%%%%%%%%%%
%%%%%%%%%%%%%%%%%%%%%%%%%%%%%%%%%%
La num\'ero de la phase trait\'ee fait partie des arguments de \fort{turrij}. On
omettra volontairement de le pr\'eciser dans ce qui suit, on indiquera par $(\ )$ la
notion de tableau s'y rattachant.

\etape{Calcul des termes de production $\tens{\mathcal{P}}$}
\begin{itemize}
\item [$\star$] Initialisation \`a z\'ero du tableau \var{PRODUC} dimensionn\'e \`a $\var{NCEL}\times 6$.
\item [$\star$] On appelle trois fois \fort{grdcel} pour calculer les gradients des composantes de la vitesse $u$, $v$ et
$w$ prises au temps $n$.

Au final, on a :\\
$\displaystyle
\begin{array} {ll}
\var{PRODUC(1,IEL)} = & \displaystyle - 2 \left[ R_{11}^{\,n} \frac{\partial u^{\,n}} {\partial x} +R_{12}^{\,n} \frac{\partial u^{\,n}} {\partial y}+R_{13}^{\,n} \frac{\partial u^{\,n}} {\partial z} \right] \text{        (production de $R_{11}^{\,n}$)}\\
\var{PRODUC(2,IEL)} = & \displaystyle - 2 \left[ R_{12}^{\,n} \frac{\partial v^{\,n}} {\partial x} +R_{22}^{\,n} \frac{\partial v^{\,n}} {\partial y}+R_{23}^{\,n} \frac{\partial v^{\,n}} {\partial z} \right] \text{        (production de $R_{22}^{\,n}$)}\\
\var{PRODUC(3,IEL)} = & \displaystyle - 2 \left[ R_{13}^{\,n} \frac{\partial w^{\,n}} {\partial x} +R_{23}^{\,n} \frac{\partial w^{\,n}} {\partial y}+R_{33}^{\,n} \frac{\partial w^{\,n}} {\partial z} \right] \text{        (production de $R_{33}^{\,n}$)}\\
\var{PRODUC(4,IEL)} = & \displaystyle - \left[ R_{12}^{\,n} \frac{\partial u^{\,n}} {\partial x} +R_{22}^{\,n} \frac{\partial u^{\,n}} {\partial y}+R_{23}^{\,n} \frac{\partial u^{\,n}} {\partial z} \right] \\
& \displaystyle - \left[ R_{11}^{\,n} \frac{\partial v^{\,n}} {\partial x} +R_{12}^{\,n} \frac{\partial v^{\,n}} {\partial y}+R_{13}^{\,n} \frac{\partial v^{\,n}} {\partial z} \right] \text{        (production de $R_{12}^{\,n}$)} \\
\var{PRODUC(5,IEL)} = & \displaystyle - \left[ R_{13}^{\,n} \frac{\partial u^{\,n}} {\partial x} +R_{23}^{\,n} \frac{\partial u^{\,n}} {\partial y}+R_{33}^{\,n} \frac{\partial u^{\,n}} {\partial z} \right] \\
& \displaystyle - \left[ R_{11}^{\,n} \frac{\partial w^{\,n}} {\partial x} +R_{12}^{\,n} \frac{\partial w^{\,n}} {\partial y}+R_{13}^{\,n} \frac{\partial w^{\,n}} {\partial z} \right] \text{        (production de $R_{13}^{\,n}$)} \\
\var{PRODUC(6,IEL)} = & \displaystyle - \left[ R_{13}^{\,n} \frac{\partial v^{\,n}} {\partial x} +R_{23}^{\,n} \frac{\partial v^{\,n}} {\partial y}+R_{33}^{\,n} \frac{\partial v^{\,n}} {\partial z} \right] \\
& \displaystyle - \left[ R_{12}^{\,n} \frac{\partial w^{\,n}} {\partial x} +R_{22}^{\,n} \frac{\partial w^{\,n}} {\partial y}+R_{23}^{\,n} \frac{\partial w^{\,n}} {\partial z} \right]  \text{        (production de $R_{23}^{\,n}$)}
\end{array}
$
\end{itemize}

\etape{Calcul du gradient de la masse volumique $\rho^n$ prise au d\'ebut du pas
de temps courant\footnote{{\it i.e.} calcul\'ee \`a partir des
variables du pas de temps pr\'ec\'edent $n$ si n\'ecessaire.} $(n+1)$}
Ce calcul n'a lieu que si les termes de gravit\'e doivent \^etre pris en compte
($\var{IGRARI()} =1$).
\begin{itemize}
\item [$\star$] Appel de \fort{grdcel}  pour calculer le gradient de $\rho^n$
dans les trois directions de l'espace. Les conditions aux limites sur $\rho^n$
sont des conditions de Dirichlet puisque la valeur de $\rho^n$ aux faces de bord
$ik$ (variable \var{IFAC}) est connue et vaut $\rho_{\,b_{\,ik}}$. Pour \'ecrire les conditions aux limites
sous la forme habituelle, $$\rho_{\,b_{\,ik}} = \var{COEFA} + \var{COEFB}
\,\rho^n_{\,I'}$$ on pose alors $\var{COEFA}=
\var{PROPCE(IFAC,IPPROB(IROM(IPHAS)))}$ et $\var{COEFB} = \var{VISCB} = 0$.\\
$\var{PROPCE(1,IPPROB(IROM(IPHAS)))}$ (resp.$\var{VISCB}$) est utilis\'e en lieu
et place de l'habituel \var{COEFA} ($\var{COEFB}$), lors de l'appel \`a \fort{grdcel}.\\
On a donc :\\
$\displaystyle \var{GRAROX}= \frac{\partial \rho^n}{\partial x}\ $,$\displaystyle \ \var{GRAROY}= \frac{\partial
\rho^n}{\partial y}$ et $
\displaystyle \ \var{GRAROZ}= \frac{\partial \rho^n}{\partial z}\ $.

\end{itemize}

Le gradient de $\rho^n$ servira \`a calculer les termes de production par effets de gravit\'e si ces derniers sont pris en compte.

\etape{Boucle \var{ISOU} de $1$ \`a $6$ sur les tensions de Reynolds}
Pour $\var{ISOU} = 1,2,3,4,5,6$, on r\'esout respectivement et dans
l'ordre  les
\'equations de $R_{11}$, $R_{22}$, $R_{33}$, $R_{12}$, $R_{13}$ et $R_{23}$ par
l'appel au sous-programme \fort{resrij}.\\
La r\'esolution se fait par incr\'ement $\delta {R}_{ij}^{\,n+1,k+1}$ , en utilisant la m\^eme m\'ethode que
celle d\'ecrite dans le sous-programme \fort{codits}. On adopte ici les m\^emes notations.
\var{SMBR} est le second membre du syst\`eme \`a inverser, syst\`eme portant sur
les incr\'ements de la variable. \var{ROVSDT} repr\'esente la diagonale de la
matrice, hors convection/diffusion.\\
On va r\'esoudre l'\'equation (\ref{Base_Turrij_Eq_Temp_Rij}) sous forme incr\'ementale en
utilisant \fort{codits}, soit :
\begin{equation}\label{Base_Turrij_Eq_Temp_deltaRij}
\begin{array}{ll}
&\displaystyle \underbrace{\left(\frac {\rho^n_L}{\Delta t^n}
+ \rho^n_L \,C_1\,\frac{\varepsilon^n_L}{k^n_L}(1-\frac{\delta_{ij}}{3})
 - m^n_{\,lm} + \Gamma_L\,+ max(-\alpha^n_{R_{ij}},0)\right)\,|\Omega_l|}
_{\text {$\var{ROVSDT}$ contribuant
\`a la diagonale de la matrice simplifi\'ee de \fort{matrix}}}\,(\delta{R}_{ij}^{\,n+1,p+1})_{\,L}\\\\
&  \underbrace{+\sum\limits_{m\in Vois(l)}\displaystyle \left[
 m^n_{\,lm} \delta R_{ij,\,f_{\,lm}}^{\,n+1,p+1}
- (\mu^n_{\,lm} + \gamma^n_{\,lm})\
\frac{({\delta R}_{ij}^{\,n+1,p+1})_{M}-({\delta R}_{ij}^{\,n+1,p+1})_{L})}{\overline{L'M'}}\,
S_{\,lm} \right]}_{\text { convection upwind pur et diffusion non reconstruite
relatives \`a la matrice simplifi\'ee de \fort{matrix}\footnotemark}} \\
% voir le texte de la footmark plus bas
&= - \displaystyle\frac {\rho^n_L}{\Delta t^n}\,\left(\,(R^{\,n+1,p}_{ij})_L - (R^{\,n}_{ij})_L\,\right)\\
&-\,\underbrace{\displaystyle\int_{\Omega_l} \left(
\dive\,[\,(\rho\,\vect{u})^n\,R^{\,n+1,p}_{ij} - (\mu^n\,+ \gamma^n\,)\,
\grad{R^{\,n+1,p}_{ij}}\,]\right)\,d\Omega}_{\text {convection et diffusion
trait\'ees par \fort{bilsc2}}}\\
&+\displaystyle \int_{\Omega_l} \left[\,\mathcal{P}^{\,n+1,p}_{ij} + \mathcal{G}^{\,n+1,p}_{ij}
- \displaystyle\rho^n \,C_1\,\frac{\varepsilon^n}{k^n}\left[R^{\,n+1,p}_{ij}-
\frac{2}{3}\,k^n\,\delta_{ij}\right] + \phi^{\,n+1,p}_{ij,2} +
\phi^{\,n+1,p}_{ij,w}\,\right]\, d\Omega \\
& + \displaystyle\int_{\Omega_l} \left[- \frac{2}{3} \rho^n \varepsilon^n \delta_{ij}
 + \Gamma\,(\,R^{\,in}_{ij} - R^{\,n+1,p}_{ij}\,) +
\alpha^n_{R_{ij}}\,R^{\,n+1,p}_{ij}+ \beta^n_{R_{ij}}\right]\, d\Omega\\
&+ \sum\limits_{m\in
Vois(l)}\displaystyle \left[\ \tens{E}^n\,\grad{R}^{\,n+1,p}_{ij} \right]_{\,lm}\,.\,\vect{n}_{\,lm}S_{\,lm}\\
&+ \sum\limits_{m\in Vois(l)}\displaystyle \left[\
\tens{D}^n\,\grad{R}^{\,n+1,p}_{ij} \right]_{\,lm}\,.\,\vect{n}_{\,lm}S_{\,lm}\\
&- \sum\limits_{m\in Vois(l)} \gamma^n_{\,lm} \left( \grad{R}^{\,n+1,p}_{ij}\,.\,\vect{n}_{\,lm} \right)  S_{\,lm}\\
&+ \sum\limits_{m\in Vois(l)}  m^n_{\,lm}\,(R^{\,n+1,p}_{ij})_L\\
\end{array}
\end{equation}
% si on ne fait pas comme ca, il n'apparait pas
\footnotetext[\thefootnote]{Si $\var{IRIJNU} = 1$, on remplace  $\mu^n_{\,lm}$ par $(\mu +
\mu_t)^n_{\,lm}$ dans l'expression de la diffusion non reconstruite
associ\'ee \`a la matrice simplifi\'ee de \fort{matrix} ($\mu_t$ d\'esigne la
viscosit\'e turbulente calcul\'ee comme en $k-\varepsilon$).}

o\`u on rappelle :\\
pour $n$ donn\'e entier positif, on d\'efinit la suite
 $({R}_{ij}^{\,n+1,p})_{p \in \grandN}$
 par :
\begin{equation}\notag
\left\{\begin{array}{l}
{R}_{ij}^{\,n+1,0} = {R}_{ij}^{\,n}\\
{R}_{ij}^{\,n+1,p+1} = {R}_{ij}^{\,n+1,p} + \delta{R}_{ij}^{\,n+1,p+1} \\
\end{array}\right.
\end{equation}
$(\delta{R}_{ij}^{\,n+1,p+1})_{\,L}$ d\'esigne la valeur sur l'\'el\'ement
$\Omega_l$ du $\text{$(\,p+1\,)$-i\`eme}$ incr\'ement de ${R}_{ij}^{\,n+1}$,
$ m^n_{\,lm}$ le flux de masse \`a l'instant $n$ \`a travers la face $lm$,
$\delta R_{ij,\,f_{\,lm}}^{\,n+1,p+1}$ vaut $({\delta
R}_{ij}^{\,n+1,p+1})_{L}$  si $ m^n_{\,lm} \geqslant 0$, $({\delta
R}_{ij}^{\,n+1,p+1})_{M}$ sinon,
$\mathcal{P}^{\,n+1,p}_{ij}$, $\phi^{\,n+1,p}_{ij,2}$, $\phi^{\,n+1,p}_{ij,w}$ les valeurs
des quantit\'es associ\'ees correspondant \`a l'incr\'ement
$(\delta{R}_{ij}^{\,n+1,p})$.\\



Tous ces termes sont calcul\'es comme suit :
\begin{itemize}
\item Terme de gauche de l'\'equation (\ref{Base_Turrij_Eq_Temp_deltaRij})\\
Dans \fort{resrij} est calcul\'ee la variable \var{ROVSDT}. Les autres
termes sont compl\'et\'es par \fort{codits}, lors de la construction de la matrice simplifi\'ee , {\it via} un
appel au sous-programme \fort{matrix}. La quantit\'e
 $(\mu^n_{\,lm} + \gamma^n_{\,lm})$ \`a la face $lm$ est calcul\'ee lors de l'appel \`a
\fort{visort}.\\
\item Second membre de l'\'equation (\ref{Base_Turrij_Eq_Temp_deltaRij})\\
Le premier terme non d\'etaill\'e est calcul\'e par le sous-programme
\fort{bilsc2}, qui applique le sch\'ema convectif choisi par l'utilisateur, qui
reconstruit ou non selon le souhait de l'utilisateur les gradients intervenants
dans la convection-diffusion.\\
Les termes sans accolade sont, eux, compl\`etement explicites et ajout\'es au fur et
\`a mesure dans \var{SMBR} pour former
l'expression $f^{\,exp}_s$ de \fort{codits}.
\end{itemize}
On d\'ecrit ci-dessous les \'etapes de \fort{resrij} :
\begin{itemize}

\item DELTIJ = 1, pour $\var{ISOU} \leqslant 3$ et DELTIJ = 0  Si $\var{ISOU} >
3$. Cette valeur repr\'esente le symbole de Kroeneker $\delta_{ij}$.

\item Initialisation \`a z\'ero de \var{SMBR} (tableau contenant le second
membre) et \var{ROVSDT} (tableau contenant la diagonale de la matrice sauf celle
relative \`a la contribution de la
diagonale des op\'erateurs de convection et de diffusion lin\'earis\'es
\footnote{qui correspondent aux sch\'emas convectif upwind pur et diffusif sans
reconstruction.}), tous deux de dimension $\var{NCEL}$.

\item Lecture et prise en compte des termes sources utilisateur pour la variable $R_{ij}$

Appel \`a \fort{ustsri} pour charger les termes sources utilisateurs. Ils sont
stock\'es comme suit. Pour la cellule $\Omega_l$ de centre $L$, repr\'esent\'ee par $\var{IEL}$, on a :\\
\begin{equation}\notag
\left\{\begin{array}{lll}
&\var{ROVSDT(IEL)}&= |\Omega_l| \ \alpha_{R_{ij}}\\
&\var{SMBR(IEL)}&=|\Omega_l| \ \beta_{R_{ij}}\\
\end{array}\right.
\end{equation}
On affecte alors les valeurs ad\'equates au second membre \var{SMBR} et \`a la
diagonale \var{ROVSDT} comme suit :
\begin{equation}\notag
\left\{\begin{array}{lll}
&\var{SMBR(IEL)} &= \var{SMBR(IEL)} +\ |\Omega_l| \ \alpha_{R_{ij}} \ (R^n_{ij})_L \\
&\var{ROVSDT(IEL)}&= \text{max }(-\ |\Omega_l| \ \alpha_{R_{ij}},0)\\
\end{array}\right.
\end{equation}
La valeur de $ \var{ROVSDT}$ est ainsi calcul\'ee pour des raisons de stabilit\'e
num\'erique. En effet, on ne rajoute sur la diagonale que les valeurs positives,
ce qui correspond physiquement \`a impliciter les termes de rappel uniquement.
\item{Calcul du terme source de masse  si $\Gamma_L > 0$}

Appel de \fort{catsma} et incr\'ementation si n\'ecessaire de \var{SMBR} et
\var{ROVSDT} {\it via} :\\
\begin{equation}\notag
\left\{\begin{array}{lll}
\displaystyle \var{SMBR(IEL)} = \var{SMBR(IEL)} + |\Omega_l| \ \Gamma_L \
\left[(R^{\,in}_{ij})_L - (R^{\,n}_{ij})_L \right] \\
\displaystyle \var{ROVSDT(IEL)}=\var{ROVSDT(IEL)} + |\Omega_l| \ \Gamma_L
\end{array}\right.
\end{equation}
\item Calcul du terme d'accumulation de masse et du terme instationnaire

On stocke $\displaystyle \var{W1}= \int_{\Omega_l}\dive\,(\rho\,\vect{u})\,d\Omega$
calcul\'e par \fort{divmas} \`a l'aide des flux de masse aux faces internes
$ m^n_{\,lm}=\sum\limits_{m\in Vois(l)}{(\rho \vect{u})_{\,lm}^n} \text{.}\,
\vect{S}_{\,lm} $ (tableau \var{FLUMAS}) et des flux de masse aux bords  $ m^n_{\,b_{lk}} = \sum\limits_{k\in{\gamma_b(l)}}{(\rho \vect{u})_{\,{b}_{lk}}^n} \text{.}\,
\vect{S}_{\,{b}_{lk}} $ (tableau \var{FLUMAB}).
On incr\'emente ensuite \var{SMBR} et \var{ROVSDT}.
\begin{equation}\notag
\left\{\begin{array}{lll}
&\var{SMBR(IEL)} &= \var{SMBR(IEL)} + \var{ICONV}\  (R^n_{ij})_L\,(\displaystyle
\int_{\Omega_l}\dive\,(\rho\,\vect{u})\ d\Omega) \\
&\var{ROVSDT(IEL)}& = \var{ROVSDT(IEL)} +  \var{ISTAT}\,\displaystyle
\frac{\rho^n_L \ |\Omega_l|}{\Delta t^n} -  \var{ICONV}\ (\displaystyle
\int_{\Omega_l}\dive\,(\rho\,\vect{u})\ d\Omega) \\
\end{array}\right.
\end{equation}
\item Calcul des termes sources de production, des termes $\displaystyle
\phi_{\,ij,1}+\phi_{\,ij,2}$ et de la dissipation~$\displaystyle-\frac{2}{3} \varepsilon\,\delta_{\,ij}$ :

On effectue une boucle d'indice \var{IEL} sur les cellules $\Omega_l$ de centre $L$ :
\begin{itemize}
\item [$\Rightarrow$] $\displaystyle \var{TRPROD}= \frac{1}{2} (\mathcal{P}^n_{ii})_L = \frac{1}{2} \left[ \var{PRODUC(1,IEL)} +  \var{PRODUC(2,IEL)} +  \var{PRODUC(3,IEL)} \right] $
\item [$\Rightarrow$] $\displaystyle \var{TRRIJ }= \frac{1}{2} (R^n_{ii})_L $
\item [$\Rightarrow$] $\displaystyle \var{SMBR(IEL)} =\ \var{SMBR(IEL)}\ +$\\
$\ \displaystyle\rho^n_L |\Omega_l| \left[ \displaystyle
\frac{2}{3}\,\delta_{\,ij} \left( \ \displaystyle \frac{ C_2}{2}\,(\mathcal{P}^n_{ii})_L\ +
(C_1-1)\ \varepsilon^n_L\, \right)\right.$\\
$ + \left.\ (1-C_2) \ \var{PRODUC(ISOU,IEL)} -
\displaystyle C_1\ \frac{2\,\varepsilon^n_L}{(R^n_{ii})_L}\ (R^n_{ij})_L \right]$
\item [$\Rightarrow$] $\displaystyle \var{ROVSDT(IEL)} = \var{ROVSDT(IEL)} +
\rho^n_L \ |\Omega_l| \ (- \displaystyle \frac{1}{3} \ \,\delta_{\,ij} + 1) \ C_1
\ \frac{2\ \varepsilon^n_L}{(R^n_{ii})_L}$
\end{itemize}
\item Appel de \fort{rijech} pour le calcul des termes d'\'echo de paroi
 $\phi^n_{ij,w}$ si $\var{IRIJEC()}=1$ et ajout dans \var{SMBR}.\\
$\var{SMBR} = \var{SMBR} + \phi^n_{ij,w}$\\
Suivant son mode de calcul (\var{ICDPAR}), la distance � la paroi est directement accessible
par \var{RA(IDIPAR+IEL-1)} (\var{|ICDPAR|} = 1) ou bien
est calcul\'ee \`a partir de $\var{IA(IIFAPA(IPHAS)+IEL - 1)}$,
qui donne pour l'\'el\'ement $\var{IEL}$ le num\'ero de la face de bord
paroi la plus  proche (\var{|ICDPAR|} = 2). Ces tableaux ont \'et\'e renseign\'e une fois pour toutes au
d\'ebut de calcul.

\item  Appel de \fort{rijthe} pour le calcul des termes de gravit\'e $\mathcal{G}^n_{ij}$ et ajout dans \var{SMBR}.

Ce calcul n'a lieu que si $\var{IGRARI()} = 1$.
$ \var{SMBR} = \var{SMBR} + \mathcal{G}^n_{ij}$
\item Calcul de la partie explicite du terme de diffusion
 $\dive{\,\left[\tens{A}\,\grad{R}^{\,n}_{ij}\right]}$, plus pr\'ecis\'ement
des contributions du terme extradiagonal pris aux faces purement internes
(remplissage du tableau \var{VISCF}), puis aux faces de bord (remplissage du
tableau \var{VISCB}).
\begin{itemize}
\item [$\star$] Appel de \fort{grdcel} pour le calcul du gradient de
$R^{\,n}_{ij}$ dans chaque direction. Ces gradients sont respectivement
stock\'es dans les tableaux de travail \var{W1}, \var{W2} et \var{W3}.

\item [$\star$] boucle d'indice \var{IEL} sur les cellules $\Omega_l$ de centre
$L$ pour le
calcul de $\tens{E}^n\,\grad{R}^{\,n}_{ij}$ aux cellules dans un premier temps :\\
\begin{itemize}
\item [$\Rightarrow$] $\displaystyle \var{TRRIJ}= \frac{1}{2} (R^{\,n}_{ii})_L $
\item [$\Rightarrow$] $\displaystyle \var{CSTRIJ} = \rho^n_L\ C_S \ \displaystyle\frac{(R^n_{ii})_L}{2\,\varepsilon^n_L}$
\item [$\Rightarrow$] $\displaystyle \var{W4(IEL)} = \rho^n_L\ C_S\
\displaystyle\frac{(R^n_{ii})_L}{2\,\varepsilon^n_L} \left[\,(R^{\,n}_{12})_L \ \var{W2(IEL)} +
(R^{\,n}_{13})_L \ \var{W3(IEL)}\,\right]$
\item [$\Rightarrow$] $\displaystyle \var{W5(IEL)} = \rho^n_L\ C_S\
\displaystyle\frac{(R^n_{ii})_L}{2\,\varepsilon^n_L} \left[\,(R^{\,n}_{12})_L \ \var{W1(IEL)} +
(R^{\,n}_{23})_L \ \var{W3(IEL)}\,\right]$
\item [$\Rightarrow$] $\displaystyle \var{W6(IEL)} = \rho^n_L\ C_S\
\displaystyle\frac{(R^n_{ii})_L}{2\,\varepsilon^n_L} \left[\,(R^{\,n}_{13})_L \ \var{W1(IEL)} + (R^{\,n}_{23})_L \ \var{W2(IEL)}\,\right]$
\end{itemize}



\item [$\star$] Appel de \fort{vectds}\footnote{Le r\'esultat est stock\'e dans
\var{VISCF} et \var{VISCB}. Dans \fort{vectds}, les valeurs aux faces internes
sont interpol\'ees lin\'eairement sans reconstruction et \var{VISCB} est mis \`a
z\'ero.} pour assembler $\displaystyle\left[ \tens{E}^n\,\grad{R}^{\,n}_{ij}
\right]\,.\,\vect{n}_{\,lm}S_{\,lm}$ aux faces $lm$.
\item [$\star$] Appel de \fort{divmas} pour calculer la divergence du flux d\'efini par \var{VISCF} et \var{VISCB}.
Le r\'esultat est stock\'e dans \var{W4}.\\
Ajout au second membre \var{SMBR}.\\
\var{SMBR} = \var{SMBR} + \var{W4}
\end{itemize}

A l'issue de cette \'etape, seule la partie extradiagonale de la diffusion prise
enti\`erement explicite~:
 $$\sum\limits_{m\in
Vois(l)}\left[\ \tens{E}^n\,\grad{R}^{\,n}_{ij} \right]_{\,lm}\,.\,\vect{n}_{\,lm}S_{\,lm}$$ a \'et\'e calcul\'ee.\\

\item Calcul de la partie diagonale du terme de diffusion\footnote{Seule la
composante normale  du  gradient de $R_{ij}$ aux faces sera implicite.} :\\
Comme on l'a d\'eja vu, une partie de cette contribution sera trait\'ee en
implicite (celle relative \`a la composante normale du gradient) et donc
ajout\'ee au second membre par \fort{bilsc2} ; l'autre
partie sera explicite et prise en compte dans $f_s^{\,exp}$.
\begin{itemize}
\item [$\star$] On effectue une boucle d'indice \var{IEL} sur les cellules
$\Omega_l$ de centre $L$ :
\begin{itemize}
\item [$\Rightarrow$] $\displaystyle \var{TRRIJ }= \frac{1}{2} (R^{\,n}_{ii})_L $
\item [$\Rightarrow$] $\displaystyle \var{CSTRIJ} = \rho^n_L \ C_S \ \frac{(R^{\,n}_{ii})_L}{2\,\varepsilon^n_L}$
\item [$\Rightarrow$] $\displaystyle \var{W4(IEL)} = \rho^n_L \ C_S \
\frac{(R^{\,n}_{ii})_L}{2\,\varepsilon^n_L} \ (R^{\,n}_{11})_L$
\item [$\Rightarrow$] $\displaystyle \var{W5(IEL)} = \rho^n_L \ C_S \ \frac{(R^{\,n}_{ii})_L}{2\,\varepsilon^n_L}\ (R^n_{22})_L$
\item [$\Rightarrow$] $\displaystyle \var{W6(IEL)} = \rho^n_L \ C_S \ \frac{(R^{\,n}_{ii})_L}{2\,\varepsilon^n_L} \ (R^n_{33})_L$
\end{itemize}

%\item Traitement du parall\'elisme et de la p\'eriodicit\'e.

\item [$\star$] On effectue une boucle d'indice \var{IFAC} sur les faces
purement internes $lm$ pour remplir le tableau \var{VISCF} :
\begin{itemize}
\item [$\Rightarrow$] Identification des cellules $\Omega_l$ et $\Omega_m$ de
centre respectif $L$ (variable \var{II}) et $M$ (variable \var{JJ}), se trouvant de chaque c\^ot\'e de la face
$lm$\footnote{La normale \`a la face est orient\'ee de L vers M.}.
\item [$\Rightarrow$] Calcul du carr\'e de la surface de la face. La valeur est
stock\'ee dans le tableau \var{SURFN2}.
\item [$\Rightarrow$] Interpolation du gradient de $R^{\,n}_{ij}$ \`a la face
$lm$ (gradient facette $\left[\grad{R}^{\,n}_{ij}\right]_{\,lm}$) :
\begin{equation}\notag
\left\{\begin{array}{ll}
\var{GRDPX} &= \displaystyle \frac{1}{2} \left(\var{W1(II)} + \var{W1(JJ)}
\right) \\
&\\
\var{GRDPY} &= \displaystyle \frac{1}{2} \left(\var{W2(II)} + \var{W2(JJ)} \right) \\
&\\
\var{GRDPZ} &= \displaystyle \frac{1}{2} \left(\var{W3(II)} + \var{W3(JJ)} \right)
\end{array}\right.
\end{equation}
\item [$\Rightarrow$] Calcul du gradient de $R^{\,n}_{ij}$ normal \`a la face
$lm$, $\left[\grad{R}^{\,n}_{ij}\right]_{\,lm}.\vect{n}_{\,lm}\,S_{\,lm}$.\\

$\displaystyle \var{GRDSN} =  \var{GRDPX} \ \var{SURFAC(1,IFAC)} + \var{GRDPY} \ \var{SURFAC(2,IFAC)} +  \var{GRDPZ} \ \var{SURFAC(3,IFAC)}$
$\var{SURFAC}$ est le vecteur surface de la face \var{IFAC}.


\item [$\Rightarrow$] calcul de
 $\left[\grad{R^{\,n}_{ij}} - (\grad
R^{\,n}_{ij}\,.\,\vect{n}_{\,lm})\vect{n}_{\,lm}\right]$, les vecteurs \'etant
calcul\'es \`a la face $lm$ :
\begin{equation}\notag
\left\{\begin{array}{lll}
&\displaystyle \var{GRDPX} &= \var{GRDPX} - \displaystyle\frac{\var{GRDSN}}{\var{SURFN2}} \ \var{SURFAC(1,IFAC)}\\
&&\\
&\displaystyle \var{GRDPY} &= \var{GRDPY} - \displaystyle\frac{\var{GRDSN}}{\var{SURFN2}} \ \var{SURFAC(2,IFAC)} \\
&&\\
&\displaystyle \var{GRDPZ} &= \var{GRDPZ} - \displaystyle\frac{\var{GRDSN}}{\var{SURFN2}} \ \var{SURFAC(3,IFAC)}
\end{array}\right.
\end{equation}
\item [$\Rightarrow$] finalisation du calcul de l'expression totalement
explicite
 $$\left[ \tens{D}^n\,\left( \grad{R^{\,n}_{ij}} - (\grad R^{\,n}_{ij}\,.\,\vect{n}_{\,lm})\,\vect{n}_{\,lm}\right) \right]\,.\,\vect{n}_{\,lm}$$
\begin{equation}\notag
\begin{array} {ll}
\displaystyle \var{VISCF} = &
 \displaystyle\frac{1}{2} (\ \var{W4(II)} +\ \var{W4(JJ)}) \ \var{GRDPX} \
\var{SURFAC(1,IFAC)})\ + \\
&\\
&  \displaystyle\frac{1}{2} (\ \var{W5(II)} +\ \var{W5(JJ)}) \ \var{GRDPY} \
\var{SURFAC(2,IFAC)})\ + \\
&\\
&  \displaystyle\frac{1}{2} (\ \var{W6(II)} +\ \var{W6(JJ)}) \ \var{GRDPZ} \ \var{SURFAC(3,IFAC)})
\end{array}
\end{equation}
\end{itemize}

\item [$\star$] Mise \`a z\'ero du tableau \var{VISCB}.

\item [$\star$] Appel de \fort{divmas} pour calculer la divergence de~:
 $$\tens{D}^{\,n}\,\left( \grad{R^{\,n}_{ij}} - (\grad R^{\,n}_{ij}\,.\,\vect{n}_{\,lm})\vect{n}_{\,lm}\right)$$ d\'efini aux faces dans \var{VISCF} et \var{VISCB}.

Le r\'esultat est stock\'e dans le tableau \var{W1}.\\
Ajout au second membre \var{SMBR}.\\
$\var{SMBR} = \var{SMBR} + \var{W1}$
\end{itemize}
\item Calcul de la viscosit\'e orthotrope $\gamma^n_{\,lm}$ \`a la face $lm$ de la variable principale
$R^{\,n}_{ij}$\\
Ce calcul permet au sous-programme \fort{codits} de compl\'eter le second membre
\var{SMBR} par :
\begin{equation}
\begin{array} {ll}
& \sum\limits_{m\in Vois(l)}
\mu^n_{\,lm}\,\left(\grad{R}^{\,n}_{ij}\,.\,\vect{n}_{\,lm}\right)S_{\,lm}
 + \sum\limits_{m\in Vois(l)} \left(\grad{R}^{\,n}_{ij}
\,.\,\vect{n}_{\,lm}\right)\left[\tens{D}^{\,n}\,\vect{n}_{\,lm}\right]_{\,lm}\,.\,\vect{n}_{\,lm}\
S_{\,lm}\\
& = \sum\limits_{m\in Vois(l)}(\,\mu^n_{\,lm}\, + \,\gamma^n_{\,lm}\,)\,\left(\grad{R}^{\,n}_{ij}\,.\,\vect{n}_{\,lm}\right)S_{\,lm}
\end{array}
\end{equation}
sans pr\'eciser la nature de la face $lm$, {\it via} l'appel \`a \fort{bilsc2}  et de disposer de la quantit\'e
$(\mu^n_{\,lm}\, + \gamma^n_{\,lm})$ pour construire sa
matrice simplifi\'ee.\\
\begin{itemize}
\item [$\star$] On effectue une boucle d'indice \var{IEL} sur les cellules
$\Omega_l$ :
\begin{itemize}
\item [$\Rightarrow$] $\displaystyle \var{TRRIJ }= \frac{1}{2} (R^{\,n}_{ii})_L $
\item [$\Rightarrow$] $\displaystyle \var{RCSTE} = \rho^n_L \ C_S \ \frac{ (R^{\,n}_{ii})_L}{2\,\varepsilon^n_L} $
\item [$\Rightarrow$] $\displaystyle \var{W1(IEL)} = \mu^n +\rho^n_L \ C_S \ \frac{
(R^{\,n}_{ii})_L}{2\,\varepsilon^n_L}\ (R^n_{11})_L$
\item [$\Rightarrow$] $\displaystyle \var{W2(IEL)} = \mu^n + \rho^n_L \ C_S \ \frac{ (R^{\,n}_{ii})_L}{2\,\varepsilon^n_L}\ (R^n_{22})_L$
\item [$\Rightarrow$] $\displaystyle \var{W3(IEL)} = \mu^n + \rho^n_L \ C_S \ \frac{ (R^{\,n}_{ii})_L}{2\,\varepsilon^n_L}\ (R^n_{33})_L$
\end{itemize}

\item [$\star$] Appel de \fort{visort} pour calculer la viscosit\'e orthotrope
\footnote{Comme dans le sous-programme \fort{viscfa}, on multiplie la viscosit\'e par
$\displaystyle \frac{S_{\,lm}}{\overline{L'M'}}$, o\`u $S_{\,lm}$ et
$\overline{L'M'}$ repr\'esentent respectivement la surface de la face $lm$ et la
mesure alg\'ebrique du segment reliant les projections des centres des cellules
voisines sur la normale \`a la face. On garde dans ce sous-programme  la possibilit\'e d'interpoler la viscosit\'e aux cellules lin\'eairement ou d'utiliser une moyenne harmonique. La viscosit\'e au bord est celle de la cellule de bord correspondante.}
$ \gamma^n_{\,lm} = (\tens{D}^{\,n}\,\vect{n}_{\,lm}).\vect{n}_{\,lm}$ aux faces $lm$

Le r\'esultat est stock\'e dans les tableaux \var{VISCF} et \var{VISCB}.
\end{itemize}

\item appel de \fort{codits} pour la r\'esolution de l'\'equation de
convection/diffusion/termes sources de la variable $R_{ij}$. Le terme source est
\var{SMBR}, la viscosit\'e \var{VISCF} aux faces purement internes (
resp. \var{VISCB} aux faces de bord) et \var{FLUMAS} le flux de masse interne
 ( resp. \var{FLUMAB} flux de masse au bord). Le r\'esultat est la variable $R_{ij}$ au temps
$n+1$, donc $R^{\,n+1}_{ij}$. Elle est stock\'ee dans le tableau \var{RTP} des
variables mises \`a jour.

\end{itemize}

\etape{Appel de \fort{reseps} pour la r\'esolution de la variable $\varepsilon$}

On d\'ecrit ci-dessous le sous-programme \fort{reseps}, les commentaires du sous-programme \fort{resrij} ne seront pas r\'ep\'et\'es puisque les deux sous-programmes ne diff\`erent que par leurs termes sources.

\begin{itemize}
\item Initialisation \`a z\'ero de \var{SMBR} et \var{ROVSDT}.

\item{Lecture et prise en compte des termes sources utilisateur pour la variable $\varepsilon$ :}

Appel de \fort{ustsri} pour charger les termes sources utilisateurs. Ils sont
stock\'es dans les tableaux suivants :\\
pour la cellule $\Omega_l$ repr\'esent\'ee par $\var{IEL}$ de centre $L$, on a :
\begin{equation}\notag
\left\{\begin{array}{lll}
&\var{ROVSDT(IEL)}&= |\Omega_l| \ \alpha_{\varepsilon}\\
&\var{SMBR(IEL)}&=|\Omega_l| \ \beta_{\varepsilon}\\
\end{array}\right.
\end{equation}
On affecte alors les valeurs ad\'equates au second membre \var{SMBR} et \`a la
diagonale \var{ROVSDT} comme suit :
\begin{equation}\notag
\left\{\begin{array}{lll}
&\var{SMBR(IEL)} &= \var{SMBR(IEL)} +\ |\Omega_l| \ \alpha_{\,\varepsilon} \
\varepsilon^n_L \\
&\var{ROVSDT(IEL)}&= \text{max }(-\ |\Omega_l| \ \alpha_{\,\varepsilon},0)\\
\end{array}\right.
\end{equation}

\item{Calcul du terme source de masse si $\Gamma_L > 0$ :
\begin{equation}\notag
\left\{\begin{array}{lll}
&\displaystyle \var{SMBR(IEL)} = \var{SMBR(IEL)} + |\Omega_l| \ \Gamma_L \
(\varepsilon^{\,in}_L -\varepsilon^n_L) \\
&\displaystyle \var{ROVSDT(IEL)}= \var{ROVSDT(IEL)} + |\Omega_l| \ \Gamma_L
\end{array}\right.
\end{equation}
\item Calcul du terme d'accumulation de masse et du terme instationnaire \\
On stocke $\displaystyle \var{W1}= \int_{\Omega_l}\dive\,(\rho\,\vect{u})\,d\Omega$
calcul\'e par \fort{divmas} \`a l'aide des flux de masse internes et aux bords.\\
On incr\'emente ensuite \var{SMBR} et \var{ROVSDT}.
\begin{equation}\notag
\left\{\begin{array}{lll}
&\var{SMBR(IEL)} &= \var{SMBR(IEL)} + \var{ICONV}\ \varepsilon^n_L\,(\displaystyle
\int_{\Omega_l}\dive\,(\rho\,\vect{u})\ d\Omega) \\
&\var{ROVSDT(IEL)}& = \var{ROVSDT(IEL)} +  \var{ISTAT}\,\displaystyle
\frac{\rho^n_L \ |\Omega_l|}{\Delta t^n} -  \var{ICONV}\ (\displaystyle
\int_{\Omega_l}\dive\,(\rho\,\vect{u})\ d\Omega) \\
\end{array}\right.
\end{equation}

\item Traitement du terme de production
 $\displaystyle \rho\,C_{\varepsilon_1}\,\frac{\varepsilon}{k}\,\mathcal{P}$
 et du terme de dissipation $-\,\displaystyle \rho\,C_{\varepsilon_2}\,\frac{\varepsilon}{k}\,\varepsilon$ \\
pour cela, on effectue une boucle d'indice \var{IEL} sur les cellules $\Omega_l$
de centre $L$ :
\begin{itemize}
\item [$\Rightarrow$] $\displaystyle \var{TRPROD}= \frac{1}{2} (\mathcal{P}^n_{ii})_L = \frac{1}{2} \left[ \var{PRODUC(1,IEL)} +  \var{PRODUC(2,IEL)} +  \var{PRODUC(3,IEL)} \right] $
\item [$\Rightarrow$] $\displaystyle \var{TRRIJ }= \frac{1}{2} (R^n_{ii})_L $
\item [$\Rightarrow$] $\displaystyle \var{SMBR(IEL)} = \var{SMBR(IEL)} + \rho^n_L
|\Omega_l| \left[ -C_{\varepsilon_2} \ \frac{2\,(\varepsilon^n_L)^2}{(R^n_{ii})_L} + C_{\varepsilon_1} \ \frac{\varepsilon^n_L}{(R^n_{ii})_L}\ (\mathcal{P}^n_{ii})_L \right] $
\item [$\Rightarrow$] $\displaystyle \var{ROVSDT(IEL)} = \var{ROVSDT(IEL)} + C_{\varepsilon_2} \ \rho^n_L \ |\Omega_l| \ \frac{2\,\varepsilon^n_L}{(R^n_{ii})_L}$
\end{itemize}

\item Appel de \fort{rijthe} pour le calcul des termes de gravit\'e $\mathcal{G}^n_{\varepsilon}$ et ajout dans \var{SMBR}.

$ \var{SMBR} = \var{SMBR} + \mathcal{G}^n_{\varepsilon}$\\
Ce calcul n'a lieu que si $\var{IGRARI()} = 1$.

\item Calcul de la diffusion de $\varepsilon$ \\
 Le terme $\dive \left[\mu\, \grad(\varepsilon) + \tens{A'}\,\grad(\varepsilon)
\right]$ est calcul\'e exactement de la m\^eme mani\`ere que pour les tensions
de Reynolds $R_{ij}$ en rempla\c cant $\tens{A}$ par $\tens{A'}$.

\item Appel de \fort{codits} pour la r\'esolution de l'\'equation de
convection/diffusion/termes sources de la variable principale $\varepsilon$. Le
r\'esultat $\varepsilon^{\,n+1}$ est stock\'e dans le tableau \var{RTP} des
variables mises \`a jour.
}
\end{itemize}

\etape{clippings finaux}
On passe enfin dans le sous-programme  \fort{clprij} pour faire un clipping \'eventuel
des variables $R^{\,n+1}_{ij}$ et $\varepsilon^{\,n+1}$. Le sous-programme
\fort{clprij} est appel\'e\footnote{L'option
$\var{ICLIP} = 1$ consiste en un clipping minimal des variables $R_{ii}$ et
$\varepsilon$ en prenant la valeur absolue de ces variables puisqu'elles ne
peuvent \^etre que positives.} avec $\var{ICLIP} = 2$ . Cette option
\footnote{Quand la valeur des grandeurs $R_{ii}$ ou $\varepsilon$ est
n\'egative, on la remplace par le minimum entre sa valeur absolue et (1,1)
fois la valeur obtenue au pas de temps pr\'ec\'edent.} contient l'option $\var{ICLIP} = 1$  et permet de v\'erifier l'in\'egalit\'e de Cauchy-Schwarz sur les grandeurs extra-diagonales du tenseur $\tens{R}$ (pour $i \neq j$, $|R_{ij}|^2 \le R_{ii} R_{jj}$).


%%%%%%%%%%%%%%%%%%%%%%%%%%%%%%%%%%
%%%%%%%%%%%%%%%%%%%%%%%%%%%%%%%%%%
\section{Points \`a traiter}
%%%%%%%%%%%%%%%%%%%%%%%%%%%%%%%%%%
%%%%%%%%%%%%%%%%%%%%%%%%%%%%%%%%%%
Sauf mention explicite, $\phi$ repr\'esentera une tension de Reynolds ou la dissipation turbulente ($\phi = R_{ij} \ \text{ou} \ \varepsilon$).

\begin{itemize}
\item {La vitesse utilis\'ee pour le calcul de la production est explicite. Est-ce qu'une implicitation peut am\'eliorer la pr\'ecision temporelle de $\phi$ \footnote{Cette remarque peut \^etre g\'en\'eralis\'ee. En effet, peut-on envisager d'actualiser les variables d\'ej\`a r\'esolues (sans r\'eactualiser les variables turbulentes apr\`es leur r\'esolution)? Ceci obligerait \`a modifier les sous-programmes tels que \fort{condli} qui sont appel\'es au d\'ebut de la boucle en temps.} ?}
\item {Dans quelle mesure le terme d'\'echo de paroi est-il valide ? En effet, ce terme est remis en question par certains auteurs.}
\item {On peut envisager la r\'esolution d'un syst\`eme hyperbolique pour les
tensions de Reynolds afin d'introduire un couplage avec le champ de vitesse.}
\item {Le flux au bord \var{VISCB} est annul\'e dans le sous-programme
\fort{vectds}. Peut-on envisager de mettre au bord la valeur de la variable
concern\'ee \`a la cellule de bord correspondant? De m\^eme, il faudrait se
pencher sur les hypoth\`eses sous-jacentes \`a l'annulation des contributions
aux bords de \var{VISCB} lors du calcul de : $$\left[ \tens{D}^n\,\left( \grad{R^{\,n}_{ij}} - (\grad R^{\,n}_{ij}\,.\,\vect{n}_{\,lm})\,\vect{n}_{\,lm}\right) \right]\,.\,\vect{n}_{\,lm}.$$}
\item {Un probl\`eme de pond\'eration appara\^\i t plus g\'en\'eralement. Si on prend la partie explicite de $\tens{D}\,\grad(\phi)$, la pond\'eration aux faces internes utilise le coefficient $\displaystyle\frac{1}{2}$ avec pond\'eration s\'epar\'ee de $\tens{D}$ et $\grad(\phi)$, alors que pour $\tens{E}\,\grad(\phi)$, on calcule d'abord ce terme aux cellules pour ensuite l'interpoler lin\'eairement aux faces \footnote{Cette interpolation se fait dans \fort{vectds} avec des coefficients de pond\'eration aux faces.}. Ceci donne donc deux types d'interpolations pour des termes de m\^eme nature.}
\item {On laisse la possibilit\'e dans \fort{visort} d'utiliser une moyenne
harmonique aux faces. Est-ce que ceci est valable puisque les interpolations
utilis\'ees lors du calcul de la partie explicite de $\tens{A}\,\grad{\phi}$
sont des moyennes arithm\'etiques ?}
\item {Les techniques adopt\'ees lors du clipping sont \`a revoir.}
\item {On utilise dans le cadre du mod\`ele $\displaystyle R_{ij}-\varepsilon $ une semi-implicitation de termes comme $\displaystyle \phi_{ij,1}$ ou $\displaystyle -\rho\,C_{\varepsilon_2}\,\frac{\varepsilon}{k}\,\varepsilon$. On peut envisager le m\^eme type d'implicitation dans \fort{turbke} m\^eme en pr\'esence du couplage $\displaystyle k-\varepsilon$.}
\item L'adoption d'une r\'esolution d\'ecoupl\'ee fait perdre l'invariance par rotation.
\item La formulation et l'implantation des conditions aux limites de paroi
devront \^etre v\'erifi\'ees. En effet, il semblerait que, dans certains cas, des ph\'enom\`enes
``oscillatoires'' apparaissent, sans qu'il soit ais\'e d'en d\'eterminer la cause.
\item L'implicitation partielle (du fait de la r\'esolution d\'ecoupl\'ee) des
conditions aux limites conduit souvent \`a des calculs instables. Il
conviendrait d'en conna\^\i tre la raison. L'implicitation partielle avait
\'et\'e mise en \oe uvre afin de tenter d'utiliser un pas de temps plus grand
dans le cas de jets axisym\'etriques en particulier.

\end{itemize}

%                      Code_Saturne version 1.3
%                      ------------------------
%
%     This file is part of the Code_Saturne Kernel, element of the
%     Code_Saturne CFD tool.
%
%     Copyright (C) 1998-2007 EDF S.A., France
%
%     contact: saturne-support@edf.fr
%
%     The Code_Saturne Kernel is free software; you can redistribute it
%     and/or modify it under the terms of the GNU General Public License
%     as published by the Free Software Foundation; either version 2 of
%     the License, or (at your option) any later version.
%
%     The Code_Saturne Kernel is distributed in the hope that it will be
%     useful, but WITHOUT ANY WARRANTY; without even the implied warranty
%     of MERCHANTABILITY or FITNESS FOR A PARTICULAR PURPOSE.  See the
%     GNU General Public License for more details.
%
%     You should have received a copy of the GNU General Public License
%     along with the Code_Saturne Kernel; if not, write to the
%     Free Software Foundation, Inc.,
%     51 Franklin St, Fifth Floor,
%     Boston, MA  02110-1301  USA
%
%-----------------------------------------------------------------------
%
\programme{vortex}
%
\vspace{1cm}
%%%%%%%%%%%%%%%%%%%%%%%%%%%%%%%%%%
%%%%%%%%%%%%%%%%%%%%%%%%%%%%%%%%%%
\section{Fonction}
%%%%%%%%%%%%%%%%%%%%%%%%%%%%%%%%%%
%%%%%%%%%%%%%%%%%%%%%%%%%%%%%%%%%%
Ce sous-programme est d�di� � la g�n�ration des conditions d'entr�e
turbulente utilis�es en LES.


La m�thode des vortex est bas�e sur une approche de tourbillons
ponctuels. L'id�e de la m�thode consiste � injecter des tourbillons 2D dans le
plan d'entr�e du calcul, puis � calculer le champ de vitesse induit par ces
tourbillons au centre des faces d'entr�e.

%                      Code_Saturne version 1.3
%                      ------------------------
%
%     This file is part of the Code_Saturne Kernel, element of the
%     Code_Saturne CFD tool.
% 
%     Copyright (C) 1998-2007 EDF S.A., France
%
%     contact: saturne-support@edf.fr
% 
%     The Code_Saturne Kernel is free software; you can redistribute it
%     and/or modify it under the terms of the GNU General Public License
%     as published by the Free Software Foundation; either version 2 of
%     the License, or (at your option) any later version.
% 
%     The Code_Saturne Kernel is distributed in the hope that it will be
%     useful, but WITHOUT ANY WARRANTY; without even the implied warranty
%     of MERCHANTABILITY or FITNESS FOR A PARTICULAR PURPOSE.  See the
%     GNU General Public License for more details.
% 
%     You should have received a copy of the GNU General Public License
%     along with the Code_Saturne Kernel; if not, write to the
%     Free Software Foundation, Inc.,
%     51 Franklin St, Fifth Floor,
%     Boston, MA  02110-1301  USA
%
%-----------------------------------------------------------------------
%
%%%%%%%%%%%%%%%%%%%%%%%%%%%%%%%%%%
%%%%%%%%%%%%%%%%%%%%%%%%%%%%%%%%%%
\section{Discr\'etisation}
%%%%%%%%%%%%%%%%%%%%%%%%%%%%%%%%%%
%%%%%%%%%%%%%%%%%%%%%%%%%%%%%%%%%%

Le terme convectif en $\dive(\underline{u} \otimes \rho\,\underline{u})$
introduit une non lin\'earit\'e et un couplage des composantes de la vitesse
$\vect{u}$ dans l'�quation (\ref{Base_Preduv_eqqdm}). Une lin\'earisation et un d\'ecouplage
des trois composantes de la 
vitesse sont r\'ealis\'es lors de la discr\'etisation de cette \'etape de
pr\'ediction.\\
En effet, soit :
\begin{equation}
\vect{\widetilde{u}}= \vect{u}^n + \delta \vect{u} 
\end{equation}
La contribution exacte du terme convectif \`a prendre en compte dans cette
\'etape de pr\'ediction serait :\\
\begin{equation}\label{Base_Preduv_Conv_exact}
\begin{array}{ll}
\dive(\vect{\widetilde{u}} \otimes \rho\,\vect{\widetilde{u}}) =
\dive(\vect{u}^{n} \otimes \rho\,\vect{u}^{n}) + \dive(\delta \vect{u} \otimes
\rho\,\vect{u}^{n}) +  \underbrace { \dive(\vect{u}^{n} \otimes
\rho\,\delta \vect{u})}_{\text {terme couplant lin\'eaire}} +  \underbrace { \dive(\delta \vect{u} \otimes
\rho\,\delta \vect{u})}_{\text {terme couplant et non lin\'eaire}}\\
\end{array} 
\end{equation}
Les deux derniers termes de l'expression (\ref{Base_Preduv_Conv_exact}) sont {\em a priori} n�glig�s
de mani�re � obtenir un syst\`eme en vitesse qui soit d\'ecoupl\'e et donc,
�viter l'inversion d'une matrice pouvant \^etre de tr\`es grande taille. Ces
deux termes peuvent n�anmoins �tre pris en compte de mani�re plus ou moins
approch�e par extrapolation explicite du flux de masse en $n+\theta_F$ (pour le
terme couplant lin�aire provenant de la convection de $\vect{u}^{n}$ par $\delta
\vect{u}$) et utilisation d'un point-fixe par sous it�ration sur le sous
programme \fort{navsto} (pour le terme non-lin�aire, en sp�cifiant $\var{NTERUP}>1$).

L'�quation (\ref{Base_Preduv_eqqdm}) est discr�tis�e au temps $n+\theta$ � l'aide d'un
$\theta$-sch�ma, et le tenseur des pertes de charges d�compos� en une partie
diagonale $\tens{K}_{d}$ et une extradiagonale $\tens{K}_{e}$ (soit
 $\tens{K}_{pdc}=\tens{K}_{d}+\tens{K}_{e}$).\\
$\bullet$ La pression est suppos�e connue en $n-1+\theta$ (d�calage temporel
pression-vitesse) et le gradient naturellement calcul� � cet instant.\\ 
$\bullet$ Les termes sources de viscosit� secondaire, de gradient transpos\'e,
ceux provenant du mod�le de turbulence\footnote{except� $\dive (\mu_t\ (\ggrad
\underline {u}))$}, $\rho\,\tens{K}_{\,e}\ \underline{u}$, $(\rho -\rho_0)
\underline {g}$ ainsi que $\underline{T}_{s}^{\,exp}$ et
$\Gamma\,\underline{u}_{\,i}$ sont pris de mani�re explicite au temps $n$, ou
extrapol�s suivant le sch�ma en temps choisi pour les propri�t�s physique et les
termes sources.\\ 
$\bullet$ Les termes sources $\underline{u}\,\,\dive (\rho\,\underline {u})$,
$\Gamma\,\,\underline{u}$, $T_{s}^{\,imp}\,\,\underline{u}$ et
$-\rho\,\tens{K}_{\,d}\,\,\underline{u}$ sont implicit�s est calcul�s �
l'instant $n+\theta$.\\ 
$\bullet$ Le terme de diffusion $\dive (\mu_{\,tot}\,\ggrad \underline{u})$ est
implicit� : la vitesse est prise � l'instant $n+\theta$ et la viscosit�
explicit�e ou extrapol�e.\\ 
$\bullet$ Enfin, le terme de convection en $\dive(\,\underline{u} \otimes
(\rho\underline{u})\,)$ est implicit� : la composante r�solue de la vitesse est
prise en $n+\theta$, et le flux de masse, explicit�, ou extrapol� en
$n+\theta_F$. 

Par souci de clart�, on suppose, en l'absence d'indication, que les propri�tes
physiques $\Phi$ ($\rho,\,\mu_{tot},\,...$) et le flux de masse
$(\rho\underline{u})$ sont pris respectivement aux instants $n+\theta_\Phi$ et
$n+\theta_F$, o� $\theta_\Phi$ et $\theta_F$ d�pendent des sch�mas en temps
sp�cifiquement utilis�s pour ces grandeurs\footnote{cf. \fort{introd}}. 

La discr�tisation temporelle de l'�quation (\ref{Base_Preduv_eqqdm}) s'�crit alors comme suit : 

\begin{equation}\label{Base_Preduv_eq_di1}
 \begin{array}{c}
\displaystyle \rho\,\ \frac{ \underline {\widetilde{u}}^{n+1} -\underline {u}^{n} }
{\Delta t} + \dive(\,\underline{\widetilde{u}}^{n+\theta} \otimes (\rho\underline{u})\,) -\dive
(\mu_{\,tot}\,\ggrad \underline{\widetilde{u}}^{n+\theta}) =
\\
\displaystyle
 - \grad p^{n-1+\theta} + \dive (\rho\,\underline {u})\,\underline{\widetilde{u}}^{n+\theta} +(\Gamma\,\underline{u}_{\,i})^{n+\theta_S}-\Gamma^n\,\,\underline{\widetilde{u}}^{n+\theta}
\\
\begin{array}{c}
\displaystyle
- \rho\,\tens{K}_{\,d}^{n}\,\,\underline{\widetilde{u}}^{n+\theta} - (\rho\,\tens{K}_{\,e}\ \underline{u})^{n+\theta_S} + (\underline{T}_{s}^{\,exp})^{\,n+\theta_S} + T_{s}^{\,imp}\,\,\underline{\widetilde{u}}^{n+\theta}
\\
\displaystyle
+[\dive (\mu_{\,tot}\,^t\ggrad \underline {u})]^{n+\theta_S}-\frac {2} {3}[\,\grad (\mu_{\,tot}\,\dive \underline {u})]^{n+\theta_S} + (\rho -\rho_0) \underline {g}
 - (\underline{turb})^{n+\theta_S}
\end{array}
\end{array}
\end{equation}
o\`u, par souci de simplification, on a pos\'e :
\begin{equation}
\mu_{\,tot}=
\begin{cases}
\mu+\mu_t & \text{pour les mod�les � viscosit� turbulente ou en LES}, \\
\mu & \text{pour les mod�les au second ordre ou en laminaire}
\end{cases} \ 
\end{equation}
\\
et :
\begin{equation}
\underline{turb}^{n}=
\begin{cases}
\displaystyle\frac {2}{3}\grad (\rho^{n}\,k^{n}) & \text{pour les mod�les � viscosit� turbulente}, \\
\dive(\rho^{n}\,\tens{R}^n) & \text{pour les mod�les au second ordre},\\
0 & \text{en laminaire ou en LES}\\
\end{cases}
\end{equation}
Par analogie avec l'�criture du $\theta$-sch�ma pour une variable scalaire, $\,
\underline {\widetilde{u}}^{n+\theta}$ est interpol�e � partir de la vitesse
pr�dite $\underline {\widetilde{u}}^{n+1}$ de la mani\`ere suivante\footnote{si
$\theta=1/2$, ou qu'une extrapolation est utilis�e, l'ordre 2 n'est obtenu que si
le pas de temps $\Delta t$ est uniforme en temps et en espace.}~: 
\begin{equation}
\underline {\widetilde{u}}^{n+\theta}=\theta\, \underline
{\widetilde{u}}^{n+1}+(1-\theta)\, \underline {u}^{n}\\ 
\end{equation}
Avec :
\begin{equation}
\left\{
\begin{array}{ll}
\theta = 1   & \text{Pour un sch\'ema de type Euler implicite d'ordre 1.}\\
\theta = 1/2 & \text{Pour un sch\'ema de type Cranck-Nicolson d'ordre 2.}\\
\end{array}
\right.
\end{equation}

L'�quation (\ref{Base_Preduv_eq_di1}) est alors r��crite sous la forme :

\begin{equation}\label{Base_Preduv_eq_di2}
\begin{array}{c}
\displaystyle \underbrace{\left(\frac{\rho}{\Delta t} -\theta \,\dive (\rho\,\underline {u}) +\theta \,\, \Gamma^n +
\theta \,\, \rho\,\tens{K}_{\,d}^n-\theta \,T_s^{\,imp} \right)}_{\displaystyle f_s^{imp}}\, (\underline {\,\widetilde{u}}^{n+1} -\underline {u}^{n})
\\
 +\, \theta\, \dive(\underline {\widetilde{u}}^{n+1} \otimes (\rho\underline{u}))-\, \theta\,\dive (\mu_{\,tot}\,\ggrad \underline {\widetilde{u}}^{n+1}) =
\\
-\,(1-\theta)\, \dive(\underline {u}^{n} \otimes (\rho\underline{u})) +\,(1-\theta)\,\dive (\mu_{\,tot}\,\ggrad \underline {u}^{n})
\\
f_s^{exp}\left\{
\begin{array}{c}
\displaystyle 
- \grad p^{n-1+\theta} + \dive (\rho\,\underline {u})\,\underline{u}^{n} +\,(\,\Gamma^{n}\,\underline{u}_{\,i}\,)^{n+\theta_S}- \Gamma^n\,\,\underline{u}^{n}
\\
\displaystyle
-(\,\rho\,\tens{K}_{\,e}\ \underline{u}\,)^{n+\theta_S} -\rho\,\tens{K}_{\,d}^n\ \underline{u}^{n}+ (\underline{T}_{s}^{\,exp})^{\,n+\theta_S} + T_s^{\,imp}\,\,\underline {u}^{n} 
\\
\displaystyle
+[\dive (\mu_{\,tot}\,^t\ggrad \underline {u}\,)]^{n+\theta_S}-\frac {2} {3}[\,\grad (\mu_{\,tot}\,\dive \underline {u}\,)]^{n+\theta_S} + (\rho -\rho_0) \underline {g}-(\underline{turb})^{n+\theta_S}
\end{array}
\right.
\end{array}
\end{equation}

d'o� l'�quation r�solue par le sous-programme \fort{codits} :
\begin{equation}\begin{array}{c}
\displaystyle
f_s^{\,imp}(\underline {\widetilde{u}}^{n+1}-\underline {u}^{n}) + \theta\, \dive(\underline{\widetilde{u}}^{n+1} \otimes (\rho
\underline{u})) - \theta\,\dive (\,\mu_{\,tot}\,\ggrad \underline{\widetilde{u}}^{n+1}) = 
\\\\
\displaystyle
-(1-\theta)\,\dive(\underline{u}^{n} \otimes (\rho \underline{u}))+(1-\theta)\,\dive (\,\mu_{\,tot}\,\ggrad \underline{u}^{n})
+ \underline{f}_{\,s}^{\,exp}
\end{array}
\end{equation}
La m\'ethode de discr\'etisation spatiale est d\'evelopp\'ee dans le sous-programme \fort{codits}.\\



\minititre{Remarques :}
{\tiny$\blacksquare$} Dans le cas standard sans extrapolation, le terme
$-\,T_s^{\,imp}$ n'est ajout� � $f_s^{\,imp}$ que s'il est positif afin de ne
pas affaiblir la dominance de la diagonale de la matrice � inverser.\\ 
{\tiny$\blacksquare$} Si une extrapolation est utilis�e, par contre,
$\,T_s^{\,imp}$ est ajout� � $f_s^{\,imp}$ quel que soit son signe. En effet, l'id�e intuitive qui
consiste � prendre~: 
\begin{equation}
\begin{cases}
\displaystyle
(\underline{T}_{s}^{\,exp} + T_{s}^{\,imp}\,\underline {u})^{\,n+\theta_S} &
\text{si } T_{s}^{\,imp} > 0\\ 
\displaystyle
(\underline{T}_{s}^{\,exp})^{\,n+\theta_S} + T_{s}^{\,imp}\,\underline{u}^{n+\theta} &\text{sinon}\\
\end{cases}
\end{equation} 
aboutit � une incoh�rence dans le traitement si $T_s^{imp}$ change de signe
entre deux pas de temps.\\ 
{\tiny$\blacksquare$} la partie diagonale $\tens{K}_{\,d}$ du terme
de perte de charge est utilis�e dans $f_s^{\,imp}$. En fait, pour \^etre rigoureux,
il faudrait ne retenir que les contributions positives (point signal\'e dans le
sous-programme utilisateur associ\'e \fort{uskpdc}). Cette prise en compte sera \`a am\'eliorer.\\
{\tiny$\blacksquare$} Le terme $\theta\,\Gamma^{n}-\theta\,\dive
(\rho\,\underline {u})$ ne pose pas de probl�me pour la 
dominance de la diagonale de la matrice car il est exactement compens� par le
terme de convection (cf. \fort{covofi}). 


%                      Code_Saturne version 1.3
%                      ------------------------
%
%     This file is part of the Code_Saturne Kernel, element of the
%     Code_Saturne CFD tool.
%
%     Copyright (C) 1998-2007 EDF S.A., France
%
%     contact: saturne-support@edf.fr
%
%     The Code_Saturne Kernel is free software; you can redistribute it
%     and/or modify it under the terms of the GNU General Public License
%     as published by the Free Software Foundation; either version 2 of
%     the License, or (at your option) any later version.
%
%     The Code_Saturne Kernel is distributed in the hope that it will be
%     useful, but WITHOUT ANY WARRANTY; without even the implied warranty
%     of MERCHANTABILITY or FITNESS FOR A PARTICULAR PURPOSE.  See the
%     GNU General Public License for more details.
%
%     You should have received a copy of the GNU General Public License
%     along with the Code_Saturne Kernel; if not, write to the
%     Free Software Foundation, Inc.,
%     51 Franklin St, Fifth Floor,
%     Boston, MA  02110-1301  USA
%
%-----------------------------------------------------------------------
%

%%%%%%%%%%%%%%%%%%%%%%%%%%%%%%%%%%
%%%%%%%%%%%%%%%%%%%%%%%%%%%%%%%%%%
\section{Mise en \oe uvre}
%%%%%%%%%%%%%%%%%%%%%%%%%%%%%%%%%%
%%%%%%%%%%%%%%%%%%%%%%%%%%%%%%%%%%
La num\'ero de la phase trait\'ee fait partie des arguments de \fort{turrij}. On
omettra volontairement de le pr\'eciser dans ce qui suit, on indiquera par $(\ )$ la
notion de tableau s'y rattachant.

\etape{Calcul des termes de production $\tens{\mathcal{P}}$}
\begin{itemize}
\item [$\star$] Initialisation \`a z\'ero du tableau \var{PRODUC} dimensionn\'e \`a $\var{NCEL}\times 6$.
\item [$\star$] On appelle trois fois \fort{grdcel} pour calculer les gradients des composantes de la vitesse $u$, $v$ et
$w$ prises au temps $n$.

Au final, on a :\\
$\displaystyle
\begin{array} {ll}
\var{PRODUC(1,IEL)} = & \displaystyle - 2 \left[ R_{11}^{\,n} \frac{\partial u^{\,n}} {\partial x} +R_{12}^{\,n} \frac{\partial u^{\,n}} {\partial y}+R_{13}^{\,n} \frac{\partial u^{\,n}} {\partial z} \right] \text{        (production de $R_{11}^{\,n}$)}\\
\var{PRODUC(2,IEL)} = & \displaystyle - 2 \left[ R_{12}^{\,n} \frac{\partial v^{\,n}} {\partial x} +R_{22}^{\,n} \frac{\partial v^{\,n}} {\partial y}+R_{23}^{\,n} \frac{\partial v^{\,n}} {\partial z} \right] \text{        (production de $R_{22}^{\,n}$)}\\
\var{PRODUC(3,IEL)} = & \displaystyle - 2 \left[ R_{13}^{\,n} \frac{\partial w^{\,n}} {\partial x} +R_{23}^{\,n} \frac{\partial w^{\,n}} {\partial y}+R_{33}^{\,n} \frac{\partial w^{\,n}} {\partial z} \right] \text{        (production de $R_{33}^{\,n}$)}\\
\var{PRODUC(4,IEL)} = & \displaystyle - \left[ R_{12}^{\,n} \frac{\partial u^{\,n}} {\partial x} +R_{22}^{\,n} \frac{\partial u^{\,n}} {\partial y}+R_{23}^{\,n} \frac{\partial u^{\,n}} {\partial z} \right] \\
& \displaystyle - \left[ R_{11}^{\,n} \frac{\partial v^{\,n}} {\partial x} +R_{12}^{\,n} \frac{\partial v^{\,n}} {\partial y}+R_{13}^{\,n} \frac{\partial v^{\,n}} {\partial z} \right] \text{        (production de $R_{12}^{\,n}$)} \\
\var{PRODUC(5,IEL)} = & \displaystyle - \left[ R_{13}^{\,n} \frac{\partial u^{\,n}} {\partial x} +R_{23}^{\,n} \frac{\partial u^{\,n}} {\partial y}+R_{33}^{\,n} \frac{\partial u^{\,n}} {\partial z} \right] \\
& \displaystyle - \left[ R_{11}^{\,n} \frac{\partial w^{\,n}} {\partial x} +R_{12}^{\,n} \frac{\partial w^{\,n}} {\partial y}+R_{13}^{\,n} \frac{\partial w^{\,n}} {\partial z} \right] \text{        (production de $R_{13}^{\,n}$)} \\
\var{PRODUC(6,IEL)} = & \displaystyle - \left[ R_{13}^{\,n} \frac{\partial v^{\,n}} {\partial x} +R_{23}^{\,n} \frac{\partial v^{\,n}} {\partial y}+R_{33}^{\,n} \frac{\partial v^{\,n}} {\partial z} \right] \\
& \displaystyle - \left[ R_{12}^{\,n} \frac{\partial w^{\,n}} {\partial x} +R_{22}^{\,n} \frac{\partial w^{\,n}} {\partial y}+R_{23}^{\,n} \frac{\partial w^{\,n}} {\partial z} \right]  \text{        (production de $R_{23}^{\,n}$)}
\end{array}
$
\end{itemize}

\etape{Calcul du gradient de la masse volumique $\rho^n$ prise au d\'ebut du pas
de temps courant\footnote{{\it i.e.} calcul\'ee \`a partir des
variables du pas de temps pr\'ec\'edent $n$ si n\'ecessaire.} $(n+1)$}
Ce calcul n'a lieu que si les termes de gravit\'e doivent \^etre pris en compte
($\var{IGRARI()} =1$).
\begin{itemize}
\item [$\star$] Appel de \fort{grdcel}  pour calculer le gradient de $\rho^n$
dans les trois directions de l'espace. Les conditions aux limites sur $\rho^n$
sont des conditions de Dirichlet puisque la valeur de $\rho^n$ aux faces de bord
$ik$ (variable \var{IFAC}) est connue et vaut $\rho_{\,b_{\,ik}}$. Pour \'ecrire les conditions aux limites
sous la forme habituelle, $$\rho_{\,b_{\,ik}} = \var{COEFA} + \var{COEFB}
\,\rho^n_{\,I'}$$ on pose alors $\var{COEFA}=
\var{PROPCE(IFAC,IPPROB(IROM(IPHAS)))}$ et $\var{COEFB} = \var{VISCB} = 0$.\\
$\var{PROPCE(1,IPPROB(IROM(IPHAS)))}$ (resp.$\var{VISCB}$) est utilis\'e en lieu
et place de l'habituel \var{COEFA} ($\var{COEFB}$), lors de l'appel \`a \fort{grdcel}.\\
On a donc :\\
$\displaystyle \var{GRAROX}= \frac{\partial \rho^n}{\partial x}\ $,$\displaystyle \ \var{GRAROY}= \frac{\partial
\rho^n}{\partial y}$ et $
\displaystyle \ \var{GRAROZ}= \frac{\partial \rho^n}{\partial z}\ $.

\end{itemize}

Le gradient de $\rho^n$ servira \`a calculer les termes de production par effets de gravit\'e si ces derniers sont pris en compte.

\etape{Boucle \var{ISOU} de $1$ \`a $6$ sur les tensions de Reynolds}
Pour $\var{ISOU} = 1,2,3,4,5,6$, on r\'esout respectivement et dans
l'ordre  les
\'equations de $R_{11}$, $R_{22}$, $R_{33}$, $R_{12}$, $R_{13}$ et $R_{23}$ par
l'appel au sous-programme \fort{resrij}.\\
La r\'esolution se fait par incr\'ement $\delta {R}_{ij}^{\,n+1,k+1}$ , en utilisant la m\^eme m\'ethode que
celle d\'ecrite dans le sous-programme \fort{codits}. On adopte ici les m\^emes notations.
\var{SMBR} est le second membre du syst\`eme \`a inverser, syst\`eme portant sur
les incr\'ements de la variable. \var{ROVSDT} repr\'esente la diagonale de la
matrice, hors convection/diffusion.\\
On va r\'esoudre l'\'equation (\ref{Base_Turrij_Eq_Temp_Rij}) sous forme incr\'ementale en
utilisant \fort{codits}, soit :
\begin{equation}\label{Base_Turrij_Eq_Temp_deltaRij}
\begin{array}{ll}
&\displaystyle \underbrace{\left(\frac {\rho^n_L}{\Delta t^n}
+ \rho^n_L \,C_1\,\frac{\varepsilon^n_L}{k^n_L}(1-\frac{\delta_{ij}}{3})
 - m^n_{\,lm} + \Gamma_L\,+ max(-\alpha^n_{R_{ij}},0)\right)\,|\Omega_l|}
_{\text {$\var{ROVSDT}$ contribuant
\`a la diagonale de la matrice simplifi\'ee de \fort{matrix}}}\,(\delta{R}_{ij}^{\,n+1,p+1})_{\,L}\\\\
&  \underbrace{+\sum\limits_{m\in Vois(l)}\displaystyle \left[
 m^n_{\,lm} \delta R_{ij,\,f_{\,lm}}^{\,n+1,p+1}
- (\mu^n_{\,lm} + \gamma^n_{\,lm})\
\frac{({\delta R}_{ij}^{\,n+1,p+1})_{M}-({\delta R}_{ij}^{\,n+1,p+1})_{L})}{\overline{L'M'}}\,
S_{\,lm} \right]}_{\text { convection upwind pur et diffusion non reconstruite
relatives \`a la matrice simplifi\'ee de \fort{matrix}\footnotemark}} \\
% voir le texte de la footmark plus bas
&= - \displaystyle\frac {\rho^n_L}{\Delta t^n}\,\left(\,(R^{\,n+1,p}_{ij})_L - (R^{\,n}_{ij})_L\,\right)\\
&-\,\underbrace{\displaystyle\int_{\Omega_l} \left(
\dive\,[\,(\rho\,\vect{u})^n\,R^{\,n+1,p}_{ij} - (\mu^n\,+ \gamma^n\,)\,
\grad{R^{\,n+1,p}_{ij}}\,]\right)\,d\Omega}_{\text {convection et diffusion
trait\'ees par \fort{bilsc2}}}\\
&+\displaystyle \int_{\Omega_l} \left[\,\mathcal{P}^{\,n+1,p}_{ij} + \mathcal{G}^{\,n+1,p}_{ij}
- \displaystyle\rho^n \,C_1\,\frac{\varepsilon^n}{k^n}\left[R^{\,n+1,p}_{ij}-
\frac{2}{3}\,k^n\,\delta_{ij}\right] + \phi^{\,n+1,p}_{ij,2} +
\phi^{\,n+1,p}_{ij,w}\,\right]\, d\Omega \\
& + \displaystyle\int_{\Omega_l} \left[- \frac{2}{3} \rho^n \varepsilon^n \delta_{ij}
 + \Gamma\,(\,R^{\,in}_{ij} - R^{\,n+1,p}_{ij}\,) +
\alpha^n_{R_{ij}}\,R^{\,n+1,p}_{ij}+ \beta^n_{R_{ij}}\right]\, d\Omega\\
&+ \sum\limits_{m\in
Vois(l)}\displaystyle \left[\ \tens{E}^n\,\grad{R}^{\,n+1,p}_{ij} \right]_{\,lm}\,.\,\vect{n}_{\,lm}S_{\,lm}\\
&+ \sum\limits_{m\in Vois(l)}\displaystyle \left[\
\tens{D}^n\,\grad{R}^{\,n+1,p}_{ij} \right]_{\,lm}\,.\,\vect{n}_{\,lm}S_{\,lm}\\
&- \sum\limits_{m\in Vois(l)} \gamma^n_{\,lm} \left( \grad{R}^{\,n+1,p}_{ij}\,.\,\vect{n}_{\,lm} \right)  S_{\,lm}\\
&+ \sum\limits_{m\in Vois(l)}  m^n_{\,lm}\,(R^{\,n+1,p}_{ij})_L\\
\end{array}
\end{equation}
% si on ne fait pas comme ca, il n'apparait pas
\footnotetext[\thefootnote]{Si $\var{IRIJNU} = 1$, on remplace  $\mu^n_{\,lm}$ par $(\mu +
\mu_t)^n_{\,lm}$ dans l'expression de la diffusion non reconstruite
associ\'ee \`a la matrice simplifi\'ee de \fort{matrix} ($\mu_t$ d\'esigne la
viscosit\'e turbulente calcul\'ee comme en $k-\varepsilon$).}

o\`u on rappelle :\\
pour $n$ donn\'e entier positif, on d\'efinit la suite
 $({R}_{ij}^{\,n+1,p})_{p \in \grandN}$
 par :
\begin{equation}\notag
\left\{\begin{array}{l}
{R}_{ij}^{\,n+1,0} = {R}_{ij}^{\,n}\\
{R}_{ij}^{\,n+1,p+1} = {R}_{ij}^{\,n+1,p} + \delta{R}_{ij}^{\,n+1,p+1} \\
\end{array}\right.
\end{equation}
$(\delta{R}_{ij}^{\,n+1,p+1})_{\,L}$ d\'esigne la valeur sur l'\'el\'ement
$\Omega_l$ du $\text{$(\,p+1\,)$-i\`eme}$ incr\'ement de ${R}_{ij}^{\,n+1}$,
$ m^n_{\,lm}$ le flux de masse \`a l'instant $n$ \`a travers la face $lm$,
$\delta R_{ij,\,f_{\,lm}}^{\,n+1,p+1}$ vaut $({\delta
R}_{ij}^{\,n+1,p+1})_{L}$  si $ m^n_{\,lm} \geqslant 0$, $({\delta
R}_{ij}^{\,n+1,p+1})_{M}$ sinon,
$\mathcal{P}^{\,n+1,p}_{ij}$, $\phi^{\,n+1,p}_{ij,2}$, $\phi^{\,n+1,p}_{ij,w}$ les valeurs
des quantit\'es associ\'ees correspondant \`a l'incr\'ement
$(\delta{R}_{ij}^{\,n+1,p})$.\\



Tous ces termes sont calcul\'es comme suit :
\begin{itemize}
\item Terme de gauche de l'\'equation (\ref{Base_Turrij_Eq_Temp_deltaRij})\\
Dans \fort{resrij} est calcul\'ee la variable \var{ROVSDT}. Les autres
termes sont compl\'et\'es par \fort{codits}, lors de la construction de la matrice simplifi\'ee , {\it via} un
appel au sous-programme \fort{matrix}. La quantit\'e
 $(\mu^n_{\,lm} + \gamma^n_{\,lm})$ \`a la face $lm$ est calcul\'ee lors de l'appel \`a
\fort{visort}.\\
\item Second membre de l'\'equation (\ref{Base_Turrij_Eq_Temp_deltaRij})\\
Le premier terme non d\'etaill\'e est calcul\'e par le sous-programme
\fort{bilsc2}, qui applique le sch\'ema convectif choisi par l'utilisateur, qui
reconstruit ou non selon le souhait de l'utilisateur les gradients intervenants
dans la convection-diffusion.\\
Les termes sans accolade sont, eux, compl\`etement explicites et ajout\'es au fur et
\`a mesure dans \var{SMBR} pour former
l'expression $f^{\,exp}_s$ de \fort{codits}.
\end{itemize}
On d\'ecrit ci-dessous les \'etapes de \fort{resrij} :
\begin{itemize}

\item DELTIJ = 1, pour $\var{ISOU} \leqslant 3$ et DELTIJ = 0  Si $\var{ISOU} >
3$. Cette valeur repr\'esente le symbole de Kroeneker $\delta_{ij}$.

\item Initialisation \`a z\'ero de \var{SMBR} (tableau contenant le second
membre) et \var{ROVSDT} (tableau contenant la diagonale de la matrice sauf celle
relative \`a la contribution de la
diagonale des op\'erateurs de convection et de diffusion lin\'earis\'es
\footnote{qui correspondent aux sch\'emas convectif upwind pur et diffusif sans
reconstruction.}), tous deux de dimension $\var{NCEL}$.

\item Lecture et prise en compte des termes sources utilisateur pour la variable $R_{ij}$

Appel \`a \fort{ustsri} pour charger les termes sources utilisateurs. Ils sont
stock\'es comme suit. Pour la cellule $\Omega_l$ de centre $L$, repr\'esent\'ee par $\var{IEL}$, on a :\\
\begin{equation}\notag
\left\{\begin{array}{lll}
&\var{ROVSDT(IEL)}&= |\Omega_l| \ \alpha_{R_{ij}}\\
&\var{SMBR(IEL)}&=|\Omega_l| \ \beta_{R_{ij}}\\
\end{array}\right.
\end{equation}
On affecte alors les valeurs ad\'equates au second membre \var{SMBR} et \`a la
diagonale \var{ROVSDT} comme suit :
\begin{equation}\notag
\left\{\begin{array}{lll}
&\var{SMBR(IEL)} &= \var{SMBR(IEL)} +\ |\Omega_l| \ \alpha_{R_{ij}} \ (R^n_{ij})_L \\
&\var{ROVSDT(IEL)}&= \text{max }(-\ |\Omega_l| \ \alpha_{R_{ij}},0)\\
\end{array}\right.
\end{equation}
La valeur de $ \var{ROVSDT}$ est ainsi calcul\'ee pour des raisons de stabilit\'e
num\'erique. En effet, on ne rajoute sur la diagonale que les valeurs positives,
ce qui correspond physiquement \`a impliciter les termes de rappel uniquement.
\item{Calcul du terme source de masse  si $\Gamma_L > 0$}

Appel de \fort{catsma} et incr\'ementation si n\'ecessaire de \var{SMBR} et
\var{ROVSDT} {\it via} :\\
\begin{equation}\notag
\left\{\begin{array}{lll}
\displaystyle \var{SMBR(IEL)} = \var{SMBR(IEL)} + |\Omega_l| \ \Gamma_L \
\left[(R^{\,in}_{ij})_L - (R^{\,n}_{ij})_L \right] \\
\displaystyle \var{ROVSDT(IEL)}=\var{ROVSDT(IEL)} + |\Omega_l| \ \Gamma_L
\end{array}\right.
\end{equation}
\item Calcul du terme d'accumulation de masse et du terme instationnaire

On stocke $\displaystyle \var{W1}= \int_{\Omega_l}\dive\,(\rho\,\vect{u})\,d\Omega$
calcul\'e par \fort{divmas} \`a l'aide des flux de masse aux faces internes
$ m^n_{\,lm}=\sum\limits_{m\in Vois(l)}{(\rho \vect{u})_{\,lm}^n} \text{.}\,
\vect{S}_{\,lm} $ (tableau \var{FLUMAS}) et des flux de masse aux bords  $ m^n_{\,b_{lk}} = \sum\limits_{k\in{\gamma_b(l)}}{(\rho \vect{u})_{\,{b}_{lk}}^n} \text{.}\,
\vect{S}_{\,{b}_{lk}} $ (tableau \var{FLUMAB}).
On incr\'emente ensuite \var{SMBR} et \var{ROVSDT}.
\begin{equation}\notag
\left\{\begin{array}{lll}
&\var{SMBR(IEL)} &= \var{SMBR(IEL)} + \var{ICONV}\  (R^n_{ij})_L\,(\displaystyle
\int_{\Omega_l}\dive\,(\rho\,\vect{u})\ d\Omega) \\
&\var{ROVSDT(IEL)}& = \var{ROVSDT(IEL)} +  \var{ISTAT}\,\displaystyle
\frac{\rho^n_L \ |\Omega_l|}{\Delta t^n} -  \var{ICONV}\ (\displaystyle
\int_{\Omega_l}\dive\,(\rho\,\vect{u})\ d\Omega) \\
\end{array}\right.
\end{equation}
\item Calcul des termes sources de production, des termes $\displaystyle
\phi_{\,ij,1}+\phi_{\,ij,2}$ et de la dissipation~$\displaystyle-\frac{2}{3} \varepsilon\,\delta_{\,ij}$ :

On effectue une boucle d'indice \var{IEL} sur les cellules $\Omega_l$ de centre $L$ :
\begin{itemize}
\item [$\Rightarrow$] $\displaystyle \var{TRPROD}= \frac{1}{2} (\mathcal{P}^n_{ii})_L = \frac{1}{2} \left[ \var{PRODUC(1,IEL)} +  \var{PRODUC(2,IEL)} +  \var{PRODUC(3,IEL)} \right] $
\item [$\Rightarrow$] $\displaystyle \var{TRRIJ }= \frac{1}{2} (R^n_{ii})_L $
\item [$\Rightarrow$] $\displaystyle \var{SMBR(IEL)} =\ \var{SMBR(IEL)}\ +$\\
$\ \displaystyle\rho^n_L |\Omega_l| \left[ \displaystyle
\frac{2}{3}\,\delta_{\,ij} \left( \ \displaystyle \frac{ C_2}{2}\,(\mathcal{P}^n_{ii})_L\ +
(C_1-1)\ \varepsilon^n_L\, \right)\right.$\\
$ + \left.\ (1-C_2) \ \var{PRODUC(ISOU,IEL)} -
\displaystyle C_1\ \frac{2\,\varepsilon^n_L}{(R^n_{ii})_L}\ (R^n_{ij})_L \right]$
\item [$\Rightarrow$] $\displaystyle \var{ROVSDT(IEL)} = \var{ROVSDT(IEL)} +
\rho^n_L \ |\Omega_l| \ (- \displaystyle \frac{1}{3} \ \,\delta_{\,ij} + 1) \ C_1
\ \frac{2\ \varepsilon^n_L}{(R^n_{ii})_L}$
\end{itemize}
\item Appel de \fort{rijech} pour le calcul des termes d'\'echo de paroi
 $\phi^n_{ij,w}$ si $\var{IRIJEC()}=1$ et ajout dans \var{SMBR}.\\
$\var{SMBR} = \var{SMBR} + \phi^n_{ij,w}$\\
Suivant son mode de calcul (\var{ICDPAR}), la distance � la paroi est directement accessible
par \var{RA(IDIPAR+IEL-1)} (\var{|ICDPAR|} = 1) ou bien
est calcul\'ee \`a partir de $\var{IA(IIFAPA(IPHAS)+IEL - 1)}$,
qui donne pour l'\'el\'ement $\var{IEL}$ le num\'ero de la face de bord
paroi la plus  proche (\var{|ICDPAR|} = 2). Ces tableaux ont \'et\'e renseign\'e une fois pour toutes au
d\'ebut de calcul.

\item  Appel de \fort{rijthe} pour le calcul des termes de gravit\'e $\mathcal{G}^n_{ij}$ et ajout dans \var{SMBR}.

Ce calcul n'a lieu que si $\var{IGRARI()} = 1$.
$ \var{SMBR} = \var{SMBR} + \mathcal{G}^n_{ij}$
\item Calcul de la partie explicite du terme de diffusion
 $\dive{\,\left[\tens{A}\,\grad{R}^{\,n}_{ij}\right]}$, plus pr\'ecis\'ement
des contributions du terme extradiagonal pris aux faces purement internes
(remplissage du tableau \var{VISCF}), puis aux faces de bord (remplissage du
tableau \var{VISCB}).
\begin{itemize}
\item [$\star$] Appel de \fort{grdcel} pour le calcul du gradient de
$R^{\,n}_{ij}$ dans chaque direction. Ces gradients sont respectivement
stock\'es dans les tableaux de travail \var{W1}, \var{W2} et \var{W3}.

\item [$\star$] boucle d'indice \var{IEL} sur les cellules $\Omega_l$ de centre
$L$ pour le
calcul de $\tens{E}^n\,\grad{R}^{\,n}_{ij}$ aux cellules dans un premier temps :\\
\begin{itemize}
\item [$\Rightarrow$] $\displaystyle \var{TRRIJ}= \frac{1}{2} (R^{\,n}_{ii})_L $
\item [$\Rightarrow$] $\displaystyle \var{CSTRIJ} = \rho^n_L\ C_S \ \displaystyle\frac{(R^n_{ii})_L}{2\,\varepsilon^n_L}$
\item [$\Rightarrow$] $\displaystyle \var{W4(IEL)} = \rho^n_L\ C_S\
\displaystyle\frac{(R^n_{ii})_L}{2\,\varepsilon^n_L} \left[\,(R^{\,n}_{12})_L \ \var{W2(IEL)} +
(R^{\,n}_{13})_L \ \var{W3(IEL)}\,\right]$
\item [$\Rightarrow$] $\displaystyle \var{W5(IEL)} = \rho^n_L\ C_S\
\displaystyle\frac{(R^n_{ii})_L}{2\,\varepsilon^n_L} \left[\,(R^{\,n}_{12})_L \ \var{W1(IEL)} +
(R^{\,n}_{23})_L \ \var{W3(IEL)}\,\right]$
\item [$\Rightarrow$] $\displaystyle \var{W6(IEL)} = \rho^n_L\ C_S\
\displaystyle\frac{(R^n_{ii})_L}{2\,\varepsilon^n_L} \left[\,(R^{\,n}_{13})_L \ \var{W1(IEL)} + (R^{\,n}_{23})_L \ \var{W2(IEL)}\,\right]$
\end{itemize}



\item [$\star$] Appel de \fort{vectds}\footnote{Le r\'esultat est stock\'e dans
\var{VISCF} et \var{VISCB}. Dans \fort{vectds}, les valeurs aux faces internes
sont interpol\'ees lin\'eairement sans reconstruction et \var{VISCB} est mis \`a
z\'ero.} pour assembler $\displaystyle\left[ \tens{E}^n\,\grad{R}^{\,n}_{ij}
\right]\,.\,\vect{n}_{\,lm}S_{\,lm}$ aux faces $lm$.
\item [$\star$] Appel de \fort{divmas} pour calculer la divergence du flux d\'efini par \var{VISCF} et \var{VISCB}.
Le r\'esultat est stock\'e dans \var{W4}.\\
Ajout au second membre \var{SMBR}.\\
\var{SMBR} = \var{SMBR} + \var{W4}
\end{itemize}

A l'issue de cette \'etape, seule la partie extradiagonale de la diffusion prise
enti\`erement explicite~:
 $$\sum\limits_{m\in
Vois(l)}\left[\ \tens{E}^n\,\grad{R}^{\,n}_{ij} \right]_{\,lm}\,.\,\vect{n}_{\,lm}S_{\,lm}$$ a \'et\'e calcul\'ee.\\

\item Calcul de la partie diagonale du terme de diffusion\footnote{Seule la
composante normale  du  gradient de $R_{ij}$ aux faces sera implicite.} :\\
Comme on l'a d\'eja vu, une partie de cette contribution sera trait\'ee en
implicite (celle relative \`a la composante normale du gradient) et donc
ajout\'ee au second membre par \fort{bilsc2} ; l'autre
partie sera explicite et prise en compte dans $f_s^{\,exp}$.
\begin{itemize}
\item [$\star$] On effectue une boucle d'indice \var{IEL} sur les cellules
$\Omega_l$ de centre $L$ :
\begin{itemize}
\item [$\Rightarrow$] $\displaystyle \var{TRRIJ }= \frac{1}{2} (R^{\,n}_{ii})_L $
\item [$\Rightarrow$] $\displaystyle \var{CSTRIJ} = \rho^n_L \ C_S \ \frac{(R^{\,n}_{ii})_L}{2\,\varepsilon^n_L}$
\item [$\Rightarrow$] $\displaystyle \var{W4(IEL)} = \rho^n_L \ C_S \
\frac{(R^{\,n}_{ii})_L}{2\,\varepsilon^n_L} \ (R^{\,n}_{11})_L$
\item [$\Rightarrow$] $\displaystyle \var{W5(IEL)} = \rho^n_L \ C_S \ \frac{(R^{\,n}_{ii})_L}{2\,\varepsilon^n_L}\ (R^n_{22})_L$
\item [$\Rightarrow$] $\displaystyle \var{W6(IEL)} = \rho^n_L \ C_S \ \frac{(R^{\,n}_{ii})_L}{2\,\varepsilon^n_L} \ (R^n_{33})_L$
\end{itemize}

%\item Traitement du parall\'elisme et de la p\'eriodicit\'e.

\item [$\star$] On effectue une boucle d'indice \var{IFAC} sur les faces
purement internes $lm$ pour remplir le tableau \var{VISCF} :
\begin{itemize}
\item [$\Rightarrow$] Identification des cellules $\Omega_l$ et $\Omega_m$ de
centre respectif $L$ (variable \var{II}) et $M$ (variable \var{JJ}), se trouvant de chaque c\^ot\'e de la face
$lm$\footnote{La normale \`a la face est orient\'ee de L vers M.}.
\item [$\Rightarrow$] Calcul du carr\'e de la surface de la face. La valeur est
stock\'ee dans le tableau \var{SURFN2}.
\item [$\Rightarrow$] Interpolation du gradient de $R^{\,n}_{ij}$ \`a la face
$lm$ (gradient facette $\left[\grad{R}^{\,n}_{ij}\right]_{\,lm}$) :
\begin{equation}\notag
\left\{\begin{array}{ll}
\var{GRDPX} &= \displaystyle \frac{1}{2} \left(\var{W1(II)} + \var{W1(JJ)}
\right) \\
&\\
\var{GRDPY} &= \displaystyle \frac{1}{2} \left(\var{W2(II)} + \var{W2(JJ)} \right) \\
&\\
\var{GRDPZ} &= \displaystyle \frac{1}{2} \left(\var{W3(II)} + \var{W3(JJ)} \right)
\end{array}\right.
\end{equation}
\item [$\Rightarrow$] Calcul du gradient de $R^{\,n}_{ij}$ normal \`a la face
$lm$, $\left[\grad{R}^{\,n}_{ij}\right]_{\,lm}.\vect{n}_{\,lm}\,S_{\,lm}$.\\

$\displaystyle \var{GRDSN} =  \var{GRDPX} \ \var{SURFAC(1,IFAC)} + \var{GRDPY} \ \var{SURFAC(2,IFAC)} +  \var{GRDPZ} \ \var{SURFAC(3,IFAC)}$
$\var{SURFAC}$ est le vecteur surface de la face \var{IFAC}.


\item [$\Rightarrow$] calcul de
 $\left[\grad{R^{\,n}_{ij}} - (\grad
R^{\,n}_{ij}\,.\,\vect{n}_{\,lm})\vect{n}_{\,lm}\right]$, les vecteurs \'etant
calcul\'es \`a la face $lm$ :
\begin{equation}\notag
\left\{\begin{array}{lll}
&\displaystyle \var{GRDPX} &= \var{GRDPX} - \displaystyle\frac{\var{GRDSN}}{\var{SURFN2}} \ \var{SURFAC(1,IFAC)}\\
&&\\
&\displaystyle \var{GRDPY} &= \var{GRDPY} - \displaystyle\frac{\var{GRDSN}}{\var{SURFN2}} \ \var{SURFAC(2,IFAC)} \\
&&\\
&\displaystyle \var{GRDPZ} &= \var{GRDPZ} - \displaystyle\frac{\var{GRDSN}}{\var{SURFN2}} \ \var{SURFAC(3,IFAC)}
\end{array}\right.
\end{equation}
\item [$\Rightarrow$] finalisation du calcul de l'expression totalement
explicite
 $$\left[ \tens{D}^n\,\left( \grad{R^{\,n}_{ij}} - (\grad R^{\,n}_{ij}\,.\,\vect{n}_{\,lm})\,\vect{n}_{\,lm}\right) \right]\,.\,\vect{n}_{\,lm}$$
\begin{equation}\notag
\begin{array} {ll}
\displaystyle \var{VISCF} = &
 \displaystyle\frac{1}{2} (\ \var{W4(II)} +\ \var{W4(JJ)}) \ \var{GRDPX} \
\var{SURFAC(1,IFAC)})\ + \\
&\\
&  \displaystyle\frac{1}{2} (\ \var{W5(II)} +\ \var{W5(JJ)}) \ \var{GRDPY} \
\var{SURFAC(2,IFAC)})\ + \\
&\\
&  \displaystyle\frac{1}{2} (\ \var{W6(II)} +\ \var{W6(JJ)}) \ \var{GRDPZ} \ \var{SURFAC(3,IFAC)})
\end{array}
\end{equation}
\end{itemize}

\item [$\star$] Mise \`a z\'ero du tableau \var{VISCB}.

\item [$\star$] Appel de \fort{divmas} pour calculer la divergence de~:
 $$\tens{D}^{\,n}\,\left( \grad{R^{\,n}_{ij}} - (\grad R^{\,n}_{ij}\,.\,\vect{n}_{\,lm})\vect{n}_{\,lm}\right)$$ d\'efini aux faces dans \var{VISCF} et \var{VISCB}.

Le r\'esultat est stock\'e dans le tableau \var{W1}.\\
Ajout au second membre \var{SMBR}.\\
$\var{SMBR} = \var{SMBR} + \var{W1}$
\end{itemize}
\item Calcul de la viscosit\'e orthotrope $\gamma^n_{\,lm}$ \`a la face $lm$ de la variable principale
$R^{\,n}_{ij}$\\
Ce calcul permet au sous-programme \fort{codits} de compl\'eter le second membre
\var{SMBR} par :
\begin{equation}
\begin{array} {ll}
& \sum\limits_{m\in Vois(l)}
\mu^n_{\,lm}\,\left(\grad{R}^{\,n}_{ij}\,.\,\vect{n}_{\,lm}\right)S_{\,lm}
 + \sum\limits_{m\in Vois(l)} \left(\grad{R}^{\,n}_{ij}
\,.\,\vect{n}_{\,lm}\right)\left[\tens{D}^{\,n}\,\vect{n}_{\,lm}\right]_{\,lm}\,.\,\vect{n}_{\,lm}\
S_{\,lm}\\
& = \sum\limits_{m\in Vois(l)}(\,\mu^n_{\,lm}\, + \,\gamma^n_{\,lm}\,)\,\left(\grad{R}^{\,n}_{ij}\,.\,\vect{n}_{\,lm}\right)S_{\,lm}
\end{array}
\end{equation}
sans pr\'eciser la nature de la face $lm$, {\it via} l'appel \`a \fort{bilsc2}  et de disposer de la quantit\'e
$(\mu^n_{\,lm}\, + \gamma^n_{\,lm})$ pour construire sa
matrice simplifi\'ee.\\
\begin{itemize}
\item [$\star$] On effectue une boucle d'indice \var{IEL} sur les cellules
$\Omega_l$ :
\begin{itemize}
\item [$\Rightarrow$] $\displaystyle \var{TRRIJ }= \frac{1}{2} (R^{\,n}_{ii})_L $
\item [$\Rightarrow$] $\displaystyle \var{RCSTE} = \rho^n_L \ C_S \ \frac{ (R^{\,n}_{ii})_L}{2\,\varepsilon^n_L} $
\item [$\Rightarrow$] $\displaystyle \var{W1(IEL)} = \mu^n +\rho^n_L \ C_S \ \frac{
(R^{\,n}_{ii})_L}{2\,\varepsilon^n_L}\ (R^n_{11})_L$
\item [$\Rightarrow$] $\displaystyle \var{W2(IEL)} = \mu^n + \rho^n_L \ C_S \ \frac{ (R^{\,n}_{ii})_L}{2\,\varepsilon^n_L}\ (R^n_{22})_L$
\item [$\Rightarrow$] $\displaystyle \var{W3(IEL)} = \mu^n + \rho^n_L \ C_S \ \frac{ (R^{\,n}_{ii})_L}{2\,\varepsilon^n_L}\ (R^n_{33})_L$
\end{itemize}

\item [$\star$] Appel de \fort{visort} pour calculer la viscosit\'e orthotrope
\footnote{Comme dans le sous-programme \fort{viscfa}, on multiplie la viscosit\'e par
$\displaystyle \frac{S_{\,lm}}{\overline{L'M'}}$, o\`u $S_{\,lm}$ et
$\overline{L'M'}$ repr\'esentent respectivement la surface de la face $lm$ et la
mesure alg\'ebrique du segment reliant les projections des centres des cellules
voisines sur la normale \`a la face. On garde dans ce sous-programme  la possibilit\'e d'interpoler la viscosit\'e aux cellules lin\'eairement ou d'utiliser une moyenne harmonique. La viscosit\'e au bord est celle de la cellule de bord correspondante.}
$ \gamma^n_{\,lm} = (\tens{D}^{\,n}\,\vect{n}_{\,lm}).\vect{n}_{\,lm}$ aux faces $lm$

Le r\'esultat est stock\'e dans les tableaux \var{VISCF} et \var{VISCB}.
\end{itemize}

\item appel de \fort{codits} pour la r\'esolution de l'\'equation de
convection/diffusion/termes sources de la variable $R_{ij}$. Le terme source est
\var{SMBR}, la viscosit\'e \var{VISCF} aux faces purement internes (
resp. \var{VISCB} aux faces de bord) et \var{FLUMAS} le flux de masse interne
 ( resp. \var{FLUMAB} flux de masse au bord). Le r\'esultat est la variable $R_{ij}$ au temps
$n+1$, donc $R^{\,n+1}_{ij}$. Elle est stock\'ee dans le tableau \var{RTP} des
variables mises \`a jour.

\end{itemize}

\etape{Appel de \fort{reseps} pour la r\'esolution de la variable $\varepsilon$}

On d\'ecrit ci-dessous le sous-programme \fort{reseps}, les commentaires du sous-programme \fort{resrij} ne seront pas r\'ep\'et\'es puisque les deux sous-programmes ne diff\`erent que par leurs termes sources.

\begin{itemize}
\item Initialisation \`a z\'ero de \var{SMBR} et \var{ROVSDT}.

\item{Lecture et prise en compte des termes sources utilisateur pour la variable $\varepsilon$ :}

Appel de \fort{ustsri} pour charger les termes sources utilisateurs. Ils sont
stock\'es dans les tableaux suivants :\\
pour la cellule $\Omega_l$ repr\'esent\'ee par $\var{IEL}$ de centre $L$, on a :
\begin{equation}\notag
\left\{\begin{array}{lll}
&\var{ROVSDT(IEL)}&= |\Omega_l| \ \alpha_{\varepsilon}\\
&\var{SMBR(IEL)}&=|\Omega_l| \ \beta_{\varepsilon}\\
\end{array}\right.
\end{equation}
On affecte alors les valeurs ad\'equates au second membre \var{SMBR} et \`a la
diagonale \var{ROVSDT} comme suit :
\begin{equation}\notag
\left\{\begin{array}{lll}
&\var{SMBR(IEL)} &= \var{SMBR(IEL)} +\ |\Omega_l| \ \alpha_{\,\varepsilon} \
\varepsilon^n_L \\
&\var{ROVSDT(IEL)}&= \text{max }(-\ |\Omega_l| \ \alpha_{\,\varepsilon},0)\\
\end{array}\right.
\end{equation}

\item{Calcul du terme source de masse si $\Gamma_L > 0$ :
\begin{equation}\notag
\left\{\begin{array}{lll}
&\displaystyle \var{SMBR(IEL)} = \var{SMBR(IEL)} + |\Omega_l| \ \Gamma_L \
(\varepsilon^{\,in}_L -\varepsilon^n_L) \\
&\displaystyle \var{ROVSDT(IEL)}= \var{ROVSDT(IEL)} + |\Omega_l| \ \Gamma_L
\end{array}\right.
\end{equation}
\item Calcul du terme d'accumulation de masse et du terme instationnaire \\
On stocke $\displaystyle \var{W1}= \int_{\Omega_l}\dive\,(\rho\,\vect{u})\,d\Omega$
calcul\'e par \fort{divmas} \`a l'aide des flux de masse internes et aux bords.\\
On incr\'emente ensuite \var{SMBR} et \var{ROVSDT}.
\begin{equation}\notag
\left\{\begin{array}{lll}
&\var{SMBR(IEL)} &= \var{SMBR(IEL)} + \var{ICONV}\ \varepsilon^n_L\,(\displaystyle
\int_{\Omega_l}\dive\,(\rho\,\vect{u})\ d\Omega) \\
&\var{ROVSDT(IEL)}& = \var{ROVSDT(IEL)} +  \var{ISTAT}\,\displaystyle
\frac{\rho^n_L \ |\Omega_l|}{\Delta t^n} -  \var{ICONV}\ (\displaystyle
\int_{\Omega_l}\dive\,(\rho\,\vect{u})\ d\Omega) \\
\end{array}\right.
\end{equation}

\item Traitement du terme de production
 $\displaystyle \rho\,C_{\varepsilon_1}\,\frac{\varepsilon}{k}\,\mathcal{P}$
 et du terme de dissipation $-\,\displaystyle \rho\,C_{\varepsilon_2}\,\frac{\varepsilon}{k}\,\varepsilon$ \\
pour cela, on effectue une boucle d'indice \var{IEL} sur les cellules $\Omega_l$
de centre $L$ :
\begin{itemize}
\item [$\Rightarrow$] $\displaystyle \var{TRPROD}= \frac{1}{2} (\mathcal{P}^n_{ii})_L = \frac{1}{2} \left[ \var{PRODUC(1,IEL)} +  \var{PRODUC(2,IEL)} +  \var{PRODUC(3,IEL)} \right] $
\item [$\Rightarrow$] $\displaystyle \var{TRRIJ }= \frac{1}{2} (R^n_{ii})_L $
\item [$\Rightarrow$] $\displaystyle \var{SMBR(IEL)} = \var{SMBR(IEL)} + \rho^n_L
|\Omega_l| \left[ -C_{\varepsilon_2} \ \frac{2\,(\varepsilon^n_L)^2}{(R^n_{ii})_L} + C_{\varepsilon_1} \ \frac{\varepsilon^n_L}{(R^n_{ii})_L}\ (\mathcal{P}^n_{ii})_L \right] $
\item [$\Rightarrow$] $\displaystyle \var{ROVSDT(IEL)} = \var{ROVSDT(IEL)} + C_{\varepsilon_2} \ \rho^n_L \ |\Omega_l| \ \frac{2\,\varepsilon^n_L}{(R^n_{ii})_L}$
\end{itemize}

\item Appel de \fort{rijthe} pour le calcul des termes de gravit\'e $\mathcal{G}^n_{\varepsilon}$ et ajout dans \var{SMBR}.

$ \var{SMBR} = \var{SMBR} + \mathcal{G}^n_{\varepsilon}$\\
Ce calcul n'a lieu que si $\var{IGRARI()} = 1$.

\item Calcul de la diffusion de $\varepsilon$ \\
 Le terme $\dive \left[\mu\, \grad(\varepsilon) + \tens{A'}\,\grad(\varepsilon)
\right]$ est calcul\'e exactement de la m\^eme mani\`ere que pour les tensions
de Reynolds $R_{ij}$ en rempla\c cant $\tens{A}$ par $\tens{A'}$.

\item Appel de \fort{codits} pour la r\'esolution de l'\'equation de
convection/diffusion/termes sources de la variable principale $\varepsilon$. Le
r\'esultat $\varepsilon^{\,n+1}$ est stock\'e dans le tableau \var{RTP} des
variables mises \`a jour.
}
\end{itemize}

\etape{clippings finaux}
On passe enfin dans le sous-programme  \fort{clprij} pour faire un clipping \'eventuel
des variables $R^{\,n+1}_{ij}$ et $\varepsilon^{\,n+1}$. Le sous-programme
\fort{clprij} est appel\'e\footnote{L'option
$\var{ICLIP} = 1$ consiste en un clipping minimal des variables $R_{ii}$ et
$\varepsilon$ en prenant la valeur absolue de ces variables puisqu'elles ne
peuvent \^etre que positives.} avec $\var{ICLIP} = 2$ . Cette option
\footnote{Quand la valeur des grandeurs $R_{ii}$ ou $\varepsilon$ est
n\'egative, on la remplace par le minimum entre sa valeur absolue et (1,1)
fois la valeur obtenue au pas de temps pr\'ec\'edent.} contient l'option $\var{ICLIP} = 1$  et permet de v\'erifier l'in\'egalit\'e de Cauchy-Schwarz sur les grandeurs extra-diagonales du tenseur $\tens{R}$ (pour $i \neq j$, $|R_{ij}|^2 \le R_{ii} R_{jj}$).


%%%%%%%%%%%%%%%%%%%%%%%%%%%%%%%%%%
%%%%%%%%%%%%%%%%%%%%%%%%%%%%%%%%%%
\section{Points \`a traiter}
%%%%%%%%%%%%%%%%%%%%%%%%%%%%%%%%%%
%%%%%%%%%%%%%%%%%%%%%%%%%%%%%%%%%%
Sauf mention explicite, $\phi$ repr\'esentera une tension de Reynolds ou la dissipation turbulente ($\phi = R_{ij} \ \text{ou} \ \varepsilon$).

\begin{itemize}
\item {La vitesse utilis\'ee pour le calcul de la production est explicite. Est-ce qu'une implicitation peut am\'eliorer la pr\'ecision temporelle de $\phi$ \footnote{Cette remarque peut \^etre g\'en\'eralis\'ee. En effet, peut-on envisager d'actualiser les variables d\'ej\`a r\'esolues (sans r\'eactualiser les variables turbulentes apr\`es leur r\'esolution)? Ceci obligerait \`a modifier les sous-programmes tels que \fort{condli} qui sont appel\'es au d\'ebut de la boucle en temps.} ?}
\item {Dans quelle mesure le terme d'\'echo de paroi est-il valide ? En effet, ce terme est remis en question par certains auteurs.}
\item {On peut envisager la r\'esolution d'un syst\`eme hyperbolique pour les
tensions de Reynolds afin d'introduire un couplage avec le champ de vitesse.}
\item {Le flux au bord \var{VISCB} est annul\'e dans le sous-programme
\fort{vectds}. Peut-on envisager de mettre au bord la valeur de la variable
concern\'ee \`a la cellule de bord correspondant? De m\^eme, il faudrait se
pencher sur les hypoth\`eses sous-jacentes \`a l'annulation des contributions
aux bords de \var{VISCB} lors du calcul de : $$\left[ \tens{D}^n\,\left( \grad{R^{\,n}_{ij}} - (\grad R^{\,n}_{ij}\,.\,\vect{n}_{\,lm})\,\vect{n}_{\,lm}\right) \right]\,.\,\vect{n}_{\,lm}.$$}
\item {Un probl\`eme de pond\'eration appara\^\i t plus g\'en\'eralement. Si on prend la partie explicite de $\tens{D}\,\grad(\phi)$, la pond\'eration aux faces internes utilise le coefficient $\displaystyle\frac{1}{2}$ avec pond\'eration s\'epar\'ee de $\tens{D}$ et $\grad(\phi)$, alors que pour $\tens{E}\,\grad(\phi)$, on calcule d'abord ce terme aux cellules pour ensuite l'interpoler lin\'eairement aux faces \footnote{Cette interpolation se fait dans \fort{vectds} avec des coefficients de pond\'eration aux faces.}. Ceci donne donc deux types d'interpolations pour des termes de m\^eme nature.}
\item {On laisse la possibilit\'e dans \fort{visort} d'utiliser une moyenne
harmonique aux faces. Est-ce que ceci est valable puisque les interpolations
utilis\'ees lors du calcul de la partie explicite de $\tens{A}\,\grad{\phi}$
sont des moyennes arithm\'etiques ?}
\item {Les techniques adopt\'ees lors du clipping sont \`a revoir.}
\item {On utilise dans le cadre du mod\`ele $\displaystyle R_{ij}-\varepsilon $ une semi-implicitation de termes comme $\displaystyle \phi_{ij,1}$ ou $\displaystyle -\rho\,C_{\varepsilon_2}\,\frac{\varepsilon}{k}\,\varepsilon$. On peut envisager le m\^eme type d'implicitation dans \fort{turbke} m\^eme en pr\'esence du couplage $\displaystyle k-\varepsilon$.}
\item L'adoption d'une r\'esolution d\'ecoupl\'ee fait perdre l'invariance par rotation.
\item La formulation et l'implantation des conditions aux limites de paroi
devront \^etre v\'erifi\'ees. En effet, il semblerait que, dans certains cas, des ph\'enom\`enes
``oscillatoires'' apparaissent, sans qu'il soit ais\'e d'en d\'eterminer la cause.
\item L'implicitation partielle (du fait de la r\'esolution d\'ecoupl\'ee) des
conditions aux limites conduit souvent \`a des calculs instables. Il
conviendrait d'en conna\^\i tre la raison. L'implicitation partielle avait
\'et\'e mise en \oe uvre afin de tenter d'utiliser un pas de temps plus grand
dans le cas de jets axisym\'etriques en particulier.

\end{itemize}

%                      Code_Saturne version 1.3
%                      ------------------------
%
%     This file is part of the Code_Saturne Kernel, element of the
%     Code_Saturne CFD tool.
%
%     Copyright (C) 1998-2007 EDF S.A., France
%
%     contact: saturne-support@edf.fr
%
%     The Code_Saturne Kernel is free software; you can redistribute it
%     and/or modify it under the terms of the GNU General Public License
%     as published by the Free Software Foundation; either version 2 of
%     the License, or (at your option) any later version.
%
%     The Code_Saturne Kernel is distributed in the hope that it will be
%     useful, but WITHOUT ANY WARRANTY; without even the implied warranty
%     of MERCHANTABILITY or FITNESS FOR A PARTICULAR PURPOSE.  See the
%     GNU General Public License for more details.
%
%     You should have received a copy of the GNU General Public License
%     along with the Code_Saturne Kernel; if not, write to the
%     Free Software Foundation, Inc.,
%     51 Franklin St, Fifth Floor,
%     Boston, MA  02110-1301  USA
%
%-----------------------------------------------------------------------
%
\programme{vortex}
%
\vspace{1cm}
%%%%%%%%%%%%%%%%%%%%%%%%%%%%%%%%%%
%%%%%%%%%%%%%%%%%%%%%%%%%%%%%%%%%%
\section{Fonction}
%%%%%%%%%%%%%%%%%%%%%%%%%%%%%%%%%%
%%%%%%%%%%%%%%%%%%%%%%%%%%%%%%%%%%
Ce sous-programme est d�di� � la g�n�ration des conditions d'entr�e
turbulente utilis�es en LES.


La m�thode des vortex est bas�e sur une approche de tourbillons
ponctuels. L'id�e de la m�thode consiste � injecter des tourbillons 2D dans le
plan d'entr�e du calcul, puis � calculer le champ de vitesse induit par ces
tourbillons au centre des faces d'entr�e.

%                      Code_Saturne version 1.3
%                      ------------------------
%
%     This file is part of the Code_Saturne Kernel, element of the
%     Code_Saturne CFD tool.
% 
%     Copyright (C) 1998-2007 EDF S.A., France
%
%     contact: saturne-support@edf.fr
% 
%     The Code_Saturne Kernel is free software; you can redistribute it
%     and/or modify it under the terms of the GNU General Public License
%     as published by the Free Software Foundation; either version 2 of
%     the License, or (at your option) any later version.
% 
%     The Code_Saturne Kernel is distributed in the hope that it will be
%     useful, but WITHOUT ANY WARRANTY; without even the implied warranty
%     of MERCHANTABILITY or FITNESS FOR A PARTICULAR PURPOSE.  See the
%     GNU General Public License for more details.
% 
%     You should have received a copy of the GNU General Public License
%     along with the Code_Saturne Kernel; if not, write to the
%     Free Software Foundation, Inc.,
%     51 Franklin St, Fifth Floor,
%     Boston, MA  02110-1301  USA
%
%-----------------------------------------------------------------------
%
%%%%%%%%%%%%%%%%%%%%%%%%%%%%%%%%%%
%%%%%%%%%%%%%%%%%%%%%%%%%%%%%%%%%%
\section{Discr\'etisation}
%%%%%%%%%%%%%%%%%%%%%%%%%%%%%%%%%%
%%%%%%%%%%%%%%%%%%%%%%%%%%%%%%%%%%

Le terme convectif en $\dive(\underline{u} \otimes \rho\,\underline{u})$
introduit une non lin\'earit\'e et un couplage des composantes de la vitesse
$\vect{u}$ dans l'�quation (\ref{Base_Preduv_eqqdm}). Une lin\'earisation et un d\'ecouplage
des trois composantes de la 
vitesse sont r\'ealis\'es lors de la discr\'etisation de cette \'etape de
pr\'ediction.\\
En effet, soit :
\begin{equation}
\vect{\widetilde{u}}= \vect{u}^n + \delta \vect{u} 
\end{equation}
La contribution exacte du terme convectif \`a prendre en compte dans cette
\'etape de pr\'ediction serait :\\
\begin{equation}\label{Base_Preduv_Conv_exact}
\begin{array}{ll}
\dive(\vect{\widetilde{u}} \otimes \rho\,\vect{\widetilde{u}}) =
\dive(\vect{u}^{n} \otimes \rho\,\vect{u}^{n}) + \dive(\delta \vect{u} \otimes
\rho\,\vect{u}^{n}) +  \underbrace { \dive(\vect{u}^{n} \otimes
\rho\,\delta \vect{u})}_{\text {terme couplant lin\'eaire}} +  \underbrace { \dive(\delta \vect{u} \otimes
\rho\,\delta \vect{u})}_{\text {terme couplant et non lin\'eaire}}\\
\end{array} 
\end{equation}
Les deux derniers termes de l'expression (\ref{Base_Preduv_Conv_exact}) sont {\em a priori} n�glig�s
de mani�re � obtenir un syst\`eme en vitesse qui soit d\'ecoupl\'e et donc,
�viter l'inversion d'une matrice pouvant \^etre de tr\`es grande taille. Ces
deux termes peuvent n�anmoins �tre pris en compte de mani�re plus ou moins
approch�e par extrapolation explicite du flux de masse en $n+\theta_F$ (pour le
terme couplant lin�aire provenant de la convection de $\vect{u}^{n}$ par $\delta
\vect{u}$) et utilisation d'un point-fixe par sous it�ration sur le sous
programme \fort{navsto} (pour le terme non-lin�aire, en sp�cifiant $\var{NTERUP}>1$).

L'�quation (\ref{Base_Preduv_eqqdm}) est discr�tis�e au temps $n+\theta$ � l'aide d'un
$\theta$-sch�ma, et le tenseur des pertes de charges d�compos� en une partie
diagonale $\tens{K}_{d}$ et une extradiagonale $\tens{K}_{e}$ (soit
 $\tens{K}_{pdc}=\tens{K}_{d}+\tens{K}_{e}$).\\
$\bullet$ La pression est suppos�e connue en $n-1+\theta$ (d�calage temporel
pression-vitesse) et le gradient naturellement calcul� � cet instant.\\ 
$\bullet$ Les termes sources de viscosit� secondaire, de gradient transpos\'e,
ceux provenant du mod�le de turbulence\footnote{except� $\dive (\mu_t\ (\ggrad
\underline {u}))$}, $\rho\,\tens{K}_{\,e}\ \underline{u}$, $(\rho -\rho_0)
\underline {g}$ ainsi que $\underline{T}_{s}^{\,exp}$ et
$\Gamma\,\underline{u}_{\,i}$ sont pris de mani�re explicite au temps $n$, ou
extrapol�s suivant le sch�ma en temps choisi pour les propri�t�s physique et les
termes sources.\\ 
$\bullet$ Les termes sources $\underline{u}\,\,\dive (\rho\,\underline {u})$,
$\Gamma\,\,\underline{u}$, $T_{s}^{\,imp}\,\,\underline{u}$ et
$-\rho\,\tens{K}_{\,d}\,\,\underline{u}$ sont implicit�s est calcul�s �
l'instant $n+\theta$.\\ 
$\bullet$ Le terme de diffusion $\dive (\mu_{\,tot}\,\ggrad \underline{u})$ est
implicit� : la vitesse est prise � l'instant $n+\theta$ et la viscosit�
explicit�e ou extrapol�e.\\ 
$\bullet$ Enfin, le terme de convection en $\dive(\,\underline{u} \otimes
(\rho\underline{u})\,)$ est implicit� : la composante r�solue de la vitesse est
prise en $n+\theta$, et le flux de masse, explicit�, ou extrapol� en
$n+\theta_F$. 

Par souci de clart�, on suppose, en l'absence d'indication, que les propri�tes
physiques $\Phi$ ($\rho,\,\mu_{tot},\,...$) et le flux de masse
$(\rho\underline{u})$ sont pris respectivement aux instants $n+\theta_\Phi$ et
$n+\theta_F$, o� $\theta_\Phi$ et $\theta_F$ d�pendent des sch�mas en temps
sp�cifiquement utilis�s pour ces grandeurs\footnote{cf. \fort{introd}}. 

La discr�tisation temporelle de l'�quation (\ref{Base_Preduv_eqqdm}) s'�crit alors comme suit : 

\begin{equation}\label{Base_Preduv_eq_di1}
 \begin{array}{c}
\displaystyle \rho\,\ \frac{ \underline {\widetilde{u}}^{n+1} -\underline {u}^{n} }
{\Delta t} + \dive(\,\underline{\widetilde{u}}^{n+\theta} \otimes (\rho\underline{u})\,) -\dive
(\mu_{\,tot}\,\ggrad \underline{\widetilde{u}}^{n+\theta}) =
\\
\displaystyle
 - \grad p^{n-1+\theta} + \dive (\rho\,\underline {u})\,\underline{\widetilde{u}}^{n+\theta} +(\Gamma\,\underline{u}_{\,i})^{n+\theta_S}-\Gamma^n\,\,\underline{\widetilde{u}}^{n+\theta}
\\
\begin{array}{c}
\displaystyle
- \rho\,\tens{K}_{\,d}^{n}\,\,\underline{\widetilde{u}}^{n+\theta} - (\rho\,\tens{K}_{\,e}\ \underline{u})^{n+\theta_S} + (\underline{T}_{s}^{\,exp})^{\,n+\theta_S} + T_{s}^{\,imp}\,\,\underline{\widetilde{u}}^{n+\theta}
\\
\displaystyle
+[\dive (\mu_{\,tot}\,^t\ggrad \underline {u})]^{n+\theta_S}-\frac {2} {3}[\,\grad (\mu_{\,tot}\,\dive \underline {u})]^{n+\theta_S} + (\rho -\rho_0) \underline {g}
 - (\underline{turb})^{n+\theta_S}
\end{array}
\end{array}
\end{equation}
o\`u, par souci de simplification, on a pos\'e :
\begin{equation}
\mu_{\,tot}=
\begin{cases}
\mu+\mu_t & \text{pour les mod�les � viscosit� turbulente ou en LES}, \\
\mu & \text{pour les mod�les au second ordre ou en laminaire}
\end{cases} \ 
\end{equation}
\\
et :
\begin{equation}
\underline{turb}^{n}=
\begin{cases}
\displaystyle\frac {2}{3}\grad (\rho^{n}\,k^{n}) & \text{pour les mod�les � viscosit� turbulente}, \\
\dive(\rho^{n}\,\tens{R}^n) & \text{pour les mod�les au second ordre},\\
0 & \text{en laminaire ou en LES}\\
\end{cases}
\end{equation}
Par analogie avec l'�criture du $\theta$-sch�ma pour une variable scalaire, $\,
\underline {\widetilde{u}}^{n+\theta}$ est interpol�e � partir de la vitesse
pr�dite $\underline {\widetilde{u}}^{n+1}$ de la mani\`ere suivante\footnote{si
$\theta=1/2$, ou qu'une extrapolation est utilis�e, l'ordre 2 n'est obtenu que si
le pas de temps $\Delta t$ est uniforme en temps et en espace.}~: 
\begin{equation}
\underline {\widetilde{u}}^{n+\theta}=\theta\, \underline
{\widetilde{u}}^{n+1}+(1-\theta)\, \underline {u}^{n}\\ 
\end{equation}
Avec :
\begin{equation}
\left\{
\begin{array}{ll}
\theta = 1   & \text{Pour un sch\'ema de type Euler implicite d'ordre 1.}\\
\theta = 1/2 & \text{Pour un sch\'ema de type Cranck-Nicolson d'ordre 2.}\\
\end{array}
\right.
\end{equation}

L'�quation (\ref{Base_Preduv_eq_di1}) est alors r��crite sous la forme :

\begin{equation}\label{Base_Preduv_eq_di2}
\begin{array}{c}
\displaystyle \underbrace{\left(\frac{\rho}{\Delta t} -\theta \,\dive (\rho\,\underline {u}) +\theta \,\, \Gamma^n +
\theta \,\, \rho\,\tens{K}_{\,d}^n-\theta \,T_s^{\,imp} \right)}_{\displaystyle f_s^{imp}}\, (\underline {\,\widetilde{u}}^{n+1} -\underline {u}^{n})
\\
 +\, \theta\, \dive(\underline {\widetilde{u}}^{n+1} \otimes (\rho\underline{u}))-\, \theta\,\dive (\mu_{\,tot}\,\ggrad \underline {\widetilde{u}}^{n+1}) =
\\
-\,(1-\theta)\, \dive(\underline {u}^{n} \otimes (\rho\underline{u})) +\,(1-\theta)\,\dive (\mu_{\,tot}\,\ggrad \underline {u}^{n})
\\
f_s^{exp}\left\{
\begin{array}{c}
\displaystyle 
- \grad p^{n-1+\theta} + \dive (\rho\,\underline {u})\,\underline{u}^{n} +\,(\,\Gamma^{n}\,\underline{u}_{\,i}\,)^{n+\theta_S}- \Gamma^n\,\,\underline{u}^{n}
\\
\displaystyle
-(\,\rho\,\tens{K}_{\,e}\ \underline{u}\,)^{n+\theta_S} -\rho\,\tens{K}_{\,d}^n\ \underline{u}^{n}+ (\underline{T}_{s}^{\,exp})^{\,n+\theta_S} + T_s^{\,imp}\,\,\underline {u}^{n} 
\\
\displaystyle
+[\dive (\mu_{\,tot}\,^t\ggrad \underline {u}\,)]^{n+\theta_S}-\frac {2} {3}[\,\grad (\mu_{\,tot}\,\dive \underline {u}\,)]^{n+\theta_S} + (\rho -\rho_0) \underline {g}-(\underline{turb})^{n+\theta_S}
\end{array}
\right.
\end{array}
\end{equation}

d'o� l'�quation r�solue par le sous-programme \fort{codits} :
\begin{equation}\begin{array}{c}
\displaystyle
f_s^{\,imp}(\underline {\widetilde{u}}^{n+1}-\underline {u}^{n}) + \theta\, \dive(\underline{\widetilde{u}}^{n+1} \otimes (\rho
\underline{u})) - \theta\,\dive (\,\mu_{\,tot}\,\ggrad \underline{\widetilde{u}}^{n+1}) = 
\\\\
\displaystyle
-(1-\theta)\,\dive(\underline{u}^{n} \otimes (\rho \underline{u}))+(1-\theta)\,\dive (\,\mu_{\,tot}\,\ggrad \underline{u}^{n})
+ \underline{f}_{\,s}^{\,exp}
\end{array}
\end{equation}
La m\'ethode de discr\'etisation spatiale est d\'evelopp\'ee dans le sous-programme \fort{codits}.\\



\minititre{Remarques :}
{\tiny$\blacksquare$} Dans le cas standard sans extrapolation, le terme
$-\,T_s^{\,imp}$ n'est ajout� � $f_s^{\,imp}$ que s'il est positif afin de ne
pas affaiblir la dominance de la diagonale de la matrice � inverser.\\ 
{\tiny$\blacksquare$} Si une extrapolation est utilis�e, par contre,
$\,T_s^{\,imp}$ est ajout� � $f_s^{\,imp}$ quel que soit son signe. En effet, l'id�e intuitive qui
consiste � prendre~: 
\begin{equation}
\begin{cases}
\displaystyle
(\underline{T}_{s}^{\,exp} + T_{s}^{\,imp}\,\underline {u})^{\,n+\theta_S} &
\text{si } T_{s}^{\,imp} > 0\\ 
\displaystyle
(\underline{T}_{s}^{\,exp})^{\,n+\theta_S} + T_{s}^{\,imp}\,\underline{u}^{n+\theta} &\text{sinon}\\
\end{cases}
\end{equation} 
aboutit � une incoh�rence dans le traitement si $T_s^{imp}$ change de signe
entre deux pas de temps.\\ 
{\tiny$\blacksquare$} la partie diagonale $\tens{K}_{\,d}$ du terme
de perte de charge est utilis�e dans $f_s^{\,imp}$. En fait, pour \^etre rigoureux,
il faudrait ne retenir que les contributions positives (point signal\'e dans le
sous-programme utilisateur associ\'e \fort{uskpdc}). Cette prise en compte sera \`a am\'eliorer.\\
{\tiny$\blacksquare$} Le terme $\theta\,\Gamma^{n}-\theta\,\dive
(\rho\,\underline {u})$ ne pose pas de probl�me pour la 
dominance de la diagonale de la matrice car il est exactement compens� par le
terme de convection (cf. \fort{covofi}). 


%                      Code_Saturne version 1.3
%                      ------------------------
%
%     This file is part of the Code_Saturne Kernel, element of the
%     Code_Saturne CFD tool.
%
%     Copyright (C) 1998-2007 EDF S.A., France
%
%     contact: saturne-support@edf.fr
%
%     The Code_Saturne Kernel is free software; you can redistribute it
%     and/or modify it under the terms of the GNU General Public License
%     as published by the Free Software Foundation; either version 2 of
%     the License, or (at your option) any later version.
%
%     The Code_Saturne Kernel is distributed in the hope that it will be
%     useful, but WITHOUT ANY WARRANTY; without even the implied warranty
%     of MERCHANTABILITY or FITNESS FOR A PARTICULAR PURPOSE.  See the
%     GNU General Public License for more details.
%
%     You should have received a copy of the GNU General Public License
%     along with the Code_Saturne Kernel; if not, write to the
%     Free Software Foundation, Inc.,
%     51 Franklin St, Fifth Floor,
%     Boston, MA  02110-1301  USA
%
%-----------------------------------------------------------------------
%

%%%%%%%%%%%%%%%%%%%%%%%%%%%%%%%%%%
%%%%%%%%%%%%%%%%%%%%%%%%%%%%%%%%%%
\section{Mise en \oe uvre}
%%%%%%%%%%%%%%%%%%%%%%%%%%%%%%%%%%
%%%%%%%%%%%%%%%%%%%%%%%%%%%%%%%%%%
La num\'ero de la phase trait\'ee fait partie des arguments de \fort{turrij}. On
omettra volontairement de le pr\'eciser dans ce qui suit, on indiquera par $(\ )$ la
notion de tableau s'y rattachant.

\etape{Calcul des termes de production $\tens{\mathcal{P}}$}
\begin{itemize}
\item [$\star$] Initialisation \`a z\'ero du tableau \var{PRODUC} dimensionn\'e \`a $\var{NCEL}\times 6$.
\item [$\star$] On appelle trois fois \fort{grdcel} pour calculer les gradients des composantes de la vitesse $u$, $v$ et
$w$ prises au temps $n$.

Au final, on a :\\
$\displaystyle
\begin{array} {ll}
\var{PRODUC(1,IEL)} = & \displaystyle - 2 \left[ R_{11}^{\,n} \frac{\partial u^{\,n}} {\partial x} +R_{12}^{\,n} \frac{\partial u^{\,n}} {\partial y}+R_{13}^{\,n} \frac{\partial u^{\,n}} {\partial z} \right] \text{        (production de $R_{11}^{\,n}$)}\\
\var{PRODUC(2,IEL)} = & \displaystyle - 2 \left[ R_{12}^{\,n} \frac{\partial v^{\,n}} {\partial x} +R_{22}^{\,n} \frac{\partial v^{\,n}} {\partial y}+R_{23}^{\,n} \frac{\partial v^{\,n}} {\partial z} \right] \text{        (production de $R_{22}^{\,n}$)}\\
\var{PRODUC(3,IEL)} = & \displaystyle - 2 \left[ R_{13}^{\,n} \frac{\partial w^{\,n}} {\partial x} +R_{23}^{\,n} \frac{\partial w^{\,n}} {\partial y}+R_{33}^{\,n} \frac{\partial w^{\,n}} {\partial z} \right] \text{        (production de $R_{33}^{\,n}$)}\\
\var{PRODUC(4,IEL)} = & \displaystyle - \left[ R_{12}^{\,n} \frac{\partial u^{\,n}} {\partial x} +R_{22}^{\,n} \frac{\partial u^{\,n}} {\partial y}+R_{23}^{\,n} \frac{\partial u^{\,n}} {\partial z} \right] \\
& \displaystyle - \left[ R_{11}^{\,n} \frac{\partial v^{\,n}} {\partial x} +R_{12}^{\,n} \frac{\partial v^{\,n}} {\partial y}+R_{13}^{\,n} \frac{\partial v^{\,n}} {\partial z} \right] \text{        (production de $R_{12}^{\,n}$)} \\
\var{PRODUC(5,IEL)} = & \displaystyle - \left[ R_{13}^{\,n} \frac{\partial u^{\,n}} {\partial x} +R_{23}^{\,n} \frac{\partial u^{\,n}} {\partial y}+R_{33}^{\,n} \frac{\partial u^{\,n}} {\partial z} \right] \\
& \displaystyle - \left[ R_{11}^{\,n} \frac{\partial w^{\,n}} {\partial x} +R_{12}^{\,n} \frac{\partial w^{\,n}} {\partial y}+R_{13}^{\,n} \frac{\partial w^{\,n}} {\partial z} \right] \text{        (production de $R_{13}^{\,n}$)} \\
\var{PRODUC(6,IEL)} = & \displaystyle - \left[ R_{13}^{\,n} \frac{\partial v^{\,n}} {\partial x} +R_{23}^{\,n} \frac{\partial v^{\,n}} {\partial y}+R_{33}^{\,n} \frac{\partial v^{\,n}} {\partial z} \right] \\
& \displaystyle - \left[ R_{12}^{\,n} \frac{\partial w^{\,n}} {\partial x} +R_{22}^{\,n} \frac{\partial w^{\,n}} {\partial y}+R_{23}^{\,n} \frac{\partial w^{\,n}} {\partial z} \right]  \text{        (production de $R_{23}^{\,n}$)}
\end{array}
$
\end{itemize}

\etape{Calcul du gradient de la masse volumique $\rho^n$ prise au d\'ebut du pas
de temps courant\footnote{{\it i.e.} calcul\'ee \`a partir des
variables du pas de temps pr\'ec\'edent $n$ si n\'ecessaire.} $(n+1)$}
Ce calcul n'a lieu que si les termes de gravit\'e doivent \^etre pris en compte
($\var{IGRARI()} =1$).
\begin{itemize}
\item [$\star$] Appel de \fort{grdcel}  pour calculer le gradient de $\rho^n$
dans les trois directions de l'espace. Les conditions aux limites sur $\rho^n$
sont des conditions de Dirichlet puisque la valeur de $\rho^n$ aux faces de bord
$ik$ (variable \var{IFAC}) est connue et vaut $\rho_{\,b_{\,ik}}$. Pour \'ecrire les conditions aux limites
sous la forme habituelle, $$\rho_{\,b_{\,ik}} = \var{COEFA} + \var{COEFB}
\,\rho^n_{\,I'}$$ on pose alors $\var{COEFA}=
\var{PROPCE(IFAC,IPPROB(IROM(IPHAS)))}$ et $\var{COEFB} = \var{VISCB} = 0$.\\
$\var{PROPCE(1,IPPROB(IROM(IPHAS)))}$ (resp.$\var{VISCB}$) est utilis\'e en lieu
et place de l'habituel \var{COEFA} ($\var{COEFB}$), lors de l'appel \`a \fort{grdcel}.\\
On a donc :\\
$\displaystyle \var{GRAROX}= \frac{\partial \rho^n}{\partial x}\ $,$\displaystyle \ \var{GRAROY}= \frac{\partial
\rho^n}{\partial y}$ et $
\displaystyle \ \var{GRAROZ}= \frac{\partial \rho^n}{\partial z}\ $.

\end{itemize}

Le gradient de $\rho^n$ servira \`a calculer les termes de production par effets de gravit\'e si ces derniers sont pris en compte.

\etape{Boucle \var{ISOU} de $1$ \`a $6$ sur les tensions de Reynolds}
Pour $\var{ISOU} = 1,2,3,4,5,6$, on r\'esout respectivement et dans
l'ordre  les
\'equations de $R_{11}$, $R_{22}$, $R_{33}$, $R_{12}$, $R_{13}$ et $R_{23}$ par
l'appel au sous-programme \fort{resrij}.\\
La r\'esolution se fait par incr\'ement $\delta {R}_{ij}^{\,n+1,k+1}$ , en utilisant la m\^eme m\'ethode que
celle d\'ecrite dans le sous-programme \fort{codits}. On adopte ici les m\^emes notations.
\var{SMBR} est le second membre du syst\`eme \`a inverser, syst\`eme portant sur
les incr\'ements de la variable. \var{ROVSDT} repr\'esente la diagonale de la
matrice, hors convection/diffusion.\\
On va r\'esoudre l'\'equation (\ref{Base_Turrij_Eq_Temp_Rij}) sous forme incr\'ementale en
utilisant \fort{codits}, soit :
\begin{equation}\label{Base_Turrij_Eq_Temp_deltaRij}
\begin{array}{ll}
&\displaystyle \underbrace{\left(\frac {\rho^n_L}{\Delta t^n}
+ \rho^n_L \,C_1\,\frac{\varepsilon^n_L}{k^n_L}(1-\frac{\delta_{ij}}{3})
 - m^n_{\,lm} + \Gamma_L\,+ max(-\alpha^n_{R_{ij}},0)\right)\,|\Omega_l|}
_{\text {$\var{ROVSDT}$ contribuant
\`a la diagonale de la matrice simplifi\'ee de \fort{matrix}}}\,(\delta{R}_{ij}^{\,n+1,p+1})_{\,L}\\\\
&  \underbrace{+\sum\limits_{m\in Vois(l)}\displaystyle \left[
 m^n_{\,lm} \delta R_{ij,\,f_{\,lm}}^{\,n+1,p+1}
- (\mu^n_{\,lm} + \gamma^n_{\,lm})\
\frac{({\delta R}_{ij}^{\,n+1,p+1})_{M}-({\delta R}_{ij}^{\,n+1,p+1})_{L})}{\overline{L'M'}}\,
S_{\,lm} \right]}_{\text { convection upwind pur et diffusion non reconstruite
relatives \`a la matrice simplifi\'ee de \fort{matrix}\footnotemark}} \\
% voir le texte de la footmark plus bas
&= - \displaystyle\frac {\rho^n_L}{\Delta t^n}\,\left(\,(R^{\,n+1,p}_{ij})_L - (R^{\,n}_{ij})_L\,\right)\\
&-\,\underbrace{\displaystyle\int_{\Omega_l} \left(
\dive\,[\,(\rho\,\vect{u})^n\,R^{\,n+1,p}_{ij} - (\mu^n\,+ \gamma^n\,)\,
\grad{R^{\,n+1,p}_{ij}}\,]\right)\,d\Omega}_{\text {convection et diffusion
trait\'ees par \fort{bilsc2}}}\\
&+\displaystyle \int_{\Omega_l} \left[\,\mathcal{P}^{\,n+1,p}_{ij} + \mathcal{G}^{\,n+1,p}_{ij}
- \displaystyle\rho^n \,C_1\,\frac{\varepsilon^n}{k^n}\left[R^{\,n+1,p}_{ij}-
\frac{2}{3}\,k^n\,\delta_{ij}\right] + \phi^{\,n+1,p}_{ij,2} +
\phi^{\,n+1,p}_{ij,w}\,\right]\, d\Omega \\
& + \displaystyle\int_{\Omega_l} \left[- \frac{2}{3} \rho^n \varepsilon^n \delta_{ij}
 + \Gamma\,(\,R^{\,in}_{ij} - R^{\,n+1,p}_{ij}\,) +
\alpha^n_{R_{ij}}\,R^{\,n+1,p}_{ij}+ \beta^n_{R_{ij}}\right]\, d\Omega\\
&+ \sum\limits_{m\in
Vois(l)}\displaystyle \left[\ \tens{E}^n\,\grad{R}^{\,n+1,p}_{ij} \right]_{\,lm}\,.\,\vect{n}_{\,lm}S_{\,lm}\\
&+ \sum\limits_{m\in Vois(l)}\displaystyle \left[\
\tens{D}^n\,\grad{R}^{\,n+1,p}_{ij} \right]_{\,lm}\,.\,\vect{n}_{\,lm}S_{\,lm}\\
&- \sum\limits_{m\in Vois(l)} \gamma^n_{\,lm} \left( \grad{R}^{\,n+1,p}_{ij}\,.\,\vect{n}_{\,lm} \right)  S_{\,lm}\\
&+ \sum\limits_{m\in Vois(l)}  m^n_{\,lm}\,(R^{\,n+1,p}_{ij})_L\\
\end{array}
\end{equation}
% si on ne fait pas comme ca, il n'apparait pas
\footnotetext[\thefootnote]{Si $\var{IRIJNU} = 1$, on remplace  $\mu^n_{\,lm}$ par $(\mu +
\mu_t)^n_{\,lm}$ dans l'expression de la diffusion non reconstruite
associ\'ee \`a la matrice simplifi\'ee de \fort{matrix} ($\mu_t$ d\'esigne la
viscosit\'e turbulente calcul\'ee comme en $k-\varepsilon$).}

o\`u on rappelle :\\
pour $n$ donn\'e entier positif, on d\'efinit la suite
 $({R}_{ij}^{\,n+1,p})_{p \in \grandN}$
 par :
\begin{equation}\notag
\left\{\begin{array}{l}
{R}_{ij}^{\,n+1,0} = {R}_{ij}^{\,n}\\
{R}_{ij}^{\,n+1,p+1} = {R}_{ij}^{\,n+1,p} + \delta{R}_{ij}^{\,n+1,p+1} \\
\end{array}\right.
\end{equation}
$(\delta{R}_{ij}^{\,n+1,p+1})_{\,L}$ d\'esigne la valeur sur l'\'el\'ement
$\Omega_l$ du $\text{$(\,p+1\,)$-i\`eme}$ incr\'ement de ${R}_{ij}^{\,n+1}$,
$ m^n_{\,lm}$ le flux de masse \`a l'instant $n$ \`a travers la face $lm$,
$\delta R_{ij,\,f_{\,lm}}^{\,n+1,p+1}$ vaut $({\delta
R}_{ij}^{\,n+1,p+1})_{L}$  si $ m^n_{\,lm} \geqslant 0$, $({\delta
R}_{ij}^{\,n+1,p+1})_{M}$ sinon,
$\mathcal{P}^{\,n+1,p}_{ij}$, $\phi^{\,n+1,p}_{ij,2}$, $\phi^{\,n+1,p}_{ij,w}$ les valeurs
des quantit\'es associ\'ees correspondant \`a l'incr\'ement
$(\delta{R}_{ij}^{\,n+1,p})$.\\



Tous ces termes sont calcul\'es comme suit :
\begin{itemize}
\item Terme de gauche de l'\'equation (\ref{Base_Turrij_Eq_Temp_deltaRij})\\
Dans \fort{resrij} est calcul\'ee la variable \var{ROVSDT}. Les autres
termes sont compl\'et\'es par \fort{codits}, lors de la construction de la matrice simplifi\'ee , {\it via} un
appel au sous-programme \fort{matrix}. La quantit\'e
 $(\mu^n_{\,lm} + \gamma^n_{\,lm})$ \`a la face $lm$ est calcul\'ee lors de l'appel \`a
\fort{visort}.\\
\item Second membre de l'\'equation (\ref{Base_Turrij_Eq_Temp_deltaRij})\\
Le premier terme non d\'etaill\'e est calcul\'e par le sous-programme
\fort{bilsc2}, qui applique le sch\'ema convectif choisi par l'utilisateur, qui
reconstruit ou non selon le souhait de l'utilisateur les gradients intervenants
dans la convection-diffusion.\\
Les termes sans accolade sont, eux, compl\`etement explicites et ajout\'es au fur et
\`a mesure dans \var{SMBR} pour former
l'expression $f^{\,exp}_s$ de \fort{codits}.
\end{itemize}
On d\'ecrit ci-dessous les \'etapes de \fort{resrij} :
\begin{itemize}

\item DELTIJ = 1, pour $\var{ISOU} \leqslant 3$ et DELTIJ = 0  Si $\var{ISOU} >
3$. Cette valeur repr\'esente le symbole de Kroeneker $\delta_{ij}$.

\item Initialisation \`a z\'ero de \var{SMBR} (tableau contenant le second
membre) et \var{ROVSDT} (tableau contenant la diagonale de la matrice sauf celle
relative \`a la contribution de la
diagonale des op\'erateurs de convection et de diffusion lin\'earis\'es
\footnote{qui correspondent aux sch\'emas convectif upwind pur et diffusif sans
reconstruction.}), tous deux de dimension $\var{NCEL}$.

\item Lecture et prise en compte des termes sources utilisateur pour la variable $R_{ij}$

Appel \`a \fort{ustsri} pour charger les termes sources utilisateurs. Ils sont
stock\'es comme suit. Pour la cellule $\Omega_l$ de centre $L$, repr\'esent\'ee par $\var{IEL}$, on a :\\
\begin{equation}\notag
\left\{\begin{array}{lll}
&\var{ROVSDT(IEL)}&= |\Omega_l| \ \alpha_{R_{ij}}\\
&\var{SMBR(IEL)}&=|\Omega_l| \ \beta_{R_{ij}}\\
\end{array}\right.
\end{equation}
On affecte alors les valeurs ad\'equates au second membre \var{SMBR} et \`a la
diagonale \var{ROVSDT} comme suit :
\begin{equation}\notag
\left\{\begin{array}{lll}
&\var{SMBR(IEL)} &= \var{SMBR(IEL)} +\ |\Omega_l| \ \alpha_{R_{ij}} \ (R^n_{ij})_L \\
&\var{ROVSDT(IEL)}&= \text{max }(-\ |\Omega_l| \ \alpha_{R_{ij}},0)\\
\end{array}\right.
\end{equation}
La valeur de $ \var{ROVSDT}$ est ainsi calcul\'ee pour des raisons de stabilit\'e
num\'erique. En effet, on ne rajoute sur la diagonale que les valeurs positives,
ce qui correspond physiquement \`a impliciter les termes de rappel uniquement.
\item{Calcul du terme source de masse  si $\Gamma_L > 0$}

Appel de \fort{catsma} et incr\'ementation si n\'ecessaire de \var{SMBR} et
\var{ROVSDT} {\it via} :\\
\begin{equation}\notag
\left\{\begin{array}{lll}
\displaystyle \var{SMBR(IEL)} = \var{SMBR(IEL)} + |\Omega_l| \ \Gamma_L \
\left[(R^{\,in}_{ij})_L - (R^{\,n}_{ij})_L \right] \\
\displaystyle \var{ROVSDT(IEL)}=\var{ROVSDT(IEL)} + |\Omega_l| \ \Gamma_L
\end{array}\right.
\end{equation}
\item Calcul du terme d'accumulation de masse et du terme instationnaire

On stocke $\displaystyle \var{W1}= \int_{\Omega_l}\dive\,(\rho\,\vect{u})\,d\Omega$
calcul\'e par \fort{divmas} \`a l'aide des flux de masse aux faces internes
$ m^n_{\,lm}=\sum\limits_{m\in Vois(l)}{(\rho \vect{u})_{\,lm}^n} \text{.}\,
\vect{S}_{\,lm} $ (tableau \var{FLUMAS}) et des flux de masse aux bords  $ m^n_{\,b_{lk}} = \sum\limits_{k\in{\gamma_b(l)}}{(\rho \vect{u})_{\,{b}_{lk}}^n} \text{.}\,
\vect{S}_{\,{b}_{lk}} $ (tableau \var{FLUMAB}).
On incr\'emente ensuite \var{SMBR} et \var{ROVSDT}.
\begin{equation}\notag
\left\{\begin{array}{lll}
&\var{SMBR(IEL)} &= \var{SMBR(IEL)} + \var{ICONV}\  (R^n_{ij})_L\,(\displaystyle
\int_{\Omega_l}\dive\,(\rho\,\vect{u})\ d\Omega) \\
&\var{ROVSDT(IEL)}& = \var{ROVSDT(IEL)} +  \var{ISTAT}\,\displaystyle
\frac{\rho^n_L \ |\Omega_l|}{\Delta t^n} -  \var{ICONV}\ (\displaystyle
\int_{\Omega_l}\dive\,(\rho\,\vect{u})\ d\Omega) \\
\end{array}\right.
\end{equation}
\item Calcul des termes sources de production, des termes $\displaystyle
\phi_{\,ij,1}+\phi_{\,ij,2}$ et de la dissipation~$\displaystyle-\frac{2}{3} \varepsilon\,\delta_{\,ij}$ :

On effectue une boucle d'indice \var{IEL} sur les cellules $\Omega_l$ de centre $L$ :
\begin{itemize}
\item [$\Rightarrow$] $\displaystyle \var{TRPROD}= \frac{1}{2} (\mathcal{P}^n_{ii})_L = \frac{1}{2} \left[ \var{PRODUC(1,IEL)} +  \var{PRODUC(2,IEL)} +  \var{PRODUC(3,IEL)} \right] $
\item [$\Rightarrow$] $\displaystyle \var{TRRIJ }= \frac{1}{2} (R^n_{ii})_L $
\item [$\Rightarrow$] $\displaystyle \var{SMBR(IEL)} =\ \var{SMBR(IEL)}\ +$\\
$\ \displaystyle\rho^n_L |\Omega_l| \left[ \displaystyle
\frac{2}{3}\,\delta_{\,ij} \left( \ \displaystyle \frac{ C_2}{2}\,(\mathcal{P}^n_{ii})_L\ +
(C_1-1)\ \varepsilon^n_L\, \right)\right.$\\
$ + \left.\ (1-C_2) \ \var{PRODUC(ISOU,IEL)} -
\displaystyle C_1\ \frac{2\,\varepsilon^n_L}{(R^n_{ii})_L}\ (R^n_{ij})_L \right]$
\item [$\Rightarrow$] $\displaystyle \var{ROVSDT(IEL)} = \var{ROVSDT(IEL)} +
\rho^n_L \ |\Omega_l| \ (- \displaystyle \frac{1}{3} \ \,\delta_{\,ij} + 1) \ C_1
\ \frac{2\ \varepsilon^n_L}{(R^n_{ii})_L}$
\end{itemize}
\item Appel de \fort{rijech} pour le calcul des termes d'\'echo de paroi
 $\phi^n_{ij,w}$ si $\var{IRIJEC()}=1$ et ajout dans \var{SMBR}.\\
$\var{SMBR} = \var{SMBR} + \phi^n_{ij,w}$\\
Suivant son mode de calcul (\var{ICDPAR}), la distance � la paroi est directement accessible
par \var{RA(IDIPAR+IEL-1)} (\var{|ICDPAR|} = 1) ou bien
est calcul\'ee \`a partir de $\var{IA(IIFAPA(IPHAS)+IEL - 1)}$,
qui donne pour l'\'el\'ement $\var{IEL}$ le num\'ero de la face de bord
paroi la plus  proche (\var{|ICDPAR|} = 2). Ces tableaux ont \'et\'e renseign\'e une fois pour toutes au
d\'ebut de calcul.

\item  Appel de \fort{rijthe} pour le calcul des termes de gravit\'e $\mathcal{G}^n_{ij}$ et ajout dans \var{SMBR}.

Ce calcul n'a lieu que si $\var{IGRARI()} = 1$.
$ \var{SMBR} = \var{SMBR} + \mathcal{G}^n_{ij}$
\item Calcul de la partie explicite du terme de diffusion
 $\dive{\,\left[\tens{A}\,\grad{R}^{\,n}_{ij}\right]}$, plus pr\'ecis\'ement
des contributions du terme extradiagonal pris aux faces purement internes
(remplissage du tableau \var{VISCF}), puis aux faces de bord (remplissage du
tableau \var{VISCB}).
\begin{itemize}
\item [$\star$] Appel de \fort{grdcel} pour le calcul du gradient de
$R^{\,n}_{ij}$ dans chaque direction. Ces gradients sont respectivement
stock\'es dans les tableaux de travail \var{W1}, \var{W2} et \var{W3}.

\item [$\star$] boucle d'indice \var{IEL} sur les cellules $\Omega_l$ de centre
$L$ pour le
calcul de $\tens{E}^n\,\grad{R}^{\,n}_{ij}$ aux cellules dans un premier temps :\\
\begin{itemize}
\item [$\Rightarrow$] $\displaystyle \var{TRRIJ}= \frac{1}{2} (R^{\,n}_{ii})_L $
\item [$\Rightarrow$] $\displaystyle \var{CSTRIJ} = \rho^n_L\ C_S \ \displaystyle\frac{(R^n_{ii})_L}{2\,\varepsilon^n_L}$
\item [$\Rightarrow$] $\displaystyle \var{W4(IEL)} = \rho^n_L\ C_S\
\displaystyle\frac{(R^n_{ii})_L}{2\,\varepsilon^n_L} \left[\,(R^{\,n}_{12})_L \ \var{W2(IEL)} +
(R^{\,n}_{13})_L \ \var{W3(IEL)}\,\right]$
\item [$\Rightarrow$] $\displaystyle \var{W5(IEL)} = \rho^n_L\ C_S\
\displaystyle\frac{(R^n_{ii})_L}{2\,\varepsilon^n_L} \left[\,(R^{\,n}_{12})_L \ \var{W1(IEL)} +
(R^{\,n}_{23})_L \ \var{W3(IEL)}\,\right]$
\item [$\Rightarrow$] $\displaystyle \var{W6(IEL)} = \rho^n_L\ C_S\
\displaystyle\frac{(R^n_{ii})_L}{2\,\varepsilon^n_L} \left[\,(R^{\,n}_{13})_L \ \var{W1(IEL)} + (R^{\,n}_{23})_L \ \var{W2(IEL)}\,\right]$
\end{itemize}



\item [$\star$] Appel de \fort{vectds}\footnote{Le r\'esultat est stock\'e dans
\var{VISCF} et \var{VISCB}. Dans \fort{vectds}, les valeurs aux faces internes
sont interpol\'ees lin\'eairement sans reconstruction et \var{VISCB} est mis \`a
z\'ero.} pour assembler $\displaystyle\left[ \tens{E}^n\,\grad{R}^{\,n}_{ij}
\right]\,.\,\vect{n}_{\,lm}S_{\,lm}$ aux faces $lm$.
\item [$\star$] Appel de \fort{divmas} pour calculer la divergence du flux d\'efini par \var{VISCF} et \var{VISCB}.
Le r\'esultat est stock\'e dans \var{W4}.\\
Ajout au second membre \var{SMBR}.\\
\var{SMBR} = \var{SMBR} + \var{W4}
\end{itemize}

A l'issue de cette \'etape, seule la partie extradiagonale de la diffusion prise
enti\`erement explicite~:
 $$\sum\limits_{m\in
Vois(l)}\left[\ \tens{E}^n\,\grad{R}^{\,n}_{ij} \right]_{\,lm}\,.\,\vect{n}_{\,lm}S_{\,lm}$$ a \'et\'e calcul\'ee.\\

\item Calcul de la partie diagonale du terme de diffusion\footnote{Seule la
composante normale  du  gradient de $R_{ij}$ aux faces sera implicite.} :\\
Comme on l'a d\'eja vu, une partie de cette contribution sera trait\'ee en
implicite (celle relative \`a la composante normale du gradient) et donc
ajout\'ee au second membre par \fort{bilsc2} ; l'autre
partie sera explicite et prise en compte dans $f_s^{\,exp}$.
\begin{itemize}
\item [$\star$] On effectue une boucle d'indice \var{IEL} sur les cellules
$\Omega_l$ de centre $L$ :
\begin{itemize}
\item [$\Rightarrow$] $\displaystyle \var{TRRIJ }= \frac{1}{2} (R^{\,n}_{ii})_L $
\item [$\Rightarrow$] $\displaystyle \var{CSTRIJ} = \rho^n_L \ C_S \ \frac{(R^{\,n}_{ii})_L}{2\,\varepsilon^n_L}$
\item [$\Rightarrow$] $\displaystyle \var{W4(IEL)} = \rho^n_L \ C_S \
\frac{(R^{\,n}_{ii})_L}{2\,\varepsilon^n_L} \ (R^{\,n}_{11})_L$
\item [$\Rightarrow$] $\displaystyle \var{W5(IEL)} = \rho^n_L \ C_S \ \frac{(R^{\,n}_{ii})_L}{2\,\varepsilon^n_L}\ (R^n_{22})_L$
\item [$\Rightarrow$] $\displaystyle \var{W6(IEL)} = \rho^n_L \ C_S \ \frac{(R^{\,n}_{ii})_L}{2\,\varepsilon^n_L} \ (R^n_{33})_L$
\end{itemize}

%\item Traitement du parall\'elisme et de la p\'eriodicit\'e.

\item [$\star$] On effectue une boucle d'indice \var{IFAC} sur les faces
purement internes $lm$ pour remplir le tableau \var{VISCF} :
\begin{itemize}
\item [$\Rightarrow$] Identification des cellules $\Omega_l$ et $\Omega_m$ de
centre respectif $L$ (variable \var{II}) et $M$ (variable \var{JJ}), se trouvant de chaque c\^ot\'e de la face
$lm$\footnote{La normale \`a la face est orient\'ee de L vers M.}.
\item [$\Rightarrow$] Calcul du carr\'e de la surface de la face. La valeur est
stock\'ee dans le tableau \var{SURFN2}.
\item [$\Rightarrow$] Interpolation du gradient de $R^{\,n}_{ij}$ \`a la face
$lm$ (gradient facette $\left[\grad{R}^{\,n}_{ij}\right]_{\,lm}$) :
\begin{equation}\notag
\left\{\begin{array}{ll}
\var{GRDPX} &= \displaystyle \frac{1}{2} \left(\var{W1(II)} + \var{W1(JJ)}
\right) \\
&\\
\var{GRDPY} &= \displaystyle \frac{1}{2} \left(\var{W2(II)} + \var{W2(JJ)} \right) \\
&\\
\var{GRDPZ} &= \displaystyle \frac{1}{2} \left(\var{W3(II)} + \var{W3(JJ)} \right)
\end{array}\right.
\end{equation}
\item [$\Rightarrow$] Calcul du gradient de $R^{\,n}_{ij}$ normal \`a la face
$lm$, $\left[\grad{R}^{\,n}_{ij}\right]_{\,lm}.\vect{n}_{\,lm}\,S_{\,lm}$.\\

$\displaystyle \var{GRDSN} =  \var{GRDPX} \ \var{SURFAC(1,IFAC)} + \var{GRDPY} \ \var{SURFAC(2,IFAC)} +  \var{GRDPZ} \ \var{SURFAC(3,IFAC)}$
$\var{SURFAC}$ est le vecteur surface de la face \var{IFAC}.


\item [$\Rightarrow$] calcul de
 $\left[\grad{R^{\,n}_{ij}} - (\grad
R^{\,n}_{ij}\,.\,\vect{n}_{\,lm})\vect{n}_{\,lm}\right]$, les vecteurs \'etant
calcul\'es \`a la face $lm$ :
\begin{equation}\notag
\left\{\begin{array}{lll}
&\displaystyle \var{GRDPX} &= \var{GRDPX} - \displaystyle\frac{\var{GRDSN}}{\var{SURFN2}} \ \var{SURFAC(1,IFAC)}\\
&&\\
&\displaystyle \var{GRDPY} &= \var{GRDPY} - \displaystyle\frac{\var{GRDSN}}{\var{SURFN2}} \ \var{SURFAC(2,IFAC)} \\
&&\\
&\displaystyle \var{GRDPZ} &= \var{GRDPZ} - \displaystyle\frac{\var{GRDSN}}{\var{SURFN2}} \ \var{SURFAC(3,IFAC)}
\end{array}\right.
\end{equation}
\item [$\Rightarrow$] finalisation du calcul de l'expression totalement
explicite
 $$\left[ \tens{D}^n\,\left( \grad{R^{\,n}_{ij}} - (\grad R^{\,n}_{ij}\,.\,\vect{n}_{\,lm})\,\vect{n}_{\,lm}\right) \right]\,.\,\vect{n}_{\,lm}$$
\begin{equation}\notag
\begin{array} {ll}
\displaystyle \var{VISCF} = &
 \displaystyle\frac{1}{2} (\ \var{W4(II)} +\ \var{W4(JJ)}) \ \var{GRDPX} \
\var{SURFAC(1,IFAC)})\ + \\
&\\
&  \displaystyle\frac{1}{2} (\ \var{W5(II)} +\ \var{W5(JJ)}) \ \var{GRDPY} \
\var{SURFAC(2,IFAC)})\ + \\
&\\
&  \displaystyle\frac{1}{2} (\ \var{W6(II)} +\ \var{W6(JJ)}) \ \var{GRDPZ} \ \var{SURFAC(3,IFAC)})
\end{array}
\end{equation}
\end{itemize}

\item [$\star$] Mise \`a z\'ero du tableau \var{VISCB}.

\item [$\star$] Appel de \fort{divmas} pour calculer la divergence de~:
 $$\tens{D}^{\,n}\,\left( \grad{R^{\,n}_{ij}} - (\grad R^{\,n}_{ij}\,.\,\vect{n}_{\,lm})\vect{n}_{\,lm}\right)$$ d\'efini aux faces dans \var{VISCF} et \var{VISCB}.

Le r\'esultat est stock\'e dans le tableau \var{W1}.\\
Ajout au second membre \var{SMBR}.\\
$\var{SMBR} = \var{SMBR} + \var{W1}$
\end{itemize}
\item Calcul de la viscosit\'e orthotrope $\gamma^n_{\,lm}$ \`a la face $lm$ de la variable principale
$R^{\,n}_{ij}$\\
Ce calcul permet au sous-programme \fort{codits} de compl\'eter le second membre
\var{SMBR} par :
\begin{equation}
\begin{array} {ll}
& \sum\limits_{m\in Vois(l)}
\mu^n_{\,lm}\,\left(\grad{R}^{\,n}_{ij}\,.\,\vect{n}_{\,lm}\right)S_{\,lm}
 + \sum\limits_{m\in Vois(l)} \left(\grad{R}^{\,n}_{ij}
\,.\,\vect{n}_{\,lm}\right)\left[\tens{D}^{\,n}\,\vect{n}_{\,lm}\right]_{\,lm}\,.\,\vect{n}_{\,lm}\
S_{\,lm}\\
& = \sum\limits_{m\in Vois(l)}(\,\mu^n_{\,lm}\, + \,\gamma^n_{\,lm}\,)\,\left(\grad{R}^{\,n}_{ij}\,.\,\vect{n}_{\,lm}\right)S_{\,lm}
\end{array}
\end{equation}
sans pr\'eciser la nature de la face $lm$, {\it via} l'appel \`a \fort{bilsc2}  et de disposer de la quantit\'e
$(\mu^n_{\,lm}\, + \gamma^n_{\,lm})$ pour construire sa
matrice simplifi\'ee.\\
\begin{itemize}
\item [$\star$] On effectue une boucle d'indice \var{IEL} sur les cellules
$\Omega_l$ :
\begin{itemize}
\item [$\Rightarrow$] $\displaystyle \var{TRRIJ }= \frac{1}{2} (R^{\,n}_{ii})_L $
\item [$\Rightarrow$] $\displaystyle \var{RCSTE} = \rho^n_L \ C_S \ \frac{ (R^{\,n}_{ii})_L}{2\,\varepsilon^n_L} $
\item [$\Rightarrow$] $\displaystyle \var{W1(IEL)} = \mu^n +\rho^n_L \ C_S \ \frac{
(R^{\,n}_{ii})_L}{2\,\varepsilon^n_L}\ (R^n_{11})_L$
\item [$\Rightarrow$] $\displaystyle \var{W2(IEL)} = \mu^n + \rho^n_L \ C_S \ \frac{ (R^{\,n}_{ii})_L}{2\,\varepsilon^n_L}\ (R^n_{22})_L$
\item [$\Rightarrow$] $\displaystyle \var{W3(IEL)} = \mu^n + \rho^n_L \ C_S \ \frac{ (R^{\,n}_{ii})_L}{2\,\varepsilon^n_L}\ (R^n_{33})_L$
\end{itemize}

\item [$\star$] Appel de \fort{visort} pour calculer la viscosit\'e orthotrope
\footnote{Comme dans le sous-programme \fort{viscfa}, on multiplie la viscosit\'e par
$\displaystyle \frac{S_{\,lm}}{\overline{L'M'}}$, o\`u $S_{\,lm}$ et
$\overline{L'M'}$ repr\'esentent respectivement la surface de la face $lm$ et la
mesure alg\'ebrique du segment reliant les projections des centres des cellules
voisines sur la normale \`a la face. On garde dans ce sous-programme  la possibilit\'e d'interpoler la viscosit\'e aux cellules lin\'eairement ou d'utiliser une moyenne harmonique. La viscosit\'e au bord est celle de la cellule de bord correspondante.}
$ \gamma^n_{\,lm} = (\tens{D}^{\,n}\,\vect{n}_{\,lm}).\vect{n}_{\,lm}$ aux faces $lm$

Le r\'esultat est stock\'e dans les tableaux \var{VISCF} et \var{VISCB}.
\end{itemize}

\item appel de \fort{codits} pour la r\'esolution de l'\'equation de
convection/diffusion/termes sources de la variable $R_{ij}$. Le terme source est
\var{SMBR}, la viscosit\'e \var{VISCF} aux faces purement internes (
resp. \var{VISCB} aux faces de bord) et \var{FLUMAS} le flux de masse interne
 ( resp. \var{FLUMAB} flux de masse au bord). Le r\'esultat est la variable $R_{ij}$ au temps
$n+1$, donc $R^{\,n+1}_{ij}$. Elle est stock\'ee dans le tableau \var{RTP} des
variables mises \`a jour.

\end{itemize}

\etape{Appel de \fort{reseps} pour la r\'esolution de la variable $\varepsilon$}

On d\'ecrit ci-dessous le sous-programme \fort{reseps}, les commentaires du sous-programme \fort{resrij} ne seront pas r\'ep\'et\'es puisque les deux sous-programmes ne diff\`erent que par leurs termes sources.

\begin{itemize}
\item Initialisation \`a z\'ero de \var{SMBR} et \var{ROVSDT}.

\item{Lecture et prise en compte des termes sources utilisateur pour la variable $\varepsilon$ :}

Appel de \fort{ustsri} pour charger les termes sources utilisateurs. Ils sont
stock\'es dans les tableaux suivants :\\
pour la cellule $\Omega_l$ repr\'esent\'ee par $\var{IEL}$ de centre $L$, on a :
\begin{equation}\notag
\left\{\begin{array}{lll}
&\var{ROVSDT(IEL)}&= |\Omega_l| \ \alpha_{\varepsilon}\\
&\var{SMBR(IEL)}&=|\Omega_l| \ \beta_{\varepsilon}\\
\end{array}\right.
\end{equation}
On affecte alors les valeurs ad\'equates au second membre \var{SMBR} et \`a la
diagonale \var{ROVSDT} comme suit :
\begin{equation}\notag
\left\{\begin{array}{lll}
&\var{SMBR(IEL)} &= \var{SMBR(IEL)} +\ |\Omega_l| \ \alpha_{\,\varepsilon} \
\varepsilon^n_L \\
&\var{ROVSDT(IEL)}&= \text{max }(-\ |\Omega_l| \ \alpha_{\,\varepsilon},0)\\
\end{array}\right.
\end{equation}

\item{Calcul du terme source de masse si $\Gamma_L > 0$ :
\begin{equation}\notag
\left\{\begin{array}{lll}
&\displaystyle \var{SMBR(IEL)} = \var{SMBR(IEL)} + |\Omega_l| \ \Gamma_L \
(\varepsilon^{\,in}_L -\varepsilon^n_L) \\
&\displaystyle \var{ROVSDT(IEL)}= \var{ROVSDT(IEL)} + |\Omega_l| \ \Gamma_L
\end{array}\right.
\end{equation}
\item Calcul du terme d'accumulation de masse et du terme instationnaire \\
On stocke $\displaystyle \var{W1}= \int_{\Omega_l}\dive\,(\rho\,\vect{u})\,d\Omega$
calcul\'e par \fort{divmas} \`a l'aide des flux de masse internes et aux bords.\\
On incr\'emente ensuite \var{SMBR} et \var{ROVSDT}.
\begin{equation}\notag
\left\{\begin{array}{lll}
&\var{SMBR(IEL)} &= \var{SMBR(IEL)} + \var{ICONV}\ \varepsilon^n_L\,(\displaystyle
\int_{\Omega_l}\dive\,(\rho\,\vect{u})\ d\Omega) \\
&\var{ROVSDT(IEL)}& = \var{ROVSDT(IEL)} +  \var{ISTAT}\,\displaystyle
\frac{\rho^n_L \ |\Omega_l|}{\Delta t^n} -  \var{ICONV}\ (\displaystyle
\int_{\Omega_l}\dive\,(\rho\,\vect{u})\ d\Omega) \\
\end{array}\right.
\end{equation}

\item Traitement du terme de production
 $\displaystyle \rho\,C_{\varepsilon_1}\,\frac{\varepsilon}{k}\,\mathcal{P}$
 et du terme de dissipation $-\,\displaystyle \rho\,C_{\varepsilon_2}\,\frac{\varepsilon}{k}\,\varepsilon$ \\
pour cela, on effectue une boucle d'indice \var{IEL} sur les cellules $\Omega_l$
de centre $L$ :
\begin{itemize}
\item [$\Rightarrow$] $\displaystyle \var{TRPROD}= \frac{1}{2} (\mathcal{P}^n_{ii})_L = \frac{1}{2} \left[ \var{PRODUC(1,IEL)} +  \var{PRODUC(2,IEL)} +  \var{PRODUC(3,IEL)} \right] $
\item [$\Rightarrow$] $\displaystyle \var{TRRIJ }= \frac{1}{2} (R^n_{ii})_L $
\item [$\Rightarrow$] $\displaystyle \var{SMBR(IEL)} = \var{SMBR(IEL)} + \rho^n_L
|\Omega_l| \left[ -C_{\varepsilon_2} \ \frac{2\,(\varepsilon^n_L)^2}{(R^n_{ii})_L} + C_{\varepsilon_1} \ \frac{\varepsilon^n_L}{(R^n_{ii})_L}\ (\mathcal{P}^n_{ii})_L \right] $
\item [$\Rightarrow$] $\displaystyle \var{ROVSDT(IEL)} = \var{ROVSDT(IEL)} + C_{\varepsilon_2} \ \rho^n_L \ |\Omega_l| \ \frac{2\,\varepsilon^n_L}{(R^n_{ii})_L}$
\end{itemize}

\item Appel de \fort{rijthe} pour le calcul des termes de gravit\'e $\mathcal{G}^n_{\varepsilon}$ et ajout dans \var{SMBR}.

$ \var{SMBR} = \var{SMBR} + \mathcal{G}^n_{\varepsilon}$\\
Ce calcul n'a lieu que si $\var{IGRARI()} = 1$.

\item Calcul de la diffusion de $\varepsilon$ \\
 Le terme $\dive \left[\mu\, \grad(\varepsilon) + \tens{A'}\,\grad(\varepsilon)
\right]$ est calcul\'e exactement de la m\^eme mani\`ere que pour les tensions
de Reynolds $R_{ij}$ en rempla\c cant $\tens{A}$ par $\tens{A'}$.

\item Appel de \fort{codits} pour la r\'esolution de l'\'equation de
convection/diffusion/termes sources de la variable principale $\varepsilon$. Le
r\'esultat $\varepsilon^{\,n+1}$ est stock\'e dans le tableau \var{RTP} des
variables mises \`a jour.
}
\end{itemize}

\etape{clippings finaux}
On passe enfin dans le sous-programme  \fort{clprij} pour faire un clipping \'eventuel
des variables $R^{\,n+1}_{ij}$ et $\varepsilon^{\,n+1}$. Le sous-programme
\fort{clprij} est appel\'e\footnote{L'option
$\var{ICLIP} = 1$ consiste en un clipping minimal des variables $R_{ii}$ et
$\varepsilon$ en prenant la valeur absolue de ces variables puisqu'elles ne
peuvent \^etre que positives.} avec $\var{ICLIP} = 2$ . Cette option
\footnote{Quand la valeur des grandeurs $R_{ii}$ ou $\varepsilon$ est
n\'egative, on la remplace par le minimum entre sa valeur absolue et (1,1)
fois la valeur obtenue au pas de temps pr\'ec\'edent.} contient l'option $\var{ICLIP} = 1$  et permet de v\'erifier l'in\'egalit\'e de Cauchy-Schwarz sur les grandeurs extra-diagonales du tenseur $\tens{R}$ (pour $i \neq j$, $|R_{ij}|^2 \le R_{ii} R_{jj}$).


%%%%%%%%%%%%%%%%%%%%%%%%%%%%%%%%%%
%%%%%%%%%%%%%%%%%%%%%%%%%%%%%%%%%%
\section{Points \`a traiter}
%%%%%%%%%%%%%%%%%%%%%%%%%%%%%%%%%%
%%%%%%%%%%%%%%%%%%%%%%%%%%%%%%%%%%
Sauf mention explicite, $\phi$ repr\'esentera une tension de Reynolds ou la dissipation turbulente ($\phi = R_{ij} \ \text{ou} \ \varepsilon$).

\begin{itemize}
\item {La vitesse utilis\'ee pour le calcul de la production est explicite. Est-ce qu'une implicitation peut am\'eliorer la pr\'ecision temporelle de $\phi$ \footnote{Cette remarque peut \^etre g\'en\'eralis\'ee. En effet, peut-on envisager d'actualiser les variables d\'ej\`a r\'esolues (sans r\'eactualiser les variables turbulentes apr\`es leur r\'esolution)? Ceci obligerait \`a modifier les sous-programmes tels que \fort{condli} qui sont appel\'es au d\'ebut de la boucle en temps.} ?}
\item {Dans quelle mesure le terme d'\'echo de paroi est-il valide ? En effet, ce terme est remis en question par certains auteurs.}
\item {On peut envisager la r\'esolution d'un syst\`eme hyperbolique pour les
tensions de Reynolds afin d'introduire un couplage avec le champ de vitesse.}
\item {Le flux au bord \var{VISCB} est annul\'e dans le sous-programme
\fort{vectds}. Peut-on envisager de mettre au bord la valeur de la variable
concern\'ee \`a la cellule de bord correspondant? De m\^eme, il faudrait se
pencher sur les hypoth\`eses sous-jacentes \`a l'annulation des contributions
aux bords de \var{VISCB} lors du calcul de : $$\left[ \tens{D}^n\,\left( \grad{R^{\,n}_{ij}} - (\grad R^{\,n}_{ij}\,.\,\vect{n}_{\,lm})\,\vect{n}_{\,lm}\right) \right]\,.\,\vect{n}_{\,lm}.$$}
\item {Un probl\`eme de pond\'eration appara\^\i t plus g\'en\'eralement. Si on prend la partie explicite de $\tens{D}\,\grad(\phi)$, la pond\'eration aux faces internes utilise le coefficient $\displaystyle\frac{1}{2}$ avec pond\'eration s\'epar\'ee de $\tens{D}$ et $\grad(\phi)$, alors que pour $\tens{E}\,\grad(\phi)$, on calcule d'abord ce terme aux cellules pour ensuite l'interpoler lin\'eairement aux faces \footnote{Cette interpolation se fait dans \fort{vectds} avec des coefficients de pond\'eration aux faces.}. Ceci donne donc deux types d'interpolations pour des termes de m\^eme nature.}
\item {On laisse la possibilit\'e dans \fort{visort} d'utiliser une moyenne
harmonique aux faces. Est-ce que ceci est valable puisque les interpolations
utilis\'ees lors du calcul de la partie explicite de $\tens{A}\,\grad{\phi}$
sont des moyennes arithm\'etiques ?}
\item {Les techniques adopt\'ees lors du clipping sont \`a revoir.}
\item {On utilise dans le cadre du mod\`ele $\displaystyle R_{ij}-\varepsilon $ une semi-implicitation de termes comme $\displaystyle \phi_{ij,1}$ ou $\displaystyle -\rho\,C_{\varepsilon_2}\,\frac{\varepsilon}{k}\,\varepsilon$. On peut envisager le m\^eme type d'implicitation dans \fort{turbke} m\^eme en pr\'esence du couplage $\displaystyle k-\varepsilon$.}
\item L'adoption d'une r\'esolution d\'ecoupl\'ee fait perdre l'invariance par rotation.
\item La formulation et l'implantation des conditions aux limites de paroi
devront \^etre v\'erifi\'ees. En effet, il semblerait que, dans certains cas, des ph\'enom\`enes
``oscillatoires'' apparaissent, sans qu'il soit ais\'e d'en d\'eterminer la cause.
\item L'implicitation partielle (du fait de la r\'esolution d\'ecoupl\'ee) des
conditions aux limites conduit souvent \`a des calculs instables. Il
conviendrait d'en conna\^\i tre la raison. L'implicitation partielle avait
\'et\'e mise en \oe uvre afin de tenter d'utiliser un pas de temps plus grand
dans le cas de jets axisym\'etriques en particulier.

\end{itemize}

%                      Code_Saturne version 1.3
%                      ------------------------
%
%     This file is part of the Code_Saturne Kernel, element of the
%     Code_Saturne CFD tool.
%
%     Copyright (C) 1998-2007 EDF S.A., France
%
%     contact: saturne-support@edf.fr
%
%     The Code_Saturne Kernel is free software; you can redistribute it
%     and/or modify it under the terms of the GNU General Public License
%     as published by the Free Software Foundation; either version 2 of
%     the License, or (at your option) any later version.
%
%     The Code_Saturne Kernel is distributed in the hope that it will be
%     useful, but WITHOUT ANY WARRANTY; without even the implied warranty
%     of MERCHANTABILITY or FITNESS FOR A PARTICULAR PURPOSE.  See the
%     GNU General Public License for more details.
%
%     You should have received a copy of the GNU General Public License
%     along with the Code_Saturne Kernel; if not, write to the
%     Free Software Foundation, Inc.,
%     51 Franklin St, Fifth Floor,
%     Boston, MA  02110-1301  USA
%
%-----------------------------------------------------------------------
%
\programme{vortex}
%
\vspace{1cm}
%%%%%%%%%%%%%%%%%%%%%%%%%%%%%%%%%%
%%%%%%%%%%%%%%%%%%%%%%%%%%%%%%%%%%
\section{Fonction}
%%%%%%%%%%%%%%%%%%%%%%%%%%%%%%%%%%
%%%%%%%%%%%%%%%%%%%%%%%%%%%%%%%%%%
Ce sous-programme est d�di� � la g�n�ration des conditions d'entr�e
turbulente utilis�es en LES.


La m�thode des vortex est bas�e sur une approche de tourbillons
ponctuels. L'id�e de la m�thode consiste � injecter des tourbillons 2D dans le
plan d'entr�e du calcul, puis � calculer le champ de vitesse induit par ces
tourbillons au centre des faces d'entr�e.

%                      Code_Saturne version 1.3
%                      ------------------------
%
%     This file is part of the Code_Saturne Kernel, element of the
%     Code_Saturne CFD tool.
% 
%     Copyright (C) 1998-2007 EDF S.A., France
%
%     contact: saturne-support@edf.fr
% 
%     The Code_Saturne Kernel is free software; you can redistribute it
%     and/or modify it under the terms of the GNU General Public License
%     as published by the Free Software Foundation; either version 2 of
%     the License, or (at your option) any later version.
% 
%     The Code_Saturne Kernel is distributed in the hope that it will be
%     useful, but WITHOUT ANY WARRANTY; without even the implied warranty
%     of MERCHANTABILITY or FITNESS FOR A PARTICULAR PURPOSE.  See the
%     GNU General Public License for more details.
% 
%     You should have received a copy of the GNU General Public License
%     along with the Code_Saturne Kernel; if not, write to the
%     Free Software Foundation, Inc.,
%     51 Franklin St, Fifth Floor,
%     Boston, MA  02110-1301  USA
%
%-----------------------------------------------------------------------
%
%%%%%%%%%%%%%%%%%%%%%%%%%%%%%%%%%%
%%%%%%%%%%%%%%%%%%%%%%%%%%%%%%%%%%
\section{Discr\'etisation}
%%%%%%%%%%%%%%%%%%%%%%%%%%%%%%%%%%
%%%%%%%%%%%%%%%%%%%%%%%%%%%%%%%%%%

Le terme convectif en $\dive(\underline{u} \otimes \rho\,\underline{u})$
introduit une non lin\'earit\'e et un couplage des composantes de la vitesse
$\vect{u}$ dans l'�quation (\ref{Base_Preduv_eqqdm}). Une lin\'earisation et un d\'ecouplage
des trois composantes de la 
vitesse sont r\'ealis\'es lors de la discr\'etisation de cette \'etape de
pr\'ediction.\\
En effet, soit :
\begin{equation}
\vect{\widetilde{u}}= \vect{u}^n + \delta \vect{u} 
\end{equation}
La contribution exacte du terme convectif \`a prendre en compte dans cette
\'etape de pr\'ediction serait :\\
\begin{equation}\label{Base_Preduv_Conv_exact}
\begin{array}{ll}
\dive(\vect{\widetilde{u}} \otimes \rho\,\vect{\widetilde{u}}) =
\dive(\vect{u}^{n} \otimes \rho\,\vect{u}^{n}) + \dive(\delta \vect{u} \otimes
\rho\,\vect{u}^{n}) +  \underbrace { \dive(\vect{u}^{n} \otimes
\rho\,\delta \vect{u})}_{\text {terme couplant lin\'eaire}} +  \underbrace { \dive(\delta \vect{u} \otimes
\rho\,\delta \vect{u})}_{\text {terme couplant et non lin\'eaire}}\\
\end{array} 
\end{equation}
Les deux derniers termes de l'expression (\ref{Base_Preduv_Conv_exact}) sont {\em a priori} n�glig�s
de mani�re � obtenir un syst\`eme en vitesse qui soit d\'ecoupl\'e et donc,
�viter l'inversion d'une matrice pouvant \^etre de tr\`es grande taille. Ces
deux termes peuvent n�anmoins �tre pris en compte de mani�re plus ou moins
approch�e par extrapolation explicite du flux de masse en $n+\theta_F$ (pour le
terme couplant lin�aire provenant de la convection de $\vect{u}^{n}$ par $\delta
\vect{u}$) et utilisation d'un point-fixe par sous it�ration sur le sous
programme \fort{navsto} (pour le terme non-lin�aire, en sp�cifiant $\var{NTERUP}>1$).

L'�quation (\ref{Base_Preduv_eqqdm}) est discr�tis�e au temps $n+\theta$ � l'aide d'un
$\theta$-sch�ma, et le tenseur des pertes de charges d�compos� en une partie
diagonale $\tens{K}_{d}$ et une extradiagonale $\tens{K}_{e}$ (soit
 $\tens{K}_{pdc}=\tens{K}_{d}+\tens{K}_{e}$).\\
$\bullet$ La pression est suppos�e connue en $n-1+\theta$ (d�calage temporel
pression-vitesse) et le gradient naturellement calcul� � cet instant.\\ 
$\bullet$ Les termes sources de viscosit� secondaire, de gradient transpos\'e,
ceux provenant du mod�le de turbulence\footnote{except� $\dive (\mu_t\ (\ggrad
\underline {u}))$}, $\rho\,\tens{K}_{\,e}\ \underline{u}$, $(\rho -\rho_0)
\underline {g}$ ainsi que $\underline{T}_{s}^{\,exp}$ et
$\Gamma\,\underline{u}_{\,i}$ sont pris de mani�re explicite au temps $n$, ou
extrapol�s suivant le sch�ma en temps choisi pour les propri�t�s physique et les
termes sources.\\ 
$\bullet$ Les termes sources $\underline{u}\,\,\dive (\rho\,\underline {u})$,
$\Gamma\,\,\underline{u}$, $T_{s}^{\,imp}\,\,\underline{u}$ et
$-\rho\,\tens{K}_{\,d}\,\,\underline{u}$ sont implicit�s est calcul�s �
l'instant $n+\theta$.\\ 
$\bullet$ Le terme de diffusion $\dive (\mu_{\,tot}\,\ggrad \underline{u})$ est
implicit� : la vitesse est prise � l'instant $n+\theta$ et la viscosit�
explicit�e ou extrapol�e.\\ 
$\bullet$ Enfin, le terme de convection en $\dive(\,\underline{u} \otimes
(\rho\underline{u})\,)$ est implicit� : la composante r�solue de la vitesse est
prise en $n+\theta$, et le flux de masse, explicit�, ou extrapol� en
$n+\theta_F$. 

Par souci de clart�, on suppose, en l'absence d'indication, que les propri�tes
physiques $\Phi$ ($\rho,\,\mu_{tot},\,...$) et le flux de masse
$(\rho\underline{u})$ sont pris respectivement aux instants $n+\theta_\Phi$ et
$n+\theta_F$, o� $\theta_\Phi$ et $\theta_F$ d�pendent des sch�mas en temps
sp�cifiquement utilis�s pour ces grandeurs\footnote{cf. \fort{introd}}. 

La discr�tisation temporelle de l'�quation (\ref{Base_Preduv_eqqdm}) s'�crit alors comme suit : 

\begin{equation}\label{Base_Preduv_eq_di1}
 \begin{array}{c}
\displaystyle \rho\,\ \frac{ \underline {\widetilde{u}}^{n+1} -\underline {u}^{n} }
{\Delta t} + \dive(\,\underline{\widetilde{u}}^{n+\theta} \otimes (\rho\underline{u})\,) -\dive
(\mu_{\,tot}\,\ggrad \underline{\widetilde{u}}^{n+\theta}) =
\\
\displaystyle
 - \grad p^{n-1+\theta} + \dive (\rho\,\underline {u})\,\underline{\widetilde{u}}^{n+\theta} +(\Gamma\,\underline{u}_{\,i})^{n+\theta_S}-\Gamma^n\,\,\underline{\widetilde{u}}^{n+\theta}
\\
\begin{array}{c}
\displaystyle
- \rho\,\tens{K}_{\,d}^{n}\,\,\underline{\widetilde{u}}^{n+\theta} - (\rho\,\tens{K}_{\,e}\ \underline{u})^{n+\theta_S} + (\underline{T}_{s}^{\,exp})^{\,n+\theta_S} + T_{s}^{\,imp}\,\,\underline{\widetilde{u}}^{n+\theta}
\\
\displaystyle
+[\dive (\mu_{\,tot}\,^t\ggrad \underline {u})]^{n+\theta_S}-\frac {2} {3}[\,\grad (\mu_{\,tot}\,\dive \underline {u})]^{n+\theta_S} + (\rho -\rho_0) \underline {g}
 - (\underline{turb})^{n+\theta_S}
\end{array}
\end{array}
\end{equation}
o\`u, par souci de simplification, on a pos\'e :
\begin{equation}
\mu_{\,tot}=
\begin{cases}
\mu+\mu_t & \text{pour les mod�les � viscosit� turbulente ou en LES}, \\
\mu & \text{pour les mod�les au second ordre ou en laminaire}
\end{cases} \ 
\end{equation}
\\
et :
\begin{equation}
\underline{turb}^{n}=
\begin{cases}
\displaystyle\frac {2}{3}\grad (\rho^{n}\,k^{n}) & \text{pour les mod�les � viscosit� turbulente}, \\
\dive(\rho^{n}\,\tens{R}^n) & \text{pour les mod�les au second ordre},\\
0 & \text{en laminaire ou en LES}\\
\end{cases}
\end{equation}
Par analogie avec l'�criture du $\theta$-sch�ma pour une variable scalaire, $\,
\underline {\widetilde{u}}^{n+\theta}$ est interpol�e � partir de la vitesse
pr�dite $\underline {\widetilde{u}}^{n+1}$ de la mani\`ere suivante\footnote{si
$\theta=1/2$, ou qu'une extrapolation est utilis�e, l'ordre 2 n'est obtenu que si
le pas de temps $\Delta t$ est uniforme en temps et en espace.}~: 
\begin{equation}
\underline {\widetilde{u}}^{n+\theta}=\theta\, \underline
{\widetilde{u}}^{n+1}+(1-\theta)\, \underline {u}^{n}\\ 
\end{equation}
Avec :
\begin{equation}
\left\{
\begin{array}{ll}
\theta = 1   & \text{Pour un sch\'ema de type Euler implicite d'ordre 1.}\\
\theta = 1/2 & \text{Pour un sch\'ema de type Cranck-Nicolson d'ordre 2.}\\
\end{array}
\right.
\end{equation}

L'�quation (\ref{Base_Preduv_eq_di1}) est alors r��crite sous la forme :

\begin{equation}\label{Base_Preduv_eq_di2}
\begin{array}{c}
\displaystyle \underbrace{\left(\frac{\rho}{\Delta t} -\theta \,\dive (\rho\,\underline {u}) +\theta \,\, \Gamma^n +
\theta \,\, \rho\,\tens{K}_{\,d}^n-\theta \,T_s^{\,imp} \right)}_{\displaystyle f_s^{imp}}\, (\underline {\,\widetilde{u}}^{n+1} -\underline {u}^{n})
\\
 +\, \theta\, \dive(\underline {\widetilde{u}}^{n+1} \otimes (\rho\underline{u}))-\, \theta\,\dive (\mu_{\,tot}\,\ggrad \underline {\widetilde{u}}^{n+1}) =
\\
-\,(1-\theta)\, \dive(\underline {u}^{n} \otimes (\rho\underline{u})) +\,(1-\theta)\,\dive (\mu_{\,tot}\,\ggrad \underline {u}^{n})
\\
f_s^{exp}\left\{
\begin{array}{c}
\displaystyle 
- \grad p^{n-1+\theta} + \dive (\rho\,\underline {u})\,\underline{u}^{n} +\,(\,\Gamma^{n}\,\underline{u}_{\,i}\,)^{n+\theta_S}- \Gamma^n\,\,\underline{u}^{n}
\\
\displaystyle
-(\,\rho\,\tens{K}_{\,e}\ \underline{u}\,)^{n+\theta_S} -\rho\,\tens{K}_{\,d}^n\ \underline{u}^{n}+ (\underline{T}_{s}^{\,exp})^{\,n+\theta_S} + T_s^{\,imp}\,\,\underline {u}^{n} 
\\
\displaystyle
+[\dive (\mu_{\,tot}\,^t\ggrad \underline {u}\,)]^{n+\theta_S}-\frac {2} {3}[\,\grad (\mu_{\,tot}\,\dive \underline {u}\,)]^{n+\theta_S} + (\rho -\rho_0) \underline {g}-(\underline{turb})^{n+\theta_S}
\end{array}
\right.
\end{array}
\end{equation}

d'o� l'�quation r�solue par le sous-programme \fort{codits} :
\begin{equation}\begin{array}{c}
\displaystyle
f_s^{\,imp}(\underline {\widetilde{u}}^{n+1}-\underline {u}^{n}) + \theta\, \dive(\underline{\widetilde{u}}^{n+1} \otimes (\rho
\underline{u})) - \theta\,\dive (\,\mu_{\,tot}\,\ggrad \underline{\widetilde{u}}^{n+1}) = 
\\\\
\displaystyle
-(1-\theta)\,\dive(\underline{u}^{n} \otimes (\rho \underline{u}))+(1-\theta)\,\dive (\,\mu_{\,tot}\,\ggrad \underline{u}^{n})
+ \underline{f}_{\,s}^{\,exp}
\end{array}
\end{equation}
La m\'ethode de discr\'etisation spatiale est d\'evelopp\'ee dans le sous-programme \fort{codits}.\\



\minititre{Remarques :}
{\tiny$\blacksquare$} Dans le cas standard sans extrapolation, le terme
$-\,T_s^{\,imp}$ n'est ajout� � $f_s^{\,imp}$ que s'il est positif afin de ne
pas affaiblir la dominance de la diagonale de la matrice � inverser.\\ 
{\tiny$\blacksquare$} Si une extrapolation est utilis�e, par contre,
$\,T_s^{\,imp}$ est ajout� � $f_s^{\,imp}$ quel que soit son signe. En effet, l'id�e intuitive qui
consiste � prendre~: 
\begin{equation}
\begin{cases}
\displaystyle
(\underline{T}_{s}^{\,exp} + T_{s}^{\,imp}\,\underline {u})^{\,n+\theta_S} &
\text{si } T_{s}^{\,imp} > 0\\ 
\displaystyle
(\underline{T}_{s}^{\,exp})^{\,n+\theta_S} + T_{s}^{\,imp}\,\underline{u}^{n+\theta} &\text{sinon}\\
\end{cases}
\end{equation} 
aboutit � une incoh�rence dans le traitement si $T_s^{imp}$ change de signe
entre deux pas de temps.\\ 
{\tiny$\blacksquare$} la partie diagonale $\tens{K}_{\,d}$ du terme
de perte de charge est utilis�e dans $f_s^{\,imp}$. En fait, pour \^etre rigoureux,
il faudrait ne retenir que les contributions positives (point signal\'e dans le
sous-programme utilisateur associ\'e \fort{uskpdc}). Cette prise en compte sera \`a am\'eliorer.\\
{\tiny$\blacksquare$} Le terme $\theta\,\Gamma^{n}-\theta\,\dive
(\rho\,\underline {u})$ ne pose pas de probl�me pour la 
dominance de la diagonale de la matrice car il est exactement compens� par le
terme de convection (cf. \fort{covofi}). 


%                      Code_Saturne version 1.3
%                      ------------------------
%
%     This file is part of the Code_Saturne Kernel, element of the
%     Code_Saturne CFD tool.
%
%     Copyright (C) 1998-2007 EDF S.A., France
%
%     contact: saturne-support@edf.fr
%
%     The Code_Saturne Kernel is free software; you can redistribute it
%     and/or modify it under the terms of the GNU General Public License
%     as published by the Free Software Foundation; either version 2 of
%     the License, or (at your option) any later version.
%
%     The Code_Saturne Kernel is distributed in the hope that it will be
%     useful, but WITHOUT ANY WARRANTY; without even the implied warranty
%     of MERCHANTABILITY or FITNESS FOR A PARTICULAR PURPOSE.  See the
%     GNU General Public License for more details.
%
%     You should have received a copy of the GNU General Public License
%     along with the Code_Saturne Kernel; if not, write to the
%     Free Software Foundation, Inc.,
%     51 Franklin St, Fifth Floor,
%     Boston, MA  02110-1301  USA
%
%-----------------------------------------------------------------------
%

%%%%%%%%%%%%%%%%%%%%%%%%%%%%%%%%%%
%%%%%%%%%%%%%%%%%%%%%%%%%%%%%%%%%%
\section{Mise en \oe uvre}
%%%%%%%%%%%%%%%%%%%%%%%%%%%%%%%%%%
%%%%%%%%%%%%%%%%%%%%%%%%%%%%%%%%%%
La num\'ero de la phase trait\'ee fait partie des arguments de \fort{turrij}. On
omettra volontairement de le pr\'eciser dans ce qui suit, on indiquera par $(\ )$ la
notion de tableau s'y rattachant.

\etape{Calcul des termes de production $\tens{\mathcal{P}}$}
\begin{itemize}
\item [$\star$] Initialisation \`a z\'ero du tableau \var{PRODUC} dimensionn\'e \`a $\var{NCEL}\times 6$.
\item [$\star$] On appelle trois fois \fort{grdcel} pour calculer les gradients des composantes de la vitesse $u$, $v$ et
$w$ prises au temps $n$.

Au final, on a :\\
$\displaystyle
\begin{array} {ll}
\var{PRODUC(1,IEL)} = & \displaystyle - 2 \left[ R_{11}^{\,n} \frac{\partial u^{\,n}} {\partial x} +R_{12}^{\,n} \frac{\partial u^{\,n}} {\partial y}+R_{13}^{\,n} \frac{\partial u^{\,n}} {\partial z} \right] \text{        (production de $R_{11}^{\,n}$)}\\
\var{PRODUC(2,IEL)} = & \displaystyle - 2 \left[ R_{12}^{\,n} \frac{\partial v^{\,n}} {\partial x} +R_{22}^{\,n} \frac{\partial v^{\,n}} {\partial y}+R_{23}^{\,n} \frac{\partial v^{\,n}} {\partial z} \right] \text{        (production de $R_{22}^{\,n}$)}\\
\var{PRODUC(3,IEL)} = & \displaystyle - 2 \left[ R_{13}^{\,n} \frac{\partial w^{\,n}} {\partial x} +R_{23}^{\,n} \frac{\partial w^{\,n}} {\partial y}+R_{33}^{\,n} \frac{\partial w^{\,n}} {\partial z} \right] \text{        (production de $R_{33}^{\,n}$)}\\
\var{PRODUC(4,IEL)} = & \displaystyle - \left[ R_{12}^{\,n} \frac{\partial u^{\,n}} {\partial x} +R_{22}^{\,n} \frac{\partial u^{\,n}} {\partial y}+R_{23}^{\,n} \frac{\partial u^{\,n}} {\partial z} \right] \\
& \displaystyle - \left[ R_{11}^{\,n} \frac{\partial v^{\,n}} {\partial x} +R_{12}^{\,n} \frac{\partial v^{\,n}} {\partial y}+R_{13}^{\,n} \frac{\partial v^{\,n}} {\partial z} \right] \text{        (production de $R_{12}^{\,n}$)} \\
\var{PRODUC(5,IEL)} = & \displaystyle - \left[ R_{13}^{\,n} \frac{\partial u^{\,n}} {\partial x} +R_{23}^{\,n} \frac{\partial u^{\,n}} {\partial y}+R_{33}^{\,n} \frac{\partial u^{\,n}} {\partial z} \right] \\
& \displaystyle - \left[ R_{11}^{\,n} \frac{\partial w^{\,n}} {\partial x} +R_{12}^{\,n} \frac{\partial w^{\,n}} {\partial y}+R_{13}^{\,n} \frac{\partial w^{\,n}} {\partial z} \right] \text{        (production de $R_{13}^{\,n}$)} \\
\var{PRODUC(6,IEL)} = & \displaystyle - \left[ R_{13}^{\,n} \frac{\partial v^{\,n}} {\partial x} +R_{23}^{\,n} \frac{\partial v^{\,n}} {\partial y}+R_{33}^{\,n} \frac{\partial v^{\,n}} {\partial z} \right] \\
& \displaystyle - \left[ R_{12}^{\,n} \frac{\partial w^{\,n}} {\partial x} +R_{22}^{\,n} \frac{\partial w^{\,n}} {\partial y}+R_{23}^{\,n} \frac{\partial w^{\,n}} {\partial z} \right]  \text{        (production de $R_{23}^{\,n}$)}
\end{array}
$
\end{itemize}

\etape{Calcul du gradient de la masse volumique $\rho^n$ prise au d\'ebut du pas
de temps courant\footnote{{\it i.e.} calcul\'ee \`a partir des
variables du pas de temps pr\'ec\'edent $n$ si n\'ecessaire.} $(n+1)$}
Ce calcul n'a lieu que si les termes de gravit\'e doivent \^etre pris en compte
($\var{IGRARI()} =1$).
\begin{itemize}
\item [$\star$] Appel de \fort{grdcel}  pour calculer le gradient de $\rho^n$
dans les trois directions de l'espace. Les conditions aux limites sur $\rho^n$
sont des conditions de Dirichlet puisque la valeur de $\rho^n$ aux faces de bord
$ik$ (variable \var{IFAC}) est connue et vaut $\rho_{\,b_{\,ik}}$. Pour \'ecrire les conditions aux limites
sous la forme habituelle, $$\rho_{\,b_{\,ik}} = \var{COEFA} + \var{COEFB}
\,\rho^n_{\,I'}$$ on pose alors $\var{COEFA}=
\var{PROPCE(IFAC,IPPROB(IROM(IPHAS)))}$ et $\var{COEFB} = \var{VISCB} = 0$.\\
$\var{PROPCE(1,IPPROB(IROM(IPHAS)))}$ (resp.$\var{VISCB}$) est utilis\'e en lieu
et place de l'habituel \var{COEFA} ($\var{COEFB}$), lors de l'appel \`a \fort{grdcel}.\\
On a donc :\\
$\displaystyle \var{GRAROX}= \frac{\partial \rho^n}{\partial x}\ $,$\displaystyle \ \var{GRAROY}= \frac{\partial
\rho^n}{\partial y}$ et $
\displaystyle \ \var{GRAROZ}= \frac{\partial \rho^n}{\partial z}\ $.

\end{itemize}

Le gradient de $\rho^n$ servira \`a calculer les termes de production par effets de gravit\'e si ces derniers sont pris en compte.

\etape{Boucle \var{ISOU} de $1$ \`a $6$ sur les tensions de Reynolds}
Pour $\var{ISOU} = 1,2,3,4,5,6$, on r\'esout respectivement et dans
l'ordre  les
\'equations de $R_{11}$, $R_{22}$, $R_{33}$, $R_{12}$, $R_{13}$ et $R_{23}$ par
l'appel au sous-programme \fort{resrij}.\\
La r\'esolution se fait par incr\'ement $\delta {R}_{ij}^{\,n+1,k+1}$ , en utilisant la m\^eme m\'ethode que
celle d\'ecrite dans le sous-programme \fort{codits}. On adopte ici les m\^emes notations.
\var{SMBR} est le second membre du syst\`eme \`a inverser, syst\`eme portant sur
les incr\'ements de la variable. \var{ROVSDT} repr\'esente la diagonale de la
matrice, hors convection/diffusion.\\
On va r\'esoudre l'\'equation (\ref{Base_Turrij_Eq_Temp_Rij}) sous forme incr\'ementale en
utilisant \fort{codits}, soit :
\begin{equation}\label{Base_Turrij_Eq_Temp_deltaRij}
\begin{array}{ll}
&\displaystyle \underbrace{\left(\frac {\rho^n_L}{\Delta t^n}
+ \rho^n_L \,C_1\,\frac{\varepsilon^n_L}{k^n_L}(1-\frac{\delta_{ij}}{3})
 - m^n_{\,lm} + \Gamma_L\,+ max(-\alpha^n_{R_{ij}},0)\right)\,|\Omega_l|}
_{\text {$\var{ROVSDT}$ contribuant
\`a la diagonale de la matrice simplifi\'ee de \fort{matrix}}}\,(\delta{R}_{ij}^{\,n+1,p+1})_{\,L}\\\\
&  \underbrace{+\sum\limits_{m\in Vois(l)}\displaystyle \left[
 m^n_{\,lm} \delta R_{ij,\,f_{\,lm}}^{\,n+1,p+1}
- (\mu^n_{\,lm} + \gamma^n_{\,lm})\
\frac{({\delta R}_{ij}^{\,n+1,p+1})_{M}-({\delta R}_{ij}^{\,n+1,p+1})_{L})}{\overline{L'M'}}\,
S_{\,lm} \right]}_{\text { convection upwind pur et diffusion non reconstruite
relatives \`a la matrice simplifi\'ee de \fort{matrix}\footnotemark}} \\
% voir le texte de la footmark plus bas
&= - \displaystyle\frac {\rho^n_L}{\Delta t^n}\,\left(\,(R^{\,n+1,p}_{ij})_L - (R^{\,n}_{ij})_L\,\right)\\
&-\,\underbrace{\displaystyle\int_{\Omega_l} \left(
\dive\,[\,(\rho\,\vect{u})^n\,R^{\,n+1,p}_{ij} - (\mu^n\,+ \gamma^n\,)\,
\grad{R^{\,n+1,p}_{ij}}\,]\right)\,d\Omega}_{\text {convection et diffusion
trait\'ees par \fort{bilsc2}}}\\
&+\displaystyle \int_{\Omega_l} \left[\,\mathcal{P}^{\,n+1,p}_{ij} + \mathcal{G}^{\,n+1,p}_{ij}
- \displaystyle\rho^n \,C_1\,\frac{\varepsilon^n}{k^n}\left[R^{\,n+1,p}_{ij}-
\frac{2}{3}\,k^n\,\delta_{ij}\right] + \phi^{\,n+1,p}_{ij,2} +
\phi^{\,n+1,p}_{ij,w}\,\right]\, d\Omega \\
& + \displaystyle\int_{\Omega_l} \left[- \frac{2}{3} \rho^n \varepsilon^n \delta_{ij}
 + \Gamma\,(\,R^{\,in}_{ij} - R^{\,n+1,p}_{ij}\,) +
\alpha^n_{R_{ij}}\,R^{\,n+1,p}_{ij}+ \beta^n_{R_{ij}}\right]\, d\Omega\\
&+ \sum\limits_{m\in
Vois(l)}\displaystyle \left[\ \tens{E}^n\,\grad{R}^{\,n+1,p}_{ij} \right]_{\,lm}\,.\,\vect{n}_{\,lm}S_{\,lm}\\
&+ \sum\limits_{m\in Vois(l)}\displaystyle \left[\
\tens{D}^n\,\grad{R}^{\,n+1,p}_{ij} \right]_{\,lm}\,.\,\vect{n}_{\,lm}S_{\,lm}\\
&- \sum\limits_{m\in Vois(l)} \gamma^n_{\,lm} \left( \grad{R}^{\,n+1,p}_{ij}\,.\,\vect{n}_{\,lm} \right)  S_{\,lm}\\
&+ \sum\limits_{m\in Vois(l)}  m^n_{\,lm}\,(R^{\,n+1,p}_{ij})_L\\
\end{array}
\end{equation}
% si on ne fait pas comme ca, il n'apparait pas
\footnotetext[\thefootnote]{Si $\var{IRIJNU} = 1$, on remplace  $\mu^n_{\,lm}$ par $(\mu +
\mu_t)^n_{\,lm}$ dans l'expression de la diffusion non reconstruite
associ\'ee \`a la matrice simplifi\'ee de \fort{matrix} ($\mu_t$ d\'esigne la
viscosit\'e turbulente calcul\'ee comme en $k-\varepsilon$).}

o\`u on rappelle :\\
pour $n$ donn\'e entier positif, on d\'efinit la suite
 $({R}_{ij}^{\,n+1,p})_{p \in \grandN}$
 par :
\begin{equation}\notag
\left\{\begin{array}{l}
{R}_{ij}^{\,n+1,0} = {R}_{ij}^{\,n}\\
{R}_{ij}^{\,n+1,p+1} = {R}_{ij}^{\,n+1,p} + \delta{R}_{ij}^{\,n+1,p+1} \\
\end{array}\right.
\end{equation}
$(\delta{R}_{ij}^{\,n+1,p+1})_{\,L}$ d\'esigne la valeur sur l'\'el\'ement
$\Omega_l$ du $\text{$(\,p+1\,)$-i\`eme}$ incr\'ement de ${R}_{ij}^{\,n+1}$,
$ m^n_{\,lm}$ le flux de masse \`a l'instant $n$ \`a travers la face $lm$,
$\delta R_{ij,\,f_{\,lm}}^{\,n+1,p+1}$ vaut $({\delta
R}_{ij}^{\,n+1,p+1})_{L}$  si $ m^n_{\,lm} \geqslant 0$, $({\delta
R}_{ij}^{\,n+1,p+1})_{M}$ sinon,
$\mathcal{P}^{\,n+1,p}_{ij}$, $\phi^{\,n+1,p}_{ij,2}$, $\phi^{\,n+1,p}_{ij,w}$ les valeurs
des quantit\'es associ\'ees correspondant \`a l'incr\'ement
$(\delta{R}_{ij}^{\,n+1,p})$.\\



Tous ces termes sont calcul\'es comme suit :
\begin{itemize}
\item Terme de gauche de l'\'equation (\ref{Base_Turrij_Eq_Temp_deltaRij})\\
Dans \fort{resrij} est calcul\'ee la variable \var{ROVSDT}. Les autres
termes sont compl\'et\'es par \fort{codits}, lors de la construction de la matrice simplifi\'ee , {\it via} un
appel au sous-programme \fort{matrix}. La quantit\'e
 $(\mu^n_{\,lm} + \gamma^n_{\,lm})$ \`a la face $lm$ est calcul\'ee lors de l'appel \`a
\fort{visort}.\\
\item Second membre de l'\'equation (\ref{Base_Turrij_Eq_Temp_deltaRij})\\
Le premier terme non d\'etaill\'e est calcul\'e par le sous-programme
\fort{bilsc2}, qui applique le sch\'ema convectif choisi par l'utilisateur, qui
reconstruit ou non selon le souhait de l'utilisateur les gradients intervenants
dans la convection-diffusion.\\
Les termes sans accolade sont, eux, compl\`etement explicites et ajout\'es au fur et
\`a mesure dans \var{SMBR} pour former
l'expression $f^{\,exp}_s$ de \fort{codits}.
\end{itemize}
On d\'ecrit ci-dessous les \'etapes de \fort{resrij} :
\begin{itemize}

\item DELTIJ = 1, pour $\var{ISOU} \leqslant 3$ et DELTIJ = 0  Si $\var{ISOU} >
3$. Cette valeur repr\'esente le symbole de Kroeneker $\delta_{ij}$.

\item Initialisation \`a z\'ero de \var{SMBR} (tableau contenant le second
membre) et \var{ROVSDT} (tableau contenant la diagonale de la matrice sauf celle
relative \`a la contribution de la
diagonale des op\'erateurs de convection et de diffusion lin\'earis\'es
\footnote{qui correspondent aux sch\'emas convectif upwind pur et diffusif sans
reconstruction.}), tous deux de dimension $\var{NCEL}$.

\item Lecture et prise en compte des termes sources utilisateur pour la variable $R_{ij}$

Appel \`a \fort{ustsri} pour charger les termes sources utilisateurs. Ils sont
stock\'es comme suit. Pour la cellule $\Omega_l$ de centre $L$, repr\'esent\'ee par $\var{IEL}$, on a :\\
\begin{equation}\notag
\left\{\begin{array}{lll}
&\var{ROVSDT(IEL)}&= |\Omega_l| \ \alpha_{R_{ij}}\\
&\var{SMBR(IEL)}&=|\Omega_l| \ \beta_{R_{ij}}\\
\end{array}\right.
\end{equation}
On affecte alors les valeurs ad\'equates au second membre \var{SMBR} et \`a la
diagonale \var{ROVSDT} comme suit :
\begin{equation}\notag
\left\{\begin{array}{lll}
&\var{SMBR(IEL)} &= \var{SMBR(IEL)} +\ |\Omega_l| \ \alpha_{R_{ij}} \ (R^n_{ij})_L \\
&\var{ROVSDT(IEL)}&= \text{max }(-\ |\Omega_l| \ \alpha_{R_{ij}},0)\\
\end{array}\right.
\end{equation}
La valeur de $ \var{ROVSDT}$ est ainsi calcul\'ee pour des raisons de stabilit\'e
num\'erique. En effet, on ne rajoute sur la diagonale que les valeurs positives,
ce qui correspond physiquement \`a impliciter les termes de rappel uniquement.
\item{Calcul du terme source de masse  si $\Gamma_L > 0$}

Appel de \fort{catsma} et incr\'ementation si n\'ecessaire de \var{SMBR} et
\var{ROVSDT} {\it via} :\\
\begin{equation}\notag
\left\{\begin{array}{lll}
\displaystyle \var{SMBR(IEL)} = \var{SMBR(IEL)} + |\Omega_l| \ \Gamma_L \
\left[(R^{\,in}_{ij})_L - (R^{\,n}_{ij})_L \right] \\
\displaystyle \var{ROVSDT(IEL)}=\var{ROVSDT(IEL)} + |\Omega_l| \ \Gamma_L
\end{array}\right.
\end{equation}
\item Calcul du terme d'accumulation de masse et du terme instationnaire

On stocke $\displaystyle \var{W1}= \int_{\Omega_l}\dive\,(\rho\,\vect{u})\,d\Omega$
calcul\'e par \fort{divmas} \`a l'aide des flux de masse aux faces internes
$ m^n_{\,lm}=\sum\limits_{m\in Vois(l)}{(\rho \vect{u})_{\,lm}^n} \text{.}\,
\vect{S}_{\,lm} $ (tableau \var{FLUMAS}) et des flux de masse aux bords  $ m^n_{\,b_{lk}} = \sum\limits_{k\in{\gamma_b(l)}}{(\rho \vect{u})_{\,{b}_{lk}}^n} \text{.}\,
\vect{S}_{\,{b}_{lk}} $ (tableau \var{FLUMAB}).
On incr\'emente ensuite \var{SMBR} et \var{ROVSDT}.
\begin{equation}\notag
\left\{\begin{array}{lll}
&\var{SMBR(IEL)} &= \var{SMBR(IEL)} + \var{ICONV}\  (R^n_{ij})_L\,(\displaystyle
\int_{\Omega_l}\dive\,(\rho\,\vect{u})\ d\Omega) \\
&\var{ROVSDT(IEL)}& = \var{ROVSDT(IEL)} +  \var{ISTAT}\,\displaystyle
\frac{\rho^n_L \ |\Omega_l|}{\Delta t^n} -  \var{ICONV}\ (\displaystyle
\int_{\Omega_l}\dive\,(\rho\,\vect{u})\ d\Omega) \\
\end{array}\right.
\end{equation}
\item Calcul des termes sources de production, des termes $\displaystyle
\phi_{\,ij,1}+\phi_{\,ij,2}$ et de la dissipation~$\displaystyle-\frac{2}{3} \varepsilon\,\delta_{\,ij}$ :

On effectue une boucle d'indice \var{IEL} sur les cellules $\Omega_l$ de centre $L$ :
\begin{itemize}
\item [$\Rightarrow$] $\displaystyle \var{TRPROD}= \frac{1}{2} (\mathcal{P}^n_{ii})_L = \frac{1}{2} \left[ \var{PRODUC(1,IEL)} +  \var{PRODUC(2,IEL)} +  \var{PRODUC(3,IEL)} \right] $
\item [$\Rightarrow$] $\displaystyle \var{TRRIJ }= \frac{1}{2} (R^n_{ii})_L $
\item [$\Rightarrow$] $\displaystyle \var{SMBR(IEL)} =\ \var{SMBR(IEL)}\ +$\\
$\ \displaystyle\rho^n_L |\Omega_l| \left[ \displaystyle
\frac{2}{3}\,\delta_{\,ij} \left( \ \displaystyle \frac{ C_2}{2}\,(\mathcal{P}^n_{ii})_L\ +
(C_1-1)\ \varepsilon^n_L\, \right)\right.$\\
$ + \left.\ (1-C_2) \ \var{PRODUC(ISOU,IEL)} -
\displaystyle C_1\ \frac{2\,\varepsilon^n_L}{(R^n_{ii})_L}\ (R^n_{ij})_L \right]$
\item [$\Rightarrow$] $\displaystyle \var{ROVSDT(IEL)} = \var{ROVSDT(IEL)} +
\rho^n_L \ |\Omega_l| \ (- \displaystyle \frac{1}{3} \ \,\delta_{\,ij} + 1) \ C_1
\ \frac{2\ \varepsilon^n_L}{(R^n_{ii})_L}$
\end{itemize}
\item Appel de \fort{rijech} pour le calcul des termes d'\'echo de paroi
 $\phi^n_{ij,w}$ si $\var{IRIJEC()}=1$ et ajout dans \var{SMBR}.\\
$\var{SMBR} = \var{SMBR} + \phi^n_{ij,w}$\\
Suivant son mode de calcul (\var{ICDPAR}), la distance � la paroi est directement accessible
par \var{RA(IDIPAR+IEL-1)} (\var{|ICDPAR|} = 1) ou bien
est calcul\'ee \`a partir de $\var{IA(IIFAPA(IPHAS)+IEL - 1)}$,
qui donne pour l'\'el\'ement $\var{IEL}$ le num\'ero de la face de bord
paroi la plus  proche (\var{|ICDPAR|} = 2). Ces tableaux ont \'et\'e renseign\'e une fois pour toutes au
d\'ebut de calcul.

\item  Appel de \fort{rijthe} pour le calcul des termes de gravit\'e $\mathcal{G}^n_{ij}$ et ajout dans \var{SMBR}.

Ce calcul n'a lieu que si $\var{IGRARI()} = 1$.
$ \var{SMBR} = \var{SMBR} + \mathcal{G}^n_{ij}$
\item Calcul de la partie explicite du terme de diffusion
 $\dive{\,\left[\tens{A}\,\grad{R}^{\,n}_{ij}\right]}$, plus pr\'ecis\'ement
des contributions du terme extradiagonal pris aux faces purement internes
(remplissage du tableau \var{VISCF}), puis aux faces de bord (remplissage du
tableau \var{VISCB}).
\begin{itemize}
\item [$\star$] Appel de \fort{grdcel} pour le calcul du gradient de
$R^{\,n}_{ij}$ dans chaque direction. Ces gradients sont respectivement
stock\'es dans les tableaux de travail \var{W1}, \var{W2} et \var{W3}.

\item [$\star$] boucle d'indice \var{IEL} sur les cellules $\Omega_l$ de centre
$L$ pour le
calcul de $\tens{E}^n\,\grad{R}^{\,n}_{ij}$ aux cellules dans un premier temps :\\
\begin{itemize}
\item [$\Rightarrow$] $\displaystyle \var{TRRIJ}= \frac{1}{2} (R^{\,n}_{ii})_L $
\item [$\Rightarrow$] $\displaystyle \var{CSTRIJ} = \rho^n_L\ C_S \ \displaystyle\frac{(R^n_{ii})_L}{2\,\varepsilon^n_L}$
\item [$\Rightarrow$] $\displaystyle \var{W4(IEL)} = \rho^n_L\ C_S\
\displaystyle\frac{(R^n_{ii})_L}{2\,\varepsilon^n_L} \left[\,(R^{\,n}_{12})_L \ \var{W2(IEL)} +
(R^{\,n}_{13})_L \ \var{W3(IEL)}\,\right]$
\item [$\Rightarrow$] $\displaystyle \var{W5(IEL)} = \rho^n_L\ C_S\
\displaystyle\frac{(R^n_{ii})_L}{2\,\varepsilon^n_L} \left[\,(R^{\,n}_{12})_L \ \var{W1(IEL)} +
(R^{\,n}_{23})_L \ \var{W3(IEL)}\,\right]$
\item [$\Rightarrow$] $\displaystyle \var{W6(IEL)} = \rho^n_L\ C_S\
\displaystyle\frac{(R^n_{ii})_L}{2\,\varepsilon^n_L} \left[\,(R^{\,n}_{13})_L \ \var{W1(IEL)} + (R^{\,n}_{23})_L \ \var{W2(IEL)}\,\right]$
\end{itemize}



\item [$\star$] Appel de \fort{vectds}\footnote{Le r\'esultat est stock\'e dans
\var{VISCF} et \var{VISCB}. Dans \fort{vectds}, les valeurs aux faces internes
sont interpol\'ees lin\'eairement sans reconstruction et \var{VISCB} est mis \`a
z\'ero.} pour assembler $\displaystyle\left[ \tens{E}^n\,\grad{R}^{\,n}_{ij}
\right]\,.\,\vect{n}_{\,lm}S_{\,lm}$ aux faces $lm$.
\item [$\star$] Appel de \fort{divmas} pour calculer la divergence du flux d\'efini par \var{VISCF} et \var{VISCB}.
Le r\'esultat est stock\'e dans \var{W4}.\\
Ajout au second membre \var{SMBR}.\\
\var{SMBR} = \var{SMBR} + \var{W4}
\end{itemize}

A l'issue de cette \'etape, seule la partie extradiagonale de la diffusion prise
enti\`erement explicite~:
 $$\sum\limits_{m\in
Vois(l)}\left[\ \tens{E}^n\,\grad{R}^{\,n}_{ij} \right]_{\,lm}\,.\,\vect{n}_{\,lm}S_{\,lm}$$ a \'et\'e calcul\'ee.\\

\item Calcul de la partie diagonale du terme de diffusion\footnote{Seule la
composante normale  du  gradient de $R_{ij}$ aux faces sera implicite.} :\\
Comme on l'a d\'eja vu, une partie de cette contribution sera trait\'ee en
implicite (celle relative \`a la composante normale du gradient) et donc
ajout\'ee au second membre par \fort{bilsc2} ; l'autre
partie sera explicite et prise en compte dans $f_s^{\,exp}$.
\begin{itemize}
\item [$\star$] On effectue une boucle d'indice \var{IEL} sur les cellules
$\Omega_l$ de centre $L$ :
\begin{itemize}
\item [$\Rightarrow$] $\displaystyle \var{TRRIJ }= \frac{1}{2} (R^{\,n}_{ii})_L $
\item [$\Rightarrow$] $\displaystyle \var{CSTRIJ} = \rho^n_L \ C_S \ \frac{(R^{\,n}_{ii})_L}{2\,\varepsilon^n_L}$
\item [$\Rightarrow$] $\displaystyle \var{W4(IEL)} = \rho^n_L \ C_S \
\frac{(R^{\,n}_{ii})_L}{2\,\varepsilon^n_L} \ (R^{\,n}_{11})_L$
\item [$\Rightarrow$] $\displaystyle \var{W5(IEL)} = \rho^n_L \ C_S \ \frac{(R^{\,n}_{ii})_L}{2\,\varepsilon^n_L}\ (R^n_{22})_L$
\item [$\Rightarrow$] $\displaystyle \var{W6(IEL)} = \rho^n_L \ C_S \ \frac{(R^{\,n}_{ii})_L}{2\,\varepsilon^n_L} \ (R^n_{33})_L$
\end{itemize}

%\item Traitement du parall\'elisme et de la p\'eriodicit\'e.

\item [$\star$] On effectue une boucle d'indice \var{IFAC} sur les faces
purement internes $lm$ pour remplir le tableau \var{VISCF} :
\begin{itemize}
\item [$\Rightarrow$] Identification des cellules $\Omega_l$ et $\Omega_m$ de
centre respectif $L$ (variable \var{II}) et $M$ (variable \var{JJ}), se trouvant de chaque c\^ot\'e de la face
$lm$\footnote{La normale \`a la face est orient\'ee de L vers M.}.
\item [$\Rightarrow$] Calcul du carr\'e de la surface de la face. La valeur est
stock\'ee dans le tableau \var{SURFN2}.
\item [$\Rightarrow$] Interpolation du gradient de $R^{\,n}_{ij}$ \`a la face
$lm$ (gradient facette $\left[\grad{R}^{\,n}_{ij}\right]_{\,lm}$) :
\begin{equation}\notag
\left\{\begin{array}{ll}
\var{GRDPX} &= \displaystyle \frac{1}{2} \left(\var{W1(II)} + \var{W1(JJ)}
\right) \\
&\\
\var{GRDPY} &= \displaystyle \frac{1}{2} \left(\var{W2(II)} + \var{W2(JJ)} \right) \\
&\\
\var{GRDPZ} &= \displaystyle \frac{1}{2} \left(\var{W3(II)} + \var{W3(JJ)} \right)
\end{array}\right.
\end{equation}
\item [$\Rightarrow$] Calcul du gradient de $R^{\,n}_{ij}$ normal \`a la face
$lm$, $\left[\grad{R}^{\,n}_{ij}\right]_{\,lm}.\vect{n}_{\,lm}\,S_{\,lm}$.\\

$\displaystyle \var{GRDSN} =  \var{GRDPX} \ \var{SURFAC(1,IFAC)} + \var{GRDPY} \ \var{SURFAC(2,IFAC)} +  \var{GRDPZ} \ \var{SURFAC(3,IFAC)}$
$\var{SURFAC}$ est le vecteur surface de la face \var{IFAC}.


\item [$\Rightarrow$] calcul de
 $\left[\grad{R^{\,n}_{ij}} - (\grad
R^{\,n}_{ij}\,.\,\vect{n}_{\,lm})\vect{n}_{\,lm}\right]$, les vecteurs \'etant
calcul\'es \`a la face $lm$ :
\begin{equation}\notag
\left\{\begin{array}{lll}
&\displaystyle \var{GRDPX} &= \var{GRDPX} - \displaystyle\frac{\var{GRDSN}}{\var{SURFN2}} \ \var{SURFAC(1,IFAC)}\\
&&\\
&\displaystyle \var{GRDPY} &= \var{GRDPY} - \displaystyle\frac{\var{GRDSN}}{\var{SURFN2}} \ \var{SURFAC(2,IFAC)} \\
&&\\
&\displaystyle \var{GRDPZ} &= \var{GRDPZ} - \displaystyle\frac{\var{GRDSN}}{\var{SURFN2}} \ \var{SURFAC(3,IFAC)}
\end{array}\right.
\end{equation}
\item [$\Rightarrow$] finalisation du calcul de l'expression totalement
explicite
 $$\left[ \tens{D}^n\,\left( \grad{R^{\,n}_{ij}} - (\grad R^{\,n}_{ij}\,.\,\vect{n}_{\,lm})\,\vect{n}_{\,lm}\right) \right]\,.\,\vect{n}_{\,lm}$$
\begin{equation}\notag
\begin{array} {ll}
\displaystyle \var{VISCF} = &
 \displaystyle\frac{1}{2} (\ \var{W4(II)} +\ \var{W4(JJ)}) \ \var{GRDPX} \
\var{SURFAC(1,IFAC)})\ + \\
&\\
&  \displaystyle\frac{1}{2} (\ \var{W5(II)} +\ \var{W5(JJ)}) \ \var{GRDPY} \
\var{SURFAC(2,IFAC)})\ + \\
&\\
&  \displaystyle\frac{1}{2} (\ \var{W6(II)} +\ \var{W6(JJ)}) \ \var{GRDPZ} \ \var{SURFAC(3,IFAC)})
\end{array}
\end{equation}
\end{itemize}

\item [$\star$] Mise \`a z\'ero du tableau \var{VISCB}.

\item [$\star$] Appel de \fort{divmas} pour calculer la divergence de~:
 $$\tens{D}^{\,n}\,\left( \grad{R^{\,n}_{ij}} - (\grad R^{\,n}_{ij}\,.\,\vect{n}_{\,lm})\vect{n}_{\,lm}\right)$$ d\'efini aux faces dans \var{VISCF} et \var{VISCB}.

Le r\'esultat est stock\'e dans le tableau \var{W1}.\\
Ajout au second membre \var{SMBR}.\\
$\var{SMBR} = \var{SMBR} + \var{W1}$
\end{itemize}
\item Calcul de la viscosit\'e orthotrope $\gamma^n_{\,lm}$ \`a la face $lm$ de la variable principale
$R^{\,n}_{ij}$\\
Ce calcul permet au sous-programme \fort{codits} de compl\'eter le second membre
\var{SMBR} par :
\begin{equation}
\begin{array} {ll}
& \sum\limits_{m\in Vois(l)}
\mu^n_{\,lm}\,\left(\grad{R}^{\,n}_{ij}\,.\,\vect{n}_{\,lm}\right)S_{\,lm}
 + \sum\limits_{m\in Vois(l)} \left(\grad{R}^{\,n}_{ij}
\,.\,\vect{n}_{\,lm}\right)\left[\tens{D}^{\,n}\,\vect{n}_{\,lm}\right]_{\,lm}\,.\,\vect{n}_{\,lm}\
S_{\,lm}\\
& = \sum\limits_{m\in Vois(l)}(\,\mu^n_{\,lm}\, + \,\gamma^n_{\,lm}\,)\,\left(\grad{R}^{\,n}_{ij}\,.\,\vect{n}_{\,lm}\right)S_{\,lm}
\end{array}
\end{equation}
sans pr\'eciser la nature de la face $lm$, {\it via} l'appel \`a \fort{bilsc2}  et de disposer de la quantit\'e
$(\mu^n_{\,lm}\, + \gamma^n_{\,lm})$ pour construire sa
matrice simplifi\'ee.\\
\begin{itemize}
\item [$\star$] On effectue une boucle d'indice \var{IEL} sur les cellules
$\Omega_l$ :
\begin{itemize}
\item [$\Rightarrow$] $\displaystyle \var{TRRIJ }= \frac{1}{2} (R^{\,n}_{ii})_L $
\item [$\Rightarrow$] $\displaystyle \var{RCSTE} = \rho^n_L \ C_S \ \frac{ (R^{\,n}_{ii})_L}{2\,\varepsilon^n_L} $
\item [$\Rightarrow$] $\displaystyle \var{W1(IEL)} = \mu^n +\rho^n_L \ C_S \ \frac{
(R^{\,n}_{ii})_L}{2\,\varepsilon^n_L}\ (R^n_{11})_L$
\item [$\Rightarrow$] $\displaystyle \var{W2(IEL)} = \mu^n + \rho^n_L \ C_S \ \frac{ (R^{\,n}_{ii})_L}{2\,\varepsilon^n_L}\ (R^n_{22})_L$
\item [$\Rightarrow$] $\displaystyle \var{W3(IEL)} = \mu^n + \rho^n_L \ C_S \ \frac{ (R^{\,n}_{ii})_L}{2\,\varepsilon^n_L}\ (R^n_{33})_L$
\end{itemize}

\item [$\star$] Appel de \fort{visort} pour calculer la viscosit\'e orthotrope
\footnote{Comme dans le sous-programme \fort{viscfa}, on multiplie la viscosit\'e par
$\displaystyle \frac{S_{\,lm}}{\overline{L'M'}}$, o\`u $S_{\,lm}$ et
$\overline{L'M'}$ repr\'esentent respectivement la surface de la face $lm$ et la
mesure alg\'ebrique du segment reliant les projections des centres des cellules
voisines sur la normale \`a la face. On garde dans ce sous-programme  la possibilit\'e d'interpoler la viscosit\'e aux cellules lin\'eairement ou d'utiliser une moyenne harmonique. La viscosit\'e au bord est celle de la cellule de bord correspondante.}
$ \gamma^n_{\,lm} = (\tens{D}^{\,n}\,\vect{n}_{\,lm}).\vect{n}_{\,lm}$ aux faces $lm$

Le r\'esultat est stock\'e dans les tableaux \var{VISCF} et \var{VISCB}.
\end{itemize}

\item appel de \fort{codits} pour la r\'esolution de l'\'equation de
convection/diffusion/termes sources de la variable $R_{ij}$. Le terme source est
\var{SMBR}, la viscosit\'e \var{VISCF} aux faces purement internes (
resp. \var{VISCB} aux faces de bord) et \var{FLUMAS} le flux de masse interne
 ( resp. \var{FLUMAB} flux de masse au bord). Le r\'esultat est la variable $R_{ij}$ au temps
$n+1$, donc $R^{\,n+1}_{ij}$. Elle est stock\'ee dans le tableau \var{RTP} des
variables mises \`a jour.

\end{itemize}

\etape{Appel de \fort{reseps} pour la r\'esolution de la variable $\varepsilon$}

On d\'ecrit ci-dessous le sous-programme \fort{reseps}, les commentaires du sous-programme \fort{resrij} ne seront pas r\'ep\'et\'es puisque les deux sous-programmes ne diff\`erent que par leurs termes sources.

\begin{itemize}
\item Initialisation \`a z\'ero de \var{SMBR} et \var{ROVSDT}.

\item{Lecture et prise en compte des termes sources utilisateur pour la variable $\varepsilon$ :}

Appel de \fort{ustsri} pour charger les termes sources utilisateurs. Ils sont
stock\'es dans les tableaux suivants :\\
pour la cellule $\Omega_l$ repr\'esent\'ee par $\var{IEL}$ de centre $L$, on a :
\begin{equation}\notag
\left\{\begin{array}{lll}
&\var{ROVSDT(IEL)}&= |\Omega_l| \ \alpha_{\varepsilon}\\
&\var{SMBR(IEL)}&=|\Omega_l| \ \beta_{\varepsilon}\\
\end{array}\right.
\end{equation}
On affecte alors les valeurs ad\'equates au second membre \var{SMBR} et \`a la
diagonale \var{ROVSDT} comme suit :
\begin{equation}\notag
\left\{\begin{array}{lll}
&\var{SMBR(IEL)} &= \var{SMBR(IEL)} +\ |\Omega_l| \ \alpha_{\,\varepsilon} \
\varepsilon^n_L \\
&\var{ROVSDT(IEL)}&= \text{max }(-\ |\Omega_l| \ \alpha_{\,\varepsilon},0)\\
\end{array}\right.
\end{equation}

\item{Calcul du terme source de masse si $\Gamma_L > 0$ :
\begin{equation}\notag
\left\{\begin{array}{lll}
&\displaystyle \var{SMBR(IEL)} = \var{SMBR(IEL)} + |\Omega_l| \ \Gamma_L \
(\varepsilon^{\,in}_L -\varepsilon^n_L) \\
&\displaystyle \var{ROVSDT(IEL)}= \var{ROVSDT(IEL)} + |\Omega_l| \ \Gamma_L
\end{array}\right.
\end{equation}
\item Calcul du terme d'accumulation de masse et du terme instationnaire \\
On stocke $\displaystyle \var{W1}= \int_{\Omega_l}\dive\,(\rho\,\vect{u})\,d\Omega$
calcul\'e par \fort{divmas} \`a l'aide des flux de masse internes et aux bords.\\
On incr\'emente ensuite \var{SMBR} et \var{ROVSDT}.
\begin{equation}\notag
\left\{\begin{array}{lll}
&\var{SMBR(IEL)} &= \var{SMBR(IEL)} + \var{ICONV}\ \varepsilon^n_L\,(\displaystyle
\int_{\Omega_l}\dive\,(\rho\,\vect{u})\ d\Omega) \\
&\var{ROVSDT(IEL)}& = \var{ROVSDT(IEL)} +  \var{ISTAT}\,\displaystyle
\frac{\rho^n_L \ |\Omega_l|}{\Delta t^n} -  \var{ICONV}\ (\displaystyle
\int_{\Omega_l}\dive\,(\rho\,\vect{u})\ d\Omega) \\
\end{array}\right.
\end{equation}

\item Traitement du terme de production
 $\displaystyle \rho\,C_{\varepsilon_1}\,\frac{\varepsilon}{k}\,\mathcal{P}$
 et du terme de dissipation $-\,\displaystyle \rho\,C_{\varepsilon_2}\,\frac{\varepsilon}{k}\,\varepsilon$ \\
pour cela, on effectue une boucle d'indice \var{IEL} sur les cellules $\Omega_l$
de centre $L$ :
\begin{itemize}
\item [$\Rightarrow$] $\displaystyle \var{TRPROD}= \frac{1}{2} (\mathcal{P}^n_{ii})_L = \frac{1}{2} \left[ \var{PRODUC(1,IEL)} +  \var{PRODUC(2,IEL)} +  \var{PRODUC(3,IEL)} \right] $
\item [$\Rightarrow$] $\displaystyle \var{TRRIJ }= \frac{1}{2} (R^n_{ii})_L $
\item [$\Rightarrow$] $\displaystyle \var{SMBR(IEL)} = \var{SMBR(IEL)} + \rho^n_L
|\Omega_l| \left[ -C_{\varepsilon_2} \ \frac{2\,(\varepsilon^n_L)^2}{(R^n_{ii})_L} + C_{\varepsilon_1} \ \frac{\varepsilon^n_L}{(R^n_{ii})_L}\ (\mathcal{P}^n_{ii})_L \right] $
\item [$\Rightarrow$] $\displaystyle \var{ROVSDT(IEL)} = \var{ROVSDT(IEL)} + C_{\varepsilon_2} \ \rho^n_L \ |\Omega_l| \ \frac{2\,\varepsilon^n_L}{(R^n_{ii})_L}$
\end{itemize}

\item Appel de \fort{rijthe} pour le calcul des termes de gravit\'e $\mathcal{G}^n_{\varepsilon}$ et ajout dans \var{SMBR}.

$ \var{SMBR} = \var{SMBR} + \mathcal{G}^n_{\varepsilon}$\\
Ce calcul n'a lieu que si $\var{IGRARI()} = 1$.

\item Calcul de la diffusion de $\varepsilon$ \\
 Le terme $\dive \left[\mu\, \grad(\varepsilon) + \tens{A'}\,\grad(\varepsilon)
\right]$ est calcul\'e exactement de la m\^eme mani\`ere que pour les tensions
de Reynolds $R_{ij}$ en rempla\c cant $\tens{A}$ par $\tens{A'}$.

\item Appel de \fort{codits} pour la r\'esolution de l'\'equation de
convection/diffusion/termes sources de la variable principale $\varepsilon$. Le
r\'esultat $\varepsilon^{\,n+1}$ est stock\'e dans le tableau \var{RTP} des
variables mises \`a jour.
}
\end{itemize}

\etape{clippings finaux}
On passe enfin dans le sous-programme  \fort{clprij} pour faire un clipping \'eventuel
des variables $R^{\,n+1}_{ij}$ et $\varepsilon^{\,n+1}$. Le sous-programme
\fort{clprij} est appel\'e\footnote{L'option
$\var{ICLIP} = 1$ consiste en un clipping minimal des variables $R_{ii}$ et
$\varepsilon$ en prenant la valeur absolue de ces variables puisqu'elles ne
peuvent \^etre que positives.} avec $\var{ICLIP} = 2$ . Cette option
\footnote{Quand la valeur des grandeurs $R_{ii}$ ou $\varepsilon$ est
n\'egative, on la remplace par le minimum entre sa valeur absolue et (1,1)
fois la valeur obtenue au pas de temps pr\'ec\'edent.} contient l'option $\var{ICLIP} = 1$  et permet de v\'erifier l'in\'egalit\'e de Cauchy-Schwarz sur les grandeurs extra-diagonales du tenseur $\tens{R}$ (pour $i \neq j$, $|R_{ij}|^2 \le R_{ii} R_{jj}$).


%%%%%%%%%%%%%%%%%%%%%%%%%%%%%%%%%%
%%%%%%%%%%%%%%%%%%%%%%%%%%%%%%%%%%
\section{Points \`a traiter}
%%%%%%%%%%%%%%%%%%%%%%%%%%%%%%%%%%
%%%%%%%%%%%%%%%%%%%%%%%%%%%%%%%%%%
Sauf mention explicite, $\phi$ repr\'esentera une tension de Reynolds ou la dissipation turbulente ($\phi = R_{ij} \ \text{ou} \ \varepsilon$).

\begin{itemize}
\item {La vitesse utilis\'ee pour le calcul de la production est explicite. Est-ce qu'une implicitation peut am\'eliorer la pr\'ecision temporelle de $\phi$ \footnote{Cette remarque peut \^etre g\'en\'eralis\'ee. En effet, peut-on envisager d'actualiser les variables d\'ej\`a r\'esolues (sans r\'eactualiser les variables turbulentes apr\`es leur r\'esolution)? Ceci obligerait \`a modifier les sous-programmes tels que \fort{condli} qui sont appel\'es au d\'ebut de la boucle en temps.} ?}
\item {Dans quelle mesure le terme d'\'echo de paroi est-il valide ? En effet, ce terme est remis en question par certains auteurs.}
\item {On peut envisager la r\'esolution d'un syst\`eme hyperbolique pour les
tensions de Reynolds afin d'introduire un couplage avec le champ de vitesse.}
\item {Le flux au bord \var{VISCB} est annul\'e dans le sous-programme
\fort{vectds}. Peut-on envisager de mettre au bord la valeur de la variable
concern\'ee \`a la cellule de bord correspondant? De m\^eme, il faudrait se
pencher sur les hypoth\`eses sous-jacentes \`a l'annulation des contributions
aux bords de \var{VISCB} lors du calcul de : $$\left[ \tens{D}^n\,\left( \grad{R^{\,n}_{ij}} - (\grad R^{\,n}_{ij}\,.\,\vect{n}_{\,lm})\,\vect{n}_{\,lm}\right) \right]\,.\,\vect{n}_{\,lm}.$$}
\item {Un probl\`eme de pond\'eration appara\^\i t plus g\'en\'eralement. Si on prend la partie explicite de $\tens{D}\,\grad(\phi)$, la pond\'eration aux faces internes utilise le coefficient $\displaystyle\frac{1}{2}$ avec pond\'eration s\'epar\'ee de $\tens{D}$ et $\grad(\phi)$, alors que pour $\tens{E}\,\grad(\phi)$, on calcule d'abord ce terme aux cellules pour ensuite l'interpoler lin\'eairement aux faces \footnote{Cette interpolation se fait dans \fort{vectds} avec des coefficients de pond\'eration aux faces.}. Ceci donne donc deux types d'interpolations pour des termes de m\^eme nature.}
\item {On laisse la possibilit\'e dans \fort{visort} d'utiliser une moyenne
harmonique aux faces. Est-ce que ceci est valable puisque les interpolations
utilis\'ees lors du calcul de la partie explicite de $\tens{A}\,\grad{\phi}$
sont des moyennes arithm\'etiques ?}
\item {Les techniques adopt\'ees lors du clipping sont \`a revoir.}
\item {On utilise dans le cadre du mod\`ele $\displaystyle R_{ij}-\varepsilon $ une semi-implicitation de termes comme $\displaystyle \phi_{ij,1}$ ou $\displaystyle -\rho\,C_{\varepsilon_2}\,\frac{\varepsilon}{k}\,\varepsilon$. On peut envisager le m\^eme type d'implicitation dans \fort{turbke} m\^eme en pr\'esence du couplage $\displaystyle k-\varepsilon$.}
\item L'adoption d'une r\'esolution d\'ecoupl\'ee fait perdre l'invariance par rotation.
\item La formulation et l'implantation des conditions aux limites de paroi
devront \^etre v\'erifi\'ees. En effet, il semblerait que, dans certains cas, des ph\'enom\`enes
``oscillatoires'' apparaissent, sans qu'il soit ais\'e d'en d\'eterminer la cause.
\item L'implicitation partielle (du fait de la r\'esolution d\'ecoupl\'ee) des
conditions aux limites conduit souvent \`a des calculs instables. Il
conviendrait d'en conna\^\i tre la raison. L'implicitation partielle avait
\'et\'e mise en \oe uvre afin de tenter d'utiliser un pas de temps plus grand
dans le cas de jets axisym\'etriques en particulier.

\end{itemize}

\part{Module compressible}
%                      Code_Saturne version 1.3
%                      ------------------------
%
%     This file is part of the Code_Saturne Kernel, element of the
%     Code_Saturne CFD tool.
%
%     Copyright (C) 1998-2007 EDF S.A., France
%
%     contact: saturne-support@edf.fr
%
%     The Code_Saturne Kernel is free software; you can redistribute it
%     and/or modify it under the terms of the GNU General Public License
%     as published by the Free Software Foundation; either version 2 of
%     the License, or (at your option) any later version.
%
%     The Code_Saturne Kernel is distributed in the hope that it will be
%     useful, but WITHOUT ANY WARRANTY; without even the implied warranty
%     of MERCHANTABILITY or FITNESS FOR A PARTICULAR PURPOSE.  See the
%     GNU General Public License for more details.
%
%     You should have received a copy of the GNU General Public License
%     along with the Code_Saturne Kernel; if not, write to the
%     Free Software Foundation, Inc.,
%     51 Franklin St, Fifth Floor,
%     Boston, MA  02110-1301  USA
%
%-----------------------------------------------------------------------
%
\programme{vortex}
%
\vspace{1cm}
%%%%%%%%%%%%%%%%%%%%%%%%%%%%%%%%%%
%%%%%%%%%%%%%%%%%%%%%%%%%%%%%%%%%%
\section{Fonction}
%%%%%%%%%%%%%%%%%%%%%%%%%%%%%%%%%%
%%%%%%%%%%%%%%%%%%%%%%%%%%%%%%%%%%
Ce sous-programme est d�di� � la g�n�ration des conditions d'entr�e
turbulente utilis�es en LES.


La m�thode des vortex est bas�e sur une approche de tourbillons
ponctuels. L'id�e de la m�thode consiste � injecter des tourbillons 2D dans le
plan d'entr�e du calcul, puis � calculer le champ de vitesse induit par ces
tourbillons au centre des faces d'entr�e.

%                      Code_Saturne version 1.3
%                      ------------------------
%
%     This file is part of the Code_Saturne Kernel, element of the
%     Code_Saturne CFD tool.
% 
%     Copyright (C) 1998-2007 EDF S.A., France
%
%     contact: saturne-support@edf.fr
% 
%     The Code_Saturne Kernel is free software; you can redistribute it
%     and/or modify it under the terms of the GNU General Public License
%     as published by the Free Software Foundation; either version 2 of
%     the License, or (at your option) any later version.
% 
%     The Code_Saturne Kernel is distributed in the hope that it will be
%     useful, but WITHOUT ANY WARRANTY; without even the implied warranty
%     of MERCHANTABILITY or FITNESS FOR A PARTICULAR PURPOSE.  See the
%     GNU General Public License for more details.
% 
%     You should have received a copy of the GNU General Public License
%     along with the Code_Saturne Kernel; if not, write to the
%     Free Software Foundation, Inc.,
%     51 Franklin St, Fifth Floor,
%     Boston, MA  02110-1301  USA
%
%-----------------------------------------------------------------------
%
%%%%%%%%%%%%%%%%%%%%%%%%%%%%%%%%%%
%%%%%%%%%%%%%%%%%%%%%%%%%%%%%%%%%%
\section{Discr\'etisation}
%%%%%%%%%%%%%%%%%%%%%%%%%%%%%%%%%%
%%%%%%%%%%%%%%%%%%%%%%%%%%%%%%%%%%

Le terme convectif en $\dive(\underline{u} \otimes \rho\,\underline{u})$
introduit une non lin\'earit\'e et un couplage des composantes de la vitesse
$\vect{u}$ dans l'�quation (\ref{Base_Preduv_eqqdm}). Une lin\'earisation et un d\'ecouplage
des trois composantes de la 
vitesse sont r\'ealis\'es lors de la discr\'etisation de cette \'etape de
pr\'ediction.\\
En effet, soit :
\begin{equation}
\vect{\widetilde{u}}= \vect{u}^n + \delta \vect{u} 
\end{equation}
La contribution exacte du terme convectif \`a prendre en compte dans cette
\'etape de pr\'ediction serait :\\
\begin{equation}\label{Base_Preduv_Conv_exact}
\begin{array}{ll}
\dive(\vect{\widetilde{u}} \otimes \rho\,\vect{\widetilde{u}}) =
\dive(\vect{u}^{n} \otimes \rho\,\vect{u}^{n}) + \dive(\delta \vect{u} \otimes
\rho\,\vect{u}^{n}) +  \underbrace { \dive(\vect{u}^{n} \otimes
\rho\,\delta \vect{u})}_{\text {terme couplant lin\'eaire}} +  \underbrace { \dive(\delta \vect{u} \otimes
\rho\,\delta \vect{u})}_{\text {terme couplant et non lin\'eaire}}\\
\end{array} 
\end{equation}
Les deux derniers termes de l'expression (\ref{Base_Preduv_Conv_exact}) sont {\em a priori} n�glig�s
de mani�re � obtenir un syst\`eme en vitesse qui soit d\'ecoupl\'e et donc,
�viter l'inversion d'une matrice pouvant \^etre de tr\`es grande taille. Ces
deux termes peuvent n�anmoins �tre pris en compte de mani�re plus ou moins
approch�e par extrapolation explicite du flux de masse en $n+\theta_F$ (pour le
terme couplant lin�aire provenant de la convection de $\vect{u}^{n}$ par $\delta
\vect{u}$) et utilisation d'un point-fixe par sous it�ration sur le sous
programme \fort{navsto} (pour le terme non-lin�aire, en sp�cifiant $\var{NTERUP}>1$).

L'�quation (\ref{Base_Preduv_eqqdm}) est discr�tis�e au temps $n+\theta$ � l'aide d'un
$\theta$-sch�ma, et le tenseur des pertes de charges d�compos� en une partie
diagonale $\tens{K}_{d}$ et une extradiagonale $\tens{K}_{e}$ (soit
 $\tens{K}_{pdc}=\tens{K}_{d}+\tens{K}_{e}$).\\
$\bullet$ La pression est suppos�e connue en $n-1+\theta$ (d�calage temporel
pression-vitesse) et le gradient naturellement calcul� � cet instant.\\ 
$\bullet$ Les termes sources de viscosit� secondaire, de gradient transpos\'e,
ceux provenant du mod�le de turbulence\footnote{except� $\dive (\mu_t\ (\ggrad
\underline {u}))$}, $\rho\,\tens{K}_{\,e}\ \underline{u}$, $(\rho -\rho_0)
\underline {g}$ ainsi que $\underline{T}_{s}^{\,exp}$ et
$\Gamma\,\underline{u}_{\,i}$ sont pris de mani�re explicite au temps $n$, ou
extrapol�s suivant le sch�ma en temps choisi pour les propri�t�s physique et les
termes sources.\\ 
$\bullet$ Les termes sources $\underline{u}\,\,\dive (\rho\,\underline {u})$,
$\Gamma\,\,\underline{u}$, $T_{s}^{\,imp}\,\,\underline{u}$ et
$-\rho\,\tens{K}_{\,d}\,\,\underline{u}$ sont implicit�s est calcul�s �
l'instant $n+\theta$.\\ 
$\bullet$ Le terme de diffusion $\dive (\mu_{\,tot}\,\ggrad \underline{u})$ est
implicit� : la vitesse est prise � l'instant $n+\theta$ et la viscosit�
explicit�e ou extrapol�e.\\ 
$\bullet$ Enfin, le terme de convection en $\dive(\,\underline{u} \otimes
(\rho\underline{u})\,)$ est implicit� : la composante r�solue de la vitesse est
prise en $n+\theta$, et le flux de masse, explicit�, ou extrapol� en
$n+\theta_F$. 

Par souci de clart�, on suppose, en l'absence d'indication, que les propri�tes
physiques $\Phi$ ($\rho,\,\mu_{tot},\,...$) et le flux de masse
$(\rho\underline{u})$ sont pris respectivement aux instants $n+\theta_\Phi$ et
$n+\theta_F$, o� $\theta_\Phi$ et $\theta_F$ d�pendent des sch�mas en temps
sp�cifiquement utilis�s pour ces grandeurs\footnote{cf. \fort{introd}}. 

La discr�tisation temporelle de l'�quation (\ref{Base_Preduv_eqqdm}) s'�crit alors comme suit : 

\begin{equation}\label{Base_Preduv_eq_di1}
 \begin{array}{c}
\displaystyle \rho\,\ \frac{ \underline {\widetilde{u}}^{n+1} -\underline {u}^{n} }
{\Delta t} + \dive(\,\underline{\widetilde{u}}^{n+\theta} \otimes (\rho\underline{u})\,) -\dive
(\mu_{\,tot}\,\ggrad \underline{\widetilde{u}}^{n+\theta}) =
\\
\displaystyle
 - \grad p^{n-1+\theta} + \dive (\rho\,\underline {u})\,\underline{\widetilde{u}}^{n+\theta} +(\Gamma\,\underline{u}_{\,i})^{n+\theta_S}-\Gamma^n\,\,\underline{\widetilde{u}}^{n+\theta}
\\
\begin{array}{c}
\displaystyle
- \rho\,\tens{K}_{\,d}^{n}\,\,\underline{\widetilde{u}}^{n+\theta} - (\rho\,\tens{K}_{\,e}\ \underline{u})^{n+\theta_S} + (\underline{T}_{s}^{\,exp})^{\,n+\theta_S} + T_{s}^{\,imp}\,\,\underline{\widetilde{u}}^{n+\theta}
\\
\displaystyle
+[\dive (\mu_{\,tot}\,^t\ggrad \underline {u})]^{n+\theta_S}-\frac {2} {3}[\,\grad (\mu_{\,tot}\,\dive \underline {u})]^{n+\theta_S} + (\rho -\rho_0) \underline {g}
 - (\underline{turb})^{n+\theta_S}
\end{array}
\end{array}
\end{equation}
o\`u, par souci de simplification, on a pos\'e :
\begin{equation}
\mu_{\,tot}=
\begin{cases}
\mu+\mu_t & \text{pour les mod�les � viscosit� turbulente ou en LES}, \\
\mu & \text{pour les mod�les au second ordre ou en laminaire}
\end{cases} \ 
\end{equation}
\\
et :
\begin{equation}
\underline{turb}^{n}=
\begin{cases}
\displaystyle\frac {2}{3}\grad (\rho^{n}\,k^{n}) & \text{pour les mod�les � viscosit� turbulente}, \\
\dive(\rho^{n}\,\tens{R}^n) & \text{pour les mod�les au second ordre},\\
0 & \text{en laminaire ou en LES}\\
\end{cases}
\end{equation}
Par analogie avec l'�criture du $\theta$-sch�ma pour une variable scalaire, $\,
\underline {\widetilde{u}}^{n+\theta}$ est interpol�e � partir de la vitesse
pr�dite $\underline {\widetilde{u}}^{n+1}$ de la mani\`ere suivante\footnote{si
$\theta=1/2$, ou qu'une extrapolation est utilis�e, l'ordre 2 n'est obtenu que si
le pas de temps $\Delta t$ est uniforme en temps et en espace.}~: 
\begin{equation}
\underline {\widetilde{u}}^{n+\theta}=\theta\, \underline
{\widetilde{u}}^{n+1}+(1-\theta)\, \underline {u}^{n}\\ 
\end{equation}
Avec :
\begin{equation}
\left\{
\begin{array}{ll}
\theta = 1   & \text{Pour un sch\'ema de type Euler implicite d'ordre 1.}\\
\theta = 1/2 & \text{Pour un sch\'ema de type Cranck-Nicolson d'ordre 2.}\\
\end{array}
\right.
\end{equation}

L'�quation (\ref{Base_Preduv_eq_di1}) est alors r��crite sous la forme :

\begin{equation}\label{Base_Preduv_eq_di2}
\begin{array}{c}
\displaystyle \underbrace{\left(\frac{\rho}{\Delta t} -\theta \,\dive (\rho\,\underline {u}) +\theta \,\, \Gamma^n +
\theta \,\, \rho\,\tens{K}_{\,d}^n-\theta \,T_s^{\,imp} \right)}_{\displaystyle f_s^{imp}}\, (\underline {\,\widetilde{u}}^{n+1} -\underline {u}^{n})
\\
 +\, \theta\, \dive(\underline {\widetilde{u}}^{n+1} \otimes (\rho\underline{u}))-\, \theta\,\dive (\mu_{\,tot}\,\ggrad \underline {\widetilde{u}}^{n+1}) =
\\
-\,(1-\theta)\, \dive(\underline {u}^{n} \otimes (\rho\underline{u})) +\,(1-\theta)\,\dive (\mu_{\,tot}\,\ggrad \underline {u}^{n})
\\
f_s^{exp}\left\{
\begin{array}{c}
\displaystyle 
- \grad p^{n-1+\theta} + \dive (\rho\,\underline {u})\,\underline{u}^{n} +\,(\,\Gamma^{n}\,\underline{u}_{\,i}\,)^{n+\theta_S}- \Gamma^n\,\,\underline{u}^{n}
\\
\displaystyle
-(\,\rho\,\tens{K}_{\,e}\ \underline{u}\,)^{n+\theta_S} -\rho\,\tens{K}_{\,d}^n\ \underline{u}^{n}+ (\underline{T}_{s}^{\,exp})^{\,n+\theta_S} + T_s^{\,imp}\,\,\underline {u}^{n} 
\\
\displaystyle
+[\dive (\mu_{\,tot}\,^t\ggrad \underline {u}\,)]^{n+\theta_S}-\frac {2} {3}[\,\grad (\mu_{\,tot}\,\dive \underline {u}\,)]^{n+\theta_S} + (\rho -\rho_0) \underline {g}-(\underline{turb})^{n+\theta_S}
\end{array}
\right.
\end{array}
\end{equation}

d'o� l'�quation r�solue par le sous-programme \fort{codits} :
\begin{equation}\begin{array}{c}
\displaystyle
f_s^{\,imp}(\underline {\widetilde{u}}^{n+1}-\underline {u}^{n}) + \theta\, \dive(\underline{\widetilde{u}}^{n+1} \otimes (\rho
\underline{u})) - \theta\,\dive (\,\mu_{\,tot}\,\ggrad \underline{\widetilde{u}}^{n+1}) = 
\\\\
\displaystyle
-(1-\theta)\,\dive(\underline{u}^{n} \otimes (\rho \underline{u}))+(1-\theta)\,\dive (\,\mu_{\,tot}\,\ggrad \underline{u}^{n})
+ \underline{f}_{\,s}^{\,exp}
\end{array}
\end{equation}
La m\'ethode de discr\'etisation spatiale est d\'evelopp\'ee dans le sous-programme \fort{codits}.\\



\minititre{Remarques :}
{\tiny$\blacksquare$} Dans le cas standard sans extrapolation, le terme
$-\,T_s^{\,imp}$ n'est ajout� � $f_s^{\,imp}$ que s'il est positif afin de ne
pas affaiblir la dominance de la diagonale de la matrice � inverser.\\ 
{\tiny$\blacksquare$} Si une extrapolation est utilis�e, par contre,
$\,T_s^{\,imp}$ est ajout� � $f_s^{\,imp}$ quel que soit son signe. En effet, l'id�e intuitive qui
consiste � prendre~: 
\begin{equation}
\begin{cases}
\displaystyle
(\underline{T}_{s}^{\,exp} + T_{s}^{\,imp}\,\underline {u})^{\,n+\theta_S} &
\text{si } T_{s}^{\,imp} > 0\\ 
\displaystyle
(\underline{T}_{s}^{\,exp})^{\,n+\theta_S} + T_{s}^{\,imp}\,\underline{u}^{n+\theta} &\text{sinon}\\
\end{cases}
\end{equation} 
aboutit � une incoh�rence dans le traitement si $T_s^{imp}$ change de signe
entre deux pas de temps.\\ 
{\tiny$\blacksquare$} la partie diagonale $\tens{K}_{\,d}$ du terme
de perte de charge est utilis�e dans $f_s^{\,imp}$. En fait, pour \^etre rigoureux,
il faudrait ne retenir que les contributions positives (point signal\'e dans le
sous-programme utilisateur associ\'e \fort{uskpdc}). Cette prise en compte sera \`a am\'eliorer.\\
{\tiny$\blacksquare$} Le terme $\theta\,\Gamma^{n}-\theta\,\dive
(\rho\,\underline {u})$ ne pose pas de probl�me pour la 
dominance de la diagonale de la matrice car il est exactement compens� par le
terme de convection (cf. \fort{covofi}). 


%                      Code_Saturne version 1.3
%                      ------------------------
%
%     This file is part of the Code_Saturne Kernel, element of the
%     Code_Saturne CFD tool.
%
%     Copyright (C) 1998-2007 EDF S.A., France
%
%     contact: saturne-support@edf.fr
%
%     The Code_Saturne Kernel is free software; you can redistribute it
%     and/or modify it under the terms of the GNU General Public License
%     as published by the Free Software Foundation; either version 2 of
%     the License, or (at your option) any later version.
%
%     The Code_Saturne Kernel is distributed in the hope that it will be
%     useful, but WITHOUT ANY WARRANTY; without even the implied warranty
%     of MERCHANTABILITY or FITNESS FOR A PARTICULAR PURPOSE.  See the
%     GNU General Public License for more details.
%
%     You should have received a copy of the GNU General Public License
%     along with the Code_Saturne Kernel; if not, write to the
%     Free Software Foundation, Inc.,
%     51 Franklin St, Fifth Floor,
%     Boston, MA  02110-1301  USA
%
%-----------------------------------------------------------------------
%

%%%%%%%%%%%%%%%%%%%%%%%%%%%%%%%%%%
%%%%%%%%%%%%%%%%%%%%%%%%%%%%%%%%%%
\section{Mise en \oe uvre}
%%%%%%%%%%%%%%%%%%%%%%%%%%%%%%%%%%
%%%%%%%%%%%%%%%%%%%%%%%%%%%%%%%%%%
La num\'ero de la phase trait\'ee fait partie des arguments de \fort{turrij}. On
omettra volontairement de le pr\'eciser dans ce qui suit, on indiquera par $(\ )$ la
notion de tableau s'y rattachant.

\etape{Calcul des termes de production $\tens{\mathcal{P}}$}
\begin{itemize}
\item [$\star$] Initialisation \`a z\'ero du tableau \var{PRODUC} dimensionn\'e \`a $\var{NCEL}\times 6$.
\item [$\star$] On appelle trois fois \fort{grdcel} pour calculer les gradients des composantes de la vitesse $u$, $v$ et
$w$ prises au temps $n$.

Au final, on a :\\
$\displaystyle
\begin{array} {ll}
\var{PRODUC(1,IEL)} = & \displaystyle - 2 \left[ R_{11}^{\,n} \frac{\partial u^{\,n}} {\partial x} +R_{12}^{\,n} \frac{\partial u^{\,n}} {\partial y}+R_{13}^{\,n} \frac{\partial u^{\,n}} {\partial z} \right] \text{        (production de $R_{11}^{\,n}$)}\\
\var{PRODUC(2,IEL)} = & \displaystyle - 2 \left[ R_{12}^{\,n} \frac{\partial v^{\,n}} {\partial x} +R_{22}^{\,n} \frac{\partial v^{\,n}} {\partial y}+R_{23}^{\,n} \frac{\partial v^{\,n}} {\partial z} \right] \text{        (production de $R_{22}^{\,n}$)}\\
\var{PRODUC(3,IEL)} = & \displaystyle - 2 \left[ R_{13}^{\,n} \frac{\partial w^{\,n}} {\partial x} +R_{23}^{\,n} \frac{\partial w^{\,n}} {\partial y}+R_{33}^{\,n} \frac{\partial w^{\,n}} {\partial z} \right] \text{        (production de $R_{33}^{\,n}$)}\\
\var{PRODUC(4,IEL)} = & \displaystyle - \left[ R_{12}^{\,n} \frac{\partial u^{\,n}} {\partial x} +R_{22}^{\,n} \frac{\partial u^{\,n}} {\partial y}+R_{23}^{\,n} \frac{\partial u^{\,n}} {\partial z} \right] \\
& \displaystyle - \left[ R_{11}^{\,n} \frac{\partial v^{\,n}} {\partial x} +R_{12}^{\,n} \frac{\partial v^{\,n}} {\partial y}+R_{13}^{\,n} \frac{\partial v^{\,n}} {\partial z} \right] \text{        (production de $R_{12}^{\,n}$)} \\
\var{PRODUC(5,IEL)} = & \displaystyle - \left[ R_{13}^{\,n} \frac{\partial u^{\,n}} {\partial x} +R_{23}^{\,n} \frac{\partial u^{\,n}} {\partial y}+R_{33}^{\,n} \frac{\partial u^{\,n}} {\partial z} \right] \\
& \displaystyle - \left[ R_{11}^{\,n} \frac{\partial w^{\,n}} {\partial x} +R_{12}^{\,n} \frac{\partial w^{\,n}} {\partial y}+R_{13}^{\,n} \frac{\partial w^{\,n}} {\partial z} \right] \text{        (production de $R_{13}^{\,n}$)} \\
\var{PRODUC(6,IEL)} = & \displaystyle - \left[ R_{13}^{\,n} \frac{\partial v^{\,n}} {\partial x} +R_{23}^{\,n} \frac{\partial v^{\,n}} {\partial y}+R_{33}^{\,n} \frac{\partial v^{\,n}} {\partial z} \right] \\
& \displaystyle - \left[ R_{12}^{\,n} \frac{\partial w^{\,n}} {\partial x} +R_{22}^{\,n} \frac{\partial w^{\,n}} {\partial y}+R_{23}^{\,n} \frac{\partial w^{\,n}} {\partial z} \right]  \text{        (production de $R_{23}^{\,n}$)}
\end{array}
$
\end{itemize}

\etape{Calcul du gradient de la masse volumique $\rho^n$ prise au d\'ebut du pas
de temps courant\footnote{{\it i.e.} calcul\'ee \`a partir des
variables du pas de temps pr\'ec\'edent $n$ si n\'ecessaire.} $(n+1)$}
Ce calcul n'a lieu que si les termes de gravit\'e doivent \^etre pris en compte
($\var{IGRARI()} =1$).
\begin{itemize}
\item [$\star$] Appel de \fort{grdcel}  pour calculer le gradient de $\rho^n$
dans les trois directions de l'espace. Les conditions aux limites sur $\rho^n$
sont des conditions de Dirichlet puisque la valeur de $\rho^n$ aux faces de bord
$ik$ (variable \var{IFAC}) est connue et vaut $\rho_{\,b_{\,ik}}$. Pour \'ecrire les conditions aux limites
sous la forme habituelle, $$\rho_{\,b_{\,ik}} = \var{COEFA} + \var{COEFB}
\,\rho^n_{\,I'}$$ on pose alors $\var{COEFA}=
\var{PROPCE(IFAC,IPPROB(IROM(IPHAS)))}$ et $\var{COEFB} = \var{VISCB} = 0$.\\
$\var{PROPCE(1,IPPROB(IROM(IPHAS)))}$ (resp.$\var{VISCB}$) est utilis\'e en lieu
et place de l'habituel \var{COEFA} ($\var{COEFB}$), lors de l'appel \`a \fort{grdcel}.\\
On a donc :\\
$\displaystyle \var{GRAROX}= \frac{\partial \rho^n}{\partial x}\ $,$\displaystyle \ \var{GRAROY}= \frac{\partial
\rho^n}{\partial y}$ et $
\displaystyle \ \var{GRAROZ}= \frac{\partial \rho^n}{\partial z}\ $.

\end{itemize}

Le gradient de $\rho^n$ servira \`a calculer les termes de production par effets de gravit\'e si ces derniers sont pris en compte.

\etape{Boucle \var{ISOU} de $1$ \`a $6$ sur les tensions de Reynolds}
Pour $\var{ISOU} = 1,2,3,4,5,6$, on r\'esout respectivement et dans
l'ordre  les
\'equations de $R_{11}$, $R_{22}$, $R_{33}$, $R_{12}$, $R_{13}$ et $R_{23}$ par
l'appel au sous-programme \fort{resrij}.\\
La r\'esolution se fait par incr\'ement $\delta {R}_{ij}^{\,n+1,k+1}$ , en utilisant la m\^eme m\'ethode que
celle d\'ecrite dans le sous-programme \fort{codits}. On adopte ici les m\^emes notations.
\var{SMBR} est le second membre du syst\`eme \`a inverser, syst\`eme portant sur
les incr\'ements de la variable. \var{ROVSDT} repr\'esente la diagonale de la
matrice, hors convection/diffusion.\\
On va r\'esoudre l'\'equation (\ref{Base_Turrij_Eq_Temp_Rij}) sous forme incr\'ementale en
utilisant \fort{codits}, soit :
\begin{equation}\label{Base_Turrij_Eq_Temp_deltaRij}
\begin{array}{ll}
&\displaystyle \underbrace{\left(\frac {\rho^n_L}{\Delta t^n}
+ \rho^n_L \,C_1\,\frac{\varepsilon^n_L}{k^n_L}(1-\frac{\delta_{ij}}{3})
 - m^n_{\,lm} + \Gamma_L\,+ max(-\alpha^n_{R_{ij}},0)\right)\,|\Omega_l|}
_{\text {$\var{ROVSDT}$ contribuant
\`a la diagonale de la matrice simplifi\'ee de \fort{matrix}}}\,(\delta{R}_{ij}^{\,n+1,p+1})_{\,L}\\\\
&  \underbrace{+\sum\limits_{m\in Vois(l)}\displaystyle \left[
 m^n_{\,lm} \delta R_{ij,\,f_{\,lm}}^{\,n+1,p+1}
- (\mu^n_{\,lm} + \gamma^n_{\,lm})\
\frac{({\delta R}_{ij}^{\,n+1,p+1})_{M}-({\delta R}_{ij}^{\,n+1,p+1})_{L})}{\overline{L'M'}}\,
S_{\,lm} \right]}_{\text { convection upwind pur et diffusion non reconstruite
relatives \`a la matrice simplifi\'ee de \fort{matrix}\footnotemark}} \\
% voir le texte de la footmark plus bas
&= - \displaystyle\frac {\rho^n_L}{\Delta t^n}\,\left(\,(R^{\,n+1,p}_{ij})_L - (R^{\,n}_{ij})_L\,\right)\\
&-\,\underbrace{\displaystyle\int_{\Omega_l} \left(
\dive\,[\,(\rho\,\vect{u})^n\,R^{\,n+1,p}_{ij} - (\mu^n\,+ \gamma^n\,)\,
\grad{R^{\,n+1,p}_{ij}}\,]\right)\,d\Omega}_{\text {convection et diffusion
trait\'ees par \fort{bilsc2}}}\\
&+\displaystyle \int_{\Omega_l} \left[\,\mathcal{P}^{\,n+1,p}_{ij} + \mathcal{G}^{\,n+1,p}_{ij}
- \displaystyle\rho^n \,C_1\,\frac{\varepsilon^n}{k^n}\left[R^{\,n+1,p}_{ij}-
\frac{2}{3}\,k^n\,\delta_{ij}\right] + \phi^{\,n+1,p}_{ij,2} +
\phi^{\,n+1,p}_{ij,w}\,\right]\, d\Omega \\
& + \displaystyle\int_{\Omega_l} \left[- \frac{2}{3} \rho^n \varepsilon^n \delta_{ij}
 + \Gamma\,(\,R^{\,in}_{ij} - R^{\,n+1,p}_{ij}\,) +
\alpha^n_{R_{ij}}\,R^{\,n+1,p}_{ij}+ \beta^n_{R_{ij}}\right]\, d\Omega\\
&+ \sum\limits_{m\in
Vois(l)}\displaystyle \left[\ \tens{E}^n\,\grad{R}^{\,n+1,p}_{ij} \right]_{\,lm}\,.\,\vect{n}_{\,lm}S_{\,lm}\\
&+ \sum\limits_{m\in Vois(l)}\displaystyle \left[\
\tens{D}^n\,\grad{R}^{\,n+1,p}_{ij} \right]_{\,lm}\,.\,\vect{n}_{\,lm}S_{\,lm}\\
&- \sum\limits_{m\in Vois(l)} \gamma^n_{\,lm} \left( \grad{R}^{\,n+1,p}_{ij}\,.\,\vect{n}_{\,lm} \right)  S_{\,lm}\\
&+ \sum\limits_{m\in Vois(l)}  m^n_{\,lm}\,(R^{\,n+1,p}_{ij})_L\\
\end{array}
\end{equation}
% si on ne fait pas comme ca, il n'apparait pas
\footnotetext[\thefootnote]{Si $\var{IRIJNU} = 1$, on remplace  $\mu^n_{\,lm}$ par $(\mu +
\mu_t)^n_{\,lm}$ dans l'expression de la diffusion non reconstruite
associ\'ee \`a la matrice simplifi\'ee de \fort{matrix} ($\mu_t$ d\'esigne la
viscosit\'e turbulente calcul\'ee comme en $k-\varepsilon$).}

o\`u on rappelle :\\
pour $n$ donn\'e entier positif, on d\'efinit la suite
 $({R}_{ij}^{\,n+1,p})_{p \in \grandN}$
 par :
\begin{equation}\notag
\left\{\begin{array}{l}
{R}_{ij}^{\,n+1,0} = {R}_{ij}^{\,n}\\
{R}_{ij}^{\,n+1,p+1} = {R}_{ij}^{\,n+1,p} + \delta{R}_{ij}^{\,n+1,p+1} \\
\end{array}\right.
\end{equation}
$(\delta{R}_{ij}^{\,n+1,p+1})_{\,L}$ d\'esigne la valeur sur l'\'el\'ement
$\Omega_l$ du $\text{$(\,p+1\,)$-i\`eme}$ incr\'ement de ${R}_{ij}^{\,n+1}$,
$ m^n_{\,lm}$ le flux de masse \`a l'instant $n$ \`a travers la face $lm$,
$\delta R_{ij,\,f_{\,lm}}^{\,n+1,p+1}$ vaut $({\delta
R}_{ij}^{\,n+1,p+1})_{L}$  si $ m^n_{\,lm} \geqslant 0$, $({\delta
R}_{ij}^{\,n+1,p+1})_{M}$ sinon,
$\mathcal{P}^{\,n+1,p}_{ij}$, $\phi^{\,n+1,p}_{ij,2}$, $\phi^{\,n+1,p}_{ij,w}$ les valeurs
des quantit\'es associ\'ees correspondant \`a l'incr\'ement
$(\delta{R}_{ij}^{\,n+1,p})$.\\



Tous ces termes sont calcul\'es comme suit :
\begin{itemize}
\item Terme de gauche de l'\'equation (\ref{Base_Turrij_Eq_Temp_deltaRij})\\
Dans \fort{resrij} est calcul\'ee la variable \var{ROVSDT}. Les autres
termes sont compl\'et\'es par \fort{codits}, lors de la construction de la matrice simplifi\'ee , {\it via} un
appel au sous-programme \fort{matrix}. La quantit\'e
 $(\mu^n_{\,lm} + \gamma^n_{\,lm})$ \`a la face $lm$ est calcul\'ee lors de l'appel \`a
\fort{visort}.\\
\item Second membre de l'\'equation (\ref{Base_Turrij_Eq_Temp_deltaRij})\\
Le premier terme non d\'etaill\'e est calcul\'e par le sous-programme
\fort{bilsc2}, qui applique le sch\'ema convectif choisi par l'utilisateur, qui
reconstruit ou non selon le souhait de l'utilisateur les gradients intervenants
dans la convection-diffusion.\\
Les termes sans accolade sont, eux, compl\`etement explicites et ajout\'es au fur et
\`a mesure dans \var{SMBR} pour former
l'expression $f^{\,exp}_s$ de \fort{codits}.
\end{itemize}
On d\'ecrit ci-dessous les \'etapes de \fort{resrij} :
\begin{itemize}

\item DELTIJ = 1, pour $\var{ISOU} \leqslant 3$ et DELTIJ = 0  Si $\var{ISOU} >
3$. Cette valeur repr\'esente le symbole de Kroeneker $\delta_{ij}$.

\item Initialisation \`a z\'ero de \var{SMBR} (tableau contenant le second
membre) et \var{ROVSDT} (tableau contenant la diagonale de la matrice sauf celle
relative \`a la contribution de la
diagonale des op\'erateurs de convection et de diffusion lin\'earis\'es
\footnote{qui correspondent aux sch\'emas convectif upwind pur et diffusif sans
reconstruction.}), tous deux de dimension $\var{NCEL}$.

\item Lecture et prise en compte des termes sources utilisateur pour la variable $R_{ij}$

Appel \`a \fort{ustsri} pour charger les termes sources utilisateurs. Ils sont
stock\'es comme suit. Pour la cellule $\Omega_l$ de centre $L$, repr\'esent\'ee par $\var{IEL}$, on a :\\
\begin{equation}\notag
\left\{\begin{array}{lll}
&\var{ROVSDT(IEL)}&= |\Omega_l| \ \alpha_{R_{ij}}\\
&\var{SMBR(IEL)}&=|\Omega_l| \ \beta_{R_{ij}}\\
\end{array}\right.
\end{equation}
On affecte alors les valeurs ad\'equates au second membre \var{SMBR} et \`a la
diagonale \var{ROVSDT} comme suit :
\begin{equation}\notag
\left\{\begin{array}{lll}
&\var{SMBR(IEL)} &= \var{SMBR(IEL)} +\ |\Omega_l| \ \alpha_{R_{ij}} \ (R^n_{ij})_L \\
&\var{ROVSDT(IEL)}&= \text{max }(-\ |\Omega_l| \ \alpha_{R_{ij}},0)\\
\end{array}\right.
\end{equation}
La valeur de $ \var{ROVSDT}$ est ainsi calcul\'ee pour des raisons de stabilit\'e
num\'erique. En effet, on ne rajoute sur la diagonale que les valeurs positives,
ce qui correspond physiquement \`a impliciter les termes de rappel uniquement.
\item{Calcul du terme source de masse  si $\Gamma_L > 0$}

Appel de \fort{catsma} et incr\'ementation si n\'ecessaire de \var{SMBR} et
\var{ROVSDT} {\it via} :\\
\begin{equation}\notag
\left\{\begin{array}{lll}
\displaystyle \var{SMBR(IEL)} = \var{SMBR(IEL)} + |\Omega_l| \ \Gamma_L \
\left[(R^{\,in}_{ij})_L - (R^{\,n}_{ij})_L \right] \\
\displaystyle \var{ROVSDT(IEL)}=\var{ROVSDT(IEL)} + |\Omega_l| \ \Gamma_L
\end{array}\right.
\end{equation}
\item Calcul du terme d'accumulation de masse et du terme instationnaire

On stocke $\displaystyle \var{W1}= \int_{\Omega_l}\dive\,(\rho\,\vect{u})\,d\Omega$
calcul\'e par \fort{divmas} \`a l'aide des flux de masse aux faces internes
$ m^n_{\,lm}=\sum\limits_{m\in Vois(l)}{(\rho \vect{u})_{\,lm}^n} \text{.}\,
\vect{S}_{\,lm} $ (tableau \var{FLUMAS}) et des flux de masse aux bords  $ m^n_{\,b_{lk}} = \sum\limits_{k\in{\gamma_b(l)}}{(\rho \vect{u})_{\,{b}_{lk}}^n} \text{.}\,
\vect{S}_{\,{b}_{lk}} $ (tableau \var{FLUMAB}).
On incr\'emente ensuite \var{SMBR} et \var{ROVSDT}.
\begin{equation}\notag
\left\{\begin{array}{lll}
&\var{SMBR(IEL)} &= \var{SMBR(IEL)} + \var{ICONV}\  (R^n_{ij})_L\,(\displaystyle
\int_{\Omega_l}\dive\,(\rho\,\vect{u})\ d\Omega) \\
&\var{ROVSDT(IEL)}& = \var{ROVSDT(IEL)} +  \var{ISTAT}\,\displaystyle
\frac{\rho^n_L \ |\Omega_l|}{\Delta t^n} -  \var{ICONV}\ (\displaystyle
\int_{\Omega_l}\dive\,(\rho\,\vect{u})\ d\Omega) \\
\end{array}\right.
\end{equation}
\item Calcul des termes sources de production, des termes $\displaystyle
\phi_{\,ij,1}+\phi_{\,ij,2}$ et de la dissipation~$\displaystyle-\frac{2}{3} \varepsilon\,\delta_{\,ij}$ :

On effectue une boucle d'indice \var{IEL} sur les cellules $\Omega_l$ de centre $L$ :
\begin{itemize}
\item [$\Rightarrow$] $\displaystyle \var{TRPROD}= \frac{1}{2} (\mathcal{P}^n_{ii})_L = \frac{1}{2} \left[ \var{PRODUC(1,IEL)} +  \var{PRODUC(2,IEL)} +  \var{PRODUC(3,IEL)} \right] $
\item [$\Rightarrow$] $\displaystyle \var{TRRIJ }= \frac{1}{2} (R^n_{ii})_L $
\item [$\Rightarrow$] $\displaystyle \var{SMBR(IEL)} =\ \var{SMBR(IEL)}\ +$\\
$\ \displaystyle\rho^n_L |\Omega_l| \left[ \displaystyle
\frac{2}{3}\,\delta_{\,ij} \left( \ \displaystyle \frac{ C_2}{2}\,(\mathcal{P}^n_{ii})_L\ +
(C_1-1)\ \varepsilon^n_L\, \right)\right.$\\
$ + \left.\ (1-C_2) \ \var{PRODUC(ISOU,IEL)} -
\displaystyle C_1\ \frac{2\,\varepsilon^n_L}{(R^n_{ii})_L}\ (R^n_{ij})_L \right]$
\item [$\Rightarrow$] $\displaystyle \var{ROVSDT(IEL)} = \var{ROVSDT(IEL)} +
\rho^n_L \ |\Omega_l| \ (- \displaystyle \frac{1}{3} \ \,\delta_{\,ij} + 1) \ C_1
\ \frac{2\ \varepsilon^n_L}{(R^n_{ii})_L}$
\end{itemize}
\item Appel de \fort{rijech} pour le calcul des termes d'\'echo de paroi
 $\phi^n_{ij,w}$ si $\var{IRIJEC()}=1$ et ajout dans \var{SMBR}.\\
$\var{SMBR} = \var{SMBR} + \phi^n_{ij,w}$\\
Suivant son mode de calcul (\var{ICDPAR}), la distance � la paroi est directement accessible
par \var{RA(IDIPAR+IEL-1)} (\var{|ICDPAR|} = 1) ou bien
est calcul\'ee \`a partir de $\var{IA(IIFAPA(IPHAS)+IEL - 1)}$,
qui donne pour l'\'el\'ement $\var{IEL}$ le num\'ero de la face de bord
paroi la plus  proche (\var{|ICDPAR|} = 2). Ces tableaux ont \'et\'e renseign\'e une fois pour toutes au
d\'ebut de calcul.

\item  Appel de \fort{rijthe} pour le calcul des termes de gravit\'e $\mathcal{G}^n_{ij}$ et ajout dans \var{SMBR}.

Ce calcul n'a lieu que si $\var{IGRARI()} = 1$.
$ \var{SMBR} = \var{SMBR} + \mathcal{G}^n_{ij}$
\item Calcul de la partie explicite du terme de diffusion
 $\dive{\,\left[\tens{A}\,\grad{R}^{\,n}_{ij}\right]}$, plus pr\'ecis\'ement
des contributions du terme extradiagonal pris aux faces purement internes
(remplissage du tableau \var{VISCF}), puis aux faces de bord (remplissage du
tableau \var{VISCB}).
\begin{itemize}
\item [$\star$] Appel de \fort{grdcel} pour le calcul du gradient de
$R^{\,n}_{ij}$ dans chaque direction. Ces gradients sont respectivement
stock\'es dans les tableaux de travail \var{W1}, \var{W2} et \var{W3}.

\item [$\star$] boucle d'indice \var{IEL} sur les cellules $\Omega_l$ de centre
$L$ pour le
calcul de $\tens{E}^n\,\grad{R}^{\,n}_{ij}$ aux cellules dans un premier temps :\\
\begin{itemize}
\item [$\Rightarrow$] $\displaystyle \var{TRRIJ}= \frac{1}{2} (R^{\,n}_{ii})_L $
\item [$\Rightarrow$] $\displaystyle \var{CSTRIJ} = \rho^n_L\ C_S \ \displaystyle\frac{(R^n_{ii})_L}{2\,\varepsilon^n_L}$
\item [$\Rightarrow$] $\displaystyle \var{W4(IEL)} = \rho^n_L\ C_S\
\displaystyle\frac{(R^n_{ii})_L}{2\,\varepsilon^n_L} \left[\,(R^{\,n}_{12})_L \ \var{W2(IEL)} +
(R^{\,n}_{13})_L \ \var{W3(IEL)}\,\right]$
\item [$\Rightarrow$] $\displaystyle \var{W5(IEL)} = \rho^n_L\ C_S\
\displaystyle\frac{(R^n_{ii})_L}{2\,\varepsilon^n_L} \left[\,(R^{\,n}_{12})_L \ \var{W1(IEL)} +
(R^{\,n}_{23})_L \ \var{W3(IEL)}\,\right]$
\item [$\Rightarrow$] $\displaystyle \var{W6(IEL)} = \rho^n_L\ C_S\
\displaystyle\frac{(R^n_{ii})_L}{2\,\varepsilon^n_L} \left[\,(R^{\,n}_{13})_L \ \var{W1(IEL)} + (R^{\,n}_{23})_L \ \var{W2(IEL)}\,\right]$
\end{itemize}



\item [$\star$] Appel de \fort{vectds}\footnote{Le r\'esultat est stock\'e dans
\var{VISCF} et \var{VISCB}. Dans \fort{vectds}, les valeurs aux faces internes
sont interpol\'ees lin\'eairement sans reconstruction et \var{VISCB} est mis \`a
z\'ero.} pour assembler $\displaystyle\left[ \tens{E}^n\,\grad{R}^{\,n}_{ij}
\right]\,.\,\vect{n}_{\,lm}S_{\,lm}$ aux faces $lm$.
\item [$\star$] Appel de \fort{divmas} pour calculer la divergence du flux d\'efini par \var{VISCF} et \var{VISCB}.
Le r\'esultat est stock\'e dans \var{W4}.\\
Ajout au second membre \var{SMBR}.\\
\var{SMBR} = \var{SMBR} + \var{W4}
\end{itemize}

A l'issue de cette \'etape, seule la partie extradiagonale de la diffusion prise
enti\`erement explicite~:
 $$\sum\limits_{m\in
Vois(l)}\left[\ \tens{E}^n\,\grad{R}^{\,n}_{ij} \right]_{\,lm}\,.\,\vect{n}_{\,lm}S_{\,lm}$$ a \'et\'e calcul\'ee.\\

\item Calcul de la partie diagonale du terme de diffusion\footnote{Seule la
composante normale  du  gradient de $R_{ij}$ aux faces sera implicite.} :\\
Comme on l'a d\'eja vu, une partie de cette contribution sera trait\'ee en
implicite (celle relative \`a la composante normale du gradient) et donc
ajout\'ee au second membre par \fort{bilsc2} ; l'autre
partie sera explicite et prise en compte dans $f_s^{\,exp}$.
\begin{itemize}
\item [$\star$] On effectue une boucle d'indice \var{IEL} sur les cellules
$\Omega_l$ de centre $L$ :
\begin{itemize}
\item [$\Rightarrow$] $\displaystyle \var{TRRIJ }= \frac{1}{2} (R^{\,n}_{ii})_L $
\item [$\Rightarrow$] $\displaystyle \var{CSTRIJ} = \rho^n_L \ C_S \ \frac{(R^{\,n}_{ii})_L}{2\,\varepsilon^n_L}$
\item [$\Rightarrow$] $\displaystyle \var{W4(IEL)} = \rho^n_L \ C_S \
\frac{(R^{\,n}_{ii})_L}{2\,\varepsilon^n_L} \ (R^{\,n}_{11})_L$
\item [$\Rightarrow$] $\displaystyle \var{W5(IEL)} = \rho^n_L \ C_S \ \frac{(R^{\,n}_{ii})_L}{2\,\varepsilon^n_L}\ (R^n_{22})_L$
\item [$\Rightarrow$] $\displaystyle \var{W6(IEL)} = \rho^n_L \ C_S \ \frac{(R^{\,n}_{ii})_L}{2\,\varepsilon^n_L} \ (R^n_{33})_L$
\end{itemize}

%\item Traitement du parall\'elisme et de la p\'eriodicit\'e.

\item [$\star$] On effectue une boucle d'indice \var{IFAC} sur les faces
purement internes $lm$ pour remplir le tableau \var{VISCF} :
\begin{itemize}
\item [$\Rightarrow$] Identification des cellules $\Omega_l$ et $\Omega_m$ de
centre respectif $L$ (variable \var{II}) et $M$ (variable \var{JJ}), se trouvant de chaque c\^ot\'e de la face
$lm$\footnote{La normale \`a la face est orient\'ee de L vers M.}.
\item [$\Rightarrow$] Calcul du carr\'e de la surface de la face. La valeur est
stock\'ee dans le tableau \var{SURFN2}.
\item [$\Rightarrow$] Interpolation du gradient de $R^{\,n}_{ij}$ \`a la face
$lm$ (gradient facette $\left[\grad{R}^{\,n}_{ij}\right]_{\,lm}$) :
\begin{equation}\notag
\left\{\begin{array}{ll}
\var{GRDPX} &= \displaystyle \frac{1}{2} \left(\var{W1(II)} + \var{W1(JJ)}
\right) \\
&\\
\var{GRDPY} &= \displaystyle \frac{1}{2} \left(\var{W2(II)} + \var{W2(JJ)} \right) \\
&\\
\var{GRDPZ} &= \displaystyle \frac{1}{2} \left(\var{W3(II)} + \var{W3(JJ)} \right)
\end{array}\right.
\end{equation}
\item [$\Rightarrow$] Calcul du gradient de $R^{\,n}_{ij}$ normal \`a la face
$lm$, $\left[\grad{R}^{\,n}_{ij}\right]_{\,lm}.\vect{n}_{\,lm}\,S_{\,lm}$.\\

$\displaystyle \var{GRDSN} =  \var{GRDPX} \ \var{SURFAC(1,IFAC)} + \var{GRDPY} \ \var{SURFAC(2,IFAC)} +  \var{GRDPZ} \ \var{SURFAC(3,IFAC)}$
$\var{SURFAC}$ est le vecteur surface de la face \var{IFAC}.


\item [$\Rightarrow$] calcul de
 $\left[\grad{R^{\,n}_{ij}} - (\grad
R^{\,n}_{ij}\,.\,\vect{n}_{\,lm})\vect{n}_{\,lm}\right]$, les vecteurs \'etant
calcul\'es \`a la face $lm$ :
\begin{equation}\notag
\left\{\begin{array}{lll}
&\displaystyle \var{GRDPX} &= \var{GRDPX} - \displaystyle\frac{\var{GRDSN}}{\var{SURFN2}} \ \var{SURFAC(1,IFAC)}\\
&&\\
&\displaystyle \var{GRDPY} &= \var{GRDPY} - \displaystyle\frac{\var{GRDSN}}{\var{SURFN2}} \ \var{SURFAC(2,IFAC)} \\
&&\\
&\displaystyle \var{GRDPZ} &= \var{GRDPZ} - \displaystyle\frac{\var{GRDSN}}{\var{SURFN2}} \ \var{SURFAC(3,IFAC)}
\end{array}\right.
\end{equation}
\item [$\Rightarrow$] finalisation du calcul de l'expression totalement
explicite
 $$\left[ \tens{D}^n\,\left( \grad{R^{\,n}_{ij}} - (\grad R^{\,n}_{ij}\,.\,\vect{n}_{\,lm})\,\vect{n}_{\,lm}\right) \right]\,.\,\vect{n}_{\,lm}$$
\begin{equation}\notag
\begin{array} {ll}
\displaystyle \var{VISCF} = &
 \displaystyle\frac{1}{2} (\ \var{W4(II)} +\ \var{W4(JJ)}) \ \var{GRDPX} \
\var{SURFAC(1,IFAC)})\ + \\
&\\
&  \displaystyle\frac{1}{2} (\ \var{W5(II)} +\ \var{W5(JJ)}) \ \var{GRDPY} \
\var{SURFAC(2,IFAC)})\ + \\
&\\
&  \displaystyle\frac{1}{2} (\ \var{W6(II)} +\ \var{W6(JJ)}) \ \var{GRDPZ} \ \var{SURFAC(3,IFAC)})
\end{array}
\end{equation}
\end{itemize}

\item [$\star$] Mise \`a z\'ero du tableau \var{VISCB}.

\item [$\star$] Appel de \fort{divmas} pour calculer la divergence de~:
 $$\tens{D}^{\,n}\,\left( \grad{R^{\,n}_{ij}} - (\grad R^{\,n}_{ij}\,.\,\vect{n}_{\,lm})\vect{n}_{\,lm}\right)$$ d\'efini aux faces dans \var{VISCF} et \var{VISCB}.

Le r\'esultat est stock\'e dans le tableau \var{W1}.\\
Ajout au second membre \var{SMBR}.\\
$\var{SMBR} = \var{SMBR} + \var{W1}$
\end{itemize}
\item Calcul de la viscosit\'e orthotrope $\gamma^n_{\,lm}$ \`a la face $lm$ de la variable principale
$R^{\,n}_{ij}$\\
Ce calcul permet au sous-programme \fort{codits} de compl\'eter le second membre
\var{SMBR} par :
\begin{equation}
\begin{array} {ll}
& \sum\limits_{m\in Vois(l)}
\mu^n_{\,lm}\,\left(\grad{R}^{\,n}_{ij}\,.\,\vect{n}_{\,lm}\right)S_{\,lm}
 + \sum\limits_{m\in Vois(l)} \left(\grad{R}^{\,n}_{ij}
\,.\,\vect{n}_{\,lm}\right)\left[\tens{D}^{\,n}\,\vect{n}_{\,lm}\right]_{\,lm}\,.\,\vect{n}_{\,lm}\
S_{\,lm}\\
& = \sum\limits_{m\in Vois(l)}(\,\mu^n_{\,lm}\, + \,\gamma^n_{\,lm}\,)\,\left(\grad{R}^{\,n}_{ij}\,.\,\vect{n}_{\,lm}\right)S_{\,lm}
\end{array}
\end{equation}
sans pr\'eciser la nature de la face $lm$, {\it via} l'appel \`a \fort{bilsc2}  et de disposer de la quantit\'e
$(\mu^n_{\,lm}\, + \gamma^n_{\,lm})$ pour construire sa
matrice simplifi\'ee.\\
\begin{itemize}
\item [$\star$] On effectue une boucle d'indice \var{IEL} sur les cellules
$\Omega_l$ :
\begin{itemize}
\item [$\Rightarrow$] $\displaystyle \var{TRRIJ }= \frac{1}{2} (R^{\,n}_{ii})_L $
\item [$\Rightarrow$] $\displaystyle \var{RCSTE} = \rho^n_L \ C_S \ \frac{ (R^{\,n}_{ii})_L}{2\,\varepsilon^n_L} $
\item [$\Rightarrow$] $\displaystyle \var{W1(IEL)} = \mu^n +\rho^n_L \ C_S \ \frac{
(R^{\,n}_{ii})_L}{2\,\varepsilon^n_L}\ (R^n_{11})_L$
\item [$\Rightarrow$] $\displaystyle \var{W2(IEL)} = \mu^n + \rho^n_L \ C_S \ \frac{ (R^{\,n}_{ii})_L}{2\,\varepsilon^n_L}\ (R^n_{22})_L$
\item [$\Rightarrow$] $\displaystyle \var{W3(IEL)} = \mu^n + \rho^n_L \ C_S \ \frac{ (R^{\,n}_{ii})_L}{2\,\varepsilon^n_L}\ (R^n_{33})_L$
\end{itemize}

\item [$\star$] Appel de \fort{visort} pour calculer la viscosit\'e orthotrope
\footnote{Comme dans le sous-programme \fort{viscfa}, on multiplie la viscosit\'e par
$\displaystyle \frac{S_{\,lm}}{\overline{L'M'}}$, o\`u $S_{\,lm}$ et
$\overline{L'M'}$ repr\'esentent respectivement la surface de la face $lm$ et la
mesure alg\'ebrique du segment reliant les projections des centres des cellules
voisines sur la normale \`a la face. On garde dans ce sous-programme  la possibilit\'e d'interpoler la viscosit\'e aux cellules lin\'eairement ou d'utiliser une moyenne harmonique. La viscosit\'e au bord est celle de la cellule de bord correspondante.}
$ \gamma^n_{\,lm} = (\tens{D}^{\,n}\,\vect{n}_{\,lm}).\vect{n}_{\,lm}$ aux faces $lm$

Le r\'esultat est stock\'e dans les tableaux \var{VISCF} et \var{VISCB}.
\end{itemize}

\item appel de \fort{codits} pour la r\'esolution de l'\'equation de
convection/diffusion/termes sources de la variable $R_{ij}$. Le terme source est
\var{SMBR}, la viscosit\'e \var{VISCF} aux faces purement internes (
resp. \var{VISCB} aux faces de bord) et \var{FLUMAS} le flux de masse interne
 ( resp. \var{FLUMAB} flux de masse au bord). Le r\'esultat est la variable $R_{ij}$ au temps
$n+1$, donc $R^{\,n+1}_{ij}$. Elle est stock\'ee dans le tableau \var{RTP} des
variables mises \`a jour.

\end{itemize}

\etape{Appel de \fort{reseps} pour la r\'esolution de la variable $\varepsilon$}

On d\'ecrit ci-dessous le sous-programme \fort{reseps}, les commentaires du sous-programme \fort{resrij} ne seront pas r\'ep\'et\'es puisque les deux sous-programmes ne diff\`erent que par leurs termes sources.

\begin{itemize}
\item Initialisation \`a z\'ero de \var{SMBR} et \var{ROVSDT}.

\item{Lecture et prise en compte des termes sources utilisateur pour la variable $\varepsilon$ :}

Appel de \fort{ustsri} pour charger les termes sources utilisateurs. Ils sont
stock\'es dans les tableaux suivants :\\
pour la cellule $\Omega_l$ repr\'esent\'ee par $\var{IEL}$ de centre $L$, on a :
\begin{equation}\notag
\left\{\begin{array}{lll}
&\var{ROVSDT(IEL)}&= |\Omega_l| \ \alpha_{\varepsilon}\\
&\var{SMBR(IEL)}&=|\Omega_l| \ \beta_{\varepsilon}\\
\end{array}\right.
\end{equation}
On affecte alors les valeurs ad\'equates au second membre \var{SMBR} et \`a la
diagonale \var{ROVSDT} comme suit :
\begin{equation}\notag
\left\{\begin{array}{lll}
&\var{SMBR(IEL)} &= \var{SMBR(IEL)} +\ |\Omega_l| \ \alpha_{\,\varepsilon} \
\varepsilon^n_L \\
&\var{ROVSDT(IEL)}&= \text{max }(-\ |\Omega_l| \ \alpha_{\,\varepsilon},0)\\
\end{array}\right.
\end{equation}

\item{Calcul du terme source de masse si $\Gamma_L > 0$ :
\begin{equation}\notag
\left\{\begin{array}{lll}
&\displaystyle \var{SMBR(IEL)} = \var{SMBR(IEL)} + |\Omega_l| \ \Gamma_L \
(\varepsilon^{\,in}_L -\varepsilon^n_L) \\
&\displaystyle \var{ROVSDT(IEL)}= \var{ROVSDT(IEL)} + |\Omega_l| \ \Gamma_L
\end{array}\right.
\end{equation}
\item Calcul du terme d'accumulation de masse et du terme instationnaire \\
On stocke $\displaystyle \var{W1}= \int_{\Omega_l}\dive\,(\rho\,\vect{u})\,d\Omega$
calcul\'e par \fort{divmas} \`a l'aide des flux de masse internes et aux bords.\\
On incr\'emente ensuite \var{SMBR} et \var{ROVSDT}.
\begin{equation}\notag
\left\{\begin{array}{lll}
&\var{SMBR(IEL)} &= \var{SMBR(IEL)} + \var{ICONV}\ \varepsilon^n_L\,(\displaystyle
\int_{\Omega_l}\dive\,(\rho\,\vect{u})\ d\Omega) \\
&\var{ROVSDT(IEL)}& = \var{ROVSDT(IEL)} +  \var{ISTAT}\,\displaystyle
\frac{\rho^n_L \ |\Omega_l|}{\Delta t^n} -  \var{ICONV}\ (\displaystyle
\int_{\Omega_l}\dive\,(\rho\,\vect{u})\ d\Omega) \\
\end{array}\right.
\end{equation}

\item Traitement du terme de production
 $\displaystyle \rho\,C_{\varepsilon_1}\,\frac{\varepsilon}{k}\,\mathcal{P}$
 et du terme de dissipation $-\,\displaystyle \rho\,C_{\varepsilon_2}\,\frac{\varepsilon}{k}\,\varepsilon$ \\
pour cela, on effectue une boucle d'indice \var{IEL} sur les cellules $\Omega_l$
de centre $L$ :
\begin{itemize}
\item [$\Rightarrow$] $\displaystyle \var{TRPROD}= \frac{1}{2} (\mathcal{P}^n_{ii})_L = \frac{1}{2} \left[ \var{PRODUC(1,IEL)} +  \var{PRODUC(2,IEL)} +  \var{PRODUC(3,IEL)} \right] $
\item [$\Rightarrow$] $\displaystyle \var{TRRIJ }= \frac{1}{2} (R^n_{ii})_L $
\item [$\Rightarrow$] $\displaystyle \var{SMBR(IEL)} = \var{SMBR(IEL)} + \rho^n_L
|\Omega_l| \left[ -C_{\varepsilon_2} \ \frac{2\,(\varepsilon^n_L)^2}{(R^n_{ii})_L} + C_{\varepsilon_1} \ \frac{\varepsilon^n_L}{(R^n_{ii})_L}\ (\mathcal{P}^n_{ii})_L \right] $
\item [$\Rightarrow$] $\displaystyle \var{ROVSDT(IEL)} = \var{ROVSDT(IEL)} + C_{\varepsilon_2} \ \rho^n_L \ |\Omega_l| \ \frac{2\,\varepsilon^n_L}{(R^n_{ii})_L}$
\end{itemize}

\item Appel de \fort{rijthe} pour le calcul des termes de gravit\'e $\mathcal{G}^n_{\varepsilon}$ et ajout dans \var{SMBR}.

$ \var{SMBR} = \var{SMBR} + \mathcal{G}^n_{\varepsilon}$\\
Ce calcul n'a lieu que si $\var{IGRARI()} = 1$.

\item Calcul de la diffusion de $\varepsilon$ \\
 Le terme $\dive \left[\mu\, \grad(\varepsilon) + \tens{A'}\,\grad(\varepsilon)
\right]$ est calcul\'e exactement de la m\^eme mani\`ere que pour les tensions
de Reynolds $R_{ij}$ en rempla\c cant $\tens{A}$ par $\tens{A'}$.

\item Appel de \fort{codits} pour la r\'esolution de l'\'equation de
convection/diffusion/termes sources de la variable principale $\varepsilon$. Le
r\'esultat $\varepsilon^{\,n+1}$ est stock\'e dans le tableau \var{RTP} des
variables mises \`a jour.
}
\end{itemize}

\etape{clippings finaux}
On passe enfin dans le sous-programme  \fort{clprij} pour faire un clipping \'eventuel
des variables $R^{\,n+1}_{ij}$ et $\varepsilon^{\,n+1}$. Le sous-programme
\fort{clprij} est appel\'e\footnote{L'option
$\var{ICLIP} = 1$ consiste en un clipping minimal des variables $R_{ii}$ et
$\varepsilon$ en prenant la valeur absolue de ces variables puisqu'elles ne
peuvent \^etre que positives.} avec $\var{ICLIP} = 2$ . Cette option
\footnote{Quand la valeur des grandeurs $R_{ii}$ ou $\varepsilon$ est
n\'egative, on la remplace par le minimum entre sa valeur absolue et (1,1)
fois la valeur obtenue au pas de temps pr\'ec\'edent.} contient l'option $\var{ICLIP} = 1$  et permet de v\'erifier l'in\'egalit\'e de Cauchy-Schwarz sur les grandeurs extra-diagonales du tenseur $\tens{R}$ (pour $i \neq j$, $|R_{ij}|^2 \le R_{ii} R_{jj}$).


%%%%%%%%%%%%%%%%%%%%%%%%%%%%%%%%%%
%%%%%%%%%%%%%%%%%%%%%%%%%%%%%%%%%%
\section{Points \`a traiter}
%%%%%%%%%%%%%%%%%%%%%%%%%%%%%%%%%%
%%%%%%%%%%%%%%%%%%%%%%%%%%%%%%%%%%
Sauf mention explicite, $\phi$ repr\'esentera une tension de Reynolds ou la dissipation turbulente ($\phi = R_{ij} \ \text{ou} \ \varepsilon$).

\begin{itemize}
\item {La vitesse utilis\'ee pour le calcul de la production est explicite. Est-ce qu'une implicitation peut am\'eliorer la pr\'ecision temporelle de $\phi$ \footnote{Cette remarque peut \^etre g\'en\'eralis\'ee. En effet, peut-on envisager d'actualiser les variables d\'ej\`a r\'esolues (sans r\'eactualiser les variables turbulentes apr\`es leur r\'esolution)? Ceci obligerait \`a modifier les sous-programmes tels que \fort{condli} qui sont appel\'es au d\'ebut de la boucle en temps.} ?}
\item {Dans quelle mesure le terme d'\'echo de paroi est-il valide ? En effet, ce terme est remis en question par certains auteurs.}
\item {On peut envisager la r\'esolution d'un syst\`eme hyperbolique pour les
tensions de Reynolds afin d'introduire un couplage avec le champ de vitesse.}
\item {Le flux au bord \var{VISCB} est annul\'e dans le sous-programme
\fort{vectds}. Peut-on envisager de mettre au bord la valeur de la variable
concern\'ee \`a la cellule de bord correspondant? De m\^eme, il faudrait se
pencher sur les hypoth\`eses sous-jacentes \`a l'annulation des contributions
aux bords de \var{VISCB} lors du calcul de : $$\left[ \tens{D}^n\,\left( \grad{R^{\,n}_{ij}} - (\grad R^{\,n}_{ij}\,.\,\vect{n}_{\,lm})\,\vect{n}_{\,lm}\right) \right]\,.\,\vect{n}_{\,lm}.$$}
\item {Un probl\`eme de pond\'eration appara\^\i t plus g\'en\'eralement. Si on prend la partie explicite de $\tens{D}\,\grad(\phi)$, la pond\'eration aux faces internes utilise le coefficient $\displaystyle\frac{1}{2}$ avec pond\'eration s\'epar\'ee de $\tens{D}$ et $\grad(\phi)$, alors que pour $\tens{E}\,\grad(\phi)$, on calcule d'abord ce terme aux cellules pour ensuite l'interpoler lin\'eairement aux faces \footnote{Cette interpolation se fait dans \fort{vectds} avec des coefficients de pond\'eration aux faces.}. Ceci donne donc deux types d'interpolations pour des termes de m\^eme nature.}
\item {On laisse la possibilit\'e dans \fort{visort} d'utiliser une moyenne
harmonique aux faces. Est-ce que ceci est valable puisque les interpolations
utilis\'ees lors du calcul de la partie explicite de $\tens{A}\,\grad{\phi}$
sont des moyennes arithm\'etiques ?}
\item {Les techniques adopt\'ees lors du clipping sont \`a revoir.}
\item {On utilise dans le cadre du mod\`ele $\displaystyle R_{ij}-\varepsilon $ une semi-implicitation de termes comme $\displaystyle \phi_{ij,1}$ ou $\displaystyle -\rho\,C_{\varepsilon_2}\,\frac{\varepsilon}{k}\,\varepsilon$. On peut envisager le m\^eme type d'implicitation dans \fort{turbke} m\^eme en pr\'esence du couplage $\displaystyle k-\varepsilon$.}
\item L'adoption d'une r\'esolution d\'ecoupl\'ee fait perdre l'invariance par rotation.
\item La formulation et l'implantation des conditions aux limites de paroi
devront \^etre v\'erifi\'ees. En effet, il semblerait que, dans certains cas, des ph\'enom\`enes
``oscillatoires'' apparaissent, sans qu'il soit ais\'e d'en d\'eterminer la cause.
\item L'implicitation partielle (du fait de la r\'esolution d\'ecoupl\'ee) des
conditions aux limites conduit souvent \`a des calculs instables. Il
conviendrait d'en conna\^\i tre la raison. L'implicitation partielle avait
\'et\'e mise en \oe uvre afin de tenter d'utiliser un pas de temps plus grand
dans le cas de jets axisym\'etriques en particulier.

\end{itemize}

%                      Code_Saturne version 1.3
%                      ------------------------
%
%     This file is part of the Code_Saturne Kernel, element of the
%     Code_Saturne CFD tool.
%
%     Copyright (C) 1998-2007 EDF S.A., France
%
%     contact: saturne-support@edf.fr
%
%     The Code_Saturne Kernel is free software; you can redistribute it
%     and/or modify it under the terms of the GNU General Public License
%     as published by the Free Software Foundation; either version 2 of
%     the License, or (at your option) any later version.
%
%     The Code_Saturne Kernel is distributed in the hope that it will be
%     useful, but WITHOUT ANY WARRANTY; without even the implied warranty
%     of MERCHANTABILITY or FITNESS FOR A PARTICULAR PURPOSE.  See the
%     GNU General Public License for more details.
%
%     You should have received a copy of the GNU General Public License
%     along with the Code_Saturne Kernel; if not, write to the
%     Free Software Foundation, Inc.,
%     51 Franklin St, Fifth Floor,
%     Boston, MA  02110-1301  USA
%
%-----------------------------------------------------------------------
%
\programme{vortex}
%
\vspace{1cm}
%%%%%%%%%%%%%%%%%%%%%%%%%%%%%%%%%%
%%%%%%%%%%%%%%%%%%%%%%%%%%%%%%%%%%
\section{Fonction}
%%%%%%%%%%%%%%%%%%%%%%%%%%%%%%%%%%
%%%%%%%%%%%%%%%%%%%%%%%%%%%%%%%%%%
Ce sous-programme est d�di� � la g�n�ration des conditions d'entr�e
turbulente utilis�es en LES.


La m�thode des vortex est bas�e sur une approche de tourbillons
ponctuels. L'id�e de la m�thode consiste � injecter des tourbillons 2D dans le
plan d'entr�e du calcul, puis � calculer le champ de vitesse induit par ces
tourbillons au centre des faces d'entr�e.

%                      Code_Saturne version 1.3
%                      ------------------------
%
%     This file is part of the Code_Saturne Kernel, element of the
%     Code_Saturne CFD tool.
% 
%     Copyright (C) 1998-2007 EDF S.A., France
%
%     contact: saturne-support@edf.fr
% 
%     The Code_Saturne Kernel is free software; you can redistribute it
%     and/or modify it under the terms of the GNU General Public License
%     as published by the Free Software Foundation; either version 2 of
%     the License, or (at your option) any later version.
% 
%     The Code_Saturne Kernel is distributed in the hope that it will be
%     useful, but WITHOUT ANY WARRANTY; without even the implied warranty
%     of MERCHANTABILITY or FITNESS FOR A PARTICULAR PURPOSE.  See the
%     GNU General Public License for more details.
% 
%     You should have received a copy of the GNU General Public License
%     along with the Code_Saturne Kernel; if not, write to the
%     Free Software Foundation, Inc.,
%     51 Franklin St, Fifth Floor,
%     Boston, MA  02110-1301  USA
%
%-----------------------------------------------------------------------
%
%%%%%%%%%%%%%%%%%%%%%%%%%%%%%%%%%%
%%%%%%%%%%%%%%%%%%%%%%%%%%%%%%%%%%
\section{Discr\'etisation}
%%%%%%%%%%%%%%%%%%%%%%%%%%%%%%%%%%
%%%%%%%%%%%%%%%%%%%%%%%%%%%%%%%%%%

Le terme convectif en $\dive(\underline{u} \otimes \rho\,\underline{u})$
introduit une non lin\'earit\'e et un couplage des composantes de la vitesse
$\vect{u}$ dans l'�quation (\ref{Base_Preduv_eqqdm}). Une lin\'earisation et un d\'ecouplage
des trois composantes de la 
vitesse sont r\'ealis\'es lors de la discr\'etisation de cette \'etape de
pr\'ediction.\\
En effet, soit :
\begin{equation}
\vect{\widetilde{u}}= \vect{u}^n + \delta \vect{u} 
\end{equation}
La contribution exacte du terme convectif \`a prendre en compte dans cette
\'etape de pr\'ediction serait :\\
\begin{equation}\label{Base_Preduv_Conv_exact}
\begin{array}{ll}
\dive(\vect{\widetilde{u}} \otimes \rho\,\vect{\widetilde{u}}) =
\dive(\vect{u}^{n} \otimes \rho\,\vect{u}^{n}) + \dive(\delta \vect{u} \otimes
\rho\,\vect{u}^{n}) +  \underbrace { \dive(\vect{u}^{n} \otimes
\rho\,\delta \vect{u})}_{\text {terme couplant lin\'eaire}} +  \underbrace { \dive(\delta \vect{u} \otimes
\rho\,\delta \vect{u})}_{\text {terme couplant et non lin\'eaire}}\\
\end{array} 
\end{equation}
Les deux derniers termes de l'expression (\ref{Base_Preduv_Conv_exact}) sont {\em a priori} n�glig�s
de mani�re � obtenir un syst\`eme en vitesse qui soit d\'ecoupl\'e et donc,
�viter l'inversion d'une matrice pouvant \^etre de tr\`es grande taille. Ces
deux termes peuvent n�anmoins �tre pris en compte de mani�re plus ou moins
approch�e par extrapolation explicite du flux de masse en $n+\theta_F$ (pour le
terme couplant lin�aire provenant de la convection de $\vect{u}^{n}$ par $\delta
\vect{u}$) et utilisation d'un point-fixe par sous it�ration sur le sous
programme \fort{navsto} (pour le terme non-lin�aire, en sp�cifiant $\var{NTERUP}>1$).

L'�quation (\ref{Base_Preduv_eqqdm}) est discr�tis�e au temps $n+\theta$ � l'aide d'un
$\theta$-sch�ma, et le tenseur des pertes de charges d�compos� en une partie
diagonale $\tens{K}_{d}$ et une extradiagonale $\tens{K}_{e}$ (soit
 $\tens{K}_{pdc}=\tens{K}_{d}+\tens{K}_{e}$).\\
$\bullet$ La pression est suppos�e connue en $n-1+\theta$ (d�calage temporel
pression-vitesse) et le gradient naturellement calcul� � cet instant.\\ 
$\bullet$ Les termes sources de viscosit� secondaire, de gradient transpos\'e,
ceux provenant du mod�le de turbulence\footnote{except� $\dive (\mu_t\ (\ggrad
\underline {u}))$}, $\rho\,\tens{K}_{\,e}\ \underline{u}$, $(\rho -\rho_0)
\underline {g}$ ainsi que $\underline{T}_{s}^{\,exp}$ et
$\Gamma\,\underline{u}_{\,i}$ sont pris de mani�re explicite au temps $n$, ou
extrapol�s suivant le sch�ma en temps choisi pour les propri�t�s physique et les
termes sources.\\ 
$\bullet$ Les termes sources $\underline{u}\,\,\dive (\rho\,\underline {u})$,
$\Gamma\,\,\underline{u}$, $T_{s}^{\,imp}\,\,\underline{u}$ et
$-\rho\,\tens{K}_{\,d}\,\,\underline{u}$ sont implicit�s est calcul�s �
l'instant $n+\theta$.\\ 
$\bullet$ Le terme de diffusion $\dive (\mu_{\,tot}\,\ggrad \underline{u})$ est
implicit� : la vitesse est prise � l'instant $n+\theta$ et la viscosit�
explicit�e ou extrapol�e.\\ 
$\bullet$ Enfin, le terme de convection en $\dive(\,\underline{u} \otimes
(\rho\underline{u})\,)$ est implicit� : la composante r�solue de la vitesse est
prise en $n+\theta$, et le flux de masse, explicit�, ou extrapol� en
$n+\theta_F$. 

Par souci de clart�, on suppose, en l'absence d'indication, que les propri�tes
physiques $\Phi$ ($\rho,\,\mu_{tot},\,...$) et le flux de masse
$(\rho\underline{u})$ sont pris respectivement aux instants $n+\theta_\Phi$ et
$n+\theta_F$, o� $\theta_\Phi$ et $\theta_F$ d�pendent des sch�mas en temps
sp�cifiquement utilis�s pour ces grandeurs\footnote{cf. \fort{introd}}. 

La discr�tisation temporelle de l'�quation (\ref{Base_Preduv_eqqdm}) s'�crit alors comme suit : 

\begin{equation}\label{Base_Preduv_eq_di1}
 \begin{array}{c}
\displaystyle \rho\,\ \frac{ \underline {\widetilde{u}}^{n+1} -\underline {u}^{n} }
{\Delta t} + \dive(\,\underline{\widetilde{u}}^{n+\theta} \otimes (\rho\underline{u})\,) -\dive
(\mu_{\,tot}\,\ggrad \underline{\widetilde{u}}^{n+\theta}) =
\\
\displaystyle
 - \grad p^{n-1+\theta} + \dive (\rho\,\underline {u})\,\underline{\widetilde{u}}^{n+\theta} +(\Gamma\,\underline{u}_{\,i})^{n+\theta_S}-\Gamma^n\,\,\underline{\widetilde{u}}^{n+\theta}
\\
\begin{array}{c}
\displaystyle
- \rho\,\tens{K}_{\,d}^{n}\,\,\underline{\widetilde{u}}^{n+\theta} - (\rho\,\tens{K}_{\,e}\ \underline{u})^{n+\theta_S} + (\underline{T}_{s}^{\,exp})^{\,n+\theta_S} + T_{s}^{\,imp}\,\,\underline{\widetilde{u}}^{n+\theta}
\\
\displaystyle
+[\dive (\mu_{\,tot}\,^t\ggrad \underline {u})]^{n+\theta_S}-\frac {2} {3}[\,\grad (\mu_{\,tot}\,\dive \underline {u})]^{n+\theta_S} + (\rho -\rho_0) \underline {g}
 - (\underline{turb})^{n+\theta_S}
\end{array}
\end{array}
\end{equation}
o\`u, par souci de simplification, on a pos\'e :
\begin{equation}
\mu_{\,tot}=
\begin{cases}
\mu+\mu_t & \text{pour les mod�les � viscosit� turbulente ou en LES}, \\
\mu & \text{pour les mod�les au second ordre ou en laminaire}
\end{cases} \ 
\end{equation}
\\
et :
\begin{equation}
\underline{turb}^{n}=
\begin{cases}
\displaystyle\frac {2}{3}\grad (\rho^{n}\,k^{n}) & \text{pour les mod�les � viscosit� turbulente}, \\
\dive(\rho^{n}\,\tens{R}^n) & \text{pour les mod�les au second ordre},\\
0 & \text{en laminaire ou en LES}\\
\end{cases}
\end{equation}
Par analogie avec l'�criture du $\theta$-sch�ma pour une variable scalaire, $\,
\underline {\widetilde{u}}^{n+\theta}$ est interpol�e � partir de la vitesse
pr�dite $\underline {\widetilde{u}}^{n+1}$ de la mani\`ere suivante\footnote{si
$\theta=1/2$, ou qu'une extrapolation est utilis�e, l'ordre 2 n'est obtenu que si
le pas de temps $\Delta t$ est uniforme en temps et en espace.}~: 
\begin{equation}
\underline {\widetilde{u}}^{n+\theta}=\theta\, \underline
{\widetilde{u}}^{n+1}+(1-\theta)\, \underline {u}^{n}\\ 
\end{equation}
Avec :
\begin{equation}
\left\{
\begin{array}{ll}
\theta = 1   & \text{Pour un sch\'ema de type Euler implicite d'ordre 1.}\\
\theta = 1/2 & \text{Pour un sch\'ema de type Cranck-Nicolson d'ordre 2.}\\
\end{array}
\right.
\end{equation}

L'�quation (\ref{Base_Preduv_eq_di1}) est alors r��crite sous la forme :

\begin{equation}\label{Base_Preduv_eq_di2}
\begin{array}{c}
\displaystyle \underbrace{\left(\frac{\rho}{\Delta t} -\theta \,\dive (\rho\,\underline {u}) +\theta \,\, \Gamma^n +
\theta \,\, \rho\,\tens{K}_{\,d}^n-\theta \,T_s^{\,imp} \right)}_{\displaystyle f_s^{imp}}\, (\underline {\,\widetilde{u}}^{n+1} -\underline {u}^{n})
\\
 +\, \theta\, \dive(\underline {\widetilde{u}}^{n+1} \otimes (\rho\underline{u}))-\, \theta\,\dive (\mu_{\,tot}\,\ggrad \underline {\widetilde{u}}^{n+1}) =
\\
-\,(1-\theta)\, \dive(\underline {u}^{n} \otimes (\rho\underline{u})) +\,(1-\theta)\,\dive (\mu_{\,tot}\,\ggrad \underline {u}^{n})
\\
f_s^{exp}\left\{
\begin{array}{c}
\displaystyle 
- \grad p^{n-1+\theta} + \dive (\rho\,\underline {u})\,\underline{u}^{n} +\,(\,\Gamma^{n}\,\underline{u}_{\,i}\,)^{n+\theta_S}- \Gamma^n\,\,\underline{u}^{n}
\\
\displaystyle
-(\,\rho\,\tens{K}_{\,e}\ \underline{u}\,)^{n+\theta_S} -\rho\,\tens{K}_{\,d}^n\ \underline{u}^{n}+ (\underline{T}_{s}^{\,exp})^{\,n+\theta_S} + T_s^{\,imp}\,\,\underline {u}^{n} 
\\
\displaystyle
+[\dive (\mu_{\,tot}\,^t\ggrad \underline {u}\,)]^{n+\theta_S}-\frac {2} {3}[\,\grad (\mu_{\,tot}\,\dive \underline {u}\,)]^{n+\theta_S} + (\rho -\rho_0) \underline {g}-(\underline{turb})^{n+\theta_S}
\end{array}
\right.
\end{array}
\end{equation}

d'o� l'�quation r�solue par le sous-programme \fort{codits} :
\begin{equation}\begin{array}{c}
\displaystyle
f_s^{\,imp}(\underline {\widetilde{u}}^{n+1}-\underline {u}^{n}) + \theta\, \dive(\underline{\widetilde{u}}^{n+1} \otimes (\rho
\underline{u})) - \theta\,\dive (\,\mu_{\,tot}\,\ggrad \underline{\widetilde{u}}^{n+1}) = 
\\\\
\displaystyle
-(1-\theta)\,\dive(\underline{u}^{n} \otimes (\rho \underline{u}))+(1-\theta)\,\dive (\,\mu_{\,tot}\,\ggrad \underline{u}^{n})
+ \underline{f}_{\,s}^{\,exp}
\end{array}
\end{equation}
La m\'ethode de discr\'etisation spatiale est d\'evelopp\'ee dans le sous-programme \fort{codits}.\\



\minititre{Remarques :}
{\tiny$\blacksquare$} Dans le cas standard sans extrapolation, le terme
$-\,T_s^{\,imp}$ n'est ajout� � $f_s^{\,imp}$ que s'il est positif afin de ne
pas affaiblir la dominance de la diagonale de la matrice � inverser.\\ 
{\tiny$\blacksquare$} Si une extrapolation est utilis�e, par contre,
$\,T_s^{\,imp}$ est ajout� � $f_s^{\,imp}$ quel que soit son signe. En effet, l'id�e intuitive qui
consiste � prendre~: 
\begin{equation}
\begin{cases}
\displaystyle
(\underline{T}_{s}^{\,exp} + T_{s}^{\,imp}\,\underline {u})^{\,n+\theta_S} &
\text{si } T_{s}^{\,imp} > 0\\ 
\displaystyle
(\underline{T}_{s}^{\,exp})^{\,n+\theta_S} + T_{s}^{\,imp}\,\underline{u}^{n+\theta} &\text{sinon}\\
\end{cases}
\end{equation} 
aboutit � une incoh�rence dans le traitement si $T_s^{imp}$ change de signe
entre deux pas de temps.\\ 
{\tiny$\blacksquare$} la partie diagonale $\tens{K}_{\,d}$ du terme
de perte de charge est utilis�e dans $f_s^{\,imp}$. En fait, pour \^etre rigoureux,
il faudrait ne retenir que les contributions positives (point signal\'e dans le
sous-programme utilisateur associ\'e \fort{uskpdc}). Cette prise en compte sera \`a am\'eliorer.\\
{\tiny$\blacksquare$} Le terme $\theta\,\Gamma^{n}-\theta\,\dive
(\rho\,\underline {u})$ ne pose pas de probl�me pour la 
dominance de la diagonale de la matrice car il est exactement compens� par le
terme de convection (cf. \fort{covofi}). 


%                      Code_Saturne version 1.3
%                      ------------------------
%
%     This file is part of the Code_Saturne Kernel, element of the
%     Code_Saturne CFD tool.
%
%     Copyright (C) 1998-2007 EDF S.A., France
%
%     contact: saturne-support@edf.fr
%
%     The Code_Saturne Kernel is free software; you can redistribute it
%     and/or modify it under the terms of the GNU General Public License
%     as published by the Free Software Foundation; either version 2 of
%     the License, or (at your option) any later version.
%
%     The Code_Saturne Kernel is distributed in the hope that it will be
%     useful, but WITHOUT ANY WARRANTY; without even the implied warranty
%     of MERCHANTABILITY or FITNESS FOR A PARTICULAR PURPOSE.  See the
%     GNU General Public License for more details.
%
%     You should have received a copy of the GNU General Public License
%     along with the Code_Saturne Kernel; if not, write to the
%     Free Software Foundation, Inc.,
%     51 Franklin St, Fifth Floor,
%     Boston, MA  02110-1301  USA
%
%-----------------------------------------------------------------------
%

%%%%%%%%%%%%%%%%%%%%%%%%%%%%%%%%%%
%%%%%%%%%%%%%%%%%%%%%%%%%%%%%%%%%%
\section{Mise en \oe uvre}
%%%%%%%%%%%%%%%%%%%%%%%%%%%%%%%%%%
%%%%%%%%%%%%%%%%%%%%%%%%%%%%%%%%%%
La num\'ero de la phase trait\'ee fait partie des arguments de \fort{turrij}. On
omettra volontairement de le pr\'eciser dans ce qui suit, on indiquera par $(\ )$ la
notion de tableau s'y rattachant.

\etape{Calcul des termes de production $\tens{\mathcal{P}}$}
\begin{itemize}
\item [$\star$] Initialisation \`a z\'ero du tableau \var{PRODUC} dimensionn\'e \`a $\var{NCEL}\times 6$.
\item [$\star$] On appelle trois fois \fort{grdcel} pour calculer les gradients des composantes de la vitesse $u$, $v$ et
$w$ prises au temps $n$.

Au final, on a :\\
$\displaystyle
\begin{array} {ll}
\var{PRODUC(1,IEL)} = & \displaystyle - 2 \left[ R_{11}^{\,n} \frac{\partial u^{\,n}} {\partial x} +R_{12}^{\,n} \frac{\partial u^{\,n}} {\partial y}+R_{13}^{\,n} \frac{\partial u^{\,n}} {\partial z} \right] \text{        (production de $R_{11}^{\,n}$)}\\
\var{PRODUC(2,IEL)} = & \displaystyle - 2 \left[ R_{12}^{\,n} \frac{\partial v^{\,n}} {\partial x} +R_{22}^{\,n} \frac{\partial v^{\,n}} {\partial y}+R_{23}^{\,n} \frac{\partial v^{\,n}} {\partial z} \right] \text{        (production de $R_{22}^{\,n}$)}\\
\var{PRODUC(3,IEL)} = & \displaystyle - 2 \left[ R_{13}^{\,n} \frac{\partial w^{\,n}} {\partial x} +R_{23}^{\,n} \frac{\partial w^{\,n}} {\partial y}+R_{33}^{\,n} \frac{\partial w^{\,n}} {\partial z} \right] \text{        (production de $R_{33}^{\,n}$)}\\
\var{PRODUC(4,IEL)} = & \displaystyle - \left[ R_{12}^{\,n} \frac{\partial u^{\,n}} {\partial x} +R_{22}^{\,n} \frac{\partial u^{\,n}} {\partial y}+R_{23}^{\,n} \frac{\partial u^{\,n}} {\partial z} \right] \\
& \displaystyle - \left[ R_{11}^{\,n} \frac{\partial v^{\,n}} {\partial x} +R_{12}^{\,n} \frac{\partial v^{\,n}} {\partial y}+R_{13}^{\,n} \frac{\partial v^{\,n}} {\partial z} \right] \text{        (production de $R_{12}^{\,n}$)} \\
\var{PRODUC(5,IEL)} = & \displaystyle - \left[ R_{13}^{\,n} \frac{\partial u^{\,n}} {\partial x} +R_{23}^{\,n} \frac{\partial u^{\,n}} {\partial y}+R_{33}^{\,n} \frac{\partial u^{\,n}} {\partial z} \right] \\
& \displaystyle - \left[ R_{11}^{\,n} \frac{\partial w^{\,n}} {\partial x} +R_{12}^{\,n} \frac{\partial w^{\,n}} {\partial y}+R_{13}^{\,n} \frac{\partial w^{\,n}} {\partial z} \right] \text{        (production de $R_{13}^{\,n}$)} \\
\var{PRODUC(6,IEL)} = & \displaystyle - \left[ R_{13}^{\,n} \frac{\partial v^{\,n}} {\partial x} +R_{23}^{\,n} \frac{\partial v^{\,n}} {\partial y}+R_{33}^{\,n} \frac{\partial v^{\,n}} {\partial z} \right] \\
& \displaystyle - \left[ R_{12}^{\,n} \frac{\partial w^{\,n}} {\partial x} +R_{22}^{\,n} \frac{\partial w^{\,n}} {\partial y}+R_{23}^{\,n} \frac{\partial w^{\,n}} {\partial z} \right]  \text{        (production de $R_{23}^{\,n}$)}
\end{array}
$
\end{itemize}

\etape{Calcul du gradient de la masse volumique $\rho^n$ prise au d\'ebut du pas
de temps courant\footnote{{\it i.e.} calcul\'ee \`a partir des
variables du pas de temps pr\'ec\'edent $n$ si n\'ecessaire.} $(n+1)$}
Ce calcul n'a lieu que si les termes de gravit\'e doivent \^etre pris en compte
($\var{IGRARI()} =1$).
\begin{itemize}
\item [$\star$] Appel de \fort{grdcel}  pour calculer le gradient de $\rho^n$
dans les trois directions de l'espace. Les conditions aux limites sur $\rho^n$
sont des conditions de Dirichlet puisque la valeur de $\rho^n$ aux faces de bord
$ik$ (variable \var{IFAC}) est connue et vaut $\rho_{\,b_{\,ik}}$. Pour \'ecrire les conditions aux limites
sous la forme habituelle, $$\rho_{\,b_{\,ik}} = \var{COEFA} + \var{COEFB}
\,\rho^n_{\,I'}$$ on pose alors $\var{COEFA}=
\var{PROPCE(IFAC,IPPROB(IROM(IPHAS)))}$ et $\var{COEFB} = \var{VISCB} = 0$.\\
$\var{PROPCE(1,IPPROB(IROM(IPHAS)))}$ (resp.$\var{VISCB}$) est utilis\'e en lieu
et place de l'habituel \var{COEFA} ($\var{COEFB}$), lors de l'appel \`a \fort{grdcel}.\\
On a donc :\\
$\displaystyle \var{GRAROX}= \frac{\partial \rho^n}{\partial x}\ $,$\displaystyle \ \var{GRAROY}= \frac{\partial
\rho^n}{\partial y}$ et $
\displaystyle \ \var{GRAROZ}= \frac{\partial \rho^n}{\partial z}\ $.

\end{itemize}

Le gradient de $\rho^n$ servira \`a calculer les termes de production par effets de gravit\'e si ces derniers sont pris en compte.

\etape{Boucle \var{ISOU} de $1$ \`a $6$ sur les tensions de Reynolds}
Pour $\var{ISOU} = 1,2,3,4,5,6$, on r\'esout respectivement et dans
l'ordre  les
\'equations de $R_{11}$, $R_{22}$, $R_{33}$, $R_{12}$, $R_{13}$ et $R_{23}$ par
l'appel au sous-programme \fort{resrij}.\\
La r\'esolution se fait par incr\'ement $\delta {R}_{ij}^{\,n+1,k+1}$ , en utilisant la m\^eme m\'ethode que
celle d\'ecrite dans le sous-programme \fort{codits}. On adopte ici les m\^emes notations.
\var{SMBR} est le second membre du syst\`eme \`a inverser, syst\`eme portant sur
les incr\'ements de la variable. \var{ROVSDT} repr\'esente la diagonale de la
matrice, hors convection/diffusion.\\
On va r\'esoudre l'\'equation (\ref{Base_Turrij_Eq_Temp_Rij}) sous forme incr\'ementale en
utilisant \fort{codits}, soit :
\begin{equation}\label{Base_Turrij_Eq_Temp_deltaRij}
\begin{array}{ll}
&\displaystyle \underbrace{\left(\frac {\rho^n_L}{\Delta t^n}
+ \rho^n_L \,C_1\,\frac{\varepsilon^n_L}{k^n_L}(1-\frac{\delta_{ij}}{3})
 - m^n_{\,lm} + \Gamma_L\,+ max(-\alpha^n_{R_{ij}},0)\right)\,|\Omega_l|}
_{\text {$\var{ROVSDT}$ contribuant
\`a la diagonale de la matrice simplifi\'ee de \fort{matrix}}}\,(\delta{R}_{ij}^{\,n+1,p+1})_{\,L}\\\\
&  \underbrace{+\sum\limits_{m\in Vois(l)}\displaystyle \left[
 m^n_{\,lm} \delta R_{ij,\,f_{\,lm}}^{\,n+1,p+1}
- (\mu^n_{\,lm} + \gamma^n_{\,lm})\
\frac{({\delta R}_{ij}^{\,n+1,p+1})_{M}-({\delta R}_{ij}^{\,n+1,p+1})_{L})}{\overline{L'M'}}\,
S_{\,lm} \right]}_{\text { convection upwind pur et diffusion non reconstruite
relatives \`a la matrice simplifi\'ee de \fort{matrix}\footnotemark}} \\
% voir le texte de la footmark plus bas
&= - \displaystyle\frac {\rho^n_L}{\Delta t^n}\,\left(\,(R^{\,n+1,p}_{ij})_L - (R^{\,n}_{ij})_L\,\right)\\
&-\,\underbrace{\displaystyle\int_{\Omega_l} \left(
\dive\,[\,(\rho\,\vect{u})^n\,R^{\,n+1,p}_{ij} - (\mu^n\,+ \gamma^n\,)\,
\grad{R^{\,n+1,p}_{ij}}\,]\right)\,d\Omega}_{\text {convection et diffusion
trait\'ees par \fort{bilsc2}}}\\
&+\displaystyle \int_{\Omega_l} \left[\,\mathcal{P}^{\,n+1,p}_{ij} + \mathcal{G}^{\,n+1,p}_{ij}
- \displaystyle\rho^n \,C_1\,\frac{\varepsilon^n}{k^n}\left[R^{\,n+1,p}_{ij}-
\frac{2}{3}\,k^n\,\delta_{ij}\right] + \phi^{\,n+1,p}_{ij,2} +
\phi^{\,n+1,p}_{ij,w}\,\right]\, d\Omega \\
& + \displaystyle\int_{\Omega_l} \left[- \frac{2}{3} \rho^n \varepsilon^n \delta_{ij}
 + \Gamma\,(\,R^{\,in}_{ij} - R^{\,n+1,p}_{ij}\,) +
\alpha^n_{R_{ij}}\,R^{\,n+1,p}_{ij}+ \beta^n_{R_{ij}}\right]\, d\Omega\\
&+ \sum\limits_{m\in
Vois(l)}\displaystyle \left[\ \tens{E}^n\,\grad{R}^{\,n+1,p}_{ij} \right]_{\,lm}\,.\,\vect{n}_{\,lm}S_{\,lm}\\
&+ \sum\limits_{m\in Vois(l)}\displaystyle \left[\
\tens{D}^n\,\grad{R}^{\,n+1,p}_{ij} \right]_{\,lm}\,.\,\vect{n}_{\,lm}S_{\,lm}\\
&- \sum\limits_{m\in Vois(l)} \gamma^n_{\,lm} \left( \grad{R}^{\,n+1,p}_{ij}\,.\,\vect{n}_{\,lm} \right)  S_{\,lm}\\
&+ \sum\limits_{m\in Vois(l)}  m^n_{\,lm}\,(R^{\,n+1,p}_{ij})_L\\
\end{array}
\end{equation}
% si on ne fait pas comme ca, il n'apparait pas
\footnotetext[\thefootnote]{Si $\var{IRIJNU} = 1$, on remplace  $\mu^n_{\,lm}$ par $(\mu +
\mu_t)^n_{\,lm}$ dans l'expression de la diffusion non reconstruite
associ\'ee \`a la matrice simplifi\'ee de \fort{matrix} ($\mu_t$ d\'esigne la
viscosit\'e turbulente calcul\'ee comme en $k-\varepsilon$).}

o\`u on rappelle :\\
pour $n$ donn\'e entier positif, on d\'efinit la suite
 $({R}_{ij}^{\,n+1,p})_{p \in \grandN}$
 par :
\begin{equation}\notag
\left\{\begin{array}{l}
{R}_{ij}^{\,n+1,0} = {R}_{ij}^{\,n}\\
{R}_{ij}^{\,n+1,p+1} = {R}_{ij}^{\,n+1,p} + \delta{R}_{ij}^{\,n+1,p+1} \\
\end{array}\right.
\end{equation}
$(\delta{R}_{ij}^{\,n+1,p+1})_{\,L}$ d\'esigne la valeur sur l'\'el\'ement
$\Omega_l$ du $\text{$(\,p+1\,)$-i\`eme}$ incr\'ement de ${R}_{ij}^{\,n+1}$,
$ m^n_{\,lm}$ le flux de masse \`a l'instant $n$ \`a travers la face $lm$,
$\delta R_{ij,\,f_{\,lm}}^{\,n+1,p+1}$ vaut $({\delta
R}_{ij}^{\,n+1,p+1})_{L}$  si $ m^n_{\,lm} \geqslant 0$, $({\delta
R}_{ij}^{\,n+1,p+1})_{M}$ sinon,
$\mathcal{P}^{\,n+1,p}_{ij}$, $\phi^{\,n+1,p}_{ij,2}$, $\phi^{\,n+1,p}_{ij,w}$ les valeurs
des quantit\'es associ\'ees correspondant \`a l'incr\'ement
$(\delta{R}_{ij}^{\,n+1,p})$.\\



Tous ces termes sont calcul\'es comme suit :
\begin{itemize}
\item Terme de gauche de l'\'equation (\ref{Base_Turrij_Eq_Temp_deltaRij})\\
Dans \fort{resrij} est calcul\'ee la variable \var{ROVSDT}. Les autres
termes sont compl\'et\'es par \fort{codits}, lors de la construction de la matrice simplifi\'ee , {\it via} un
appel au sous-programme \fort{matrix}. La quantit\'e
 $(\mu^n_{\,lm} + \gamma^n_{\,lm})$ \`a la face $lm$ est calcul\'ee lors de l'appel \`a
\fort{visort}.\\
\item Second membre de l'\'equation (\ref{Base_Turrij_Eq_Temp_deltaRij})\\
Le premier terme non d\'etaill\'e est calcul\'e par le sous-programme
\fort{bilsc2}, qui applique le sch\'ema convectif choisi par l'utilisateur, qui
reconstruit ou non selon le souhait de l'utilisateur les gradients intervenants
dans la convection-diffusion.\\
Les termes sans accolade sont, eux, compl\`etement explicites et ajout\'es au fur et
\`a mesure dans \var{SMBR} pour former
l'expression $f^{\,exp}_s$ de \fort{codits}.
\end{itemize}
On d\'ecrit ci-dessous les \'etapes de \fort{resrij} :
\begin{itemize}

\item DELTIJ = 1, pour $\var{ISOU} \leqslant 3$ et DELTIJ = 0  Si $\var{ISOU} >
3$. Cette valeur repr\'esente le symbole de Kroeneker $\delta_{ij}$.

\item Initialisation \`a z\'ero de \var{SMBR} (tableau contenant le second
membre) et \var{ROVSDT} (tableau contenant la diagonale de la matrice sauf celle
relative \`a la contribution de la
diagonale des op\'erateurs de convection et de diffusion lin\'earis\'es
\footnote{qui correspondent aux sch\'emas convectif upwind pur et diffusif sans
reconstruction.}), tous deux de dimension $\var{NCEL}$.

\item Lecture et prise en compte des termes sources utilisateur pour la variable $R_{ij}$

Appel \`a \fort{ustsri} pour charger les termes sources utilisateurs. Ils sont
stock\'es comme suit. Pour la cellule $\Omega_l$ de centre $L$, repr\'esent\'ee par $\var{IEL}$, on a :\\
\begin{equation}\notag
\left\{\begin{array}{lll}
&\var{ROVSDT(IEL)}&= |\Omega_l| \ \alpha_{R_{ij}}\\
&\var{SMBR(IEL)}&=|\Omega_l| \ \beta_{R_{ij}}\\
\end{array}\right.
\end{equation}
On affecte alors les valeurs ad\'equates au second membre \var{SMBR} et \`a la
diagonale \var{ROVSDT} comme suit :
\begin{equation}\notag
\left\{\begin{array}{lll}
&\var{SMBR(IEL)} &= \var{SMBR(IEL)} +\ |\Omega_l| \ \alpha_{R_{ij}} \ (R^n_{ij})_L \\
&\var{ROVSDT(IEL)}&= \text{max }(-\ |\Omega_l| \ \alpha_{R_{ij}},0)\\
\end{array}\right.
\end{equation}
La valeur de $ \var{ROVSDT}$ est ainsi calcul\'ee pour des raisons de stabilit\'e
num\'erique. En effet, on ne rajoute sur la diagonale que les valeurs positives,
ce qui correspond physiquement \`a impliciter les termes de rappel uniquement.
\item{Calcul du terme source de masse  si $\Gamma_L > 0$}

Appel de \fort{catsma} et incr\'ementation si n\'ecessaire de \var{SMBR} et
\var{ROVSDT} {\it via} :\\
\begin{equation}\notag
\left\{\begin{array}{lll}
\displaystyle \var{SMBR(IEL)} = \var{SMBR(IEL)} + |\Omega_l| \ \Gamma_L \
\left[(R^{\,in}_{ij})_L - (R^{\,n}_{ij})_L \right] \\
\displaystyle \var{ROVSDT(IEL)}=\var{ROVSDT(IEL)} + |\Omega_l| \ \Gamma_L
\end{array}\right.
\end{equation}
\item Calcul du terme d'accumulation de masse et du terme instationnaire

On stocke $\displaystyle \var{W1}= \int_{\Omega_l}\dive\,(\rho\,\vect{u})\,d\Omega$
calcul\'e par \fort{divmas} \`a l'aide des flux de masse aux faces internes
$ m^n_{\,lm}=\sum\limits_{m\in Vois(l)}{(\rho \vect{u})_{\,lm}^n} \text{.}\,
\vect{S}_{\,lm} $ (tableau \var{FLUMAS}) et des flux de masse aux bords  $ m^n_{\,b_{lk}} = \sum\limits_{k\in{\gamma_b(l)}}{(\rho \vect{u})_{\,{b}_{lk}}^n} \text{.}\,
\vect{S}_{\,{b}_{lk}} $ (tableau \var{FLUMAB}).
On incr\'emente ensuite \var{SMBR} et \var{ROVSDT}.
\begin{equation}\notag
\left\{\begin{array}{lll}
&\var{SMBR(IEL)} &= \var{SMBR(IEL)} + \var{ICONV}\  (R^n_{ij})_L\,(\displaystyle
\int_{\Omega_l}\dive\,(\rho\,\vect{u})\ d\Omega) \\
&\var{ROVSDT(IEL)}& = \var{ROVSDT(IEL)} +  \var{ISTAT}\,\displaystyle
\frac{\rho^n_L \ |\Omega_l|}{\Delta t^n} -  \var{ICONV}\ (\displaystyle
\int_{\Omega_l}\dive\,(\rho\,\vect{u})\ d\Omega) \\
\end{array}\right.
\end{equation}
\item Calcul des termes sources de production, des termes $\displaystyle
\phi_{\,ij,1}+\phi_{\,ij,2}$ et de la dissipation~$\displaystyle-\frac{2}{3} \varepsilon\,\delta_{\,ij}$ :

On effectue une boucle d'indice \var{IEL} sur les cellules $\Omega_l$ de centre $L$ :
\begin{itemize}
\item [$\Rightarrow$] $\displaystyle \var{TRPROD}= \frac{1}{2} (\mathcal{P}^n_{ii})_L = \frac{1}{2} \left[ \var{PRODUC(1,IEL)} +  \var{PRODUC(2,IEL)} +  \var{PRODUC(3,IEL)} \right] $
\item [$\Rightarrow$] $\displaystyle \var{TRRIJ }= \frac{1}{2} (R^n_{ii})_L $
\item [$\Rightarrow$] $\displaystyle \var{SMBR(IEL)} =\ \var{SMBR(IEL)}\ +$\\
$\ \displaystyle\rho^n_L |\Omega_l| \left[ \displaystyle
\frac{2}{3}\,\delta_{\,ij} \left( \ \displaystyle \frac{ C_2}{2}\,(\mathcal{P}^n_{ii})_L\ +
(C_1-1)\ \varepsilon^n_L\, \right)\right.$\\
$ + \left.\ (1-C_2) \ \var{PRODUC(ISOU,IEL)} -
\displaystyle C_1\ \frac{2\,\varepsilon^n_L}{(R^n_{ii})_L}\ (R^n_{ij})_L \right]$
\item [$\Rightarrow$] $\displaystyle \var{ROVSDT(IEL)} = \var{ROVSDT(IEL)} +
\rho^n_L \ |\Omega_l| \ (- \displaystyle \frac{1}{3} \ \,\delta_{\,ij} + 1) \ C_1
\ \frac{2\ \varepsilon^n_L}{(R^n_{ii})_L}$
\end{itemize}
\item Appel de \fort{rijech} pour le calcul des termes d'\'echo de paroi
 $\phi^n_{ij,w}$ si $\var{IRIJEC()}=1$ et ajout dans \var{SMBR}.\\
$\var{SMBR} = \var{SMBR} + \phi^n_{ij,w}$\\
Suivant son mode de calcul (\var{ICDPAR}), la distance � la paroi est directement accessible
par \var{RA(IDIPAR+IEL-1)} (\var{|ICDPAR|} = 1) ou bien
est calcul\'ee \`a partir de $\var{IA(IIFAPA(IPHAS)+IEL - 1)}$,
qui donne pour l'\'el\'ement $\var{IEL}$ le num\'ero de la face de bord
paroi la plus  proche (\var{|ICDPAR|} = 2). Ces tableaux ont \'et\'e renseign\'e une fois pour toutes au
d\'ebut de calcul.

\item  Appel de \fort{rijthe} pour le calcul des termes de gravit\'e $\mathcal{G}^n_{ij}$ et ajout dans \var{SMBR}.

Ce calcul n'a lieu que si $\var{IGRARI()} = 1$.
$ \var{SMBR} = \var{SMBR} + \mathcal{G}^n_{ij}$
\item Calcul de la partie explicite du terme de diffusion
 $\dive{\,\left[\tens{A}\,\grad{R}^{\,n}_{ij}\right]}$, plus pr\'ecis\'ement
des contributions du terme extradiagonal pris aux faces purement internes
(remplissage du tableau \var{VISCF}), puis aux faces de bord (remplissage du
tableau \var{VISCB}).
\begin{itemize}
\item [$\star$] Appel de \fort{grdcel} pour le calcul du gradient de
$R^{\,n}_{ij}$ dans chaque direction. Ces gradients sont respectivement
stock\'es dans les tableaux de travail \var{W1}, \var{W2} et \var{W3}.

\item [$\star$] boucle d'indice \var{IEL} sur les cellules $\Omega_l$ de centre
$L$ pour le
calcul de $\tens{E}^n\,\grad{R}^{\,n}_{ij}$ aux cellules dans un premier temps :\\
\begin{itemize}
\item [$\Rightarrow$] $\displaystyle \var{TRRIJ}= \frac{1}{2} (R^{\,n}_{ii})_L $
\item [$\Rightarrow$] $\displaystyle \var{CSTRIJ} = \rho^n_L\ C_S \ \displaystyle\frac{(R^n_{ii})_L}{2\,\varepsilon^n_L}$
\item [$\Rightarrow$] $\displaystyle \var{W4(IEL)} = \rho^n_L\ C_S\
\displaystyle\frac{(R^n_{ii})_L}{2\,\varepsilon^n_L} \left[\,(R^{\,n}_{12})_L \ \var{W2(IEL)} +
(R^{\,n}_{13})_L \ \var{W3(IEL)}\,\right]$
\item [$\Rightarrow$] $\displaystyle \var{W5(IEL)} = \rho^n_L\ C_S\
\displaystyle\frac{(R^n_{ii})_L}{2\,\varepsilon^n_L} \left[\,(R^{\,n}_{12})_L \ \var{W1(IEL)} +
(R^{\,n}_{23})_L \ \var{W3(IEL)}\,\right]$
\item [$\Rightarrow$] $\displaystyle \var{W6(IEL)} = \rho^n_L\ C_S\
\displaystyle\frac{(R^n_{ii})_L}{2\,\varepsilon^n_L} \left[\,(R^{\,n}_{13})_L \ \var{W1(IEL)} + (R^{\,n}_{23})_L \ \var{W2(IEL)}\,\right]$
\end{itemize}



\item [$\star$] Appel de \fort{vectds}\footnote{Le r\'esultat est stock\'e dans
\var{VISCF} et \var{VISCB}. Dans \fort{vectds}, les valeurs aux faces internes
sont interpol\'ees lin\'eairement sans reconstruction et \var{VISCB} est mis \`a
z\'ero.} pour assembler $\displaystyle\left[ \tens{E}^n\,\grad{R}^{\,n}_{ij}
\right]\,.\,\vect{n}_{\,lm}S_{\,lm}$ aux faces $lm$.
\item [$\star$] Appel de \fort{divmas} pour calculer la divergence du flux d\'efini par \var{VISCF} et \var{VISCB}.
Le r\'esultat est stock\'e dans \var{W4}.\\
Ajout au second membre \var{SMBR}.\\
\var{SMBR} = \var{SMBR} + \var{W4}
\end{itemize}

A l'issue de cette \'etape, seule la partie extradiagonale de la diffusion prise
enti\`erement explicite~:
 $$\sum\limits_{m\in
Vois(l)}\left[\ \tens{E}^n\,\grad{R}^{\,n}_{ij} \right]_{\,lm}\,.\,\vect{n}_{\,lm}S_{\,lm}$$ a \'et\'e calcul\'ee.\\

\item Calcul de la partie diagonale du terme de diffusion\footnote{Seule la
composante normale  du  gradient de $R_{ij}$ aux faces sera implicite.} :\\
Comme on l'a d\'eja vu, une partie de cette contribution sera trait\'ee en
implicite (celle relative \`a la composante normale du gradient) et donc
ajout\'ee au second membre par \fort{bilsc2} ; l'autre
partie sera explicite et prise en compte dans $f_s^{\,exp}$.
\begin{itemize}
\item [$\star$] On effectue une boucle d'indice \var{IEL} sur les cellules
$\Omega_l$ de centre $L$ :
\begin{itemize}
\item [$\Rightarrow$] $\displaystyle \var{TRRIJ }= \frac{1}{2} (R^{\,n}_{ii})_L $
\item [$\Rightarrow$] $\displaystyle \var{CSTRIJ} = \rho^n_L \ C_S \ \frac{(R^{\,n}_{ii})_L}{2\,\varepsilon^n_L}$
\item [$\Rightarrow$] $\displaystyle \var{W4(IEL)} = \rho^n_L \ C_S \
\frac{(R^{\,n}_{ii})_L}{2\,\varepsilon^n_L} \ (R^{\,n}_{11})_L$
\item [$\Rightarrow$] $\displaystyle \var{W5(IEL)} = \rho^n_L \ C_S \ \frac{(R^{\,n}_{ii})_L}{2\,\varepsilon^n_L}\ (R^n_{22})_L$
\item [$\Rightarrow$] $\displaystyle \var{W6(IEL)} = \rho^n_L \ C_S \ \frac{(R^{\,n}_{ii})_L}{2\,\varepsilon^n_L} \ (R^n_{33})_L$
\end{itemize}

%\item Traitement du parall\'elisme et de la p\'eriodicit\'e.

\item [$\star$] On effectue une boucle d'indice \var{IFAC} sur les faces
purement internes $lm$ pour remplir le tableau \var{VISCF} :
\begin{itemize}
\item [$\Rightarrow$] Identification des cellules $\Omega_l$ et $\Omega_m$ de
centre respectif $L$ (variable \var{II}) et $M$ (variable \var{JJ}), se trouvant de chaque c\^ot\'e de la face
$lm$\footnote{La normale \`a la face est orient\'ee de L vers M.}.
\item [$\Rightarrow$] Calcul du carr\'e de la surface de la face. La valeur est
stock\'ee dans le tableau \var{SURFN2}.
\item [$\Rightarrow$] Interpolation du gradient de $R^{\,n}_{ij}$ \`a la face
$lm$ (gradient facette $\left[\grad{R}^{\,n}_{ij}\right]_{\,lm}$) :
\begin{equation}\notag
\left\{\begin{array}{ll}
\var{GRDPX} &= \displaystyle \frac{1}{2} \left(\var{W1(II)} + \var{W1(JJ)}
\right) \\
&\\
\var{GRDPY} &= \displaystyle \frac{1}{2} \left(\var{W2(II)} + \var{W2(JJ)} \right) \\
&\\
\var{GRDPZ} &= \displaystyle \frac{1}{2} \left(\var{W3(II)} + \var{W3(JJ)} \right)
\end{array}\right.
\end{equation}
\item [$\Rightarrow$] Calcul du gradient de $R^{\,n}_{ij}$ normal \`a la face
$lm$, $\left[\grad{R}^{\,n}_{ij}\right]_{\,lm}.\vect{n}_{\,lm}\,S_{\,lm}$.\\

$\displaystyle \var{GRDSN} =  \var{GRDPX} \ \var{SURFAC(1,IFAC)} + \var{GRDPY} \ \var{SURFAC(2,IFAC)} +  \var{GRDPZ} \ \var{SURFAC(3,IFAC)}$
$\var{SURFAC}$ est le vecteur surface de la face \var{IFAC}.


\item [$\Rightarrow$] calcul de
 $\left[\grad{R^{\,n}_{ij}} - (\grad
R^{\,n}_{ij}\,.\,\vect{n}_{\,lm})\vect{n}_{\,lm}\right]$, les vecteurs \'etant
calcul\'es \`a la face $lm$ :
\begin{equation}\notag
\left\{\begin{array}{lll}
&\displaystyle \var{GRDPX} &= \var{GRDPX} - \displaystyle\frac{\var{GRDSN}}{\var{SURFN2}} \ \var{SURFAC(1,IFAC)}\\
&&\\
&\displaystyle \var{GRDPY} &= \var{GRDPY} - \displaystyle\frac{\var{GRDSN}}{\var{SURFN2}} \ \var{SURFAC(2,IFAC)} \\
&&\\
&\displaystyle \var{GRDPZ} &= \var{GRDPZ} - \displaystyle\frac{\var{GRDSN}}{\var{SURFN2}} \ \var{SURFAC(3,IFAC)}
\end{array}\right.
\end{equation}
\item [$\Rightarrow$] finalisation du calcul de l'expression totalement
explicite
 $$\left[ \tens{D}^n\,\left( \grad{R^{\,n}_{ij}} - (\grad R^{\,n}_{ij}\,.\,\vect{n}_{\,lm})\,\vect{n}_{\,lm}\right) \right]\,.\,\vect{n}_{\,lm}$$
\begin{equation}\notag
\begin{array} {ll}
\displaystyle \var{VISCF} = &
 \displaystyle\frac{1}{2} (\ \var{W4(II)} +\ \var{W4(JJ)}) \ \var{GRDPX} \
\var{SURFAC(1,IFAC)})\ + \\
&\\
&  \displaystyle\frac{1}{2} (\ \var{W5(II)} +\ \var{W5(JJ)}) \ \var{GRDPY} \
\var{SURFAC(2,IFAC)})\ + \\
&\\
&  \displaystyle\frac{1}{2} (\ \var{W6(II)} +\ \var{W6(JJ)}) \ \var{GRDPZ} \ \var{SURFAC(3,IFAC)})
\end{array}
\end{equation}
\end{itemize}

\item [$\star$] Mise \`a z\'ero du tableau \var{VISCB}.

\item [$\star$] Appel de \fort{divmas} pour calculer la divergence de~:
 $$\tens{D}^{\,n}\,\left( \grad{R^{\,n}_{ij}} - (\grad R^{\,n}_{ij}\,.\,\vect{n}_{\,lm})\vect{n}_{\,lm}\right)$$ d\'efini aux faces dans \var{VISCF} et \var{VISCB}.

Le r\'esultat est stock\'e dans le tableau \var{W1}.\\
Ajout au second membre \var{SMBR}.\\
$\var{SMBR} = \var{SMBR} + \var{W1}$
\end{itemize}
\item Calcul de la viscosit\'e orthotrope $\gamma^n_{\,lm}$ \`a la face $lm$ de la variable principale
$R^{\,n}_{ij}$\\
Ce calcul permet au sous-programme \fort{codits} de compl\'eter le second membre
\var{SMBR} par :
\begin{equation}
\begin{array} {ll}
& \sum\limits_{m\in Vois(l)}
\mu^n_{\,lm}\,\left(\grad{R}^{\,n}_{ij}\,.\,\vect{n}_{\,lm}\right)S_{\,lm}
 + \sum\limits_{m\in Vois(l)} \left(\grad{R}^{\,n}_{ij}
\,.\,\vect{n}_{\,lm}\right)\left[\tens{D}^{\,n}\,\vect{n}_{\,lm}\right]_{\,lm}\,.\,\vect{n}_{\,lm}\
S_{\,lm}\\
& = \sum\limits_{m\in Vois(l)}(\,\mu^n_{\,lm}\, + \,\gamma^n_{\,lm}\,)\,\left(\grad{R}^{\,n}_{ij}\,.\,\vect{n}_{\,lm}\right)S_{\,lm}
\end{array}
\end{equation}
sans pr\'eciser la nature de la face $lm$, {\it via} l'appel \`a \fort{bilsc2}  et de disposer de la quantit\'e
$(\mu^n_{\,lm}\, + \gamma^n_{\,lm})$ pour construire sa
matrice simplifi\'ee.\\
\begin{itemize}
\item [$\star$] On effectue une boucle d'indice \var{IEL} sur les cellules
$\Omega_l$ :
\begin{itemize}
\item [$\Rightarrow$] $\displaystyle \var{TRRIJ }= \frac{1}{2} (R^{\,n}_{ii})_L $
\item [$\Rightarrow$] $\displaystyle \var{RCSTE} = \rho^n_L \ C_S \ \frac{ (R^{\,n}_{ii})_L}{2\,\varepsilon^n_L} $
\item [$\Rightarrow$] $\displaystyle \var{W1(IEL)} = \mu^n +\rho^n_L \ C_S \ \frac{
(R^{\,n}_{ii})_L}{2\,\varepsilon^n_L}\ (R^n_{11})_L$
\item [$\Rightarrow$] $\displaystyle \var{W2(IEL)} = \mu^n + \rho^n_L \ C_S \ \frac{ (R^{\,n}_{ii})_L}{2\,\varepsilon^n_L}\ (R^n_{22})_L$
\item [$\Rightarrow$] $\displaystyle \var{W3(IEL)} = \mu^n + \rho^n_L \ C_S \ \frac{ (R^{\,n}_{ii})_L}{2\,\varepsilon^n_L}\ (R^n_{33})_L$
\end{itemize}

\item [$\star$] Appel de \fort{visort} pour calculer la viscosit\'e orthotrope
\footnote{Comme dans le sous-programme \fort{viscfa}, on multiplie la viscosit\'e par
$\displaystyle \frac{S_{\,lm}}{\overline{L'M'}}$, o\`u $S_{\,lm}$ et
$\overline{L'M'}$ repr\'esentent respectivement la surface de la face $lm$ et la
mesure alg\'ebrique du segment reliant les projections des centres des cellules
voisines sur la normale \`a la face. On garde dans ce sous-programme  la possibilit\'e d'interpoler la viscosit\'e aux cellules lin\'eairement ou d'utiliser une moyenne harmonique. La viscosit\'e au bord est celle de la cellule de bord correspondante.}
$ \gamma^n_{\,lm} = (\tens{D}^{\,n}\,\vect{n}_{\,lm}).\vect{n}_{\,lm}$ aux faces $lm$

Le r\'esultat est stock\'e dans les tableaux \var{VISCF} et \var{VISCB}.
\end{itemize}

\item appel de \fort{codits} pour la r\'esolution de l'\'equation de
convection/diffusion/termes sources de la variable $R_{ij}$. Le terme source est
\var{SMBR}, la viscosit\'e \var{VISCF} aux faces purement internes (
resp. \var{VISCB} aux faces de bord) et \var{FLUMAS} le flux de masse interne
 ( resp. \var{FLUMAB} flux de masse au bord). Le r\'esultat est la variable $R_{ij}$ au temps
$n+1$, donc $R^{\,n+1}_{ij}$. Elle est stock\'ee dans le tableau \var{RTP} des
variables mises \`a jour.

\end{itemize}

\etape{Appel de \fort{reseps} pour la r\'esolution de la variable $\varepsilon$}

On d\'ecrit ci-dessous le sous-programme \fort{reseps}, les commentaires du sous-programme \fort{resrij} ne seront pas r\'ep\'et\'es puisque les deux sous-programmes ne diff\`erent que par leurs termes sources.

\begin{itemize}
\item Initialisation \`a z\'ero de \var{SMBR} et \var{ROVSDT}.

\item{Lecture et prise en compte des termes sources utilisateur pour la variable $\varepsilon$ :}

Appel de \fort{ustsri} pour charger les termes sources utilisateurs. Ils sont
stock\'es dans les tableaux suivants :\\
pour la cellule $\Omega_l$ repr\'esent\'ee par $\var{IEL}$ de centre $L$, on a :
\begin{equation}\notag
\left\{\begin{array}{lll}
&\var{ROVSDT(IEL)}&= |\Omega_l| \ \alpha_{\varepsilon}\\
&\var{SMBR(IEL)}&=|\Omega_l| \ \beta_{\varepsilon}\\
\end{array}\right.
\end{equation}
On affecte alors les valeurs ad\'equates au second membre \var{SMBR} et \`a la
diagonale \var{ROVSDT} comme suit :
\begin{equation}\notag
\left\{\begin{array}{lll}
&\var{SMBR(IEL)} &= \var{SMBR(IEL)} +\ |\Omega_l| \ \alpha_{\,\varepsilon} \
\varepsilon^n_L \\
&\var{ROVSDT(IEL)}&= \text{max }(-\ |\Omega_l| \ \alpha_{\,\varepsilon},0)\\
\end{array}\right.
\end{equation}

\item{Calcul du terme source de masse si $\Gamma_L > 0$ :
\begin{equation}\notag
\left\{\begin{array}{lll}
&\displaystyle \var{SMBR(IEL)} = \var{SMBR(IEL)} + |\Omega_l| \ \Gamma_L \
(\varepsilon^{\,in}_L -\varepsilon^n_L) \\
&\displaystyle \var{ROVSDT(IEL)}= \var{ROVSDT(IEL)} + |\Omega_l| \ \Gamma_L
\end{array}\right.
\end{equation}
\item Calcul du terme d'accumulation de masse et du terme instationnaire \\
On stocke $\displaystyle \var{W1}= \int_{\Omega_l}\dive\,(\rho\,\vect{u})\,d\Omega$
calcul\'e par \fort{divmas} \`a l'aide des flux de masse internes et aux bords.\\
On incr\'emente ensuite \var{SMBR} et \var{ROVSDT}.
\begin{equation}\notag
\left\{\begin{array}{lll}
&\var{SMBR(IEL)} &= \var{SMBR(IEL)} + \var{ICONV}\ \varepsilon^n_L\,(\displaystyle
\int_{\Omega_l}\dive\,(\rho\,\vect{u})\ d\Omega) \\
&\var{ROVSDT(IEL)}& = \var{ROVSDT(IEL)} +  \var{ISTAT}\,\displaystyle
\frac{\rho^n_L \ |\Omega_l|}{\Delta t^n} -  \var{ICONV}\ (\displaystyle
\int_{\Omega_l}\dive\,(\rho\,\vect{u})\ d\Omega) \\
\end{array}\right.
\end{equation}

\item Traitement du terme de production
 $\displaystyle \rho\,C_{\varepsilon_1}\,\frac{\varepsilon}{k}\,\mathcal{P}$
 et du terme de dissipation $-\,\displaystyle \rho\,C_{\varepsilon_2}\,\frac{\varepsilon}{k}\,\varepsilon$ \\
pour cela, on effectue une boucle d'indice \var{IEL} sur les cellules $\Omega_l$
de centre $L$ :
\begin{itemize}
\item [$\Rightarrow$] $\displaystyle \var{TRPROD}= \frac{1}{2} (\mathcal{P}^n_{ii})_L = \frac{1}{2} \left[ \var{PRODUC(1,IEL)} +  \var{PRODUC(2,IEL)} +  \var{PRODUC(3,IEL)} \right] $
\item [$\Rightarrow$] $\displaystyle \var{TRRIJ }= \frac{1}{2} (R^n_{ii})_L $
\item [$\Rightarrow$] $\displaystyle \var{SMBR(IEL)} = \var{SMBR(IEL)} + \rho^n_L
|\Omega_l| \left[ -C_{\varepsilon_2} \ \frac{2\,(\varepsilon^n_L)^2}{(R^n_{ii})_L} + C_{\varepsilon_1} \ \frac{\varepsilon^n_L}{(R^n_{ii})_L}\ (\mathcal{P}^n_{ii})_L \right] $
\item [$\Rightarrow$] $\displaystyle \var{ROVSDT(IEL)} = \var{ROVSDT(IEL)} + C_{\varepsilon_2} \ \rho^n_L \ |\Omega_l| \ \frac{2\,\varepsilon^n_L}{(R^n_{ii})_L}$
\end{itemize}

\item Appel de \fort{rijthe} pour le calcul des termes de gravit\'e $\mathcal{G}^n_{\varepsilon}$ et ajout dans \var{SMBR}.

$ \var{SMBR} = \var{SMBR} + \mathcal{G}^n_{\varepsilon}$\\
Ce calcul n'a lieu que si $\var{IGRARI()} = 1$.

\item Calcul de la diffusion de $\varepsilon$ \\
 Le terme $\dive \left[\mu\, \grad(\varepsilon) + \tens{A'}\,\grad(\varepsilon)
\right]$ est calcul\'e exactement de la m\^eme mani\`ere que pour les tensions
de Reynolds $R_{ij}$ en rempla\c cant $\tens{A}$ par $\tens{A'}$.

\item Appel de \fort{codits} pour la r\'esolution de l'\'equation de
convection/diffusion/termes sources de la variable principale $\varepsilon$. Le
r\'esultat $\varepsilon^{\,n+1}$ est stock\'e dans le tableau \var{RTP} des
variables mises \`a jour.
}
\end{itemize}

\etape{clippings finaux}
On passe enfin dans le sous-programme  \fort{clprij} pour faire un clipping \'eventuel
des variables $R^{\,n+1}_{ij}$ et $\varepsilon^{\,n+1}$. Le sous-programme
\fort{clprij} est appel\'e\footnote{L'option
$\var{ICLIP} = 1$ consiste en un clipping minimal des variables $R_{ii}$ et
$\varepsilon$ en prenant la valeur absolue de ces variables puisqu'elles ne
peuvent \^etre que positives.} avec $\var{ICLIP} = 2$ . Cette option
\footnote{Quand la valeur des grandeurs $R_{ii}$ ou $\varepsilon$ est
n\'egative, on la remplace par le minimum entre sa valeur absolue et (1,1)
fois la valeur obtenue au pas de temps pr\'ec\'edent.} contient l'option $\var{ICLIP} = 1$  et permet de v\'erifier l'in\'egalit\'e de Cauchy-Schwarz sur les grandeurs extra-diagonales du tenseur $\tens{R}$ (pour $i \neq j$, $|R_{ij}|^2 \le R_{ii} R_{jj}$).


%%%%%%%%%%%%%%%%%%%%%%%%%%%%%%%%%%
%%%%%%%%%%%%%%%%%%%%%%%%%%%%%%%%%%
\section{Points \`a traiter}
%%%%%%%%%%%%%%%%%%%%%%%%%%%%%%%%%%
%%%%%%%%%%%%%%%%%%%%%%%%%%%%%%%%%%
Sauf mention explicite, $\phi$ repr\'esentera une tension de Reynolds ou la dissipation turbulente ($\phi = R_{ij} \ \text{ou} \ \varepsilon$).

\begin{itemize}
\item {La vitesse utilis\'ee pour le calcul de la production est explicite. Est-ce qu'une implicitation peut am\'eliorer la pr\'ecision temporelle de $\phi$ \footnote{Cette remarque peut \^etre g\'en\'eralis\'ee. En effet, peut-on envisager d'actualiser les variables d\'ej\`a r\'esolues (sans r\'eactualiser les variables turbulentes apr\`es leur r\'esolution)? Ceci obligerait \`a modifier les sous-programmes tels que \fort{condli} qui sont appel\'es au d\'ebut de la boucle en temps.} ?}
\item {Dans quelle mesure le terme d'\'echo de paroi est-il valide ? En effet, ce terme est remis en question par certains auteurs.}
\item {On peut envisager la r\'esolution d'un syst\`eme hyperbolique pour les
tensions de Reynolds afin d'introduire un couplage avec le champ de vitesse.}
\item {Le flux au bord \var{VISCB} est annul\'e dans le sous-programme
\fort{vectds}. Peut-on envisager de mettre au bord la valeur de la variable
concern\'ee \`a la cellule de bord correspondant? De m\^eme, il faudrait se
pencher sur les hypoth\`eses sous-jacentes \`a l'annulation des contributions
aux bords de \var{VISCB} lors du calcul de : $$\left[ \tens{D}^n\,\left( \grad{R^{\,n}_{ij}} - (\grad R^{\,n}_{ij}\,.\,\vect{n}_{\,lm})\,\vect{n}_{\,lm}\right) \right]\,.\,\vect{n}_{\,lm}.$$}
\item {Un probl\`eme de pond\'eration appara\^\i t plus g\'en\'eralement. Si on prend la partie explicite de $\tens{D}\,\grad(\phi)$, la pond\'eration aux faces internes utilise le coefficient $\displaystyle\frac{1}{2}$ avec pond\'eration s\'epar\'ee de $\tens{D}$ et $\grad(\phi)$, alors que pour $\tens{E}\,\grad(\phi)$, on calcule d'abord ce terme aux cellules pour ensuite l'interpoler lin\'eairement aux faces \footnote{Cette interpolation se fait dans \fort{vectds} avec des coefficients de pond\'eration aux faces.}. Ceci donne donc deux types d'interpolations pour des termes de m\^eme nature.}
\item {On laisse la possibilit\'e dans \fort{visort} d'utiliser une moyenne
harmonique aux faces. Est-ce que ceci est valable puisque les interpolations
utilis\'ees lors du calcul de la partie explicite de $\tens{A}\,\grad{\phi}$
sont des moyennes arithm\'etiques ?}
\item {Les techniques adopt\'ees lors du clipping sont \`a revoir.}
\item {On utilise dans le cadre du mod\`ele $\displaystyle R_{ij}-\varepsilon $ une semi-implicitation de termes comme $\displaystyle \phi_{ij,1}$ ou $\displaystyle -\rho\,C_{\varepsilon_2}\,\frac{\varepsilon}{k}\,\varepsilon$. On peut envisager le m\^eme type d'implicitation dans \fort{turbke} m\^eme en pr\'esence du couplage $\displaystyle k-\varepsilon$.}
\item L'adoption d'une r\'esolution d\'ecoupl\'ee fait perdre l'invariance par rotation.
\item La formulation et l'implantation des conditions aux limites de paroi
devront \^etre v\'erifi\'ees. En effet, il semblerait que, dans certains cas, des ph\'enom\`enes
``oscillatoires'' apparaissent, sans qu'il soit ais\'e d'en d\'eterminer la cause.
\item L'implicitation partielle (du fait de la r\'esolution d\'ecoupl\'ee) des
conditions aux limites conduit souvent \`a des calculs instables. Il
conviendrait d'en conna\^\i tre la raison. L'implicitation partielle avait
\'et\'e mise en \oe uvre afin de tenter d'utiliser un pas de temps plus grand
dans le cas de jets axisym\'etriques en particulier.

\end{itemize}

%                      Code_Saturne version 1.3
%                      ------------------------
%
%     This file is part of the Code_Saturne Kernel, element of the
%     Code_Saturne CFD tool.
%
%     Copyright (C) 1998-2007 EDF S.A., France
%
%     contact: saturne-support@edf.fr
%
%     The Code_Saturne Kernel is free software; you can redistribute it
%     and/or modify it under the terms of the GNU General Public License
%     as published by the Free Software Foundation; either version 2 of
%     the License, or (at your option) any later version.
%
%     The Code_Saturne Kernel is distributed in the hope that it will be
%     useful, but WITHOUT ANY WARRANTY; without even the implied warranty
%     of MERCHANTABILITY or FITNESS FOR A PARTICULAR PURPOSE.  See the
%     GNU General Public License for more details.
%
%     You should have received a copy of the GNU General Public License
%     along with the Code_Saturne Kernel; if not, write to the
%     Free Software Foundation, Inc.,
%     51 Franklin St, Fifth Floor,
%     Boston, MA  02110-1301  USA
%
%-----------------------------------------------------------------------
%
\programme{vortex}
%
\vspace{1cm}
%%%%%%%%%%%%%%%%%%%%%%%%%%%%%%%%%%
%%%%%%%%%%%%%%%%%%%%%%%%%%%%%%%%%%
\section{Fonction}
%%%%%%%%%%%%%%%%%%%%%%%%%%%%%%%%%%
%%%%%%%%%%%%%%%%%%%%%%%%%%%%%%%%%%
Ce sous-programme est d�di� � la g�n�ration des conditions d'entr�e
turbulente utilis�es en LES.


La m�thode des vortex est bas�e sur une approche de tourbillons
ponctuels. L'id�e de la m�thode consiste � injecter des tourbillons 2D dans le
plan d'entr�e du calcul, puis � calculer le champ de vitesse induit par ces
tourbillons au centre des faces d'entr�e.

%                      Code_Saturne version 1.3
%                      ------------------------
%
%     This file is part of the Code_Saturne Kernel, element of the
%     Code_Saturne CFD tool.
% 
%     Copyright (C) 1998-2007 EDF S.A., France
%
%     contact: saturne-support@edf.fr
% 
%     The Code_Saturne Kernel is free software; you can redistribute it
%     and/or modify it under the terms of the GNU General Public License
%     as published by the Free Software Foundation; either version 2 of
%     the License, or (at your option) any later version.
% 
%     The Code_Saturne Kernel is distributed in the hope that it will be
%     useful, but WITHOUT ANY WARRANTY; without even the implied warranty
%     of MERCHANTABILITY or FITNESS FOR A PARTICULAR PURPOSE.  See the
%     GNU General Public License for more details.
% 
%     You should have received a copy of the GNU General Public License
%     along with the Code_Saturne Kernel; if not, write to the
%     Free Software Foundation, Inc.,
%     51 Franklin St, Fifth Floor,
%     Boston, MA  02110-1301  USA
%
%-----------------------------------------------------------------------
%
%%%%%%%%%%%%%%%%%%%%%%%%%%%%%%%%%%
%%%%%%%%%%%%%%%%%%%%%%%%%%%%%%%%%%
\section{Discr\'etisation}
%%%%%%%%%%%%%%%%%%%%%%%%%%%%%%%%%%
%%%%%%%%%%%%%%%%%%%%%%%%%%%%%%%%%%

Le terme convectif en $\dive(\underline{u} \otimes \rho\,\underline{u})$
introduit une non lin\'earit\'e et un couplage des composantes de la vitesse
$\vect{u}$ dans l'�quation (\ref{Base_Preduv_eqqdm}). Une lin\'earisation et un d\'ecouplage
des trois composantes de la 
vitesse sont r\'ealis\'es lors de la discr\'etisation de cette \'etape de
pr\'ediction.\\
En effet, soit :
\begin{equation}
\vect{\widetilde{u}}= \vect{u}^n + \delta \vect{u} 
\end{equation}
La contribution exacte du terme convectif \`a prendre en compte dans cette
\'etape de pr\'ediction serait :\\
\begin{equation}\label{Base_Preduv_Conv_exact}
\begin{array}{ll}
\dive(\vect{\widetilde{u}} \otimes \rho\,\vect{\widetilde{u}}) =
\dive(\vect{u}^{n} \otimes \rho\,\vect{u}^{n}) + \dive(\delta \vect{u} \otimes
\rho\,\vect{u}^{n}) +  \underbrace { \dive(\vect{u}^{n} \otimes
\rho\,\delta \vect{u})}_{\text {terme couplant lin\'eaire}} +  \underbrace { \dive(\delta \vect{u} \otimes
\rho\,\delta \vect{u})}_{\text {terme couplant et non lin\'eaire}}\\
\end{array} 
\end{equation}
Les deux derniers termes de l'expression (\ref{Base_Preduv_Conv_exact}) sont {\em a priori} n�glig�s
de mani�re � obtenir un syst\`eme en vitesse qui soit d\'ecoupl\'e et donc,
�viter l'inversion d'une matrice pouvant \^etre de tr\`es grande taille. Ces
deux termes peuvent n�anmoins �tre pris en compte de mani�re plus ou moins
approch�e par extrapolation explicite du flux de masse en $n+\theta_F$ (pour le
terme couplant lin�aire provenant de la convection de $\vect{u}^{n}$ par $\delta
\vect{u}$) et utilisation d'un point-fixe par sous it�ration sur le sous
programme \fort{navsto} (pour le terme non-lin�aire, en sp�cifiant $\var{NTERUP}>1$).

L'�quation (\ref{Base_Preduv_eqqdm}) est discr�tis�e au temps $n+\theta$ � l'aide d'un
$\theta$-sch�ma, et le tenseur des pertes de charges d�compos� en une partie
diagonale $\tens{K}_{d}$ et une extradiagonale $\tens{K}_{e}$ (soit
 $\tens{K}_{pdc}=\tens{K}_{d}+\tens{K}_{e}$).\\
$\bullet$ La pression est suppos�e connue en $n-1+\theta$ (d�calage temporel
pression-vitesse) et le gradient naturellement calcul� � cet instant.\\ 
$\bullet$ Les termes sources de viscosit� secondaire, de gradient transpos\'e,
ceux provenant du mod�le de turbulence\footnote{except� $\dive (\mu_t\ (\ggrad
\underline {u}))$}, $\rho\,\tens{K}_{\,e}\ \underline{u}$, $(\rho -\rho_0)
\underline {g}$ ainsi que $\underline{T}_{s}^{\,exp}$ et
$\Gamma\,\underline{u}_{\,i}$ sont pris de mani�re explicite au temps $n$, ou
extrapol�s suivant le sch�ma en temps choisi pour les propri�t�s physique et les
termes sources.\\ 
$\bullet$ Les termes sources $\underline{u}\,\,\dive (\rho\,\underline {u})$,
$\Gamma\,\,\underline{u}$, $T_{s}^{\,imp}\,\,\underline{u}$ et
$-\rho\,\tens{K}_{\,d}\,\,\underline{u}$ sont implicit�s est calcul�s �
l'instant $n+\theta$.\\ 
$\bullet$ Le terme de diffusion $\dive (\mu_{\,tot}\,\ggrad \underline{u})$ est
implicit� : la vitesse est prise � l'instant $n+\theta$ et la viscosit�
explicit�e ou extrapol�e.\\ 
$\bullet$ Enfin, le terme de convection en $\dive(\,\underline{u} \otimes
(\rho\underline{u})\,)$ est implicit� : la composante r�solue de la vitesse est
prise en $n+\theta$, et le flux de masse, explicit�, ou extrapol� en
$n+\theta_F$. 

Par souci de clart�, on suppose, en l'absence d'indication, que les propri�tes
physiques $\Phi$ ($\rho,\,\mu_{tot},\,...$) et le flux de masse
$(\rho\underline{u})$ sont pris respectivement aux instants $n+\theta_\Phi$ et
$n+\theta_F$, o� $\theta_\Phi$ et $\theta_F$ d�pendent des sch�mas en temps
sp�cifiquement utilis�s pour ces grandeurs\footnote{cf. \fort{introd}}. 

La discr�tisation temporelle de l'�quation (\ref{Base_Preduv_eqqdm}) s'�crit alors comme suit : 

\begin{equation}\label{Base_Preduv_eq_di1}
 \begin{array}{c}
\displaystyle \rho\,\ \frac{ \underline {\widetilde{u}}^{n+1} -\underline {u}^{n} }
{\Delta t} + \dive(\,\underline{\widetilde{u}}^{n+\theta} \otimes (\rho\underline{u})\,) -\dive
(\mu_{\,tot}\,\ggrad \underline{\widetilde{u}}^{n+\theta}) =
\\
\displaystyle
 - \grad p^{n-1+\theta} + \dive (\rho\,\underline {u})\,\underline{\widetilde{u}}^{n+\theta} +(\Gamma\,\underline{u}_{\,i})^{n+\theta_S}-\Gamma^n\,\,\underline{\widetilde{u}}^{n+\theta}
\\
\begin{array}{c}
\displaystyle
- \rho\,\tens{K}_{\,d}^{n}\,\,\underline{\widetilde{u}}^{n+\theta} - (\rho\,\tens{K}_{\,e}\ \underline{u})^{n+\theta_S} + (\underline{T}_{s}^{\,exp})^{\,n+\theta_S} + T_{s}^{\,imp}\,\,\underline{\widetilde{u}}^{n+\theta}
\\
\displaystyle
+[\dive (\mu_{\,tot}\,^t\ggrad \underline {u})]^{n+\theta_S}-\frac {2} {3}[\,\grad (\mu_{\,tot}\,\dive \underline {u})]^{n+\theta_S} + (\rho -\rho_0) \underline {g}
 - (\underline{turb})^{n+\theta_S}
\end{array}
\end{array}
\end{equation}
o\`u, par souci de simplification, on a pos\'e :
\begin{equation}
\mu_{\,tot}=
\begin{cases}
\mu+\mu_t & \text{pour les mod�les � viscosit� turbulente ou en LES}, \\
\mu & \text{pour les mod�les au second ordre ou en laminaire}
\end{cases} \ 
\end{equation}
\\
et :
\begin{equation}
\underline{turb}^{n}=
\begin{cases}
\displaystyle\frac {2}{3}\grad (\rho^{n}\,k^{n}) & \text{pour les mod�les � viscosit� turbulente}, \\
\dive(\rho^{n}\,\tens{R}^n) & \text{pour les mod�les au second ordre},\\
0 & \text{en laminaire ou en LES}\\
\end{cases}
\end{equation}
Par analogie avec l'�criture du $\theta$-sch�ma pour une variable scalaire, $\,
\underline {\widetilde{u}}^{n+\theta}$ est interpol�e � partir de la vitesse
pr�dite $\underline {\widetilde{u}}^{n+1}$ de la mani\`ere suivante\footnote{si
$\theta=1/2$, ou qu'une extrapolation est utilis�e, l'ordre 2 n'est obtenu que si
le pas de temps $\Delta t$ est uniforme en temps et en espace.}~: 
\begin{equation}
\underline {\widetilde{u}}^{n+\theta}=\theta\, \underline
{\widetilde{u}}^{n+1}+(1-\theta)\, \underline {u}^{n}\\ 
\end{equation}
Avec :
\begin{equation}
\left\{
\begin{array}{ll}
\theta = 1   & \text{Pour un sch\'ema de type Euler implicite d'ordre 1.}\\
\theta = 1/2 & \text{Pour un sch\'ema de type Cranck-Nicolson d'ordre 2.}\\
\end{array}
\right.
\end{equation}

L'�quation (\ref{Base_Preduv_eq_di1}) est alors r��crite sous la forme :

\begin{equation}\label{Base_Preduv_eq_di2}
\begin{array}{c}
\displaystyle \underbrace{\left(\frac{\rho}{\Delta t} -\theta \,\dive (\rho\,\underline {u}) +\theta \,\, \Gamma^n +
\theta \,\, \rho\,\tens{K}_{\,d}^n-\theta \,T_s^{\,imp} \right)}_{\displaystyle f_s^{imp}}\, (\underline {\,\widetilde{u}}^{n+1} -\underline {u}^{n})
\\
 +\, \theta\, \dive(\underline {\widetilde{u}}^{n+1} \otimes (\rho\underline{u}))-\, \theta\,\dive (\mu_{\,tot}\,\ggrad \underline {\widetilde{u}}^{n+1}) =
\\
-\,(1-\theta)\, \dive(\underline {u}^{n} \otimes (\rho\underline{u})) +\,(1-\theta)\,\dive (\mu_{\,tot}\,\ggrad \underline {u}^{n})
\\
f_s^{exp}\left\{
\begin{array}{c}
\displaystyle 
- \grad p^{n-1+\theta} + \dive (\rho\,\underline {u})\,\underline{u}^{n} +\,(\,\Gamma^{n}\,\underline{u}_{\,i}\,)^{n+\theta_S}- \Gamma^n\,\,\underline{u}^{n}
\\
\displaystyle
-(\,\rho\,\tens{K}_{\,e}\ \underline{u}\,)^{n+\theta_S} -\rho\,\tens{K}_{\,d}^n\ \underline{u}^{n}+ (\underline{T}_{s}^{\,exp})^{\,n+\theta_S} + T_s^{\,imp}\,\,\underline {u}^{n} 
\\
\displaystyle
+[\dive (\mu_{\,tot}\,^t\ggrad \underline {u}\,)]^{n+\theta_S}-\frac {2} {3}[\,\grad (\mu_{\,tot}\,\dive \underline {u}\,)]^{n+\theta_S} + (\rho -\rho_0) \underline {g}-(\underline{turb})^{n+\theta_S}
\end{array}
\right.
\end{array}
\end{equation}

d'o� l'�quation r�solue par le sous-programme \fort{codits} :
\begin{equation}\begin{array}{c}
\displaystyle
f_s^{\,imp}(\underline {\widetilde{u}}^{n+1}-\underline {u}^{n}) + \theta\, \dive(\underline{\widetilde{u}}^{n+1} \otimes (\rho
\underline{u})) - \theta\,\dive (\,\mu_{\,tot}\,\ggrad \underline{\widetilde{u}}^{n+1}) = 
\\\\
\displaystyle
-(1-\theta)\,\dive(\underline{u}^{n} \otimes (\rho \underline{u}))+(1-\theta)\,\dive (\,\mu_{\,tot}\,\ggrad \underline{u}^{n})
+ \underline{f}_{\,s}^{\,exp}
\end{array}
\end{equation}
La m\'ethode de discr\'etisation spatiale est d\'evelopp\'ee dans le sous-programme \fort{codits}.\\



\minititre{Remarques :}
{\tiny$\blacksquare$} Dans le cas standard sans extrapolation, le terme
$-\,T_s^{\,imp}$ n'est ajout� � $f_s^{\,imp}$ que s'il est positif afin de ne
pas affaiblir la dominance de la diagonale de la matrice � inverser.\\ 
{\tiny$\blacksquare$} Si une extrapolation est utilis�e, par contre,
$\,T_s^{\,imp}$ est ajout� � $f_s^{\,imp}$ quel que soit son signe. En effet, l'id�e intuitive qui
consiste � prendre~: 
\begin{equation}
\begin{cases}
\displaystyle
(\underline{T}_{s}^{\,exp} + T_{s}^{\,imp}\,\underline {u})^{\,n+\theta_S} &
\text{si } T_{s}^{\,imp} > 0\\ 
\displaystyle
(\underline{T}_{s}^{\,exp})^{\,n+\theta_S} + T_{s}^{\,imp}\,\underline{u}^{n+\theta} &\text{sinon}\\
\end{cases}
\end{equation} 
aboutit � une incoh�rence dans le traitement si $T_s^{imp}$ change de signe
entre deux pas de temps.\\ 
{\tiny$\blacksquare$} la partie diagonale $\tens{K}_{\,d}$ du terme
de perte de charge est utilis�e dans $f_s^{\,imp}$. En fait, pour \^etre rigoureux,
il faudrait ne retenir que les contributions positives (point signal\'e dans le
sous-programme utilisateur associ\'e \fort{uskpdc}). Cette prise en compte sera \`a am\'eliorer.\\
{\tiny$\blacksquare$} Le terme $\theta\,\Gamma^{n}-\theta\,\dive
(\rho\,\underline {u})$ ne pose pas de probl�me pour la 
dominance de la diagonale de la matrice car il est exactement compens� par le
terme de convection (cf. \fort{covofi}). 


%                      Code_Saturne version 1.3
%                      ------------------------
%
%     This file is part of the Code_Saturne Kernel, element of the
%     Code_Saturne CFD tool.
%
%     Copyright (C) 1998-2007 EDF S.A., France
%
%     contact: saturne-support@edf.fr
%
%     The Code_Saturne Kernel is free software; you can redistribute it
%     and/or modify it under the terms of the GNU General Public License
%     as published by the Free Software Foundation; either version 2 of
%     the License, or (at your option) any later version.
%
%     The Code_Saturne Kernel is distributed in the hope that it will be
%     useful, but WITHOUT ANY WARRANTY; without even the implied warranty
%     of MERCHANTABILITY or FITNESS FOR A PARTICULAR PURPOSE.  See the
%     GNU General Public License for more details.
%
%     You should have received a copy of the GNU General Public License
%     along with the Code_Saturne Kernel; if not, write to the
%     Free Software Foundation, Inc.,
%     51 Franklin St, Fifth Floor,
%     Boston, MA  02110-1301  USA
%
%-----------------------------------------------------------------------
%

%%%%%%%%%%%%%%%%%%%%%%%%%%%%%%%%%%
%%%%%%%%%%%%%%%%%%%%%%%%%%%%%%%%%%
\section{Mise en \oe uvre}
%%%%%%%%%%%%%%%%%%%%%%%%%%%%%%%%%%
%%%%%%%%%%%%%%%%%%%%%%%%%%%%%%%%%%
La num\'ero de la phase trait\'ee fait partie des arguments de \fort{turrij}. On
omettra volontairement de le pr\'eciser dans ce qui suit, on indiquera par $(\ )$ la
notion de tableau s'y rattachant.

\etape{Calcul des termes de production $\tens{\mathcal{P}}$}
\begin{itemize}
\item [$\star$] Initialisation \`a z\'ero du tableau \var{PRODUC} dimensionn\'e \`a $\var{NCEL}\times 6$.
\item [$\star$] On appelle trois fois \fort{grdcel} pour calculer les gradients des composantes de la vitesse $u$, $v$ et
$w$ prises au temps $n$.

Au final, on a :\\
$\displaystyle
\begin{array} {ll}
\var{PRODUC(1,IEL)} = & \displaystyle - 2 \left[ R_{11}^{\,n} \frac{\partial u^{\,n}} {\partial x} +R_{12}^{\,n} \frac{\partial u^{\,n}} {\partial y}+R_{13}^{\,n} \frac{\partial u^{\,n}} {\partial z} \right] \text{        (production de $R_{11}^{\,n}$)}\\
\var{PRODUC(2,IEL)} = & \displaystyle - 2 \left[ R_{12}^{\,n} \frac{\partial v^{\,n}} {\partial x} +R_{22}^{\,n} \frac{\partial v^{\,n}} {\partial y}+R_{23}^{\,n} \frac{\partial v^{\,n}} {\partial z} \right] \text{        (production de $R_{22}^{\,n}$)}\\
\var{PRODUC(3,IEL)} = & \displaystyle - 2 \left[ R_{13}^{\,n} \frac{\partial w^{\,n}} {\partial x} +R_{23}^{\,n} \frac{\partial w^{\,n}} {\partial y}+R_{33}^{\,n} \frac{\partial w^{\,n}} {\partial z} \right] \text{        (production de $R_{33}^{\,n}$)}\\
\var{PRODUC(4,IEL)} = & \displaystyle - \left[ R_{12}^{\,n} \frac{\partial u^{\,n}} {\partial x} +R_{22}^{\,n} \frac{\partial u^{\,n}} {\partial y}+R_{23}^{\,n} \frac{\partial u^{\,n}} {\partial z} \right] \\
& \displaystyle - \left[ R_{11}^{\,n} \frac{\partial v^{\,n}} {\partial x} +R_{12}^{\,n} \frac{\partial v^{\,n}} {\partial y}+R_{13}^{\,n} \frac{\partial v^{\,n}} {\partial z} \right] \text{        (production de $R_{12}^{\,n}$)} \\
\var{PRODUC(5,IEL)} = & \displaystyle - \left[ R_{13}^{\,n} \frac{\partial u^{\,n}} {\partial x} +R_{23}^{\,n} \frac{\partial u^{\,n}} {\partial y}+R_{33}^{\,n} \frac{\partial u^{\,n}} {\partial z} \right] \\
& \displaystyle - \left[ R_{11}^{\,n} \frac{\partial w^{\,n}} {\partial x} +R_{12}^{\,n} \frac{\partial w^{\,n}} {\partial y}+R_{13}^{\,n} \frac{\partial w^{\,n}} {\partial z} \right] \text{        (production de $R_{13}^{\,n}$)} \\
\var{PRODUC(6,IEL)} = & \displaystyle - \left[ R_{13}^{\,n} \frac{\partial v^{\,n}} {\partial x} +R_{23}^{\,n} \frac{\partial v^{\,n}} {\partial y}+R_{33}^{\,n} \frac{\partial v^{\,n}} {\partial z} \right] \\
& \displaystyle - \left[ R_{12}^{\,n} \frac{\partial w^{\,n}} {\partial x} +R_{22}^{\,n} \frac{\partial w^{\,n}} {\partial y}+R_{23}^{\,n} \frac{\partial w^{\,n}} {\partial z} \right]  \text{        (production de $R_{23}^{\,n}$)}
\end{array}
$
\end{itemize}

\etape{Calcul du gradient de la masse volumique $\rho^n$ prise au d\'ebut du pas
de temps courant\footnote{{\it i.e.} calcul\'ee \`a partir des
variables du pas de temps pr\'ec\'edent $n$ si n\'ecessaire.} $(n+1)$}
Ce calcul n'a lieu que si les termes de gravit\'e doivent \^etre pris en compte
($\var{IGRARI()} =1$).
\begin{itemize}
\item [$\star$] Appel de \fort{grdcel}  pour calculer le gradient de $\rho^n$
dans les trois directions de l'espace. Les conditions aux limites sur $\rho^n$
sont des conditions de Dirichlet puisque la valeur de $\rho^n$ aux faces de bord
$ik$ (variable \var{IFAC}) est connue et vaut $\rho_{\,b_{\,ik}}$. Pour \'ecrire les conditions aux limites
sous la forme habituelle, $$\rho_{\,b_{\,ik}} = \var{COEFA} + \var{COEFB}
\,\rho^n_{\,I'}$$ on pose alors $\var{COEFA}=
\var{PROPCE(IFAC,IPPROB(IROM(IPHAS)))}$ et $\var{COEFB} = \var{VISCB} = 0$.\\
$\var{PROPCE(1,IPPROB(IROM(IPHAS)))}$ (resp.$\var{VISCB}$) est utilis\'e en lieu
et place de l'habituel \var{COEFA} ($\var{COEFB}$), lors de l'appel \`a \fort{grdcel}.\\
On a donc :\\
$\displaystyle \var{GRAROX}= \frac{\partial \rho^n}{\partial x}\ $,$\displaystyle \ \var{GRAROY}= \frac{\partial
\rho^n}{\partial y}$ et $
\displaystyle \ \var{GRAROZ}= \frac{\partial \rho^n}{\partial z}\ $.

\end{itemize}

Le gradient de $\rho^n$ servira \`a calculer les termes de production par effets de gravit\'e si ces derniers sont pris en compte.

\etape{Boucle \var{ISOU} de $1$ \`a $6$ sur les tensions de Reynolds}
Pour $\var{ISOU} = 1,2,3,4,5,6$, on r\'esout respectivement et dans
l'ordre  les
\'equations de $R_{11}$, $R_{22}$, $R_{33}$, $R_{12}$, $R_{13}$ et $R_{23}$ par
l'appel au sous-programme \fort{resrij}.\\
La r\'esolution se fait par incr\'ement $\delta {R}_{ij}^{\,n+1,k+1}$ , en utilisant la m\^eme m\'ethode que
celle d\'ecrite dans le sous-programme \fort{codits}. On adopte ici les m\^emes notations.
\var{SMBR} est le second membre du syst\`eme \`a inverser, syst\`eme portant sur
les incr\'ements de la variable. \var{ROVSDT} repr\'esente la diagonale de la
matrice, hors convection/diffusion.\\
On va r\'esoudre l'\'equation (\ref{Base_Turrij_Eq_Temp_Rij}) sous forme incr\'ementale en
utilisant \fort{codits}, soit :
\begin{equation}\label{Base_Turrij_Eq_Temp_deltaRij}
\begin{array}{ll}
&\displaystyle \underbrace{\left(\frac {\rho^n_L}{\Delta t^n}
+ \rho^n_L \,C_1\,\frac{\varepsilon^n_L}{k^n_L}(1-\frac{\delta_{ij}}{3})
 - m^n_{\,lm} + \Gamma_L\,+ max(-\alpha^n_{R_{ij}},0)\right)\,|\Omega_l|}
_{\text {$\var{ROVSDT}$ contribuant
\`a la diagonale de la matrice simplifi\'ee de \fort{matrix}}}\,(\delta{R}_{ij}^{\,n+1,p+1})_{\,L}\\\\
&  \underbrace{+\sum\limits_{m\in Vois(l)}\displaystyle \left[
 m^n_{\,lm} \delta R_{ij,\,f_{\,lm}}^{\,n+1,p+1}
- (\mu^n_{\,lm} + \gamma^n_{\,lm})\
\frac{({\delta R}_{ij}^{\,n+1,p+1})_{M}-({\delta R}_{ij}^{\,n+1,p+1})_{L})}{\overline{L'M'}}\,
S_{\,lm} \right]}_{\text { convection upwind pur et diffusion non reconstruite
relatives \`a la matrice simplifi\'ee de \fort{matrix}\footnotemark}} \\
% voir le texte de la footmark plus bas
&= - \displaystyle\frac {\rho^n_L}{\Delta t^n}\,\left(\,(R^{\,n+1,p}_{ij})_L - (R^{\,n}_{ij})_L\,\right)\\
&-\,\underbrace{\displaystyle\int_{\Omega_l} \left(
\dive\,[\,(\rho\,\vect{u})^n\,R^{\,n+1,p}_{ij} - (\mu^n\,+ \gamma^n\,)\,
\grad{R^{\,n+1,p}_{ij}}\,]\right)\,d\Omega}_{\text {convection et diffusion
trait\'ees par \fort{bilsc2}}}\\
&+\displaystyle \int_{\Omega_l} \left[\,\mathcal{P}^{\,n+1,p}_{ij} + \mathcal{G}^{\,n+1,p}_{ij}
- \displaystyle\rho^n \,C_1\,\frac{\varepsilon^n}{k^n}\left[R^{\,n+1,p}_{ij}-
\frac{2}{3}\,k^n\,\delta_{ij}\right] + \phi^{\,n+1,p}_{ij,2} +
\phi^{\,n+1,p}_{ij,w}\,\right]\, d\Omega \\
& + \displaystyle\int_{\Omega_l} \left[- \frac{2}{3} \rho^n \varepsilon^n \delta_{ij}
 + \Gamma\,(\,R^{\,in}_{ij} - R^{\,n+1,p}_{ij}\,) +
\alpha^n_{R_{ij}}\,R^{\,n+1,p}_{ij}+ \beta^n_{R_{ij}}\right]\, d\Omega\\
&+ \sum\limits_{m\in
Vois(l)}\displaystyle \left[\ \tens{E}^n\,\grad{R}^{\,n+1,p}_{ij} \right]_{\,lm}\,.\,\vect{n}_{\,lm}S_{\,lm}\\
&+ \sum\limits_{m\in Vois(l)}\displaystyle \left[\
\tens{D}^n\,\grad{R}^{\,n+1,p}_{ij} \right]_{\,lm}\,.\,\vect{n}_{\,lm}S_{\,lm}\\
&- \sum\limits_{m\in Vois(l)} \gamma^n_{\,lm} \left( \grad{R}^{\,n+1,p}_{ij}\,.\,\vect{n}_{\,lm} \right)  S_{\,lm}\\
&+ \sum\limits_{m\in Vois(l)}  m^n_{\,lm}\,(R^{\,n+1,p}_{ij})_L\\
\end{array}
\end{equation}
% si on ne fait pas comme ca, il n'apparait pas
\footnotetext[\thefootnote]{Si $\var{IRIJNU} = 1$, on remplace  $\mu^n_{\,lm}$ par $(\mu +
\mu_t)^n_{\,lm}$ dans l'expression de la diffusion non reconstruite
associ\'ee \`a la matrice simplifi\'ee de \fort{matrix} ($\mu_t$ d\'esigne la
viscosit\'e turbulente calcul\'ee comme en $k-\varepsilon$).}

o\`u on rappelle :\\
pour $n$ donn\'e entier positif, on d\'efinit la suite
 $({R}_{ij}^{\,n+1,p})_{p \in \grandN}$
 par :
\begin{equation}\notag
\left\{\begin{array}{l}
{R}_{ij}^{\,n+1,0} = {R}_{ij}^{\,n}\\
{R}_{ij}^{\,n+1,p+1} = {R}_{ij}^{\,n+1,p} + \delta{R}_{ij}^{\,n+1,p+1} \\
\end{array}\right.
\end{equation}
$(\delta{R}_{ij}^{\,n+1,p+1})_{\,L}$ d\'esigne la valeur sur l'\'el\'ement
$\Omega_l$ du $\text{$(\,p+1\,)$-i\`eme}$ incr\'ement de ${R}_{ij}^{\,n+1}$,
$ m^n_{\,lm}$ le flux de masse \`a l'instant $n$ \`a travers la face $lm$,
$\delta R_{ij,\,f_{\,lm}}^{\,n+1,p+1}$ vaut $({\delta
R}_{ij}^{\,n+1,p+1})_{L}$  si $ m^n_{\,lm} \geqslant 0$, $({\delta
R}_{ij}^{\,n+1,p+1})_{M}$ sinon,
$\mathcal{P}^{\,n+1,p}_{ij}$, $\phi^{\,n+1,p}_{ij,2}$, $\phi^{\,n+1,p}_{ij,w}$ les valeurs
des quantit\'es associ\'ees correspondant \`a l'incr\'ement
$(\delta{R}_{ij}^{\,n+1,p})$.\\



Tous ces termes sont calcul\'es comme suit :
\begin{itemize}
\item Terme de gauche de l'\'equation (\ref{Base_Turrij_Eq_Temp_deltaRij})\\
Dans \fort{resrij} est calcul\'ee la variable \var{ROVSDT}. Les autres
termes sont compl\'et\'es par \fort{codits}, lors de la construction de la matrice simplifi\'ee , {\it via} un
appel au sous-programme \fort{matrix}. La quantit\'e
 $(\mu^n_{\,lm} + \gamma^n_{\,lm})$ \`a la face $lm$ est calcul\'ee lors de l'appel \`a
\fort{visort}.\\
\item Second membre de l'\'equation (\ref{Base_Turrij_Eq_Temp_deltaRij})\\
Le premier terme non d\'etaill\'e est calcul\'e par le sous-programme
\fort{bilsc2}, qui applique le sch\'ema convectif choisi par l'utilisateur, qui
reconstruit ou non selon le souhait de l'utilisateur les gradients intervenants
dans la convection-diffusion.\\
Les termes sans accolade sont, eux, compl\`etement explicites et ajout\'es au fur et
\`a mesure dans \var{SMBR} pour former
l'expression $f^{\,exp}_s$ de \fort{codits}.
\end{itemize}
On d\'ecrit ci-dessous les \'etapes de \fort{resrij} :
\begin{itemize}

\item DELTIJ = 1, pour $\var{ISOU} \leqslant 3$ et DELTIJ = 0  Si $\var{ISOU} >
3$. Cette valeur repr\'esente le symbole de Kroeneker $\delta_{ij}$.

\item Initialisation \`a z\'ero de \var{SMBR} (tableau contenant le second
membre) et \var{ROVSDT} (tableau contenant la diagonale de la matrice sauf celle
relative \`a la contribution de la
diagonale des op\'erateurs de convection et de diffusion lin\'earis\'es
\footnote{qui correspondent aux sch\'emas convectif upwind pur et diffusif sans
reconstruction.}), tous deux de dimension $\var{NCEL}$.

\item Lecture et prise en compte des termes sources utilisateur pour la variable $R_{ij}$

Appel \`a \fort{ustsri} pour charger les termes sources utilisateurs. Ils sont
stock\'es comme suit. Pour la cellule $\Omega_l$ de centre $L$, repr\'esent\'ee par $\var{IEL}$, on a :\\
\begin{equation}\notag
\left\{\begin{array}{lll}
&\var{ROVSDT(IEL)}&= |\Omega_l| \ \alpha_{R_{ij}}\\
&\var{SMBR(IEL)}&=|\Omega_l| \ \beta_{R_{ij}}\\
\end{array}\right.
\end{equation}
On affecte alors les valeurs ad\'equates au second membre \var{SMBR} et \`a la
diagonale \var{ROVSDT} comme suit :
\begin{equation}\notag
\left\{\begin{array}{lll}
&\var{SMBR(IEL)} &= \var{SMBR(IEL)} +\ |\Omega_l| \ \alpha_{R_{ij}} \ (R^n_{ij})_L \\
&\var{ROVSDT(IEL)}&= \text{max }(-\ |\Omega_l| \ \alpha_{R_{ij}},0)\\
\end{array}\right.
\end{equation}
La valeur de $ \var{ROVSDT}$ est ainsi calcul\'ee pour des raisons de stabilit\'e
num\'erique. En effet, on ne rajoute sur la diagonale que les valeurs positives,
ce qui correspond physiquement \`a impliciter les termes de rappel uniquement.
\item{Calcul du terme source de masse  si $\Gamma_L > 0$}

Appel de \fort{catsma} et incr\'ementation si n\'ecessaire de \var{SMBR} et
\var{ROVSDT} {\it via} :\\
\begin{equation}\notag
\left\{\begin{array}{lll}
\displaystyle \var{SMBR(IEL)} = \var{SMBR(IEL)} + |\Omega_l| \ \Gamma_L \
\left[(R^{\,in}_{ij})_L - (R^{\,n}_{ij})_L \right] \\
\displaystyle \var{ROVSDT(IEL)}=\var{ROVSDT(IEL)} + |\Omega_l| \ \Gamma_L
\end{array}\right.
\end{equation}
\item Calcul du terme d'accumulation de masse et du terme instationnaire

On stocke $\displaystyle \var{W1}= \int_{\Omega_l}\dive\,(\rho\,\vect{u})\,d\Omega$
calcul\'e par \fort{divmas} \`a l'aide des flux de masse aux faces internes
$ m^n_{\,lm}=\sum\limits_{m\in Vois(l)}{(\rho \vect{u})_{\,lm}^n} \text{.}\,
\vect{S}_{\,lm} $ (tableau \var{FLUMAS}) et des flux de masse aux bords  $ m^n_{\,b_{lk}} = \sum\limits_{k\in{\gamma_b(l)}}{(\rho \vect{u})_{\,{b}_{lk}}^n} \text{.}\,
\vect{S}_{\,{b}_{lk}} $ (tableau \var{FLUMAB}).
On incr\'emente ensuite \var{SMBR} et \var{ROVSDT}.
\begin{equation}\notag
\left\{\begin{array}{lll}
&\var{SMBR(IEL)} &= \var{SMBR(IEL)} + \var{ICONV}\  (R^n_{ij})_L\,(\displaystyle
\int_{\Omega_l}\dive\,(\rho\,\vect{u})\ d\Omega) \\
&\var{ROVSDT(IEL)}& = \var{ROVSDT(IEL)} +  \var{ISTAT}\,\displaystyle
\frac{\rho^n_L \ |\Omega_l|}{\Delta t^n} -  \var{ICONV}\ (\displaystyle
\int_{\Omega_l}\dive\,(\rho\,\vect{u})\ d\Omega) \\
\end{array}\right.
\end{equation}
\item Calcul des termes sources de production, des termes $\displaystyle
\phi_{\,ij,1}+\phi_{\,ij,2}$ et de la dissipation~$\displaystyle-\frac{2}{3} \varepsilon\,\delta_{\,ij}$ :

On effectue une boucle d'indice \var{IEL} sur les cellules $\Omega_l$ de centre $L$ :
\begin{itemize}
\item [$\Rightarrow$] $\displaystyle \var{TRPROD}= \frac{1}{2} (\mathcal{P}^n_{ii})_L = \frac{1}{2} \left[ \var{PRODUC(1,IEL)} +  \var{PRODUC(2,IEL)} +  \var{PRODUC(3,IEL)} \right] $
\item [$\Rightarrow$] $\displaystyle \var{TRRIJ }= \frac{1}{2} (R^n_{ii})_L $
\item [$\Rightarrow$] $\displaystyle \var{SMBR(IEL)} =\ \var{SMBR(IEL)}\ +$\\
$\ \displaystyle\rho^n_L |\Omega_l| \left[ \displaystyle
\frac{2}{3}\,\delta_{\,ij} \left( \ \displaystyle \frac{ C_2}{2}\,(\mathcal{P}^n_{ii})_L\ +
(C_1-1)\ \varepsilon^n_L\, \right)\right.$\\
$ + \left.\ (1-C_2) \ \var{PRODUC(ISOU,IEL)} -
\displaystyle C_1\ \frac{2\,\varepsilon^n_L}{(R^n_{ii})_L}\ (R^n_{ij})_L \right]$
\item [$\Rightarrow$] $\displaystyle \var{ROVSDT(IEL)} = \var{ROVSDT(IEL)} +
\rho^n_L \ |\Omega_l| \ (- \displaystyle \frac{1}{3} \ \,\delta_{\,ij} + 1) \ C_1
\ \frac{2\ \varepsilon^n_L}{(R^n_{ii})_L}$
\end{itemize}
\item Appel de \fort{rijech} pour le calcul des termes d'\'echo de paroi
 $\phi^n_{ij,w}$ si $\var{IRIJEC()}=1$ et ajout dans \var{SMBR}.\\
$\var{SMBR} = \var{SMBR} + \phi^n_{ij,w}$\\
Suivant son mode de calcul (\var{ICDPAR}), la distance � la paroi est directement accessible
par \var{RA(IDIPAR+IEL-1)} (\var{|ICDPAR|} = 1) ou bien
est calcul\'ee \`a partir de $\var{IA(IIFAPA(IPHAS)+IEL - 1)}$,
qui donne pour l'\'el\'ement $\var{IEL}$ le num\'ero de la face de bord
paroi la plus  proche (\var{|ICDPAR|} = 2). Ces tableaux ont \'et\'e renseign\'e une fois pour toutes au
d\'ebut de calcul.

\item  Appel de \fort{rijthe} pour le calcul des termes de gravit\'e $\mathcal{G}^n_{ij}$ et ajout dans \var{SMBR}.

Ce calcul n'a lieu que si $\var{IGRARI()} = 1$.
$ \var{SMBR} = \var{SMBR} + \mathcal{G}^n_{ij}$
\item Calcul de la partie explicite du terme de diffusion
 $\dive{\,\left[\tens{A}\,\grad{R}^{\,n}_{ij}\right]}$, plus pr\'ecis\'ement
des contributions du terme extradiagonal pris aux faces purement internes
(remplissage du tableau \var{VISCF}), puis aux faces de bord (remplissage du
tableau \var{VISCB}).
\begin{itemize}
\item [$\star$] Appel de \fort{grdcel} pour le calcul du gradient de
$R^{\,n}_{ij}$ dans chaque direction. Ces gradients sont respectivement
stock\'es dans les tableaux de travail \var{W1}, \var{W2} et \var{W3}.

\item [$\star$] boucle d'indice \var{IEL} sur les cellules $\Omega_l$ de centre
$L$ pour le
calcul de $\tens{E}^n\,\grad{R}^{\,n}_{ij}$ aux cellules dans un premier temps :\\
\begin{itemize}
\item [$\Rightarrow$] $\displaystyle \var{TRRIJ}= \frac{1}{2} (R^{\,n}_{ii})_L $
\item [$\Rightarrow$] $\displaystyle \var{CSTRIJ} = \rho^n_L\ C_S \ \displaystyle\frac{(R^n_{ii})_L}{2\,\varepsilon^n_L}$
\item [$\Rightarrow$] $\displaystyle \var{W4(IEL)} = \rho^n_L\ C_S\
\displaystyle\frac{(R^n_{ii})_L}{2\,\varepsilon^n_L} \left[\,(R^{\,n}_{12})_L \ \var{W2(IEL)} +
(R^{\,n}_{13})_L \ \var{W3(IEL)}\,\right]$
\item [$\Rightarrow$] $\displaystyle \var{W5(IEL)} = \rho^n_L\ C_S\
\displaystyle\frac{(R^n_{ii})_L}{2\,\varepsilon^n_L} \left[\,(R^{\,n}_{12})_L \ \var{W1(IEL)} +
(R^{\,n}_{23})_L \ \var{W3(IEL)}\,\right]$
\item [$\Rightarrow$] $\displaystyle \var{W6(IEL)} = \rho^n_L\ C_S\
\displaystyle\frac{(R^n_{ii})_L}{2\,\varepsilon^n_L} \left[\,(R^{\,n}_{13})_L \ \var{W1(IEL)} + (R^{\,n}_{23})_L \ \var{W2(IEL)}\,\right]$
\end{itemize}



\item [$\star$] Appel de \fort{vectds}\footnote{Le r\'esultat est stock\'e dans
\var{VISCF} et \var{VISCB}. Dans \fort{vectds}, les valeurs aux faces internes
sont interpol\'ees lin\'eairement sans reconstruction et \var{VISCB} est mis \`a
z\'ero.} pour assembler $\displaystyle\left[ \tens{E}^n\,\grad{R}^{\,n}_{ij}
\right]\,.\,\vect{n}_{\,lm}S_{\,lm}$ aux faces $lm$.
\item [$\star$] Appel de \fort{divmas} pour calculer la divergence du flux d\'efini par \var{VISCF} et \var{VISCB}.
Le r\'esultat est stock\'e dans \var{W4}.\\
Ajout au second membre \var{SMBR}.\\
\var{SMBR} = \var{SMBR} + \var{W4}
\end{itemize}

A l'issue de cette \'etape, seule la partie extradiagonale de la diffusion prise
enti\`erement explicite~:
 $$\sum\limits_{m\in
Vois(l)}\left[\ \tens{E}^n\,\grad{R}^{\,n}_{ij} \right]_{\,lm}\,.\,\vect{n}_{\,lm}S_{\,lm}$$ a \'et\'e calcul\'ee.\\

\item Calcul de la partie diagonale du terme de diffusion\footnote{Seule la
composante normale  du  gradient de $R_{ij}$ aux faces sera implicite.} :\\
Comme on l'a d\'eja vu, une partie de cette contribution sera trait\'ee en
implicite (celle relative \`a la composante normale du gradient) et donc
ajout\'ee au second membre par \fort{bilsc2} ; l'autre
partie sera explicite et prise en compte dans $f_s^{\,exp}$.
\begin{itemize}
\item [$\star$] On effectue une boucle d'indice \var{IEL} sur les cellules
$\Omega_l$ de centre $L$ :
\begin{itemize}
\item [$\Rightarrow$] $\displaystyle \var{TRRIJ }= \frac{1}{2} (R^{\,n}_{ii})_L $
\item [$\Rightarrow$] $\displaystyle \var{CSTRIJ} = \rho^n_L \ C_S \ \frac{(R^{\,n}_{ii})_L}{2\,\varepsilon^n_L}$
\item [$\Rightarrow$] $\displaystyle \var{W4(IEL)} = \rho^n_L \ C_S \
\frac{(R^{\,n}_{ii})_L}{2\,\varepsilon^n_L} \ (R^{\,n}_{11})_L$
\item [$\Rightarrow$] $\displaystyle \var{W5(IEL)} = \rho^n_L \ C_S \ \frac{(R^{\,n}_{ii})_L}{2\,\varepsilon^n_L}\ (R^n_{22})_L$
\item [$\Rightarrow$] $\displaystyle \var{W6(IEL)} = \rho^n_L \ C_S \ \frac{(R^{\,n}_{ii})_L}{2\,\varepsilon^n_L} \ (R^n_{33})_L$
\end{itemize}

%\item Traitement du parall\'elisme et de la p\'eriodicit\'e.

\item [$\star$] On effectue une boucle d'indice \var{IFAC} sur les faces
purement internes $lm$ pour remplir le tableau \var{VISCF} :
\begin{itemize}
\item [$\Rightarrow$] Identification des cellules $\Omega_l$ et $\Omega_m$ de
centre respectif $L$ (variable \var{II}) et $M$ (variable \var{JJ}), se trouvant de chaque c\^ot\'e de la face
$lm$\footnote{La normale \`a la face est orient\'ee de L vers M.}.
\item [$\Rightarrow$] Calcul du carr\'e de la surface de la face. La valeur est
stock\'ee dans le tableau \var{SURFN2}.
\item [$\Rightarrow$] Interpolation du gradient de $R^{\,n}_{ij}$ \`a la face
$lm$ (gradient facette $\left[\grad{R}^{\,n}_{ij}\right]_{\,lm}$) :
\begin{equation}\notag
\left\{\begin{array}{ll}
\var{GRDPX} &= \displaystyle \frac{1}{2} \left(\var{W1(II)} + \var{W1(JJ)}
\right) \\
&\\
\var{GRDPY} &= \displaystyle \frac{1}{2} \left(\var{W2(II)} + \var{W2(JJ)} \right) \\
&\\
\var{GRDPZ} &= \displaystyle \frac{1}{2} \left(\var{W3(II)} + \var{W3(JJ)} \right)
\end{array}\right.
\end{equation}
\item [$\Rightarrow$] Calcul du gradient de $R^{\,n}_{ij}$ normal \`a la face
$lm$, $\left[\grad{R}^{\,n}_{ij}\right]_{\,lm}.\vect{n}_{\,lm}\,S_{\,lm}$.\\

$\displaystyle \var{GRDSN} =  \var{GRDPX} \ \var{SURFAC(1,IFAC)} + \var{GRDPY} \ \var{SURFAC(2,IFAC)} +  \var{GRDPZ} \ \var{SURFAC(3,IFAC)}$
$\var{SURFAC}$ est le vecteur surface de la face \var{IFAC}.


\item [$\Rightarrow$] calcul de
 $\left[\grad{R^{\,n}_{ij}} - (\grad
R^{\,n}_{ij}\,.\,\vect{n}_{\,lm})\vect{n}_{\,lm}\right]$, les vecteurs \'etant
calcul\'es \`a la face $lm$ :
\begin{equation}\notag
\left\{\begin{array}{lll}
&\displaystyle \var{GRDPX} &= \var{GRDPX} - \displaystyle\frac{\var{GRDSN}}{\var{SURFN2}} \ \var{SURFAC(1,IFAC)}\\
&&\\
&\displaystyle \var{GRDPY} &= \var{GRDPY} - \displaystyle\frac{\var{GRDSN}}{\var{SURFN2}} \ \var{SURFAC(2,IFAC)} \\
&&\\
&\displaystyle \var{GRDPZ} &= \var{GRDPZ} - \displaystyle\frac{\var{GRDSN}}{\var{SURFN2}} \ \var{SURFAC(3,IFAC)}
\end{array}\right.
\end{equation}
\item [$\Rightarrow$] finalisation du calcul de l'expression totalement
explicite
 $$\left[ \tens{D}^n\,\left( \grad{R^{\,n}_{ij}} - (\grad R^{\,n}_{ij}\,.\,\vect{n}_{\,lm})\,\vect{n}_{\,lm}\right) \right]\,.\,\vect{n}_{\,lm}$$
\begin{equation}\notag
\begin{array} {ll}
\displaystyle \var{VISCF} = &
 \displaystyle\frac{1}{2} (\ \var{W4(II)} +\ \var{W4(JJ)}) \ \var{GRDPX} \
\var{SURFAC(1,IFAC)})\ + \\
&\\
&  \displaystyle\frac{1}{2} (\ \var{W5(II)} +\ \var{W5(JJ)}) \ \var{GRDPY} \
\var{SURFAC(2,IFAC)})\ + \\
&\\
&  \displaystyle\frac{1}{2} (\ \var{W6(II)} +\ \var{W6(JJ)}) \ \var{GRDPZ} \ \var{SURFAC(3,IFAC)})
\end{array}
\end{equation}
\end{itemize}

\item [$\star$] Mise \`a z\'ero du tableau \var{VISCB}.

\item [$\star$] Appel de \fort{divmas} pour calculer la divergence de~:
 $$\tens{D}^{\,n}\,\left( \grad{R^{\,n}_{ij}} - (\grad R^{\,n}_{ij}\,.\,\vect{n}_{\,lm})\vect{n}_{\,lm}\right)$$ d\'efini aux faces dans \var{VISCF} et \var{VISCB}.

Le r\'esultat est stock\'e dans le tableau \var{W1}.\\
Ajout au second membre \var{SMBR}.\\
$\var{SMBR} = \var{SMBR} + \var{W1}$
\end{itemize}
\item Calcul de la viscosit\'e orthotrope $\gamma^n_{\,lm}$ \`a la face $lm$ de la variable principale
$R^{\,n}_{ij}$\\
Ce calcul permet au sous-programme \fort{codits} de compl\'eter le second membre
\var{SMBR} par :
\begin{equation}
\begin{array} {ll}
& \sum\limits_{m\in Vois(l)}
\mu^n_{\,lm}\,\left(\grad{R}^{\,n}_{ij}\,.\,\vect{n}_{\,lm}\right)S_{\,lm}
 + \sum\limits_{m\in Vois(l)} \left(\grad{R}^{\,n}_{ij}
\,.\,\vect{n}_{\,lm}\right)\left[\tens{D}^{\,n}\,\vect{n}_{\,lm}\right]_{\,lm}\,.\,\vect{n}_{\,lm}\
S_{\,lm}\\
& = \sum\limits_{m\in Vois(l)}(\,\mu^n_{\,lm}\, + \,\gamma^n_{\,lm}\,)\,\left(\grad{R}^{\,n}_{ij}\,.\,\vect{n}_{\,lm}\right)S_{\,lm}
\end{array}
\end{equation}
sans pr\'eciser la nature de la face $lm$, {\it via} l'appel \`a \fort{bilsc2}  et de disposer de la quantit\'e
$(\mu^n_{\,lm}\, + \gamma^n_{\,lm})$ pour construire sa
matrice simplifi\'ee.\\
\begin{itemize}
\item [$\star$] On effectue une boucle d'indice \var{IEL} sur les cellules
$\Omega_l$ :
\begin{itemize}
\item [$\Rightarrow$] $\displaystyle \var{TRRIJ }= \frac{1}{2} (R^{\,n}_{ii})_L $
\item [$\Rightarrow$] $\displaystyle \var{RCSTE} = \rho^n_L \ C_S \ \frac{ (R^{\,n}_{ii})_L}{2\,\varepsilon^n_L} $
\item [$\Rightarrow$] $\displaystyle \var{W1(IEL)} = \mu^n +\rho^n_L \ C_S \ \frac{
(R^{\,n}_{ii})_L}{2\,\varepsilon^n_L}\ (R^n_{11})_L$
\item [$\Rightarrow$] $\displaystyle \var{W2(IEL)} = \mu^n + \rho^n_L \ C_S \ \frac{ (R^{\,n}_{ii})_L}{2\,\varepsilon^n_L}\ (R^n_{22})_L$
\item [$\Rightarrow$] $\displaystyle \var{W3(IEL)} = \mu^n + \rho^n_L \ C_S \ \frac{ (R^{\,n}_{ii})_L}{2\,\varepsilon^n_L}\ (R^n_{33})_L$
\end{itemize}

\item [$\star$] Appel de \fort{visort} pour calculer la viscosit\'e orthotrope
\footnote{Comme dans le sous-programme \fort{viscfa}, on multiplie la viscosit\'e par
$\displaystyle \frac{S_{\,lm}}{\overline{L'M'}}$, o\`u $S_{\,lm}$ et
$\overline{L'M'}$ repr\'esentent respectivement la surface de la face $lm$ et la
mesure alg\'ebrique du segment reliant les projections des centres des cellules
voisines sur la normale \`a la face. On garde dans ce sous-programme  la possibilit\'e d'interpoler la viscosit\'e aux cellules lin\'eairement ou d'utiliser une moyenne harmonique. La viscosit\'e au bord est celle de la cellule de bord correspondante.}
$ \gamma^n_{\,lm} = (\tens{D}^{\,n}\,\vect{n}_{\,lm}).\vect{n}_{\,lm}$ aux faces $lm$

Le r\'esultat est stock\'e dans les tableaux \var{VISCF} et \var{VISCB}.
\end{itemize}

\item appel de \fort{codits} pour la r\'esolution de l'\'equation de
convection/diffusion/termes sources de la variable $R_{ij}$. Le terme source est
\var{SMBR}, la viscosit\'e \var{VISCF} aux faces purement internes (
resp. \var{VISCB} aux faces de bord) et \var{FLUMAS} le flux de masse interne
 ( resp. \var{FLUMAB} flux de masse au bord). Le r\'esultat est la variable $R_{ij}$ au temps
$n+1$, donc $R^{\,n+1}_{ij}$. Elle est stock\'ee dans le tableau \var{RTP} des
variables mises \`a jour.

\end{itemize}

\etape{Appel de \fort{reseps} pour la r\'esolution de la variable $\varepsilon$}

On d\'ecrit ci-dessous le sous-programme \fort{reseps}, les commentaires du sous-programme \fort{resrij} ne seront pas r\'ep\'et\'es puisque les deux sous-programmes ne diff\`erent que par leurs termes sources.

\begin{itemize}
\item Initialisation \`a z\'ero de \var{SMBR} et \var{ROVSDT}.

\item{Lecture et prise en compte des termes sources utilisateur pour la variable $\varepsilon$ :}

Appel de \fort{ustsri} pour charger les termes sources utilisateurs. Ils sont
stock\'es dans les tableaux suivants :\\
pour la cellule $\Omega_l$ repr\'esent\'ee par $\var{IEL}$ de centre $L$, on a :
\begin{equation}\notag
\left\{\begin{array}{lll}
&\var{ROVSDT(IEL)}&= |\Omega_l| \ \alpha_{\varepsilon}\\
&\var{SMBR(IEL)}&=|\Omega_l| \ \beta_{\varepsilon}\\
\end{array}\right.
\end{equation}
On affecte alors les valeurs ad\'equates au second membre \var{SMBR} et \`a la
diagonale \var{ROVSDT} comme suit :
\begin{equation}\notag
\left\{\begin{array}{lll}
&\var{SMBR(IEL)} &= \var{SMBR(IEL)} +\ |\Omega_l| \ \alpha_{\,\varepsilon} \
\varepsilon^n_L \\
&\var{ROVSDT(IEL)}&= \text{max }(-\ |\Omega_l| \ \alpha_{\,\varepsilon},0)\\
\end{array}\right.
\end{equation}

\item{Calcul du terme source de masse si $\Gamma_L > 0$ :
\begin{equation}\notag
\left\{\begin{array}{lll}
&\displaystyle \var{SMBR(IEL)} = \var{SMBR(IEL)} + |\Omega_l| \ \Gamma_L \
(\varepsilon^{\,in}_L -\varepsilon^n_L) \\
&\displaystyle \var{ROVSDT(IEL)}= \var{ROVSDT(IEL)} + |\Omega_l| \ \Gamma_L
\end{array}\right.
\end{equation}
\item Calcul du terme d'accumulation de masse et du terme instationnaire \\
On stocke $\displaystyle \var{W1}= \int_{\Omega_l}\dive\,(\rho\,\vect{u})\,d\Omega$
calcul\'e par \fort{divmas} \`a l'aide des flux de masse internes et aux bords.\\
On incr\'emente ensuite \var{SMBR} et \var{ROVSDT}.
\begin{equation}\notag
\left\{\begin{array}{lll}
&\var{SMBR(IEL)} &= \var{SMBR(IEL)} + \var{ICONV}\ \varepsilon^n_L\,(\displaystyle
\int_{\Omega_l}\dive\,(\rho\,\vect{u})\ d\Omega) \\
&\var{ROVSDT(IEL)}& = \var{ROVSDT(IEL)} +  \var{ISTAT}\,\displaystyle
\frac{\rho^n_L \ |\Omega_l|}{\Delta t^n} -  \var{ICONV}\ (\displaystyle
\int_{\Omega_l}\dive\,(\rho\,\vect{u})\ d\Omega) \\
\end{array}\right.
\end{equation}

\item Traitement du terme de production
 $\displaystyle \rho\,C_{\varepsilon_1}\,\frac{\varepsilon}{k}\,\mathcal{P}$
 et du terme de dissipation $-\,\displaystyle \rho\,C_{\varepsilon_2}\,\frac{\varepsilon}{k}\,\varepsilon$ \\
pour cela, on effectue une boucle d'indice \var{IEL} sur les cellules $\Omega_l$
de centre $L$ :
\begin{itemize}
\item [$\Rightarrow$] $\displaystyle \var{TRPROD}= \frac{1}{2} (\mathcal{P}^n_{ii})_L = \frac{1}{2} \left[ \var{PRODUC(1,IEL)} +  \var{PRODUC(2,IEL)} +  \var{PRODUC(3,IEL)} \right] $
\item [$\Rightarrow$] $\displaystyle \var{TRRIJ }= \frac{1}{2} (R^n_{ii})_L $
\item [$\Rightarrow$] $\displaystyle \var{SMBR(IEL)} = \var{SMBR(IEL)} + \rho^n_L
|\Omega_l| \left[ -C_{\varepsilon_2} \ \frac{2\,(\varepsilon^n_L)^2}{(R^n_{ii})_L} + C_{\varepsilon_1} \ \frac{\varepsilon^n_L}{(R^n_{ii})_L}\ (\mathcal{P}^n_{ii})_L \right] $
\item [$\Rightarrow$] $\displaystyle \var{ROVSDT(IEL)} = \var{ROVSDT(IEL)} + C_{\varepsilon_2} \ \rho^n_L \ |\Omega_l| \ \frac{2\,\varepsilon^n_L}{(R^n_{ii})_L}$
\end{itemize}

\item Appel de \fort{rijthe} pour le calcul des termes de gravit\'e $\mathcal{G}^n_{\varepsilon}$ et ajout dans \var{SMBR}.

$ \var{SMBR} = \var{SMBR} + \mathcal{G}^n_{\varepsilon}$\\
Ce calcul n'a lieu que si $\var{IGRARI()} = 1$.

\item Calcul de la diffusion de $\varepsilon$ \\
 Le terme $\dive \left[\mu\, \grad(\varepsilon) + \tens{A'}\,\grad(\varepsilon)
\right]$ est calcul\'e exactement de la m\^eme mani\`ere que pour les tensions
de Reynolds $R_{ij}$ en rempla\c cant $\tens{A}$ par $\tens{A'}$.

\item Appel de \fort{codits} pour la r\'esolution de l'\'equation de
convection/diffusion/termes sources de la variable principale $\varepsilon$. Le
r\'esultat $\varepsilon^{\,n+1}$ est stock\'e dans le tableau \var{RTP} des
variables mises \`a jour.
}
\end{itemize}

\etape{clippings finaux}
On passe enfin dans le sous-programme  \fort{clprij} pour faire un clipping \'eventuel
des variables $R^{\,n+1}_{ij}$ et $\varepsilon^{\,n+1}$. Le sous-programme
\fort{clprij} est appel\'e\footnote{L'option
$\var{ICLIP} = 1$ consiste en un clipping minimal des variables $R_{ii}$ et
$\varepsilon$ en prenant la valeur absolue de ces variables puisqu'elles ne
peuvent \^etre que positives.} avec $\var{ICLIP} = 2$ . Cette option
\footnote{Quand la valeur des grandeurs $R_{ii}$ ou $\varepsilon$ est
n\'egative, on la remplace par le minimum entre sa valeur absolue et (1,1)
fois la valeur obtenue au pas de temps pr\'ec\'edent.} contient l'option $\var{ICLIP} = 1$  et permet de v\'erifier l'in\'egalit\'e de Cauchy-Schwarz sur les grandeurs extra-diagonales du tenseur $\tens{R}$ (pour $i \neq j$, $|R_{ij}|^2 \le R_{ii} R_{jj}$).


%%%%%%%%%%%%%%%%%%%%%%%%%%%%%%%%%%
%%%%%%%%%%%%%%%%%%%%%%%%%%%%%%%%%%
\section{Points \`a traiter}
%%%%%%%%%%%%%%%%%%%%%%%%%%%%%%%%%%
%%%%%%%%%%%%%%%%%%%%%%%%%%%%%%%%%%
Sauf mention explicite, $\phi$ repr\'esentera une tension de Reynolds ou la dissipation turbulente ($\phi = R_{ij} \ \text{ou} \ \varepsilon$).

\begin{itemize}
\item {La vitesse utilis\'ee pour le calcul de la production est explicite. Est-ce qu'une implicitation peut am\'eliorer la pr\'ecision temporelle de $\phi$ \footnote{Cette remarque peut \^etre g\'en\'eralis\'ee. En effet, peut-on envisager d'actualiser les variables d\'ej\`a r\'esolues (sans r\'eactualiser les variables turbulentes apr\`es leur r\'esolution)? Ceci obligerait \`a modifier les sous-programmes tels que \fort{condli} qui sont appel\'es au d\'ebut de la boucle en temps.} ?}
\item {Dans quelle mesure le terme d'\'echo de paroi est-il valide ? En effet, ce terme est remis en question par certains auteurs.}
\item {On peut envisager la r\'esolution d'un syst\`eme hyperbolique pour les
tensions de Reynolds afin d'introduire un couplage avec le champ de vitesse.}
\item {Le flux au bord \var{VISCB} est annul\'e dans le sous-programme
\fort{vectds}. Peut-on envisager de mettre au bord la valeur de la variable
concern\'ee \`a la cellule de bord correspondant? De m\^eme, il faudrait se
pencher sur les hypoth\`eses sous-jacentes \`a l'annulation des contributions
aux bords de \var{VISCB} lors du calcul de : $$\left[ \tens{D}^n\,\left( \grad{R^{\,n}_{ij}} - (\grad R^{\,n}_{ij}\,.\,\vect{n}_{\,lm})\,\vect{n}_{\,lm}\right) \right]\,.\,\vect{n}_{\,lm}.$$}
\item {Un probl\`eme de pond\'eration appara\^\i t plus g\'en\'eralement. Si on prend la partie explicite de $\tens{D}\,\grad(\phi)$, la pond\'eration aux faces internes utilise le coefficient $\displaystyle\frac{1}{2}$ avec pond\'eration s\'epar\'ee de $\tens{D}$ et $\grad(\phi)$, alors que pour $\tens{E}\,\grad(\phi)$, on calcule d'abord ce terme aux cellules pour ensuite l'interpoler lin\'eairement aux faces \footnote{Cette interpolation se fait dans \fort{vectds} avec des coefficients de pond\'eration aux faces.}. Ceci donne donc deux types d'interpolations pour des termes de m\^eme nature.}
\item {On laisse la possibilit\'e dans \fort{visort} d'utiliser une moyenne
harmonique aux faces. Est-ce que ceci est valable puisque les interpolations
utilis\'ees lors du calcul de la partie explicite de $\tens{A}\,\grad{\phi}$
sont des moyennes arithm\'etiques ?}
\item {Les techniques adopt\'ees lors du clipping sont \`a revoir.}
\item {On utilise dans le cadre du mod\`ele $\displaystyle R_{ij}-\varepsilon $ une semi-implicitation de termes comme $\displaystyle \phi_{ij,1}$ ou $\displaystyle -\rho\,C_{\varepsilon_2}\,\frac{\varepsilon}{k}\,\varepsilon$. On peut envisager le m\^eme type d'implicitation dans \fort{turbke} m\^eme en pr\'esence du couplage $\displaystyle k-\varepsilon$.}
\item L'adoption d'une r\'esolution d\'ecoupl\'ee fait perdre l'invariance par rotation.
\item La formulation et l'implantation des conditions aux limites de paroi
devront \^etre v\'erifi\'ees. En effet, il semblerait que, dans certains cas, des ph\'enom\`enes
``oscillatoires'' apparaissent, sans qu'il soit ais\'e d'en d\'eterminer la cause.
\item L'implicitation partielle (du fait de la r\'esolution d\'ecoupl\'ee) des
conditions aux limites conduit souvent \`a des calculs instables. Il
conviendrait d'en conna\^\i tre la raison. L'implicitation partielle avait
\'et\'e mise en \oe uvre afin de tenter d'utiliser un pas de temps plus grand
dans le cas de jets axisym\'etriques en particulier.

\end{itemize}

%                      Code_Saturne version 1.3
%                      ------------------------
%
%     This file is part of the Code_Saturne Kernel, element of the
%     Code_Saturne CFD tool.
%
%     Copyright (C) 1998-2007 EDF S.A., France
%
%     contact: saturne-support@edf.fr
%
%     The Code_Saturne Kernel is free software; you can redistribute it
%     and/or modify it under the terms of the GNU General Public License
%     as published by the Free Software Foundation; either version 2 of
%     the License, or (at your option) any later version.
%
%     The Code_Saturne Kernel is distributed in the hope that it will be
%     useful, but WITHOUT ANY WARRANTY; without even the implied warranty
%     of MERCHANTABILITY or FITNESS FOR A PARTICULAR PURPOSE.  See the
%     GNU General Public License for more details.
%
%     You should have received a copy of the GNU General Public License
%     along with the Code_Saturne Kernel; if not, write to the
%     Free Software Foundation, Inc.,
%     51 Franklin St, Fifth Floor,
%     Boston, MA  02110-1301  USA
%
%-----------------------------------------------------------------------
%
\programme{vortex}
%
\vspace{1cm}
%%%%%%%%%%%%%%%%%%%%%%%%%%%%%%%%%%
%%%%%%%%%%%%%%%%%%%%%%%%%%%%%%%%%%
\section{Fonction}
%%%%%%%%%%%%%%%%%%%%%%%%%%%%%%%%%%
%%%%%%%%%%%%%%%%%%%%%%%%%%%%%%%%%%
Ce sous-programme est d�di� � la g�n�ration des conditions d'entr�e
turbulente utilis�es en LES.


La m�thode des vortex est bas�e sur une approche de tourbillons
ponctuels. L'id�e de la m�thode consiste � injecter des tourbillons 2D dans le
plan d'entr�e du calcul, puis � calculer le champ de vitesse induit par ces
tourbillons au centre des faces d'entr�e.

%                      Code_Saturne version 1.3
%                      ------------------------
%
%     This file is part of the Code_Saturne Kernel, element of the
%     Code_Saturne CFD tool.
% 
%     Copyright (C) 1998-2007 EDF S.A., France
%
%     contact: saturne-support@edf.fr
% 
%     The Code_Saturne Kernel is free software; you can redistribute it
%     and/or modify it under the terms of the GNU General Public License
%     as published by the Free Software Foundation; either version 2 of
%     the License, or (at your option) any later version.
% 
%     The Code_Saturne Kernel is distributed in the hope that it will be
%     useful, but WITHOUT ANY WARRANTY; without even the implied warranty
%     of MERCHANTABILITY or FITNESS FOR A PARTICULAR PURPOSE.  See the
%     GNU General Public License for more details.
% 
%     You should have received a copy of the GNU General Public License
%     along with the Code_Saturne Kernel; if not, write to the
%     Free Software Foundation, Inc.,
%     51 Franklin St, Fifth Floor,
%     Boston, MA  02110-1301  USA
%
%-----------------------------------------------------------------------
%
%%%%%%%%%%%%%%%%%%%%%%%%%%%%%%%%%%
%%%%%%%%%%%%%%%%%%%%%%%%%%%%%%%%%%
\section{Discr\'etisation}
%%%%%%%%%%%%%%%%%%%%%%%%%%%%%%%%%%
%%%%%%%%%%%%%%%%%%%%%%%%%%%%%%%%%%

Le terme convectif en $\dive(\underline{u} \otimes \rho\,\underline{u})$
introduit une non lin\'earit\'e et un couplage des composantes de la vitesse
$\vect{u}$ dans l'�quation (\ref{Base_Preduv_eqqdm}). Une lin\'earisation et un d\'ecouplage
des trois composantes de la 
vitesse sont r\'ealis\'es lors de la discr\'etisation de cette \'etape de
pr\'ediction.\\
En effet, soit :
\begin{equation}
\vect{\widetilde{u}}= \vect{u}^n + \delta \vect{u} 
\end{equation}
La contribution exacte du terme convectif \`a prendre en compte dans cette
\'etape de pr\'ediction serait :\\
\begin{equation}\label{Base_Preduv_Conv_exact}
\begin{array}{ll}
\dive(\vect{\widetilde{u}} \otimes \rho\,\vect{\widetilde{u}}) =
\dive(\vect{u}^{n} \otimes \rho\,\vect{u}^{n}) + \dive(\delta \vect{u} \otimes
\rho\,\vect{u}^{n}) +  \underbrace { \dive(\vect{u}^{n} \otimes
\rho\,\delta \vect{u})}_{\text {terme couplant lin\'eaire}} +  \underbrace { \dive(\delta \vect{u} \otimes
\rho\,\delta \vect{u})}_{\text {terme couplant et non lin\'eaire}}\\
\end{array} 
\end{equation}
Les deux derniers termes de l'expression (\ref{Base_Preduv_Conv_exact}) sont {\em a priori} n�glig�s
de mani�re � obtenir un syst\`eme en vitesse qui soit d\'ecoupl\'e et donc,
�viter l'inversion d'une matrice pouvant \^etre de tr\`es grande taille. Ces
deux termes peuvent n�anmoins �tre pris en compte de mani�re plus ou moins
approch�e par extrapolation explicite du flux de masse en $n+\theta_F$ (pour le
terme couplant lin�aire provenant de la convection de $\vect{u}^{n}$ par $\delta
\vect{u}$) et utilisation d'un point-fixe par sous it�ration sur le sous
programme \fort{navsto} (pour le terme non-lin�aire, en sp�cifiant $\var{NTERUP}>1$).

L'�quation (\ref{Base_Preduv_eqqdm}) est discr�tis�e au temps $n+\theta$ � l'aide d'un
$\theta$-sch�ma, et le tenseur des pertes de charges d�compos� en une partie
diagonale $\tens{K}_{d}$ et une extradiagonale $\tens{K}_{e}$ (soit
 $\tens{K}_{pdc}=\tens{K}_{d}+\tens{K}_{e}$).\\
$\bullet$ La pression est suppos�e connue en $n-1+\theta$ (d�calage temporel
pression-vitesse) et le gradient naturellement calcul� � cet instant.\\ 
$\bullet$ Les termes sources de viscosit� secondaire, de gradient transpos\'e,
ceux provenant du mod�le de turbulence\footnote{except� $\dive (\mu_t\ (\ggrad
\underline {u}))$}, $\rho\,\tens{K}_{\,e}\ \underline{u}$, $(\rho -\rho_0)
\underline {g}$ ainsi que $\underline{T}_{s}^{\,exp}$ et
$\Gamma\,\underline{u}_{\,i}$ sont pris de mani�re explicite au temps $n$, ou
extrapol�s suivant le sch�ma en temps choisi pour les propri�t�s physique et les
termes sources.\\ 
$\bullet$ Les termes sources $\underline{u}\,\,\dive (\rho\,\underline {u})$,
$\Gamma\,\,\underline{u}$, $T_{s}^{\,imp}\,\,\underline{u}$ et
$-\rho\,\tens{K}_{\,d}\,\,\underline{u}$ sont implicit�s est calcul�s �
l'instant $n+\theta$.\\ 
$\bullet$ Le terme de diffusion $\dive (\mu_{\,tot}\,\ggrad \underline{u})$ est
implicit� : la vitesse est prise � l'instant $n+\theta$ et la viscosit�
explicit�e ou extrapol�e.\\ 
$\bullet$ Enfin, le terme de convection en $\dive(\,\underline{u} \otimes
(\rho\underline{u})\,)$ est implicit� : la composante r�solue de la vitesse est
prise en $n+\theta$, et le flux de masse, explicit�, ou extrapol� en
$n+\theta_F$. 

Par souci de clart�, on suppose, en l'absence d'indication, que les propri�tes
physiques $\Phi$ ($\rho,\,\mu_{tot},\,...$) et le flux de masse
$(\rho\underline{u})$ sont pris respectivement aux instants $n+\theta_\Phi$ et
$n+\theta_F$, o� $\theta_\Phi$ et $\theta_F$ d�pendent des sch�mas en temps
sp�cifiquement utilis�s pour ces grandeurs\footnote{cf. \fort{introd}}. 

La discr�tisation temporelle de l'�quation (\ref{Base_Preduv_eqqdm}) s'�crit alors comme suit : 

\begin{equation}\label{Base_Preduv_eq_di1}
 \begin{array}{c}
\displaystyle \rho\,\ \frac{ \underline {\widetilde{u}}^{n+1} -\underline {u}^{n} }
{\Delta t} + \dive(\,\underline{\widetilde{u}}^{n+\theta} \otimes (\rho\underline{u})\,) -\dive
(\mu_{\,tot}\,\ggrad \underline{\widetilde{u}}^{n+\theta}) =
\\
\displaystyle
 - \grad p^{n-1+\theta} + \dive (\rho\,\underline {u})\,\underline{\widetilde{u}}^{n+\theta} +(\Gamma\,\underline{u}_{\,i})^{n+\theta_S}-\Gamma^n\,\,\underline{\widetilde{u}}^{n+\theta}
\\
\begin{array}{c}
\displaystyle
- \rho\,\tens{K}_{\,d}^{n}\,\,\underline{\widetilde{u}}^{n+\theta} - (\rho\,\tens{K}_{\,e}\ \underline{u})^{n+\theta_S} + (\underline{T}_{s}^{\,exp})^{\,n+\theta_S} + T_{s}^{\,imp}\,\,\underline{\widetilde{u}}^{n+\theta}
\\
\displaystyle
+[\dive (\mu_{\,tot}\,^t\ggrad \underline {u})]^{n+\theta_S}-\frac {2} {3}[\,\grad (\mu_{\,tot}\,\dive \underline {u})]^{n+\theta_S} + (\rho -\rho_0) \underline {g}
 - (\underline{turb})^{n+\theta_S}
\end{array}
\end{array}
\end{equation}
o\`u, par souci de simplification, on a pos\'e :
\begin{equation}
\mu_{\,tot}=
\begin{cases}
\mu+\mu_t & \text{pour les mod�les � viscosit� turbulente ou en LES}, \\
\mu & \text{pour les mod�les au second ordre ou en laminaire}
\end{cases} \ 
\end{equation}
\\
et :
\begin{equation}
\underline{turb}^{n}=
\begin{cases}
\displaystyle\frac {2}{3}\grad (\rho^{n}\,k^{n}) & \text{pour les mod�les � viscosit� turbulente}, \\
\dive(\rho^{n}\,\tens{R}^n) & \text{pour les mod�les au second ordre},\\
0 & \text{en laminaire ou en LES}\\
\end{cases}
\end{equation}
Par analogie avec l'�criture du $\theta$-sch�ma pour une variable scalaire, $\,
\underline {\widetilde{u}}^{n+\theta}$ est interpol�e � partir de la vitesse
pr�dite $\underline {\widetilde{u}}^{n+1}$ de la mani\`ere suivante\footnote{si
$\theta=1/2$, ou qu'une extrapolation est utilis�e, l'ordre 2 n'est obtenu que si
le pas de temps $\Delta t$ est uniforme en temps et en espace.}~: 
\begin{equation}
\underline {\widetilde{u}}^{n+\theta}=\theta\, \underline
{\widetilde{u}}^{n+1}+(1-\theta)\, \underline {u}^{n}\\ 
\end{equation}
Avec :
\begin{equation}
\left\{
\begin{array}{ll}
\theta = 1   & \text{Pour un sch\'ema de type Euler implicite d'ordre 1.}\\
\theta = 1/2 & \text{Pour un sch\'ema de type Cranck-Nicolson d'ordre 2.}\\
\end{array}
\right.
\end{equation}

L'�quation (\ref{Base_Preduv_eq_di1}) est alors r��crite sous la forme :

\begin{equation}\label{Base_Preduv_eq_di2}
\begin{array}{c}
\displaystyle \underbrace{\left(\frac{\rho}{\Delta t} -\theta \,\dive (\rho\,\underline {u}) +\theta \,\, \Gamma^n +
\theta \,\, \rho\,\tens{K}_{\,d}^n-\theta \,T_s^{\,imp} \right)}_{\displaystyle f_s^{imp}}\, (\underline {\,\widetilde{u}}^{n+1} -\underline {u}^{n})
\\
 +\, \theta\, \dive(\underline {\widetilde{u}}^{n+1} \otimes (\rho\underline{u}))-\, \theta\,\dive (\mu_{\,tot}\,\ggrad \underline {\widetilde{u}}^{n+1}) =
\\
-\,(1-\theta)\, \dive(\underline {u}^{n} \otimes (\rho\underline{u})) +\,(1-\theta)\,\dive (\mu_{\,tot}\,\ggrad \underline {u}^{n})
\\
f_s^{exp}\left\{
\begin{array}{c}
\displaystyle 
- \grad p^{n-1+\theta} + \dive (\rho\,\underline {u})\,\underline{u}^{n} +\,(\,\Gamma^{n}\,\underline{u}_{\,i}\,)^{n+\theta_S}- \Gamma^n\,\,\underline{u}^{n}
\\
\displaystyle
-(\,\rho\,\tens{K}_{\,e}\ \underline{u}\,)^{n+\theta_S} -\rho\,\tens{K}_{\,d}^n\ \underline{u}^{n}+ (\underline{T}_{s}^{\,exp})^{\,n+\theta_S} + T_s^{\,imp}\,\,\underline {u}^{n} 
\\
\displaystyle
+[\dive (\mu_{\,tot}\,^t\ggrad \underline {u}\,)]^{n+\theta_S}-\frac {2} {3}[\,\grad (\mu_{\,tot}\,\dive \underline {u}\,)]^{n+\theta_S} + (\rho -\rho_0) \underline {g}-(\underline{turb})^{n+\theta_S}
\end{array}
\right.
\end{array}
\end{equation}

d'o� l'�quation r�solue par le sous-programme \fort{codits} :
\begin{equation}\begin{array}{c}
\displaystyle
f_s^{\,imp}(\underline {\widetilde{u}}^{n+1}-\underline {u}^{n}) + \theta\, \dive(\underline{\widetilde{u}}^{n+1} \otimes (\rho
\underline{u})) - \theta\,\dive (\,\mu_{\,tot}\,\ggrad \underline{\widetilde{u}}^{n+1}) = 
\\\\
\displaystyle
-(1-\theta)\,\dive(\underline{u}^{n} \otimes (\rho \underline{u}))+(1-\theta)\,\dive (\,\mu_{\,tot}\,\ggrad \underline{u}^{n})
+ \underline{f}_{\,s}^{\,exp}
\end{array}
\end{equation}
La m\'ethode de discr\'etisation spatiale est d\'evelopp\'ee dans le sous-programme \fort{codits}.\\



\minititre{Remarques :}
{\tiny$\blacksquare$} Dans le cas standard sans extrapolation, le terme
$-\,T_s^{\,imp}$ n'est ajout� � $f_s^{\,imp}$ que s'il est positif afin de ne
pas affaiblir la dominance de la diagonale de la matrice � inverser.\\ 
{\tiny$\blacksquare$} Si une extrapolation est utilis�e, par contre,
$\,T_s^{\,imp}$ est ajout� � $f_s^{\,imp}$ quel que soit son signe. En effet, l'id�e intuitive qui
consiste � prendre~: 
\begin{equation}
\begin{cases}
\displaystyle
(\underline{T}_{s}^{\,exp} + T_{s}^{\,imp}\,\underline {u})^{\,n+\theta_S} &
\text{si } T_{s}^{\,imp} > 0\\ 
\displaystyle
(\underline{T}_{s}^{\,exp})^{\,n+\theta_S} + T_{s}^{\,imp}\,\underline{u}^{n+\theta} &\text{sinon}\\
\end{cases}
\end{equation} 
aboutit � une incoh�rence dans le traitement si $T_s^{imp}$ change de signe
entre deux pas de temps.\\ 
{\tiny$\blacksquare$} la partie diagonale $\tens{K}_{\,d}$ du terme
de perte de charge est utilis�e dans $f_s^{\,imp}$. En fait, pour \^etre rigoureux,
il faudrait ne retenir que les contributions positives (point signal\'e dans le
sous-programme utilisateur associ\'e \fort{uskpdc}). Cette prise en compte sera \`a am\'eliorer.\\
{\tiny$\blacksquare$} Le terme $\theta\,\Gamma^{n}-\theta\,\dive
(\rho\,\underline {u})$ ne pose pas de probl�me pour la 
dominance de la diagonale de la matrice car il est exactement compens� par le
terme de convection (cf. \fort{covofi}). 


%                      Code_Saturne version 1.3
%                      ------------------------
%
%     This file is part of the Code_Saturne Kernel, element of the
%     Code_Saturne CFD tool.
%
%     Copyright (C) 1998-2007 EDF S.A., France
%
%     contact: saturne-support@edf.fr
%
%     The Code_Saturne Kernel is free software; you can redistribute it
%     and/or modify it under the terms of the GNU General Public License
%     as published by the Free Software Foundation; either version 2 of
%     the License, or (at your option) any later version.
%
%     The Code_Saturne Kernel is distributed in the hope that it will be
%     useful, but WITHOUT ANY WARRANTY; without even the implied warranty
%     of MERCHANTABILITY or FITNESS FOR A PARTICULAR PURPOSE.  See the
%     GNU General Public License for more details.
%
%     You should have received a copy of the GNU General Public License
%     along with the Code_Saturne Kernel; if not, write to the
%     Free Software Foundation, Inc.,
%     51 Franklin St, Fifth Floor,
%     Boston, MA  02110-1301  USA
%
%-----------------------------------------------------------------------
%

%%%%%%%%%%%%%%%%%%%%%%%%%%%%%%%%%%
%%%%%%%%%%%%%%%%%%%%%%%%%%%%%%%%%%
\section{Mise en \oe uvre}
%%%%%%%%%%%%%%%%%%%%%%%%%%%%%%%%%%
%%%%%%%%%%%%%%%%%%%%%%%%%%%%%%%%%%
La num\'ero de la phase trait\'ee fait partie des arguments de \fort{turrij}. On
omettra volontairement de le pr\'eciser dans ce qui suit, on indiquera par $(\ )$ la
notion de tableau s'y rattachant.

\etape{Calcul des termes de production $\tens{\mathcal{P}}$}
\begin{itemize}
\item [$\star$] Initialisation \`a z\'ero du tableau \var{PRODUC} dimensionn\'e \`a $\var{NCEL}\times 6$.
\item [$\star$] On appelle trois fois \fort{grdcel} pour calculer les gradients des composantes de la vitesse $u$, $v$ et
$w$ prises au temps $n$.

Au final, on a :\\
$\displaystyle
\begin{array} {ll}
\var{PRODUC(1,IEL)} = & \displaystyle - 2 \left[ R_{11}^{\,n} \frac{\partial u^{\,n}} {\partial x} +R_{12}^{\,n} \frac{\partial u^{\,n}} {\partial y}+R_{13}^{\,n} \frac{\partial u^{\,n}} {\partial z} \right] \text{        (production de $R_{11}^{\,n}$)}\\
\var{PRODUC(2,IEL)} = & \displaystyle - 2 \left[ R_{12}^{\,n} \frac{\partial v^{\,n}} {\partial x} +R_{22}^{\,n} \frac{\partial v^{\,n}} {\partial y}+R_{23}^{\,n} \frac{\partial v^{\,n}} {\partial z} \right] \text{        (production de $R_{22}^{\,n}$)}\\
\var{PRODUC(3,IEL)} = & \displaystyle - 2 \left[ R_{13}^{\,n} \frac{\partial w^{\,n}} {\partial x} +R_{23}^{\,n} \frac{\partial w^{\,n}} {\partial y}+R_{33}^{\,n} \frac{\partial w^{\,n}} {\partial z} \right] \text{        (production de $R_{33}^{\,n}$)}\\
\var{PRODUC(4,IEL)} = & \displaystyle - \left[ R_{12}^{\,n} \frac{\partial u^{\,n}} {\partial x} +R_{22}^{\,n} \frac{\partial u^{\,n}} {\partial y}+R_{23}^{\,n} \frac{\partial u^{\,n}} {\partial z} \right] \\
& \displaystyle - \left[ R_{11}^{\,n} \frac{\partial v^{\,n}} {\partial x} +R_{12}^{\,n} \frac{\partial v^{\,n}} {\partial y}+R_{13}^{\,n} \frac{\partial v^{\,n}} {\partial z} \right] \text{        (production de $R_{12}^{\,n}$)} \\
\var{PRODUC(5,IEL)} = & \displaystyle - \left[ R_{13}^{\,n} \frac{\partial u^{\,n}} {\partial x} +R_{23}^{\,n} \frac{\partial u^{\,n}} {\partial y}+R_{33}^{\,n} \frac{\partial u^{\,n}} {\partial z} \right] \\
& \displaystyle - \left[ R_{11}^{\,n} \frac{\partial w^{\,n}} {\partial x} +R_{12}^{\,n} \frac{\partial w^{\,n}} {\partial y}+R_{13}^{\,n} \frac{\partial w^{\,n}} {\partial z} \right] \text{        (production de $R_{13}^{\,n}$)} \\
\var{PRODUC(6,IEL)} = & \displaystyle - \left[ R_{13}^{\,n} \frac{\partial v^{\,n}} {\partial x} +R_{23}^{\,n} \frac{\partial v^{\,n}} {\partial y}+R_{33}^{\,n} \frac{\partial v^{\,n}} {\partial z} \right] \\
& \displaystyle - \left[ R_{12}^{\,n} \frac{\partial w^{\,n}} {\partial x} +R_{22}^{\,n} \frac{\partial w^{\,n}} {\partial y}+R_{23}^{\,n} \frac{\partial w^{\,n}} {\partial z} \right]  \text{        (production de $R_{23}^{\,n}$)}
\end{array}
$
\end{itemize}

\etape{Calcul du gradient de la masse volumique $\rho^n$ prise au d\'ebut du pas
de temps courant\footnote{{\it i.e.} calcul\'ee \`a partir des
variables du pas de temps pr\'ec\'edent $n$ si n\'ecessaire.} $(n+1)$}
Ce calcul n'a lieu que si les termes de gravit\'e doivent \^etre pris en compte
($\var{IGRARI()} =1$).
\begin{itemize}
\item [$\star$] Appel de \fort{grdcel}  pour calculer le gradient de $\rho^n$
dans les trois directions de l'espace. Les conditions aux limites sur $\rho^n$
sont des conditions de Dirichlet puisque la valeur de $\rho^n$ aux faces de bord
$ik$ (variable \var{IFAC}) est connue et vaut $\rho_{\,b_{\,ik}}$. Pour \'ecrire les conditions aux limites
sous la forme habituelle, $$\rho_{\,b_{\,ik}} = \var{COEFA} + \var{COEFB}
\,\rho^n_{\,I'}$$ on pose alors $\var{COEFA}=
\var{PROPCE(IFAC,IPPROB(IROM(IPHAS)))}$ et $\var{COEFB} = \var{VISCB} = 0$.\\
$\var{PROPCE(1,IPPROB(IROM(IPHAS)))}$ (resp.$\var{VISCB}$) est utilis\'e en lieu
et place de l'habituel \var{COEFA} ($\var{COEFB}$), lors de l'appel \`a \fort{grdcel}.\\
On a donc :\\
$\displaystyle \var{GRAROX}= \frac{\partial \rho^n}{\partial x}\ $,$\displaystyle \ \var{GRAROY}= \frac{\partial
\rho^n}{\partial y}$ et $
\displaystyle \ \var{GRAROZ}= \frac{\partial \rho^n}{\partial z}\ $.

\end{itemize}

Le gradient de $\rho^n$ servira \`a calculer les termes de production par effets de gravit\'e si ces derniers sont pris en compte.

\etape{Boucle \var{ISOU} de $1$ \`a $6$ sur les tensions de Reynolds}
Pour $\var{ISOU} = 1,2,3,4,5,6$, on r\'esout respectivement et dans
l'ordre  les
\'equations de $R_{11}$, $R_{22}$, $R_{33}$, $R_{12}$, $R_{13}$ et $R_{23}$ par
l'appel au sous-programme \fort{resrij}.\\
La r\'esolution se fait par incr\'ement $\delta {R}_{ij}^{\,n+1,k+1}$ , en utilisant la m\^eme m\'ethode que
celle d\'ecrite dans le sous-programme \fort{codits}. On adopte ici les m\^emes notations.
\var{SMBR} est le second membre du syst\`eme \`a inverser, syst\`eme portant sur
les incr\'ements de la variable. \var{ROVSDT} repr\'esente la diagonale de la
matrice, hors convection/diffusion.\\
On va r\'esoudre l'\'equation (\ref{Base_Turrij_Eq_Temp_Rij}) sous forme incr\'ementale en
utilisant \fort{codits}, soit :
\begin{equation}\label{Base_Turrij_Eq_Temp_deltaRij}
\begin{array}{ll}
&\displaystyle \underbrace{\left(\frac {\rho^n_L}{\Delta t^n}
+ \rho^n_L \,C_1\,\frac{\varepsilon^n_L}{k^n_L}(1-\frac{\delta_{ij}}{3})
 - m^n_{\,lm} + \Gamma_L\,+ max(-\alpha^n_{R_{ij}},0)\right)\,|\Omega_l|}
_{\text {$\var{ROVSDT}$ contribuant
\`a la diagonale de la matrice simplifi\'ee de \fort{matrix}}}\,(\delta{R}_{ij}^{\,n+1,p+1})_{\,L}\\\\
&  \underbrace{+\sum\limits_{m\in Vois(l)}\displaystyle \left[
 m^n_{\,lm} \delta R_{ij,\,f_{\,lm}}^{\,n+1,p+1}
- (\mu^n_{\,lm} + \gamma^n_{\,lm})\
\frac{({\delta R}_{ij}^{\,n+1,p+1})_{M}-({\delta R}_{ij}^{\,n+1,p+1})_{L})}{\overline{L'M'}}\,
S_{\,lm} \right]}_{\text { convection upwind pur et diffusion non reconstruite
relatives \`a la matrice simplifi\'ee de \fort{matrix}\footnotemark}} \\
% voir le texte de la footmark plus bas
&= - \displaystyle\frac {\rho^n_L}{\Delta t^n}\,\left(\,(R^{\,n+1,p}_{ij})_L - (R^{\,n}_{ij})_L\,\right)\\
&-\,\underbrace{\displaystyle\int_{\Omega_l} \left(
\dive\,[\,(\rho\,\vect{u})^n\,R^{\,n+1,p}_{ij} - (\mu^n\,+ \gamma^n\,)\,
\grad{R^{\,n+1,p}_{ij}}\,]\right)\,d\Omega}_{\text {convection et diffusion
trait\'ees par \fort{bilsc2}}}\\
&+\displaystyle \int_{\Omega_l} \left[\,\mathcal{P}^{\,n+1,p}_{ij} + \mathcal{G}^{\,n+1,p}_{ij}
- \displaystyle\rho^n \,C_1\,\frac{\varepsilon^n}{k^n}\left[R^{\,n+1,p}_{ij}-
\frac{2}{3}\,k^n\,\delta_{ij}\right] + \phi^{\,n+1,p}_{ij,2} +
\phi^{\,n+1,p}_{ij,w}\,\right]\, d\Omega \\
& + \displaystyle\int_{\Omega_l} \left[- \frac{2}{3} \rho^n \varepsilon^n \delta_{ij}
 + \Gamma\,(\,R^{\,in}_{ij} - R^{\,n+1,p}_{ij}\,) +
\alpha^n_{R_{ij}}\,R^{\,n+1,p}_{ij}+ \beta^n_{R_{ij}}\right]\, d\Omega\\
&+ \sum\limits_{m\in
Vois(l)}\displaystyle \left[\ \tens{E}^n\,\grad{R}^{\,n+1,p}_{ij} \right]_{\,lm}\,.\,\vect{n}_{\,lm}S_{\,lm}\\
&+ \sum\limits_{m\in Vois(l)}\displaystyle \left[\
\tens{D}^n\,\grad{R}^{\,n+1,p}_{ij} \right]_{\,lm}\,.\,\vect{n}_{\,lm}S_{\,lm}\\
&- \sum\limits_{m\in Vois(l)} \gamma^n_{\,lm} \left( \grad{R}^{\,n+1,p}_{ij}\,.\,\vect{n}_{\,lm} \right)  S_{\,lm}\\
&+ \sum\limits_{m\in Vois(l)}  m^n_{\,lm}\,(R^{\,n+1,p}_{ij})_L\\
\end{array}
\end{equation}
% si on ne fait pas comme ca, il n'apparait pas
\footnotetext[\thefootnote]{Si $\var{IRIJNU} = 1$, on remplace  $\mu^n_{\,lm}$ par $(\mu +
\mu_t)^n_{\,lm}$ dans l'expression de la diffusion non reconstruite
associ\'ee \`a la matrice simplifi\'ee de \fort{matrix} ($\mu_t$ d\'esigne la
viscosit\'e turbulente calcul\'ee comme en $k-\varepsilon$).}

o\`u on rappelle :\\
pour $n$ donn\'e entier positif, on d\'efinit la suite
 $({R}_{ij}^{\,n+1,p})_{p \in \grandN}$
 par :
\begin{equation}\notag
\left\{\begin{array}{l}
{R}_{ij}^{\,n+1,0} = {R}_{ij}^{\,n}\\
{R}_{ij}^{\,n+1,p+1} = {R}_{ij}^{\,n+1,p} + \delta{R}_{ij}^{\,n+1,p+1} \\
\end{array}\right.
\end{equation}
$(\delta{R}_{ij}^{\,n+1,p+1})_{\,L}$ d\'esigne la valeur sur l'\'el\'ement
$\Omega_l$ du $\text{$(\,p+1\,)$-i\`eme}$ incr\'ement de ${R}_{ij}^{\,n+1}$,
$ m^n_{\,lm}$ le flux de masse \`a l'instant $n$ \`a travers la face $lm$,
$\delta R_{ij,\,f_{\,lm}}^{\,n+1,p+1}$ vaut $({\delta
R}_{ij}^{\,n+1,p+1})_{L}$  si $ m^n_{\,lm} \geqslant 0$, $({\delta
R}_{ij}^{\,n+1,p+1})_{M}$ sinon,
$\mathcal{P}^{\,n+1,p}_{ij}$, $\phi^{\,n+1,p}_{ij,2}$, $\phi^{\,n+1,p}_{ij,w}$ les valeurs
des quantit\'es associ\'ees correspondant \`a l'incr\'ement
$(\delta{R}_{ij}^{\,n+1,p})$.\\



Tous ces termes sont calcul\'es comme suit :
\begin{itemize}
\item Terme de gauche de l'\'equation (\ref{Base_Turrij_Eq_Temp_deltaRij})\\
Dans \fort{resrij} est calcul\'ee la variable \var{ROVSDT}. Les autres
termes sont compl\'et\'es par \fort{codits}, lors de la construction de la matrice simplifi\'ee , {\it via} un
appel au sous-programme \fort{matrix}. La quantit\'e
 $(\mu^n_{\,lm} + \gamma^n_{\,lm})$ \`a la face $lm$ est calcul\'ee lors de l'appel \`a
\fort{visort}.\\
\item Second membre de l'\'equation (\ref{Base_Turrij_Eq_Temp_deltaRij})\\
Le premier terme non d\'etaill\'e est calcul\'e par le sous-programme
\fort{bilsc2}, qui applique le sch\'ema convectif choisi par l'utilisateur, qui
reconstruit ou non selon le souhait de l'utilisateur les gradients intervenants
dans la convection-diffusion.\\
Les termes sans accolade sont, eux, compl\`etement explicites et ajout\'es au fur et
\`a mesure dans \var{SMBR} pour former
l'expression $f^{\,exp}_s$ de \fort{codits}.
\end{itemize}
On d\'ecrit ci-dessous les \'etapes de \fort{resrij} :
\begin{itemize}

\item DELTIJ = 1, pour $\var{ISOU} \leqslant 3$ et DELTIJ = 0  Si $\var{ISOU} >
3$. Cette valeur repr\'esente le symbole de Kroeneker $\delta_{ij}$.

\item Initialisation \`a z\'ero de \var{SMBR} (tableau contenant le second
membre) et \var{ROVSDT} (tableau contenant la diagonale de la matrice sauf celle
relative \`a la contribution de la
diagonale des op\'erateurs de convection et de diffusion lin\'earis\'es
\footnote{qui correspondent aux sch\'emas convectif upwind pur et diffusif sans
reconstruction.}), tous deux de dimension $\var{NCEL}$.

\item Lecture et prise en compte des termes sources utilisateur pour la variable $R_{ij}$

Appel \`a \fort{ustsri} pour charger les termes sources utilisateurs. Ils sont
stock\'es comme suit. Pour la cellule $\Omega_l$ de centre $L$, repr\'esent\'ee par $\var{IEL}$, on a :\\
\begin{equation}\notag
\left\{\begin{array}{lll}
&\var{ROVSDT(IEL)}&= |\Omega_l| \ \alpha_{R_{ij}}\\
&\var{SMBR(IEL)}&=|\Omega_l| \ \beta_{R_{ij}}\\
\end{array}\right.
\end{equation}
On affecte alors les valeurs ad\'equates au second membre \var{SMBR} et \`a la
diagonale \var{ROVSDT} comme suit :
\begin{equation}\notag
\left\{\begin{array}{lll}
&\var{SMBR(IEL)} &= \var{SMBR(IEL)} +\ |\Omega_l| \ \alpha_{R_{ij}} \ (R^n_{ij})_L \\
&\var{ROVSDT(IEL)}&= \text{max }(-\ |\Omega_l| \ \alpha_{R_{ij}},0)\\
\end{array}\right.
\end{equation}
La valeur de $ \var{ROVSDT}$ est ainsi calcul\'ee pour des raisons de stabilit\'e
num\'erique. En effet, on ne rajoute sur la diagonale que les valeurs positives,
ce qui correspond physiquement \`a impliciter les termes de rappel uniquement.
\item{Calcul du terme source de masse  si $\Gamma_L > 0$}

Appel de \fort{catsma} et incr\'ementation si n\'ecessaire de \var{SMBR} et
\var{ROVSDT} {\it via} :\\
\begin{equation}\notag
\left\{\begin{array}{lll}
\displaystyle \var{SMBR(IEL)} = \var{SMBR(IEL)} + |\Omega_l| \ \Gamma_L \
\left[(R^{\,in}_{ij})_L - (R^{\,n}_{ij})_L \right] \\
\displaystyle \var{ROVSDT(IEL)}=\var{ROVSDT(IEL)} + |\Omega_l| \ \Gamma_L
\end{array}\right.
\end{equation}
\item Calcul du terme d'accumulation de masse et du terme instationnaire

On stocke $\displaystyle \var{W1}= \int_{\Omega_l}\dive\,(\rho\,\vect{u})\,d\Omega$
calcul\'e par \fort{divmas} \`a l'aide des flux de masse aux faces internes
$ m^n_{\,lm}=\sum\limits_{m\in Vois(l)}{(\rho \vect{u})_{\,lm}^n} \text{.}\,
\vect{S}_{\,lm} $ (tableau \var{FLUMAS}) et des flux de masse aux bords  $ m^n_{\,b_{lk}} = \sum\limits_{k\in{\gamma_b(l)}}{(\rho \vect{u})_{\,{b}_{lk}}^n} \text{.}\,
\vect{S}_{\,{b}_{lk}} $ (tableau \var{FLUMAB}).
On incr\'emente ensuite \var{SMBR} et \var{ROVSDT}.
\begin{equation}\notag
\left\{\begin{array}{lll}
&\var{SMBR(IEL)} &= \var{SMBR(IEL)} + \var{ICONV}\  (R^n_{ij})_L\,(\displaystyle
\int_{\Omega_l}\dive\,(\rho\,\vect{u})\ d\Omega) \\
&\var{ROVSDT(IEL)}& = \var{ROVSDT(IEL)} +  \var{ISTAT}\,\displaystyle
\frac{\rho^n_L \ |\Omega_l|}{\Delta t^n} -  \var{ICONV}\ (\displaystyle
\int_{\Omega_l}\dive\,(\rho\,\vect{u})\ d\Omega) \\
\end{array}\right.
\end{equation}
\item Calcul des termes sources de production, des termes $\displaystyle
\phi_{\,ij,1}+\phi_{\,ij,2}$ et de la dissipation~$\displaystyle-\frac{2}{3} \varepsilon\,\delta_{\,ij}$ :

On effectue une boucle d'indice \var{IEL} sur les cellules $\Omega_l$ de centre $L$ :
\begin{itemize}
\item [$\Rightarrow$] $\displaystyle \var{TRPROD}= \frac{1}{2} (\mathcal{P}^n_{ii})_L = \frac{1}{2} \left[ \var{PRODUC(1,IEL)} +  \var{PRODUC(2,IEL)} +  \var{PRODUC(3,IEL)} \right] $
\item [$\Rightarrow$] $\displaystyle \var{TRRIJ }= \frac{1}{2} (R^n_{ii})_L $
\item [$\Rightarrow$] $\displaystyle \var{SMBR(IEL)} =\ \var{SMBR(IEL)}\ +$\\
$\ \displaystyle\rho^n_L |\Omega_l| \left[ \displaystyle
\frac{2}{3}\,\delta_{\,ij} \left( \ \displaystyle \frac{ C_2}{2}\,(\mathcal{P}^n_{ii})_L\ +
(C_1-1)\ \varepsilon^n_L\, \right)\right.$\\
$ + \left.\ (1-C_2) \ \var{PRODUC(ISOU,IEL)} -
\displaystyle C_1\ \frac{2\,\varepsilon^n_L}{(R^n_{ii})_L}\ (R^n_{ij})_L \right]$
\item [$\Rightarrow$] $\displaystyle \var{ROVSDT(IEL)} = \var{ROVSDT(IEL)} +
\rho^n_L \ |\Omega_l| \ (- \displaystyle \frac{1}{3} \ \,\delta_{\,ij} + 1) \ C_1
\ \frac{2\ \varepsilon^n_L}{(R^n_{ii})_L}$
\end{itemize}
\item Appel de \fort{rijech} pour le calcul des termes d'\'echo de paroi
 $\phi^n_{ij,w}$ si $\var{IRIJEC()}=1$ et ajout dans \var{SMBR}.\\
$\var{SMBR} = \var{SMBR} + \phi^n_{ij,w}$\\
Suivant son mode de calcul (\var{ICDPAR}), la distance � la paroi est directement accessible
par \var{RA(IDIPAR+IEL-1)} (\var{|ICDPAR|} = 1) ou bien
est calcul\'ee \`a partir de $\var{IA(IIFAPA(IPHAS)+IEL - 1)}$,
qui donne pour l'\'el\'ement $\var{IEL}$ le num\'ero de la face de bord
paroi la plus  proche (\var{|ICDPAR|} = 2). Ces tableaux ont \'et\'e renseign\'e une fois pour toutes au
d\'ebut de calcul.

\item  Appel de \fort{rijthe} pour le calcul des termes de gravit\'e $\mathcal{G}^n_{ij}$ et ajout dans \var{SMBR}.

Ce calcul n'a lieu que si $\var{IGRARI()} = 1$.
$ \var{SMBR} = \var{SMBR} + \mathcal{G}^n_{ij}$
\item Calcul de la partie explicite du terme de diffusion
 $\dive{\,\left[\tens{A}\,\grad{R}^{\,n}_{ij}\right]}$, plus pr\'ecis\'ement
des contributions du terme extradiagonal pris aux faces purement internes
(remplissage du tableau \var{VISCF}), puis aux faces de bord (remplissage du
tableau \var{VISCB}).
\begin{itemize}
\item [$\star$] Appel de \fort{grdcel} pour le calcul du gradient de
$R^{\,n}_{ij}$ dans chaque direction. Ces gradients sont respectivement
stock\'es dans les tableaux de travail \var{W1}, \var{W2} et \var{W3}.

\item [$\star$] boucle d'indice \var{IEL} sur les cellules $\Omega_l$ de centre
$L$ pour le
calcul de $\tens{E}^n\,\grad{R}^{\,n}_{ij}$ aux cellules dans un premier temps :\\
\begin{itemize}
\item [$\Rightarrow$] $\displaystyle \var{TRRIJ}= \frac{1}{2} (R^{\,n}_{ii})_L $
\item [$\Rightarrow$] $\displaystyle \var{CSTRIJ} = \rho^n_L\ C_S \ \displaystyle\frac{(R^n_{ii})_L}{2\,\varepsilon^n_L}$
\item [$\Rightarrow$] $\displaystyle \var{W4(IEL)} = \rho^n_L\ C_S\
\displaystyle\frac{(R^n_{ii})_L}{2\,\varepsilon^n_L} \left[\,(R^{\,n}_{12})_L \ \var{W2(IEL)} +
(R^{\,n}_{13})_L \ \var{W3(IEL)}\,\right]$
\item [$\Rightarrow$] $\displaystyle \var{W5(IEL)} = \rho^n_L\ C_S\
\displaystyle\frac{(R^n_{ii})_L}{2\,\varepsilon^n_L} \left[\,(R^{\,n}_{12})_L \ \var{W1(IEL)} +
(R^{\,n}_{23})_L \ \var{W3(IEL)}\,\right]$
\item [$\Rightarrow$] $\displaystyle \var{W6(IEL)} = \rho^n_L\ C_S\
\displaystyle\frac{(R^n_{ii})_L}{2\,\varepsilon^n_L} \left[\,(R^{\,n}_{13})_L \ \var{W1(IEL)} + (R^{\,n}_{23})_L \ \var{W2(IEL)}\,\right]$
\end{itemize}



\item [$\star$] Appel de \fort{vectds}\footnote{Le r\'esultat est stock\'e dans
\var{VISCF} et \var{VISCB}. Dans \fort{vectds}, les valeurs aux faces internes
sont interpol\'ees lin\'eairement sans reconstruction et \var{VISCB} est mis \`a
z\'ero.} pour assembler $\displaystyle\left[ \tens{E}^n\,\grad{R}^{\,n}_{ij}
\right]\,.\,\vect{n}_{\,lm}S_{\,lm}$ aux faces $lm$.
\item [$\star$] Appel de \fort{divmas} pour calculer la divergence du flux d\'efini par \var{VISCF} et \var{VISCB}.
Le r\'esultat est stock\'e dans \var{W4}.\\
Ajout au second membre \var{SMBR}.\\
\var{SMBR} = \var{SMBR} + \var{W4}
\end{itemize}

A l'issue de cette \'etape, seule la partie extradiagonale de la diffusion prise
enti\`erement explicite~:
 $$\sum\limits_{m\in
Vois(l)}\left[\ \tens{E}^n\,\grad{R}^{\,n}_{ij} \right]_{\,lm}\,.\,\vect{n}_{\,lm}S_{\,lm}$$ a \'et\'e calcul\'ee.\\

\item Calcul de la partie diagonale du terme de diffusion\footnote{Seule la
composante normale  du  gradient de $R_{ij}$ aux faces sera implicite.} :\\
Comme on l'a d\'eja vu, une partie de cette contribution sera trait\'ee en
implicite (celle relative \`a la composante normale du gradient) et donc
ajout\'ee au second membre par \fort{bilsc2} ; l'autre
partie sera explicite et prise en compte dans $f_s^{\,exp}$.
\begin{itemize}
\item [$\star$] On effectue une boucle d'indice \var{IEL} sur les cellules
$\Omega_l$ de centre $L$ :
\begin{itemize}
\item [$\Rightarrow$] $\displaystyle \var{TRRIJ }= \frac{1}{2} (R^{\,n}_{ii})_L $
\item [$\Rightarrow$] $\displaystyle \var{CSTRIJ} = \rho^n_L \ C_S \ \frac{(R^{\,n}_{ii})_L}{2\,\varepsilon^n_L}$
\item [$\Rightarrow$] $\displaystyle \var{W4(IEL)} = \rho^n_L \ C_S \
\frac{(R^{\,n}_{ii})_L}{2\,\varepsilon^n_L} \ (R^{\,n}_{11})_L$
\item [$\Rightarrow$] $\displaystyle \var{W5(IEL)} = \rho^n_L \ C_S \ \frac{(R^{\,n}_{ii})_L}{2\,\varepsilon^n_L}\ (R^n_{22})_L$
\item [$\Rightarrow$] $\displaystyle \var{W6(IEL)} = \rho^n_L \ C_S \ \frac{(R^{\,n}_{ii})_L}{2\,\varepsilon^n_L} \ (R^n_{33})_L$
\end{itemize}

%\item Traitement du parall\'elisme et de la p\'eriodicit\'e.

\item [$\star$] On effectue une boucle d'indice \var{IFAC} sur les faces
purement internes $lm$ pour remplir le tableau \var{VISCF} :
\begin{itemize}
\item [$\Rightarrow$] Identification des cellules $\Omega_l$ et $\Omega_m$ de
centre respectif $L$ (variable \var{II}) et $M$ (variable \var{JJ}), se trouvant de chaque c\^ot\'e de la face
$lm$\footnote{La normale \`a la face est orient\'ee de L vers M.}.
\item [$\Rightarrow$] Calcul du carr\'e de la surface de la face. La valeur est
stock\'ee dans le tableau \var{SURFN2}.
\item [$\Rightarrow$] Interpolation du gradient de $R^{\,n}_{ij}$ \`a la face
$lm$ (gradient facette $\left[\grad{R}^{\,n}_{ij}\right]_{\,lm}$) :
\begin{equation}\notag
\left\{\begin{array}{ll}
\var{GRDPX} &= \displaystyle \frac{1}{2} \left(\var{W1(II)} + \var{W1(JJ)}
\right) \\
&\\
\var{GRDPY} &= \displaystyle \frac{1}{2} \left(\var{W2(II)} + \var{W2(JJ)} \right) \\
&\\
\var{GRDPZ} &= \displaystyle \frac{1}{2} \left(\var{W3(II)} + \var{W3(JJ)} \right)
\end{array}\right.
\end{equation}
\item [$\Rightarrow$] Calcul du gradient de $R^{\,n}_{ij}$ normal \`a la face
$lm$, $\left[\grad{R}^{\,n}_{ij}\right]_{\,lm}.\vect{n}_{\,lm}\,S_{\,lm}$.\\

$\displaystyle \var{GRDSN} =  \var{GRDPX} \ \var{SURFAC(1,IFAC)} + \var{GRDPY} \ \var{SURFAC(2,IFAC)} +  \var{GRDPZ} \ \var{SURFAC(3,IFAC)}$
$\var{SURFAC}$ est le vecteur surface de la face \var{IFAC}.


\item [$\Rightarrow$] calcul de
 $\left[\grad{R^{\,n}_{ij}} - (\grad
R^{\,n}_{ij}\,.\,\vect{n}_{\,lm})\vect{n}_{\,lm}\right]$, les vecteurs \'etant
calcul\'es \`a la face $lm$ :
\begin{equation}\notag
\left\{\begin{array}{lll}
&\displaystyle \var{GRDPX} &= \var{GRDPX} - \displaystyle\frac{\var{GRDSN}}{\var{SURFN2}} \ \var{SURFAC(1,IFAC)}\\
&&\\
&\displaystyle \var{GRDPY} &= \var{GRDPY} - \displaystyle\frac{\var{GRDSN}}{\var{SURFN2}} \ \var{SURFAC(2,IFAC)} \\
&&\\
&\displaystyle \var{GRDPZ} &= \var{GRDPZ} - \displaystyle\frac{\var{GRDSN}}{\var{SURFN2}} \ \var{SURFAC(3,IFAC)}
\end{array}\right.
\end{equation}
\item [$\Rightarrow$] finalisation du calcul de l'expression totalement
explicite
 $$\left[ \tens{D}^n\,\left( \grad{R^{\,n}_{ij}} - (\grad R^{\,n}_{ij}\,.\,\vect{n}_{\,lm})\,\vect{n}_{\,lm}\right) \right]\,.\,\vect{n}_{\,lm}$$
\begin{equation}\notag
\begin{array} {ll}
\displaystyle \var{VISCF} = &
 \displaystyle\frac{1}{2} (\ \var{W4(II)} +\ \var{W4(JJ)}) \ \var{GRDPX} \
\var{SURFAC(1,IFAC)})\ + \\
&\\
&  \displaystyle\frac{1}{2} (\ \var{W5(II)} +\ \var{W5(JJ)}) \ \var{GRDPY} \
\var{SURFAC(2,IFAC)})\ + \\
&\\
&  \displaystyle\frac{1}{2} (\ \var{W6(II)} +\ \var{W6(JJ)}) \ \var{GRDPZ} \ \var{SURFAC(3,IFAC)})
\end{array}
\end{equation}
\end{itemize}

\item [$\star$] Mise \`a z\'ero du tableau \var{VISCB}.

\item [$\star$] Appel de \fort{divmas} pour calculer la divergence de~:
 $$\tens{D}^{\,n}\,\left( \grad{R^{\,n}_{ij}} - (\grad R^{\,n}_{ij}\,.\,\vect{n}_{\,lm})\vect{n}_{\,lm}\right)$$ d\'efini aux faces dans \var{VISCF} et \var{VISCB}.

Le r\'esultat est stock\'e dans le tableau \var{W1}.\\
Ajout au second membre \var{SMBR}.\\
$\var{SMBR} = \var{SMBR} + \var{W1}$
\end{itemize}
\item Calcul de la viscosit\'e orthotrope $\gamma^n_{\,lm}$ \`a la face $lm$ de la variable principale
$R^{\,n}_{ij}$\\
Ce calcul permet au sous-programme \fort{codits} de compl\'eter le second membre
\var{SMBR} par :
\begin{equation}
\begin{array} {ll}
& \sum\limits_{m\in Vois(l)}
\mu^n_{\,lm}\,\left(\grad{R}^{\,n}_{ij}\,.\,\vect{n}_{\,lm}\right)S_{\,lm}
 + \sum\limits_{m\in Vois(l)} \left(\grad{R}^{\,n}_{ij}
\,.\,\vect{n}_{\,lm}\right)\left[\tens{D}^{\,n}\,\vect{n}_{\,lm}\right]_{\,lm}\,.\,\vect{n}_{\,lm}\
S_{\,lm}\\
& = \sum\limits_{m\in Vois(l)}(\,\mu^n_{\,lm}\, + \,\gamma^n_{\,lm}\,)\,\left(\grad{R}^{\,n}_{ij}\,.\,\vect{n}_{\,lm}\right)S_{\,lm}
\end{array}
\end{equation}
sans pr\'eciser la nature de la face $lm$, {\it via} l'appel \`a \fort{bilsc2}  et de disposer de la quantit\'e
$(\mu^n_{\,lm}\, + \gamma^n_{\,lm})$ pour construire sa
matrice simplifi\'ee.\\
\begin{itemize}
\item [$\star$] On effectue une boucle d'indice \var{IEL} sur les cellules
$\Omega_l$ :
\begin{itemize}
\item [$\Rightarrow$] $\displaystyle \var{TRRIJ }= \frac{1}{2} (R^{\,n}_{ii})_L $
\item [$\Rightarrow$] $\displaystyle \var{RCSTE} = \rho^n_L \ C_S \ \frac{ (R^{\,n}_{ii})_L}{2\,\varepsilon^n_L} $
\item [$\Rightarrow$] $\displaystyle \var{W1(IEL)} = \mu^n +\rho^n_L \ C_S \ \frac{
(R^{\,n}_{ii})_L}{2\,\varepsilon^n_L}\ (R^n_{11})_L$
\item [$\Rightarrow$] $\displaystyle \var{W2(IEL)} = \mu^n + \rho^n_L \ C_S \ \frac{ (R^{\,n}_{ii})_L}{2\,\varepsilon^n_L}\ (R^n_{22})_L$
\item [$\Rightarrow$] $\displaystyle \var{W3(IEL)} = \mu^n + \rho^n_L \ C_S \ \frac{ (R^{\,n}_{ii})_L}{2\,\varepsilon^n_L}\ (R^n_{33})_L$
\end{itemize}

\item [$\star$] Appel de \fort{visort} pour calculer la viscosit\'e orthotrope
\footnote{Comme dans le sous-programme \fort{viscfa}, on multiplie la viscosit\'e par
$\displaystyle \frac{S_{\,lm}}{\overline{L'M'}}$, o\`u $S_{\,lm}$ et
$\overline{L'M'}$ repr\'esentent respectivement la surface de la face $lm$ et la
mesure alg\'ebrique du segment reliant les projections des centres des cellules
voisines sur la normale \`a la face. On garde dans ce sous-programme  la possibilit\'e d'interpoler la viscosit\'e aux cellules lin\'eairement ou d'utiliser une moyenne harmonique. La viscosit\'e au bord est celle de la cellule de bord correspondante.}
$ \gamma^n_{\,lm} = (\tens{D}^{\,n}\,\vect{n}_{\,lm}).\vect{n}_{\,lm}$ aux faces $lm$

Le r\'esultat est stock\'e dans les tableaux \var{VISCF} et \var{VISCB}.
\end{itemize}

\item appel de \fort{codits} pour la r\'esolution de l'\'equation de
convection/diffusion/termes sources de la variable $R_{ij}$. Le terme source est
\var{SMBR}, la viscosit\'e \var{VISCF} aux faces purement internes (
resp. \var{VISCB} aux faces de bord) et \var{FLUMAS} le flux de masse interne
 ( resp. \var{FLUMAB} flux de masse au bord). Le r\'esultat est la variable $R_{ij}$ au temps
$n+1$, donc $R^{\,n+1}_{ij}$. Elle est stock\'ee dans le tableau \var{RTP} des
variables mises \`a jour.

\end{itemize}

\etape{Appel de \fort{reseps} pour la r\'esolution de la variable $\varepsilon$}

On d\'ecrit ci-dessous le sous-programme \fort{reseps}, les commentaires du sous-programme \fort{resrij} ne seront pas r\'ep\'et\'es puisque les deux sous-programmes ne diff\`erent que par leurs termes sources.

\begin{itemize}
\item Initialisation \`a z\'ero de \var{SMBR} et \var{ROVSDT}.

\item{Lecture et prise en compte des termes sources utilisateur pour la variable $\varepsilon$ :}

Appel de \fort{ustsri} pour charger les termes sources utilisateurs. Ils sont
stock\'es dans les tableaux suivants :\\
pour la cellule $\Omega_l$ repr\'esent\'ee par $\var{IEL}$ de centre $L$, on a :
\begin{equation}\notag
\left\{\begin{array}{lll}
&\var{ROVSDT(IEL)}&= |\Omega_l| \ \alpha_{\varepsilon}\\
&\var{SMBR(IEL)}&=|\Omega_l| \ \beta_{\varepsilon}\\
\end{array}\right.
\end{equation}
On affecte alors les valeurs ad\'equates au second membre \var{SMBR} et \`a la
diagonale \var{ROVSDT} comme suit :
\begin{equation}\notag
\left\{\begin{array}{lll}
&\var{SMBR(IEL)} &= \var{SMBR(IEL)} +\ |\Omega_l| \ \alpha_{\,\varepsilon} \
\varepsilon^n_L \\
&\var{ROVSDT(IEL)}&= \text{max }(-\ |\Omega_l| \ \alpha_{\,\varepsilon},0)\\
\end{array}\right.
\end{equation}

\item{Calcul du terme source de masse si $\Gamma_L > 0$ :
\begin{equation}\notag
\left\{\begin{array}{lll}
&\displaystyle \var{SMBR(IEL)} = \var{SMBR(IEL)} + |\Omega_l| \ \Gamma_L \
(\varepsilon^{\,in}_L -\varepsilon^n_L) \\
&\displaystyle \var{ROVSDT(IEL)}= \var{ROVSDT(IEL)} + |\Omega_l| \ \Gamma_L
\end{array}\right.
\end{equation}
\item Calcul du terme d'accumulation de masse et du terme instationnaire \\
On stocke $\displaystyle \var{W1}= \int_{\Omega_l}\dive\,(\rho\,\vect{u})\,d\Omega$
calcul\'e par \fort{divmas} \`a l'aide des flux de masse internes et aux bords.\\
On incr\'emente ensuite \var{SMBR} et \var{ROVSDT}.
\begin{equation}\notag
\left\{\begin{array}{lll}
&\var{SMBR(IEL)} &= \var{SMBR(IEL)} + \var{ICONV}\ \varepsilon^n_L\,(\displaystyle
\int_{\Omega_l}\dive\,(\rho\,\vect{u})\ d\Omega) \\
&\var{ROVSDT(IEL)}& = \var{ROVSDT(IEL)} +  \var{ISTAT}\,\displaystyle
\frac{\rho^n_L \ |\Omega_l|}{\Delta t^n} -  \var{ICONV}\ (\displaystyle
\int_{\Omega_l}\dive\,(\rho\,\vect{u})\ d\Omega) \\
\end{array}\right.
\end{equation}

\item Traitement du terme de production
 $\displaystyle \rho\,C_{\varepsilon_1}\,\frac{\varepsilon}{k}\,\mathcal{P}$
 et du terme de dissipation $-\,\displaystyle \rho\,C_{\varepsilon_2}\,\frac{\varepsilon}{k}\,\varepsilon$ \\
pour cela, on effectue une boucle d'indice \var{IEL} sur les cellules $\Omega_l$
de centre $L$ :
\begin{itemize}
\item [$\Rightarrow$] $\displaystyle \var{TRPROD}= \frac{1}{2} (\mathcal{P}^n_{ii})_L = \frac{1}{2} \left[ \var{PRODUC(1,IEL)} +  \var{PRODUC(2,IEL)} +  \var{PRODUC(3,IEL)} \right] $
\item [$\Rightarrow$] $\displaystyle \var{TRRIJ }= \frac{1}{2} (R^n_{ii})_L $
\item [$\Rightarrow$] $\displaystyle \var{SMBR(IEL)} = \var{SMBR(IEL)} + \rho^n_L
|\Omega_l| \left[ -C_{\varepsilon_2} \ \frac{2\,(\varepsilon^n_L)^2}{(R^n_{ii})_L} + C_{\varepsilon_1} \ \frac{\varepsilon^n_L}{(R^n_{ii})_L}\ (\mathcal{P}^n_{ii})_L \right] $
\item [$\Rightarrow$] $\displaystyle \var{ROVSDT(IEL)} = \var{ROVSDT(IEL)} + C_{\varepsilon_2} \ \rho^n_L \ |\Omega_l| \ \frac{2\,\varepsilon^n_L}{(R^n_{ii})_L}$
\end{itemize}

\item Appel de \fort{rijthe} pour le calcul des termes de gravit\'e $\mathcal{G}^n_{\varepsilon}$ et ajout dans \var{SMBR}.

$ \var{SMBR} = \var{SMBR} + \mathcal{G}^n_{\varepsilon}$\\
Ce calcul n'a lieu que si $\var{IGRARI()} = 1$.

\item Calcul de la diffusion de $\varepsilon$ \\
 Le terme $\dive \left[\mu\, \grad(\varepsilon) + \tens{A'}\,\grad(\varepsilon)
\right]$ est calcul\'e exactement de la m\^eme mani\`ere que pour les tensions
de Reynolds $R_{ij}$ en rempla\c cant $\tens{A}$ par $\tens{A'}$.

\item Appel de \fort{codits} pour la r\'esolution de l'\'equation de
convection/diffusion/termes sources de la variable principale $\varepsilon$. Le
r\'esultat $\varepsilon^{\,n+1}$ est stock\'e dans le tableau \var{RTP} des
variables mises \`a jour.
}
\end{itemize}

\etape{clippings finaux}
On passe enfin dans le sous-programme  \fort{clprij} pour faire un clipping \'eventuel
des variables $R^{\,n+1}_{ij}$ et $\varepsilon^{\,n+1}$. Le sous-programme
\fort{clprij} est appel\'e\footnote{L'option
$\var{ICLIP} = 1$ consiste en un clipping minimal des variables $R_{ii}$ et
$\varepsilon$ en prenant la valeur absolue de ces variables puisqu'elles ne
peuvent \^etre que positives.} avec $\var{ICLIP} = 2$ . Cette option
\footnote{Quand la valeur des grandeurs $R_{ii}$ ou $\varepsilon$ est
n\'egative, on la remplace par le minimum entre sa valeur absolue et (1,1)
fois la valeur obtenue au pas de temps pr\'ec\'edent.} contient l'option $\var{ICLIP} = 1$  et permet de v\'erifier l'in\'egalit\'e de Cauchy-Schwarz sur les grandeurs extra-diagonales du tenseur $\tens{R}$ (pour $i \neq j$, $|R_{ij}|^2 \le R_{ii} R_{jj}$).


%%%%%%%%%%%%%%%%%%%%%%%%%%%%%%%%%%
%%%%%%%%%%%%%%%%%%%%%%%%%%%%%%%%%%
\section{Points \`a traiter}
%%%%%%%%%%%%%%%%%%%%%%%%%%%%%%%%%%
%%%%%%%%%%%%%%%%%%%%%%%%%%%%%%%%%%
Sauf mention explicite, $\phi$ repr\'esentera une tension de Reynolds ou la dissipation turbulente ($\phi = R_{ij} \ \text{ou} \ \varepsilon$).

\begin{itemize}
\item {La vitesse utilis\'ee pour le calcul de la production est explicite. Est-ce qu'une implicitation peut am\'eliorer la pr\'ecision temporelle de $\phi$ \footnote{Cette remarque peut \^etre g\'en\'eralis\'ee. En effet, peut-on envisager d'actualiser les variables d\'ej\`a r\'esolues (sans r\'eactualiser les variables turbulentes apr\`es leur r\'esolution)? Ceci obligerait \`a modifier les sous-programmes tels que \fort{condli} qui sont appel\'es au d\'ebut de la boucle en temps.} ?}
\item {Dans quelle mesure le terme d'\'echo de paroi est-il valide ? En effet, ce terme est remis en question par certains auteurs.}
\item {On peut envisager la r\'esolution d'un syst\`eme hyperbolique pour les
tensions de Reynolds afin d'introduire un couplage avec le champ de vitesse.}
\item {Le flux au bord \var{VISCB} est annul\'e dans le sous-programme
\fort{vectds}. Peut-on envisager de mettre au bord la valeur de la variable
concern\'ee \`a la cellule de bord correspondant? De m\^eme, il faudrait se
pencher sur les hypoth\`eses sous-jacentes \`a l'annulation des contributions
aux bords de \var{VISCB} lors du calcul de : $$\left[ \tens{D}^n\,\left( \grad{R^{\,n}_{ij}} - (\grad R^{\,n}_{ij}\,.\,\vect{n}_{\,lm})\,\vect{n}_{\,lm}\right) \right]\,.\,\vect{n}_{\,lm}.$$}
\item {Un probl\`eme de pond\'eration appara\^\i t plus g\'en\'eralement. Si on prend la partie explicite de $\tens{D}\,\grad(\phi)$, la pond\'eration aux faces internes utilise le coefficient $\displaystyle\frac{1}{2}$ avec pond\'eration s\'epar\'ee de $\tens{D}$ et $\grad(\phi)$, alors que pour $\tens{E}\,\grad(\phi)$, on calcule d'abord ce terme aux cellules pour ensuite l'interpoler lin\'eairement aux faces \footnote{Cette interpolation se fait dans \fort{vectds} avec des coefficients de pond\'eration aux faces.}. Ceci donne donc deux types d'interpolations pour des termes de m\^eme nature.}
\item {On laisse la possibilit\'e dans \fort{visort} d'utiliser une moyenne
harmonique aux faces. Est-ce que ceci est valable puisque les interpolations
utilis\'ees lors du calcul de la partie explicite de $\tens{A}\,\grad{\phi}$
sont des moyennes arithm\'etiques ?}
\item {Les techniques adopt\'ees lors du clipping sont \`a revoir.}
\item {On utilise dans le cadre du mod\`ele $\displaystyle R_{ij}-\varepsilon $ une semi-implicitation de termes comme $\displaystyle \phi_{ij,1}$ ou $\displaystyle -\rho\,C_{\varepsilon_2}\,\frac{\varepsilon}{k}\,\varepsilon$. On peut envisager le m\^eme type d'implicitation dans \fort{turbke} m\^eme en pr\'esence du couplage $\displaystyle k-\varepsilon$.}
\item L'adoption d'une r\'esolution d\'ecoupl\'ee fait perdre l'invariance par rotation.
\item La formulation et l'implantation des conditions aux limites de paroi
devront \^etre v\'erifi\'ees. En effet, il semblerait que, dans certains cas, des ph\'enom\`enes
``oscillatoires'' apparaissent, sans qu'il soit ais\'e d'en d\'eterminer la cause.
\item L'implicitation partielle (du fait de la r\'esolution d\'ecoupl\'ee) des
conditions aux limites conduit souvent \`a des calculs instables. Il
conviendrait d'en conna\^\i tre la raison. L'implicitation partielle avait
\'et\'e mise en \oe uvre afin de tenter d'utiliser un pas de temps plus grand
dans le cas de jets axisym\'etriques en particulier.

\end{itemize}

%                      Code_Saturne version 1.3
%                      ------------------------
%
%     This file is part of the Code_Saturne Kernel, element of the
%     Code_Saturne CFD tool.
%
%     Copyright (C) 1998-2007 EDF S.A., France
%
%     contact: saturne-support@edf.fr
%
%     The Code_Saturne Kernel is free software; you can redistribute it
%     and/or modify it under the terms of the GNU General Public License
%     as published by the Free Software Foundation; either version 2 of
%     the License, or (at your option) any later version.
%
%     The Code_Saturne Kernel is distributed in the hope that it will be
%     useful, but WITHOUT ANY WARRANTY; without even the implied warranty
%     of MERCHANTABILITY or FITNESS FOR A PARTICULAR PURPOSE.  See the
%     GNU General Public License for more details.
%
%     You should have received a copy of the GNU General Public License
%     along with the Code_Saturne Kernel; if not, write to the
%     Free Software Foundation, Inc.,
%     51 Franklin St, Fifth Floor,
%     Boston, MA  02110-1301  USA
%
%-----------------------------------------------------------------------
%
\programme{vortex}
%
\vspace{1cm}
%%%%%%%%%%%%%%%%%%%%%%%%%%%%%%%%%%
%%%%%%%%%%%%%%%%%%%%%%%%%%%%%%%%%%
\section{Fonction}
%%%%%%%%%%%%%%%%%%%%%%%%%%%%%%%%%%
%%%%%%%%%%%%%%%%%%%%%%%%%%%%%%%%%%
Ce sous-programme est d�di� � la g�n�ration des conditions d'entr�e
turbulente utilis�es en LES.


La m�thode des vortex est bas�e sur une approche de tourbillons
ponctuels. L'id�e de la m�thode consiste � injecter des tourbillons 2D dans le
plan d'entr�e du calcul, puis � calculer le champ de vitesse induit par ces
tourbillons au centre des faces d'entr�e.

%                      Code_Saturne version 1.3
%                      ------------------------
%
%     This file is part of the Code_Saturne Kernel, element of the
%     Code_Saturne CFD tool.
% 
%     Copyright (C) 1998-2007 EDF S.A., France
%
%     contact: saturne-support@edf.fr
% 
%     The Code_Saturne Kernel is free software; you can redistribute it
%     and/or modify it under the terms of the GNU General Public License
%     as published by the Free Software Foundation; either version 2 of
%     the License, or (at your option) any later version.
% 
%     The Code_Saturne Kernel is distributed in the hope that it will be
%     useful, but WITHOUT ANY WARRANTY; without even the implied warranty
%     of MERCHANTABILITY or FITNESS FOR A PARTICULAR PURPOSE.  See the
%     GNU General Public License for more details.
% 
%     You should have received a copy of the GNU General Public License
%     along with the Code_Saturne Kernel; if not, write to the
%     Free Software Foundation, Inc.,
%     51 Franklin St, Fifth Floor,
%     Boston, MA  02110-1301  USA
%
%-----------------------------------------------------------------------
%
%%%%%%%%%%%%%%%%%%%%%%%%%%%%%%%%%%
%%%%%%%%%%%%%%%%%%%%%%%%%%%%%%%%%%
\section{Discr\'etisation}
%%%%%%%%%%%%%%%%%%%%%%%%%%%%%%%%%%
%%%%%%%%%%%%%%%%%%%%%%%%%%%%%%%%%%

Le terme convectif en $\dive(\underline{u} \otimes \rho\,\underline{u})$
introduit une non lin\'earit\'e et un couplage des composantes de la vitesse
$\vect{u}$ dans l'�quation (\ref{Base_Preduv_eqqdm}). Une lin\'earisation et un d\'ecouplage
des trois composantes de la 
vitesse sont r\'ealis\'es lors de la discr\'etisation de cette \'etape de
pr\'ediction.\\
En effet, soit :
\begin{equation}
\vect{\widetilde{u}}= \vect{u}^n + \delta \vect{u} 
\end{equation}
La contribution exacte du terme convectif \`a prendre en compte dans cette
\'etape de pr\'ediction serait :\\
\begin{equation}\label{Base_Preduv_Conv_exact}
\begin{array}{ll}
\dive(\vect{\widetilde{u}} \otimes \rho\,\vect{\widetilde{u}}) =
\dive(\vect{u}^{n} \otimes \rho\,\vect{u}^{n}) + \dive(\delta \vect{u} \otimes
\rho\,\vect{u}^{n}) +  \underbrace { \dive(\vect{u}^{n} \otimes
\rho\,\delta \vect{u})}_{\text {terme couplant lin\'eaire}} +  \underbrace { \dive(\delta \vect{u} \otimes
\rho\,\delta \vect{u})}_{\text {terme couplant et non lin\'eaire}}\\
\end{array} 
\end{equation}
Les deux derniers termes de l'expression (\ref{Base_Preduv_Conv_exact}) sont {\em a priori} n�glig�s
de mani�re � obtenir un syst\`eme en vitesse qui soit d\'ecoupl\'e et donc,
�viter l'inversion d'une matrice pouvant \^etre de tr\`es grande taille. Ces
deux termes peuvent n�anmoins �tre pris en compte de mani�re plus ou moins
approch�e par extrapolation explicite du flux de masse en $n+\theta_F$ (pour le
terme couplant lin�aire provenant de la convection de $\vect{u}^{n}$ par $\delta
\vect{u}$) et utilisation d'un point-fixe par sous it�ration sur le sous
programme \fort{navsto} (pour le terme non-lin�aire, en sp�cifiant $\var{NTERUP}>1$).

L'�quation (\ref{Base_Preduv_eqqdm}) est discr�tis�e au temps $n+\theta$ � l'aide d'un
$\theta$-sch�ma, et le tenseur des pertes de charges d�compos� en une partie
diagonale $\tens{K}_{d}$ et une extradiagonale $\tens{K}_{e}$ (soit
 $\tens{K}_{pdc}=\tens{K}_{d}+\tens{K}_{e}$).\\
$\bullet$ La pression est suppos�e connue en $n-1+\theta$ (d�calage temporel
pression-vitesse) et le gradient naturellement calcul� � cet instant.\\ 
$\bullet$ Les termes sources de viscosit� secondaire, de gradient transpos\'e,
ceux provenant du mod�le de turbulence\footnote{except� $\dive (\mu_t\ (\ggrad
\underline {u}))$}, $\rho\,\tens{K}_{\,e}\ \underline{u}$, $(\rho -\rho_0)
\underline {g}$ ainsi que $\underline{T}_{s}^{\,exp}$ et
$\Gamma\,\underline{u}_{\,i}$ sont pris de mani�re explicite au temps $n$, ou
extrapol�s suivant le sch�ma en temps choisi pour les propri�t�s physique et les
termes sources.\\ 
$\bullet$ Les termes sources $\underline{u}\,\,\dive (\rho\,\underline {u})$,
$\Gamma\,\,\underline{u}$, $T_{s}^{\,imp}\,\,\underline{u}$ et
$-\rho\,\tens{K}_{\,d}\,\,\underline{u}$ sont implicit�s est calcul�s �
l'instant $n+\theta$.\\ 
$\bullet$ Le terme de diffusion $\dive (\mu_{\,tot}\,\ggrad \underline{u})$ est
implicit� : la vitesse est prise � l'instant $n+\theta$ et la viscosit�
explicit�e ou extrapol�e.\\ 
$\bullet$ Enfin, le terme de convection en $\dive(\,\underline{u} \otimes
(\rho\underline{u})\,)$ est implicit� : la composante r�solue de la vitesse est
prise en $n+\theta$, et le flux de masse, explicit�, ou extrapol� en
$n+\theta_F$. 

Par souci de clart�, on suppose, en l'absence d'indication, que les propri�tes
physiques $\Phi$ ($\rho,\,\mu_{tot},\,...$) et le flux de masse
$(\rho\underline{u})$ sont pris respectivement aux instants $n+\theta_\Phi$ et
$n+\theta_F$, o� $\theta_\Phi$ et $\theta_F$ d�pendent des sch�mas en temps
sp�cifiquement utilis�s pour ces grandeurs\footnote{cf. \fort{introd}}. 

La discr�tisation temporelle de l'�quation (\ref{Base_Preduv_eqqdm}) s'�crit alors comme suit : 

\begin{equation}\label{Base_Preduv_eq_di1}
 \begin{array}{c}
\displaystyle \rho\,\ \frac{ \underline {\widetilde{u}}^{n+1} -\underline {u}^{n} }
{\Delta t} + \dive(\,\underline{\widetilde{u}}^{n+\theta} \otimes (\rho\underline{u})\,) -\dive
(\mu_{\,tot}\,\ggrad \underline{\widetilde{u}}^{n+\theta}) =
\\
\displaystyle
 - \grad p^{n-1+\theta} + \dive (\rho\,\underline {u})\,\underline{\widetilde{u}}^{n+\theta} +(\Gamma\,\underline{u}_{\,i})^{n+\theta_S}-\Gamma^n\,\,\underline{\widetilde{u}}^{n+\theta}
\\
\begin{array}{c}
\displaystyle
- \rho\,\tens{K}_{\,d}^{n}\,\,\underline{\widetilde{u}}^{n+\theta} - (\rho\,\tens{K}_{\,e}\ \underline{u})^{n+\theta_S} + (\underline{T}_{s}^{\,exp})^{\,n+\theta_S} + T_{s}^{\,imp}\,\,\underline{\widetilde{u}}^{n+\theta}
\\
\displaystyle
+[\dive (\mu_{\,tot}\,^t\ggrad \underline {u})]^{n+\theta_S}-\frac {2} {3}[\,\grad (\mu_{\,tot}\,\dive \underline {u})]^{n+\theta_S} + (\rho -\rho_0) \underline {g}
 - (\underline{turb})^{n+\theta_S}
\end{array}
\end{array}
\end{equation}
o\`u, par souci de simplification, on a pos\'e :
\begin{equation}
\mu_{\,tot}=
\begin{cases}
\mu+\mu_t & \text{pour les mod�les � viscosit� turbulente ou en LES}, \\
\mu & \text{pour les mod�les au second ordre ou en laminaire}
\end{cases} \ 
\end{equation}
\\
et :
\begin{equation}
\underline{turb}^{n}=
\begin{cases}
\displaystyle\frac {2}{3}\grad (\rho^{n}\,k^{n}) & \text{pour les mod�les � viscosit� turbulente}, \\
\dive(\rho^{n}\,\tens{R}^n) & \text{pour les mod�les au second ordre},\\
0 & \text{en laminaire ou en LES}\\
\end{cases}
\end{equation}
Par analogie avec l'�criture du $\theta$-sch�ma pour une variable scalaire, $\,
\underline {\widetilde{u}}^{n+\theta}$ est interpol�e � partir de la vitesse
pr�dite $\underline {\widetilde{u}}^{n+1}$ de la mani\`ere suivante\footnote{si
$\theta=1/2$, ou qu'une extrapolation est utilis�e, l'ordre 2 n'est obtenu que si
le pas de temps $\Delta t$ est uniforme en temps et en espace.}~: 
\begin{equation}
\underline {\widetilde{u}}^{n+\theta}=\theta\, \underline
{\widetilde{u}}^{n+1}+(1-\theta)\, \underline {u}^{n}\\ 
\end{equation}
Avec :
\begin{equation}
\left\{
\begin{array}{ll}
\theta = 1   & \text{Pour un sch\'ema de type Euler implicite d'ordre 1.}\\
\theta = 1/2 & \text{Pour un sch\'ema de type Cranck-Nicolson d'ordre 2.}\\
\end{array}
\right.
\end{equation}

L'�quation (\ref{Base_Preduv_eq_di1}) est alors r��crite sous la forme :

\begin{equation}\label{Base_Preduv_eq_di2}
\begin{array}{c}
\displaystyle \underbrace{\left(\frac{\rho}{\Delta t} -\theta \,\dive (\rho\,\underline {u}) +\theta \,\, \Gamma^n +
\theta \,\, \rho\,\tens{K}_{\,d}^n-\theta \,T_s^{\,imp} \right)}_{\displaystyle f_s^{imp}}\, (\underline {\,\widetilde{u}}^{n+1} -\underline {u}^{n})
\\
 +\, \theta\, \dive(\underline {\widetilde{u}}^{n+1} \otimes (\rho\underline{u}))-\, \theta\,\dive (\mu_{\,tot}\,\ggrad \underline {\widetilde{u}}^{n+1}) =
\\
-\,(1-\theta)\, \dive(\underline {u}^{n} \otimes (\rho\underline{u})) +\,(1-\theta)\,\dive (\mu_{\,tot}\,\ggrad \underline {u}^{n})
\\
f_s^{exp}\left\{
\begin{array}{c}
\displaystyle 
- \grad p^{n-1+\theta} + \dive (\rho\,\underline {u})\,\underline{u}^{n} +\,(\,\Gamma^{n}\,\underline{u}_{\,i}\,)^{n+\theta_S}- \Gamma^n\,\,\underline{u}^{n}
\\
\displaystyle
-(\,\rho\,\tens{K}_{\,e}\ \underline{u}\,)^{n+\theta_S} -\rho\,\tens{K}_{\,d}^n\ \underline{u}^{n}+ (\underline{T}_{s}^{\,exp})^{\,n+\theta_S} + T_s^{\,imp}\,\,\underline {u}^{n} 
\\
\displaystyle
+[\dive (\mu_{\,tot}\,^t\ggrad \underline {u}\,)]^{n+\theta_S}-\frac {2} {3}[\,\grad (\mu_{\,tot}\,\dive \underline {u}\,)]^{n+\theta_S} + (\rho -\rho_0) \underline {g}-(\underline{turb})^{n+\theta_S}
\end{array}
\right.
\end{array}
\end{equation}

d'o� l'�quation r�solue par le sous-programme \fort{codits} :
\begin{equation}\begin{array}{c}
\displaystyle
f_s^{\,imp}(\underline {\widetilde{u}}^{n+1}-\underline {u}^{n}) + \theta\, \dive(\underline{\widetilde{u}}^{n+1} \otimes (\rho
\underline{u})) - \theta\,\dive (\,\mu_{\,tot}\,\ggrad \underline{\widetilde{u}}^{n+1}) = 
\\\\
\displaystyle
-(1-\theta)\,\dive(\underline{u}^{n} \otimes (\rho \underline{u}))+(1-\theta)\,\dive (\,\mu_{\,tot}\,\ggrad \underline{u}^{n})
+ \underline{f}_{\,s}^{\,exp}
\end{array}
\end{equation}
La m\'ethode de discr\'etisation spatiale est d\'evelopp\'ee dans le sous-programme \fort{codits}.\\



\minititre{Remarques :}
{\tiny$\blacksquare$} Dans le cas standard sans extrapolation, le terme
$-\,T_s^{\,imp}$ n'est ajout� � $f_s^{\,imp}$ que s'il est positif afin de ne
pas affaiblir la dominance de la diagonale de la matrice � inverser.\\ 
{\tiny$\blacksquare$} Si une extrapolation est utilis�e, par contre,
$\,T_s^{\,imp}$ est ajout� � $f_s^{\,imp}$ quel que soit son signe. En effet, l'id�e intuitive qui
consiste � prendre~: 
\begin{equation}
\begin{cases}
\displaystyle
(\underline{T}_{s}^{\,exp} + T_{s}^{\,imp}\,\underline {u})^{\,n+\theta_S} &
\text{si } T_{s}^{\,imp} > 0\\ 
\displaystyle
(\underline{T}_{s}^{\,exp})^{\,n+\theta_S} + T_{s}^{\,imp}\,\underline{u}^{n+\theta} &\text{sinon}\\
\end{cases}
\end{equation} 
aboutit � une incoh�rence dans le traitement si $T_s^{imp}$ change de signe
entre deux pas de temps.\\ 
{\tiny$\blacksquare$} la partie diagonale $\tens{K}_{\,d}$ du terme
de perte de charge est utilis�e dans $f_s^{\,imp}$. En fait, pour \^etre rigoureux,
il faudrait ne retenir que les contributions positives (point signal\'e dans le
sous-programme utilisateur associ\'e \fort{uskpdc}). Cette prise en compte sera \`a am\'eliorer.\\
{\tiny$\blacksquare$} Le terme $\theta\,\Gamma^{n}-\theta\,\dive
(\rho\,\underline {u})$ ne pose pas de probl�me pour la 
dominance de la diagonale de la matrice car il est exactement compens� par le
terme de convection (cf. \fort{covofi}). 


%                      Code_Saturne version 1.3
%                      ------------------------
%
%     This file is part of the Code_Saturne Kernel, element of the
%     Code_Saturne CFD tool.
%
%     Copyright (C) 1998-2007 EDF S.A., France
%
%     contact: saturne-support@edf.fr
%
%     The Code_Saturne Kernel is free software; you can redistribute it
%     and/or modify it under the terms of the GNU General Public License
%     as published by the Free Software Foundation; either version 2 of
%     the License, or (at your option) any later version.
%
%     The Code_Saturne Kernel is distributed in the hope that it will be
%     useful, but WITHOUT ANY WARRANTY; without even the implied warranty
%     of MERCHANTABILITY or FITNESS FOR A PARTICULAR PURPOSE.  See the
%     GNU General Public License for more details.
%
%     You should have received a copy of the GNU General Public License
%     along with the Code_Saturne Kernel; if not, write to the
%     Free Software Foundation, Inc.,
%     51 Franklin St, Fifth Floor,
%     Boston, MA  02110-1301  USA
%
%-----------------------------------------------------------------------
%

%%%%%%%%%%%%%%%%%%%%%%%%%%%%%%%%%%
%%%%%%%%%%%%%%%%%%%%%%%%%%%%%%%%%%
\section{Mise en \oe uvre}
%%%%%%%%%%%%%%%%%%%%%%%%%%%%%%%%%%
%%%%%%%%%%%%%%%%%%%%%%%%%%%%%%%%%%
La num\'ero de la phase trait\'ee fait partie des arguments de \fort{turrij}. On
omettra volontairement de le pr\'eciser dans ce qui suit, on indiquera par $(\ )$ la
notion de tableau s'y rattachant.

\etape{Calcul des termes de production $\tens{\mathcal{P}}$}
\begin{itemize}
\item [$\star$] Initialisation \`a z\'ero du tableau \var{PRODUC} dimensionn\'e \`a $\var{NCEL}\times 6$.
\item [$\star$] On appelle trois fois \fort{grdcel} pour calculer les gradients des composantes de la vitesse $u$, $v$ et
$w$ prises au temps $n$.

Au final, on a :\\
$\displaystyle
\begin{array} {ll}
\var{PRODUC(1,IEL)} = & \displaystyle - 2 \left[ R_{11}^{\,n} \frac{\partial u^{\,n}} {\partial x} +R_{12}^{\,n} \frac{\partial u^{\,n}} {\partial y}+R_{13}^{\,n} \frac{\partial u^{\,n}} {\partial z} \right] \text{        (production de $R_{11}^{\,n}$)}\\
\var{PRODUC(2,IEL)} = & \displaystyle - 2 \left[ R_{12}^{\,n} \frac{\partial v^{\,n}} {\partial x} +R_{22}^{\,n} \frac{\partial v^{\,n}} {\partial y}+R_{23}^{\,n} \frac{\partial v^{\,n}} {\partial z} \right] \text{        (production de $R_{22}^{\,n}$)}\\
\var{PRODUC(3,IEL)} = & \displaystyle - 2 \left[ R_{13}^{\,n} \frac{\partial w^{\,n}} {\partial x} +R_{23}^{\,n} \frac{\partial w^{\,n}} {\partial y}+R_{33}^{\,n} \frac{\partial w^{\,n}} {\partial z} \right] \text{        (production de $R_{33}^{\,n}$)}\\
\var{PRODUC(4,IEL)} = & \displaystyle - \left[ R_{12}^{\,n} \frac{\partial u^{\,n}} {\partial x} +R_{22}^{\,n} \frac{\partial u^{\,n}} {\partial y}+R_{23}^{\,n} \frac{\partial u^{\,n}} {\partial z} \right] \\
& \displaystyle - \left[ R_{11}^{\,n} \frac{\partial v^{\,n}} {\partial x} +R_{12}^{\,n} \frac{\partial v^{\,n}} {\partial y}+R_{13}^{\,n} \frac{\partial v^{\,n}} {\partial z} \right] \text{        (production de $R_{12}^{\,n}$)} \\
\var{PRODUC(5,IEL)} = & \displaystyle - \left[ R_{13}^{\,n} \frac{\partial u^{\,n}} {\partial x} +R_{23}^{\,n} \frac{\partial u^{\,n}} {\partial y}+R_{33}^{\,n} \frac{\partial u^{\,n}} {\partial z} \right] \\
& \displaystyle - \left[ R_{11}^{\,n} \frac{\partial w^{\,n}} {\partial x} +R_{12}^{\,n} \frac{\partial w^{\,n}} {\partial y}+R_{13}^{\,n} \frac{\partial w^{\,n}} {\partial z} \right] \text{        (production de $R_{13}^{\,n}$)} \\
\var{PRODUC(6,IEL)} = & \displaystyle - \left[ R_{13}^{\,n} \frac{\partial v^{\,n}} {\partial x} +R_{23}^{\,n} \frac{\partial v^{\,n}} {\partial y}+R_{33}^{\,n} \frac{\partial v^{\,n}} {\partial z} \right] \\
& \displaystyle - \left[ R_{12}^{\,n} \frac{\partial w^{\,n}} {\partial x} +R_{22}^{\,n} \frac{\partial w^{\,n}} {\partial y}+R_{23}^{\,n} \frac{\partial w^{\,n}} {\partial z} \right]  \text{        (production de $R_{23}^{\,n}$)}
\end{array}
$
\end{itemize}

\etape{Calcul du gradient de la masse volumique $\rho^n$ prise au d\'ebut du pas
de temps courant\footnote{{\it i.e.} calcul\'ee \`a partir des
variables du pas de temps pr\'ec\'edent $n$ si n\'ecessaire.} $(n+1)$}
Ce calcul n'a lieu que si les termes de gravit\'e doivent \^etre pris en compte
($\var{IGRARI()} =1$).
\begin{itemize}
\item [$\star$] Appel de \fort{grdcel}  pour calculer le gradient de $\rho^n$
dans les trois directions de l'espace. Les conditions aux limites sur $\rho^n$
sont des conditions de Dirichlet puisque la valeur de $\rho^n$ aux faces de bord
$ik$ (variable \var{IFAC}) est connue et vaut $\rho_{\,b_{\,ik}}$. Pour \'ecrire les conditions aux limites
sous la forme habituelle, $$\rho_{\,b_{\,ik}} = \var{COEFA} + \var{COEFB}
\,\rho^n_{\,I'}$$ on pose alors $\var{COEFA}=
\var{PROPCE(IFAC,IPPROB(IROM(IPHAS)))}$ et $\var{COEFB} = \var{VISCB} = 0$.\\
$\var{PROPCE(1,IPPROB(IROM(IPHAS)))}$ (resp.$\var{VISCB}$) est utilis\'e en lieu
et place de l'habituel \var{COEFA} ($\var{COEFB}$), lors de l'appel \`a \fort{grdcel}.\\
On a donc :\\
$\displaystyle \var{GRAROX}= \frac{\partial \rho^n}{\partial x}\ $,$\displaystyle \ \var{GRAROY}= \frac{\partial
\rho^n}{\partial y}$ et $
\displaystyle \ \var{GRAROZ}= \frac{\partial \rho^n}{\partial z}\ $.

\end{itemize}

Le gradient de $\rho^n$ servira \`a calculer les termes de production par effets de gravit\'e si ces derniers sont pris en compte.

\etape{Boucle \var{ISOU} de $1$ \`a $6$ sur les tensions de Reynolds}
Pour $\var{ISOU} = 1,2,3,4,5,6$, on r\'esout respectivement et dans
l'ordre  les
\'equations de $R_{11}$, $R_{22}$, $R_{33}$, $R_{12}$, $R_{13}$ et $R_{23}$ par
l'appel au sous-programme \fort{resrij}.\\
La r\'esolution se fait par incr\'ement $\delta {R}_{ij}^{\,n+1,k+1}$ , en utilisant la m\^eme m\'ethode que
celle d\'ecrite dans le sous-programme \fort{codits}. On adopte ici les m\^emes notations.
\var{SMBR} est le second membre du syst\`eme \`a inverser, syst\`eme portant sur
les incr\'ements de la variable. \var{ROVSDT} repr\'esente la diagonale de la
matrice, hors convection/diffusion.\\
On va r\'esoudre l'\'equation (\ref{Base_Turrij_Eq_Temp_Rij}) sous forme incr\'ementale en
utilisant \fort{codits}, soit :
\begin{equation}\label{Base_Turrij_Eq_Temp_deltaRij}
\begin{array}{ll}
&\displaystyle \underbrace{\left(\frac {\rho^n_L}{\Delta t^n}
+ \rho^n_L \,C_1\,\frac{\varepsilon^n_L}{k^n_L}(1-\frac{\delta_{ij}}{3})
 - m^n_{\,lm} + \Gamma_L\,+ max(-\alpha^n_{R_{ij}},0)\right)\,|\Omega_l|}
_{\text {$\var{ROVSDT}$ contribuant
\`a la diagonale de la matrice simplifi\'ee de \fort{matrix}}}\,(\delta{R}_{ij}^{\,n+1,p+1})_{\,L}\\\\
&  \underbrace{+\sum\limits_{m\in Vois(l)}\displaystyle \left[
 m^n_{\,lm} \delta R_{ij,\,f_{\,lm}}^{\,n+1,p+1}
- (\mu^n_{\,lm} + \gamma^n_{\,lm})\
\frac{({\delta R}_{ij}^{\,n+1,p+1})_{M}-({\delta R}_{ij}^{\,n+1,p+1})_{L})}{\overline{L'M'}}\,
S_{\,lm} \right]}_{\text { convection upwind pur et diffusion non reconstruite
relatives \`a la matrice simplifi\'ee de \fort{matrix}\footnotemark}} \\
% voir le texte de la footmark plus bas
&= - \displaystyle\frac {\rho^n_L}{\Delta t^n}\,\left(\,(R^{\,n+1,p}_{ij})_L - (R^{\,n}_{ij})_L\,\right)\\
&-\,\underbrace{\displaystyle\int_{\Omega_l} \left(
\dive\,[\,(\rho\,\vect{u})^n\,R^{\,n+1,p}_{ij} - (\mu^n\,+ \gamma^n\,)\,
\grad{R^{\,n+1,p}_{ij}}\,]\right)\,d\Omega}_{\text {convection et diffusion
trait\'ees par \fort{bilsc2}}}\\
&+\displaystyle \int_{\Omega_l} \left[\,\mathcal{P}^{\,n+1,p}_{ij} + \mathcal{G}^{\,n+1,p}_{ij}
- \displaystyle\rho^n \,C_1\,\frac{\varepsilon^n}{k^n}\left[R^{\,n+1,p}_{ij}-
\frac{2}{3}\,k^n\,\delta_{ij}\right] + \phi^{\,n+1,p}_{ij,2} +
\phi^{\,n+1,p}_{ij,w}\,\right]\, d\Omega \\
& + \displaystyle\int_{\Omega_l} \left[- \frac{2}{3} \rho^n \varepsilon^n \delta_{ij}
 + \Gamma\,(\,R^{\,in}_{ij} - R^{\,n+1,p}_{ij}\,) +
\alpha^n_{R_{ij}}\,R^{\,n+1,p}_{ij}+ \beta^n_{R_{ij}}\right]\, d\Omega\\
&+ \sum\limits_{m\in
Vois(l)}\displaystyle \left[\ \tens{E}^n\,\grad{R}^{\,n+1,p}_{ij} \right]_{\,lm}\,.\,\vect{n}_{\,lm}S_{\,lm}\\
&+ \sum\limits_{m\in Vois(l)}\displaystyle \left[\
\tens{D}^n\,\grad{R}^{\,n+1,p}_{ij} \right]_{\,lm}\,.\,\vect{n}_{\,lm}S_{\,lm}\\
&- \sum\limits_{m\in Vois(l)} \gamma^n_{\,lm} \left( \grad{R}^{\,n+1,p}_{ij}\,.\,\vect{n}_{\,lm} \right)  S_{\,lm}\\
&+ \sum\limits_{m\in Vois(l)}  m^n_{\,lm}\,(R^{\,n+1,p}_{ij})_L\\
\end{array}
\end{equation}
% si on ne fait pas comme ca, il n'apparait pas
\footnotetext[\thefootnote]{Si $\var{IRIJNU} = 1$, on remplace  $\mu^n_{\,lm}$ par $(\mu +
\mu_t)^n_{\,lm}$ dans l'expression de la diffusion non reconstruite
associ\'ee \`a la matrice simplifi\'ee de \fort{matrix} ($\mu_t$ d\'esigne la
viscosit\'e turbulente calcul\'ee comme en $k-\varepsilon$).}

o\`u on rappelle :\\
pour $n$ donn\'e entier positif, on d\'efinit la suite
 $({R}_{ij}^{\,n+1,p})_{p \in \grandN}$
 par :
\begin{equation}\notag
\left\{\begin{array}{l}
{R}_{ij}^{\,n+1,0} = {R}_{ij}^{\,n}\\
{R}_{ij}^{\,n+1,p+1} = {R}_{ij}^{\,n+1,p} + \delta{R}_{ij}^{\,n+1,p+1} \\
\end{array}\right.
\end{equation}
$(\delta{R}_{ij}^{\,n+1,p+1})_{\,L}$ d\'esigne la valeur sur l'\'el\'ement
$\Omega_l$ du $\text{$(\,p+1\,)$-i\`eme}$ incr\'ement de ${R}_{ij}^{\,n+1}$,
$ m^n_{\,lm}$ le flux de masse \`a l'instant $n$ \`a travers la face $lm$,
$\delta R_{ij,\,f_{\,lm}}^{\,n+1,p+1}$ vaut $({\delta
R}_{ij}^{\,n+1,p+1})_{L}$  si $ m^n_{\,lm} \geqslant 0$, $({\delta
R}_{ij}^{\,n+1,p+1})_{M}$ sinon,
$\mathcal{P}^{\,n+1,p}_{ij}$, $\phi^{\,n+1,p}_{ij,2}$, $\phi^{\,n+1,p}_{ij,w}$ les valeurs
des quantit\'es associ\'ees correspondant \`a l'incr\'ement
$(\delta{R}_{ij}^{\,n+1,p})$.\\



Tous ces termes sont calcul\'es comme suit :
\begin{itemize}
\item Terme de gauche de l'\'equation (\ref{Base_Turrij_Eq_Temp_deltaRij})\\
Dans \fort{resrij} est calcul\'ee la variable \var{ROVSDT}. Les autres
termes sont compl\'et\'es par \fort{codits}, lors de la construction de la matrice simplifi\'ee , {\it via} un
appel au sous-programme \fort{matrix}. La quantit\'e
 $(\mu^n_{\,lm} + \gamma^n_{\,lm})$ \`a la face $lm$ est calcul\'ee lors de l'appel \`a
\fort{visort}.\\
\item Second membre de l'\'equation (\ref{Base_Turrij_Eq_Temp_deltaRij})\\
Le premier terme non d\'etaill\'e est calcul\'e par le sous-programme
\fort{bilsc2}, qui applique le sch\'ema convectif choisi par l'utilisateur, qui
reconstruit ou non selon le souhait de l'utilisateur les gradients intervenants
dans la convection-diffusion.\\
Les termes sans accolade sont, eux, compl\`etement explicites et ajout\'es au fur et
\`a mesure dans \var{SMBR} pour former
l'expression $f^{\,exp}_s$ de \fort{codits}.
\end{itemize}
On d\'ecrit ci-dessous les \'etapes de \fort{resrij} :
\begin{itemize}

\item DELTIJ = 1, pour $\var{ISOU} \leqslant 3$ et DELTIJ = 0  Si $\var{ISOU} >
3$. Cette valeur repr\'esente le symbole de Kroeneker $\delta_{ij}$.

\item Initialisation \`a z\'ero de \var{SMBR} (tableau contenant le second
membre) et \var{ROVSDT} (tableau contenant la diagonale de la matrice sauf celle
relative \`a la contribution de la
diagonale des op\'erateurs de convection et de diffusion lin\'earis\'es
\footnote{qui correspondent aux sch\'emas convectif upwind pur et diffusif sans
reconstruction.}), tous deux de dimension $\var{NCEL}$.

\item Lecture et prise en compte des termes sources utilisateur pour la variable $R_{ij}$

Appel \`a \fort{ustsri} pour charger les termes sources utilisateurs. Ils sont
stock\'es comme suit. Pour la cellule $\Omega_l$ de centre $L$, repr\'esent\'ee par $\var{IEL}$, on a :\\
\begin{equation}\notag
\left\{\begin{array}{lll}
&\var{ROVSDT(IEL)}&= |\Omega_l| \ \alpha_{R_{ij}}\\
&\var{SMBR(IEL)}&=|\Omega_l| \ \beta_{R_{ij}}\\
\end{array}\right.
\end{equation}
On affecte alors les valeurs ad\'equates au second membre \var{SMBR} et \`a la
diagonale \var{ROVSDT} comme suit :
\begin{equation}\notag
\left\{\begin{array}{lll}
&\var{SMBR(IEL)} &= \var{SMBR(IEL)} +\ |\Omega_l| \ \alpha_{R_{ij}} \ (R^n_{ij})_L \\
&\var{ROVSDT(IEL)}&= \text{max }(-\ |\Omega_l| \ \alpha_{R_{ij}},0)\\
\end{array}\right.
\end{equation}
La valeur de $ \var{ROVSDT}$ est ainsi calcul\'ee pour des raisons de stabilit\'e
num\'erique. En effet, on ne rajoute sur la diagonale que les valeurs positives,
ce qui correspond physiquement \`a impliciter les termes de rappel uniquement.
\item{Calcul du terme source de masse  si $\Gamma_L > 0$}

Appel de \fort{catsma} et incr\'ementation si n\'ecessaire de \var{SMBR} et
\var{ROVSDT} {\it via} :\\
\begin{equation}\notag
\left\{\begin{array}{lll}
\displaystyle \var{SMBR(IEL)} = \var{SMBR(IEL)} + |\Omega_l| \ \Gamma_L \
\left[(R^{\,in}_{ij})_L - (R^{\,n}_{ij})_L \right] \\
\displaystyle \var{ROVSDT(IEL)}=\var{ROVSDT(IEL)} + |\Omega_l| \ \Gamma_L
\end{array}\right.
\end{equation}
\item Calcul du terme d'accumulation de masse et du terme instationnaire

On stocke $\displaystyle \var{W1}= \int_{\Omega_l}\dive\,(\rho\,\vect{u})\,d\Omega$
calcul\'e par \fort{divmas} \`a l'aide des flux de masse aux faces internes
$ m^n_{\,lm}=\sum\limits_{m\in Vois(l)}{(\rho \vect{u})_{\,lm}^n} \text{.}\,
\vect{S}_{\,lm} $ (tableau \var{FLUMAS}) et des flux de masse aux bords  $ m^n_{\,b_{lk}} = \sum\limits_{k\in{\gamma_b(l)}}{(\rho \vect{u})_{\,{b}_{lk}}^n} \text{.}\,
\vect{S}_{\,{b}_{lk}} $ (tableau \var{FLUMAB}).
On incr\'emente ensuite \var{SMBR} et \var{ROVSDT}.
\begin{equation}\notag
\left\{\begin{array}{lll}
&\var{SMBR(IEL)} &= \var{SMBR(IEL)} + \var{ICONV}\  (R^n_{ij})_L\,(\displaystyle
\int_{\Omega_l}\dive\,(\rho\,\vect{u})\ d\Omega) \\
&\var{ROVSDT(IEL)}& = \var{ROVSDT(IEL)} +  \var{ISTAT}\,\displaystyle
\frac{\rho^n_L \ |\Omega_l|}{\Delta t^n} -  \var{ICONV}\ (\displaystyle
\int_{\Omega_l}\dive\,(\rho\,\vect{u})\ d\Omega) \\
\end{array}\right.
\end{equation}
\item Calcul des termes sources de production, des termes $\displaystyle
\phi_{\,ij,1}+\phi_{\,ij,2}$ et de la dissipation~$\displaystyle-\frac{2}{3} \varepsilon\,\delta_{\,ij}$ :

On effectue une boucle d'indice \var{IEL} sur les cellules $\Omega_l$ de centre $L$ :
\begin{itemize}
\item [$\Rightarrow$] $\displaystyle \var{TRPROD}= \frac{1}{2} (\mathcal{P}^n_{ii})_L = \frac{1}{2} \left[ \var{PRODUC(1,IEL)} +  \var{PRODUC(2,IEL)} +  \var{PRODUC(3,IEL)} \right] $
\item [$\Rightarrow$] $\displaystyle \var{TRRIJ }= \frac{1}{2} (R^n_{ii})_L $
\item [$\Rightarrow$] $\displaystyle \var{SMBR(IEL)} =\ \var{SMBR(IEL)}\ +$\\
$\ \displaystyle\rho^n_L |\Omega_l| \left[ \displaystyle
\frac{2}{3}\,\delta_{\,ij} \left( \ \displaystyle \frac{ C_2}{2}\,(\mathcal{P}^n_{ii})_L\ +
(C_1-1)\ \varepsilon^n_L\, \right)\right.$\\
$ + \left.\ (1-C_2) \ \var{PRODUC(ISOU,IEL)} -
\displaystyle C_1\ \frac{2\,\varepsilon^n_L}{(R^n_{ii})_L}\ (R^n_{ij})_L \right]$
\item [$\Rightarrow$] $\displaystyle \var{ROVSDT(IEL)} = \var{ROVSDT(IEL)} +
\rho^n_L \ |\Omega_l| \ (- \displaystyle \frac{1}{3} \ \,\delta_{\,ij} + 1) \ C_1
\ \frac{2\ \varepsilon^n_L}{(R^n_{ii})_L}$
\end{itemize}
\item Appel de \fort{rijech} pour le calcul des termes d'\'echo de paroi
 $\phi^n_{ij,w}$ si $\var{IRIJEC()}=1$ et ajout dans \var{SMBR}.\\
$\var{SMBR} = \var{SMBR} + \phi^n_{ij,w}$\\
Suivant son mode de calcul (\var{ICDPAR}), la distance � la paroi est directement accessible
par \var{RA(IDIPAR+IEL-1)} (\var{|ICDPAR|} = 1) ou bien
est calcul\'ee \`a partir de $\var{IA(IIFAPA(IPHAS)+IEL - 1)}$,
qui donne pour l'\'el\'ement $\var{IEL}$ le num\'ero de la face de bord
paroi la plus  proche (\var{|ICDPAR|} = 2). Ces tableaux ont \'et\'e renseign\'e une fois pour toutes au
d\'ebut de calcul.

\item  Appel de \fort{rijthe} pour le calcul des termes de gravit\'e $\mathcal{G}^n_{ij}$ et ajout dans \var{SMBR}.

Ce calcul n'a lieu que si $\var{IGRARI()} = 1$.
$ \var{SMBR} = \var{SMBR} + \mathcal{G}^n_{ij}$
\item Calcul de la partie explicite du terme de diffusion
 $\dive{\,\left[\tens{A}\,\grad{R}^{\,n}_{ij}\right]}$, plus pr\'ecis\'ement
des contributions du terme extradiagonal pris aux faces purement internes
(remplissage du tableau \var{VISCF}), puis aux faces de bord (remplissage du
tableau \var{VISCB}).
\begin{itemize}
\item [$\star$] Appel de \fort{grdcel} pour le calcul du gradient de
$R^{\,n}_{ij}$ dans chaque direction. Ces gradients sont respectivement
stock\'es dans les tableaux de travail \var{W1}, \var{W2} et \var{W3}.

\item [$\star$] boucle d'indice \var{IEL} sur les cellules $\Omega_l$ de centre
$L$ pour le
calcul de $\tens{E}^n\,\grad{R}^{\,n}_{ij}$ aux cellules dans un premier temps :\\
\begin{itemize}
\item [$\Rightarrow$] $\displaystyle \var{TRRIJ}= \frac{1}{2} (R^{\,n}_{ii})_L $
\item [$\Rightarrow$] $\displaystyle \var{CSTRIJ} = \rho^n_L\ C_S \ \displaystyle\frac{(R^n_{ii})_L}{2\,\varepsilon^n_L}$
\item [$\Rightarrow$] $\displaystyle \var{W4(IEL)} = \rho^n_L\ C_S\
\displaystyle\frac{(R^n_{ii})_L}{2\,\varepsilon^n_L} \left[\,(R^{\,n}_{12})_L \ \var{W2(IEL)} +
(R^{\,n}_{13})_L \ \var{W3(IEL)}\,\right]$
\item [$\Rightarrow$] $\displaystyle \var{W5(IEL)} = \rho^n_L\ C_S\
\displaystyle\frac{(R^n_{ii})_L}{2\,\varepsilon^n_L} \left[\,(R^{\,n}_{12})_L \ \var{W1(IEL)} +
(R^{\,n}_{23})_L \ \var{W3(IEL)}\,\right]$
\item [$\Rightarrow$] $\displaystyle \var{W6(IEL)} = \rho^n_L\ C_S\
\displaystyle\frac{(R^n_{ii})_L}{2\,\varepsilon^n_L} \left[\,(R^{\,n}_{13})_L \ \var{W1(IEL)} + (R^{\,n}_{23})_L \ \var{W2(IEL)}\,\right]$
\end{itemize}



\item [$\star$] Appel de \fort{vectds}\footnote{Le r\'esultat est stock\'e dans
\var{VISCF} et \var{VISCB}. Dans \fort{vectds}, les valeurs aux faces internes
sont interpol\'ees lin\'eairement sans reconstruction et \var{VISCB} est mis \`a
z\'ero.} pour assembler $\displaystyle\left[ \tens{E}^n\,\grad{R}^{\,n}_{ij}
\right]\,.\,\vect{n}_{\,lm}S_{\,lm}$ aux faces $lm$.
\item [$\star$] Appel de \fort{divmas} pour calculer la divergence du flux d\'efini par \var{VISCF} et \var{VISCB}.
Le r\'esultat est stock\'e dans \var{W4}.\\
Ajout au second membre \var{SMBR}.\\
\var{SMBR} = \var{SMBR} + \var{W4}
\end{itemize}

A l'issue de cette \'etape, seule la partie extradiagonale de la diffusion prise
enti\`erement explicite~:
 $$\sum\limits_{m\in
Vois(l)}\left[\ \tens{E}^n\,\grad{R}^{\,n}_{ij} \right]_{\,lm}\,.\,\vect{n}_{\,lm}S_{\,lm}$$ a \'et\'e calcul\'ee.\\

\item Calcul de la partie diagonale du terme de diffusion\footnote{Seule la
composante normale  du  gradient de $R_{ij}$ aux faces sera implicite.} :\\
Comme on l'a d\'eja vu, une partie de cette contribution sera trait\'ee en
implicite (celle relative \`a la composante normale du gradient) et donc
ajout\'ee au second membre par \fort{bilsc2} ; l'autre
partie sera explicite et prise en compte dans $f_s^{\,exp}$.
\begin{itemize}
\item [$\star$] On effectue une boucle d'indice \var{IEL} sur les cellules
$\Omega_l$ de centre $L$ :
\begin{itemize}
\item [$\Rightarrow$] $\displaystyle \var{TRRIJ }= \frac{1}{2} (R^{\,n}_{ii})_L $
\item [$\Rightarrow$] $\displaystyle \var{CSTRIJ} = \rho^n_L \ C_S \ \frac{(R^{\,n}_{ii})_L}{2\,\varepsilon^n_L}$
\item [$\Rightarrow$] $\displaystyle \var{W4(IEL)} = \rho^n_L \ C_S \
\frac{(R^{\,n}_{ii})_L}{2\,\varepsilon^n_L} \ (R^{\,n}_{11})_L$
\item [$\Rightarrow$] $\displaystyle \var{W5(IEL)} = \rho^n_L \ C_S \ \frac{(R^{\,n}_{ii})_L}{2\,\varepsilon^n_L}\ (R^n_{22})_L$
\item [$\Rightarrow$] $\displaystyle \var{W6(IEL)} = \rho^n_L \ C_S \ \frac{(R^{\,n}_{ii})_L}{2\,\varepsilon^n_L} \ (R^n_{33})_L$
\end{itemize}

%\item Traitement du parall\'elisme et de la p\'eriodicit\'e.

\item [$\star$] On effectue une boucle d'indice \var{IFAC} sur les faces
purement internes $lm$ pour remplir le tableau \var{VISCF} :
\begin{itemize}
\item [$\Rightarrow$] Identification des cellules $\Omega_l$ et $\Omega_m$ de
centre respectif $L$ (variable \var{II}) et $M$ (variable \var{JJ}), se trouvant de chaque c\^ot\'e de la face
$lm$\footnote{La normale \`a la face est orient\'ee de L vers M.}.
\item [$\Rightarrow$] Calcul du carr\'e de la surface de la face. La valeur est
stock\'ee dans le tableau \var{SURFN2}.
\item [$\Rightarrow$] Interpolation du gradient de $R^{\,n}_{ij}$ \`a la face
$lm$ (gradient facette $\left[\grad{R}^{\,n}_{ij}\right]_{\,lm}$) :
\begin{equation}\notag
\left\{\begin{array}{ll}
\var{GRDPX} &= \displaystyle \frac{1}{2} \left(\var{W1(II)} + \var{W1(JJ)}
\right) \\
&\\
\var{GRDPY} &= \displaystyle \frac{1}{2} \left(\var{W2(II)} + \var{W2(JJ)} \right) \\
&\\
\var{GRDPZ} &= \displaystyle \frac{1}{2} \left(\var{W3(II)} + \var{W3(JJ)} \right)
\end{array}\right.
\end{equation}
\item [$\Rightarrow$] Calcul du gradient de $R^{\,n}_{ij}$ normal \`a la face
$lm$, $\left[\grad{R}^{\,n}_{ij}\right]_{\,lm}.\vect{n}_{\,lm}\,S_{\,lm}$.\\

$\displaystyle \var{GRDSN} =  \var{GRDPX} \ \var{SURFAC(1,IFAC)} + \var{GRDPY} \ \var{SURFAC(2,IFAC)} +  \var{GRDPZ} \ \var{SURFAC(3,IFAC)}$
$\var{SURFAC}$ est le vecteur surface de la face \var{IFAC}.


\item [$\Rightarrow$] calcul de
 $\left[\grad{R^{\,n}_{ij}} - (\grad
R^{\,n}_{ij}\,.\,\vect{n}_{\,lm})\vect{n}_{\,lm}\right]$, les vecteurs \'etant
calcul\'es \`a la face $lm$ :
\begin{equation}\notag
\left\{\begin{array}{lll}
&\displaystyle \var{GRDPX} &= \var{GRDPX} - \displaystyle\frac{\var{GRDSN}}{\var{SURFN2}} \ \var{SURFAC(1,IFAC)}\\
&&\\
&\displaystyle \var{GRDPY} &= \var{GRDPY} - \displaystyle\frac{\var{GRDSN}}{\var{SURFN2}} \ \var{SURFAC(2,IFAC)} \\
&&\\
&\displaystyle \var{GRDPZ} &= \var{GRDPZ} - \displaystyle\frac{\var{GRDSN}}{\var{SURFN2}} \ \var{SURFAC(3,IFAC)}
\end{array}\right.
\end{equation}
\item [$\Rightarrow$] finalisation du calcul de l'expression totalement
explicite
 $$\left[ \tens{D}^n\,\left( \grad{R^{\,n}_{ij}} - (\grad R^{\,n}_{ij}\,.\,\vect{n}_{\,lm})\,\vect{n}_{\,lm}\right) \right]\,.\,\vect{n}_{\,lm}$$
\begin{equation}\notag
\begin{array} {ll}
\displaystyle \var{VISCF} = &
 \displaystyle\frac{1}{2} (\ \var{W4(II)} +\ \var{W4(JJ)}) \ \var{GRDPX} \
\var{SURFAC(1,IFAC)})\ + \\
&\\
&  \displaystyle\frac{1}{2} (\ \var{W5(II)} +\ \var{W5(JJ)}) \ \var{GRDPY} \
\var{SURFAC(2,IFAC)})\ + \\
&\\
&  \displaystyle\frac{1}{2} (\ \var{W6(II)} +\ \var{W6(JJ)}) \ \var{GRDPZ} \ \var{SURFAC(3,IFAC)})
\end{array}
\end{equation}
\end{itemize}

\item [$\star$] Mise \`a z\'ero du tableau \var{VISCB}.

\item [$\star$] Appel de \fort{divmas} pour calculer la divergence de~:
 $$\tens{D}^{\,n}\,\left( \grad{R^{\,n}_{ij}} - (\grad R^{\,n}_{ij}\,.\,\vect{n}_{\,lm})\vect{n}_{\,lm}\right)$$ d\'efini aux faces dans \var{VISCF} et \var{VISCB}.

Le r\'esultat est stock\'e dans le tableau \var{W1}.\\
Ajout au second membre \var{SMBR}.\\
$\var{SMBR} = \var{SMBR} + \var{W1}$
\end{itemize}
\item Calcul de la viscosit\'e orthotrope $\gamma^n_{\,lm}$ \`a la face $lm$ de la variable principale
$R^{\,n}_{ij}$\\
Ce calcul permet au sous-programme \fort{codits} de compl\'eter le second membre
\var{SMBR} par :
\begin{equation}
\begin{array} {ll}
& \sum\limits_{m\in Vois(l)}
\mu^n_{\,lm}\,\left(\grad{R}^{\,n}_{ij}\,.\,\vect{n}_{\,lm}\right)S_{\,lm}
 + \sum\limits_{m\in Vois(l)} \left(\grad{R}^{\,n}_{ij}
\,.\,\vect{n}_{\,lm}\right)\left[\tens{D}^{\,n}\,\vect{n}_{\,lm}\right]_{\,lm}\,.\,\vect{n}_{\,lm}\
S_{\,lm}\\
& = \sum\limits_{m\in Vois(l)}(\,\mu^n_{\,lm}\, + \,\gamma^n_{\,lm}\,)\,\left(\grad{R}^{\,n}_{ij}\,.\,\vect{n}_{\,lm}\right)S_{\,lm}
\end{array}
\end{equation}
sans pr\'eciser la nature de la face $lm$, {\it via} l'appel \`a \fort{bilsc2}  et de disposer de la quantit\'e
$(\mu^n_{\,lm}\, + \gamma^n_{\,lm})$ pour construire sa
matrice simplifi\'ee.\\
\begin{itemize}
\item [$\star$] On effectue une boucle d'indice \var{IEL} sur les cellules
$\Omega_l$ :
\begin{itemize}
\item [$\Rightarrow$] $\displaystyle \var{TRRIJ }= \frac{1}{2} (R^{\,n}_{ii})_L $
\item [$\Rightarrow$] $\displaystyle \var{RCSTE} = \rho^n_L \ C_S \ \frac{ (R^{\,n}_{ii})_L}{2\,\varepsilon^n_L} $
\item [$\Rightarrow$] $\displaystyle \var{W1(IEL)} = \mu^n +\rho^n_L \ C_S \ \frac{
(R^{\,n}_{ii})_L}{2\,\varepsilon^n_L}\ (R^n_{11})_L$
\item [$\Rightarrow$] $\displaystyle \var{W2(IEL)} = \mu^n + \rho^n_L \ C_S \ \frac{ (R^{\,n}_{ii})_L}{2\,\varepsilon^n_L}\ (R^n_{22})_L$
\item [$\Rightarrow$] $\displaystyle \var{W3(IEL)} = \mu^n + \rho^n_L \ C_S \ \frac{ (R^{\,n}_{ii})_L}{2\,\varepsilon^n_L}\ (R^n_{33})_L$
\end{itemize}

\item [$\star$] Appel de \fort{visort} pour calculer la viscosit\'e orthotrope
\footnote{Comme dans le sous-programme \fort{viscfa}, on multiplie la viscosit\'e par
$\displaystyle \frac{S_{\,lm}}{\overline{L'M'}}$, o\`u $S_{\,lm}$ et
$\overline{L'M'}$ repr\'esentent respectivement la surface de la face $lm$ et la
mesure alg\'ebrique du segment reliant les projections des centres des cellules
voisines sur la normale \`a la face. On garde dans ce sous-programme  la possibilit\'e d'interpoler la viscosit\'e aux cellules lin\'eairement ou d'utiliser une moyenne harmonique. La viscosit\'e au bord est celle de la cellule de bord correspondante.}
$ \gamma^n_{\,lm} = (\tens{D}^{\,n}\,\vect{n}_{\,lm}).\vect{n}_{\,lm}$ aux faces $lm$

Le r\'esultat est stock\'e dans les tableaux \var{VISCF} et \var{VISCB}.
\end{itemize}

\item appel de \fort{codits} pour la r\'esolution de l'\'equation de
convection/diffusion/termes sources de la variable $R_{ij}$. Le terme source est
\var{SMBR}, la viscosit\'e \var{VISCF} aux faces purement internes (
resp. \var{VISCB} aux faces de bord) et \var{FLUMAS} le flux de masse interne
 ( resp. \var{FLUMAB} flux de masse au bord). Le r\'esultat est la variable $R_{ij}$ au temps
$n+1$, donc $R^{\,n+1}_{ij}$. Elle est stock\'ee dans le tableau \var{RTP} des
variables mises \`a jour.

\end{itemize}

\etape{Appel de \fort{reseps} pour la r\'esolution de la variable $\varepsilon$}

On d\'ecrit ci-dessous le sous-programme \fort{reseps}, les commentaires du sous-programme \fort{resrij} ne seront pas r\'ep\'et\'es puisque les deux sous-programmes ne diff\`erent que par leurs termes sources.

\begin{itemize}
\item Initialisation \`a z\'ero de \var{SMBR} et \var{ROVSDT}.

\item{Lecture et prise en compte des termes sources utilisateur pour la variable $\varepsilon$ :}

Appel de \fort{ustsri} pour charger les termes sources utilisateurs. Ils sont
stock\'es dans les tableaux suivants :\\
pour la cellule $\Omega_l$ repr\'esent\'ee par $\var{IEL}$ de centre $L$, on a :
\begin{equation}\notag
\left\{\begin{array}{lll}
&\var{ROVSDT(IEL)}&= |\Omega_l| \ \alpha_{\varepsilon}\\
&\var{SMBR(IEL)}&=|\Omega_l| \ \beta_{\varepsilon}\\
\end{array}\right.
\end{equation}
On affecte alors les valeurs ad\'equates au second membre \var{SMBR} et \`a la
diagonale \var{ROVSDT} comme suit :
\begin{equation}\notag
\left\{\begin{array}{lll}
&\var{SMBR(IEL)} &= \var{SMBR(IEL)} +\ |\Omega_l| \ \alpha_{\,\varepsilon} \
\varepsilon^n_L \\
&\var{ROVSDT(IEL)}&= \text{max }(-\ |\Omega_l| \ \alpha_{\,\varepsilon},0)\\
\end{array}\right.
\end{equation}

\item{Calcul du terme source de masse si $\Gamma_L > 0$ :
\begin{equation}\notag
\left\{\begin{array}{lll}
&\displaystyle \var{SMBR(IEL)} = \var{SMBR(IEL)} + |\Omega_l| \ \Gamma_L \
(\varepsilon^{\,in}_L -\varepsilon^n_L) \\
&\displaystyle \var{ROVSDT(IEL)}= \var{ROVSDT(IEL)} + |\Omega_l| \ \Gamma_L
\end{array}\right.
\end{equation}
\item Calcul du terme d'accumulation de masse et du terme instationnaire \\
On stocke $\displaystyle \var{W1}= \int_{\Omega_l}\dive\,(\rho\,\vect{u})\,d\Omega$
calcul\'e par \fort{divmas} \`a l'aide des flux de masse internes et aux bords.\\
On incr\'emente ensuite \var{SMBR} et \var{ROVSDT}.
\begin{equation}\notag
\left\{\begin{array}{lll}
&\var{SMBR(IEL)} &= \var{SMBR(IEL)} + \var{ICONV}\ \varepsilon^n_L\,(\displaystyle
\int_{\Omega_l}\dive\,(\rho\,\vect{u})\ d\Omega) \\
&\var{ROVSDT(IEL)}& = \var{ROVSDT(IEL)} +  \var{ISTAT}\,\displaystyle
\frac{\rho^n_L \ |\Omega_l|}{\Delta t^n} -  \var{ICONV}\ (\displaystyle
\int_{\Omega_l}\dive\,(\rho\,\vect{u})\ d\Omega) \\
\end{array}\right.
\end{equation}

\item Traitement du terme de production
 $\displaystyle \rho\,C_{\varepsilon_1}\,\frac{\varepsilon}{k}\,\mathcal{P}$
 et du terme de dissipation $-\,\displaystyle \rho\,C_{\varepsilon_2}\,\frac{\varepsilon}{k}\,\varepsilon$ \\
pour cela, on effectue une boucle d'indice \var{IEL} sur les cellules $\Omega_l$
de centre $L$ :
\begin{itemize}
\item [$\Rightarrow$] $\displaystyle \var{TRPROD}= \frac{1}{2} (\mathcal{P}^n_{ii})_L = \frac{1}{2} \left[ \var{PRODUC(1,IEL)} +  \var{PRODUC(2,IEL)} +  \var{PRODUC(3,IEL)} \right] $
\item [$\Rightarrow$] $\displaystyle \var{TRRIJ }= \frac{1}{2} (R^n_{ii})_L $
\item [$\Rightarrow$] $\displaystyle \var{SMBR(IEL)} = \var{SMBR(IEL)} + \rho^n_L
|\Omega_l| \left[ -C_{\varepsilon_2} \ \frac{2\,(\varepsilon^n_L)^2}{(R^n_{ii})_L} + C_{\varepsilon_1} \ \frac{\varepsilon^n_L}{(R^n_{ii})_L}\ (\mathcal{P}^n_{ii})_L \right] $
\item [$\Rightarrow$] $\displaystyle \var{ROVSDT(IEL)} = \var{ROVSDT(IEL)} + C_{\varepsilon_2} \ \rho^n_L \ |\Omega_l| \ \frac{2\,\varepsilon^n_L}{(R^n_{ii})_L}$
\end{itemize}

\item Appel de \fort{rijthe} pour le calcul des termes de gravit\'e $\mathcal{G}^n_{\varepsilon}$ et ajout dans \var{SMBR}.

$ \var{SMBR} = \var{SMBR} + \mathcal{G}^n_{\varepsilon}$\\
Ce calcul n'a lieu que si $\var{IGRARI()} = 1$.

\item Calcul de la diffusion de $\varepsilon$ \\
 Le terme $\dive \left[\mu\, \grad(\varepsilon) + \tens{A'}\,\grad(\varepsilon)
\right]$ est calcul\'e exactement de la m\^eme mani\`ere que pour les tensions
de Reynolds $R_{ij}$ en rempla\c cant $\tens{A}$ par $\tens{A'}$.

\item Appel de \fort{codits} pour la r\'esolution de l'\'equation de
convection/diffusion/termes sources de la variable principale $\varepsilon$. Le
r\'esultat $\varepsilon^{\,n+1}$ est stock\'e dans le tableau \var{RTP} des
variables mises \`a jour.
}
\end{itemize}

\etape{clippings finaux}
On passe enfin dans le sous-programme  \fort{clprij} pour faire un clipping \'eventuel
des variables $R^{\,n+1}_{ij}$ et $\varepsilon^{\,n+1}$. Le sous-programme
\fort{clprij} est appel\'e\footnote{L'option
$\var{ICLIP} = 1$ consiste en un clipping minimal des variables $R_{ii}$ et
$\varepsilon$ en prenant la valeur absolue de ces variables puisqu'elles ne
peuvent \^etre que positives.} avec $\var{ICLIP} = 2$ . Cette option
\footnote{Quand la valeur des grandeurs $R_{ii}$ ou $\varepsilon$ est
n\'egative, on la remplace par le minimum entre sa valeur absolue et (1,1)
fois la valeur obtenue au pas de temps pr\'ec\'edent.} contient l'option $\var{ICLIP} = 1$  et permet de v\'erifier l'in\'egalit\'e de Cauchy-Schwarz sur les grandeurs extra-diagonales du tenseur $\tens{R}$ (pour $i \neq j$, $|R_{ij}|^2 \le R_{ii} R_{jj}$).


%%%%%%%%%%%%%%%%%%%%%%%%%%%%%%%%%%
%%%%%%%%%%%%%%%%%%%%%%%%%%%%%%%%%%
\section{Points \`a traiter}
%%%%%%%%%%%%%%%%%%%%%%%%%%%%%%%%%%
%%%%%%%%%%%%%%%%%%%%%%%%%%%%%%%%%%
Sauf mention explicite, $\phi$ repr\'esentera une tension de Reynolds ou la dissipation turbulente ($\phi = R_{ij} \ \text{ou} \ \varepsilon$).

\begin{itemize}
\item {La vitesse utilis\'ee pour le calcul de la production est explicite. Est-ce qu'une implicitation peut am\'eliorer la pr\'ecision temporelle de $\phi$ \footnote{Cette remarque peut \^etre g\'en\'eralis\'ee. En effet, peut-on envisager d'actualiser les variables d\'ej\`a r\'esolues (sans r\'eactualiser les variables turbulentes apr\`es leur r\'esolution)? Ceci obligerait \`a modifier les sous-programmes tels que \fort{condli} qui sont appel\'es au d\'ebut de la boucle en temps.} ?}
\item {Dans quelle mesure le terme d'\'echo de paroi est-il valide ? En effet, ce terme est remis en question par certains auteurs.}
\item {On peut envisager la r\'esolution d'un syst\`eme hyperbolique pour les
tensions de Reynolds afin d'introduire un couplage avec le champ de vitesse.}
\item {Le flux au bord \var{VISCB} est annul\'e dans le sous-programme
\fort{vectds}. Peut-on envisager de mettre au bord la valeur de la variable
concern\'ee \`a la cellule de bord correspondant? De m\^eme, il faudrait se
pencher sur les hypoth\`eses sous-jacentes \`a l'annulation des contributions
aux bords de \var{VISCB} lors du calcul de : $$\left[ \tens{D}^n\,\left( \grad{R^{\,n}_{ij}} - (\grad R^{\,n}_{ij}\,.\,\vect{n}_{\,lm})\,\vect{n}_{\,lm}\right) \right]\,.\,\vect{n}_{\,lm}.$$}
\item {Un probl\`eme de pond\'eration appara\^\i t plus g\'en\'eralement. Si on prend la partie explicite de $\tens{D}\,\grad(\phi)$, la pond\'eration aux faces internes utilise le coefficient $\displaystyle\frac{1}{2}$ avec pond\'eration s\'epar\'ee de $\tens{D}$ et $\grad(\phi)$, alors que pour $\tens{E}\,\grad(\phi)$, on calcule d'abord ce terme aux cellules pour ensuite l'interpoler lin\'eairement aux faces \footnote{Cette interpolation se fait dans \fort{vectds} avec des coefficients de pond\'eration aux faces.}. Ceci donne donc deux types d'interpolations pour des termes de m\^eme nature.}
\item {On laisse la possibilit\'e dans \fort{visort} d'utiliser une moyenne
harmonique aux faces. Est-ce que ceci est valable puisque les interpolations
utilis\'ees lors du calcul de la partie explicite de $\tens{A}\,\grad{\phi}$
sont des moyennes arithm\'etiques ?}
\item {Les techniques adopt\'ees lors du clipping sont \`a revoir.}
\item {On utilise dans le cadre du mod\`ele $\displaystyle R_{ij}-\varepsilon $ une semi-implicitation de termes comme $\displaystyle \phi_{ij,1}$ ou $\displaystyle -\rho\,C_{\varepsilon_2}\,\frac{\varepsilon}{k}\,\varepsilon$. On peut envisager le m\^eme type d'implicitation dans \fort{turbke} m\^eme en pr\'esence du couplage $\displaystyle k-\varepsilon$.}
\item L'adoption d'une r\'esolution d\'ecoupl\'ee fait perdre l'invariance par rotation.
\item La formulation et l'implantation des conditions aux limites de paroi
devront \^etre v\'erifi\'ees. En effet, il semblerait que, dans certains cas, des ph\'enom\`enes
``oscillatoires'' apparaissent, sans qu'il soit ais\'e d'en d\'eterminer la cause.
\item L'implicitation partielle (du fait de la r\'esolution d\'ecoupl\'ee) des
conditions aux limites conduit souvent \`a des calculs instables. Il
conviendrait d'en conna\^\i tre la raison. L'implicitation partielle avait
\'et\'e mise en \oe uvre afin de tenter d'utiliser un pas de temps plus grand
dans le cas de jets axisym\'etriques en particulier.

\end{itemize}

\part{Module �lectrique}
%                      Code_Saturne version 1.3
%                      ------------------------
%
%     This file is part of the Code_Saturne Kernel, element of the
%     Code_Saturne CFD tool.
%
%     Copyright (C) 1998-2007 EDF S.A., France
%
%     contact: saturne-support@edf.fr
%
%     The Code_Saturne Kernel is free software; you can redistribute it
%     and/or modify it under the terms of the GNU General Public License
%     as published by the Free Software Foundation; either version 2 of
%     the License, or (at your option) any later version.
%
%     The Code_Saturne Kernel is distributed in the hope that it will be
%     useful, but WITHOUT ANY WARRANTY; without even the implied warranty
%     of MERCHANTABILITY or FITNESS FOR A PARTICULAR PURPOSE.  See the
%     GNU General Public License for more details.
%
%     You should have received a copy of the GNU General Public License
%     along with the Code_Saturne Kernel; if not, write to the
%     Free Software Foundation, Inc.,
%     51 Franklin St, Fifth Floor,
%     Boston, MA  02110-1301  USA
%
%-----------------------------------------------------------------------
%
\programme{vortex}
%
\vspace{1cm}
%%%%%%%%%%%%%%%%%%%%%%%%%%%%%%%%%%
%%%%%%%%%%%%%%%%%%%%%%%%%%%%%%%%%%
\section{Fonction}
%%%%%%%%%%%%%%%%%%%%%%%%%%%%%%%%%%
%%%%%%%%%%%%%%%%%%%%%%%%%%%%%%%%%%
Ce sous-programme est d�di� � la g�n�ration des conditions d'entr�e
turbulente utilis�es en LES.


La m�thode des vortex est bas�e sur une approche de tourbillons
ponctuels. L'id�e de la m�thode consiste � injecter des tourbillons 2D dans le
plan d'entr�e du calcul, puis � calculer le champ de vitesse induit par ces
tourbillons au centre des faces d'entr�e.

%                      Code_Saturne version 1.3
%                      ------------------------
%
%     This file is part of the Code_Saturne Kernel, element of the
%     Code_Saturne CFD tool.
% 
%     Copyright (C) 1998-2007 EDF S.A., France
%
%     contact: saturne-support@edf.fr
% 
%     The Code_Saturne Kernel is free software; you can redistribute it
%     and/or modify it under the terms of the GNU General Public License
%     as published by the Free Software Foundation; either version 2 of
%     the License, or (at your option) any later version.
% 
%     The Code_Saturne Kernel is distributed in the hope that it will be
%     useful, but WITHOUT ANY WARRANTY; without even the implied warranty
%     of MERCHANTABILITY or FITNESS FOR A PARTICULAR PURPOSE.  See the
%     GNU General Public License for more details.
% 
%     You should have received a copy of the GNU General Public License
%     along with the Code_Saturne Kernel; if not, write to the
%     Free Software Foundation, Inc.,
%     51 Franklin St, Fifth Floor,
%     Boston, MA  02110-1301  USA
%
%-----------------------------------------------------------------------
%
%%%%%%%%%%%%%%%%%%%%%%%%%%%%%%%%%%
%%%%%%%%%%%%%%%%%%%%%%%%%%%%%%%%%%
\section{Discr\'etisation}
%%%%%%%%%%%%%%%%%%%%%%%%%%%%%%%%%%
%%%%%%%%%%%%%%%%%%%%%%%%%%%%%%%%%%

Le terme convectif en $\dive(\underline{u} \otimes \rho\,\underline{u})$
introduit une non lin\'earit\'e et un couplage des composantes de la vitesse
$\vect{u}$ dans l'�quation (\ref{Base_Preduv_eqqdm}). Une lin\'earisation et un d\'ecouplage
des trois composantes de la 
vitesse sont r\'ealis\'es lors de la discr\'etisation de cette \'etape de
pr\'ediction.\\
En effet, soit :
\begin{equation}
\vect{\widetilde{u}}= \vect{u}^n + \delta \vect{u} 
\end{equation}
La contribution exacte du terme convectif \`a prendre en compte dans cette
\'etape de pr\'ediction serait :\\
\begin{equation}\label{Base_Preduv_Conv_exact}
\begin{array}{ll}
\dive(\vect{\widetilde{u}} \otimes \rho\,\vect{\widetilde{u}}) =
\dive(\vect{u}^{n} \otimes \rho\,\vect{u}^{n}) + \dive(\delta \vect{u} \otimes
\rho\,\vect{u}^{n}) +  \underbrace { \dive(\vect{u}^{n} \otimes
\rho\,\delta \vect{u})}_{\text {terme couplant lin\'eaire}} +  \underbrace { \dive(\delta \vect{u} \otimes
\rho\,\delta \vect{u})}_{\text {terme couplant et non lin\'eaire}}\\
\end{array} 
\end{equation}
Les deux derniers termes de l'expression (\ref{Base_Preduv_Conv_exact}) sont {\em a priori} n�glig�s
de mani�re � obtenir un syst\`eme en vitesse qui soit d\'ecoupl\'e et donc,
�viter l'inversion d'une matrice pouvant \^etre de tr\`es grande taille. Ces
deux termes peuvent n�anmoins �tre pris en compte de mani�re plus ou moins
approch�e par extrapolation explicite du flux de masse en $n+\theta_F$ (pour le
terme couplant lin�aire provenant de la convection de $\vect{u}^{n}$ par $\delta
\vect{u}$) et utilisation d'un point-fixe par sous it�ration sur le sous
programme \fort{navsto} (pour le terme non-lin�aire, en sp�cifiant $\var{NTERUP}>1$).

L'�quation (\ref{Base_Preduv_eqqdm}) est discr�tis�e au temps $n+\theta$ � l'aide d'un
$\theta$-sch�ma, et le tenseur des pertes de charges d�compos� en une partie
diagonale $\tens{K}_{d}$ et une extradiagonale $\tens{K}_{e}$ (soit
 $\tens{K}_{pdc}=\tens{K}_{d}+\tens{K}_{e}$).\\
$\bullet$ La pression est suppos�e connue en $n-1+\theta$ (d�calage temporel
pression-vitesse) et le gradient naturellement calcul� � cet instant.\\ 
$\bullet$ Les termes sources de viscosit� secondaire, de gradient transpos\'e,
ceux provenant du mod�le de turbulence\footnote{except� $\dive (\mu_t\ (\ggrad
\underline {u}))$}, $\rho\,\tens{K}_{\,e}\ \underline{u}$, $(\rho -\rho_0)
\underline {g}$ ainsi que $\underline{T}_{s}^{\,exp}$ et
$\Gamma\,\underline{u}_{\,i}$ sont pris de mani�re explicite au temps $n$, ou
extrapol�s suivant le sch�ma en temps choisi pour les propri�t�s physique et les
termes sources.\\ 
$\bullet$ Les termes sources $\underline{u}\,\,\dive (\rho\,\underline {u})$,
$\Gamma\,\,\underline{u}$, $T_{s}^{\,imp}\,\,\underline{u}$ et
$-\rho\,\tens{K}_{\,d}\,\,\underline{u}$ sont implicit�s est calcul�s �
l'instant $n+\theta$.\\ 
$\bullet$ Le terme de diffusion $\dive (\mu_{\,tot}\,\ggrad \underline{u})$ est
implicit� : la vitesse est prise � l'instant $n+\theta$ et la viscosit�
explicit�e ou extrapol�e.\\ 
$\bullet$ Enfin, le terme de convection en $\dive(\,\underline{u} \otimes
(\rho\underline{u})\,)$ est implicit� : la composante r�solue de la vitesse est
prise en $n+\theta$, et le flux de masse, explicit�, ou extrapol� en
$n+\theta_F$. 

Par souci de clart�, on suppose, en l'absence d'indication, que les propri�tes
physiques $\Phi$ ($\rho,\,\mu_{tot},\,...$) et le flux de masse
$(\rho\underline{u})$ sont pris respectivement aux instants $n+\theta_\Phi$ et
$n+\theta_F$, o� $\theta_\Phi$ et $\theta_F$ d�pendent des sch�mas en temps
sp�cifiquement utilis�s pour ces grandeurs\footnote{cf. \fort{introd}}. 

La discr�tisation temporelle de l'�quation (\ref{Base_Preduv_eqqdm}) s'�crit alors comme suit : 

\begin{equation}\label{Base_Preduv_eq_di1}
 \begin{array}{c}
\displaystyle \rho\,\ \frac{ \underline {\widetilde{u}}^{n+1} -\underline {u}^{n} }
{\Delta t} + \dive(\,\underline{\widetilde{u}}^{n+\theta} \otimes (\rho\underline{u})\,) -\dive
(\mu_{\,tot}\,\ggrad \underline{\widetilde{u}}^{n+\theta}) =
\\
\displaystyle
 - \grad p^{n-1+\theta} + \dive (\rho\,\underline {u})\,\underline{\widetilde{u}}^{n+\theta} +(\Gamma\,\underline{u}_{\,i})^{n+\theta_S}-\Gamma^n\,\,\underline{\widetilde{u}}^{n+\theta}
\\
\begin{array}{c}
\displaystyle
- \rho\,\tens{K}_{\,d}^{n}\,\,\underline{\widetilde{u}}^{n+\theta} - (\rho\,\tens{K}_{\,e}\ \underline{u})^{n+\theta_S} + (\underline{T}_{s}^{\,exp})^{\,n+\theta_S} + T_{s}^{\,imp}\,\,\underline{\widetilde{u}}^{n+\theta}
\\
\displaystyle
+[\dive (\mu_{\,tot}\,^t\ggrad \underline {u})]^{n+\theta_S}-\frac {2} {3}[\,\grad (\mu_{\,tot}\,\dive \underline {u})]^{n+\theta_S} + (\rho -\rho_0) \underline {g}
 - (\underline{turb})^{n+\theta_S}
\end{array}
\end{array}
\end{equation}
o\`u, par souci de simplification, on a pos\'e :
\begin{equation}
\mu_{\,tot}=
\begin{cases}
\mu+\mu_t & \text{pour les mod�les � viscosit� turbulente ou en LES}, \\
\mu & \text{pour les mod�les au second ordre ou en laminaire}
\end{cases} \ 
\end{equation}
\\
et :
\begin{equation}
\underline{turb}^{n}=
\begin{cases}
\displaystyle\frac {2}{3}\grad (\rho^{n}\,k^{n}) & \text{pour les mod�les � viscosit� turbulente}, \\
\dive(\rho^{n}\,\tens{R}^n) & \text{pour les mod�les au second ordre},\\
0 & \text{en laminaire ou en LES}\\
\end{cases}
\end{equation}
Par analogie avec l'�criture du $\theta$-sch�ma pour une variable scalaire, $\,
\underline {\widetilde{u}}^{n+\theta}$ est interpol�e � partir de la vitesse
pr�dite $\underline {\widetilde{u}}^{n+1}$ de la mani\`ere suivante\footnote{si
$\theta=1/2$, ou qu'une extrapolation est utilis�e, l'ordre 2 n'est obtenu que si
le pas de temps $\Delta t$ est uniforme en temps et en espace.}~: 
\begin{equation}
\underline {\widetilde{u}}^{n+\theta}=\theta\, \underline
{\widetilde{u}}^{n+1}+(1-\theta)\, \underline {u}^{n}\\ 
\end{equation}
Avec :
\begin{equation}
\left\{
\begin{array}{ll}
\theta = 1   & \text{Pour un sch\'ema de type Euler implicite d'ordre 1.}\\
\theta = 1/2 & \text{Pour un sch\'ema de type Cranck-Nicolson d'ordre 2.}\\
\end{array}
\right.
\end{equation}

L'�quation (\ref{Base_Preduv_eq_di1}) est alors r��crite sous la forme :

\begin{equation}\label{Base_Preduv_eq_di2}
\begin{array}{c}
\displaystyle \underbrace{\left(\frac{\rho}{\Delta t} -\theta \,\dive (\rho\,\underline {u}) +\theta \,\, \Gamma^n +
\theta \,\, \rho\,\tens{K}_{\,d}^n-\theta \,T_s^{\,imp} \right)}_{\displaystyle f_s^{imp}}\, (\underline {\,\widetilde{u}}^{n+1} -\underline {u}^{n})
\\
 +\, \theta\, \dive(\underline {\widetilde{u}}^{n+1} \otimes (\rho\underline{u}))-\, \theta\,\dive (\mu_{\,tot}\,\ggrad \underline {\widetilde{u}}^{n+1}) =
\\
-\,(1-\theta)\, \dive(\underline {u}^{n} \otimes (\rho\underline{u})) +\,(1-\theta)\,\dive (\mu_{\,tot}\,\ggrad \underline {u}^{n})
\\
f_s^{exp}\left\{
\begin{array}{c}
\displaystyle 
- \grad p^{n-1+\theta} + \dive (\rho\,\underline {u})\,\underline{u}^{n} +\,(\,\Gamma^{n}\,\underline{u}_{\,i}\,)^{n+\theta_S}- \Gamma^n\,\,\underline{u}^{n}
\\
\displaystyle
-(\,\rho\,\tens{K}_{\,e}\ \underline{u}\,)^{n+\theta_S} -\rho\,\tens{K}_{\,d}^n\ \underline{u}^{n}+ (\underline{T}_{s}^{\,exp})^{\,n+\theta_S} + T_s^{\,imp}\,\,\underline {u}^{n} 
\\
\displaystyle
+[\dive (\mu_{\,tot}\,^t\ggrad \underline {u}\,)]^{n+\theta_S}-\frac {2} {3}[\,\grad (\mu_{\,tot}\,\dive \underline {u}\,)]^{n+\theta_S} + (\rho -\rho_0) \underline {g}-(\underline{turb})^{n+\theta_S}
\end{array}
\right.
\end{array}
\end{equation}

d'o� l'�quation r�solue par le sous-programme \fort{codits} :
\begin{equation}\begin{array}{c}
\displaystyle
f_s^{\,imp}(\underline {\widetilde{u}}^{n+1}-\underline {u}^{n}) + \theta\, \dive(\underline{\widetilde{u}}^{n+1} \otimes (\rho
\underline{u})) - \theta\,\dive (\,\mu_{\,tot}\,\ggrad \underline{\widetilde{u}}^{n+1}) = 
\\\\
\displaystyle
-(1-\theta)\,\dive(\underline{u}^{n} \otimes (\rho \underline{u}))+(1-\theta)\,\dive (\,\mu_{\,tot}\,\ggrad \underline{u}^{n})
+ \underline{f}_{\,s}^{\,exp}
\end{array}
\end{equation}
La m\'ethode de discr\'etisation spatiale est d\'evelopp\'ee dans le sous-programme \fort{codits}.\\



\minititre{Remarques :}
{\tiny$\blacksquare$} Dans le cas standard sans extrapolation, le terme
$-\,T_s^{\,imp}$ n'est ajout� � $f_s^{\,imp}$ que s'il est positif afin de ne
pas affaiblir la dominance de la diagonale de la matrice � inverser.\\ 
{\tiny$\blacksquare$} Si une extrapolation est utilis�e, par contre,
$\,T_s^{\,imp}$ est ajout� � $f_s^{\,imp}$ quel que soit son signe. En effet, l'id�e intuitive qui
consiste � prendre~: 
\begin{equation}
\begin{cases}
\displaystyle
(\underline{T}_{s}^{\,exp} + T_{s}^{\,imp}\,\underline {u})^{\,n+\theta_S} &
\text{si } T_{s}^{\,imp} > 0\\ 
\displaystyle
(\underline{T}_{s}^{\,exp})^{\,n+\theta_S} + T_{s}^{\,imp}\,\underline{u}^{n+\theta} &\text{sinon}\\
\end{cases}
\end{equation} 
aboutit � une incoh�rence dans le traitement si $T_s^{imp}$ change de signe
entre deux pas de temps.\\ 
{\tiny$\blacksquare$} la partie diagonale $\tens{K}_{\,d}$ du terme
de perte de charge est utilis�e dans $f_s^{\,imp}$. En fait, pour \^etre rigoureux,
il faudrait ne retenir que les contributions positives (point signal\'e dans le
sous-programme utilisateur associ\'e \fort{uskpdc}). Cette prise en compte sera \`a am\'eliorer.\\
{\tiny$\blacksquare$} Le terme $\theta\,\Gamma^{n}-\theta\,\dive
(\rho\,\underline {u})$ ne pose pas de probl�me pour la 
dominance de la diagonale de la matrice car il est exactement compens� par le
terme de convection (cf. \fort{covofi}). 


%                      Code_Saturne version 1.3
%                      ------------------------
%
%     This file is part of the Code_Saturne Kernel, element of the
%     Code_Saturne CFD tool.
%
%     Copyright (C) 1998-2007 EDF S.A., France
%
%     contact: saturne-support@edf.fr
%
%     The Code_Saturne Kernel is free software; you can redistribute it
%     and/or modify it under the terms of the GNU General Public License
%     as published by the Free Software Foundation; either version 2 of
%     the License, or (at your option) any later version.
%
%     The Code_Saturne Kernel is distributed in the hope that it will be
%     useful, but WITHOUT ANY WARRANTY; without even the implied warranty
%     of MERCHANTABILITY or FITNESS FOR A PARTICULAR PURPOSE.  See the
%     GNU General Public License for more details.
%
%     You should have received a copy of the GNU General Public License
%     along with the Code_Saturne Kernel; if not, write to the
%     Free Software Foundation, Inc.,
%     51 Franklin St, Fifth Floor,
%     Boston, MA  02110-1301  USA
%
%-----------------------------------------------------------------------
%

%%%%%%%%%%%%%%%%%%%%%%%%%%%%%%%%%%
%%%%%%%%%%%%%%%%%%%%%%%%%%%%%%%%%%
\section{Mise en \oe uvre}
%%%%%%%%%%%%%%%%%%%%%%%%%%%%%%%%%%
%%%%%%%%%%%%%%%%%%%%%%%%%%%%%%%%%%
La num\'ero de la phase trait\'ee fait partie des arguments de \fort{turrij}. On
omettra volontairement de le pr\'eciser dans ce qui suit, on indiquera par $(\ )$ la
notion de tableau s'y rattachant.

\etape{Calcul des termes de production $\tens{\mathcal{P}}$}
\begin{itemize}
\item [$\star$] Initialisation \`a z\'ero du tableau \var{PRODUC} dimensionn\'e \`a $\var{NCEL}\times 6$.
\item [$\star$] On appelle trois fois \fort{grdcel} pour calculer les gradients des composantes de la vitesse $u$, $v$ et
$w$ prises au temps $n$.

Au final, on a :\\
$\displaystyle
\begin{array} {ll}
\var{PRODUC(1,IEL)} = & \displaystyle - 2 \left[ R_{11}^{\,n} \frac{\partial u^{\,n}} {\partial x} +R_{12}^{\,n} \frac{\partial u^{\,n}} {\partial y}+R_{13}^{\,n} \frac{\partial u^{\,n}} {\partial z} \right] \text{        (production de $R_{11}^{\,n}$)}\\
\var{PRODUC(2,IEL)} = & \displaystyle - 2 \left[ R_{12}^{\,n} \frac{\partial v^{\,n}} {\partial x} +R_{22}^{\,n} \frac{\partial v^{\,n}} {\partial y}+R_{23}^{\,n} \frac{\partial v^{\,n}} {\partial z} \right] \text{        (production de $R_{22}^{\,n}$)}\\
\var{PRODUC(3,IEL)} = & \displaystyle - 2 \left[ R_{13}^{\,n} \frac{\partial w^{\,n}} {\partial x} +R_{23}^{\,n} \frac{\partial w^{\,n}} {\partial y}+R_{33}^{\,n} \frac{\partial w^{\,n}} {\partial z} \right] \text{        (production de $R_{33}^{\,n}$)}\\
\var{PRODUC(4,IEL)} = & \displaystyle - \left[ R_{12}^{\,n} \frac{\partial u^{\,n}} {\partial x} +R_{22}^{\,n} \frac{\partial u^{\,n}} {\partial y}+R_{23}^{\,n} \frac{\partial u^{\,n}} {\partial z} \right] \\
& \displaystyle - \left[ R_{11}^{\,n} \frac{\partial v^{\,n}} {\partial x} +R_{12}^{\,n} \frac{\partial v^{\,n}} {\partial y}+R_{13}^{\,n} \frac{\partial v^{\,n}} {\partial z} \right] \text{        (production de $R_{12}^{\,n}$)} \\
\var{PRODUC(5,IEL)} = & \displaystyle - \left[ R_{13}^{\,n} \frac{\partial u^{\,n}} {\partial x} +R_{23}^{\,n} \frac{\partial u^{\,n}} {\partial y}+R_{33}^{\,n} \frac{\partial u^{\,n}} {\partial z} \right] \\
& \displaystyle - \left[ R_{11}^{\,n} \frac{\partial w^{\,n}} {\partial x} +R_{12}^{\,n} \frac{\partial w^{\,n}} {\partial y}+R_{13}^{\,n} \frac{\partial w^{\,n}} {\partial z} \right] \text{        (production de $R_{13}^{\,n}$)} \\
\var{PRODUC(6,IEL)} = & \displaystyle - \left[ R_{13}^{\,n} \frac{\partial v^{\,n}} {\partial x} +R_{23}^{\,n} \frac{\partial v^{\,n}} {\partial y}+R_{33}^{\,n} \frac{\partial v^{\,n}} {\partial z} \right] \\
& \displaystyle - \left[ R_{12}^{\,n} \frac{\partial w^{\,n}} {\partial x} +R_{22}^{\,n} \frac{\partial w^{\,n}} {\partial y}+R_{23}^{\,n} \frac{\partial w^{\,n}} {\partial z} \right]  \text{        (production de $R_{23}^{\,n}$)}
\end{array}
$
\end{itemize}

\etape{Calcul du gradient de la masse volumique $\rho^n$ prise au d\'ebut du pas
de temps courant\footnote{{\it i.e.} calcul\'ee \`a partir des
variables du pas de temps pr\'ec\'edent $n$ si n\'ecessaire.} $(n+1)$}
Ce calcul n'a lieu que si les termes de gravit\'e doivent \^etre pris en compte
($\var{IGRARI()} =1$).
\begin{itemize}
\item [$\star$] Appel de \fort{grdcel}  pour calculer le gradient de $\rho^n$
dans les trois directions de l'espace. Les conditions aux limites sur $\rho^n$
sont des conditions de Dirichlet puisque la valeur de $\rho^n$ aux faces de bord
$ik$ (variable \var{IFAC}) est connue et vaut $\rho_{\,b_{\,ik}}$. Pour \'ecrire les conditions aux limites
sous la forme habituelle, $$\rho_{\,b_{\,ik}} = \var{COEFA} + \var{COEFB}
\,\rho^n_{\,I'}$$ on pose alors $\var{COEFA}=
\var{PROPCE(IFAC,IPPROB(IROM(IPHAS)))}$ et $\var{COEFB} = \var{VISCB} = 0$.\\
$\var{PROPCE(1,IPPROB(IROM(IPHAS)))}$ (resp.$\var{VISCB}$) est utilis\'e en lieu
et place de l'habituel \var{COEFA} ($\var{COEFB}$), lors de l'appel \`a \fort{grdcel}.\\
On a donc :\\
$\displaystyle \var{GRAROX}= \frac{\partial \rho^n}{\partial x}\ $,$\displaystyle \ \var{GRAROY}= \frac{\partial
\rho^n}{\partial y}$ et $
\displaystyle \ \var{GRAROZ}= \frac{\partial \rho^n}{\partial z}\ $.

\end{itemize}

Le gradient de $\rho^n$ servira \`a calculer les termes de production par effets de gravit\'e si ces derniers sont pris en compte.

\etape{Boucle \var{ISOU} de $1$ \`a $6$ sur les tensions de Reynolds}
Pour $\var{ISOU} = 1,2,3,4,5,6$, on r\'esout respectivement et dans
l'ordre  les
\'equations de $R_{11}$, $R_{22}$, $R_{33}$, $R_{12}$, $R_{13}$ et $R_{23}$ par
l'appel au sous-programme \fort{resrij}.\\
La r\'esolution se fait par incr\'ement $\delta {R}_{ij}^{\,n+1,k+1}$ , en utilisant la m\^eme m\'ethode que
celle d\'ecrite dans le sous-programme \fort{codits}. On adopte ici les m\^emes notations.
\var{SMBR} est le second membre du syst\`eme \`a inverser, syst\`eme portant sur
les incr\'ements de la variable. \var{ROVSDT} repr\'esente la diagonale de la
matrice, hors convection/diffusion.\\
On va r\'esoudre l'\'equation (\ref{Base_Turrij_Eq_Temp_Rij}) sous forme incr\'ementale en
utilisant \fort{codits}, soit :
\begin{equation}\label{Base_Turrij_Eq_Temp_deltaRij}
\begin{array}{ll}
&\displaystyle \underbrace{\left(\frac {\rho^n_L}{\Delta t^n}
+ \rho^n_L \,C_1\,\frac{\varepsilon^n_L}{k^n_L}(1-\frac{\delta_{ij}}{3})
 - m^n_{\,lm} + \Gamma_L\,+ max(-\alpha^n_{R_{ij}},0)\right)\,|\Omega_l|}
_{\text {$\var{ROVSDT}$ contribuant
\`a la diagonale de la matrice simplifi\'ee de \fort{matrix}}}\,(\delta{R}_{ij}^{\,n+1,p+1})_{\,L}\\\\
&  \underbrace{+\sum\limits_{m\in Vois(l)}\displaystyle \left[
 m^n_{\,lm} \delta R_{ij,\,f_{\,lm}}^{\,n+1,p+1}
- (\mu^n_{\,lm} + \gamma^n_{\,lm})\
\frac{({\delta R}_{ij}^{\,n+1,p+1})_{M}-({\delta R}_{ij}^{\,n+1,p+1})_{L})}{\overline{L'M'}}\,
S_{\,lm} \right]}_{\text { convection upwind pur et diffusion non reconstruite
relatives \`a la matrice simplifi\'ee de \fort{matrix}\footnotemark}} \\
% voir le texte de la footmark plus bas
&= - \displaystyle\frac {\rho^n_L}{\Delta t^n}\,\left(\,(R^{\,n+1,p}_{ij})_L - (R^{\,n}_{ij})_L\,\right)\\
&-\,\underbrace{\displaystyle\int_{\Omega_l} \left(
\dive\,[\,(\rho\,\vect{u})^n\,R^{\,n+1,p}_{ij} - (\mu^n\,+ \gamma^n\,)\,
\grad{R^{\,n+1,p}_{ij}}\,]\right)\,d\Omega}_{\text {convection et diffusion
trait\'ees par \fort{bilsc2}}}\\
&+\displaystyle \int_{\Omega_l} \left[\,\mathcal{P}^{\,n+1,p}_{ij} + \mathcal{G}^{\,n+1,p}_{ij}
- \displaystyle\rho^n \,C_1\,\frac{\varepsilon^n}{k^n}\left[R^{\,n+1,p}_{ij}-
\frac{2}{3}\,k^n\,\delta_{ij}\right] + \phi^{\,n+1,p}_{ij,2} +
\phi^{\,n+1,p}_{ij,w}\,\right]\, d\Omega \\
& + \displaystyle\int_{\Omega_l} \left[- \frac{2}{3} \rho^n \varepsilon^n \delta_{ij}
 + \Gamma\,(\,R^{\,in}_{ij} - R^{\,n+1,p}_{ij}\,) +
\alpha^n_{R_{ij}}\,R^{\,n+1,p}_{ij}+ \beta^n_{R_{ij}}\right]\, d\Omega\\
&+ \sum\limits_{m\in
Vois(l)}\displaystyle \left[\ \tens{E}^n\,\grad{R}^{\,n+1,p}_{ij} \right]_{\,lm}\,.\,\vect{n}_{\,lm}S_{\,lm}\\
&+ \sum\limits_{m\in Vois(l)}\displaystyle \left[\
\tens{D}^n\,\grad{R}^{\,n+1,p}_{ij} \right]_{\,lm}\,.\,\vect{n}_{\,lm}S_{\,lm}\\
&- \sum\limits_{m\in Vois(l)} \gamma^n_{\,lm} \left( \grad{R}^{\,n+1,p}_{ij}\,.\,\vect{n}_{\,lm} \right)  S_{\,lm}\\
&+ \sum\limits_{m\in Vois(l)}  m^n_{\,lm}\,(R^{\,n+1,p}_{ij})_L\\
\end{array}
\end{equation}
% si on ne fait pas comme ca, il n'apparait pas
\footnotetext[\thefootnote]{Si $\var{IRIJNU} = 1$, on remplace  $\mu^n_{\,lm}$ par $(\mu +
\mu_t)^n_{\,lm}$ dans l'expression de la diffusion non reconstruite
associ\'ee \`a la matrice simplifi\'ee de \fort{matrix} ($\mu_t$ d\'esigne la
viscosit\'e turbulente calcul\'ee comme en $k-\varepsilon$).}

o\`u on rappelle :\\
pour $n$ donn\'e entier positif, on d\'efinit la suite
 $({R}_{ij}^{\,n+1,p})_{p \in \grandN}$
 par :
\begin{equation}\notag
\left\{\begin{array}{l}
{R}_{ij}^{\,n+1,0} = {R}_{ij}^{\,n}\\
{R}_{ij}^{\,n+1,p+1} = {R}_{ij}^{\,n+1,p} + \delta{R}_{ij}^{\,n+1,p+1} \\
\end{array}\right.
\end{equation}
$(\delta{R}_{ij}^{\,n+1,p+1})_{\,L}$ d\'esigne la valeur sur l'\'el\'ement
$\Omega_l$ du $\text{$(\,p+1\,)$-i\`eme}$ incr\'ement de ${R}_{ij}^{\,n+1}$,
$ m^n_{\,lm}$ le flux de masse \`a l'instant $n$ \`a travers la face $lm$,
$\delta R_{ij,\,f_{\,lm}}^{\,n+1,p+1}$ vaut $({\delta
R}_{ij}^{\,n+1,p+1})_{L}$  si $ m^n_{\,lm} \geqslant 0$, $({\delta
R}_{ij}^{\,n+1,p+1})_{M}$ sinon,
$\mathcal{P}^{\,n+1,p}_{ij}$, $\phi^{\,n+1,p}_{ij,2}$, $\phi^{\,n+1,p}_{ij,w}$ les valeurs
des quantit\'es associ\'ees correspondant \`a l'incr\'ement
$(\delta{R}_{ij}^{\,n+1,p})$.\\



Tous ces termes sont calcul\'es comme suit :
\begin{itemize}
\item Terme de gauche de l'\'equation (\ref{Base_Turrij_Eq_Temp_deltaRij})\\
Dans \fort{resrij} est calcul\'ee la variable \var{ROVSDT}. Les autres
termes sont compl\'et\'es par \fort{codits}, lors de la construction de la matrice simplifi\'ee , {\it via} un
appel au sous-programme \fort{matrix}. La quantit\'e
 $(\mu^n_{\,lm} + \gamma^n_{\,lm})$ \`a la face $lm$ est calcul\'ee lors de l'appel \`a
\fort{visort}.\\
\item Second membre de l'\'equation (\ref{Base_Turrij_Eq_Temp_deltaRij})\\
Le premier terme non d\'etaill\'e est calcul\'e par le sous-programme
\fort{bilsc2}, qui applique le sch\'ema convectif choisi par l'utilisateur, qui
reconstruit ou non selon le souhait de l'utilisateur les gradients intervenants
dans la convection-diffusion.\\
Les termes sans accolade sont, eux, compl\`etement explicites et ajout\'es au fur et
\`a mesure dans \var{SMBR} pour former
l'expression $f^{\,exp}_s$ de \fort{codits}.
\end{itemize}
On d\'ecrit ci-dessous les \'etapes de \fort{resrij} :
\begin{itemize}

\item DELTIJ = 1, pour $\var{ISOU} \leqslant 3$ et DELTIJ = 0  Si $\var{ISOU} >
3$. Cette valeur repr\'esente le symbole de Kroeneker $\delta_{ij}$.

\item Initialisation \`a z\'ero de \var{SMBR} (tableau contenant le second
membre) et \var{ROVSDT} (tableau contenant la diagonale de la matrice sauf celle
relative \`a la contribution de la
diagonale des op\'erateurs de convection et de diffusion lin\'earis\'es
\footnote{qui correspondent aux sch\'emas convectif upwind pur et diffusif sans
reconstruction.}), tous deux de dimension $\var{NCEL}$.

\item Lecture et prise en compte des termes sources utilisateur pour la variable $R_{ij}$

Appel \`a \fort{ustsri} pour charger les termes sources utilisateurs. Ils sont
stock\'es comme suit. Pour la cellule $\Omega_l$ de centre $L$, repr\'esent\'ee par $\var{IEL}$, on a :\\
\begin{equation}\notag
\left\{\begin{array}{lll}
&\var{ROVSDT(IEL)}&= |\Omega_l| \ \alpha_{R_{ij}}\\
&\var{SMBR(IEL)}&=|\Omega_l| \ \beta_{R_{ij}}\\
\end{array}\right.
\end{equation}
On affecte alors les valeurs ad\'equates au second membre \var{SMBR} et \`a la
diagonale \var{ROVSDT} comme suit :
\begin{equation}\notag
\left\{\begin{array}{lll}
&\var{SMBR(IEL)} &= \var{SMBR(IEL)} +\ |\Omega_l| \ \alpha_{R_{ij}} \ (R^n_{ij})_L \\
&\var{ROVSDT(IEL)}&= \text{max }(-\ |\Omega_l| \ \alpha_{R_{ij}},0)\\
\end{array}\right.
\end{equation}
La valeur de $ \var{ROVSDT}$ est ainsi calcul\'ee pour des raisons de stabilit\'e
num\'erique. En effet, on ne rajoute sur la diagonale que les valeurs positives,
ce qui correspond physiquement \`a impliciter les termes de rappel uniquement.
\item{Calcul du terme source de masse  si $\Gamma_L > 0$}

Appel de \fort{catsma} et incr\'ementation si n\'ecessaire de \var{SMBR} et
\var{ROVSDT} {\it via} :\\
\begin{equation}\notag
\left\{\begin{array}{lll}
\displaystyle \var{SMBR(IEL)} = \var{SMBR(IEL)} + |\Omega_l| \ \Gamma_L \
\left[(R^{\,in}_{ij})_L - (R^{\,n}_{ij})_L \right] \\
\displaystyle \var{ROVSDT(IEL)}=\var{ROVSDT(IEL)} + |\Omega_l| \ \Gamma_L
\end{array}\right.
\end{equation}
\item Calcul du terme d'accumulation de masse et du terme instationnaire

On stocke $\displaystyle \var{W1}= \int_{\Omega_l}\dive\,(\rho\,\vect{u})\,d\Omega$
calcul\'e par \fort{divmas} \`a l'aide des flux de masse aux faces internes
$ m^n_{\,lm}=\sum\limits_{m\in Vois(l)}{(\rho \vect{u})_{\,lm}^n} \text{.}\,
\vect{S}_{\,lm} $ (tableau \var{FLUMAS}) et des flux de masse aux bords  $ m^n_{\,b_{lk}} = \sum\limits_{k\in{\gamma_b(l)}}{(\rho \vect{u})_{\,{b}_{lk}}^n} \text{.}\,
\vect{S}_{\,{b}_{lk}} $ (tableau \var{FLUMAB}).
On incr\'emente ensuite \var{SMBR} et \var{ROVSDT}.
\begin{equation}\notag
\left\{\begin{array}{lll}
&\var{SMBR(IEL)} &= \var{SMBR(IEL)} + \var{ICONV}\  (R^n_{ij})_L\,(\displaystyle
\int_{\Omega_l}\dive\,(\rho\,\vect{u})\ d\Omega) \\
&\var{ROVSDT(IEL)}& = \var{ROVSDT(IEL)} +  \var{ISTAT}\,\displaystyle
\frac{\rho^n_L \ |\Omega_l|}{\Delta t^n} -  \var{ICONV}\ (\displaystyle
\int_{\Omega_l}\dive\,(\rho\,\vect{u})\ d\Omega) \\
\end{array}\right.
\end{equation}
\item Calcul des termes sources de production, des termes $\displaystyle
\phi_{\,ij,1}+\phi_{\,ij,2}$ et de la dissipation~$\displaystyle-\frac{2}{3} \varepsilon\,\delta_{\,ij}$ :

On effectue une boucle d'indice \var{IEL} sur les cellules $\Omega_l$ de centre $L$ :
\begin{itemize}
\item [$\Rightarrow$] $\displaystyle \var{TRPROD}= \frac{1}{2} (\mathcal{P}^n_{ii})_L = \frac{1}{2} \left[ \var{PRODUC(1,IEL)} +  \var{PRODUC(2,IEL)} +  \var{PRODUC(3,IEL)} \right] $
\item [$\Rightarrow$] $\displaystyle \var{TRRIJ }= \frac{1}{2} (R^n_{ii})_L $
\item [$\Rightarrow$] $\displaystyle \var{SMBR(IEL)} =\ \var{SMBR(IEL)}\ +$\\
$\ \displaystyle\rho^n_L |\Omega_l| \left[ \displaystyle
\frac{2}{3}\,\delta_{\,ij} \left( \ \displaystyle \frac{ C_2}{2}\,(\mathcal{P}^n_{ii})_L\ +
(C_1-1)\ \varepsilon^n_L\, \right)\right.$\\
$ + \left.\ (1-C_2) \ \var{PRODUC(ISOU,IEL)} -
\displaystyle C_1\ \frac{2\,\varepsilon^n_L}{(R^n_{ii})_L}\ (R^n_{ij})_L \right]$
\item [$\Rightarrow$] $\displaystyle \var{ROVSDT(IEL)} = \var{ROVSDT(IEL)} +
\rho^n_L \ |\Omega_l| \ (- \displaystyle \frac{1}{3} \ \,\delta_{\,ij} + 1) \ C_1
\ \frac{2\ \varepsilon^n_L}{(R^n_{ii})_L}$
\end{itemize}
\item Appel de \fort{rijech} pour le calcul des termes d'\'echo de paroi
 $\phi^n_{ij,w}$ si $\var{IRIJEC()}=1$ et ajout dans \var{SMBR}.\\
$\var{SMBR} = \var{SMBR} + \phi^n_{ij,w}$\\
Suivant son mode de calcul (\var{ICDPAR}), la distance � la paroi est directement accessible
par \var{RA(IDIPAR+IEL-1)} (\var{|ICDPAR|} = 1) ou bien
est calcul\'ee \`a partir de $\var{IA(IIFAPA(IPHAS)+IEL - 1)}$,
qui donne pour l'\'el\'ement $\var{IEL}$ le num\'ero de la face de bord
paroi la plus  proche (\var{|ICDPAR|} = 2). Ces tableaux ont \'et\'e renseign\'e une fois pour toutes au
d\'ebut de calcul.

\item  Appel de \fort{rijthe} pour le calcul des termes de gravit\'e $\mathcal{G}^n_{ij}$ et ajout dans \var{SMBR}.

Ce calcul n'a lieu que si $\var{IGRARI()} = 1$.
$ \var{SMBR} = \var{SMBR} + \mathcal{G}^n_{ij}$
\item Calcul de la partie explicite du terme de diffusion
 $\dive{\,\left[\tens{A}\,\grad{R}^{\,n}_{ij}\right]}$, plus pr\'ecis\'ement
des contributions du terme extradiagonal pris aux faces purement internes
(remplissage du tableau \var{VISCF}), puis aux faces de bord (remplissage du
tableau \var{VISCB}).
\begin{itemize}
\item [$\star$] Appel de \fort{grdcel} pour le calcul du gradient de
$R^{\,n}_{ij}$ dans chaque direction. Ces gradients sont respectivement
stock\'es dans les tableaux de travail \var{W1}, \var{W2} et \var{W3}.

\item [$\star$] boucle d'indice \var{IEL} sur les cellules $\Omega_l$ de centre
$L$ pour le
calcul de $\tens{E}^n\,\grad{R}^{\,n}_{ij}$ aux cellules dans un premier temps :\\
\begin{itemize}
\item [$\Rightarrow$] $\displaystyle \var{TRRIJ}= \frac{1}{2} (R^{\,n}_{ii})_L $
\item [$\Rightarrow$] $\displaystyle \var{CSTRIJ} = \rho^n_L\ C_S \ \displaystyle\frac{(R^n_{ii})_L}{2\,\varepsilon^n_L}$
\item [$\Rightarrow$] $\displaystyle \var{W4(IEL)} = \rho^n_L\ C_S\
\displaystyle\frac{(R^n_{ii})_L}{2\,\varepsilon^n_L} \left[\,(R^{\,n}_{12})_L \ \var{W2(IEL)} +
(R^{\,n}_{13})_L \ \var{W3(IEL)}\,\right]$
\item [$\Rightarrow$] $\displaystyle \var{W5(IEL)} = \rho^n_L\ C_S\
\displaystyle\frac{(R^n_{ii})_L}{2\,\varepsilon^n_L} \left[\,(R^{\,n}_{12})_L \ \var{W1(IEL)} +
(R^{\,n}_{23})_L \ \var{W3(IEL)}\,\right]$
\item [$\Rightarrow$] $\displaystyle \var{W6(IEL)} = \rho^n_L\ C_S\
\displaystyle\frac{(R^n_{ii})_L}{2\,\varepsilon^n_L} \left[\,(R^{\,n}_{13})_L \ \var{W1(IEL)} + (R^{\,n}_{23})_L \ \var{W2(IEL)}\,\right]$
\end{itemize}



\item [$\star$] Appel de \fort{vectds}\footnote{Le r\'esultat est stock\'e dans
\var{VISCF} et \var{VISCB}. Dans \fort{vectds}, les valeurs aux faces internes
sont interpol\'ees lin\'eairement sans reconstruction et \var{VISCB} est mis \`a
z\'ero.} pour assembler $\displaystyle\left[ \tens{E}^n\,\grad{R}^{\,n}_{ij}
\right]\,.\,\vect{n}_{\,lm}S_{\,lm}$ aux faces $lm$.
\item [$\star$] Appel de \fort{divmas} pour calculer la divergence du flux d\'efini par \var{VISCF} et \var{VISCB}.
Le r\'esultat est stock\'e dans \var{W4}.\\
Ajout au second membre \var{SMBR}.\\
\var{SMBR} = \var{SMBR} + \var{W4}
\end{itemize}

A l'issue de cette \'etape, seule la partie extradiagonale de la diffusion prise
enti\`erement explicite~:
 $$\sum\limits_{m\in
Vois(l)}\left[\ \tens{E}^n\,\grad{R}^{\,n}_{ij} \right]_{\,lm}\,.\,\vect{n}_{\,lm}S_{\,lm}$$ a \'et\'e calcul\'ee.\\

\item Calcul de la partie diagonale du terme de diffusion\footnote{Seule la
composante normale  du  gradient de $R_{ij}$ aux faces sera implicite.} :\\
Comme on l'a d\'eja vu, une partie de cette contribution sera trait\'ee en
implicite (celle relative \`a la composante normale du gradient) et donc
ajout\'ee au second membre par \fort{bilsc2} ; l'autre
partie sera explicite et prise en compte dans $f_s^{\,exp}$.
\begin{itemize}
\item [$\star$] On effectue une boucle d'indice \var{IEL} sur les cellules
$\Omega_l$ de centre $L$ :
\begin{itemize}
\item [$\Rightarrow$] $\displaystyle \var{TRRIJ }= \frac{1}{2} (R^{\,n}_{ii})_L $
\item [$\Rightarrow$] $\displaystyle \var{CSTRIJ} = \rho^n_L \ C_S \ \frac{(R^{\,n}_{ii})_L}{2\,\varepsilon^n_L}$
\item [$\Rightarrow$] $\displaystyle \var{W4(IEL)} = \rho^n_L \ C_S \
\frac{(R^{\,n}_{ii})_L}{2\,\varepsilon^n_L} \ (R^{\,n}_{11})_L$
\item [$\Rightarrow$] $\displaystyle \var{W5(IEL)} = \rho^n_L \ C_S \ \frac{(R^{\,n}_{ii})_L}{2\,\varepsilon^n_L}\ (R^n_{22})_L$
\item [$\Rightarrow$] $\displaystyle \var{W6(IEL)} = \rho^n_L \ C_S \ \frac{(R^{\,n}_{ii})_L}{2\,\varepsilon^n_L} \ (R^n_{33})_L$
\end{itemize}

%\item Traitement du parall\'elisme et de la p\'eriodicit\'e.

\item [$\star$] On effectue une boucle d'indice \var{IFAC} sur les faces
purement internes $lm$ pour remplir le tableau \var{VISCF} :
\begin{itemize}
\item [$\Rightarrow$] Identification des cellules $\Omega_l$ et $\Omega_m$ de
centre respectif $L$ (variable \var{II}) et $M$ (variable \var{JJ}), se trouvant de chaque c\^ot\'e de la face
$lm$\footnote{La normale \`a la face est orient\'ee de L vers M.}.
\item [$\Rightarrow$] Calcul du carr\'e de la surface de la face. La valeur est
stock\'ee dans le tableau \var{SURFN2}.
\item [$\Rightarrow$] Interpolation du gradient de $R^{\,n}_{ij}$ \`a la face
$lm$ (gradient facette $\left[\grad{R}^{\,n}_{ij}\right]_{\,lm}$) :
\begin{equation}\notag
\left\{\begin{array}{ll}
\var{GRDPX} &= \displaystyle \frac{1}{2} \left(\var{W1(II)} + \var{W1(JJ)}
\right) \\
&\\
\var{GRDPY} &= \displaystyle \frac{1}{2} \left(\var{W2(II)} + \var{W2(JJ)} \right) \\
&\\
\var{GRDPZ} &= \displaystyle \frac{1}{2} \left(\var{W3(II)} + \var{W3(JJ)} \right)
\end{array}\right.
\end{equation}
\item [$\Rightarrow$] Calcul du gradient de $R^{\,n}_{ij}$ normal \`a la face
$lm$, $\left[\grad{R}^{\,n}_{ij}\right]_{\,lm}.\vect{n}_{\,lm}\,S_{\,lm}$.\\

$\displaystyle \var{GRDSN} =  \var{GRDPX} \ \var{SURFAC(1,IFAC)} + \var{GRDPY} \ \var{SURFAC(2,IFAC)} +  \var{GRDPZ} \ \var{SURFAC(3,IFAC)}$
$\var{SURFAC}$ est le vecteur surface de la face \var{IFAC}.


\item [$\Rightarrow$] calcul de
 $\left[\grad{R^{\,n}_{ij}} - (\grad
R^{\,n}_{ij}\,.\,\vect{n}_{\,lm})\vect{n}_{\,lm}\right]$, les vecteurs \'etant
calcul\'es \`a la face $lm$ :
\begin{equation}\notag
\left\{\begin{array}{lll}
&\displaystyle \var{GRDPX} &= \var{GRDPX} - \displaystyle\frac{\var{GRDSN}}{\var{SURFN2}} \ \var{SURFAC(1,IFAC)}\\
&&\\
&\displaystyle \var{GRDPY} &= \var{GRDPY} - \displaystyle\frac{\var{GRDSN}}{\var{SURFN2}} \ \var{SURFAC(2,IFAC)} \\
&&\\
&\displaystyle \var{GRDPZ} &= \var{GRDPZ} - \displaystyle\frac{\var{GRDSN}}{\var{SURFN2}} \ \var{SURFAC(3,IFAC)}
\end{array}\right.
\end{equation}
\item [$\Rightarrow$] finalisation du calcul de l'expression totalement
explicite
 $$\left[ \tens{D}^n\,\left( \grad{R^{\,n}_{ij}} - (\grad R^{\,n}_{ij}\,.\,\vect{n}_{\,lm})\,\vect{n}_{\,lm}\right) \right]\,.\,\vect{n}_{\,lm}$$
\begin{equation}\notag
\begin{array} {ll}
\displaystyle \var{VISCF} = &
 \displaystyle\frac{1}{2} (\ \var{W4(II)} +\ \var{W4(JJ)}) \ \var{GRDPX} \
\var{SURFAC(1,IFAC)})\ + \\
&\\
&  \displaystyle\frac{1}{2} (\ \var{W5(II)} +\ \var{W5(JJ)}) \ \var{GRDPY} \
\var{SURFAC(2,IFAC)})\ + \\
&\\
&  \displaystyle\frac{1}{2} (\ \var{W6(II)} +\ \var{W6(JJ)}) \ \var{GRDPZ} \ \var{SURFAC(3,IFAC)})
\end{array}
\end{equation}
\end{itemize}

\item [$\star$] Mise \`a z\'ero du tableau \var{VISCB}.

\item [$\star$] Appel de \fort{divmas} pour calculer la divergence de~:
 $$\tens{D}^{\,n}\,\left( \grad{R^{\,n}_{ij}} - (\grad R^{\,n}_{ij}\,.\,\vect{n}_{\,lm})\vect{n}_{\,lm}\right)$$ d\'efini aux faces dans \var{VISCF} et \var{VISCB}.

Le r\'esultat est stock\'e dans le tableau \var{W1}.\\
Ajout au second membre \var{SMBR}.\\
$\var{SMBR} = \var{SMBR} + \var{W1}$
\end{itemize}
\item Calcul de la viscosit\'e orthotrope $\gamma^n_{\,lm}$ \`a la face $lm$ de la variable principale
$R^{\,n}_{ij}$\\
Ce calcul permet au sous-programme \fort{codits} de compl\'eter le second membre
\var{SMBR} par :
\begin{equation}
\begin{array} {ll}
& \sum\limits_{m\in Vois(l)}
\mu^n_{\,lm}\,\left(\grad{R}^{\,n}_{ij}\,.\,\vect{n}_{\,lm}\right)S_{\,lm}
 + \sum\limits_{m\in Vois(l)} \left(\grad{R}^{\,n}_{ij}
\,.\,\vect{n}_{\,lm}\right)\left[\tens{D}^{\,n}\,\vect{n}_{\,lm}\right]_{\,lm}\,.\,\vect{n}_{\,lm}\
S_{\,lm}\\
& = \sum\limits_{m\in Vois(l)}(\,\mu^n_{\,lm}\, + \,\gamma^n_{\,lm}\,)\,\left(\grad{R}^{\,n}_{ij}\,.\,\vect{n}_{\,lm}\right)S_{\,lm}
\end{array}
\end{equation}
sans pr\'eciser la nature de la face $lm$, {\it via} l'appel \`a \fort{bilsc2}  et de disposer de la quantit\'e
$(\mu^n_{\,lm}\, + \gamma^n_{\,lm})$ pour construire sa
matrice simplifi\'ee.\\
\begin{itemize}
\item [$\star$] On effectue une boucle d'indice \var{IEL} sur les cellules
$\Omega_l$ :
\begin{itemize}
\item [$\Rightarrow$] $\displaystyle \var{TRRIJ }= \frac{1}{2} (R^{\,n}_{ii})_L $
\item [$\Rightarrow$] $\displaystyle \var{RCSTE} = \rho^n_L \ C_S \ \frac{ (R^{\,n}_{ii})_L}{2\,\varepsilon^n_L} $
\item [$\Rightarrow$] $\displaystyle \var{W1(IEL)} = \mu^n +\rho^n_L \ C_S \ \frac{
(R^{\,n}_{ii})_L}{2\,\varepsilon^n_L}\ (R^n_{11})_L$
\item [$\Rightarrow$] $\displaystyle \var{W2(IEL)} = \mu^n + \rho^n_L \ C_S \ \frac{ (R^{\,n}_{ii})_L}{2\,\varepsilon^n_L}\ (R^n_{22})_L$
\item [$\Rightarrow$] $\displaystyle \var{W3(IEL)} = \mu^n + \rho^n_L \ C_S \ \frac{ (R^{\,n}_{ii})_L}{2\,\varepsilon^n_L}\ (R^n_{33})_L$
\end{itemize}

\item [$\star$] Appel de \fort{visort} pour calculer la viscosit\'e orthotrope
\footnote{Comme dans le sous-programme \fort{viscfa}, on multiplie la viscosit\'e par
$\displaystyle \frac{S_{\,lm}}{\overline{L'M'}}$, o\`u $S_{\,lm}$ et
$\overline{L'M'}$ repr\'esentent respectivement la surface de la face $lm$ et la
mesure alg\'ebrique du segment reliant les projections des centres des cellules
voisines sur la normale \`a la face. On garde dans ce sous-programme  la possibilit\'e d'interpoler la viscosit\'e aux cellules lin\'eairement ou d'utiliser une moyenne harmonique. La viscosit\'e au bord est celle de la cellule de bord correspondante.}
$ \gamma^n_{\,lm} = (\tens{D}^{\,n}\,\vect{n}_{\,lm}).\vect{n}_{\,lm}$ aux faces $lm$

Le r\'esultat est stock\'e dans les tableaux \var{VISCF} et \var{VISCB}.
\end{itemize}

\item appel de \fort{codits} pour la r\'esolution de l'\'equation de
convection/diffusion/termes sources de la variable $R_{ij}$. Le terme source est
\var{SMBR}, la viscosit\'e \var{VISCF} aux faces purement internes (
resp. \var{VISCB} aux faces de bord) et \var{FLUMAS} le flux de masse interne
 ( resp. \var{FLUMAB} flux de masse au bord). Le r\'esultat est la variable $R_{ij}$ au temps
$n+1$, donc $R^{\,n+1}_{ij}$. Elle est stock\'ee dans le tableau \var{RTP} des
variables mises \`a jour.

\end{itemize}

\etape{Appel de \fort{reseps} pour la r\'esolution de la variable $\varepsilon$}

On d\'ecrit ci-dessous le sous-programme \fort{reseps}, les commentaires du sous-programme \fort{resrij} ne seront pas r\'ep\'et\'es puisque les deux sous-programmes ne diff\`erent que par leurs termes sources.

\begin{itemize}
\item Initialisation \`a z\'ero de \var{SMBR} et \var{ROVSDT}.

\item{Lecture et prise en compte des termes sources utilisateur pour la variable $\varepsilon$ :}

Appel de \fort{ustsri} pour charger les termes sources utilisateurs. Ils sont
stock\'es dans les tableaux suivants :\\
pour la cellule $\Omega_l$ repr\'esent\'ee par $\var{IEL}$ de centre $L$, on a :
\begin{equation}\notag
\left\{\begin{array}{lll}
&\var{ROVSDT(IEL)}&= |\Omega_l| \ \alpha_{\varepsilon}\\
&\var{SMBR(IEL)}&=|\Omega_l| \ \beta_{\varepsilon}\\
\end{array}\right.
\end{equation}
On affecte alors les valeurs ad\'equates au second membre \var{SMBR} et \`a la
diagonale \var{ROVSDT} comme suit :
\begin{equation}\notag
\left\{\begin{array}{lll}
&\var{SMBR(IEL)} &= \var{SMBR(IEL)} +\ |\Omega_l| \ \alpha_{\,\varepsilon} \
\varepsilon^n_L \\
&\var{ROVSDT(IEL)}&= \text{max }(-\ |\Omega_l| \ \alpha_{\,\varepsilon},0)\\
\end{array}\right.
\end{equation}

\item{Calcul du terme source de masse si $\Gamma_L > 0$ :
\begin{equation}\notag
\left\{\begin{array}{lll}
&\displaystyle \var{SMBR(IEL)} = \var{SMBR(IEL)} + |\Omega_l| \ \Gamma_L \
(\varepsilon^{\,in}_L -\varepsilon^n_L) \\
&\displaystyle \var{ROVSDT(IEL)}= \var{ROVSDT(IEL)} + |\Omega_l| \ \Gamma_L
\end{array}\right.
\end{equation}
\item Calcul du terme d'accumulation de masse et du terme instationnaire \\
On stocke $\displaystyle \var{W1}= \int_{\Omega_l}\dive\,(\rho\,\vect{u})\,d\Omega$
calcul\'e par \fort{divmas} \`a l'aide des flux de masse internes et aux bords.\\
On incr\'emente ensuite \var{SMBR} et \var{ROVSDT}.
\begin{equation}\notag
\left\{\begin{array}{lll}
&\var{SMBR(IEL)} &= \var{SMBR(IEL)} + \var{ICONV}\ \varepsilon^n_L\,(\displaystyle
\int_{\Omega_l}\dive\,(\rho\,\vect{u})\ d\Omega) \\
&\var{ROVSDT(IEL)}& = \var{ROVSDT(IEL)} +  \var{ISTAT}\,\displaystyle
\frac{\rho^n_L \ |\Omega_l|}{\Delta t^n} -  \var{ICONV}\ (\displaystyle
\int_{\Omega_l}\dive\,(\rho\,\vect{u})\ d\Omega) \\
\end{array}\right.
\end{equation}

\item Traitement du terme de production
 $\displaystyle \rho\,C_{\varepsilon_1}\,\frac{\varepsilon}{k}\,\mathcal{P}$
 et du terme de dissipation $-\,\displaystyle \rho\,C_{\varepsilon_2}\,\frac{\varepsilon}{k}\,\varepsilon$ \\
pour cela, on effectue une boucle d'indice \var{IEL} sur les cellules $\Omega_l$
de centre $L$ :
\begin{itemize}
\item [$\Rightarrow$] $\displaystyle \var{TRPROD}= \frac{1}{2} (\mathcal{P}^n_{ii})_L = \frac{1}{2} \left[ \var{PRODUC(1,IEL)} +  \var{PRODUC(2,IEL)} +  \var{PRODUC(3,IEL)} \right] $
\item [$\Rightarrow$] $\displaystyle \var{TRRIJ }= \frac{1}{2} (R^n_{ii})_L $
\item [$\Rightarrow$] $\displaystyle \var{SMBR(IEL)} = \var{SMBR(IEL)} + \rho^n_L
|\Omega_l| \left[ -C_{\varepsilon_2} \ \frac{2\,(\varepsilon^n_L)^2}{(R^n_{ii})_L} + C_{\varepsilon_1} \ \frac{\varepsilon^n_L}{(R^n_{ii})_L}\ (\mathcal{P}^n_{ii})_L \right] $
\item [$\Rightarrow$] $\displaystyle \var{ROVSDT(IEL)} = \var{ROVSDT(IEL)} + C_{\varepsilon_2} \ \rho^n_L \ |\Omega_l| \ \frac{2\,\varepsilon^n_L}{(R^n_{ii})_L}$
\end{itemize}

\item Appel de \fort{rijthe} pour le calcul des termes de gravit\'e $\mathcal{G}^n_{\varepsilon}$ et ajout dans \var{SMBR}.

$ \var{SMBR} = \var{SMBR} + \mathcal{G}^n_{\varepsilon}$\\
Ce calcul n'a lieu que si $\var{IGRARI()} = 1$.

\item Calcul de la diffusion de $\varepsilon$ \\
 Le terme $\dive \left[\mu\, \grad(\varepsilon) + \tens{A'}\,\grad(\varepsilon)
\right]$ est calcul\'e exactement de la m\^eme mani\`ere que pour les tensions
de Reynolds $R_{ij}$ en rempla\c cant $\tens{A}$ par $\tens{A'}$.

\item Appel de \fort{codits} pour la r\'esolution de l'\'equation de
convection/diffusion/termes sources de la variable principale $\varepsilon$. Le
r\'esultat $\varepsilon^{\,n+1}$ est stock\'e dans le tableau \var{RTP} des
variables mises \`a jour.
}
\end{itemize}

\etape{clippings finaux}
On passe enfin dans le sous-programme  \fort{clprij} pour faire un clipping \'eventuel
des variables $R^{\,n+1}_{ij}$ et $\varepsilon^{\,n+1}$. Le sous-programme
\fort{clprij} est appel\'e\footnote{L'option
$\var{ICLIP} = 1$ consiste en un clipping minimal des variables $R_{ii}$ et
$\varepsilon$ en prenant la valeur absolue de ces variables puisqu'elles ne
peuvent \^etre que positives.} avec $\var{ICLIP} = 2$ . Cette option
\footnote{Quand la valeur des grandeurs $R_{ii}$ ou $\varepsilon$ est
n\'egative, on la remplace par le minimum entre sa valeur absolue et (1,1)
fois la valeur obtenue au pas de temps pr\'ec\'edent.} contient l'option $\var{ICLIP} = 1$  et permet de v\'erifier l'in\'egalit\'e de Cauchy-Schwarz sur les grandeurs extra-diagonales du tenseur $\tens{R}$ (pour $i \neq j$, $|R_{ij}|^2 \le R_{ii} R_{jj}$).


%%%%%%%%%%%%%%%%%%%%%%%%%%%%%%%%%%
%%%%%%%%%%%%%%%%%%%%%%%%%%%%%%%%%%
\section{Points \`a traiter}
%%%%%%%%%%%%%%%%%%%%%%%%%%%%%%%%%%
%%%%%%%%%%%%%%%%%%%%%%%%%%%%%%%%%%
Sauf mention explicite, $\phi$ repr\'esentera une tension de Reynolds ou la dissipation turbulente ($\phi = R_{ij} \ \text{ou} \ \varepsilon$).

\begin{itemize}
\item {La vitesse utilis\'ee pour le calcul de la production est explicite. Est-ce qu'une implicitation peut am\'eliorer la pr\'ecision temporelle de $\phi$ \footnote{Cette remarque peut \^etre g\'en\'eralis\'ee. En effet, peut-on envisager d'actualiser les variables d\'ej\`a r\'esolues (sans r\'eactualiser les variables turbulentes apr\`es leur r\'esolution)? Ceci obligerait \`a modifier les sous-programmes tels que \fort{condli} qui sont appel\'es au d\'ebut de la boucle en temps.} ?}
\item {Dans quelle mesure le terme d'\'echo de paroi est-il valide ? En effet, ce terme est remis en question par certains auteurs.}
\item {On peut envisager la r\'esolution d'un syst\`eme hyperbolique pour les
tensions de Reynolds afin d'introduire un couplage avec le champ de vitesse.}
\item {Le flux au bord \var{VISCB} est annul\'e dans le sous-programme
\fort{vectds}. Peut-on envisager de mettre au bord la valeur de la variable
concern\'ee \`a la cellule de bord correspondant? De m\^eme, il faudrait se
pencher sur les hypoth\`eses sous-jacentes \`a l'annulation des contributions
aux bords de \var{VISCB} lors du calcul de : $$\left[ \tens{D}^n\,\left( \grad{R^{\,n}_{ij}} - (\grad R^{\,n}_{ij}\,.\,\vect{n}_{\,lm})\,\vect{n}_{\,lm}\right) \right]\,.\,\vect{n}_{\,lm}.$$}
\item {Un probl\`eme de pond\'eration appara\^\i t plus g\'en\'eralement. Si on prend la partie explicite de $\tens{D}\,\grad(\phi)$, la pond\'eration aux faces internes utilise le coefficient $\displaystyle\frac{1}{2}$ avec pond\'eration s\'epar\'ee de $\tens{D}$ et $\grad(\phi)$, alors que pour $\tens{E}\,\grad(\phi)$, on calcule d'abord ce terme aux cellules pour ensuite l'interpoler lin\'eairement aux faces \footnote{Cette interpolation se fait dans \fort{vectds} avec des coefficients de pond\'eration aux faces.}. Ceci donne donc deux types d'interpolations pour des termes de m\^eme nature.}
\item {On laisse la possibilit\'e dans \fort{visort} d'utiliser une moyenne
harmonique aux faces. Est-ce que ceci est valable puisque les interpolations
utilis\'ees lors du calcul de la partie explicite de $\tens{A}\,\grad{\phi}$
sont des moyennes arithm\'etiques ?}
\item {Les techniques adopt\'ees lors du clipping sont \`a revoir.}
\item {On utilise dans le cadre du mod\`ele $\displaystyle R_{ij}-\varepsilon $ une semi-implicitation de termes comme $\displaystyle \phi_{ij,1}$ ou $\displaystyle -\rho\,C_{\varepsilon_2}\,\frac{\varepsilon}{k}\,\varepsilon$. On peut envisager le m\^eme type d'implicitation dans \fort{turbke} m\^eme en pr\'esence du couplage $\displaystyle k-\varepsilon$.}
\item L'adoption d'une r\'esolution d\'ecoupl\'ee fait perdre l'invariance par rotation.
\item La formulation et l'implantation des conditions aux limites de paroi
devront \^etre v\'erifi\'ees. En effet, il semblerait que, dans certains cas, des ph\'enom\`enes
``oscillatoires'' apparaissent, sans qu'il soit ais\'e d'en d\'eterminer la cause.
\item L'implicitation partielle (du fait de la r\'esolution d\'ecoupl\'ee) des
conditions aux limites conduit souvent \`a des calculs instables. Il
conviendrait d'en conna\^\i tre la raison. L'implicitation partielle avait
\'et\'e mise en \oe uvre afin de tenter d'utiliser un pas de temps plus grand
dans le cas de jets axisym\'etriques en particulier.

\end{itemize}

\part{Module combustion}
%                      Code_Saturne version 1.3
%                      ------------------------
%
%     This file is part of the Code_Saturne Kernel, element of the
%     Code_Saturne CFD tool.
%
%     Copyright (C) 1998-2007 EDF S.A., France
%
%     contact: saturne-support@edf.fr
%
%     The Code_Saturne Kernel is free software; you can redistribute it
%     and/or modify it under the terms of the GNU General Public License
%     as published by the Free Software Foundation; either version 2 of
%     the License, or (at your option) any later version.
%
%     The Code_Saturne Kernel is distributed in the hope that it will be
%     useful, but WITHOUT ANY WARRANTY; without even the implied warranty
%     of MERCHANTABILITY or FITNESS FOR A PARTICULAR PURPOSE.  See the
%     GNU General Public License for more details.
%
%     You should have received a copy of the GNU General Public License
%     along with the Code_Saturne Kernel; if not, write to the
%     Free Software Foundation, Inc.,
%     51 Franklin St, Fifth Floor,
%     Boston, MA  02110-1301  USA
%
%-----------------------------------------------------------------------
%
\programme{vortex}
%
\vspace{1cm}
%%%%%%%%%%%%%%%%%%%%%%%%%%%%%%%%%%
%%%%%%%%%%%%%%%%%%%%%%%%%%%%%%%%%%
\section{Fonction}
%%%%%%%%%%%%%%%%%%%%%%%%%%%%%%%%%%
%%%%%%%%%%%%%%%%%%%%%%%%%%%%%%%%%%
Ce sous-programme est d�di� � la g�n�ration des conditions d'entr�e
turbulente utilis�es en LES.


La m�thode des vortex est bas�e sur une approche de tourbillons
ponctuels. L'id�e de la m�thode consiste � injecter des tourbillons 2D dans le
plan d'entr�e du calcul, puis � calculer le champ de vitesse induit par ces
tourbillons au centre des faces d'entr�e.

%
%%%%%%%%%%%%%%%%%%%%%%%%%%%%%%%%%%%%%%%%%%%%%%%%%%%%%%%%%%%%%%%%%%%%%%
% FIN DU DOCUMENT
\end{document}
%
%%%%%%%%%%%%%%%%%%%%%%%%%%%%%%%%%%%%%%%%%%%%%%%%%%%%%%%%%%%%%%%%%%%%%%

%%% Local Variables:
%%% mode: latex
%%% TeX-master: None
%%% End:

