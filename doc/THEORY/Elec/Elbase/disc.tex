%                      Code_Saturne version 1.3
%                      ------------------------
%
%     This file is part of the Code_Saturne Kernel, element of the
%     Code_Saturne CFD tool.
%
%     Copyright (C) 1998-2007 EDF S.A., France
%
%     contact: saturne-support@edf.fr
%
%     The Code_Saturne Kernel is free software; you can redistribute it
%     and/or modify it under the terms of the GNU General Public License
%     as published by the Free Software Foundation; either version 2 of
%     the License, or (at your option) any later version.
%
%     The Code_Saturne Kernel is distributed in the hope that it will be
%     useful, but WITHOUT ANY WARRANTY; without even the implied warranty
%     of MERCHANTABILITY or FITNESS FOR A PARTICULAR PURPOSE.  See the
%     GNU General Public License for more details.
%
%     You should have received a copy of the GNU General Public License
%     along with the Code_Saturne Kernel; if not, write to the
%     Free Software Foundation, Inc.,
%     51 Franklin St, Fifth Floor,
%     Boston, MA  02110-1301  USA
%
%-----------------------------------------------------------------------
%
%%%%%%%%%%%%%%%%%%%%%%%%%%%%%%%%%%
%%%%%%%%%%%%%%%%%%%%%%%%%%%%%%%%%%
\section{Discr\'etisation}
%%%%%%%%%%%%%%%%%%%%%%%%%%%%%%%%%%
%%%%%%%%%%%%%%%%%%%%%%%%%%%%%%%%%%

La discr\'etisation des \'equations ne pose pas de probl\`eme particulier
(ajout de termes sources explicites pour l'effet Joule et les forces de Laplace,
\'equations de Poisson pour la d\'etermination des potentiels).

Un point sur les conditions aux limites doit cependant \^etre fait ici, en
particulier pour pr\'eciser la m\'ethode de recalage automatique des
potentiels.



\subsection{Arcs \'electriques}

\subsubsection{Conditions aux limites}

Seules les conditions aux limites pour les potentiels sont \`a pr\'eciser.

{\bf Les conditions aux limites sur le potentiel scalaire} sont des conditions de
Neumann homog\`enes sur toutes les fronti\`eres hormis \`a la cathode et \`a
l'anode. \`A la cathode, on impose une condition de Dirichlet homog\`ene (potentiel nul par convention). \`A l'anode, on impose une
condition de Dirichlet permettant de fixer la diff\'erence de potentiel
souhait\'ee entre l'anode et la cathode.
L'utilisateur peut fixer le potentiel de l'anode directement ou
demander qu'un recalage automatique du potentiel soit effectu\'e pour atteindre
une intensit\'e de courant pr\'ed\'etermin\'ee.

Lorsque le recalage automatique est demand\'e (\var{IELCOR}=1), l'utilisateur doit fixer la
valeur cible de l'intensit\'e, \var{COUIMP}, ($A$) et une valeur \'elev\'ee
de d\'epart
de la diff\'erence de potentiel entre l'anode et la cathode\footnote{Plus pr\'ecis\'ement, l'utilisateur doit imposer un potentiel nul
en cathode et le potentiel \var{DPOT} \`a l'anode, en utilisant explicitement,
dans le sous-programme utilisateur \fort{uselcl}, la variable \var{DPOT} qui
sera automatiquement recal\'ee au cours du calcul.}, \var{DPOT}, ($V$).
Le recalage est effectu\'e en fin de pas temps et permet de disposer, pour le pas
de temps suivant, de valeurs recal\'ees des forces de Laplace et de l'effet
Joule.
\begin{itemize}
\item Pour effectuer le recalage, \CS\ d\'etermine l'int\'egrale de l'effet Joule
estim\'e sur le domaine (en $W$) et en compare la
valeur au produit de l'intensit\'e \var{COUIMP} par la diff\'erence de
potentiel\footnote{\var{DPOT} est la diff\'erence de
potentiel impos\'ee entre l'anode et la cathode au
pas de temps qui s'ach\`eve. \var{DPOT} a conditionn\'e le champ \'electrique et
la densit\'e de courant utilis\'es pour le calcul de l'effet Joule.} \var{DPOT}.
Un coefficient multiplicatif de recalage \var{COEPOT} en
est d\'eduit (pour \'eviter des variations trop brusques,
on s'assure qu'il reste born\'e). % entre 0,75 et 1,5).
\item On multiplie alors par \var{COEPOT} la
diff\'erence de potentiel entre l'anode et la cathode, \var{DPOT}, et le vecteur $\vect{j}$. L'effet
Joule, produit de $\vect{j}$ par $\vect{E}$, est multipli\'e par
le carr\'e de \var{COEPOT}. Pour assurer la coh\'erence du post-traitement des
variables, le potentiel vecteur et le potentiel scalaire sont \'egalement
multipli\'es par \var{COEPOT}.
\item Le champ \'electrique n'\'etant pas explicitement
stock\'e, on ne le recale pas. Le potentiel vecteur et les forces de Laplace seront d\'eduits de la
densit\'e de courant et int\'egreront donc naturellement le recalage.
\end{itemize}

\bigskip
{\bf Les conditions aux limites sur le potentiel vecteur} sont des conditions de
Neumann homog\`ene sur toutes les fronti\`eres hormis sur une zone de bord
arbitrairement choisie (paroi par exemple) pour laquelle une condition de
Dirichlet est utilis\'ee afin que le syst\`eme soit inversible
(la valeur impos\'ee est la valeur du potentiel vecteur
calcul\'ee au pas de temps pr\'ec\'edent).



\subsection{Effet Joule}

\subsubsection{Conditions aux limites}

Seules les conditions aux limites pour les potentiels sont \`a pr\'eciser.

{\bf Les conditions aux limites sur le potentiel scalaire} sont \`a pr\'eciser
au cas par cas selon la configuration des \'electrodes. Ainsi, on dispose
classiquement de conditions de Neumann homog\`enes ou de Dirichlet (potentiel
impos\'e). On peut \'egalement avoir besoin d'imposer des conditions
d'antisym\'etrie (en utilisant des conditions de Dirichlet homog\`enes par exemple).
L'utilisateur peut \'egalement souhaiter qu'un recalage automatique du potentiel
soit effectu\'e pour atteindre une valeur pr\'ed\'etermin\'ee de la puissance
dissip\'ee par effet Joule.

Lorsque le recalage automatique est demand\'e (\var{IELCOR}=1), l'utilisateur doit fixer la
valeur cible de la puissance dissip\'ee dans le domaine, \var{PUISIM}, ($V.A$).
Il doit en outre, sur les fronti\`eres o\`u il
souhaite que le potentiel (r\'eel ou complexe) s'adapte automatiquement, fournir en
condition \`a la limite une valeur initiale du potentiel et la multiplier par
la variable \var{COEJOU} qui sera automatiquement recal\'ee au cours du calcul
(\var{COEJOU} vaut 1 au premier pas de temps).
Le recalage est effectu\'e en fin de pas temps et permet de disposer, pour le pas
de temps suivant, d'une valeur recal\'ee de l'effet
Joule.
\begin{itemize}
\item Pour effectuer le recalage, \CS\ d\'etermine l'int\'egrale de l'effet Joule
estim\'e sur le domaine (en $W$) et en compare la
valeur \`a la puissance cible. Un coefficient multiplicatif de recalage \var{COEPOT} en
est d\'eduit (pour \'eviter des variations trop brusques,
on s'assure qu'il reste born\'e entre 0,75 et 1,5).
\item On multiplie alors par
\var{COEPOT} le facteur multiplicatif \var{COEJOU} utilis\'e pour les conditions aux
limites. La puissance dissip\'ee par effet
Joule est multipli\'ee par
le carr\'e de \var{COEPOT}. Pour assurer la coh\'erence du post-traitement des
variables, le potentiel est \'egalement
multipli\'e par \var{COEPOT}.
\item Le champ \'electrique n'\'etant pas explicitement
stock\'e, on ne le recale pas.
\end{itemize}

\bigskip
On notera que la variable \var{DPOT} est \'egalement recal\'ee et qu'elle
peut donc \^etre utilis\'ee si besoin pour imposer les conditions aux limites.

