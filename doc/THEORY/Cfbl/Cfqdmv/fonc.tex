%                      Code_Saturne version 1.3
%                      ------------------------
%
%     This file is part of the Code_Saturne Kernel, element of the
%     Code_Saturne CFD tool.
% 
%     Copyright (C) 1998-2007 EDF S.A., France
%
%     contact: saturne-support@edf.fr
% 
%     The Code_Saturne Kernel is free software; you can redistribute it
%     and/or modify it under the terms of the GNU General Public License
%     as published by the Free Software Foundation; either version 2 of
%     the License, or (at your option) any later version.
% 
%     The Code_Saturne Kernel is distributed in the hope that it will be
%     useful, but WITHOUT ANY WARRANTY; without even the implied warranty
%     of MERCHANTABILITY or FITNESS FOR A PARTICULAR PURPOSE.  See the
%     GNU General Public License for more details.
% 
%     You should have received a copy of the GNU General Public License
%     along with the Code_Saturne Kernel; if not, write to the
%     Free Software Foundation, Inc.,
%     51 Franklin St, Fifth Floor,
%     Boston, MA  02110-1301  USA
%
%-----------------------------------------------------------------------
%
\programme{cfqdmv}
%
\vspace{1cm}
%%%%%%%%%%%%%%%%%%%%%%%%%%%%%%%%%%
%%%%%%%%%%%%%%%%%%%%%%%%%%%%%%%%%%
\section{Fonction}
%%%%%%%%%%%%%%%%%%%%%%%%%%%%%%%%%%
%%%%%%%%%%%%%%%%%%%%%%%%%%%%%%%%%%

Pour les notations et l'algorithme dans son ensemble, 
on se reportera \`a \fort{cfbase}. 

Dans le premier pas fractionnaire (\fort{cfmsvl}), on a r\'esolu une 
\'equation sur la masse volumique, obtenu une pr\'ediction de la pression
et un flux convectif "acoustique".
On consid\`ere ici un second pas fractionnaire au cours duquel seul varie
le vecteur flux de masse $\vect{Q}=\rho\vect{u}$
(seule varie la vitesse au centre des cellules).
On r\'esout l'\'equation de Navier-Stokes ind\'ependamment
pour chaque direction d'espace, et l'on utilise le flux de masse acoustique
calcul\'e pr\'ec\'edemment comme flux convecteur (on pourrait aussi utiliser
le vecteur quantit\'e de mouvement du pas de temps pr\'ec\'edent).
De plus, on r\'esout en variable $\vect{u}$ et non $\vect{Q}$.

Le syst\`eme \`a r\'esoudre entre $t^*$ et $t^{**}$ est (on exclut 
la turbulence, dont le traitement n'a rien de particulier dans le 
module compressible)~:

\begin{equation}\label{Cfbl_Cfqdmv_eq_qdm_cfqdmv}
\left\{\begin{array}{l}

\rho^{**}=\rho^{*}=\rho^{n+1}\\
\\
\displaystyle\frac{\partial \rho\vect{u}}{\partial t}+
\divv(\vect{u} \otimes \vect{Q}_{ac}) + \gradv{P}
= \rho \vect{f}_v + \divv(\tens{\Sigma}^v)\\
\\
e^{**}=e^{*}=e^n\\

\end{array}\right.
\end{equation}

La r\'esolution de cette \'etape est similaire \`a l'\'etape
de pr\'ediction des vitesses du sch\'ema de base de \CS.
