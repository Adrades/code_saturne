%                      Code_Saturne version 1.3
%                      ------------------------
%
%     This file is part of the Code_Saturne Kernel, element of the
%     Code_Saturne CFD tool.
% 
%     Copyright (C) 1998-2007 EDF S.A., France
%
%     contact: saturne-support@edf.fr
% 
%     The Code_Saturne Kernel is free software; you can redistribute it
%     and/or modify it under the terms of the GNU General Public License
%     as published by the Free Software Foundation; either version 2 of
%     the License, or (at your option) any later version.
% 
%     The Code_Saturne Kernel is distributed in the hope that it will be
%     useful, but WITHOUT ANY WARRANTY; without even the implied warranty
%     of MERCHANTABILITY or FITNESS FOR A PARTICULAR PURPOSE.  See the
%     GNU General Public License for more details.
% 
%     You should have received a copy of the GNU General Public License
%     along with the Code_Saturne Kernel; if not, write to the
%     Free Software Foundation, Inc.,
%     51 Franklin St, Fifth Floor,
%     Boston, MA  02110-1301  USA
%
%-----------------------------------------------------------------------
%
\programme{cfmsvl}
%
\vspace{1cm}
%%%%%%%%%%%%%%%%%%%%%%%%%%%%%%%%%%
%%%%%%%%%%%%%%%%%%%%%%%%%%%%%%%%%%
\section{Fonction}
%%%%%%%%%%%%%%%%%%%%%%%%%%%%%%%%%%
%%%%%%%%%%%%%%%%%%%%%%%%%%%%%%%%%%

Pour les notations et l'algorithme dans son ensemble, 
on se reportera \`a \fort{cfbase}. 

On consid\`ere un premier pas fractionnaire au cours duquel l'\'energie totale
est fixe. Seules varient la masse volumique et le flux de masse acoustique
normal aux faces (d\'efini et calcul\'e aux faces). 

On a donc le syst\`eme suivant, entre $t^n$ et $t^*$~:
\begin{equation}\label{Cfbl_Cfmsvl_eq_acoustique_cfmsvl}
\left\{\begin{array}{l}

\displaystyle\frac{\partial\rho}{\partial t}+\divs{\vect{Q}_{ac}} = 0 \\
\\
\displaystyle\frac{\partial\vect{Q}_{ac}}{\partial t}+\gradv{P} =
\rho \vect{f}\\ 
\\
\vect{Q}^*=\vect{Q}^n\\
\\
e^*=e^n\\

\end{array}\right.
\end{equation}

Une partie des termes sources de l'\'equation de la
quantit\'e de mouvement peut \^etre prise en compte dans cette \'etape 
(les termes les plus importants, en pr\^etant attention aux sous-\'equilibres).

Il faut noter que si $\vect{f}$ est effectivement nul, on aura bien un
syst\`eme ``acoustique'', mais que si l'on place des termes suppl\'ementaires
dans $\vect{f}$, la d\'enomination est abusive (on la conservera cependant). 

On obtient $\rho^* = \rho^{n+1}$ en r\'esolvant (\ref{Cfbl_Cfmsvl_eq_acoustique_cfmsvl}),
et l'on actualise alors le flux de masse acoustique $\vect{Q}_{ac}^{n+1}$,
qui servira pour la convection (en particulier pour la convection de
l'enthalpie totale et de tous les scalaires transport\'es).
 
Suivant la valeur de \var{IGRDPP}, on actualise �ventuellement la pression, en
utilisant la loi d'\'etat : 
$$
\displaystyle P^{Pred}=P(\rho^{n+1},\varepsilon^{n})
$$
