%                      Code_Saturne version 1.3
%                      ------------------------
%
%     This file is part of the Code_Saturne Kernel, element of the
%     Code_Saturne CFD tool.
%
%     Copyright (C) 1998-2007 EDF S.A., France
%
%     contact: saturne-support@edf.fr
%
%     The Code_Saturne Kernel is free software; you can redistribute it
%     and/or modify it under the terms of the GNU General Public License
%     as published by the Free Software Foundation; either version 2 of
%     the License, or (at your option) any later version.
%
%     The Code_Saturne Kernel is distributed in the hope that it will be
%     useful, but WITHOUT ANY WARRANTY; without even the implied warranty
%     of MERCHANTABILITY or FITNESS FOR A PARTICULAR PURPOSE.  See the
%     GNU General Public License for more details.
%
%     You should have received a copy of the GNU General Public License
%     along with the Code_Saturne Kernel; if not, write to the
%     Free Software Foundation, Inc.,
%     51 Franklin St, Fifth Floor,
%     Boston, MA  02110-1301  USA
%
%-----------------------------------------------------------------------
%
%%%%%%%%%%%%%%%%%%%%%%%%%%%%%%%%%
%%%%%%%%%%%%%%%%%%%%%%%%%%%%%%%%%%
\section{Discr\'etisation}
%%%%%%%%%%%%%%%%%%%%%%%%%%%%%%%%%%
%%%%%%%%%%%%%%%%%%%%%%%%%%%%%%%%%%
%---------------------------------
\subsection{Discr\'etisation en temps}
%---------------------------------

Le syst\`eme (\ref{Cfbl_Cfmsvl_eq_acoustique_cfmsvl}) discr\'etis\'e en temps donne :
\begin{equation}\label{Cfbl_Cfmsvl_eq_acoustique_discrete_cfmsvl}
\left\{\begin{array}{l}

\displaystyle\frac{\rho^{n+1}-\rho^n}{\Delta t^n}
+ \divs{\vect{Q}_{ac}^{n+1}} = 0 \\
\\
\displaystyle\frac{\vect{Q}_{ac}^{n+1}-\vect{Q}^n}{\Delta t^n}+\gradv{P^*} =
\rho^n \vect{f}^n\\
\\
Q^*=Q^n\\
\\
e^*=e^n\\

\end{array}\right.
\end{equation}

\begin{equation}\label{Cfbl_Cfmsvl_eq_forces_supplementaires_cfmsvl}
\begin{array}{llll}
\text{avec\ }&\vect{f}^n &=& \vect{0} \\
\text{ou\ }&\vect{f}^n &=& \vect{g} \\
\text{ou m\^eme\ }&\vect{f}^n &=& \vect{f}_v
 + \displaystyle\frac{1}{\rho^n}
\left( - \divs(\vect{u} \otimes \vect{Q}) + \divv(\tens{\Sigma}^v)
 + \vect{j}\wedge\vect{B} \right)^n
\end{array}
\end{equation}

Dans la pratique nous avons d�cid� de prendre $\vect{f}^n=\vect{g}$~:
\begin{itemize}
  \item le terme $\vect{j}\wedge\vect{B}$ n'a pas �t� test�,
  \item le terme $\divv(\tens{\Sigma}^v)$ \'etait n�gligeable sur les tests
        r�alis�s,
  \item le terme $\divs(\vect{u} \otimes \vect{Q})$ a paru d�stabiliser les
        calculs (mais au moins une partie des tests a \'et\'e r\'ealis\'ee
        avec une erreur de programmation et il faudrait donc les reprendre).
\end{itemize}
\bigskip

Le terme $\vect{Q}^n$ dans la 2\textsuperscript{\`eme} \'equation
de (\ref{Cfbl_Cfmsvl_eq_acoustique_discrete_cfmsvl}) est le vecteur ``quantit\'e de mouvement''
qui provient de l'\'etape de r\'esolution de la quantit\'e de mouvement du pas
de temps pr\'ec\'edent, $\vect{Q}^n = \rho^n \vect{u}^n$.
On pourrait th�oriquement utiliser un vecteur quantit\'e de mouvement issu
de l'\'etape acoustique du pas de temps pr\'ec\'edent, mais il ne constitue
qu'un ``pr\'edicteur'' plus ou moins satisfaisant (il n'a pas ``vu'' les termes
sources qui ne sont pas dans  $\vect{f}^n$) et cette solution
n'a pas �t� test�e.

\bigskip
On \'ecrit alors la pression sous la forme~:
\begin{equation}
\gradv{P}=c^2\,\gradv{\rho}+\beta\,\gradv{s}
\end{equation}

avec $c^2 = \left.\displaystyle\frac{\partial P}{\partial \rho}\right|_s$
et $\beta = \left.\displaystyle\frac{\partial P}{\partial s}\right|_\rho$
tabul\'es ou analytiques \`a partir de la loi d'\'etat.

On discr\'etise l'expression pr\'ec\'edente en~:
\begin{equation}
\gradv{P^*}=(c^2)^n\gradv(\rho^{n+1})+\beta^n\gradv(s^n)
\end{equation}

On obtient alors une \'equation
portant sur $\rho^{n+1}$ en substituant l'expression de $\vect{Q}_{ac}^{n+1}$
issue de la 2\textsuperscript{\`eme} \'equation
de~(\ref{Cfbl_Cfmsvl_eq_acoustique_discrete_cfmsvl})
dans la 1\textsuperscript{\`ere} \'equation
de~(\ref{Cfbl_Cfmsvl_eq_acoustique_discrete_cfmsvl})~:
\begin{equation}\label{Cfbl_Cfmsvl_eq_densite_cfmsvl}
\displaystyle\frac{\rho^{n+1}-\rho^n}{\Delta t^n}
+\divs(\vect{w}^n \rho^n)
-\divs\left(\Delta t^n (c^2)^n \gradv(\rho^{n+1})\right) = 0
\end{equation}

o\`u~:
\begin{equation}
\begin{array}{lll}
\vect{w}^n&=&  \vect{u}^n + \Delta t^n
\displaystyle\left(\vect{f}^n-\frac{\beta^n}{\rho^n}\gradv(s^n)\right)
\end{array}
\end{equation}

Formulation alternative (programm\'ee mais non test\'ee)
avec le terme de convection implicite~:
\begin{equation}\label{Cfbl_Cfmsvl_eq_densite_bis_cfmsvl}
\displaystyle\frac{\rho^{n+1}-\rho^n}{\Delta t^n}
+\divs(\vect{w}^n \rho^{n+1})
-\divs\left(\Delta t^n (c^2)^n \gradv(\rho^{n+1})\right) = 0
\end{equation}


%---------------------------------
\subsection{Discr\'etisation en espace}
%---------------------------------


%.................................
\subsubsection{Introduction}
%.................................

On int\`egre l'\'equation pr\'ec\'edente ( (\ref{Cfbl_Cfmsvl_eq_densite_cfmsvl})
ou (\ref{Cfbl_Cfmsvl_eq_densite_bis_cfmsvl}) ) sur la cellule $i$ de volume $\Omega_i$.
On transforme les int\'egrales de volume en int\'egrales surfaciques
et l'on discr\'etise ces int\'egrales. Pour simplifier l'expos�, on se
place sur une cellule $i$ dont aucune face n'est sur le bord du domaine.

On obtient alors l'\'equation discr\`ete
suivante\footnote{L'exposant $^{n+\frac{1}{2}}$ signifie que le terme
peut \^etre implicite ou explicite. En pratique on a choisi
$\rho^{n+\frac{1}{2}} = \rho^{n}$.}~:
\begin{equation}\label{Cfbl_Cfmsvl_eq_densite_discrete_cfmsvl}
\Omega_i \displaystyle\frac{\rho_i^{n+1}-\rho_i^n}{\Delta t^n}
+\sum\limits_{j\in Vois(i)}(\rho^{n+\frac{1}{2}} \vect{w}^n)_{ij} \cdot \vect{S}_{ij}
-\sum\limits_{j\in Vois(i)} \left(\Delta t^n (c^2)^n
\gradv(\rho^{n+1})\right)_{ij} \cdot \vect{S}_{ij}
= 0
\end{equation}

%.................................
\subsubsection{Discr\'etisation de la partie ``convective''}
%.................................

La valeur \`a la face s'\'ecrit~:
\begin{equation}
(\rho^{n+\frac{1}{2}} \vect{w}^n)_{ij} \cdot \vect{S}_{ij}
= \rho^{n+\frac{1}{2}}_{ij} \vect{w}^n_{ij} \cdot \vect{S}_{ij}
\end{equation}
avec, pour $\vect{w}^n_{ij}$,
une simple interpolation lin\'eaire~:
\begin{equation}
\vect{w}^n_{ij}
= \alpha_{ij} \vect{w}^n_i + (1-\alpha_{ij}) \vect{w}^n_j
\end{equation}
et un d\'ecentrement sur la valeur de $\rho^{n+\frac{1}{2}}$ aux faces~:
\begin{equation}
\begin{array}{lllll}
\displaystyle\rho_{ij}^{n+\frac{1}{2}} &=& \rho_{I'}^{n+\frac{1}{2}}
                &\text{si\ }& \vect{w}^n_{ij} \cdot \vect{S}_{ij} \geqslant 0 \\
                         &=& \rho_{J'}^{n+\frac{1}{2}}
                &\text{si\ }& \vect{w}^n_{ij} \cdot \vect{S}_{ij} < 0 \\
\end{array}
\end{equation}
que l'on peut noter~:
\begin{equation}
\displaystyle\rho_{ij}^{n+\frac{1}{2}}
 = \beta_{ij}\rho_{I'}^{n+\frac{1}{2}} + (1-\beta_{ij})\rho_{J'}^{n+\frac{1}{2}}
\end{equation}
avec
\begin{equation}
\left\{\begin{array}{lll}
\beta_{ij} = 1 & \text{si\ } & \vect{w}^n_{ij} \cdot \vect{S}_{ij} \geqslant 0 \\
\beta_{ij} = 0 & \text{si\ } & \vect{w}^n_{ij} \cdot \vect{S}_{ij} < 0 \\
\end{array}\right.
\end{equation}

%.................................
\subsubsection{Discr\'etisation de la partie ``diffusive''}
%.................................

La valeur \`a la face s'\'ecrit~:
\begin{equation}
\left(\Delta t^n (c^2)^n \gradv(\rho^{n+1})\right)_{ij}\cdot \vect{S}_{ij}
= \Delta t^n (c^2)^n_{ij}
\displaystyle \left( \frac{\partial \rho}{\partial n} \right)^{n+1}_{ij}S_{ij}
\end{equation}
avec, pour assurer la continuit\'e du flux normal \`a l'interface,
une interpolation harmonique de $(c^2)^n$~:
\begin{equation}\label{Cfbl_Cfmsvl_eq_harmonique_cfmsvl}
\displaystyle(c^2)_{ij}^n
= \frac{(c^2)_{i}^n (c^2)_{j}^n}
{\alpha_{ij}(c^2)_{i}^n+(1-\alpha_{ij})(c^2)_{j}^n}
\end{equation}
et un sch\'ema centr\'e pour le gradient normal aux faces~:
\begin{equation}
\displaystyle \left( \frac{\partial \rho}{\partial n} \right)^{n+1}_{ij}
= \displaystyle\frac{\rho_{J'}^{n+1}-\rho_{I'}^{n+1}}{\overline{I'J'}}
\end{equation}

%.................................
\subsubsection{Syst\`eme final}
%.................................

On obtient maintenant le syst\`eme final, portant sur
$(\rho_i^{n+1})_{i=1 \ldots N}$~:
\begin{equation}\label{Cfbl_Cfmsvl_eq_densite_finale_cfmsvl}
\displaystyle\frac{\Omega_i}{\Delta t^n} (\rho_i^{n+1}-\rho_i^n)
+\sum\limits_{j\in Vois(i)}\rho_{ij}^{n+\frac{1}{2}}
\vect{w}_{ij}^n \cdot \vect{S}_{ij}
-\sum\limits_{j\in Vois(i)} \Delta t^n (c^2)_{ij}^n
\displaystyle\frac{\rho_{J'}^{n+1}-\rho_{I'}^{n+1}}{\overline{I'J'}}\ S_{ij}
= 0
\end{equation}



%.................................
\subsubsection{Remarque~: interpolation aux faces pour le terme de diffusion}
%.................................

Le choix de la forme de la moyenne pour le cofacteur du flux
normal n'est pas sans cons\'equence sur la vitesse de convergence, surtout
lorsque l'on est en pr\'esence de fortes inhomog\'en\'eit\'es.

On utilise une interpolation harmonique pour $c^2$
afin de conserver la continuit\'e du flux diffusif normal
$\Delta t (c^2) \displaystyle\frac{\partial \rho}{\partial n}$
\`a l'interface $ij$. En effet, on suppose que le flux est d\'erivable \`a
l'interface. Il doit donc y \^etre continu.\\
%
\'Ecrivons la continuit\'e du flux normal \`a l'interface,
avec la discr\'etisation
suivante\footnote{On ne reconstruit pas les valeurs de $\Delta\,t\,c^2$
aux points $I'$ et
$J'$.}~:
\begin{equation}
\left(\Delta t (c^2)\displaystyle\frac{\partial \rho}{\partial n}\right)_{ij}
= \Delta t (c^2)_i  \displaystyle\frac{\rho_{ij} - \rho_{I'}   }{\overline{I'F}}
=  \Delta t (c^2)_j  \displaystyle\frac{\rho_{J'}    - \rho_{ij}}{\overline{FJ'}}
\end{equation}
En \'egalant les flux \`a gauche et \`a droite de l'interface, on obtient
\begin{equation}
\rho_{ij} = \displaystyle\frac{\overline{I'F}\,(c^2)_j\rho_{J'} + \overline{FJ'}\,(c^2)_i\rho_{I'}}
{\overline{I'F}\,(c^2)_j + \overline{FJ'}\,(c^2)_i}
\end{equation}
On introduit cette formulation dans la d\'efinition du flux (par exemple, du
flux \`a gauche)~:
\begin{equation}
\left(\Delta t (c^2)\displaystyle\frac{\partial \rho}{\partial n}\right)_{ij}
= \Delta t (c^2)_i  \displaystyle\frac{\rho_{ij} - \rho_{I'}   }{\overline{I'F}}
\end{equation}
et on utilise la d\'efinition de $(c^2)_{ij}$ en fonction de ce m\^eme flux
\begin{equation}
\left(\Delta t (c^2)\displaystyle\frac{\partial \rho}{\partial n}\right)_{ij}
 \stackrel{\text{d\'ef}}{=}
 \Delta t (c^2)_{ij} \displaystyle\frac{\rho_{J'}    - \rho_{I'}   }{\overline{I'J'}}
\end{equation}
pour obtenir la valeur de $(c^2)_{ij}$ correspondant \`a l'\'equation (\ref{Cfbl_Cfmsvl_eq_harmonique_cfmsvl})~:
\begin{equation}
(c^2)_{ij} = \displaystyle\frac{\overline{I'J'}\,(c^2)_i(c^2)_j}{\overline{FJ'}\,(c^2)_i + \overline{I'F}\,(c^2)_j}
\end{equation}

