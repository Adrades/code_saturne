%                      Code_Saturne version 1.3
%                      ------------------------
%
%     This file is part of the Code_Saturne Kernel, element of the
%     Code_Saturne CFD tool.
% 
%     Copyright (C) 1998-2007 EDF S.A., France
%
%     contact: saturne-support@edf.fr
% 
%     The Code_Saturne Kernel is free software; you can redistribute it
%     and/or modify it under the terms of the GNU General Public License
%     as published by the Free Software Foundation; either version 2 of
%     the License, or (at your option) any later version.
% 
%     The Code_Saturne Kernel is distributed in the hope that it will be
%     useful, but WITHOUT ANY WARRANTY; without even the implied warranty
%     of MERCHANTABILITY or FITNESS FOR A PARTICULAR PURPOSE.  See the
%     GNU General Public License for more details.
% 
%     You should have received a copy of the GNU General Public License
%     along with the Code_Saturne Kernel; if not, write to the
%     Free Software Foundation, Inc.,
%     51 Franklin St, Fifth Floor,
%     Boston, MA  02110-1301  USA
%
%-----------------------------------------------------------------------
%


\programme{elec**}

\vspace{0,5cm}
On s'int\'eresse \`a la r\'esolution des \'equations de la
magn\'etohydrodynamique, constitu\'ees de la r\'eunion des \'equations de
l'a\'erothermodynamique et des \'equations de Maxwell. 

On se place dans deux cadres d'utilisation bien sp\'ecifiques et distincts, 
qui permettront chacun de r\'ealiser des simplifications~: les \'etudes dites 
``d'arc
\'electrique'' (dans lesquelles sont prises en compte les forces de Laplace et
l'effet Joule) et les \'etudes dites ``Joule'' (dans lesquelles seul
l'effet Joule est pris en compte). 

Les \'etudes d'arc \'electrique sont associ\'ees en grande partie, pour EDF, aux
probl\'ematiques relatives aux transformateurs. Les \'etudes Joule sont plus
s\'ecifiquement li\'ees aux ph\'enom\`enes rencontr\'es dans les fours verriers.

Outre la prise en compte ou non des forces de Laplace, ces deux types d'\'etudes
se diff\'erencient \'egalement par le mode de d\'etermination de l'effet Joule 
(utilisation d'un potentiel complexe pour les \'etudes Joule faisant intervenir
un courant alternatif non monophas\'e). 

On d\'ecrit tout d'abord les \'equations r\'esolues pour les \'etudes d'arc
\'electrique. Les sp\'ecificit\'es des \'etudes Joule seront abord\'ees ensuite.


Pour l'arc \'electrique, 
les r\'ef\'erences [douce] et [delalondre] pourront compl\'eter la
pr\'esentation~: 

\noindent{\bf [delalondre] }Delalondre, Clarisse~: ``Mod\'elisation a\'erothermodynamique d'arcs
\'electriques \`a forte intensit\'e avec prise en compte du d\'es\'equilibre
thermodynamique local et du transfert thermique \`a la cathode'', Th\`ese de
l'Universit\'e de Rouen, 1990
 
\noindent{\bf [douce]} Douce, Alexandre~: ``Mod�lisation 3-D du chauffage d'un
bain m�tallique par plasma d'arc transf\'er\'e. Application \`a un r\'eacteur
axisym\'etrique'', HE-26/99/027A, HE-44/99/043A, Th\`ese de l'Ecole Centrale
Paris et EDF, 1999 




%%%%%%%%%%%%%%%%%%%%%%%%%%%%%%%%%%
%%%%%%%%%%%%%%%%%%%%%%%%%%%%%%%%%%
\section{Fonction}
%%%%%%%%%%%%%%%%%%%%%%%%%%%%%%%%%%
%%%%%%%%%%%%%%%%%%%%%%%%%%%%%%%%%%

\subsection{Notations}

{\bf Variables utilis\'ees}
\nopagebreak

\begin{tabular}{lp{6cm}l}
$\vect{A}$	&potentiel vecteur r\'eel	&$kg\,m\,s^{-2}\,A^{-1}$ \\
$\vect{B}$	&champ magn\'etique 		&$T$ (ou $kg\,s^{-2}\,A^{-1}$) \\
$\vect{D}$ 	&d\'eplacement \'electrique	&$A\,s\,m^{-2}$ \\
$\vect{E}$ 	&champ \'electrique 		&$V\,m^{-1}$ \\
$E$	 	&\'energie totale massique 	&$J\,kg^{-1}$ (ou $m^{2}\,s^{-2}$) \\
$e$	 	&\'energie interne massique 	&$J\,kg^{-1}$ (ou $m^{2}\,s^{-2}$) \\
$e_c$	 	&\'energie cin\'etique massique &$J\,kg^{-1}$ (ou $m^{2}\,s^{-2}$) \\
$\vect{H}$ 	&excitation magn\'etique	&$A\,m^{-1}$ \\
$h$ 	  	&enthalpie massique 		&$J\,kg^{-1}$ (ou $m^{2}\,s^{-2}$) \\
$\vect{j}$	&densit\'e de courant 		&$A\,m^{-2}$  \\
$P$ 	  	&pression 			&$kg\,m^{-1}\,s^{-2}$ \\
$P_R$, $P_I$ 	&potentiel scalaire r\'eel, imaginaire	
						&$V$ (ou $kg\,m^{2}\,s^{-3}\,A^{-1}$) \\
$\vect{u}$ 	&vitesse			&$m\,s^{-1}$  \\
 		&				& \\
$\varepsilon$ &permittivit\'e \'electrique
						&$F\,m^{-1}$ (ou $m^{-3}\,kg^{-1}\,s^{4}\,A^{2}$) \\
$\varepsilon_0$ &permittivit\'e \'electrique du vide
						&$8,854\,10^{-12}\,\,F\,m^{-1}$ (ou $m^{-3}\,kg^{-1}\,s^{4}\,A^{2}$) \\
$\mu$           &perm\'eabilit\'e \'electrique
						&$H\,m^{-1}$ (ou $m\,kg\,s^{-2}\,A^{-2}$)\\
$\mu_0$ 	&perm\'eabilit\'e \'electrique du vide
						&$4\,\pi\,10^{-7}\,\,H\,m^{-1}$ (ou $m\,kg\,s^{-2}\,A^{-2}$)\\
$\sigma$ 	&conductivit\'e \'electrique	&$S\,m^{-1}$ (ou $m^{-3}\,kg^{-1}\,s^3\,A^2$)\\
\end{tabular}

\vspace*{0,5cm}
{\bf Notations d'analyse vectorielle}
\nopagebreak

On rappelle \'egalement la d\'efinition des notations employ\'ees\footnote{en
utilisant la convention de sommation d'Einstein.}~:
\begin{equation}\notag
\left\{\begin{array}{lll}
\left[\ggrad{\vect{a}}\right]_{ij} &=& \partial_j a_i\\
\left[\dive(\tens{\sigma})\right]_i &=& \partial_j \sigma_{ij}\\
\left[\vect{a}\otimes\vect{b}\right]_{ij} &= &
a_i\,b_j\\
\end{array}\right.
\end{equation}
et donc :
\begin{equation}\notag
\begin{array}{lll}
\left[\dive(\vect{a}\otimes\vect{b})\right]_i &= &
\partial_j (a_i\,b_j)
\end{array} 
\end{equation}


\subsection{Arcs \'electriques}

\subsubsection{Introduction}

Pour les \'etudes d'arc \'electrique, on calcule, 
\`a un pas de temps donn\'e~:
\begin{itemize}
\item la vitesse $\vect{u}$, la pression $P$, la variable \'energ\'etique
enthalpie $h$ (et les grandeurs turbulentes), 
\item un potentiel scalaire r\'eel $P_R$ 
(dont le gradient permet d'obtenir le champ \'electrique $\vect{E}$ et 
la densit\'e de courant $\vect{j}$),  
\item un potentiel vecteur r\'eel $\vect{A}$ (dont
le rotationnel permet d'obtenir  le champ magn\'etique $\vect{B}$). 
\end{itemize}

\bigskip
Le champ \'electrique, la
densit\'e de courant et le champ magn\'etique sont utilis\'es pour calculer les
termes sources d'effet Joule et les forces de Laplace qui interviennent
respectivement dans l'\'equation de l'enthalpie et dans celle 
de la quantit\'e de mouvement. 


\subsubsection{\'Equations continues}

{\bf Syst\`eme d'\'equations} 
\nopagebreak

Les \'equations continues qui sont r\'esolues sont les suivantes~: 
\begin{equation}
\left\{\begin{array}{l}
{\color{blue}\dive(\rho \vect{u}) = 0}\\
{\color{blue}\displaystyle\frac{\partial}{\partial t}(\rho \vect{u}) 
+\dive(\rho\, \vect{u} \otimes \vect{u})
=\dive(\tens{\sigma}) + \vect{TS} + {\color{red}\vect{j} \times \vect{B}}}\\ 
{\color{blue}\displaystyle\frac{\partial}{\partial t}(\rho h) 
+\dive(\rho\, \vect{u} h)
=\Phi_v -
\dive{\left(\left(\frac{\lambda}{C_p}+\frac{\mu_t}{\sigma_t}\right)\grad{h}\right)}
+ {\color{red}P_J}}\\
{\color{red}\dive(\sigma\,\grad{P_R})=0}\\
{\color{red}\dive(\ggrad{\vect{A}})=-\mu_0\vect{j}}
\end{array}\right.
\end{equation}

avec les relations suivantes~:
\begin{equation}
\left\{\begin{array}{l}
{\color{red}P_J=\vect{j}\cdot\vect{E}}\\
{\color{red}\vect{E}=-\grad{P_R}}\\
{\color{red}\vect{j}=\sigma\vect{E}}\\
\end{array}\right.
\end{equation}

%On donne ci-apr\`es diff\'erents \'el\'ements permettant de pr\'eciser le mode
%d'obtention de ces \'equations. 

\vspace*{0,5cm}
{\bf \'Equation de la masse} 
\nopagebreak

C'est l'\'equation r\'esolue en standard par \CS\ (contrainte
stationnaire). 
Elle n'a pas de traitement particulier dans
le cadre du module pr\'esent. Un terme source de  masse peut \^etre pris en
compte au second membre si l'utilisateur le souhaite. Pour simplifier l'expos\'e
le terme source sera suppos\'e nul ici, dans la mesure o\`u il n'est pas
sp\'ecifique au module \'electrique. 

\vspace*{0,5cm}
{\bf \'Equation de la quantit\'e de mouvement} 
\nopagebreak

Elle pr\'esente, par rapport \`a
l'\'equation standard r\'esolue par \CS, un seul terme additionnel 
($\vect{j} \times \vect{B}$) qui rend compte des forces de Laplace. 
Pour l'obtenir, on fait l'hypoth\`ese que le milieu est 
\'electriquement neutre. 

En effet, une charge $q_i$ (Coulomb) anim\'ee d'une
vitesse $\vect{v}_i$ subit, 
sous l'effet du champ \'electrique $\vect{E}$ ($V\,m^{-1}$) et du champ magn\'etique 
$\vect{B}$ (Tesla),  une force $\vect{f}_i$ ($kg\,m\,s^{-2}$)~: 
\begin{equation}
\vect{f}_i=q_i\left(\vect{E} + \vect{v}_i \times \vect{B}\right)
\end{equation}
Avec $n_i$ charges de type $q_i$ par unit\'e de volume et en sommant sur tous
les types de charge $i$ (\'electrons, ions, mol\'ecules ionis\'ees...), on
obtient la force de Laplace totale $\vect{F}_L$ ($kg\,m^{-2}\,s^{-2}$) subie par unit\'e de
volume~:    
\begin{equation}
\vect{F}_L=\sum\limits_i\left[{n_i\,q_i\left(\vect{E} + \vect{v}_i \times \vect{B}\right)}\right]
\end{equation}
On introduit alors la densit\'e de courant $\vect{j}$ ($A\,m^{-2}$)~:
\begin{equation}
\vect{j}=\sum\limits_i n_i\,q_i\,\vect{v}_i 
\end{equation}
Avec l'hypoth\`ese que le milieu est \'electriquement neutre (\`a un
niveau macroscopique)~: 
\begin{equation}
\sum\limits_i n_i\,q_i = 0 
\end{equation}
la force totale $\vect{F}_L$ s'\'ecrit alors~: 
\begin{equation}
\vect{F}_L=\vect{j} \times \vect{B}
\end{equation}
et on peut donc \'ecrire l'\'equation de la quantit\'e de mouvement~: 
\begin{equation}
\displaystyle{\color{blue}\frac{\partial}{\partial t}(\rho \vect{u}) 
+\dive(\rho\, \vect{u} \otimes \vect{u})
=\dive(\tens{\sigma}) + \vect{TS} + {\color{red}\vect{j} \times \vect{B}}}
\end{equation}

\vspace*{0,5cm}
{\bf \'Equation de l'enthalpie} 
\nopagebreak

Elle est obtenue \`a partir de l'�quation de
l'\'energie apr\`es plusieurs approximations utilis\'ees en standard dans \CS\ et en
prenant en compte le terme d'effet Joule li\'e \`a l'\'energie
\'electromagn\'etique. 

\underline{\'Energie \'electromagn\'etique}
\nopagebreak

Avec les m\^emes notations que pr\'ec\'edemment mais sans qu'il soit 
besoin de supposer que le milieu est \'electriquement neutre, 
la puissance re\c cue par une charge $q_i$ (particule dou\'ee de masse) 
de vitesse $\vect{v}_i$ (vitesse du porteur de charge, contenant \'eventuellement 
l'effet de la vitesse du fluide) sous l'effet 
du champ \'electrique $\vect{E}$ ($V\,m^{-1}$) 
et du champ magn\'etique  $\vect{B}$ ($T$) est (sans
sommation sur $i$)~:  
\begin{equation}
P_i=\vect{f}_i\cdot\vect{v}_i=
q_i(\vect{E}+\vect{v}_i\times\vect{B})\cdot\vect{v}_i 
= q_i\vect{v}_i\cdot\vect{E}
\end{equation} 
Avec $n_i$ charges par unit\'e de volume et en sommant sur tous les types 
de charges $i$, on obtient la puissance totale par unit\'e de volume~:
\begin{equation}
P_J=
\sum\limits_i n_i\,q_i\,\vect{v}_i\cdot\vect{E}
\end{equation} 
On introduit alors la densit\'e de courant $\vect{j}=\sum\limits_i n_i\,q_i\,\vect{v}_i$ (en $A\,m^{-2}$) et on obtient l'expression usuelle de la puissance 
\'electromagn\'etique dissip\'ee par effet Joule (en $W\,m^{-3}$)~:
\begin{equation}
P_J=\vect{j}\cdot\vect{E}
\end{equation} 

Pour reformuler la puissance dissip\'ee par effet Joule et obtenir 
une \'equation d'\'evolution de l'\'energie \'electromagn\'etique, on utilise
alors les \'equations de Maxwell. 
Les \'equations s'\'ecrivent (lois d'Amp\`ere
et de Faraday)~: 
\begin{equation}
\left\{
\begin{array}{l}
\displaystyle\frac{\partial \vect{D}}{\partial t} - \rot\vect{H} = -\vect{j}\\
\displaystyle\frac{\partial \vect{B}}{\partial t} + \rot\vect{E} = 0 
\end{array}
\right.
\end{equation} 

On a donc~:
\begin{equation}
P_J=\vect{j}\cdot\vect{E}=\left(-\frac{\partial \vect{D}}{\partial t} +\rot\vect{H}\right)\cdot\vect{E}
\end{equation} 
On utilise alors la relation suivante~:
\begin{equation}
\rot\vect{H}\cdot\vect{E}=\vect{H}\cdot\rot\vect{E}-\dive(\vect{E}\times\vect{H})
\end{equation} 
En effet, elle permet de faire appara\^\i tre un 
terme en divergence, caract\'eristique d'une redistribution spatiale~:
\begin{equation}
\begin{array}{lll}
\vect{j}\cdot\vect{E}&=&
\displaystyle-\frac{\partial \vect{D}}{\partial t}\cdot\vect{E} 
+\vect{H}\cdot\rot\vect{E}-\dive(\vect{E}\times\vect{H})\\
\end{array}
\end{equation} 
Et en utilisant la loi de Faraday pour faire appara\^\i tre la d\'eriv\'ee en temps du champ magn\'etique~: 
\begin{equation}
\begin{array}{lll}
\vect{j}\cdot\vect{E}&=&
\displaystyle-\frac{\partial \vect{D}}{\partial t}\cdot\vect{E} 
-\vect{H}\cdot\frac{\partial \vect{B}}{\partial t}-\dive(\vect{E}\times\vect{H})\\
\end{array}
\end{equation} 

Dans le cadre de \CS, on fait les hypoth\`eses suivantes~:
\begin{itemize}
\item la perm\'eabilit\'e $\varepsilon$ et la permittivit\'e $\mu$
sont constantes et uniformes (pour les gaz, en pratique, on utilise 
les propri\'et\'es du vide $\varepsilon_0$ et $\mu_0$). 
\item on utilise $\vect{B} = \mu \vect{H}$ et $\vect{D} = \varepsilon \vect{E}$
\end{itemize}

On a alors~: 

\begin{equation}
\begin{array}{lll}
\displaystyle\vect{j}\cdot\vect{E}&=&
\displaystyle-\frac{\varepsilon_0}{2}\frac{\partial E^2}{\partial t} 
-\frac{1}{2\,\mu_0}\frac{\partial B^2}{\partial t} 
-\frac{1}{\mu_0}\dive(\vect{E}\times\vect{B})
\end{array}
\end{equation} 

\underline{\'Energie totale}
\nopagebreak

On \'etablit l'\'equation de l'\'energie totale en prenant en compte la
puissance des forces de Laplace et le terme
d'effet Joule.  

Sans prendre en compte l'\'energie \'electromagn\'etique, 
le premier principe de la thermodynamique s'\'ecrit d'ordinaire sous la
forme suivante (pour un volume mat\'eriel suivi sur une unit\'e de temps)~: 
\begin{equation}\label{Elec_Elbase_premier_ppe_eq}
d\int_V \rho E dV=\delta Q+\delta W 
\end{equation} 
Dans cette relation, $E$ est l'\'energie totale par unit\'e de masse\footnote{Ne pas
confondre le scalaire $E$, \'energie totale, avec le vecteur $\vect{E}$, champ
\'electrique.}, soit $E=e+e_c$, $e$ \'etant l'\'energie interne massique et
$e_c=\frac{1}{2}\,\vect{u}\cdot\vect{u}$ l'\'energie cin\'etique massique. Le terme  
$\delta Q$  repr\'esente  la chaleur re\c cue au travers des fronti\`eres du 
domaine consid\'er\'e tandis que le terme  $\delta W$ repr\'esente le travail 
des forces ext\'erieures re\c cu par le syst\`eme (y compris les forces
d\'erivant d'une \'energie potentielle). 

Pour prendre en compte l'\'energie \'electromagn\'etique, il suffit d'int\'egrer
\`a la relation (\ref{Elec_Elbase_premier_ppe_eq}) la puissance des forces de Laplace $(\vect{j} \times
\vect{B})\cdot\vect{u}$ et le terme d'effet Joule $\vect{j}\cdot\vect{E}$ 
(transformation volumique d'\'energie \'electromagn\'etique en \'energie 
totale\footnote{Le terme en divergence
$-\frac{1}{\mu_0}\dive(\vect{E}\times\vect{B})$ 
traduit une redistribution spatiale d'\'energie \'electromagn\'etique~: 
ce n'est donc pas un terme source pour l'\'energie totale.}).
Dans cette relation, la vitesse $\vect{u}$ est la vitesse du fluide et non pas 
celle des porteurs de charge~: elle n'est donc pas n\'ecessairement coli\'eaire
au vecteur \vect{j} (par exemple, si le courant est d\^u \`a des \'electrons, 
la vitesse du fluide pourra \^etre consid\'er\'ee comme d\'ecorr\'el\'ee de la 
vitesse des porteurs de charges~; par contre, si le courant est d\^u \`a des ions, 
la vitesse du fluide pourra \^etre plus directement influenc\'ee par 
le d\'eplacement des porteurs de charge).
Ainsi, le premier principe de la thermodynamique s'\'ecrit~: 
\begin{equation}
d\int_V \rho E dV=\delta Q+\delta W+\vect{j}\cdot\vect{E}\,V\,dt +(\vect{j} \times
\vect{B})\cdot\vect{u}\,V\,dt
\end{equation} 
et l'\'equation locale pour l'\'energie totale est alors~:
\begin{equation} 
\displaystyle\frac{\partial}{\partial t}(\rho E) 
+\dive(\rho\, \vect{u} E)
=\dive(\tens{\sigma}\,\vect{u}) + \vect{TS}\cdot\vect{u} + {\color{red}(\vect{j} \times
\vect{B})\cdot\vect{u}} + \Phi_v - \dive{\vect{\Phi}_s} + {\color{red}\vect{j}\cdot \vect{E}}
\end{equation} 
Le terme $\Phi_v$ repr\'esente les termes sources volumiques d'\'energie
autres que l'effet Joule (par exemple, il inclut le terme source de
rayonnement, pour un milieu optiquement non transparent). Le terme $\vect{\Phi}_s$ est
le flux d'\'energie surfacique\footnote{Dans \CS, il est mod\'elis\'e par 
une hypoth\`ese de gradient et inclut �galement la ``diffusion'' turbulente.}. 

\underline{Enthalpie}
\nopagebreak

Pour obtenir une \'equation sur l'enthalpie, qui est la variable \'energ\'etique
choisie dans \CS\ dans le module \'electrique, on
soustrait tout d'abord \`a l'\'equation de l'\'energie totale celle de l'\'energie
cin\'etique pour obtenir une \'equation sur l'\'energie interne. 

L'\'equation de l'\'energie
cin\'etique (obtenue \`a partir de l'\'equation de la quantit\'e de mouvement
\'ecrite sous forme non conservative) est~: 
\begin{equation} 
\displaystyle\frac{\partial}{\partial t}(\rho e_c) 
+\dive(\rho\, \vect{u} e_c)
=\dive(\tens{\sigma}\,\vect{u}) - \tens{\sigma}:\left(\ggrad(\vect{u})\right)^t +
\vect{TS}\cdot\vect{u} + (\vect{j} \times \vect{B})\cdot\vect{u} 
\end{equation} 
de sorte que, pour l'\'energie interne, on a~:
\begin{equation} 
\displaystyle\frac{\partial}{\partial t}(\rho e) 
+\dive(\rho\, \vect{u} e)
=\tens{\sigma}:\left(\ggrad(\vect{u})\right)^t + \Phi_v - \dive{\Phi_s} + \vect{j}\cdot \vect{E}
\end{equation} 
et enfin, pour l'enthalpie $h=e+\frac{P}{\rho}$~: 
\begin{equation} 
\displaystyle\frac{\partial}{\partial t}(\rho h) 
+\dive(\rho\, \vect{u} h)
=\tens{\sigma}:\left(\ggrad(\vect{u})\right)^t + \Phi_v - \dive{\Phi_s} + \vect{j}\cdot \vect{E}+\rho\frac{d}{dt}\left(\frac{P}{\rho}\right)
\end{equation} 
En faisant appara\^itre la pression dans le tenseur des contraintes
$\tens{\sigma}=-P\tens{Id}+\tens{\tau}$, on peut \'ecrire~:
\begin{equation} 
\displaystyle\frac{\partial}{\partial t}(\rho h) 
+\dive(\rho\, \vect{u} h)
=\tens{\tau}:\left(\ggrad(\vect{u})\right)^t + \Phi_v - \dive{\Phi_s} 
+ \vect{j}\cdot\vect{E} + \frac{dP}{dt}
\end{equation} 

Les approximations habituelles de \CS\ consistent alors 
\`a n\'egliger le terme ``d'\'echauffement'' issu du tenseur des contraintes  
$\tens{\tau}:\left(\ggrad(\vect{u})\right)^t$ et le terme en d\'eriv\'ee totale de la
pression $\frac{dP}{dt}$, suppos\'es faibles en comparaison des autres termes 
dans les applications trait\'ees (exemple~: terme d'effet Joule important, effets de
compressibilit\'e faibles...). 
De plus, le terme de flux est mod\'elis\'e en suivant
une hypoth\`ese de gradient appliqu\'e \`a l'enthalpie (et non pas \`a la
temp\'erature), soit donc~:
\begin{equation} 
{\color{blue}\displaystyle\frac{\partial}{\partial t}(\rho h) 
+\dive(\rho\, \vect{u} h)
=\Phi_v -
\dive{\left(\left(\frac{\lambda}{C_p}+\frac{\mu_t}{\sigma_t}\right)\grad h\right)} + {\color{red}\vect{j}\cdot\vect{E}}}
\end{equation} 


\vspace*{0,5cm}
{\bf \'Equations \'electromagn\'etiques} 
\nopagebreak

Elles sont obtenues \`a partir des
\'equations de Maxwell sous les hypoth\`eses d\'etaill\'ees dans [douce], 
paragraphe 3.3.

\underline{Densit\'e de courant}
\nopagebreak

La relation liant la densit\'e de courant et le champ \'electrique est issue de
la loi d'Ohm que l'on suppose pouvoir utiliser sous la forme
simplifi\'ee suivante~:
\begin{equation}\label{Elec_Elbase_ohm_eq} 
{\color{red}\vect{j}=\sigma\,\vect{E}} 
\end{equation} 

\underline{Champ \'electrique}
\nopagebreak

Le champ \'electrique s'obtient \`a partir d'un potentiel vecteur. 

En effet, la loi de Faraday s'\'ecrit~:
\begin{equation}
\frac{\partial\vect{B}}{\partial t}+\rot\vect{E}=0
\end{equation}  
Avec une hypoth\`ese quasi-stationnaire, il reste~: 
\begin{equation}
\rot\vect{E}=0
\end{equation}  
Il est donc possible de postuler l'existence d'un potentiel scalaire $P_R$
tel que~:
\begin{equation}\label{Elec_Elbase_e_eq}
{\color{red}\vect{E}=-\grad{P_R}}
\end{equation}  

\underline{Potentiel scalaire}
\nopagebreak

Le potentiel scalaire est solution d'une \'equation de Poisson.

En effet, la conservation de la charge $q$ s'\'ecrit~: 
\begin{equation} 
\displaystyle\frac{\partial q}{\partial t} 
+\dive(\vect{j}) = 0
\end{equation} 
Pour un milieu \'electriquement neutre (\`a l'\'echelle macroscopique), on a
$\displaystyle\frac{\partial q}{\partial t}=0$ soit donc~:
\begin{equation} 
\dive(\vect{j}) = 0
\end{equation} 
C'est-\`a-dire, avec la loi d'Ohm (\ref{Elec_Elbase_ohm_eq}), 
\begin{equation} \label{Elec_Elbase_div_sigma_e_eq}
\dive(\sigma\,\vect{E}) = 0
\end{equation} 
Avec (\ref{Elec_Elbase_e_eq}), on obtient donc une \'equation permettant de
calculer le potentiel scalaire~:
\begin{equation} 
{\color{red}\dive(\sigma\,\grad{P_R}) = 0}
\end{equation} 

\underline{Champ magn\'etique}
\nopagebreak

Le champ magn\'etique s'obtient \`a partir d'un potentiel vecteur. 

En effet, la loi d'Amp\`ere s'\'ecrit~:
\begin{equation}
\displaystyle\frac{\partial\vect{D}}{\partial t}-\rot\vect{H}=-\vect{j}
\end{equation}  
Sous les hypoth\`eses indiqu\'ees pr\'ec\'edemment, on \'ecrit~: 
\begin{equation}
\displaystyle\varepsilon_0\,\mu_0\,\frac{\partial\vect{E}}{\partial t}-\rot\vect{B}=-\mu_0\vect{j}
\end{equation}  
Avec une hypoth\`ese quasi-stationnaire, il reste~:
\begin{equation}\label{Elec_Elbase_rot_b_eq}
\rot\vect{B}=\mu_0\vect{j}
\end{equation}  
De plus, la conservation du flux magn\'etique s'\'ecrit\footnote{Prendre la
divergence de la loi de Faraday, avec $\dive(\rot\vect{E})=0$ (par analyse
vectorielle) donne $\dive\vect{B} = \text{cst}$.}~: 
\begin{equation} 
\dive\,\vect{B} = 0
\end{equation} 
et on peut donc postuler l'existence d'un potentiel vecteur $\vect{A}$ tel que~:
\begin{equation}\label{Elec_Elbase_b_eq} 
{\color{red}\vect{B} = \rot{\vect{A}}}
\end{equation}
  
\underline{Potentiel vecteur}
\nopagebreak

Le potentiel vecteur est solution d'une \'equation de Poisson.

En prenant le rotationnel de (\ref{Elec_Elbase_b_eq}) et avec (\ref{Elec_Elbase_rot_b_eq}), on obtient~:
\begin{equation}
-\rot(\rot{\vect{A}}) = -\mu_0\vect{j}
\end{equation}
Avec la relation donnant le Laplacien\footnote{En
coordonn\'ees cart\'esiennes, le 
Laplacien du vecteur $\vect{a}$ est le vecteur dont les
composantes sont \'egales au Laplacien de chacune des composantes de $\vect{a}$.}
d'un vecteur $\dive(\ggrad\vect{a})=\grad(\dive{\vect{a}})-\rot(\rot\vect{a})$ et sous
la contrainte\footnote{La condition $\dive\vect{A}=0$, dite ``jauge de
Coulomb'', est n\'ecessaire pour assurer
l'unicit\'e du potentiel vecteur.} 
que $\dive\vect{A}=0$, on obtient finalement une \'equation
permettant de calculer le potentiel vecteur~:
\begin{equation}
{\color{red}\dive\,(\ggrad{\vect{A}}) = -\mu_0\vect{j}}
\end{equation}



\subsection{Effet Joule}

\subsubsection{Introduction}

Pour les \'etudes Joule, on calcule, 
\`a un pas de temps donn\'e~:
\begin{itemize}
\item la vitesse $\vect{u}$, la pression $P$, la variable \'energ\'etique
enthalpie $h$ (et les grandeurs turbulentes \'eventuelles), 
\item un potentiel scalaire r\'eel $P_R$, 
\item et, si le courant n'est ni continu, ni alternatif 
monophas\'e, un potentiel scalaire imaginaire $P_I$. 
\end{itemize}

\bigskip
Le gradient du potentiel permet d'obtenir  le champ \'electrique $\vect{E}$ et 
la densit\'e de courant $\vect{j}$ (partie r\'eelle et, \'eventuellement, partie
imaginaire). Le champ \'electrique et la
densit\'e de courant sont utilis\'es pour calculer le 
terme source d'effet Joule qui intervient dans l'\'equation de l'enthalpie. 

{\bf La puissance instantan\'ee dissip\'ee par effet Joule} est \'egale 
au produit instantan\'e $\vect{j}\cdot\vect{E}$. 
Dans le cas g\'en\'eral, $\vect{j}$ et $\vect{E}$ sont des signaux alternatifs 
($\vect{j}=\vect{|j|}cos(\omega\,t+\phi_j)$ et 
$\vect{E}=\vect{|E|}cos(\omega\,t+\phi_E)$) que l'on peut repr\'esenter 
par des complexes  
($\vect{j}=\vect{|j|}\,e^{i\,(\omega\,t+\phi_j)}$  et 
$\vect{E}=\vect{|E|}\,e^{i\,(\omega\,t+\phi_E)}$).
La puissance instantan\'ee s'\'ecrit alors  
$(\vect{|j|}\cdot\vect{|E|})cos(\omega\,t+\phi_j)cos(\omega\,t+\phi_E)$. 

\begin{itemize}

\item {\bf En courant continu} ($\omega=\phi_j=\phi_E=0$), 
la puissance se calcule donc simplement comme 
le produit scalaire $P_J=\vect{|j|}\cdot\vect{|E|}$. 
Le calcul de la puissance 
dissip\'ee par effet Joule ne pose donc pas de probl\`eme particulier
car les variables densit\'e de courant et champ \'electrique r\'esolues 
par \CS\ sont pr\'ecis\'ement $\vect{|j|}$ et $\vect{|E|}$ (les variables sont r\'eelles). 

\item {\bf En courant alternatif}, la periode du courant est beaucoup plus petite que 
les \'echelles de temps des ph\'enom\`enes thermohydrauliques pris en compte. 
Il n'est donc pas utile de disposer de la puissance instantan\'ee dissip\'ee 
par effet Joule~: la moyenne sur une p\'eriode est suffisante et elle 
s'\'ecrit\footnote{L'int\'egrale de $cos^2 x$ sur un intervalle de longueur
$2\,\pi$ est $\pi$.}~:
% eh oui, car l'integrale de cos^2+sin^2 (qui vaut 1), c'est 2 \pi !! 
$P_J=\frac{1}{2}(\vect{|j|}\cdot\vect{|E|})cos(\phi_j-\phi_E)$. Cette formule 
peut \'egalement s'\'ecrire de mani\`ere \'equivalente sous forme complexe~: 
$P_J=\frac{1}{2}\vect{j}\cdot\vect{E}^*$, o\`u $\vect{E}^*$ est le complexe 
conjugu\'e de $\vect{E}$. 

  \begin{itemize}
  \item En courant alternatif monophas\'e ($\phi_j=\phi_E$), en particulier, 
la formule donnant la puissance se simplifie sous la forme 
$P_J=\frac{1}{2}(\vect{|j|}\cdot\vect{|E|})$, ou encore~:
$P_J=\frac{1}{\sqrt{2}}\vect{|j|}\cdot\frac{1}{\sqrt{2}}\vect{|E|}$. 
Il s'agit donc du produit des valeurs efficaces. Or, les variables r\'esolues
par \CS\ en courant alternatif monophas\'e sont pr\'ecis\'ement les 
valeurs efficaces (valeurs que l'on 
d\'enomme abusivement "valeurs r\'eelles" dans le code source).

  \item En courant alternatif non monophas\'e (triphas\'e, en particulier), 
la formule donnant la puissance est utilis\'ee directement sous la forme  
$P_J=\frac{1}{2}\vect{j}\cdot\vect{E}^*$. 
On utilise pour la calculer les variables r\'esolues qui sont  
la partie r\'eelle et la partie imaginaire de $\vect{j}$ et $\vect{E}$.
 
  \end{itemize}

\item {\bf En conclusion}, 

  \begin{itemize}
  \item en continu, les variables r\'esolues 
$\vect{j}_{Res}$ et $\vect{E}_{Res}$ 
sont les variables r\'eelles continues 
et la puissance se calcule par la formule suivante~: 
$P_J=\vect{j}_{Res}\cdot\vect{E}_{Res}$
  \item en alternatif monophas\'e, les variables r\'esolues 
$\vect{j}_{Res}$ et $\vect{E}_{Res}$ 
sont les valeurs efficaces
et la puissance se calcule par la formule suivante~: 
$P_J=\vect{j}_{Res}\cdot\vect{E}_{Res}$
  \item en alternatif non monophas\'e, les variables r\'esolues  
$\vect{j}_{Res,R}$, $\vect{j}_{Res,I}$ et $\vect{E}_{Res,R}$, $\vect{E}_{Res,I}$ 
sont la partie r\'eelle et la partie imaginaire de $\vect{j}$ et $\vect{E}$, 
et la puissance se calcule par la formule suivante~: 
$P_J=\frac{1}{2}(\vect{j}_{Res,R}\cdot\vect{E}_{Res,R}-\vect{j}_{Res,I}\vect{E}_{Res,I})$ 
  \end{itemize}

\end{itemize}


{\bf Le potentiel imaginaire n'est donc utilis\'e dans le code que lorsque 
le courant est alternatif et non monophas\'e.}
En particulier, le potentiel imaginaire n'est pas utilis\'e lorsque le courant est
continu ou alternatif monophas\'e. 
En effet, la partie imaginaire n'est introduite en compl\'ement de la partie
r\'eelle que dans le cas o\`u il est n\'ecessaire de disposer de deux grandeurs
pour d\'efinir le potentiel, c'est-\`a-dire lorsqu'il importe de conna\^itre son
amplitude et sa phase. 
En courant continu, on n'a naturellement besoin que d'une seule information. En 
alternatif monophas\'e, la valeur de la phase importe peu 
(on ne travaille pas sur des grandeurs \'electriques instantan\'ees)~: 
il suffit de conna\^itre l'amplitude du potentiel et il est donc inutile  
d'introduire une variable imaginaire.


{\bf La variable d\'enomm\'ee ``potentiel r\'eel'', $P_R$, repr\'esente une 
valeur efficace
si le courant est monophas\'e et une partie r\'eelle sinon.}
De mani\`ere plus explicite, pour un potentiel physique alternatif sinuso\"idal
$Pp$, de valeur maximale not\'ee $Pp_\text{max}$, de phase not\'ee $\phi$, la
variable $P_R$ repr\'esente $\frac{1}{\sqrt{2}}\,Pp_\text{max}$ en
monophas\'e et $Pp_\text{max}\,cos\phi$ sinon. En courant continu, $P_R$
repr\'esente naturellement le potentiel (r\'eel, continu). 
{\bf Il est donc indispensable de pr\^eter une attention particuli\`ere aux
valeurs de potentiel impos\'ees aux limites} (facteur $\frac{1}{\sqrt{2}}$ ou
$cos\phi$). 

\subsubsection{\'Equations continues}

{\bf Syst\`eme d'\'equations} 
\nopagebreak

Les \'equations continues qui sont r\'esolues sont les suivantes~: 
\begin{equation}
\left\{\begin{array}{l}
{\color{blue}\dive(\rho \vect{u}) = 0}\\
{\color{blue}\displaystyle\frac{\partial}{\partial t}(\rho \vect{u}) 
+\dive(\rho\, \vect{u} \otimes \vect{u})
=\dive(\tens{\sigma}) + \vect{TS}}  \\ 
{\color{blue}\displaystyle\frac{\partial}{\partial t}(\rho h) 
+\dive(\rho\, \vect{u} h)
=\Phi_v -
\dive{\left(\left(\frac{\lambda}{C_p}+\frac{\mu_t}{\sigma_t}\right)\grad{h}\right)}
+ {\color{red}P_J}}\\
{\color{red}\dive(\sigma\,\grad{P_R})=0}\\
{\color{red}\dive(\sigma\,\grad{P_I})=0}\text{\ \ \ en alternatif non monophas\'e uniquement}\\
\end{array}\right.
\end{equation}

avec, en continu ou alternatif monophas\'e~:
\begin{equation}
\left\{\begin{array}{l}
{\color{red}P_J=\vect{j}\cdot\vect{E}} \\
{\color{red}\vect{E}=-\grad{P_R}}\\
{\color{red}\vect{j}=\sigma\vect{E}}\\
\end{array}\right.
\end{equation}

et, en alternatif non monophas\'e (avec $i^2=-1$)~:
\begin{equation}
\left\{\begin{array}{l}
{\color{red}\displaystyle P_J=\frac{1}{2}\,\vect{j}\cdot\vect{E}^*}\\
{\color{red}\vect{E}=-\grad{(P_R+i\,P_I)}}\\
{\color{red}\vect{j}=\sigma\vect{E}}\\
\end{array}\right.
\end{equation}

\vspace*{0,5cm}
{\bf \'Equation de la masse} 
\nopagebreak

C'est l'\'equation r\'esolue en standard par \CS\ (contrainte
stationnaire d'incompressibilit\'e). Elle n'a pas de traitement particulier dans
le cadre du module pr\'esent. Un terme source de  masse peut \^etre pris en
compte au second membre si l'utilisateur le souhaite. Pour simplifier
l'expos\'e, 
le terme source sera suppos\'e nul ici, dans la mesure o\`u il n'est pas
sp\'ecifique au module \'electrique. 

\vspace*{0,5cm}
{\bf \'Equation de la quantit\'e de mouvement} 
\nopagebreak

C'est l'\'equation r\'esolue en standard par \CS\ (les forces de Laplace 
($\vect{j} \times \vect{B}$) sont suppos\'ees n\'egligeables). 

\vspace*{0,5cm}
{\bf \'Equation de l'enthalpie} 
\nopagebreak

On l'\'etablit comme dans le cas des arcs \'electriques\footnote{\`A ceci pr\`es
que la puissance des
forces de Laplace n'appara\^it pas du tout, au lieu de dispara\^itre lorsque
l'on soustrait l'\'equation de l'\'energie cin\'etique \`a celle de l'\'energie
totale.} \`a partir de l'�quation de
l'\'energie apr\`es plusieurs approximations utilis\'ees en standard dans \CS\ et en
prenant en compte le terme d'effet Joule li\'e \`a l'\'energie
\'electromagn\'etique. 

Par rapport \`a l'\'equation utilis\'ee pour les \'etudes d'arc \'electrique, 
seule l'expression de l'effet Joule diff\`ere lorsque le courant est
alternatif non monophas\'e.

\vspace*{0,5cm}
{\bf \'Equations \'electromagn\'etiques} 
\nopagebreak

Elles sont obtenues comme indiqu\'e dans la partie relative aux arcs
\'electriques, mais on ne conserve que les relations associ\'ees \`a la
densit\'e de courant, au champ \'electrique et au potentiel dont il d\'erive.




