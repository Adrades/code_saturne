%                      Code_Saturne version 1.3
%                      ------------------------
%
%     This file is part of the Code_Saturne Kernel, element of the
%     Code_Saturne CFD tool.
% 
%     Copyright (C) 1998-2007 EDF S.A., France
%
%     contact: saturne-support@edf.fr
% 
%     The Code_Saturne Kernel is free software; you can redistribute it
%     and/or modify it under the terms of the GNU General Public License
%     as published by the Free Software Foundation; either version 2 of
%     the License, or (at your option) any later version.
% 
%     The Code_Saturne Kernel is distributed in the hope that it will be
%     useful, but WITHOUT ANY WARRANTY; without even the implied warranty
%     of MERCHANTABILITY or FITNESS FOR A PARTICULAR PURPOSE.  See the
%     GNU General Public License for more details.
% 
%     You should have received a copy of the GNU General Public License
%     along with the Code_Saturne Kernel; if not, write to the
%     Free Software Foundation, Inc.,
%     51 Franklin St, Fifth Floor,
%     Boston, MA  02110-1301  USA
%
%-----------------------------------------------------------------------
%

%%%%%%%%%%%%%%%%%%%%%%%%%%%%%%%%%%
%%%%%%%%%%%%%%%%%%%%%%%%%%%%%%%%%%
\section{Mise en \oe uvre}
%%%%%%%%%%%%%%%%%%%%%%%%%%%%%%%%%%
%%%%%%%%%%%%%%%%%%%%%%%%%%%%%%%%%%

Le module compressible est une ``physique particuli\`ere'' activ\'ee lorsque le
mot-cl\'e \var{IPPMOD(ICOMPF)} est positif ou nul.
 
Dans ce qui suit, on pr\'ecise les inconnues et les propri\'et\'es
principales utilis\'ees dans le module. 
On fournit \'egalement un arbre d'appel simplifi\'e des sous-programmes du
module~: initialisation avec \fort{initi1} puis (\fort{iniva0} et) \fort{inivar} et
enfin, boucle en temps avec \fort{tridim}.  


\subsection{Inconnues et propri\'et\'es} 

Les \var{NSCAPP} inconnues scalaires associ\'ees \`a la physique
particuli\`ere sont d\'efinies dans \fort{cfvarp} dans l'ordre
suivant~:
\begin{itemize} 
\item la masse volumique \var{RTP(*,ISCA(IRHO(IPHAS)))}, 
\item l'�nergie totale   \var{RTP(*,ISCA(IENERG(IPHAS)))}, 
\item la temp�rature     \var{RTP(*,ISCA(ITEMPK(IPHAS)))} 
\end{itemize}

On souligne que la temp\'erature est d\'efinie en tant que variable ``\var{RTP}'' et
non pas en tant que propri\'et\'e physique ``\var{PROPCE}''. Ce choix a \'et\'e
motiv\'e par la volont\'e de simplifier la gestion des conditions aux limites,
au prix cependant d'un encombrement m\'emoire l\'eg\`erement sup\'erieur (une
grandeur \var{RTP} consomme plus qu'une grandeur \var{PROPCE}). 

La pression et la vitesse sont classiquement associ\'ees aux tableaux suivants~: 
\begin{itemize} 
\item pression~: \var{RTP(*,IPR(IPHAS))} 
\item vitesse~: \var{RTP(*,IU(IPHAS))}, \var{RTP(*,IV(IPHAS))}, \var{RTP(*,IW(IPHAS))}. 
\end{itemize}


\bigskip
Outre les propri\'et\'es associ\'ees en standard aux variables 
identifi\'ees ci-dessus, le
tableau \var{PROPCE} contient \'egalement~:
 \begin{itemize} 
\item la chaleur massique � volume constant $C_v$, stock\'ee dans
\var{PROPCE(*,IPPROC(ICV(IPHAS)))},  
      si  l'utilisateur a indiqu� dans \fort{uscfth} qu'elle \'etait variable.  
\item la viscosit� en volume \var{PROPCE(*,IPPROC(IVISCV(IPHAS)))}
      si  l'utilisateur a indiqu� dans \fort{uscfx2} qu'elle \'etait variable.
\end{itemize}


\bigskip
Pour la gestion des conditions aux limites et en particulier pour le calcul du 
flux convectif par le sch\'ema de Rusanov 
aux entr\'ees et sorties (hormis en sortie supersonique), on
dispose des tableaux suivants dans  \var{PROPFB}~:
\begin{itemize} 
\item flux convectif de quantit\'e de mouvement au bord pour les trois
composantes dans les tableaux 
\var{PROPFB(*,IPPROB(IFBRHU(IPHAS)))} (composante $x$), 
\var{PROPFB(*,IPPROB(IFBRHV(IPHAS)))} (composante $y$) et 
\var{PROPFB(*,IPPROB(IFBRHW(IPHAS)))} (composante $z$)
\item flux convectif d'\'energie au bord 
\var{PROPFB(*,IPPROB(IFBENE(IPHAS)))}
\end{itemize}
et on dispose \'egalement dans \var{IA}~: 
\begin{itemize} 
\item d'un tableau d'entiers dont la premi\`ere ``case'' est \var{IA(IIFBRU)}, dimensionn\'e au nombre de faces de bord 
et permettant de rep\'erer les faces de bord pour lesquelles on calcule 
le flux convectif par le sch\'ema de Rusanov,
\item d'un tableau  d'entiers dont la premi\`ere ``case'' est \var{IA(IIFBET)}, dimensionn\'e au nombre de faces de bord 
et permettant de rep\'erer les faces de paroi \`a temp\'erature ou \`a flux 
thermique impos\'e.
\end{itemize}


 
\newpage

\subsection{Arbre d'appel simplifi\'e} 
\nopagebreak
\begin{table}[h!]
\begin{center}
\begin{tabular}{lllllp{8cm}} 
\fort{usini1} 	& 		&		&		&
	& Initialisation des mots-cl\'es utilisateur g\'en\'eraux et positionnement des variables\\
		&\fort{usppmo} 	&		&		&
	& D\'efinition du module ``physique particuli\`ere'' employ\'e\\
		&\fort{varpos} 	&		&		&
	& Positionnement des variables \\
		&	 	& \fort{pplecd} &		&
	& Branchement des physiques particuli\`eres pour la lecture du fichier de donn\'ees \'eventuel \\
		&	 	& \fort{ppvarp} &		&
	& Branchement des physiques particuli\`eres pour le positionnement des inconnues \\
		&	 	& 		& \fort{cfvarp} &
	& Positionnement des inconnues sp\'ecifiques au module compressible \\
		&	 	& 		&               & \fort{uscfth}
	& Appel� avec ICCFTH=-1, pour indiquer que $C_p$ et $C_v$ sont constants ou variables\\
		&	 	& 		&               & \fort{uscfx2}
	& Conductivit� thermique mol\'eculaire constante ou variable et viscosit� en volume
           constante ou variable (ainsi que leur valeur, si elles sont constantes)\\
		&	 	& \fort{ppprop} &		&
	& Branchement des physiques particuli\`eres pour le positionnement des propri\'et\'es\\
		&	 	& 		& \fort{cfprop} &
	& Positionnement des propri\'et\'es sp\'ecifiques au module compressible \\
%
\fort{ppini1} 	&		&		&		&
	& Branchement des physiques particuli\`eres pour l'initialisation des
mots-cl\'es sp\'ecifiques \\
		&\fort{cfini1} 	&		&		&
	& Initialisation des mots-cl\'es sp\'ecifiques au module compressible\\
		&\fort{uscfi1} 	&		&		&
	& Initialisation des mots-cl\'es utilisateur sp\'ecifiques au module compressible\\
\end{tabular}
\caption{Sous-programme \fort{initi1}~: initialisation des mots-cl\'es et
positionnement des variables}
\end{center}
\end{table}

\newpage

\begin{table}[h!]
\begin{center}
\begin{tabular}{llllp{10cm}} 
\fort{ppiniv} 	&		&		&
	& Branchement des physiques particuli\`eres pour l'initialisation des variables \\
		& \fort{cfiniv} &		&
	& Initialisation des variables sp\'ecifiques au module compressible \\
	 	& 		& \fort{memcfv} &
	& R\'eservation de tableaux de travail locaux   \\
	 	& 		& \fort{uscfth} &
	& Initialisation des variables par d�faut (en calcul suite~: seulement
$C_v$~; si le calcul n'est pas une suite~: $C_v$, la masse volumique et l'\'energie) \\
	 	& 		& \fort{uscfxi} &
	& Initialisation des variables par l'utilisateur (seulement si le calcul
n'est pas une suite)  \\
\end{tabular}
\caption{Sous-programme \fort{inivar}~: initialisation des variables}
\end{center}
\end{table}


\begin{table}[h!]
\begin{center}
\begin{tabular}{llllp{10cm}} 
\fort{phyvar} 	&		&		&
	& Calcul des propri\'et\'es physiques variables \\
		& \fort{ppphyv} &		&
	& Branchement des physiques particuli\`eres pour le calcul des 
		propri\'et\'es physiques variables \\
		&               & \fort{cfphyv} &
	& Calcul des propri\'et\'es physiques variables pour le module
		compressible \\
	 	& 		&               & \fort{uscfpv}
	& Calcul par l'utilisateur des propri\'et\'es physiques variables pour
		le module
		compressible ($C_v$ est calcul\'e dans \fort{uscfth} qui est
		appel\'e par \fort{uscfpv}) \\
\end{tabular}
\caption{Sous-programme \fort{tridim}~: partie 1 (propri\'et\'es physiques)}
\end{center}
\end{table}

\newpage

\begin{table}[h!]
\begin{center}
\begin{tabular}{llllp{10cm}} 
\fort{dttvar} 	&		&		&
	& Calcul du pas de temps variable  \\
		& \fort{cfdttv} &		&
	& Calcul de la contrainte li�e au CFL en compressible \\
		&    		&\fort{memcft} 	&
	& Gestion de la m\'emoire pour le calcul de la contrainte en CFL \\
		&    		&\fort{cfmsfl} 	&
	& Calcul du flux associ\'e \`a la contrainte en CFL \\

\fort{precli} 	& 		 &		&
	& Initialisation des tableaux avant calcul des conditions aux 
		limites (\var{IITYPF}, \var{ICODCL}, \var{RCODCL})\\
		& \fort{ppprcl} &		&
	& Initialisations sp\'ecifiques aux diff\'erentes physiques 
		particuli\`eres avant calcul des conditions aux limites 
		(pour le module compressible~: \var{IZFPPP}, \var{IA(IIFBRU)}, 
		\var{IA(IIFBET)}, \var{RCODCL}, flux convectifs pour la 
		quantit\'e de mouvement et l'\'energie)\\

\fort{ppclim} 	& 		 &		&
	& Branchement des physiques particuli\`eres pour les conditions aux limites (en lieu et place de \fort{usclim})\\
		& \fort{uscfcl} &		&
	& Intervention de l'utilisateur pour les conditions aux limites (en lieu
		et place de \fort{usclim}, m\^eme pour les variables qui ne sont
		pas sp\'ecifiques au module compressible) \\

\fort{condli} 	& 		 &		&
	& Traitement des conditions aux limites\\
		& \fort{pptycl} &		&
	& Branchement des physiques particuli\`eres pour le traitement des conditions aux limites \\
		& 		&\fort{cfxtcl} 	&
	& Traitement des conditions aux limites pour le compressible \\
		& 		&		&\fort{uscfth} 
	& Calculs de thermodynamique pour le calcul des conditions aux limites \\
		& 		&		&\fort{cfrusb} 
	& Flux de Rusanov (entr\'ees ou sorties sauf sortie supersonique) \\
\end{tabular}
\caption{Sous-programme \fort{tridim}~: partie 2 (pas de temps variable et conditions 
                                                  aux limites)}
\end{center}
\end{table}

\newpage

\begin{table}[h!]
\begin{center}
\begin{tabular}{llllp{10cm}} 
\fort{memcfm}	& 		 &		&
	& Gestion de la m\'emoire pour la r\'esolution de l'\'etape ``acoustique'' \\
\fort{cfmsvl} 	& 		 &		&
	& R\'esolution de l'�tape ``acoustique'' \\
 	        & \fort{cfmsfl}  &		&
	& Calcul du "flux de masse" aux faces 
		(not\'e $\rho\,\vect{w}\cdot\vect{n}\,S$ dans la documentation
		du sous-programme \fort{cfmsvl}) \\
 	        & 		 & \fort{cfdivs}&
	& Calcul du terme en divergence du tenseur des contraintes visqueuses
          	(trois appels), �ventuellement \\
 	        & 		 &		&
	& Apr\`es \fort{cfmsfl}, on impose le flux de masse aux faces de bord  
		\`a partir des conditions aux limites \\
 	        & \fort{cfmsvs}  &		&
	& Calcul de la "viscosit�" aux faces 
		(not\'ee $\Delta\,t\,c^2\frac{S}{d}$ dans la documentation
		du sous-programme \fort{cfmsvl}) \\
 	        & 		 &		&
	& Apr\`es \fort{cfmsvs}, on annule la viscosit\'e aux faces de bord  
		pour que le flux de masse soit bien celui souhait\'e \\
 	        & \fort{codits}  &		&
	& R\'esolution du syst\`eme portant sur la masse volumique \\
 	        & \fort{clpsca}  &		&
	& Impression des bornes et clipping \'eventuel (pas de clipping en standard)  \\
 	        & \fort{uscfth}  &		&
	& Gestion  \'eventuelle des bornes par l'utilisateur  \\
 	        & \fort{cfbsc3}  &		&
	& Calcul du flux de masse acoustique aux faces 
		(not\'e $\vect{Q}_{ac}\cdot\vect{n}$ dans la documentation
		du sous-programme \fort{cfmsvl}) \\
 	        & \fort{uscfth}  &		&
	& Actualisation de la pression, �ventuellement  \\
\fort{cfqdmv} 	& 		 &		&
	& R\'esolution de la quantit� de mouvement\\
		& \fort{cfcdts}	 &		&
	& R�solution du syst\`eme\\
		& 		 & \fort{cfbsc2}&
	& Calcul des termes de convection et de diffusion au second membre\\
\end{tabular}
\caption{Sous-programme \fort{tridim}~: partie 3 (Navier-Stokes)}
\end{center}
\end{table}

\newpage

\begin{table}[h!]
\begin{center}
\begin{tabular}{llllp{10cm}} 
\fort{scalai}  	& 		 &		&
	& R\'esolution des \'equations sur les scalaires  \\
		& \fort{cfener}	 &		& 
	& R\'esolution de l'�quation sur l'�nergie totale\\
		& 		 & \fort{memcfe}& 
	& Gestion de la m\'emoire locale\\
		& 		 & \fort{cfdivs}& 
	& Calcul du terme en divergence du produit 
 	  ``tenseur des contraintes par vitesse''\\
		& 		 & \fort{uscfth}& 
	& Calcul de l'\'ecart  ``\'energie interne - $C_v\,T$'' 
		($\varepsilon_{sup}$)\\
		& 		 & \fort{cfcdts}&
	& R�solution du syst\`eme\\
		& 		 &  		&\fort{cfbsc2}
	& Calcul des termes de convection et de diffusion au second membre\\
 	        & 		 & \fort{clpsca}&
	& Impression des bornes et clipping \'eventuel (pas de clipping en standard)  \\
 	        & 		 & \fort{uscfth}&
	& Gestion \'eventuelle des bornes par l'utilisateur  \\
 	        & 		 & \fort{uscfth}&
	& Mise \`a jour de la pression  \\
\end{tabular}
\caption{Sous-programme \fort{tridim}~: partie 4 (scalaires)}
\end{center}
\end{table}

%\newpage

Le sous-programme \fort{cfbsc3} est similaire \`a \fort{bilsc2}, mais il produit
des flux aux faces et n'est \'ecrit que pour un sch\'ema upwind, \`a l'ordre 1
en temps (ce qui est coh\'erent avec les choix faits dans l'algorithme compressible). 

Le sous-programme \fort{cfbsc2} est similaire \`a \fort{bilsc2}, mais 
n'est \'ecrit que pour un sch\'ema d'ordre 1 en
temps.
%et fait encore appara\^itre la variable IITURB au lieu de IITYTU (il
%faudrait corriger ce dernier point). 
Le sous-programme \fort{cfbsc2} permet d'effectuer un traitement 
sp\'ecifique aux faces de bord pour lesquelles on a appliqu\'e 
un sch\'ema de Rusanov pour calculer le flux convectif total. 
Ce sous-programme est appel\'e pour la r\'esolution de l'\'equation de 
la quantit\'e de mouvement et de l'\'equation de l'\'energie. 
On pourra se reporter \`a la documentation du sous-programme \fort{cfxtcl}.

Le sous-programme \fort{cfcdts} est similaire \`a \fort{codits} mais fait appel
\`a \fort{cfbsc2} et non pas \`a \fort{bilsc2}. 
Il diff\`ere de \fort{codits} par quelques autres d\'etails qui ne sont pas
g\^enants dans l'imm\'ediat~:
initialisation de PVARA et de SMBINI, 
%nombre d'it\'erations pour le second membre (NSWRSM-1 au lieu de NSWRSM), 
%mode de d\'etermination du solveur (IRESLP), 
%test de convergence sur RNORM (compar\'e \`a 0.D0 au lieu de EPZERO), 
ordre en temps (ordre 2 non pris en compte).
%Mis \`a part pour l'ordre en temps (l'algorithme
%compressible est \`a l'ordre 1), il serait bon de modifier \fort{cfcdts} pour
%qu'il soit conforme \`a  \fort{codits}. 

\newpage 
%%%%%%%%%%%%%%%%%%%%%%%%%%%%%%%%%%
%%%%%%%%%%%%%%%%%%%%%%%%%%%%%%%%%%
\section{Points \`a traiter}
%%%%%%%%%%%%%%%%%%%%%%%%%%%%%%%%%%
%%%%%%%%%%%%%%%%%%%%%%%%%%%%%%%%%%

Des actions compl\'ementaires sont identifi\'ees ci-apr\`es, dans l'ordre
d'urgence d\'ecroissante (on se reportera 
\'egalement \`a la section "Points \`a traiter" de la documentation 
des autres sous-programmes du module compressible). 

\begin{itemize}
\item Assurer la coh\'erence des sous-programmes suivants (ou, \'eventuellement,
les fusionner pour \'eviter qu'ils ne divergent)~:
	\begin{itemize}
	\item \fort{cfcdts} et \fort{codits}, 
%(actuellement pour PVARA et
%		SMBINI, mais \`a plus long terme pour \'eviter que les 
%		deux sous-programmes ne divergent),
% propose en patch 1.2.1		
%	\item \fort{cfcdts} et \fort{codits} 
%		(au moins pour PVARA, SMBINI, NSWRSM,
%		IRESLP, RNORM),
	\item \fort{cfbsc2} et \fort{bilsc2}, 
	\item \fort{cfbsc3} et \fort{bilsc2}.
	\end{itemize} 
% propose en patch 1.2.1	
%	\item Remplacer la valeur 100 par 90 pour ICCFTH dans \fort{uscfth}
%		(plus grande coh\'erence avec les choix faits dans le reste de ce
%		sous-programme). 
% propose en patch 1.2.1	
%	\item \'Eliminer \fort{memcff} qui ne sert plus. 
\item Permettre les suites de calcul incompressible/compressible et 
	compressible/incompressible.  
\item Apporter un compl\'ement de validation (exemple~: IPHYDR). 
\item Assurer la compatibilit\'e avec certaines physiques particuli\`eres, selon
	les besoins. Par exemple~: arc \'electrique, rayonnement, combustion.
\item Identifier les causes des difficult\'es rencontr\'ees sur certains cas
acad\'emiques, en particulier~:
	\begin{itemize}
	\item canal subsonique (comment s'affranchir des effets ind\'esirables
	associ\'es aux conditions d'entr\'ee et de sortie, comment r\'ealiser un
	calcul p\'eriodique, en particulier pour la temp\'erature dont le
	gradient dans la direction de l'\'ecoulement n'est pas nul, si
	les parois sont adiabatiques),
	\item cavit\'e ferm\'ee sans vitesse ni effets de gravit\'e, 
	avec temp\'erature ou flux thermique impos\'e en paroi (il pourrait
	\^etre utile d'extrapoler le gradient de pression au bord~: 
	la pression d\'epend de la temp\'erature et une simple condition de 
	Neumann homog\`ene est susceptible de cr\'eer un terme source de 
	quantit\'e de mouvement parasite),
	\item maillage non conforme (non conformit\'e dans la direction
	transverse d'un canal),
      \item ``tube � choc'' avec terme source d'�nergie.
	\end{itemize} 
\item Compl\'eter certains points de documentation, en particulier les 
	conditions aux limites thermiques pour le couplage avec Syrthes.
\item Am\'eliorer la rapidit\'e \`a faible nombre de Mach (est-il 
possible de lever la limite
actuelle sur la valeur du pas de temps~?). 
\item Enrichir, au besoin~:
	\begin{itemize}
	\item les thermodynamiques prises en compte (multiconstituant,
	gamma variable, Van der Waals...), 
	\item la gamme des conditions aux limites d'entr\'ee
	disponibles (condition \`a d\'ebit massique et d\'ebit enthalpique
	impos\'es par exemple). 
	\end{itemize}
\item Tester des variantes de l'algorithme~:
	\begin{itemize}
	\item prise en compte des termes sources de l'\'equation de la
	quantit\'e de mouvement autres que la gravit\'e dans l'\'equation de la
	masse r\'esolue lors de l'\'etape ``acoustique'' (les tests r\'ealis\'es
	avec cette variante de l'algorithme devront \^etre repris dans la 
	mesure o\`u, dans \fort{cfmsfl}, IIROM et IIROMB n'\'etaient pas 
	initialis\'es),  
	\item implicitation du terme de convection dans
	l'\'equation de la masse (\'eliminer cette possibilit\'e si 
	elle n'apporte rien),
	\item \'etape de pr\'ediction de la pression, 
	\item non reconstruction de la masse volumique pour le terme convectif 
	(actuellement, les termes convectifs sont trait\'es avec 
	d\'ecentrement amont, d'ordre 1 en espace~; 
	pour l'\'equation de la quantit\'e de mouvement et l'\'equation de 
	l'\'energie, on utilise les valeurs prises au centre des cellules 
	sans reconstruction~: c'est l'approche standard de \CS, traduite 
	dans \fort{cfbsc2}~; par contre, dans \fort{cfmsvl}, on reconstruit 
	les valeurs de la masse volumique utilis\'ees pour le terme 
	convectif~; il n'y a pas de raison d'adopter des strat\'egies 
	diff\'erentes, d'autant plus que la reconstruction de la masse 
	volumique ne permet pas de monter en ordre et augmente le risque 
	de d\'epassement des bornes physiques),
	\item mont\'ee en ordre en espace (en v\'erifier l'utilit\'e et 
	la robustesse, en particulier relativement au principe du 
	maximum pour la masse volumique), 
	\item mont\'ee en ordre en temps (en v\'erifier l'utilit\'e et 
	la robustesse).
	\end{itemize}
\item Optimiser l'encombrement m\'emoire. 
\end{itemize} 

