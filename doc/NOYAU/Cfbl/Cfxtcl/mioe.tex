%                      Code_Saturne version 1.3
%                      ------------------------
%
%     This file is part of the Code_Saturne Kernel, element of the
%     Code_Saturne CFD tool.
% 
%     Copyright (C) 1998-2007 EDF S.A., France
%
%     contact: saturne-support@edf.fr
% 
%     The Code_Saturne Kernel is free software; you can redistribute it
%     and/or modify it under the terms of the GNU General Public License
%     as published by the Free Software Foundation; either version 2 of
%     the License, or (at your option) any later version.
% 
%     The Code_Saturne Kernel is distributed in the hope that it will be
%     useful, but WITHOUT ANY WARRANTY; without even the implied warranty
%     of MERCHANTABILITY or FITNESS FOR A PARTICULAR PURPOSE.  See the
%     GNU General Public License for more details.
% 
%     You should have received a copy of the GNU General Public License
%     along with the Code_Saturne Kernel; if not, write to the
%     Free Software Foundation, Inc.,
%     51 Franklin St, Fifth Floor,
%     Boston, MA  02110-1301  USA
%
%-----------------------------------------------------------------------
%

%%%%%%%%%%%%%%%%%%%%%%%%%%%%%%%%%%
%%%%%%%%%%%%%%%%%%%%%%%%%%%%%%%%%%
\section{Mise en \oe uvre}
%%%%%%%%%%%%%%%%%%%%%%%%%%%%%%%%%%
%%%%%%%%%%%%%%%%%%%%%%%%%%%%%%%%%%
  
%=================================
\subsection{Introduction}
%=================================

Les conditions aux limites sont impos\'ees par une suite de sous-programmes, 
dans la mesure o\`u l'on a cherch\'e \`a rester coh\'erent avec la structure 
standard de \CS. 

Dans \fort{ppprcl} (appel\'e par \fort{precli}), on initialise les tableaux 
avant le calcul des conditions aux limites~:
\begin{itemize}
\item \var{IZFPPP} (num\'ero de zone, inutilis\'e, fix\'e \`a z\'ero),
\item \var{IA(IIFBRU)} (rep\'erage des faces de bord pour 
lesquelles on applique un sch\'ema de Rusanov~: initialis\'e \`a z\'ero, 
on imposera la valeur 1 dans \fort{cfrusb} pour les faces auxquelles on applique le sch\'ema 
de Rusanov) 
\item \var{IA(IIFBET)} (rep\'erage des faces de paroi \`a temp\'erature ou 
\`a flux thermique impos\'e~: initialis\'e \`a 0, on imposera la valeur 1
dans \fort{cfxtcl} lorsque la temp�rature ou le flux est impos�),
\item \var{RCODCL(*,*,1)} (initialis\'e \`a \var{-RINFIN} en pr\'evision 
du traitement des sorties r\'eentrantes pour lesquelles l'utilisateur 
aurait fourni une valeur \`a imposer en Dirichlet),
\item flux convectifs de bord pour la quantit\'e de mouvement et l'\'energie 
(initialis\'es \`a z\'ero).
\end{itemize}


\bigskip
Les types de fronti\`ere (\var{ITYPFB}) et les valeurs n\'ecessaires 
(\var{ICODCL}, \var{RCODCL}) sont impos\'es par l'utilisateur dans \fort{uscfcl}.   

On convertit ensuite ces donn\'ees dans \fort{condli} pour qu'elles 
soient directement utilisables lors du calcul des matrices et des seconds membres. 

Pour cela, \fort{cfxtcl} permet de r\'ealiser le calcul des valeurs de bord et,
pour certaines fronti\`eres, des flux convectifs. On fait appel, 
en particulier, 
\`a \fort{uscfth} (utilisation de la thermodynamique) et \`a \fort{cfrusb} 
(flux convectifs par le sch\'ema de Rusanov). Lors de ces calculs, on utilise 
\var{COEFA} et \var{COEFB} comme tableaux de travail (transmission de valeurs 
\`a \fort{uscfth} en particulier) afin de renseigner \var{ICODCL} et
\var{RCODCL}. 
Apr\`es \fort{cfxtcl}, 
le sous-programme \fort{typecl} compl\`ete quelques valeurs par d\'efaut 
de \var{ICODCL} et de \var{RCODCL}, en particulier pour les scalaires passifs.   

Apr\`es \fort{cfxtcl} et \fort{typecl}, les tableaux \var{ICODCL} et \var{RCODCL} 
sont complets. Ils sont utilis\'es dans la suite de \fort{condli} et en particulier 
dans \fort{clptur} pour construire les tableaux \var{COEFA} et \var{COEFB} 
(pour l'\'energie, on dispose de deux couples (\var{COEFA}, \var{COEFB}) afin de 
traiter les parois). 
  
On pr\'esente ci-apr\`es les points dont l'implantation diff\`ere 
de l'approche standard. Il s'agit de 
l'utilisation d'un sch\'ema de Rusanov pour le calcul des flux convectifs 
en entr\'ee et sortie (hormis sortie supersonique) 
et du mode de calcul des flux diffusifs d'\'energie en paroi. 
On insiste en particulier sur l'impact des conditions aux limites 
sur la construction des seconds membres de l'\'equation de la quantit\'e 
de mouvement et de l'\'equation de l'\'energie (\fort{cfqdmv} et \fort{cfener}).

%=================================
\subsection{Flux de Rusanov pour le calcul des flux convectifs en entr\'ee et sortie}
%=================================

Le sch\'ema de Rusanov est utilis\'e pour calculer des flux convectifs de bord  
(masse, quantit\'e de mouvement et \'energie) aux entr\'ees et des sorties 
de type IESICF, ISOPCF, IERUCF, IEQHCF. 

La gestion des conditions aux limites est diff\'erente de celle adopt\'ee
classiquement dans \CS, bien que l'on se soit efforc\'e de s'y conformer le
mieux possible. 

En volumes finis, il faut disposer de conditions aux
limites pour trois utilisations principales au moins~:
 	\begin{itemize}
	\item imposer les flux de convection,
	\item imposer les flux de diffusion,
	\item calculer les gradients pour les reconstructions. 
	\end{itemize}
Dans l'approche standard de \CS, les conditions aux limites sont d\'efinies par
variable et non pas par terme discret\footnote{Par exemple, pour un scalaire  
convect\'e et diffus\'e, on d\'efinit une valeur de bord unique {\it pour le scalaire}
et non pas une valeur de bord pour le {\it flux convectif} et une valeur de bord
pour le {\it flux diffusif}.}. On dispose donc, {\it pour chaque variable}, 
d'une valeur de bord dont devront \^etre d\'eduits les flux de
convection, les flux de diffusion et les gradients\footnote{N\'eanmoins, pour
certaines variables comme la vitesse par exemple, \CS\ dispose de deux valeurs
de bord (et non pas d'une seule) afin de pouvoir imposer de mani\`ere
ind\'ependante le gradient normal et le flux de diffusion.}. 
Ici, avec l'utilisation d'un sch\'ema de
Rusanov, dans lequel le flux convectif est trait\'e dans son ensemble, 
il est imp\'eratif 
de disposer d'un moyen d'imposer directement sa valeur au bord\footnote{Il
serait possible de calculer une valeur de bord fictive des variables d'\'etat qui
permette de retrouver le flux convectif calcul\'e par le sch\'ema de Rusanov, 
mais cette valeur ne permettrait pas d'obtenir
un flux de diffusion et un gradient satisfaisants.}. 

Le flux convectif calcul\'e par le sch\'ema de Rusanov 
sera ajout\'e directement au second membre
des \'equations de masse, de quantit\'e de mouvement et d'\'energie. Comme ce
flux contient, outre la contribution des termes convectifs ``usuels'' 
($\dive(\vect{Q})$, $\dive(\vect{u}\otimes\vect{Q})$ et 
$\dive(\vect{Q}\,e)$), celle des termes en $\grad\,P$ (quantit\'e de
mouvement) et $\dive(\vect{Q}\,\frac{P}{\rho})$ 
(\'energie), il faut veiller \`a ne pas
ajouter une seconde fois les termes de bord issus de  $\grad\,P$ et de
$\dive(\vect{Q}\,\frac{P}{\rho})$
au second membre des \'equations de quantit\'e de
mouvement et d'\'energie. 


Pour la masse, le flux convectif calcul\'e par le sch\'ema de Rusanov 
d\'efinit simplement le flux de masse au bord
(\var{PROPFB(IFAC,IPPROB(IFLUMA(ISCA(IENERG(IPHAS)))))}). 

Pour la quantit\'e de mouvement, le flux convectif calcul\'e par le sch\'ema de
Rusanov  est stock\'e dans les tableaux  
\var{PROPFB(IFAC,IPPROB(IFBRHU(IPHAS)))}, \var{PROPFB(IFAC,IPPROB(IFBRHV(IPHAS)))} et
\var{PROPFB(IFAC,IPPROB(IFBRHW(IPHAS)))}. Il est ensuite ajout\'e au second membre de
l'\'equation directement dans \fort{cfqdmv} (boucle sur les faces de bord). 
Comme ce flux contient la contribution du terme convectif usuel
$\divv(\vect{u}\otimes\vect{Q})$, il ne faut pas l'ajouter dans 
le sous-programme \fort{cfbsc2}.
De plus, le flux convectif calcul\'e par le sch\'ema de Rusanov 
contient la contribution du
gradient de pression. Or, le gradient de pression est calcul\'e dans
\fort{cfqdmv} au moyen de \fort{grdcel} et ajout\'e au second membre 
sous forme de contribution volumique (par cellule)~: il faut donc retirer 
la contribution des faces de bord auxquelles est appliqu\'e le sch\'ema de
Rusanov, pour ne pas la compter deux fois (cette op\'eration est r\'ealis\'ee
dans \fort{cfqdmv}). 

Pour l'\'energie, le flux convectif calcul\'e par le sch\'ema de 
Rusanov est stock\'e dans le tableau  
\var{PROPFB(IFAC,IPPROB(IFBENE(IPHAS)))}. Pour les faces auxquelles n'est pas
appliqu\'e le sch\'ema de Rusanov, on ajoute la contribution
du terme de transport de pression $\dive(\vect{Q}\,\frac{P}{\rho})$ 
au second membre de l'\'equation dans \fort{cfener}  
et on compl\`ete le second membre dans \fort{cfbsc2} avec la contribution du
terme convectif usuel $\dive(\vect{Q}\,e)$. Pour les faces auxquelles est
appliqu\'e le sch\'ema de Rusanov, on ajoute directement le flux de Rusanov au second
membre de l'\'equation dans \fort{cfener}, en lieu et place de la contribution 
du terme de transport de pression et l'on prend garde de ne pas
comptabiliser une seconde fois le flux convectif usuel 
$\divv(\vect{Q}\,e)$ dans le sous-programme \fort{cfbsc2}. 

C'est l'indicateur \var{IA(IIFBRU)}  
(renseign\'e dans \fort{cfrusb}) qui permet, dans \fort{cfbsc2}, 
\fort{cfqdmv} et \fort{cfener}, 
de rep\'erer les faces de bord pour lesquelles on a calcul\'e  
un flux convectif avec le sch\'ema de Rusanov. 


%=================================
\subsection{Flux diffusif d'\'energie}
%=================================

%---------------------------------
\subsubsection{Introduction}
%---------------------------------

Une condition doit \^etre fournie sur toutes les fronti\`eres pour le calcul du 
flux diffusif d'\'energie. 

Il n'y a pas lieu de
s'\'etendre particuli\`erement sur le traitement de certaines fronti\`eres.
Ainsi, aux entr\'ees et sorties, on dispose 
d'une valeur de bord (issue de la r\'esolution du probl\`eme 
de Riemann) 
que l'on utilise dans la formule discr\`ete classique donnant le
flux\footnote{Les valeurs de $u^2$ et de $\varepsilon_{sup}$ ne sont pas
reconstruites pour le calcul du gradient au bord dans 
$\displaystyle\dive{\left(\frac{K}{C_v}\,\grad{(\frac{1}{2}\,u^2+\varepsilon_{sup})}\right)}$}. 
La situation est simple aux sym\'etries \'egalement, o\`u un flux nul est impos\'e. 

Par contre, en paroi, les conditions de temp\'erature ou de flux thermique 
impos\'e doivent \^etre trait\'ees avec plus d'attention, en particulier 
lorsqu'une couche limite turbulente est pr\'esente. 

%---------------------------------
\subsubsection{Coexistence de deux conditions de bord}
%---------------------------------

Comme indiqu\'e dans la partie "discr\'etisation", 
les conditions de temp\'erature ou de flux conductif  
impos\'e en paroi se traduisent, 
pour le flux d'\'energie, au travers du terme 
$\displaystyle\dive{\left(\frac{K}{C_v}\,\grad\,e\right)}$,  
en imposant une condition de flux nul sur le terme 
$\displaystyle-\dive{\left(\frac{K}{C_v}\,\grad{(\frac{1}{2}\,u^2+\varepsilon_{sup})}\right)}$.
Les faces IFAC 
concern\'ees sont rep\'er\'ees dans \fort{cfxtcl} par l'indicateur 
\var{IA(IIFBET+IFAC-1+(IPHAS-1)*NFABOR) = 1} (qui vaut 0 sinon, initialis\'e
dans \fort{ppprcl}).

Sur ces faces, 
on calcule une valeur de bord de l'\'energie, qui, introduite dans la 
formule g\'en\'erale de flux utilis\'ee au bord dans \CS, permettra de retouver le 
flux souhait\'e. La valeur de bord est une simple valeur num\'erique sans
signification physique et ne doit \^etre utilis\'ee que pour calculer le flux 
diffusif.  

En plus de cette valeur de bord destin\'ee \`a retrouver le
flux diffusif, il est n\'ecessaire de disposer 
d'une seconde valeur de bord de l'\'energie afin de pouvoir en calculer le
gradient.

Ainsi, comme pour la vitesse en $k-\varepsilon$, il est n\'ecessaire de 
disposer pour l'\'energie de deux couples de coefficients 
(\var{COEFA},\var{COEFB}), correspondant \`a deux valeurs de bord distinctes,
dont l'une est utilis\'ee pour le calcul du flux diffusif sp\'ecifiquement. 

%---------------------------------
\subsubsection{Calcul des \var{COEFA} et \var{COEFB} pour les faces de paroi 
\`a temp\'erature impos\'ee}
%---------------------------------

Les  faces de paroi  \var{IFAC} \`a temp�rature impos\'ee sont identif\'ees par
l'utilisateur dans \fort{uscfcl} au moyen de  l'indicateur
\var{ICODCL(IFAC,ISCA(ITEMPK(IPHAS)))=5} (noter que 
ce tableau est associ\'e \`a la temp\'erature).  

Dans \fort{cfxtcl}, on impose alors \var{ICODCL(IFAC,ISCA(IENERG(IPHAS)))=5} et 
on calcule la quantit\'e
$C_v\,T_{imp,ext}+\frac{1}{2}u^2_{I}+\varepsilon_{sup,I}$, que l'on  
stocke dans \var{RCODCL(IFAC,ISCA(IENERG(IPHAS)),1)} (on ne reconstruit pas les
valeurs de $u^2$ et $\varepsilon_{sup}$ au bord, cf. \S\ref{Cfbl_Cfxtcl_prg_a_traiter}). 

\`A partir de ces valeurs de \var{ICODCL} et \var{RCODCL}, 
on renseigne ensuite dans \fort{clptur} 
les tableaux de conditions aux limites  permettant le calcul du flux~:
\var{COEFA(*,ICLRTP(ISCA(IENERG(IPHAS)),ICOEFF))} et 
\var{COEFB(*,ICLRTP(ISCA(IENERG(IPHAS)),ICOEFF))} (noter  
l'indicateur \var{ICOEFF} qui renvoie aux coefficients d\'edi\'es au flux
diffusif).  


%---------------------------------
\subsubsection{Calcul des \var{COEFA} et \var{COEFB} pour les faces de paroi 
\`a flux thermique impos\'e}
%---------------------------------

Les  faces de paroi  \var{IFAC} \`a flux thermique 
impos\'e sont identif\'ees par
l'utilisateur dans \fort{uscfcl} au moyen de  l'indicateur
\var{ICODCL(IFAC,ISCA(ITEMPK(IPHAS)))=3} (noter que le tableau est 
associ\'e \`a la temp\'erature).  

Dans \fort{cfxtcl}, on impose alors \var{ICODCL(IFAC,ISCA(IENERG(IPHAS)))=3} et 
on transf\`ere la valeur du flux de  \var{RCODCL(IFAC,ISCA(ITEMPK(IPHAS)),3)}
\`a \var{RCODCL(IFAC,ISCA(IENERG(IPHAS)),3)}. 
  
\`A partir de ces valeurs de \var{ICODCL} et \var{RCODCL}, 
on renseigne ensuite dans \fort{condli} les tableaux de conditions aux limites
permettant le calcul du flux, 
\var{COEFA(*,ICLRTP(ISCA(IENERG(IPHAS)),ICOEFF))} et 
\var{COEFB(*,ICLRTP(ISCA(IENERG(IPHAS)),ICOEFF))} (noter  
l'indicateur \var{ICOEFF} qui renvoie aux coefficients d\'edi\'es au flux
diffusif).  

%---------------------------------
\subsubsection{Gradient de l'\'energie en paroi \`a temp\'erature ou \`a flux thermique impos\'e}
%---------------------------------

Dans les deux cas (paroi \`a temp\'erature ou \`a flux thermique impos\'e), 
on utilise les tableaux 
\var{COEFA(*,ICLRTP(ISCA(II),ICOEF))},  
\var{COEFB(*,ICLRTP(ISCA(II),ICOEF))} (noter le \var{ICOEF}) pour disposer d'une 
condition de flux nul pour l'\'energie (avec \var{II=IENERG(IPHAS)}) et 
pour la temp\'erature (avec \var{II=ITEMPK(IPHAS)}) 
si un calcul de gradient est requis. 

Un gradient est en particulier utile pour les reconstructions 
de l'\'energie sur maillage non orthogonal. 
Pour la temp\'erature, il s'agit d'une pr\'ecaution, au cas 
o\`u l'utilisateur aurait besoin d'en calculer le gradient.   

%---------------------------------
\subsubsection{Autres fronti\`eres que les parois \`a temp\'erature ou \`a flux thermique impos\'e}
%---------------------------------

Pour les fronti\`eres qui ne sont pas des parois \`a temp\'erature ou 
\`a flux thermique impos\'e, les conditions aux limites de l'\'energie et 
de la temp\'erature sont compl\'et\'ees classiquement dans \fort{condli} selon 
les choix faits dans \fort{cfxtcl} pour \var{ICODCL} et \var{RCODCL}. 

En particulier, 
dans le cas de conditions de Dirichlet sur l'\'energie (entr\'ees, sorties), les
deux jeux de conditions aux limites sont identiques (tableaux  
\var{COEFA}, \var{COEFB} avec \var{ICOEFF} et \var{ICOEF}). 

Si un flux est impos\'e pour l'\'energie totale (condition assez rare,  
l'utilisateur ne raisonnant pas,
d'ordinaire, en \'energie totale), on le stocke au moyen de   
\var{COEFA(*,ICLRTP(ISCA(IENERG(IPHAS)),ICOEFF))} et 
\var{COEFB(*,ICLRTP(ISCA(IENERG(IPHAS)),ICOEFF))} (tableaux associ\'es au flux
diffusif). Pour le gradient, une condition de flux nul est stock\'ee 
dans    
\var{COEFA(*,ICLRTP(ISCA(IENERG(IPHAS)),ICOEF))} et 
\var{COEFB(*,ICLRTP(ISCA(IENERG(IPHAS)),ICOEF))}. On peut remarquer que les deux
jeux de conditions aux limites sont identiques pour les faces de sym\'etrie. 

%---------------------------------
\subsubsection{Impact dans \fort{cfener}}
%---------------------------------

Lors de la construction des seconds membres, dans \fort{cfener}, on utilise les 
conditions aux limites stock\'ees dans les tableaux associ\'es au flux
diffusif 
\var{COEFA(*,ICLRTP(ISCA(IENERG(IPHAS)),ICOEFF))} et 
\var{COEFB(*,ICLRTP(ISCA(IENERG(IPHAS)),ICOEFF))} pour le terme de flux diffusif 
$\displaystyle\dive{\left(\frac{K}{C_v}\,\grad\,e\right)}$
en prenant soin d'annuler la contribution de bord du terme 
$\displaystyle-\dive{\left(\frac{K}{C_v}\,\grad{(\frac{1}{2}\,u^2+\varepsilon_{sup})}\right)}$
sur les faces pour lesquelles cette condition 
prend les deux termes en compte, c'est-\`a-dire sur les faces pour lesquelles 
\var{IA(IIFBET+IFAC-1+(IPHAS-1)*NFABOR) = 1}. 

Pour tous les autres termes qui requi\`erent une valeur de bord, on utilise les 
conditions aux limites que l'on a stock\'ees au moyen des deux tableaux 
\var{COEFA(*,ICLRTP(ISCA(IENERG(IPHAS)),ICOEF))} et 
\var{COEFB(*,ICLRTP(ISCA(IENERG(IPHAS)),ICOEF))}. Ces conditions sont 
donc en particulier utilis\'ees pour le calcul du gradient de l'\'energie, 
lors des reconstructions sur maillage non orthogonal. 


\newpage
%%%%%%%%%%%%%%%%%%%%%%%%%%%%%%%%%%
%%%%%%%%%%%%%%%%%%%%%%%%%%%%%%%%%%
\section{Points \`a traiter}
%%%%%%%%%%%%%%%%%%%%%%%%%%%%%%%%%%
%%%%%%%%%%%%%%%%%%%%%%%%%%%%%%%%%%
\label{Cfbl_Cfxtcl_prg_a_traiter}%
% propose en patch 1.2.1 
%Corriger \fort{ppprcl} pour que l'indicateur 
%\var{IA(IIFBET+IFAC-1+(IPHAS-1)*NFABOR)} soit 
%initialis\'e \`a 0, positionn\'e \`a 1 aux faces de paroi \`a temp\'erature 
%ou flux thermique impos\'e. Dans \fort{cfener}, lorsque l'indicateur vaudra 1,
%on ne prendra pas en compte le flux correspondant \`a
%$\displaystyle-\dive{\left(\frac{K}{C_v}\,\grad{(\frac{1}{2}\,u^2+\varepsilon_{sup})}\right)}$.

% propose en patch 1.2.1 
%Pour l'\'energie, on utilise comme diffusivit\'e turbulente la valeur  
%$\displaystyle \frac{K}{C_v}=\frac{\lambda}{C_v}+\frac{\mu_t}{\sigma_t}$. 
%Par coh\'erence avec une \'equation
%portant sur la temp\'erature, il serait plus logique d'utiliser 
%$\displaystyle \frac{K}{C_v}=\frac{\lambda}{C_v}+\frac{C_p}{C_v}\,\frac{\mu_t}{\sigma_t}$. 
%On peut temporairement utiliser le nombre de Prandtl turbulent pour prendre en compte 
%le rapport $\displaystyle\frac{C_p}{C_v}$, mais il serait
%souhaitable de corriger en ce sens le calcul de \var{W1} pour \fort{viscfa} dans
%le sous-programme \fort{cfener} et le calcul similaire de \var{HINT} dans 
%\fort{condli} et \fort{clptur} (RAS pour les conversions en couplage avec Syrthes). 

Apporter un compl\'ement de test sur une cavit\'e ferm\'ee 
sans vitesse et sans gravit\'e, avec flux de bord ou temp\'erature de bord impos\'ee. 
Il semble que le transfert d'\'energie {\it via} les termes de pression g\'en\`ere de 
fortes vitesses non physiques dans la premi\`ere maille de paroi et que la 
conduction thermique ne parvienne pas \`a \'etablir le profil de temp\'erature 
recherch\'e. Il est \'egalement possible que la condition de bord sur la pression 
g\'en\`ere une perturbation (une extrapolation pourrait se r\'ev\'eler
indispensable).   
 
Il pourrait \^etre utile de g\'en\'eraliser \`a l'incompressible l'approche
utilis\'ee en compressible pour unifier simplement le traitement 
des sorties de type 9 et 10.  

Il pourrait \^etre utile d'\'etudier plus en d\'etail l'influence de la non 
orthogonalit\'e des mailles en sortie supersonique (pas de reconstruction, 
ce qui n'est pas consistant pour les flux de diffusion). 

De m\^eme, il serait utile d'\'etudier l'influence de l'absence de 
reconstruction pour la vitesse et $\varepsilon_{sup}$ dans la relation
$\displaystyle T_{I'}=\frac{1}{C_v}\left(e_{I'}-\frac{1}{2}u^2_{i}-\varepsilon_{sup,i}\right)$ 
utilis\'ee pour les parois \`a temp\'erature impos\'ee. 

Apporter un compl\'ement de documentation pour le couplage avec Syrthes (conversion 
\'energie temp\'erature). Ce n'est pas une priorit\'e. 
 
Pour les thermodynamiques \`a $\gamma$ variable, il sera n\'ecessaire de
modifier non 
seulement \fort{uscfth} mais \'egalement \fort{cfrusb} qui doit disposer de 
$\gamma$ en argument. 

Pour les thermodynamiques \`a $C_v$ variable, il sera n\'ecessaire de
prendre en compte un terme en $\grad\,C_v$, issu des flux diffusifs, 
au second membre de l'\'equation de
l'\'energie (on pourra cependant remarquer qu'actuellement, en incompressible,
on n\'eglige le terme en $\grad\,C_p$ dans l'\'equation de l'enthalpie). 
