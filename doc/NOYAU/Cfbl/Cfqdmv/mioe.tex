%                      Code_Saturne version 1.3
%                      ------------------------
%
%     This file is part of the Code_Saturne Kernel, element of the
%     Code_Saturne CFD tool.
% 
%     Copyright (C) 1998-2007 EDF S.A., France
%
%     contact: saturne-support@edf.fr
% 
%     The Code_Saturne Kernel is free software; you can redistribute it
%     and/or modify it under the terms of the GNU General Public License
%     as published by the Free Software Foundation; either version 2 of
%     the License, or (at your option) any later version.
% 
%     The Code_Saturne Kernel is distributed in the hope that it will be
%     useful, but WITHOUT ANY WARRANTY; without even the implied warranty
%     of MERCHANTABILITY or FITNESS FOR A PARTICULAR PURPOSE.  See the
%     GNU General Public License for more details.
% 
%     You should have received a copy of the GNU General Public License
%     along with the Code_Saturne Kernel; if not, write to the
%     Free Software Foundation, Inc.,
%     51 Franklin St, Fifth Floor,
%     Boston, MA  02110-1301  USA
%
%-----------------------------------------------------------------------
%

%%%%%%%%%%%%%%%%%%%%%%%%%%%%%%%%%%
%%%%%%%%%%%%%%%%%%%%%%%%%%%%%%%%%%
\section{Mise en \oe uvre}
%%%%%%%%%%%%%%%%%%%%%%%%%%%%%%%%%%
%%%%%%%%%%%%%%%%%%%%%%%%%%%%%%%%%%
On r\'esout les trois directions d'espace du syst\`eme
(\ref{Cfbl_Cfqdmv_eq_vitesse_discrete_cfqdmv}) successivement et ind\'ependamment~:
\begin{equation}\label{Cfbl_Cfqdmv_eq_vitesse_finale_cfqdmv}
\left\{\begin{array}{l}
\displaystyle\frac{\Omega_i}{\Delta t^n}
(\rho_i^{n+1}{u_i}_{(\alpha)}^{n+1}-\rho_i^n{u_i}_{(\alpha)}^n)
+ \displaystyle\sum\limits_{j\in V(i)}
{u_{ij}}_{(\alpha)}^{n+1}(\vect{Q}_{ac}^{n+1})_{ij}\cdot\vect{S}_{ij}
- \displaystyle\sum\limits_{j\in V(i)}
\mu_{ij}^n\frac{{u_j}_{(\alpha)}^{n+1}-{u_i}_{(\alpha)}^{n+1}}{\overline{I'J'}}S_{ij}\\
\qquad\qquad\qquad\qquad= \Omega_i\rho_i^{n+1} {{f_v}_i}_{(\alpha)}
- \Omega_i{(\gradv{\widetilde{P}})_{i}}_{(\alpha)}\\ 
\qquad\qquad\qquad\qquad + \displaystyle\sum\limits_{j\in V(i)}
\left((\mu^n\ ^t\gradt{\vect{u}^n})_{ij}\cdot\vect{S}_{ij}\right)_{(\alpha)}
 + \displaystyle\sum\limits_{j\in V(i)} \left((\kappa^n-\frac{2}{3}\mu^n)
\divs{\vect{u}^n}\right)_{ij}{S_{ij}}_{(\alpha)}\\
i = 1\ldots N \qquad \text{et} \qquad (\alpha) = x,\ y,\ z\\
\end{array}\right.
\end{equation}

Chaque syst\`eme associ\'e \`a une direction est r\'esolu par une m\'ethode
d'incr\'ement et r\'esidu en utilisant une m\'ethode de Jacobi.


%%%%%%%%%%%%%%%%%%%%%%%%%%%%%%%%%%
%%%%%%%%%%%%%%%%%%%%%%%%%%%%%%%%%%
%\section{Points \`a traiter}
%%%%%%%%%%%%%%%%%%%%%%%%%%%%%%%%%%
%%%%%%%%%%%%%%%%%%%%%%%%%%%%%%%%%%

% propose en patch 1.2.1

%Compl\'eter le commentaire en ent\^ete de \fort{vissec} pour prendre en compte 
%la viscosit\'e en volume. 