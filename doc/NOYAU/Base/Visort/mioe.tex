%                      Code_Saturne version 1.3
%                      ------------------------
%
%     This file is part of the Code_Saturne Kernel, element of the
%     Code_Saturne CFD tool.
% 
%     Copyright (C) 1998-2007 EDF S.A., France
%
%     contact: saturne-support@edf.fr
% 
%     The Code_Saturne Kernel is free software; you can redistribute it
%     and/or modify it under the terms of the GNU General Public License
%     as published by the Free Software Foundation; either version 2 of
%     the License, or (at your option) any later version.
% 
%     The Code_Saturne Kernel is distributed in the hope that it will be
%     useful, but WITHOUT ANY WARRANTY; without even the implied warranty
%     of MERCHANTABILITY or FITNESS FOR A PARTICULAR PURPOSE.  See the
%     GNU General Public License for more details.
% 
%     You should have received a copy of the GNU General Public License
%     along with the Code_Saturne Kernel; if not, write to the
%     Free Software Foundation, Inc.,
%     51 Franklin St, Fifth Floor,
%     Boston, MA  02110-1301  USA
%
%-----------------------------------------------------------------------
%

%%%%%%%%%%%%%%%%%%%%%%%%%%%%%%%%%%
%%%%%%%%%%%%%%%%%%%%%%%%%%%%%%%%%%
\section{Mise en \oe uvre}
%%%%%%%%%%%%%%%%%%%%%%%%%%%%%%%%%%
%%%%%%%%%%%%%%%%%%%%%%%%%%%%%%%%%%
La viscosit� orthotrope au centre des cellules est entr�e en argument {\it via}
les variables $\var{W}_1$, $\var{W}_2$ et $\var{W}_3$. On calcule la valeur
moyenne de chaque viscosit� aux faces de fa�on arithm�tique ou
harmonique. Ensuite, on calcule la viscosit� �quivalente correspondant �
$\displaystyle (\tens{\mu}\ \underline{n}_{\,ij})\,.\,\frac{\underline{S}_{\,ij}}{\overline{I'J'}}$ pour les
faces internes et � $\displaystyle (\tens{\mu}\ \underline{n}_{\,b_{ik}})\,.\,
\frac{\underline{S}_{\,b_{ik}}}{\overline{I'F}}$ pour les faces de bord.\\

Cette \'ecriture fait intervenir les vecteurs surface stock\'es dans le tableau 
\var{SURFAC}, la norme de la surface \var{SURFN}
 et la distance alg\'ebrique \var{DIST} pour une face interne (\var{SURFBO},
\var{SURFBN} et \var{DISTBR} respectivement pour une face de bord). La valeur du
terme de diffusion r�sultant est mise dans le vecteur \var{VISCF} (\var{VISCB} aux faces de bord).\\
La variable \var{IMVISF} d�termine quel type de moyenne est utilis� pour
calculer la viscosit� dans une direction \`a la face. Si \var{IMVISF}$=0$, alors
la moyenne est arithm�tique, sinon la moyenne est harmonique).
%%%%%%%%%%%%%%%%%%%%%%%%%%%%%%%%%%
%%%%%%%%%%%%%%%%%%%%%%%%%%%%%%%%%%
\section{Points \`a traiter}
%%%%%%%%%%%%%%%%%%%%%%%%%%%%%%%%%%
%%%%%%%%%%%%%%%%%%%%%%%%%%%%%%%%%%
L'obtention des interpolations utilis�es dans le code \CS \ du paragraphe
\ref{Base_Visort_paragraphe2} est r�sum�e dans le rapport de Davroux et al\footnote{Davroux A., Archambeau F. et H�rard J.M., Tests num�riques sur
quelques m�thodes de r�solution d'une �quation de diffusion en volumes finis,
HI-83/00/027/A.}.
Les auteurs de ce rapport ont montr� que, pour un maillage monodimensionnel irr�gulier et avec une
viscosit� non constante, la convergence mesur�e est d'ordre 2 en espace avec
l'interpolation harmonique et d'ordre 1 en espace avec l'interpolation
lin�aire (pour des solutions r�guli�res). Par cons�quent, il serait pr�f�rable d'utiliser l'interpolation
harmonique pour calculer la valeur de la viscosit� aux faces. Des tests de stabilit� seront n�cessaires au pr�alable.\\
De m�me, on envisage d'extrapoler la viscosit� sur les faces de bord plut�t que
de prendre la valeur de la viscosit� de la cellule jouxtant cette face.\\
Dans le cas de la moyenne arithm\'etique, l'utilisation de la valeur $0.5$ pour les coefficients $\alpha_{\,ij}$ serait \`a revoir.
