%                      Code_Saturne version 1.3
%                      ------------------------
%
%     This file is part of the Code_Saturne Kernel, element of the
%     Code_Saturne CFD tool.
% 
%     Copyright (C) 1998-2007 EDF S.A., France
%
%     contact: saturne-support@edf.fr
% 
%     The Code_Saturne Kernel is free software; you can redistribute it
%     and/or modify it under the terms of the GNU General Public License
%     as published by the Free Software Foundation; either version 2 of
%     the License, or (at your option) any later version.
% 
%     The Code_Saturne Kernel is distributed in the hope that it will be
%     useful, but WITHOUT ANY WARRANTY; without even the implied warranty
%     of MERCHANTABILITY or FITNESS FOR A PARTICULAR PURPOSE.  See the
%     GNU General Public License for more details.
% 
%     You should have received a copy of the GNU General Public License
%     along with the Code_Saturne Kernel; if not, write to the
%     Free Software Foundation, Inc.,
%     51 Franklin St, Fifth Floor,
%     Boston, MA  02110-1301  USA
%
%-----------------------------------------------------------------------
%
%%%%%%%%%%%%%%%%%%%%%%%%%%%%%%%%%
%%%%%%%%%%%%%%%%%%%%%%%%%%%%%%%%%%
\section{Discr\'etisation}
%%%%%%%%%%%%%%%%%%%%%%%%%%%%%%%%%%
%%%%%%%%%%%%%%%%%%%%%%%%%%%%%%%%%%
Pour int�grer l'�quation (\ref{Base_Covofi_Eq_cv_scal}), une discr�tisation temporelle de
type $\theta$-sch�ma est appliqu�e � la variable r�solue\footnote{Si
$\theta=1/2$, ou qu'une extrapolation est utilis�e, le pas de temps $\Delta t$
est suppos� uniforme en temps et en espace.}~: 
\begin{equation}
f^{n+\theta} = \theta \,\, f^{n+1} + (1-\theta)\,\, f^{n}
\end{equation}

L'�quation (\ref{Base_Covofi_Eq_cv_scal}) est discr\'etis\'ee au temps $n+\theta$ en
supposant les termes sources explicites pris au temps $n+\theta_{S}$, et ceux
implicites en $n+\theta$. 
Par souci de clart�, on suppose, en l'absence d'indication, que les propri�tes
physiques $\Phi$ ($K,\,\rho$,...) et le flux de masse $(\rho\,\underline{u})$
sont pris respectivement aux instants $n+\theta_\Phi$ et $n+\theta_F$, o�
$\theta_\Phi$ et $\theta_F$ d�pendent des sch�mas en temps sp�cifiquement
utilis�s pour ces grandeurs\footnote{cf. \fort{introd}}. 

\begin{equation}
\begin{array}{lcl}
&\displaystyle
 \rho \frac {f^{n+1}-f^{n}}{\Delta t} +
\underbrace{\dive\,((\rho\,\underline{u})\,f^{n+\theta})}_{\text{convection}} 
- \underbrace{\dive\,(\,K\,\grad f^{n+\theta})}_{\text{diffusion}} =
T_s^{\,imp}\,f^{n+\theta} + T_s^{\,exp,\,n+\theta_{S}}\\  
& + (\Gamma\,f_i)^{n+\theta_{S}}-\Gamma^{n}\,f^{n+\theta} +\
T_s^{\,pd,\,n+\theta_S} + f^{n+\theta}\,\dive\,(\rho\underline{u}) 
\end{array}
\end{equation}
o� :
\begin{equation}
 T_s^{\,pd,\,n+\theta_S} = 
\begin{cases}
0 & \text{pour $f=a$}, \\
2 \displaystyle \left[\frac{\mu_t}{\sigma_t}(\grad \widetilde{a})^2\right]^{n+\theta_S}-\frac{\rho\,
\varepsilon^{n}}{R_f\,k^{n}}(\widetilde{{a"}^2})^{n+\theta}& \text{pour $f=\widetilde{{a"}^2}$ }
\end{cases}
\end{equation}
Le terme de production affect� d'un indice $n+\theta_{S}$ est un terme source
explicite et il est donc trait� comme tel : 
\begin{equation}
\begin{array}{rll}
\displaystyle
\left[\frac{\mu_t}{\sigma_t}(\grad
\widetilde{a})^2\right]^{n+\theta_{S}}&=&\displaystyle
(1+\theta_{S})\,\,\frac{\mu_t^{n}}{\sigma_t}(\grad
\widetilde{a}^{n})^2-\theta_{S}\,\,\frac{\mu_t^{n-1}}{\sigma_t}(\grad
\widetilde{a}^{n-1})^2\\ 
\end{array}
\end{equation}
\\

L'�quation (\ref{Base_Covofi_Eq_cv_scal}) s'\'ecrit :
\begin{equation}\label{Base_Covofi_Eq_scal_tempo}
\begin{array}{c}
\displaystyle 
 \rho\,\frac {f^{n+1}-f^{n}}{\Delta t} +
\theta \,\,\dive\,((\rho\,\underline{u})\,f^{n+1})- \theta \,\,\dive\,(\,K\ \grad f^{n+1})
\\
-\left[ \theta\,\, T_s^{\,imp}\,- \theta\,\, \Gamma^{n} + \theta\,\, T_s^{\,pd,\,imp}+\theta\,\,
\dive\,(\rho\ \underline{u})\right]\,f^{n+1}
\\
= (1-\theta)\,\,T_s^{\,imp}\,f^{n} + T_s^{\,exp,\,n+\theta_S} +
(\Gamma\,f_i)^{n+\theta_S}-(1-\theta)\,\,\Gamma^{n}\, 
f^{n}+ T_s^{\,pd,\,exp}-\theta\,T_s^{\,pd,\,imp}\,f^{n} 
\\
+ (1-\theta) \,\, f^{n}\,\dive\,(\rho\ \underline{u})- (1-\theta) \,\, \dive\,((\rho\,\underline{u})\,f^{n})
+ (1-\theta)\,\, \dive\,(\,K\ \grad f^{n}) 
\end{array}
\end{equation}
avec :
\begin{equation}
T_s^{\,pd,\,imp} = 
\begin{cases}
0 & \text{pour $f=a$}, \\
- \displaystyle \frac{\rho\,\varepsilon^n}{R_f \,k^n} &  \text{pour $f=\widetilde{{a"}^2}$}
\end{cases}
\end{equation}
\begin{equation}
T_s^{\,pd,\,exp}= 
\begin{cases}
0 & \text{pour $f=a$}, \\
2\ \displaystyle\left[\frac{\mu_t}{\sigma_t}(\grad
\widetilde{a})^2\right]^{n+\theta_S} -
\frac{\rho\,\varepsilon^n}{R_f\,k^n}(\widetilde{{a"}^2})^n & \text{pour
$f=\widetilde{{a"}^2}$} 
\end{cases}
\end{equation}
On rappelle que, pour un scalaire $f$, le sous-programme \fort{codits} 
r�sout une �quation du type suivant
\label{Base_Covofi_Eq_Codits}
\begin{equation}
\begin{array}{c}
\displaystyle f_s^{\,imp} (f^{n+1} - f^{n}) +
\theta \,\, \dive((\rho\,\underline{u})\,f^{n+1})- \theta \,\, \dive\,(\,K\,\grad f^{n+1})
\\
= f_s^{\,exp} -\underbrace{(1-\theta) \,\, \dive((\rho\,\underline{u})\,f^{n}) + (1-\theta)
\,\, \dive\,(\,K\,\grad f^{n})}_{\text{convection diffusion explicite}} 
\end{array}
\end{equation}
$f_s^{exp}$ repr�sente les termes sources discr\'etis\'es de mani\`ere explicite
en temps (hormis contributions de la convection diffusion explicite provenant du
$\theta$-sch\'ema) et $f_s^{imp}\,f^{n+1}$ repr\'esente les termes lin\'eaires
en $f^{n+1}$ dans l'\'equation discr\'etis\'ee en temps.\\ 
On r��crit l'�quation (\ref{Base_Covofi_Eq_scal_tempo}) sous la forme (\ref{Base_Covofi_Eq_scal_final})
qui est ensuite r�solue par \fort{codits}. 
\begin{equation}
\label{Base_Covofi_Eq_scal_final}
\begin{array}{c}
\displaystyle 
\underbrace{\left(\frac {\rho}{\Delta t}- \theta\,\, T_s^{\,imp}+ \theta\,\,
\Gamma^{n} -\theta\,\, T_s^{\,pd,\,imp} - \theta\,\,
\dive\,(\rho\,\underline{u})\right)}_{\text {$f_s^{\,imp}$}}\ \delta f^{n+1} 
\\
+\theta\,\, \dive(\,(\rho \underline{u})\,f^{n+1}\,)
-\theta\,\, \dive\,(K\,\grad \,f^{n+1}) = \\
\underbrace{T_s^{\,imp}\,f^n +  T_s^{\,exp,\,n+\theta_S}
+\,(\Gamma f_i)^{n+\theta_S}\, - \Gamma^{n}\,f^n +\ T_s^{\,pd,\,exp} + 
 f^{n}\,\dive(\rho\,\underline{u})}_{\text{$f_s^{exp}$}}\\
-(1-\theta)\,\dive(\,(\rho \underline{u})\,f^{n}\,)
+(1-\theta)\,\dive\,(K\,\grad f^{n}) 
\end{array}
\end{equation}
\\
%Pour la discr�tisation spatiale de ce syst�me, on pourra se reporter au
%sous-programme \fort{navsto} 

