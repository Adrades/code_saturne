%                      Code_Saturne version 1.3
%                      ------------------------
%
%     This file is part of the Code_Saturne Kernel, element of the
%     Code_Saturne CFD tool.
% 
%     Copyright (C) 1998-2007 EDF S.A., France
%
%     contact: saturne-support@edf.fr
% 
%     The Code_Saturne Kernel is free software; you can redistribute it
%     and/or modify it under the terms of the GNU General Public License
%     as published by the Free Software Foundation; either version 2 of
%     the License, or (at your option) any later version.
% 
%     The Code_Saturne Kernel is distributed in the hope that it will be
%     useful, but WITHOUT ANY WARRANTY; without even the implied warranty
%     of MERCHANTABILITY or FITNESS FOR A PARTICULAR PURPOSE.  See the
%     GNU General Public License for more details.
% 
%     You should have received a copy of the GNU General Public License
%     along with the Code_Saturne Kernel; if not, write to the
%     Free Software Foundation, Inc.,
%     51 Franklin St, Fifth Floor,
%     Boston, MA  02110-1301  USA
%
%-----------------------------------------------------------------------
%

\programme{viscfa}
%

\vspace{1cm}
%%%%%%%%%%%%%%%%%%%%%%%%%%%%%%%%%%
%%%%%%%%%%%%%%%%%%%%%%%%%%%%%%%%%%
\section{Fonction}
%%%%%%%%%%%%%%%%%%%%%%%%%%%%%%%%%%
%%%%%%%%%%%%%%%%%%%%%%%%%%%%%%%%%%
Dans ce sous-programme est calcul\'e le coefficient de diffusion isotrope aux
faces. Ce coefficient fait intervenir la valeur de la viscosit� aux faces
multipli�e par le rapport surface de la face sur la distance alg\'ebrique $\overline{I'J'}$ ou $\overline{I'F}$({\it cf.} figure \ref{Base_Viscfa_fig_geom}), 
rapport r�sultant de l'int�gration du terme de diffusion.
 Par analogie du terme calcul�, ce sous-programme est aussi appel� par le
sous-programme \fort{resolp} pour calculer le coefficient ``diffusif'' de la pression faisant intervenir le pas de temps.\\
La valeur de la viscosit� aux faces est d�termin�e soit par une moyenne
arithm�tique, soit par une moyenne harmonique de la viscosit� au centre des
cellules, suivant le choix de l'utilisateur. Par d�faut, cette valeur est calcul�e par une moyenne arithm�tique.

