%                      Code_Saturne version 1.3
%                      ------------------------
%
%     This file is part of the Code_Saturne Kernel, element of the
%     Code_Saturne CFD tool.
% 
%     Copyright (C) 1998-2007 EDF S.A., France
%
%     contact: saturne-support@edf.fr
% 
%     The Code_Saturne Kernel is free software; you can redistribute it
%     and/or modify it under the terms of the GNU General Public License
%     as published by the Free Software Foundation; either version 2 of
%     the License, or (at your option) any later version.
% 
%     The Code_Saturne Kernel is distributed in the hope that it will be
%     useful, but WITHOUT ANY WARRANTY; without even the implied warranty
%     of MERCHANTABILITY or FITNESS FOR A PARTICULAR PURPOSE.  See the
%     GNU General Public License for more details.
% 
%     You should have received a copy of the GNU General Public License
%     along with the Code_Saturne Kernel; if not, write to the
%     Free Software Foundation, Inc.,
%     51 Franklin St, Fifth Floor,
%     Boston, MA  02110-1301  USA
%
%-----------------------------------------------------------------------
%

%%%%%%%%%%%%%%%%%%%%%%%%%%%%%%%%%%
%%%%%%%%%%%%%%%%%%%%%%%%%%%%%%%%%%
\section{Discr\'etisation}
%%%%%%%%%%%%%%%%%%%%%%%%%%%%%%%%%%
%%%%%%%%%%%%%%%%%%%%%%%%%%%%%%%%%%
Soit une cellule $\Omega_i$, $\phi_{ij}$  le flux de masse (total ou uniquement la partie en
gradient de pression) \`a travers la face la
s\'eparant d'une cellule voisine $\Omega_j$ et $\phi_{\,b_ik}$ le flux de masse (total ou uniquement la partie en
gradient de pression)\`a travers la face de bord $\,b_{ik}$.
L'id\'eal serait de pouvoir trouver un vecteur $\vect{v}_i$ telle que, pour toute cellule voisine $\Omega_j$ on ait :
\begin{equation}
\rho_i\vect{v}_i.\vect{S}_{ij} = \phi_{ij}
\end{equation}
et l'\'equivalent aux faces de bords, {\it i.e.} :
\begin{equation}
\rho_i \vect{v}_i.\vect{S}_{\,b_{ik}} = \phi_{\,b_{ik}}
\end{equation}
Comme c'est g\'en\'eralement impossible d'obtenir les deux \'egalit\'es pr\'ec\'edentes\footnote{%
sauf en incompressible pour des triangles en 2D et des
t\'etra\`edres en 3D}, on va simplement chercher \`a minimiser la fonction $F_i$ :
\begin{equation}
F_i=\sum\limits_{j\in Vois(i)}\left[
\rho_i\vect{v}_i.\vect{S}_{ij}-\phi_{ij}\right]^2 + \sum\limits_{k\in {\gamma_b(i)}}\left[\rho_i\vect{v}_i.\vect{S}_{\,b_{ik}}-\phi_{\,b_{ik}}\right]^2
\end{equation}

Pour ce faire, on d\'erive $F_i$ par rapport aux trois composantes du vecteur $\vect{v}_i$,
et on r\'esout le syst\`eme $3\times3$ local qui r\'esulte :\\
\begin{equation}
\begin{array}{lll}
&\displaystyle \tens{\mathcal{S}}^{\,i} \,
\left[\begin{array}{c}
v_{i,x} \\ v_{i,y} \\ v_{i,z}
\end{array}\right]
&=\left[\begin{array}{c}
\displaystyle
\frac{1}{\rho_i}(\sum\limits_{j\in Vois(i)}\phi_{ij}S_{ij,x} +\sum\limits_{k\in {\gamma_b(i)}}\phi_{\,b_{ik}}S_{{\,b_{ik}},x})\\
\displaystyle
\frac{1}{\rho_i}(\sum\limits_{j\in Vois(i)}\phi_{ij}S_{ij,y} +\sum\limits_{k\in {\gamma_b(i)}}\phi_{\,b_{ik}}S_{{\,b_{ik}},y})\\
\displaystyle
\frac{1}{\rho_i}(\sum\limits_{j\in Vois(i)}\phi_{ij}S_{ij,z} +\sum\limits_{k\in {\gamma_b(i)}}\phi_{\,b_{ik}}S_{{\,b_{ik}},z}) 
\end{array}\right]
\end{array}
\end{equation}

avec $\tens{\mathcal{S}}^{\,i}$ matrice carr\'ee $3\times3$ d'\'el\'ement $S^{\,i}_{\,ml}$ courant d\'efini par :\\
\begin{equation}
S^{\,i}_{\,ml} = \sum\limits_{j\in Vois(i)}S_{ij,\,l}\,S_{ij,\,m} + \sum\limits_{k\in {\gamma_b(i)}}S_{{\,b_{ik}},\,l}\,S_{{\,b_{ik}},\,m}
\end{equation}

%\begin{equation}
%\left[\begin{array}{ccc}
%\displaystyle
%\sum\limits_jS_{ij,x}S_{ij,x} & \sum\limits_jS_{ij,x}S_{ij,y} 
%& \sum\limits_jS_{ij,x}S_{ij,z}\\
%\displaystyle
%\sum\limits_jS_{ij,x}S_{ij,y} & \sum\limits_jS_{ij,y}S_{ij,y} 
%& \sum\limits_jS_{ij,y}S_{ij,z}\\
%\displaystyle
%\sum\limits_jS_{ij,x}S_{ij,z} & \sum\limits_jS_{ij,y}S_{ij,z} 
%& \sum\limits_jS_{ij,z}S_{ij,z}
%\end{array}\right]
%\left[\begin{array}{c}
%u_{i,x} \\ u_{i,y} \\ u_{i,z}
%\end{array}\right]
%=\left[\begin{array}{c}
%\displaystyle
%\frac{1}{\rho_i}\sum\limits_j\phi_{ij}S_{ij,x}\\
%\displaystyle
%\frac{1}{\rho_i}\sum\limits_j\phi_{ij}S_{ij,y}\\
%\displaystyle
%\frac{1}{\rho_i}\sum\limits_j\phi_{ij}S_{ij,z}
%\end{array}\right]
%\end{equation}
