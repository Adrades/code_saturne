%                      Code_Saturne version 1.3
%                      ------------------------
%
%     This file is part of the Code_Saturne Kernel, element of the
%     Code_Saturne CFD tool.
% 
%     Copyright (C) 1998-2007 EDF S.A., France
%
%     contact: saturne-support@edf.fr
% 
%     The Code_Saturne Kernel is free software; you can redistribute it
%     and/or modify it under the terms of the GNU General Public License
%     as published by the Free Software Foundation; either version 2 of
%     the License, or (at your option) any later version.
% 
%     The Code_Saturne Kernel is distributed in the hope that it will be
%     useful, but WITHOUT ANY WARRANTY; without even the implied warranty
%     of MERCHANTABILITY or FITNESS FOR A PARTICULAR PURPOSE.  See the
%     GNU General Public License for more details.
% 
%     You should have received a copy of the GNU General Public License
%     along with the Code_Saturne Kernel; if not, write to the
%     Free Software Foundation, Inc.,
%     51 Franklin St, Fifth Floor,
%     Boston, MA  02110-1301  USA
%
%-----------------------------------------------------------------------
%

\programme{itrmas/itrgrp}

\vspace{1cm}
%%%%%%%%%%%%%%%%%%%%%%%%%%%%%%%%%%
%%%%%%%%%%%%%%%%%%%%%%%%%%%%%%%%%%
\section{Fonction}
%%%%%%%%%%%%%%%%%%%%%%%%%%%%%%%%%%
%%%%%%%%%%%%%%%%%%%%%%%%%%%%%%%%%%
Le but du sous-programme \fort{itrmas} est de calculer un gradient de pression
``facette''. Il est utilis\'e dans la phase de correction de pression
(deuxi\`eme phase de \fort{navsto}) o\`u le flux de masse est mis \`a jour \`a l'aide de termes en $-\Delta t_{\,ij}(\grad_f P)_{\,ij}.\vect{S}_{\,ij}$ et en $-\Delta t_{\,b_{ik}}(\grad_f P)_{\,b_{ik}}\,.\,\vect{S}_{\,b_{ik}}$.

Le sous-programme \fort{itrgrp} calcule un gradient ``facette'' de pression et
en prend la divergence, c'est-\`a-dire calcule le terme :
\begin{displaymath}
-\sum\limits_{j\in Vois(i)}\Delta t_{\,ij}(\grad_f P)_{\,ij}.\vect{S}_{\,ij}
-\sum\limits_{k\in\gamma_b(i)}\Delta t_{\,b_{ik}}(\grad_f P)_{\,b_{ik}}\,.\,\vect{S}_{\,b_{ik}}
\end{displaymath}
En pratique \fort{itrgrp} correspond \`a la combinaison de \fort{itrmas} et de
\fort{divmas}, mais permet par son traitement en un seul bloc d'\'eviter la
d\'efinition de tableaux de travail de taille \var{NFAC} et \var{NFABOR}.


