%-----------------------------------------------------------------------
%
%     This file is part of the Code_Saturne Kernel, element of the
%     Code_Saturne CFD tool.
%
%     Copyright (C) 1998-2008 EDF S.A., France
%
%     contact: saturne-support@edf.fr
%
%     The Code_Saturne Kernel is free software; you can redistribute it
%     and/or modify it under the terms of the GNU General Public License
%     as published by the Free Software Foundation; either version 2 of
%     the License, or (at your option) any later version.
%
%     The Code_Saturne Kernel is distributed in the hope that it will be
%     useful, but WITHOUT ANY WARRANTY; without even the implied warranty
%     of MERCHANTABILITY or FITNESS FOR A PARTICULAR PURPOSE.  See the
%     GNU General Public License for more details.
%
%     You should have received a copy of the GNU General Public License
%     along with the Code_Saturne Kernel; if not, write to the
%     Free Software Foundation, Inc.,
%     51 Franklin St, Fifth Floor,
%     Boston, MA  02110-1301  USA
%
%-----------------------------------------------------------------------
%

\programme{recvmc}

\vspace{1cm}
%%%%%%%%%%%%%%%%%%%%%%%%%%%%%%%%%%
%%%%%%%%%%%%%%%%%%%%%%%%%%%%%%%%%%
\section{Fonction}
%%%%%%%%%%%%%%%%%%%%%%%%%%%%%%%%%%
%%%%%%%%%%%%%%%%%%%%%%%%%%%%%%%%%%
Le but de ce sous-programme est de calculer la vitesse au centre des cellules
\`a partir du flux de masse aux faces, par moindres carr\'es. Utilis\'ee apr\`es
l'\'etape de correction de pression ({\it cf.}~\fort{navsto}) cette m\'ethode est une
alternative \`a la technique de reconstruction \`a partir du gradient de
l'incr\'ement de pression (technique standard).
Elle est activ\'ee quand l'indicateur \var{IREVMC} vaut~1 ou 2.

On rappelle que, � la fin de l'�tape de correction de pression, le flux de masse aux faces vaut :
\begin{equation}
(\rho \vect{u})^{n+1}_{\,ij}\text{.}\vect{S}_{\,ij} =
(\rho
\vect{\widetilde{u}})^{n+1}_{\,ij}.\,\vect{S}_{\,ij}
-\vect{D}_{\,ij}(\Delta t^n,\delta P^{n+\theta})
+\text{RC}_{\,ij}
\end{equation}
o� $\vect{\widetilde{u}}$ est la vitesse issue de l'�tape de pr�diction, $D_{\,ij}$ un op�rateur de gradient aux faces
et $\text{RC}_{\,ij}$ le terme d'Arakawa (cf. \fort{navsto} pour une d�finition pr�cise des notations).
Une premi�re m�thode, activ�e par \var{IREVMC} = 2, consiste � partir directement de
$(\rho \vect{u})^{n+1}_{\,ij}\text{.}\vect{S}_{\,ij}$ pour calculer $\vect{u}^{n+1}$ par moindres carr�s. Son utilisation a
montr\'e qu'elle semblait plus diffusive que la m\'ethode standard (par exemple, dans le cas de la cavit\'e entra\^\i n\'ee)
et pouvait conduire � des r�sultats erron�s sur des maillages ne comportant pas uniquement des t�tra�dres
(ou des prismes � base triangulaire en ``2D'') et des pav�s (hexa�dres orthogonaux).\\
On note que, dans la m�thode ci-dessus, on est parti d'une vitesse $\vect{\widetilde{u}}$ au centre des cellules, qu'on a projet�e aux faces pour obtenir le flux de masse, et qu'on ram�ne au centre des cellules par moindres carr�s. Fort de cette constatation, une m�thode alternative est disponible, activ�e par \var{IREVMC} = 1. Elle consiste � n'appliquer la m�thode des moindres carr�s qu'� la partie $-\vect{D}_{\,ij}(\Delta t^n,\delta P^{n+\theta}) +\text{RC}_{\,ij}$ du flux de masse et � rajouter directement
$\vect{\widetilde{u}}$ (connu au centre des cellules) au r�sultat obtenu\footnote{cette derni�re �tape est faite dans
\fort{navsto}.}. Cette m�thode donne des r�sultats sensiblement meilleurs.

%%%%%%%%%%%%%%%%%%%%%%%%%%%%%%%%%%
%%%%%%%%%%%%%%%%%%%%%%%%%%%%%%%%%%
\section{Discr\'etisation}
%%%%%%%%%%%%%%%%%%%%%%%%%%%%%%%%%%
%%%%%%%%%%%%%%%%%%%%%%%%%%%%%%%%%%
Soit une cellule $\Omega_i$, $\phi_{ij}$  le flux de masse (total ou uniquement la partie en
gradient de pression) \`a travers la face la
s\'eparant d'une cellule voisine $\Omega_j$ et $\phi_{\,b_ik}$ le flux de masse (total ou uniquement la partie en
gradient de pression)\`a travers la face de bord $\,b_{ik}$.
L'id\'eal serait de pouvoir trouver un vecteur $\vect{v}_i$ telle que, pour toute cellule voisine $\Omega_j$ on ait :
\begin{equation}
\rho_i\vect{v}_i.\vect{S}_{ij} = \phi_{ij}
\end{equation}
et l'\'equivalent aux faces de bords, {\it i.e.} :
\begin{equation}
\rho_i \vect{v}_i.\vect{S}_{\,b_{ik}} = \phi_{\,b_{ik}}
\end{equation}
Comme c'est g\'en\'eralement impossible d'obtenir les deux \'egalit\'es pr\'ec\'edentes\footnote{%
sauf en incompressible pour des triangles en 2D et des
t\'etra\`edres en 3D}, on va simplement chercher \`a minimiser la fonction $F_i$ :
\begin{equation}
F_i=\sum\limits_{j\in Vois(i)}\left[
\rho_i\vect{v}_i.\vect{S}_{ij}-\phi_{ij}\right]^2 + \sum\limits_{k\in {\gamma_b(i)}}\left[\rho_i\vect{v}_i.\vect{S}_{\,b_{ik}}-\phi_{\,b_{ik}}\right]^2
\end{equation}

Pour ce faire, on d\'erive $F_i$ par rapport aux trois composantes du vecteur $\vect{v}_i$,
et on r\'esout le syst\`eme $3\times3$ local qui r\'esulte :\\
\begin{equation}
\begin{array}{lll}
&\displaystyle \tens{\mathcal{S}}^{\,i} \,
\left[\begin{array}{c}
v_{i,x} \\ v_{i,y} \\ v_{i,z}
\end{array}\right]
&=\left[\begin{array}{c}
\displaystyle
\frac{1}{\rho_i}(\sum\limits_{j\in Vois(i)}\phi_{ij}S_{ij,x} +\sum\limits_{k\in {\gamma_b(i)}}\phi_{\,b_{ik}}S_{{\,b_{ik}},x})\\
\displaystyle
\frac{1}{\rho_i}(\sum\limits_{j\in Vois(i)}\phi_{ij}S_{ij,y} +\sum\limits_{k\in {\gamma_b(i)}}\phi_{\,b_{ik}}S_{{\,b_{ik}},y})\\
\displaystyle
\frac{1}{\rho_i}(\sum\limits_{j\in Vois(i)}\phi_{ij}S_{ij,z} +\sum\limits_{k\in {\gamma_b(i)}}\phi_{\,b_{ik}}S_{{\,b_{ik}},z})
\end{array}\right]
\end{array}
\end{equation}

avec $\tens{\mathcal{S}}^{\,i}$ matrice carr\'ee $3\times3$ d'\'el\'ement $S^{\,i}_{\,ml}$ courant d\'efini par :\\
\begin{equation}
S^{\,i}_{\,ml} = \sum\limits_{j\in Vois(i)}S_{ij,\,l}\,S_{ij,\,m} + \sum\limits_{k\in {\gamma_b(i)}}S_{{\,b_{ik}},\,l}\,S_{{\,b_{ik}},\,m}
\end{equation}

%\begin{equation}
%\left[\begin{array}{ccc}
%\displaystyle
%\sum\limits_jS_{ij,x}S_{ij,x} & \sum\limits_jS_{ij,x}S_{ij,y}
%& \sum\limits_jS_{ij,x}S_{ij,z}\\
%\displaystyle
%\sum\limits_jS_{ij,x}S_{ij,y} & \sum\limits_jS_{ij,y}S_{ij,y}
%& \sum\limits_jS_{ij,y}S_{ij,z}\\
%\displaystyle
%\sum\limits_jS_{ij,x}S_{ij,z} & \sum\limits_jS_{ij,y}S_{ij,z}
%& \sum\limits_jS_{ij,z}S_{ij,z}
%\end{array}\right]
%\left[\begin{array}{c}
%u_{i,x} \\ u_{i,y} \\ u_{i,z}
%\end{array}\right]
%=\left[\begin{array}{c}
%\displaystyle
%\frac{1}{\rho_i}\sum\limits_j\phi_{ij}S_{ij,x}\\
%\displaystyle
%\frac{1}{\rho_i}\sum\limits_j\phi_{ij}S_{ij,y}\\
%\displaystyle
%\frac{1}{\rho_i}\sum\limits_j\phi_{ij}S_{ij,z}
%\end{array}\right]
%\end{equation}

%%%%%%%%%%%%%%%%%%%%%%%%%%%%%%%%%%
%%%%%%%%%%%%%%%%%%%%%%%%%%%%%%%%%%
\section{Mise en \oe uvre}
%%%%%%%%%%%%%%%%%%%%%%%%%%%%%%%%%%
%%%%%%%%%%%%%%%%%%%%%%%%%%%%%%%%%%
Le flux de masse est pass\'e par les arguments \var{FLUMAS} et \var{FLUMAB}.

\etape{Calcul de la matrice}
Les \var{NCEL} matrices $3\times 3$ sont stock\'ees dans le tableau de travail
\var{COCG},
de dimension $NCELET\times 3\times 3$. Ce dernier est d'abord mis \`a z\'ero, puis
son remplissage se fait dans des boucles sur les faces internes et les faces de
bord. La matrice \'etant sym\'etrique, ces boucles ne
servent qu'\`a remplir la partie triangulaire sup\'erieure, le reste \'etant
rempli par sym\'etrie \`a la fin.

\etape{Inversion de la matrice}
On calcule les coefficients de la comatrice, puis l'inverse.
Pour des questions de vectorisation, la boucle sur les \var{NCEL} \'el\'ements
est remplac\'ee par une
s\'erie de boucles en vectorisation forc\'ee sur des blocs de \var{NBLOC=1024}
\'el\'ements. Le reliquat ($\var{NCEL}-E(\var{NCEL}/1024)\times 1024$) est
trait\'e apr\`es les boucles.
\`A la fin, la matrice inverse est stock\'ee dans \var{COCG}
(toujours en utilisant sa propri\'et\'e de sym\'etrie).

\etape{Calcul du second membre et r\'esolution}
Le second membre est stock\'e dans \var{BX}, \var{BY} et \var{BZ}. La vitesse
finale est stock\'ee dans \var{UX}, \var{UY} et \var{UZ}.


%%%%%%%%%%%%%%%%%%%%%%%%%%%%%%%%%%
%%%%%%%%%%%%%%%%%%%%%%%%%%%%%%%%%%
\section{Points \`a traiter}
%%%%%%%%%%%%%%%%%%%%%%%%%%%%%%%%%%
%%%%%%%%%%%%%%%%%%%%%%%%%%%%%%%%%%
\etape{Vectorisation forc\'ee}
Le d\'ecoupage en boucles de 1024 est-il vraiment n\'ecessaire ? Les machines
vectorielles et les compilateurs sont-ils aujourd'hui capables
d'effectuer la vectorisation sans cette aide ? On note cependant que ce
d\'ecoupage en boucles de 1024 n'a pas de co\^ut CPU suppl\'ementaire, et un
co\^ut m\'emoire n\'egligeable. Le seul inconv\'enient r\'eside dans la
complexit\'e de l'\'ecriture.

\etape{Suppression de la m�thode \var{IREVMC} = 2}
Sur un maillage ``1D'' d'hexa�dres tous orthogonaux sauf une face, on peut montrer que la m�thode fait appara�tre
une composante de vitesse aberrante non nulle et directement d�termin�e par l'angle de non orthogonalit� de la
face (non consistance). On pourrait donc songer � supprimer purement cette m�thode, dans la mesure o� elle n'est
{\em a priori} consistante que sur une classe r�duite de maillages.

