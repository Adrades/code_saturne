%-------------------------------------------------------------------------------

% This file is part of Code_Saturne, a general-purpose CFD tool.
%
% Copyright (C) 1998-2011 EDF S.A.
%
% This program is free software; you can redistribute it and/or modify it under
% the terms of the GNU General Public License as published by the Free Software
% Foundation; either version 2 of the License, or (at your option) any later
% version.
%
% This program is distributed in the hope that it will be useful, but WITHOUT
% ANY WARRANTY; without even the implied warranty of MERCHANTABILITY or FITNESS
% FOR A PARTICULAR PURPOSE.  See the GNU General Public License for more
% details.
%
% You should have received a copy of the GNU General Public License along with
% this program; if not, write to the Free Software Foundation, Inc., 51 Franklin
% Street, Fifth Floor, Boston, MA 02110-1301, USA.

%-------------------------------------------------------------------------------

\programme{ Introduction}
{\huge sub-routines: co**, cp**, fu** ...}

%%%%%%%%%%%%%%%%%%%%%%%%%%%%%%%%%%
%%%%%%%%%%%%%%%%%%%%%%%%%%%%%%%%%%
\section{Use \& call}
%%%%%%%%%%%%%%%%%%%%%%%%%%%%%%%%%%
%%%%%%%%%%%%%%%%%%%%%%%%%%%%%%%%%%

From a CFD point of view combustion is a ({\small sometimes very})
complicated way to determine $\rho$.\\ Models needs few extra scalar
fields with regular transport equations, some of them with a
reactive or interfacial source term.\\ Modelling of combustion is able
to deal with gas phase combustion ({\small diffusion, premix, partial
premix}), and with solid or liquid fuels.\\ Combustion of condensed
fuels involves one-way interfacial flux due to phenomena in the
condensed phase ({\small evaporation or pyrolisis}) and reciprocal
ones ({\small heterogeneous combustion}). Many of the species injected
in the gas phase are afterwards involved in gas phase combsution.\\
That is the reason why many modules are similar for gas, coal and fuel
combustion modelling. Obviously, the thermodynamical description of
gas species is similar in every version as close as possible to the
JANAF rules.\\ All models are developped in both adiabatic and
unadiabatic ({\small permeatic: heat loss, eg. by radiation})
version, beyond the standard, the rule to call models is:

IPPMOD(index model)  =  -1     unused

IPPMOD(index model)  =   0     simplest adiabatic version

IPPMOD(index model)  =   1     simplest permeatic version

Eventually

IPPMOD(index model)  =  2.p    P� adiabatic version

IPPMOD(index model)  =  2.p+1  P� permeatic version 


Every permeatic version involves the transport of enthalpy ({\small one more variable}). 

%=================================
\subsection{Gas combustion modelling}
%=================================

Gas combustion is limited by disponibility ({\small in the same fluid
particle}) of both fuel and oxidant and by kinetic effects ({\small a
lot of chemical reactions involved can be described using an Arrhenius
law with high activation energy}). The mixing of mass ({\small atoms})
incoming with fuel and oxydant is described by a mixture fraction
({\small mass fraction of matter incoming with fuel}), this variable
is not affected by combustion. A progress variable is used to describe
the transformation of the mixture from fuel and oxydant to products
({\small carbon dioxyde and so on}).Combustion of gas is,
traditionnaly, splitted in premix and diffusion regimes.\\

In premix combustion process a first stage of mixing have been
realised ({\small without blast ...}) and the mixture is introduced in
the boiler ({\small or combustor can}). In common industrial
conditions the combustion is mainly limited by the mixing of fresh
gases ({\small inert}) and burnt ones resulting in the inflammation of
the first and their conversion to burnt ones; so an assumption of
chemistry much faster than mixing induces an intermittent regime. The
gas flow is constituted of totally fresh and totally burnt gases
({\small the flamme containing the gases during their transformation
is "extremely" thin}). With this previous assumptions,
Spalding \cite{1} established the "Eddy Break Up" model, which allows
a complete description with only one progress variable ({\small
mixture fraction is homogeneous}).\\

In diffusion flames the fuel and the oxydant are introduced by two
({\small at least}) inlets, in common industrial conditions, their
mixing is the main limitation and the mixture fraction is enough to
qualify a fluid particle, but in turbulent flows a {\em P}robability
{\em D}ensity {\em F}unction of the mixture fraction is needed to
qualify the thermodynamical state of the bulk. So both the mean and
the variance of the mixture fraction are needed ({\small two
variables}).\\

Real world's chemistry is not so fast and, unfortunately, the mixing
can not be so homogeneous as wished. Then industrial combustion occurs
in partial premix regime. Partial premix occurs if mixing is not
finished ({\small at molecular level}) when the mixture is introduced,
or if air or fuel, are staggered, or if a diffusion flame is blown
off. For these situations, and specifically for lean premix gas
turbines Libby \& Williams \cite{2} developped a model allowing a
description of both mixing and chemical limitations. A collaboration
between the LCD Poitiers \cite{3} and EDF R\&D allows a simpler
version of their algorithm. Not only the mean and the variance of both
mixture fraction and progress variable are needed but also their
covariance ({\small five variables}).


%=================================
\subsection{Coal combustion modelling}
%=================================

Coal combustion is the main way to produce electricity in the world.
Coal is a natural product with a very complex composition. During the
industrial process of milling the raw coal is broken in tiny particles
of different sizes. After its introduction in the boiler, coal
particles undergoes drying, devolatilisation ({\small heating of coal
turn it in a mixture of char and gases}), heterogenous combustion
({\small of char in carbon monoxide}) leaving ash particles.\\ Each of
these phenomena are taken into account for some class of particles: a
class is caracterised by a coal ({\small it is useful to burn mixture
of coals with differents ranks or mixture of coal with biomasse ...})
and an initial diameter. For each class, \CS computes the number and
the mass of particles by unit mass of mixture.\\ The main assumption
is to solve only one velocity ({\small and pressure}) field: it means
the discrepancy of velocity between coal particles and gases is
assumed negligible.\\ Due to the radiation and heterogeneous
combustion, temperature can be different for gas and different size
particles: so the specific enthalpy of each particle class is
solved.\\ The description of coal pyrolysis proposed by Kobayashi \&
Ubhayakar \cite{4} is used, leading to two source terms for light and
heavy volatile matters ({\small the moderate temperature reaction
produces gases with low molecular mass, the high temperature reaction
produces heavier gases and less char}) represented by two passive
scalars: mixture fractions.  The description of the heterogeneous
reaction ({\small which produce carbon monoxide}) produces a source
term for the carbon: a mixture fraction who can't be greater than the
results of stoechiometric oxidation of char by air ({\small carbon
can't be free in gas phase, it is always linked in an oxide}).\\ The
retained model for the gas phase combustion is the assumption of
diffusion flammelets surrounding each particle, so the previous
gaseous fuels are mixed in a local mean fuel and the mixing with air
is represented by a pdf between air and the mean local fuel
constructed with the variance of a passive scalar linked with air
({\small interfacial mass flux produce a source term for this
scalar}).
 



%=================================
\subsection{Heavy Fuel Oil combustion modelling}
%=================================

Heavy fuel oil combustion have been hugely used to produced electrical
energy. Environmental regulation turning it more difficult and less
acceptable, a focus is needed on pollutant emission mainly sulphur
oxide and particles ({\small condensation of sulphuric acid can
aggregate soot}).\\ The description of fuel evaporation is done with
respect to its heaviness: after a minimum temperature is reached, the
gain of enthalpy is splitted between heating and evaporation. This way
the evaporation takes place on a range of temperature ({\small which
can be large}). The "total" evaporation is common for light ({\small
domestic}) oil but impossible for heavy ones: at high temperature,
during the last evaporation, a crakink reaction appears: so a
particle similar to char leaves. The heterogeneous oxydation of this
char particle is very similar to coal char ones.\\ Fuel injection is
described ({\small 2006 version}) with only one class of particles
({\small i.e. initial diameter}), the number, mass and specific
enthalpy of particles are calculated eveywhere. So three variables are
used to describe the condensed phase. In the same way as for coal,
only one velocity field is computed.\\ The model for gas combustion is
very similar to coal one but a special attention is paid to sulphur
({\small assumed to leave the particle as H2S during evaporation and
to be converted to SO2 during gas combustion}).


%==================================
%==================================
\section{Bibliography}
%==================================
%==================================
\begin{thebibliography}{4}


\bibitem{1}
{\sc Spalding, D.B., {\em et al.}},\\
{\em Mixing and chemical reaction in steady confined turbulent turbulent flames},\\
13th Int.Symp. on Combustion , pp. 649-657, (1971).

\bibitem{2}
{\sc Libby, P.A. and Williams, F.A.},\\
{\em A presumed PDF analysis of lean partially premixed turbulent combustion},\\
Combust. Sci. Technol., 161, pp. 351-390, (2000)

\bibitem{3}
{\sc Ribert, G.; Champion, M. and Plion, P.},\\
{\em Modeling turbulent reactive flows with variable equivalence ratio: application to the calculation of a reactive shear layer},\\
Combust. Sci. Technol., 176, pp. 907-923, (2004)

\bibitem{4}
{\sc Kobayashi, H. {\em et al.}},\\
16th Int.Symp. on Combustion , pp. 425-441, (1976).





\end{thebibliography}
\newpage
%%%%%%%%%%%%%%%%%%%%%%%%%%%%%%%%%%
%%%%%%%%%%%%%%%%%%%%%%%%%%%%%%%%%%
\section{Discr\'etisation}
%%%%%%%%%%%%%%%%%%%%%%%%%%%%%%%%%%
%%%%%%%%%%%%%%%%%%%%%%%%%%%%%%%%%%

On se reportera aux sections relatives aux sous-programmes 
\fort{cfmsvl} (masse volumique), \fort{cfqdmv} 
(quantit\'e de mouvement) et \fort{cfener} (\'energie). 
La documentation du sous-programme 
\fort{cfxtcl} fournit des \'el\'ements relatifs aux 
conditions
aux limites. 
