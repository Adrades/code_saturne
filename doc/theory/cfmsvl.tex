%-------------------------------------------------------------------------------

% This file is part of Code_Saturne, a general-purpose CFD tool.
%
% Copyright (C) 1998-2011 EDF S.A.
%
% This program is free software; you can redistribute it and/or modify it under
% the terms of the GNU General Public License as published by the Free Software
% Foundation; either version 2 of the License, or (at your option) any later
% version.
%
% This program is distributed in the hope that it will be useful, but WITHOUT
% ANY WARRANTY; without even the implied warranty of MERCHANTABILITY or FITNESS
% FOR A PARTICULAR PURPOSE.  See the GNU General Public License for more
% details.
%
% You should have received a copy of the GNU General Public License along with
% this program; if not, write to the Free Software Foundation, Inc., 51 Franklin
% Street, Fifth Floor, Boston, MA 02110-1301, USA.

%-------------------------------------------------------------------------------

\programme{cfmsvl}
%
\vspace{1cm}
%%%%%%%%%%%%%%%%%%%%%%%%%%%%%%%%%%
%%%%%%%%%%%%%%%%%%%%%%%%%%%%%%%%%%
\section*{Fonction}
%%%%%%%%%%%%%%%%%%%%%%%%%%%%%%%%%%
%%%%%%%%%%%%%%%%%%%%%%%%%%%%%%%%%%

Pour les notations et l'algorithme dans son ensemble,
on se reportera \`a \fort{cfbase}.

On consid\`ere un premier pas fractionnaire au cours duquel l'\'energie totale
est fixe. Seules varient la masse volumique et le flux de masse acoustique
normal aux faces (d\'efini et calcul\'e aux faces).

On a donc le syst\`eme suivant, entre $t^n$ et $t^*$~:
\begin{equation}\label{Cfbl_Cfmsvl_eq_acoustique_cfmsvl}
\left\{\begin{array}{l}

\displaystyle\frac{\partial\rho}{\partial t}+\divs{\vect{Q}_{ac}} = 0 \\
\\
\displaystyle\frac{\partial\vect{Q}_{ac}}{\partial t}+\gradv{P} =
\rho \vect{f}\\
\\
\vect{Q}^*=\vect{Q}^n\\
\\
e^*=e^n\\

\end{array}\right.
\end{equation}

Une partie des termes sources de l'\'equation de la
quantit\'e de mouvement peut \^etre prise en compte dans cette \'etape
(les termes les plus importants, en pr\^etant attention aux sous-\'equilibres).

Il faut noter que si $\vect{f}$ est effectivement nul, on aura bien un
syst\`eme ``acoustique'', mais que si l'on place des termes suppl\'ementaires
dans $\vect{f}$, la d\'enomination est abusive (on la conservera cependant).

On obtient $\rho^* = \rho^{n+1}$ en r\'esolvant (\ref{Cfbl_Cfmsvl_eq_acoustique_cfmsvl}),
et l'on actualise alors le flux de masse acoustique $\vect{Q}_{ac}^{n+1}$,
qui servira pour la convection (en particulier pour la convection de
l'enthalpie totale et de tous les scalaires transport\'es).

Suivant la valeur de \var{IGRDPP}, on actualise �ventuellement la pression, en
utilisant la loi d'\'etat :
$$
\displaystyle P^{Pred}=P(\rho^{n+1},\varepsilon^{n})
$$

%%%%%%%%%%%%%%%%%%%%%%%%%%%%%%%%%
%%%%%%%%%%%%%%%%%%%%%%%%%%%%%%%%%%
\section*{Discr\'etisation}
%%%%%%%%%%%%%%%%%%%%%%%%%%%%%%%%%%
%%%%%%%%%%%%%%%%%%%%%%%%%%%%%%%%%%
%---------------------------------
\subsection*{Discr\'etisation en temps}
%---------------------------------

Le syst\`eme (\ref{Cfbl_Cfmsvl_eq_acoustique_cfmsvl}) discr\'etis\'e en temps donne :
\begin{equation}\label{Cfbl_Cfmsvl_eq_acoustique_discrete_cfmsvl}
\left\{\begin{array}{l}

\displaystyle\frac{\rho^{n+1}-\rho^n}{\Delta t^n}
+ \divs{\vect{Q}_{ac}^{n+1}} = 0 \\
\\
\displaystyle\frac{\vect{Q}_{ac}^{n+1}-\vect{Q}^n}{\Delta t^n}+\gradv{P^*} =
\rho^n \vect{f}^n\\
\\
Q^*=Q^n\\
\\
e^*=e^n\\

\end{array}\right.
\end{equation}

\begin{equation}\label{Cfbl_Cfmsvl_eq_forces_supplementaires_cfmsvl}
\begin{array}{llll}
\text{avec\ }&\vect{f}^n &=& \vect{0} \\
\text{ou\ }&\vect{f}^n &=& \vect{g} \\
\text{ou m\^eme\ }&\vect{f}^n &=& \vect{f}_v
 + \displaystyle\frac{1}{\rho^n}
\left( - \divs(\vect{u} \otimes \vect{Q}) + \divv(\tens{\Sigma}^v)
 + \vect{j}\wedge\vect{B} \right)^n
\end{array}
\end{equation}

Dans la pratique nous avons d�cid� de prendre $\vect{f}^n=\vect{g}$~:
\begin{itemize}
  \item le terme $\vect{j}\wedge\vect{B}$ n'a pas �t� test�,
  \item le terme $\divv(\tens{\Sigma}^v)$ \'etait n�gligeable sur les tests
        r�alis�s,
  \item le terme $\divs(\vect{u} \otimes \vect{Q})$ a paru d�stabiliser les
        calculs (mais au moins une partie des tests a \'et\'e r\'ealis\'ee
        avec une erreur de programmation et il faudrait donc les reprendre).
\end{itemize}
\bigskip

Le terme $\vect{Q}^n$ dans la 2\textsuperscript{\`eme} \'equation
de (\ref{Cfbl_Cfmsvl_eq_acoustique_discrete_cfmsvl}) est le vecteur ``quantit\'e de mouvement''
qui provient de l'\'etape de r\'esolution de la quantit\'e de mouvement du pas
de temps pr\'ec\'edent, $\vect{Q}^n = \rho^n \vect{u}^n$.
On pourrait th�oriquement utiliser un vecteur quantit\'e de mouvement issu
de l'\'etape acoustique du pas de temps pr\'ec\'edent, mais il ne constitue
qu'un ``pr\'edicteur'' plus ou moins satisfaisant (il n'a pas ``vu'' les termes
sources qui ne sont pas dans  $\vect{f}^n$) et cette solution
n'a pas �t� test�e.

\bigskip
On \'ecrit alors la pression sous la forme~:
\begin{equation}
\gradv{P}=c^2\,\gradv{\rho}+\beta\,\gradv{s}
\end{equation}

avec $c^2 = \left.\displaystyle\frac{\partial P}{\partial \rho}\right|_s$
et $\beta = \left.\displaystyle\frac{\partial P}{\partial s}\right|_\rho$
tabul\'es ou analytiques \`a partir de la loi d'\'etat.

On discr\'etise l'expression pr\'ec\'edente en~:
\begin{equation}
\gradv{P^*}=(c^2)^n\gradv(\rho^{n+1})+\beta^n\gradv(s^n)
\end{equation}

On obtient alors une \'equation
portant sur $\rho^{n+1}$ en substituant l'expression de $\vect{Q}_{ac}^{n+1}$
issue de la 2\textsuperscript{\`eme} \'equation
de~(\ref{Cfbl_Cfmsvl_eq_acoustique_discrete_cfmsvl})
dans la 1\textsuperscript{\`ere} \'equation
de~(\ref{Cfbl_Cfmsvl_eq_acoustique_discrete_cfmsvl})~:
\begin{equation}\label{Cfbl_Cfmsvl_eq_densite_cfmsvl}
\displaystyle\frac{\rho^{n+1}-\rho^n}{\Delta t^n}
+\divs(\vect{w}^n \rho^n)
-\divs\left(\Delta t^n (c^2)^n \gradv(\rho^{n+1})\right) = 0
\end{equation}

o\`u~:
\begin{equation}
\begin{array}{lll}
\vect{w}^n&=&  \vect{u}^n + \Delta t^n
\displaystyle\left(\vect{f}^n-\frac{\beta^n}{\rho^n}\gradv(s^n)\right)
\end{array}
\end{equation}

Formulation alternative (programm\'ee mais non test\'ee)
avec le terme de convection implicite~:
\begin{equation}\label{Cfbl_Cfmsvl_eq_densite_bis_cfmsvl}
\displaystyle\frac{\rho^{n+1}-\rho^n}{\Delta t^n}
+\divs(\vect{w}^n \rho^{n+1})
-\divs\left(\Delta t^n (c^2)^n \gradv(\rho^{n+1})\right) = 0
\end{equation}


%---------------------------------
\subsection*{Discr\'etisation en espace}
%---------------------------------


%.................................
\subsubsection*{Introduction}
%.................................

On int\`egre l'\'equation pr\'ec\'edente ( (\ref{Cfbl_Cfmsvl_eq_densite_cfmsvl})
ou (\ref{Cfbl_Cfmsvl_eq_densite_bis_cfmsvl}) ) sur la cellule $i$ de volume $\Omega_i$.
On transforme les int\'egrales de volume en int\'egrales surfaciques
et l'on discr\'etise ces int\'egrales. Pour simplifier l'expos�, on se
place sur une cellule $i$ dont aucune face n'est sur le bord du domaine.

On obtient alors l'\'equation discr\`ete
suivante\footnote{L'exposant $^{n+\frac{1}{2}}$ signifie que le terme
peut \^etre implicite ou explicite. En pratique on a choisi
$\rho^{n+\frac{1}{2}} = \rho^{n}$.}~:
\begin{equation}\label{Cfbl_Cfmsvl_eq_densite_discrete_cfmsvl}
\Omega_i \displaystyle\frac{\rho_i^{n+1}-\rho_i^n}{\Delta t^n}
+\sum\limits_{j\in Vois(i)}(\rho^{n+\frac{1}{2}} \vect{w}^n)_{ij} \cdot \vect{S}_{ij}
-\sum\limits_{j\in Vois(i)} \left(\Delta t^n (c^2)^n
\gradv(\rho^{n+1})\right)_{ij} \cdot \vect{S}_{ij}
= 0
\end{equation}

%.................................
\subsubsection*{Discr\'etisation de la partie ``convective''}
%.................................

La valeur \`a la face s'\'ecrit~:
\begin{equation}
(\rho^{n+\frac{1}{2}} \vect{w}^n)_{ij} \cdot \vect{S}_{ij}
= \rho^{n+\frac{1}{2}}_{ij} \vect{w}^n_{ij} \cdot \vect{S}_{ij}
\end{equation}
avec, pour $\vect{w}^n_{ij}$,
une simple interpolation lin\'eaire~:
\begin{equation}
\vect{w}^n_{ij}
= \alpha_{ij} \vect{w}^n_i + (1-\alpha_{ij}) \vect{w}^n_j
\end{equation}
et un d\'ecentrement sur la valeur de $\rho^{n+\frac{1}{2}}$ aux faces~:
\begin{equation}
\begin{array}{lllll}
\displaystyle\rho_{ij}^{n+\frac{1}{2}} &=& \rho_{I'}^{n+\frac{1}{2}}
                &\text{si\ }& \vect{w}^n_{ij} \cdot \vect{S}_{ij} \geqslant 0 \\
                         &=& \rho_{J'}^{n+\frac{1}{2}}
                &\text{si\ }& \vect{w}^n_{ij} \cdot \vect{S}_{ij} < 0 \\
\end{array}
\end{equation}
que l'on peut noter~:
\begin{equation}
\displaystyle\rho_{ij}^{n+\frac{1}{2}}
 = \beta_{ij}\rho_{I'}^{n+\frac{1}{2}} + (1-\beta_{ij})\rho_{J'}^{n+\frac{1}{2}}
\end{equation}
avec
\begin{equation}
\left\{\begin{array}{lll}
\beta_{ij} = 1 & \text{si\ } & \vect{w}^n_{ij} \cdot \vect{S}_{ij} \geqslant 0 \\
\beta_{ij} = 0 & \text{si\ } & \vect{w}^n_{ij} \cdot \vect{S}_{ij} < 0 \\
\end{array}\right.
\end{equation}

%.................................
\subsubsection*{Discr\'etisation de la partie ``diffusive''}
%.................................

La valeur \`a la face s'\'ecrit~:
\begin{equation}
\left(\Delta t^n (c^2)^n \gradv(\rho^{n+1})\right)_{ij}\cdot \vect{S}_{ij}
= \Delta t^n (c^2)^n_{ij}
\displaystyle \left( \frac{\partial \rho}{\partial n} \right)^{n+1}_{ij}S_{ij}
\end{equation}
avec, pour assurer la continuit\'e du flux normal \`a l'interface,
une interpolation harmonique de $(c^2)^n$~:
\begin{equation}\label{Cfbl_Cfmsvl_eq_harmonique_cfmsvl}
\displaystyle(c^2)_{ij}^n
= \frac{(c^2)_{i}^n (c^2)_{j}^n}
{\alpha_{ij}(c^2)_{i}^n+(1-\alpha_{ij})(c^2)_{j}^n}
\end{equation}
et un sch\'ema centr\'e pour le gradient normal aux faces~:
\begin{equation}
\displaystyle \left( \frac{\partial \rho}{\partial n} \right)^{n+1}_{ij}
= \displaystyle\frac{\rho_{J'}^{n+1}-\rho_{I'}^{n+1}}{\overline{I'J'}}
\end{equation}

%.................................
\subsubsection*{Syst\`eme final}
%.................................

On obtient maintenant le syst\`eme final, portant sur
$(\rho_i^{n+1})_{i=1 \ldots N}$~:
\begin{equation}\label{Cfbl_Cfmsvl_eq_densite_finale_cfmsvl}
\displaystyle\frac{\Omega_i}{\Delta t^n} (\rho_i^{n+1}-\rho_i^n)
+\sum\limits_{j\in Vois(i)}\rho_{ij}^{n+\frac{1}{2}}
\vect{w}_{ij}^n \cdot \vect{S}_{ij}
-\sum\limits_{j\in Vois(i)} \Delta t^n (c^2)_{ij}^n
\displaystyle\frac{\rho_{J'}^{n+1}-\rho_{I'}^{n+1}}{\overline{I'J'}}\ S_{ij}
= 0
\end{equation}



%.................................
\subsubsection*{Remarque~: interpolation aux faces pour le terme de diffusion}
%.................................

Le choix de la forme de la moyenne pour le cofacteur du flux
normal n'est pas sans cons\'equence sur la vitesse de convergence, surtout
lorsque l'on est en pr\'esence de fortes inhomog\'en\'eit\'es.

On utilise une interpolation harmonique pour $c^2$
afin de conserver la continuit\'e du flux diffusif normal
$\Delta t (c^2) \displaystyle\frac{\partial \rho}{\partial n}$
\`a l'interface $ij$. En effet, on suppose que le flux est d\'erivable \`a
l'interface. Il doit donc y \^etre continu.\\
%
\'Ecrivons la continuit\'e du flux normal \`a l'interface,
avec la discr\'etisation
suivante\footnote{On ne reconstruit pas les valeurs de $\Delta\,t\,c^2$
aux points $I'$ et
$J'$.}~:
\begin{equation}
\left(\Delta t (c^2)\displaystyle\frac{\partial \rho}{\partial n}\right)_{ij}
= \Delta t (c^2)_i  \displaystyle\frac{\rho_{ij} - \rho_{I'}   }{\overline{I'F}}
=  \Delta t (c^2)_j  \displaystyle\frac{\rho_{J'}    - \rho_{ij}}{\overline{FJ'}}
\end{equation}
En \'egalant les flux \`a gauche et \`a droite de l'interface, on obtient
\begin{equation}
\rho_{ij} = \displaystyle\frac{\overline{I'F}\,(c^2)_j\rho_{J'} + \overline{FJ'}\,(c^2)_i\rho_{I'}}
{\overline{I'F}\,(c^2)_j + \overline{FJ'}\,(c^2)_i}
\end{equation}
On introduit cette formulation dans la d\'efinition du flux (par exemple, du
flux \`a gauche)~:
\begin{equation}
\left(\Delta t (c^2)\displaystyle\frac{\partial \rho}{\partial n}\right)_{ij}
= \Delta t (c^2)_i  \displaystyle\frac{\rho_{ij} - \rho_{I'}   }{\overline{I'F}}
\end{equation}
et on utilise la d\'efinition de $(c^2)_{ij}$ en fonction de ce m\^eme flux
\begin{equation}
\left(\Delta t (c^2)\displaystyle\frac{\partial \rho}{\partial n}\right)_{ij}
 \stackrel{\text{d\'ef}}{=}
 \Delta t (c^2)_{ij} \displaystyle\frac{\rho_{J'}    - \rho_{I'}   }{\overline{I'J'}}
\end{equation}
pour obtenir la valeur de $(c^2)_{ij}$ correspondant \`a l'\'equation (\ref{Cfbl_Cfmsvl_eq_harmonique_cfmsvl})~:
\begin{equation}
(c^2)_{ij} = \displaystyle\frac{\overline{I'J'}\,(c^2)_i(c^2)_j}{\overline{FJ'}\,(c^2)_i + \overline{I'F}\,(c^2)_j}
\end{equation}

%%%%%%%%%%%%%%%%%%%%%%%%%%%%%%%%%%
%%%%%%%%%%%%%%%%%%%%%%%%%%%%%%%%%%
\section*{Mise en \oe uvre}
%%%%%%%%%%%%%%%%%%%%%%%%%%%%%%%%%%
%%%%%%%%%%%%%%%%%%%%%%%%%%%%%%%%%%
Le syst\`eme (\ref{Cfbl_Cfmsvl_eq_densite_finale_cfmsvl}) est r\'esolu par une m\'ethode
d'incr\'ement et r\'esidu en utilisant
une m\'ethode de Jacobi pour inverser le syst\`eme si le terme convectif
est implicite et en utilisant une m\'ethode de gradient conjugu\'e
si le terme convectif est explicite (qui est le cas par d�faut).

Attention, les valeurs du flux de masse $\rho\,\vect{w}\cdot\vect{S}$ et
de la viscosit\'e $\Delta\,t\,c^2\frac{S}{d}$ aux faces de
bord, qui sont calcul\'ees dans \fort{cfmsfl} et \fort{cfmsvs} respectivement,
sont modifi\'ees imm\'ediatement apr\`es l'appel \`a ces sous-programmes.
En effet, il est indispensable que la contribution de bord de
$\left(\rho\,\vect{w}-\Delta\,t\,(c^2)\,\gradv\,\rho\right)\cdot\vect{S}$
repr\'esente exactement $\vect{Q}_{ac}\cdot\vect{S}$.
Pour cela,
\begin{itemize}
\item imm\'ediatement apr\`es l'appel \`a
\fort{cfmsfl}, on remplace la contribution de bord de
$\rho\,\vect{w}\cdot\vect{S}$
par le flux de masse exact, $\vect{Q}_{ac}\cdot\vect{S}$,
d\'etermin\'e \`a partir des conditions aux limites,
\item puis, imm\'ediatement apr\`es l'appel \`a
\fort{cfmsvs}, on annule la viscosit\'e au bord $\Delta\,t\,(c^2)$ pour
\'eliminer la contribution de $-\Delta\,t\,(c^2)\,(\gradv\,\rho)\cdot\vect{S}$
(l'annulation de la viscosit\'e n'est pas probl\'ematique pour la matrice,
puisqu'elle porte sur des incr\'ements).
\end{itemize}

\bigskip

Une fois qu'on a obtenu $\rho^{n+1}$,
on peut actualiser le flux de masse acoustique
aux faces $(\vect{Q}_{ac}^{n+1})_{ij} \cdot \vect{S}_{ij}$,
qui servira pour la convection des autres variables~:
\begin{equation}\label{Cfbl_Cfmsvl_eq_flux_masse_acoustique_cfmsvl}
\displaystyle(\vect{Q}_{ac}^{n+1})_{ij}\cdot\vect{S}_{ij}=
-\left(\Delta t^n (c^2)^n \gradv(\rho^{n+1})\right)_{ij}\cdot\vect{S}_{ij}
+\left(\rho^{n+\frac{1}{2}} \vect{w}^n\right)_{ij}\cdot\vect{S}_{ij}\\
\end{equation}
Ce calcul de flux est r\'ealis\'e par \fort{cfbsc3}.
Si l'on a choisi l'algorithme standard, \'equation~(\ref{Cfbl_Cfmsvl_eq_densite_cfmsvl}),
on compl\`ete le flux dans \fort{cfmsvl} imm\'ediatement apr\`es l'appel
\`a \fort{cfbsc3}.
En effet, dans ce cas,
la convection est explicite ($\rho^{n+\frac{1}{2}}=\rho^{n}$,
obtenu en imposant \var{ICONV(ISCA(IRHO(IPHAS)))=0})
et le sous-programme \fort{cfbsc3},
qui calcule le flux de masse aux faces,
ne prend pas en compte la contribution du terme
$\rho^{n+\frac{1}{2}}\,\vect{w}^n\cdot\vect{S}$. On ajoute donc cette
contribution dans \fort{cfmsvl}, apr\`es l'appel \`a \fort{cfbsc3}.
Au bord, en particulier, c'est bien le flux de masse calcul\'e \`a partir
des conditions aux limites que l'on obtient.

On actualise la pression \`a la fin de l'\'etape, en utilisant la loi d'\'etat~:
\begin{equation}
\displaystyle P_i^{pred}=P(\rho_i^{n+1},\varepsilon_i^{n})
\end{equation}


%%%%%%%%%%%%%%%%%%%%%%%%%%%%%%%%%%
%%%%%%%%%%%%%%%%%%%%%%%%%%%%%%%%%%
\section*{Points \`a traiter}
%%%%%%%%%%%%%%%%%%%%%%%%%%%%%%%%%%
%%%%%%%%%%%%%%%%%%%%%%%%%%%%%%%%%%
Le calcul du flux de masse au  bord n'est pas enti\`erement satisfaisant
si la convection est trait\'ee de mani\`ere implicite
(algorithme non standard, non test\'e,
associ\'e \`a l'\'equation~(\ref{Cfbl_Cfmsvl_eq_densite_bis_cfmsvl}),
correspondant au choix $\rho^{n+\frac{1}{2}}=\rho^{n+1}$ et
obtenu en imposant \var{ICONV(ISCA(IRHO(IPHAS)))=1}).
En effet, apr\`es \fort{cfmsfl}, il faut d\'eterminer la vitesse de
convection $\vect{w}^n$ pour qu'apparaisse
$\rho^{n+1} \vect{w}^n\cdot\vect{n}$
au cours de la r\'esolution par \fort{codits}. De ce fait, on doit d\'eduire
une valeur de $\vect{w}^n$ \`a partir de la valeur
du flux de masse. Au bord, en particulier, il faut
donc diviser le flux de masse
issu des conditions aux limites par la valeur de bord de $\rho^{n+1}$.
Or, lorsque des conditions de Neumann sont appliqu\'ees \`a la
masse volumique,
la valeur de $\rho^{n+1}$ au bord n'est pas connue avant la r\'esolution du
syst\`eme. On utilise donc, au lieu de la valeur de bord inconnue de
$\rho^{n+1}$ la valeur de bord prise au pas de temps
pr\'ec\'edent $\rho^{n}$. Cette approximation est susceptible
d'affecter la valeur du flux de masse au bord.
