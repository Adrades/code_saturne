%-------------------------------------------------------------------------------

% This file is part of Code_Saturne, a general-purpose CFD tool.
%
% Copyright (C) 1998-2011 EDF S.A.
%
% This program is free software; you can redistribute it and/or modify it under
% the terms of the GNU General Public License as published by the Free Software
% Foundation; either version 2 of the License, or (at your option) any later
% version.
%
% This program is distributed in the hope that it will be useful, but WITHOUT
% ANY WARRANTY; without even the implied warranty of MERCHANTABILITY or FITNESS
% FOR A PARTICULAR PURPOSE.  See the GNU General Public License for more
% details.
%
% You should have received a copy of the GNU General Public License along with
% this program; if not, write to the Free Software Foundation, Inc., 51 Franklin
% Street, Fifth Floor, Boston, MA 02110-1301, USA.

%-------------------------------------------------------------------------------

\programme{Thermodynamics}
{\huge sub-routines : pptbht, cothht, colecd, cplecd, fulecd ...}

%%%%%%%%%%%%%%%%%%%%%%%%%%%%%%%%%%
%%%%%%%%%%%%%%%%%%%%%%%%%%%%%%%%%%
\section{Function}
%%%%%%%%%%%%%%%%%%%%%%%%%%%%%%%%%%
%%%%%%%%%%%%%%%%%%%%%%%%%%%%%%%%%%

The description of the thermodynamical of gases mixture is as close as possible
of the JANAF standard. The gases mixture is, often, considered as composed of
some {\em global} species ({\small eg. oxidizer, products, fuel}) each of them
beeing a mixture ({\small with known ratio}) of {\em elementary} species
({\small oxygen, nitrogen, carbon dioxide, ...}).\\ A tabulation of the enthalpy
of both elementary and global species for some temperatures is constructed
({\small using JANAF polynoms}) or read ({\small if the user found useful to
define a global specie not simply related to elementary ones ; eg. unspecified
hydrocarbon known by C, H, O, N, S analysis and heating value.}).\\ The
thermodynamic properties of condensed phase are more simple : formation enthalpy
is computed using properties of gaseous products of combustion with air ({\small
formation enthalpy of wich is zero valued as O2 and N2 are reference state}) and
the lower heating value. The heat capacity of condensed phase is assumed
constant and it is a data the user have to type ({\small in the corresponding
data file dp\_FCP or dp\_FUE}).


%d
%%%%%%%%%%%%%%%%%%%%%%%%%%%%%%%%%%
%%%%%%%%%%%%%%%%%%%%%%%%%%%%%%%%%%
\section{Gases enthalpy discretisation}
%%%%%%%%%%%%%%%%%%%%%%%%%%%%%%%%%%
%%%%%%%%%%%%%%%%%%%%%%%%%%%%%%%%%%

A table of gases ({\small both elementary species and global ones}) enthalpy for
some temperatures ({\small user choses number of points, temperature in dp\_***
file}) is computed ({\small enthalpy of elementary species is computed using
JANAF polynomia ; enthalpy for global species are computed by weighting of
elementary ones})or read ({\small subroutine PPTBHT}). Then the entahlpy is
supposed to be linear vs. temperature in each temperature gap ({\small
i.e. continuous piece wise linear on the whole temperature range}). As a
consequence, temperature is a linear function of entahlpy ; and a simple
algorithm ({\small subroutine COTHHT}) allows to determine the enthalpy of a
mixture of gases ({\small for inlet conditions it is more useful to indicate
temperature and mass fractions}) or to determine temperature from enthalpy of
the mixture and mass fractions of global species ({\small common use in every
fluid particle, at every time step}).
%%%%%%%%%%%%%%%%%%%%%%%%%%%%%%%%%%
%%%%%%%%%%%%%%%%%%%%%%%%%%%%%%%%%%
\section{Particles enthalpy discretisation}
%%%%%%%%%%%%%%%%%%%%%%%%%%%%%%%%%%
%%%%%%%%%%%%%%%%%%%%%%%%%%%%%%%%%%

Enthalpy of condensed material are rarely known. Commonly, the thermal power and
ultimate analysis are determined. So, using simple assumptions and the enthalpy
of known released species ({\small after burning with air}) the formation
enthalpy of coal or heavy oil can be computed. Assuming the thermal capacity is
constant for every condensed material a table can be build with ... two
temperatures, allowing the use of the same simple algorithm for
temperature-enthalpy conversion. When intermediate gaseous species ({\small
volatile or vapour}) are thermodynamically known, simple assumptions({\small eg
: char is thermodynamically equivalent to pure carbon in reference state ; ashes
are inert}) allow to deduce enthalpy for heterogeneous reactions ({\small these
energies have not to be explicitely taken in account for the energy budget of
particles}).
