%-------------------------------------------------------------------------------

% This file is part of Code_Saturne, a general-purpose CFD tool.
%
% Copyright (C) 1998-2011 EDF S.A.
%
% This program is free software; you can redistribute it and/or modify it under
% the terms of the GNU General Public License as published by the Free Software
% Foundation; either version 2 of the License, or (at your option) any later
% version.
%
% This program is distributed in the hope that it will be useful, but WITHOUT
% ANY WARRANTY; without even the implied warranty of MERCHANTABILITY or FITNESS
% FOR A PARTICULAR PURPOSE.  See the GNU General Public License for more
% details.
%
% You should have received a copy of the GNU General Public License along with
% this program; if not, write to the Free Software Foundation, Inc., 51 Franklin
% Street, Fifth Floor, Boston, MA 02110-1301, USA.

%-------------------------------------------------------------------------------

\programme{fubase}

%%%%%%%%%%%%%%%%%%%%%%%%%%%%%%%%%%
%%%%%%%%%%%%%%%%%%%%%%%%%%%%%%%%%%
\section{Function}
%%%%%%%%%%%%%%%%%%%%%%%%%%%%%%%%%%
%%%%%%%%%%%%%%%%%%%%%%%%%%%%%%%%%%

The combustion of heavy fuel oil is described. After a droplet enters
the furnace, its temperature increases reaching an evaporating range
({\small heavy fuel oil is'nt a specy with a well defined boiling
point}):
\begin{enumerate}
  \item As droplet's temperature reachs the begin of evaporating
          range, vapour begins to be given off. Gaseous hydrocarbons
          are then involved in a diffusion flame surrouding each
          droplet. The heavy fuel oil beeing constitued of long
          hydrocarbons molecules, cracking mays occur and char is
          leaved. So the complex chemistry inside the droplet is
          summarized by: \\ Heavy fuel oil $\longrightarrow$ gaseous
          hydrocarbons + char
 
  \item After evaporation, when the whole vapour are burnt, oxygen is
  able to reach the surface of the char particle. So heterogenous
  combustion can take place. It is very similar to the oxydation of
  coal char: diffusion of oxygen from bulk, heterogeneous reaction
  ({\small kinetically limited}) and back diffusion of carbone
  monoxide. The heterogenous combustion is achieved if all the carbon
  of the char particle is converted, leaving an ash particle ({\small
  heavy fuel oil can contain inerts, some of which beeing heavy
  metals}). Unburnt carbon can leave the boiler as flying
  particle. The heterogeneous reaction is wrotten: \\
\centerline{$Char + \frac{1}{2} 0_{2} =(k0het, T0het)=> CO$} 
\end{enumerate}

\vspace{0.5cm}
\noindent{} Fuel jet combsution strongly depends on injection conditions and namely on mean diameter of drops (size distribution). In the first (2006) version \CS take in account only one diameter: monodisperse injection.


\vspace{0.5cm}

\noindent{} For heavy fuel oil, like for coal, the french reference [1] can be useful:
 
\noindent{\bf [1]} Escaich, Alain~: ``Mise en oeuvre dans Code\_Saturne des
mod�lisations physiques particuli�res. Tome 2: Combustion du charbon
pulv�ris�.'', HI-81/02/09/A, Rapport EDF, 2002.

\newpage
%=================================
\subsection{Notations}
%=================================

\begin{table}[h!]
\begin{tabular}{ccp{10,5cm}}

{\bf Symbol} & {\bf Unit} & {\bf Meaning}\\


$H$ 		& $J/kg$ 	& specific enthalpy \\
$K$ 		& $kg/(m.\,s)$ 	& thermal diffusivity\\
$\lambda$ 	& $W/(m.\,K)$ 	& thermal conductivity\\
$\mu$	 	& $kg/(m.\,s)$ 	& dynamical viscosity\\
$\rho$ 		& $kg/m^3$ 	& density\\
$M$, $M_i$ 	& $kg/mol$ 	& molar mass ($M_i$ for  $i$� constituant)\\
$P$ 		& $Pa$ 		& pressure\\
$R$ 		& $J/(mol.\,K)$ & perfect gas constant\\
$T$ 		& $K$ 		& temperature ($>0$)\\
$Y_i$ 		& 		& mass fraction of constituant $i$ 
					($0 \leqslant Y_i \leqslant 1$)\\
$D^{t}$         & $kg/(m.\,s)$  & turbulent viscosity \\
$\alpha_{i}$    & 		& mass fraction of phase k \\
$t$ 		& $s$ 		& time\\
\end{tabular}
\end{table}

\clearpage

%=================================
\subsection{Budget Equations}
%=================================

The bulk, done of gases and droplets, is assumed to be describable
with only one pressure and velocity. The slipping velocitiy between
droplets and gases is supposed negligible compared to this mean
velocity.  Scalars for the bulk are:
\vspace{0.5cm}
\begin{itemize}
  \item Bulk density 
     \begin{equation} 
        \rho_{m} = \alpha_{1}\rho_{1} + \alpha_{2}\rho_{2}
     \end{equation} 
  \item Bulk velocity 
     \begin{equation} 
       U_{m} = \frac{ \alpha_{1}\rho_{1} U_{1} 
                    + \alpha_{2}\rho_{2} U_{2} }{\rho_{m}}
     \end{equation} 
  \item Bulk enthalpy 
     \begin{equation} 
        H_{m} = \frac{ \alpha_{1}\rho_{1} H_{1} 
                     + \alpha_{2}\rho_{2} H_{2} }{\rho_{m}}
     \end{equation} 
  \item Bulk pressure 
     \begin{equation} 
       P_{m} = P_{1}
     \end{equation} 
\end{itemize}  

Mass fractions of gaseous medium ($Y_{1}^{*}$) and of droplets are defined by:
\begin{eqnarray}
  Y_{1}^{*} = \frac{\alpha_{1}\rho_{1}}{\rho_{m}} &\\
  Y_{2}^{*} = \frac{\alpha_{2}\rho_{2}}{\rho_{m}} &
\end{eqnarray}

So budget equations for the bulk can be written:

\begin{equation}
  \frac{\partial}{\partial t    } \rho_{m}
 +\frac{\partial}{\partial x_{j}} (\rho_{m}U_{m,j}) = 0 
\end{equation}

\begin{equation}
  \frac{\partial}{\partial t    } (\rho_{m}U_{m,i})
 +\frac{\partial}{\partial x_{j}} (\rho_{m}U_{m,i}U_{m,j})
       =  \frac{\partial}{\partial x_{j}}
              \left[ \rho_{m} \left[ D_{m}^{t}( \frac{\partial U_{m,i}}{\partial x_{j}}
                                  +\frac{\partial U_{m,i}}{\partial x_{j}} ) \\
                      -\frac{2}{3}\delta_{ij}
                            ( q_{m}^{2}
                             +D_{m}^{t}\frac{\partial U_{m,l}}{\partial x_{l}}) \right]  \right]
           - \frac{\partial P_{m}}{\partial x_{i}}+\rho_{m}g_{i}
\end{equation}

\begin{equation}
  \frac{\partial}{\partial t    } (\rho_{m} H_{m})
 +\frac{\partial}{\partial x_{j}} (\rho_{m}U_{m,j}H_{m})
              = \frac{\partial}{\partial x_{j}} 
                       (\rho_{m}D_{m}^{t} \frac{\partial H_{m}}{\partial x_{j}})
               +S_{m,R}
\end{equation}
 
With the ({\small velocity}) homogeneity assumption, mainly budget equation for bulk caracteristic  are pertinent. So transport equation for the scalar $\Phi_{k}$, where k is the phase, can be written:

\begin{equation}
  \frac{\partial}{\partial t    } (\rho_{m} Y_{k}^{*}\Phi_{k})
 +\frac{\partial}{\partial x_{j}} (\rho_{m} U_{m,j} Y_{k}^{*} \Phi_{k})
              = \frac{\partial}{\partial x_{j}} 
                       (\rho_{m}D_{m}^{t} \frac{\partial Y_{k}^{*} \Phi_{k}}{\partial x_{j}})
               +S_{\Phi_{k}}+\Gamma_{\Phi_{k}}
\end{equation}

%=================================
\subsection{Fuel combustion scalars}
%=================================

\subsubsection{Bulk enthalpy: $H_{m}$ }
Budget equation for the specific enthalpy of the mixture ({\small gas
+ droplets}) admits only one source term for radiative effects
$S_{m,R}$:
\begin{equation}
    S_{m,R}= S_{1,R}+ S_{2,R}
\end{equation}
With contributions of each phases liable to be described by different
models ({\small eg: wide band for gases, black body for particles}).

\subsubsection{Droplets enthalpy: $Y_{2}^{*}H_{2}$ }
Enthalpy of droplets ({\small J in droplets/kg bulk}) is the product
of liquid phase mass fraction ({\small kg liq/kg bulk}) by the
specific enthalpy of liquid ({\small kg liq/kg bulk}). So the budget
equation for liquid enthalpy has four source terms:
\begin{equation}
     \Pi_{2}^{'}+S_{2,R}-\Gamma_{evap}H_{vap}(T_{2})
                        +\Gamma_{het}\left( \frac{M_{O}}{M_{C}}H_{O_{2}}(T_{1})
                                      -\frac{M_{CO   }}{M_{C}}H_{CO   }(T_{2})\right) 
\end{equation}
with
\begin{itemize}
  \item $\Pi_{2}^{'}$: heat flux between phases
  \item $S_{2,R}$: radiative source term for droplets
  \item $\Gamma_{evap}H_{vap}(T_{2})$ the vapor flux leaves at droplet temperature ($H_{vap}$ includes latent heat)
   \item $\Gamma_{het}(...)$ heterogenous combustion induces reciprocal mass flux: oxygen arriving at gas temperature and carbone monoxide leaving at char particle one.

\end{itemize}
    
\subsubsection{Dispersed phase mass fraction: $Y_{2}^{*}$}
In budget equation for the mass fraction of the dispersed phase
({\small first droplets, then char particles, at last ashes}) the
source terms are interfacial mass fluxes ({\small first evaporation,
then net flux for heterogeneous combustion}):

\begin{equation}
     -\Gamma_{evap}-\Gamma_{het}
\end{equation}
          
\subsubsection{Number of droplets: $N_{p}^{*}$}
No source term in the budget equation for number of droplets: a
droplet became a particle ({\small eventually a tiny flying ash}) but
never vanish ({\small particles it have to get out}).

                                
\subsubsection{Mean of the passive scalar for fuel vapor: $F_{1}$}  
This scalar represent the amount of matter which have leaved the
particle as fuel vapour, whatever it happens after. It's a mass
fraction of gaseous matter ({\small in hydrocarbon form or carbon
oxide ones}). So the source term in its budget is only evaporation
mass flux:
\begin{equation}
   \Gamma_{evap}
\end{equation}     

\subsubsection{ Variance of $F_{1}$: $F_{1}^{'2}$}
Budget equation for $F_{1}^{'2}$ have three source term:
\begin{equation}
   \Gamma_{F_{1}^{'2}}
   -2\rho_{m}Y_{1}^{*}\frac{F_{1}^{'2}}{\tau_{\chi_{F_{1}^{'2}}}} 
   + \rho_{m}Y_{1}^{*}D_{m}^{t}\frac{\partial F_{1}}{\partial x_{j}} 
                               \frac{\partial F_{1}}{\partial x_{j}}
\end{equation} 
where $\Gamma_{F_{1}^{'2}}$ is due to interfacial mass fluxes ({\small
every interfacial mass fluxes impact gaseous phase variances}).
                                                 
\subsubsection{Mean of the passive scalar for carbon from char: $F_{3}$}  
Budget equation for $F_{3}$ have for lone source term the mass flux
due to heterogeneous combsution ({\small mass flux of carbon monoxide
minus oxygene mass flux}). As for $F_{1}$ oxidation in the gaseous
phase does not modifiy this {\em passive} scalar:
\begin{equation}
   \Gamma_{het}
\end{equation}   
         
                       
\subsubsection{ Variance of the passive scalar for air: $F_{4}^{'2}$} 
 
This passive scalar incomes with air but is'nt destroyed by any
({\small in gaseous phase or heterogeneous}). No budget equation
needed for it, $F_{4}$ can be determined from the wholeness
relation.\\

Budget equation for $F_{4}^{'2}$ have, like other passive scalar
variance budget equaiton, four source terms:
\begin{equation} 
    \Gamma_{F_{4}^{'2}}
   -2\rho_{m}Y_{1}^{*}\frac{F_{4}^{'2}}{\tau_{\chi_{F_{4}^{'2}}}} 
   + \rho_{m}Y_{1}^{*}D_{m}^{t}\frac{\partial F_{4}}{\partial x_{j}} 
                               \frac{\partial F_{4}}{\partial x_{j}}
\end{equation} 
where $\Gamma_{F_{4}^{'2}}$ is due to interfacial mass fluxes ({\small
every interfacial mass fluxes impact gaseous phase variances}).






%%%%%%%%%%%%%%%%%%%%%%%%%%%%%%%%%%
%%%%%%%%%%%%%%%%%%%%%%%%%%%%%%%%%%
\section{Discretisation}
%%%%%%%%%%%%%%%%%%%%%%%%%%%%%%%%%%
%%%%%%%%%%%%%%%%%%%%%%%%%%%%%%%%%%

The discretisation of equation are not problematic. Details are in
sections: \fort{fuflux} (interfacial fluxes of mass and energy),
\fort{futssc} (source term for fuel specific scalars)
and \fort{fucym1} (gas phase combustion). 
