%-----------------------------------------------------------------------
%
%     This file is part of the Code_Saturne Kernel, element of the
%     Code_Saturne CFD tool.
%
%     Copyright (C) 1998-2008 EDF S.A., France
%
%     contact: saturne-support@edf.fr
%
%     The Code_Saturne Kernel is free software; you can redistribute it
%     and/or modify it under the terms of the GNU General Public License
%     as published by the Free Software Foundation; either version 2 of
%     the License, or (at your option) any later version.
%
%     The Code_Saturne Kernel is distributed in the hope that it will be
%     useful, but WITHOUT ANY WARRANTY; without even the implied warranty
%     of MERCHANTABILITY or FITNESS FOR A PARTICULAR PURPOSE.  See the
%     GNU General Public License for more details.
%
%     You should have received a copy of the GNU General Public License
%     along with the Code_Saturne Kernel; if not, write to the
%     Free Software Foundation, Inc.,
%     51 Franklin St, Fifth Floor,
%     Boston, MA  02110-1301  USA
%
%-----------------------------------------------------------------------
%


\programme{elec**}

\vspace{0,5cm}
On s'int\'eresse \`a la r\'esolution des \'equations de la
magn\'etohydrodynamique, constitu\'ees de la r\'eunion des \'equations de
l'a\'erothermodynamique et des \'equations de Maxwell.

On se place dans deux cadres d'utilisation bien sp\'ecifiques et distincts,
qui permettront chacun de r\'ealiser des simplifications~: les \'etudes dites
``d'arc
\'electrique'' (dans lesquelles sont prises en compte les forces de Laplace et
l'effet Joule) et les \'etudes dites ``Joule'' (dans lesquelles seul
l'effet Joule est pris en compte).

Les \'etudes d'arc \'electrique sont associ\'ees en grande partie, pour EDF, aux
probl\'ematiques relatives aux transformateurs. Les \'etudes Joule sont plus
s\'ecifiquement li\'ees aux ph\'enom\`enes rencontr\'es dans les fours verriers.

Outre la prise en compte ou non des forces de Laplace, ces deux types d'\'etudes
se diff\'erencient \'egalement par le mode de d\'etermination de l'effet Joule
(utilisation d'un potentiel complexe pour les \'etudes Joule faisant intervenir
un courant alternatif non monophas\'e).

On d\'ecrit tout d'abord les \'equations r\'esolues pour les \'etudes d'arc
\'electrique. Les sp\'ecificit\'es des \'etudes Joule seront abord\'ees ensuite.


Pour l'arc \'electrique,
les r\'ef\'erences [douce] et [delalondre] pourront compl\'eter la
pr\'esentation~:

\noindent{\bf [delalondre] }Delalondre, Clarisse~: ``Mod\'elisation a\'erothermodynamique d'arcs
\'electriques \`a forte intensit\'e avec prise en compte du d\'es\'equilibre
thermodynamique local et du transfert thermique \`a la cathode'', Th\`ese de
l'Universit\'e de Rouen, 1990

\noindent{\bf [douce]} Douce, Alexandre~: ``Mod�lisation 3-D du chauffage d'un
bain m�tallique par plasma d'arc transf\'er\'e. Application \`a un r\'eacteur
axisym\'etrique'', HE-26/99/027A, HE-44/99/043A, Th\`ese de l'Ecole Centrale
Paris et EDF, 1999




%%%%%%%%%%%%%%%%%%%%%%%%%%%%%%%%%%
%%%%%%%%%%%%%%%%%%%%%%%%%%%%%%%%%%
\section{Fonction}
%%%%%%%%%%%%%%%%%%%%%%%%%%%%%%%%%%
%%%%%%%%%%%%%%%%%%%%%%%%%%%%%%%%%%

\subsection{Notations}

{\bf Variables utilis\'ees}
\nopagebreak

\begin{tabular}{lp{6cm}l}
$\vect{A}$        &potentiel vecteur r\'eel        &$kg\,m\,s^{-2}\,A^{-1}$ \\
$\vect{B}$        &champ magn\'etique                 &$T$ (ou $kg\,s^{-2}\,A^{-1}$) \\
$\vect{D}$         &d\'eplacement \'electrique        &$A\,s\,m^{-2}$ \\
$\vect{E}$         &champ \'electrique                 &$V\,m^{-1}$ \\
$E$                 &\'energie totale massique         &$J\,kg^{-1}$ (ou $m^{2}\,s^{-2}$) \\
$e$                 &\'energie interne massique         &$J\,kg^{-1}$ (ou $m^{2}\,s^{-2}$) \\
$e_c$                 &\'energie cin\'etique massique &$J\,kg^{-1}$ (ou $m^{2}\,s^{-2}$) \\
$\vect{H}$         &excitation magn\'etique        &$A\,m^{-1}$ \\
$h$                   &enthalpie massique                 &$J\,kg^{-1}$ (ou $m^{2}\,s^{-2}$) \\
$\vect{j}$        &densit\'e de courant                 &$A\,m^{-2}$  \\
$P$                   &pression                         &$kg\,m^{-1}\,s^{-2}$ \\
$P_R$, $P_I$         &potentiel scalaire r\'eel, imaginaire        
                                                &$V$ (ou $kg\,m^{2}\,s^{-3}\,A^{-1}$) \\
$\vect{u}$         &vitesse                        &$m\,s^{-1}$  \\
                 &                                & \\
$\varepsilon$ &permittivit\'e \'electrique
                                                &$F\,m^{-1}$ (ou $m^{-3}\,kg^{-1}\,s^{4}\,A^{2}$) \\
$\varepsilon_0$ &permittivit\'e \'electrique du vide
                                                &$8,854\,10^{-12}\,\,F\,m^{-1}$ (ou $m^{-3}\,kg^{-1}\,s^{4}\,A^{2}$) \\
$\mu$           &perm\'eabilit\'e \'electrique
                                                &$H\,m^{-1}$ (ou $m\,kg\,s^{-2}\,A^{-2}$)\\
$\mu_0$         &perm\'eabilit\'e \'electrique du vide
                                                &$4\,\pi\,10^{-7}\,\,H\,m^{-1}$ (ou $m\,kg\,s^{-2}\,A^{-2}$)\\
$\sigma$         &conductivit\'e \'electrique        &$S\,m^{-1}$ (ou $m^{-3}\,kg^{-1}\,s^3\,A^2$)\\
\end{tabular}

\vspace*{0,5cm}
{\bf Notations d'analyse vectorielle}
\nopagebreak

On rappelle \'egalement la d\'efinition des notations employ\'ees\footnote{en
utilisant la convention de sommation d'Einstein.}~:
\begin{equation}\notag
\left\{\begin{array}{lll}
\left[\ggrad{\vect{a}}\right]_{ij} &=& \partial_j a_i\\
\left[\dive(\tens{\sigma})\right]_i &=& \partial_j \sigma_{ij}\\
\left[\vect{a}\otimes\vect{b}\right]_{ij} &= &
a_i\,b_j\\
\end{array}\right.
\end{equation}
et donc :
\begin{equation}\notag
\begin{array}{lll}
\left[\dive(\vect{a}\otimes\vect{b})\right]_i &= &
\partial_j (a_i\,b_j)
\end{array}
\end{equation}


\subsection{Arcs \'electriques}

\subsubsection{Introduction}

Pour les \'etudes d'arc \'electrique, on calcule,
\`a un pas de temps donn\'e~:
\begin{itemize}
\item la vitesse $\vect{u}$, la pression $P$, la variable \'energ\'etique
enthalpie $h$ (et les grandeurs turbulentes),
\item un potentiel scalaire r\'eel $P_R$
(dont le gradient permet d'obtenir le champ \'electrique $\vect{E}$ et
la densit\'e de courant $\vect{j}$),
\item un potentiel vecteur r\'eel $\vect{A}$ (dont
le rotationnel permet d'obtenir  le champ magn\'etique $\vect{B}$).
\end{itemize}

\bigskip
Le champ \'electrique, la
densit\'e de courant et le champ magn\'etique sont utilis\'es pour calculer les
termes sources d'effet Joule et les forces de Laplace qui interviennent
respectivement dans l'\'equation de l'enthalpie et dans celle
de la quantit\'e de mouvement.


\subsubsection{\'Equations continues}

{\bf Syst\`eme d'\'equations}
\nopagebreak

Les \'equations continues qui sont r\'esolues sont les suivantes~:
\begin{equation}
\left\{\begin{array}{l}
{\color{blue}\dive(\rho \vect{u}) = 0}\\
{\color{blue}\displaystyle\frac{\partial}{\partial t}(\rho \vect{u})
+\dive(\rho\, \vect{u} \otimes \vect{u})
=\dive(\tens{\sigma}) + \vect{TS} + {\color{red}\vect{j} \times \vect{B}}}\\
{\color{blue}\displaystyle\frac{\partial}{\partial t}(\rho h)
+\dive(\rho\, \vect{u} h)
=\Phi_v -
\dive{\left(\left(\frac{\lambda}{C_p}+\frac{\mu_t}{\sigma_t}\right)\grad{h}\right)}
+ {\color{red}P_J}}\\
{\color{red}\dive(\sigma\,\grad{P_R})=0}\\
{\color{red}\dive(\ggrad{\vect{A}})=-\mu_0\vect{j}}
\end{array}\right.
\end{equation}

avec les relations suivantes~:
\begin{equation}
\left\{\begin{array}{l}
{\color{red}P_J=\vect{j}\cdot\vect{E}}\\
{\color{red}\vect{E}=-\grad{P_R}}\\
{\color{red}\vect{j}=\sigma\vect{E}}\\
\end{array}\right.
\end{equation}

%On donne ci-apr\`es diff\'erents \'el\'ements permettant de pr\'eciser le mode
%d'obtention de ces \'equations.

\vspace*{0,5cm}
{\bf \'Equation de la masse}
\nopagebreak

C'est l'\'equation r\'esolue en standard par \CS (contrainte
stationnaire).
Elle n'a pas de traitement particulier dans
le cadre du module pr\'esent. Un terme source de  masse peut \^etre pris en
compte au second membre si l'utilisateur le souhaite. Pour simplifier l'expos\'e
le terme source sera suppos\'e nul ici, dans la mesure o\`u il n'est pas
sp\'ecifique au module \'electrique.

\vspace*{0,5cm}
{\bf \'Equation de la quantit\'e de mouvement}
\nopagebreak

Elle pr\'esente, par rapport \`a
l'\'equation standard r\'esolue par \CS, un seul terme additionnel
($\vect{j} \times \vect{B}$) qui rend compte des forces de Laplace.
Pour l'obtenir, on fait l'hypoth\`ese que le milieu est
\'electriquement neutre.

En effet, une charge $q_i$ (Coulomb) anim\'ee d'une
vitesse $\vect{v}_i$ subit,
sous l'effet du champ \'electrique $\vect{E}$ ($V\,m^{-1}$) et du champ magn\'etique
$\vect{B}$ (Tesla),  une force $\vect{f}_i$ ($kg\,m\,s^{-2}$)~:
\begin{equation}
\vect{f}_i=q_i\left(\vect{E} + \vect{v}_i \times \vect{B}\right)
\end{equation}
Avec $n_i$ charges de type $q_i$ par unit\'e de volume et en sommant sur tous
les types de charge $i$ (\'electrons, ions, mol\'ecules ionis\'ees...), on
obtient la force de Laplace totale $\vect{F}_L$ ($kg\,m^{-2}\,s^{-2}$) subie par unit\'e de
volume~:
\begin{equation}
\vect{F}_L=\sum\limits_i\left[{n_i\,q_i\left(\vect{E} + \vect{v}_i \times \vect{B}\right)}\right]
\end{equation}
On introduit alors la densit\'e de courant $\vect{j}$ ($A\,m^{-2}$)~:
\begin{equation}
\vect{j}=\sum\limits_i n_i\,q_i\,\vect{v}_i
\end{equation}
Avec l'hypoth\`ese que le milieu est \'electriquement neutre (\`a un
niveau macroscopique)~:
\begin{equation}
\sum\limits_i n_i\,q_i = 0
\end{equation}
la force totale $\vect{F}_L$ s'\'ecrit alors~:
\begin{equation}
\vect{F}_L=\vect{j} \times \vect{B}
\end{equation}
et on peut donc \'ecrire l'\'equation de la quantit\'e de mouvement~:
\begin{equation}
\displaystyle{\color{blue}\frac{\partial}{\partial t}(\rho \vect{u})
+\dive(\rho\, \vect{u} \otimes \vect{u})
=\dive(\tens{\sigma}) + \vect{TS} + {\color{red}\vect{j} \times \vect{B}}}
\end{equation}

\vspace*{0,5cm}
{\bf \'Equation de l'enthalpie}
\nopagebreak

Elle est obtenue \`a partir de l'�quation de
l'\'energie apr\`es plusieurs approximations utilis\'ees en standard dans \CS et en
prenant en compte le terme d'effet Joule li\'e \`a l'\'energie
\'electromagn\'etique.

\underline{\'Energie \'electromagn\'etique}
\nopagebreak

Avec les m\^emes notations que pr\'ec\'edemment mais sans qu'il soit
besoin de supposer que le milieu est \'electriquement neutre,
la puissance re\c cue par une charge $q_i$ (particule dou\'ee de masse)
de vitesse $\vect{v}_i$ (vitesse du porteur de charge, contenant \'eventuellement
l'effet de la vitesse du fluide) sous l'effet
du champ \'electrique $\vect{E}$ ($V\,m^{-1}$)
et du champ magn\'etique  $\vect{B}$ ($T$) est (sans
sommation sur $i$)~:
\begin{equation}
P_i=\vect{f}_i\cdot\vect{v}_i=
q_i(\vect{E}+\vect{v}_i\times\vect{B})\cdot\vect{v}_i
= q_i\vect{v}_i\cdot\vect{E}
\end{equation}
Avec $n_i$ charges par unit\'e de volume et en sommant sur tous les types
de charges $i$, on obtient la puissance totale par unit\'e de volume~:
\begin{equation}
P_J=
\sum\limits_i n_i\,q_i\,\vect{v}_i\cdot\vect{E}
\end{equation}
On introduit alors la densit\'e de courant $\vect{j}=\sum\limits_i n_i\,q_i\,\vect{v}_i$ (en $A\,m^{-2}$) et on obtient l'expression usuelle de la puissance
\'electromagn\'etique dissip\'ee par effet Joule (en $W\,m^{-3}$)~:
\begin{equation}
P_J=\vect{j}\cdot\vect{E}
\end{equation}

Pour reformuler la puissance dissip\'ee par effet Joule et obtenir
une \'equation d'\'evolution de l'\'energie \'electromagn\'etique, on utilise
alors les \'equations de Maxwell.
Les \'equations s'\'ecrivent (lois d'Amp\`ere
et de Faraday)~:
\begin{equation}
\left\{
\begin{array}{l}
\displaystyle\frac{\partial \vect{D}}{\partial t} - \rot\vect{H} = -\vect{j}\\
\displaystyle\frac{\partial \vect{B}}{\partial t} + \rot\vect{E} = 0
\end{array}
\right.
\end{equation}

On a donc~:
\begin{equation}
P_J=\vect{j}\cdot\vect{E}=\left(-\frac{\partial \vect{D}}{\partial t} +\rot\vect{H}\right)\cdot\vect{E}
\end{equation}
On utilise alors la relation suivante~:
\begin{equation}
\rot\vect{H}\cdot\vect{E}=\vect{H}\cdot\rot\vect{E}-\dive(\vect{E}\times\vect{H})
\end{equation}
En effet, elle permet de faire appara\^\i tre un
terme en divergence, caract\'eristique d'une redistribution spatiale~:
\begin{equation}
\begin{array}{lll}
\vect{j}\cdot\vect{E}&=&
\displaystyle-\frac{\partial \vect{D}}{\partial t}\cdot\vect{E}
+\vect{H}\cdot\rot\vect{E}-\dive(\vect{E}\times\vect{H})\\
\end{array}
\end{equation}
Et en utilisant la loi de Faraday pour faire appara\^\i tre la d\'eriv\'ee en temps du champ magn\'etique~:
\begin{equation}
\begin{array}{lll}
\vect{j}\cdot\vect{E}&=&
\displaystyle-\frac{\partial \vect{D}}{\partial t}\cdot\vect{E}
-\vect{H}\cdot\frac{\partial \vect{B}}{\partial t}-\dive(\vect{E}\times\vect{H})\\
\end{array}
\end{equation}

Dans le cadre de \CS, on fait les hypoth\`eses suivantes~:
\begin{itemize}
\item la perm\'eabilit\'e $\varepsilon$ et la permittivit\'e $\mu$
sont constantes et uniformes (pour les gaz, en pratique, on utilise
les propri\'et\'es du vide $\varepsilon_0$ et $\mu_0$).
\item on utilise $\vect{B} = \mu \vect{H}$ et $\vect{D} = \varepsilon \vect{E}$
\end{itemize}

On a alors~:

\begin{equation}
\begin{array}{lll}
\displaystyle\vect{j}\cdot\vect{E}&=&
\displaystyle-\frac{\varepsilon_0}{2}\frac{\partial E^2}{\partial t}
-\frac{1}{2\,\mu_0}\frac{\partial B^2}{\partial t}
-\frac{1}{\mu_0}\dive(\vect{E}\times\vect{B})
\end{array}
\end{equation}

\underline{\'Energie totale}
\nopagebreak

On \'etablit l'\'equation de l'\'energie totale en prenant en compte la
puissance des forces de Laplace et le terme
d'effet Joule.

Sans prendre en compte l'\'energie \'electromagn\'etique,
le premier principe de la thermodynamique s'\'ecrit d'ordinaire sous la
forme suivante (pour un volume mat\'eriel suivi sur une unit\'e de temps)~:
\begin{equation}\label{Elec_Elbase_premier_ppe_eq}
d\int_V \rho E dV=\delta Q+\delta W
\end{equation}
Dans cette relation, $E$ est l'\'energie totale par unit\'e de masse\footnote{Ne pas
confondre le scalaire $E$, \'energie totale, avec le vecteur $\vect{E}$, champ
\'electrique.}, soit $E=e+e_c$, $e$ \'etant l'\'energie interne massique et
$e_c=\frac{1}{2}\,\vect{u}\cdot\vect{u}$ l'\'energie cin\'etique massique. Le terme
$\delta Q$  repr\'esente  la chaleur re\c cue au travers des fronti\`eres du
domaine consid\'er\'e tandis que le terme  $\delta W$ repr\'esente le travail
des forces ext\'erieures re\c cu par le syst\`eme (y compris les forces
d\'erivant d'une \'energie potentielle).

Pour prendre en compte l'\'energie \'electromagn\'etique, il suffit d'int\'egrer
\`a la relation (\ref{Elec_Elbase_premier_ppe_eq}) la puissance des forces de Laplace $(\vect{j} \times
\vect{B})\cdot\vect{u}$ et le terme d'effet Joule $\vect{j}\cdot\vect{E}$
(transformation volumique d'\'energie \'electromagn\'etique en \'energie
totale\footnote{Le terme en divergence
$-\frac{1}{\mu_0}\dive(\vect{E}\times\vect{B})$
traduit une redistribution spatiale d'\'energie \'electromagn\'etique~:
ce n'est donc pas un terme source pour l'\'energie totale.}).
Dans cette relation, la vitesse $\vect{u}$ est la vitesse du fluide et non pas
celle des porteurs de charge~: elle n'est donc pas n\'ecessairement coli\'eaire
au vecteur \vect{j} (par exemple, si le courant est d\^u \`a des \'electrons,
la vitesse du fluide pourra \^etre consid\'er\'ee comme d\'ecorr\'el\'ee de la
vitesse des porteurs de charges~; par contre, si le courant est d\^u \`a des ions,
la vitesse du fluide pourra \^etre plus directement influenc\'ee par
le d\'eplacement des porteurs de charge).
Ainsi, le premier principe de la thermodynamique s'\'ecrit~:
\begin{equation}
d\int_V \rho E dV=\delta Q+\delta W+\vect{j}\cdot\vect{E}\,V\,dt +(\vect{j} \times
\vect{B})\cdot\vect{u}\,V\,dt
\end{equation}
et l'\'equation locale pour l'\'energie totale est alors~:
\begin{equation}
\displaystyle\frac{\partial}{\partial t}(\rho E)
+\dive(\rho\, \vect{u} E)
=\dive(\tens{\sigma}\,\vect{u}) + \vect{TS}\cdot\vect{u} + {\color{red}(\vect{j} \times
\vect{B})\cdot\vect{u}} + \Phi_v - \dive{\vect{\Phi}_s} + {\color{red}\vect{j}\cdot \vect{E}}
\end{equation}
Le terme $\Phi_v$ repr\'esente les termes sources volumiques d'\'energie
autres que l'effet Joule (par exemple, il inclut le terme source de
rayonnement, pour un milieu optiquement non transparent). Le terme $\vect{\Phi}_s$ est
le flux d'\'energie surfacique\footnote{Dans \CS, il est mod\'elis\'e par
une hypoth\`ese de gradient et inclut �galement la ``diffusion'' turbulente.}.

\underline{Enthalpie}
\nopagebreak

Pour obtenir une \'equation sur l'enthalpie, qui est la variable \'energ\'etique
choisie dans \CS dans le module \'electrique, on
soustrait tout d'abord \`a l'\'equation de l'\'energie totale celle de l'\'energie
cin\'etique pour obtenir une \'equation sur l'\'energie interne.

L'\'equation de l'\'energie
cin\'etique (obtenue \`a partir de l'\'equation de la quantit\'e de mouvement
\'ecrite sous forme non conservative) est~:
\begin{equation}
\displaystyle\frac{\partial}{\partial t}(\rho e_c)
+\dive(\rho\, \vect{u} e_c)
=\dive(\tens{\sigma}\,\vect{u}) - \tens{\sigma}:\left(\ggrad(\vect{u})\right)^t +
\vect{TS}\cdot\vect{u} + (\vect{j} \times \vect{B})\cdot\vect{u}
\end{equation}
de sorte que, pour l'\'energie interne, on a~:
\begin{equation}
\displaystyle\frac{\partial}{\partial t}(\rho e)
+\dive(\rho\, \vect{u} e)
=\tens{\sigma}:\left(\ggrad(\vect{u})\right)^t + \Phi_v - \dive{\Phi_s} + \vect{j}\cdot \vect{E}
\end{equation}
et enfin, pour l'enthalpie $h=e+\frac{P}{\rho}$~:
\begin{equation}
\displaystyle\frac{\partial}{\partial t}(\rho h)
+\dive(\rho\, \vect{u} h)
=\tens{\sigma}:\left(\ggrad(\vect{u})\right)^t + \Phi_v - \dive{\Phi_s} + \vect{j}\cdot \vect{E}+\rho\frac{d}{dt}\left(\frac{P}{\rho}\right)
\end{equation}
En faisant appara\^itre la pression dans le tenseur des contraintes
$\tens{\sigma}=-P\tens{Id}+\tens{\tau}$, on peut \'ecrire~:
\begin{equation}
\displaystyle\frac{\partial}{\partial t}(\rho h)
+\dive(\rho\, \vect{u} h)
=\tens{\tau}:\left(\ggrad(\vect{u})\right)^t + \Phi_v - \dive{\Phi_s}
+ \vect{j}\cdot\vect{E} + \frac{dP}{dt}
\end{equation}

Les approximations habituelles de \CS consistent alors
\`a n\'egliger le terme ``d'\'echauffement'' issu du tenseur des contraintes
$\tens{\tau}:\left(\ggrad(\vect{u})\right)^t$ et le terme en d\'eriv\'ee totale de la
pression $\frac{dP}{dt}$, suppos\'es faibles en comparaison des autres termes
dans les applications trait\'ees (exemple~: terme d'effet Joule important, effets de
compressibilit\'e faibles...).
De plus, le terme de flux est mod\'elis\'e en suivant
une hypoth\`ese de gradient appliqu\'e \`a l'enthalpie (et non pas \`a la
temp\'erature), soit donc~:
\begin{equation}
{\color{blue}\displaystyle\frac{\partial}{\partial t}(\rho h)
+\dive(\rho\, \vect{u} h)
=\Phi_v -
\dive{\left(\left(\frac{\lambda}{C_p}+\frac{\mu_t}{\sigma_t}\right)\grad h\right)} + {\color{red}\vect{j}\cdot\vect{E}}}
\end{equation}


\vspace*{0,5cm}
{\bf \'Equations \'electromagn\'etiques}
\nopagebreak

Elles sont obtenues \`a partir des
\'equations de Maxwell sous les hypoth\`eses d\'etaill\'ees dans [douce],
paragraphe 3.3.

\underline{Densit\'e de courant}
\nopagebreak

La relation liant la densit\'e de courant et le champ \'electrique est issue de
la loi d'Ohm que l'on suppose pouvoir utiliser sous la forme
simplifi\'ee suivante~:
\begin{equation}\label{Elec_Elbase_ohm_eq}
{\color{red}\vect{j}=\sigma\,\vect{E}}
\end{equation}

\underline{Champ \'electrique}
\nopagebreak

Le champ \'electrique s'obtient \`a partir d'un potentiel vecteur.

En effet, la loi de Faraday s'\'ecrit~:
\begin{equation}
\frac{\partial\vect{B}}{\partial t}+\rot\vect{E}=0
\end{equation}
Avec une hypoth\`ese quasi-stationnaire, il reste~:
\begin{equation}
\rot\vect{E}=0
\end{equation}
Il est donc possible de postuler l'existence d'un potentiel scalaire $P_R$
tel que~:
\begin{equation}\label{Elec_Elbase_e_eq}
{\color{red}\vect{E}=-\grad{P_R}}
\end{equation}

\underline{Potentiel scalaire}
\nopagebreak

Le potentiel scalaire est solution d'une \'equation de Poisson.

En effet, la conservation de la charge $q$ s'\'ecrit~:
\begin{equation}
\displaystyle\frac{\partial q}{\partial t}
+\dive(\vect{j}) = 0
\end{equation}
Pour un milieu \'electriquement neutre (\`a l'\'echelle macroscopique), on a
$\displaystyle\frac{\partial q}{\partial t}=0$ soit donc~:
\begin{equation}
\dive(\vect{j}) = 0
\end{equation}
C'est-\`a-dire, avec la loi d'Ohm (\ref{Elec_Elbase_ohm_eq}),
\begin{equation} \label{Elec_Elbase_div_sigma_e_eq}
\dive(\sigma\,\vect{E}) = 0
\end{equation}
Avec (\ref{Elec_Elbase_e_eq}), on obtient donc une \'equation permettant de
calculer le potentiel scalaire~:
\begin{equation}
{\color{red}\dive(\sigma\,\grad{P_R}) = 0}
\end{equation}

\underline{Champ magn\'etique}
\nopagebreak

Le champ magn\'etique s'obtient \`a partir d'un potentiel vecteur.

En effet, la loi d'Amp\`ere s'\'ecrit~:
\begin{equation}
\displaystyle\frac{\partial\vect{D}}{\partial t}-\rot\vect{H}=-\vect{j}
\end{equation}
Sous les hypoth\`eses indiqu\'ees pr\'ec\'edemment, on \'ecrit~:
\begin{equation}
\displaystyle\varepsilon_0\,\mu_0\,\frac{\partial\vect{E}}{\partial t}-\rot\vect{B}=-\mu_0\vect{j}
\end{equation}
Avec une hypoth\`ese quasi-stationnaire, il reste~:
\begin{equation}\label{Elec_Elbase_rot_b_eq}
\rot\vect{B}=\mu_0\vect{j}
\end{equation}
De plus, la conservation du flux magn\'etique s'\'ecrit\footnote{Prendre la
divergence de la loi de Faraday, avec $\dive(\rot\vect{E})=0$ (par analyse
vectorielle) donne $\dive\vect{B} = \text{cst}$.}~:
\begin{equation}
\dive\,\vect{B} = 0
\end{equation}
et on peut donc postuler l'existence d'un potentiel vecteur $\vect{A}$ tel que~:
\begin{equation}\label{Elec_Elbase_b_eq}
{\color{red}\vect{B} = \rot{\vect{A}}}
\end{equation}

\underline{Potentiel vecteur}
\nopagebreak

Le potentiel vecteur est solution d'une \'equation de Poisson.

En prenant le rotationnel de (\ref{Elec_Elbase_b_eq}) et avec (\ref{Elec_Elbase_rot_b_eq}), on obtient~:
\begin{equation}
-\rot(\rot{\vect{A}}) = -\mu_0\vect{j}
\end{equation}
Avec la relation donnant le Laplacien\footnote{En
coordonn\'ees cart\'esiennes, le
Laplacien du vecteur $\vect{a}$ est le vecteur dont les
composantes sont \'egales au Laplacien de chacune des composantes de $\vect{a}$.}
d'un vecteur $\dive(\ggrad\vect{a})=\grad(\dive{\vect{a}})-\rot(\rot\vect{a})$ et sous
la contrainte\footnote{La condition $\dive\vect{A}=0$, dite ``jauge de
Coulomb'', est n\'ecessaire pour assurer
l'unicit\'e du potentiel vecteur.}
que $\dive\vect{A}=0$, on obtient finalement une \'equation
permettant de calculer le potentiel vecteur~:
\begin{equation}
{\color{red}\dive\,(\ggrad{\vect{A}}) = -\mu_0\vect{j}}
\end{equation}



\subsection{Effet Joule}

\subsubsection{Introduction}

Pour les \'etudes Joule, on calcule,
\`a un pas de temps donn\'e~:
\begin{itemize}
\item la vitesse $\vect{u}$, la pression $P$, la variable \'energ\'etique
enthalpie $h$ (et les grandeurs turbulentes \'eventuelles),
\item un potentiel scalaire r\'eel $P_R$,
\item et, si le courant n'est ni continu, ni alternatif
monophas\'e, un potentiel scalaire imaginaire $P_I$.
\end{itemize}

\bigskip
Le gradient du potentiel permet d'obtenir  le champ \'electrique $\vect{E}$ et
la densit\'e de courant $\vect{j}$ (partie r\'eelle et, \'eventuellement, partie
imaginaire). Le champ \'electrique et la
densit\'e de courant sont utilis\'es pour calculer le
terme source d'effet Joule qui intervient dans l'\'equation de l'enthalpie.

{\bf La puissance instantan\'ee dissip\'ee par effet Joule} est \'egale
au produit instantan\'e $\vect{j}\cdot\vect{E}$.
Dans le cas g\'en\'eral, $\vect{j}$ et $\vect{E}$ sont des signaux alternatifs
($\vect{j}=\vect{|j|}cos(\omega\,t+\phi_j)$ et
$\vect{E}=\vect{|E|}cos(\omega\,t+\phi_E)$) que l'on peut repr\'esenter
par des complexes
($\vect{j}=\vect{|j|}\,e^{i\,(\omega\,t+\phi_j)}$  et
$\vect{E}=\vect{|E|}\,e^{i\,(\omega\,t+\phi_E)}$).
La puissance instantan\'ee s'\'ecrit alors
$(\vect{|j|}\cdot\vect{|E|})cos(\omega\,t+\phi_j)cos(\omega\,t+\phi_E)$.

\begin{itemize}

\item {\bf En courant continu} ($\omega=\phi_j=\phi_E=0$),
la puissance se calcule donc simplement comme
le produit scalaire $P_J=\vect{|j|}\cdot\vect{|E|}$.
Le calcul de la puissance
dissip\'ee par effet Joule ne pose donc pas de probl\`eme particulier
car les variables densit\'e de courant et champ \'electrique r\'esolues
par \CS sont pr\'ecis\'ement $\vect{|j|}$ et $\vect{|E|}$ (les variables sont r\'eelles).

\item {\bf En courant alternatif}, la periode du courant est beaucoup plus petite que
les \'echelles de temps des ph\'enom\`enes thermohydrauliques pris en compte.
Il n'est donc pas utile de disposer de la puissance instantan\'ee dissip\'ee
par effet Joule~: la moyenne sur une p\'eriode est suffisante et elle
s'\'ecrit\footnote{L'int\'egrale de $cos^2 x$ sur un intervalle de longueur
$2\,\pi$ est $\pi$.}~:
% eh oui, car l'integrale de cos^2+sin^2 (qui vaut 1), c'est 2 \pi !!
$P_J=\frac{1}{2}(\vect{|j|}\cdot\vect{|E|})cos(\phi_j-\phi_E)$. Cette formule
peut \'egalement s'\'ecrire de mani\`ere \'equivalente sous forme complexe~:
$P_J=\frac{1}{2}\vect{j}\cdot\vect{E}^*$, o\`u $\vect{E}^*$ est le complexe
conjugu\'e de $\vect{E}$.

  \begin{itemize}
  \item En courant alternatif monophas\'e ($\phi_j=\phi_E$), en particulier,
la formule donnant la puissance se simplifie sous la forme
$P_J=\frac{1}{2}(\vect{|j|}\cdot\vect{|E|})$, ou encore~:
$P_J=\frac{1}{\sqrt{2}}\vect{|j|}\cdot\frac{1}{\sqrt{2}}\vect{|E|}$.
Il s'agit donc du produit des valeurs efficaces. Or, les variables r\'esolues
par \CS en courant alternatif monophas\'e sont pr\'ecis\'ement les
valeurs efficaces (valeurs que l'on
d\'enomme abusivement "valeurs r\'eelles" dans le code source).

  \item En courant alternatif non monophas\'e (triphas\'e, en particulier),
la formule donnant la puissance est utilis\'ee directement sous la forme
$P_J=\frac{1}{2}\vect{j}\cdot\vect{E}^*$.
On utilise pour la calculer les variables r\'esolues qui sont
la partie r\'eelle et la partie imaginaire de $\vect{j}$ et $\vect{E}$.

  \end{itemize}

\item {\bf En conclusion},

  \begin{itemize}
  \item en continu, les variables r\'esolues
$\vect{j}_{Res}$ et $\vect{E}_{Res}$
sont les variables r\'eelles continues
et la puissance se calcule par la formule suivante~:
$P_J=\vect{j}_{Res}\cdot\vect{E}_{Res}$
  \item en alternatif monophas\'e, les variables r\'esolues
$\vect{j}_{Res}$ et $\vect{E}_{Res}$
sont les valeurs efficaces
et la puissance se calcule par la formule suivante~:
$P_J=\vect{j}_{Res}\cdot\vect{E}_{Res}$
  \item en alternatif non monophas\'e, les variables r\'esolues
$\vect{j}_{Res,R}$, $\vect{j}_{Res,I}$ et $\vect{E}_{Res,R}$, $\vect{E}_{Res,I}$
sont la partie r\'eelle et la partie imaginaire de $\vect{j}$ et $\vect{E}$,
et la puissance se calcule par la formule suivante~:
$P_J=\frac{1}{2}(\vect{j}_{Res,R}\cdot\vect{E}_{Res,R}-\vect{j}_{Res,I}\vect{E}_{Res,I})$
  \end{itemize}

\end{itemize}


{\bf Le potentiel imaginaire n'est donc utilis\'e dans le code que lorsque
le courant est alternatif et non monophas\'e.}
En particulier, le potentiel imaginaire n'est pas utilis\'e lorsque le courant est
continu ou alternatif monophas\'e.
En effet, la partie imaginaire n'est introduite en compl\'ement de la partie
r\'eelle que dans le cas o\`u il est n\'ecessaire de disposer de deux grandeurs
pour d\'efinir le potentiel, c'est-\`a-dire lorsqu'il importe de conna\^itre son
amplitude et sa phase.
En courant continu, on n'a naturellement besoin que d'une seule information. En
alternatif monophas\'e, la valeur de la phase importe peu
(on ne travaille pas sur des grandeurs \'electriques instantan\'ees)~:
il suffit de conna\^itre l'amplitude du potentiel et il est donc inutile
d'introduire une variable imaginaire.


{\bf La variable d\'enomm\'ee ``potentiel r\'eel'', $P_R$, repr\'esente une
valeur efficace
si le courant est monophas\'e et une partie r\'eelle sinon.}
De mani\`ere plus explicite, pour un potentiel physique alternatif sinuso\"idal
$Pp$, de valeur maximale not\'ee $Pp_\text{max}$, de phase not\'ee $\phi$, la
variable $P_R$ repr\'esente $\frac{1}{\sqrt{2}}\,Pp_\text{max}$ en
monophas\'e et $Pp_\text{max}\,cos\phi$ sinon. En courant continu, $P_R$
repr\'esente naturellement le potentiel (r\'eel, continu).
{\bf Il est donc indispensable de pr\^eter une attention particuli\`ere aux
valeurs de potentiel impos\'ees aux limites} (facteur $\frac{1}{\sqrt{2}}$ ou
$cos\phi$).

\subsubsection{\'Equations continues}

{\bf Syst\`eme d'\'equations}
\nopagebreak

Les \'equations continues qui sont r\'esolues sont les suivantes~:
\begin{equation}
\left\{\begin{array}{l}
{\color{blue}\dive(\rho \vect{u}) = 0}\\
{\color{blue}\displaystyle\frac{\partial}{\partial t}(\rho \vect{u})
+\dive(\rho\, \vect{u} \otimes \vect{u})
=\dive(\tens{\sigma}) + \vect{TS}}  \\
{\color{blue}\displaystyle\frac{\partial}{\partial t}(\rho h)
+\dive(\rho\, \vect{u} h)
=\Phi_v -
\dive{\left(\left(\frac{\lambda}{C_p}+\frac{\mu_t}{\sigma_t}\right)\grad{h}\right)}
+ {\color{red}P_J}}\\
{\color{red}\dive(\sigma\,\grad{P_R})=0}\\
{\color{red}\dive(\sigma\,\grad{P_I})=0}\text{\ \ \ en alternatif non monophas\'e uniquement}\\
\end{array}\right.
\end{equation}

avec, en continu ou alternatif monophas\'e~:
\begin{equation}
\left\{\begin{array}{l}
{\color{red}P_J=\vect{j}\cdot\vect{E}} \\
{\color{red}\vect{E}=-\grad{P_R}}\\
{\color{red}\vect{j}=\sigma\vect{E}}\\
\end{array}\right.
\end{equation}

et, en alternatif non monophas\'e (avec $i^2=-1$)~:
\begin{equation}
\left\{\begin{array}{l}
{\color{red}\displaystyle P_J=\frac{1}{2}\,\vect{j}\cdot\vect{E}^*}\\
{\color{red}\vect{E}=-\grad{(P_R+i\,P_I)}}\\
{\color{red}\vect{j}=\sigma\vect{E}}\\
\end{array}\right.
\end{equation}

\vspace*{0,5cm}
{\bf \'Equation de la masse}
\nopagebreak

C'est l'\'equation r\'esolue en standard par \CS (contrainte
stationnaire d'incompressibilit\'e). Elle n'a pas de traitement particulier dans
le cadre du module pr\'esent. Un terme source de  masse peut \^etre pris en
compte au second membre si l'utilisateur le souhaite. Pour simplifier
l'expos\'e,
le terme source sera suppos\'e nul ici, dans la mesure o\`u il n'est pas
sp\'ecifique au module \'electrique.

\vspace*{0,5cm}
{\bf \'Equation de la quantit\'e de mouvement}
\nopagebreak

C'est l'\'equation r\'esolue en standard par \CS (les forces de Laplace
($\vect{j} \times \vect{B}$) sont suppos\'ees n\'egligeables).

\vspace*{0,5cm}
{\bf \'Equation de l'enthalpie}
\nopagebreak

On l'\'etablit comme dans le cas des arcs \'electriques\footnote{\`A ceci pr\`es
que la puissance des
forces de Laplace n'appara\^it pas du tout, au lieu de dispara\^itre lorsque
l'on soustrait l'\'equation de l'\'energie cin\'etique \`a celle de l'\'energie
totale.} \`a partir de l'�quation de
l'\'energie apr\`es plusieurs approximations utilis\'ees en standard dans \CS et en
prenant en compte le terme d'effet Joule li\'e \`a l'\'energie
\'electromagn\'etique.

Par rapport \`a l'\'equation utilis\'ee pour les \'etudes d'arc \'electrique,
seule l'expression de l'effet Joule diff\`ere lorsque le courant est
alternatif non monophas\'e.

\vspace*{0,5cm}
{\bf \'Equations \'electromagn\'etiques}
\nopagebreak

Elles sont obtenues comme indiqu\'e dans la partie relative aux arcs
\'electriques, mais on ne conserve que les relations associ\'ees \`a la
densit\'e de courant, au champ \'electrique et au potentiel dont il d\'erive.

%%%%%%%%%%%%%%%%%%%%%%%%%%%%%%%%%%
%%%%%%%%%%%%%%%%%%%%%%%%%%%%%%%%%%
\section{Discr\'etisation}
%%%%%%%%%%%%%%%%%%%%%%%%%%%%%%%%%%
%%%%%%%%%%%%%%%%%%%%%%%%%%%%%%%%%%

La discr\'etisation des \'equations ne pose pas de probl\`eme particulier
(ajout de termes sources explicites pour l'effet Joule et les forces de Laplace,
\'equations de Poisson pour la d\'etermination des potentiels).

Un point sur les conditions aux limites doit cependant \^etre fait ici, en
particulier pour pr\'eciser la m\'ethode de recalage automatique des
potentiels.



\subsection{Arcs \'electriques}

\subsubsection{Conditions aux limites}

Seules les conditions aux limites pour les potentiels sont \`a pr\'eciser.

{\bf Les conditions aux limites sur le potentiel scalaire} sont des conditions de
Neumann homog\`enes sur toutes les fronti\`eres hormis \`a la cathode et \`a
l'anode. \`A la cathode, on impose une condition de Dirichlet homog\`ene (potentiel nul par convention). \`A l'anode, on impose une
condition de Dirichlet permettant de fixer la diff\'erence de potentiel
souhait\'ee entre l'anode et la cathode.
L'utilisateur peut fixer le potentiel de l'anode directement ou
demander qu'un recalage automatique du potentiel soit effectu\'e pour atteindre
une intensit\'e de courant pr\'ed\'etermin\'ee.

Lorsque le recalage automatique est demand\'e (\var{IELCOR}=1), l'utilisateur doit fixer la
valeur cible de l'intensit\'e, \var{COUIMP}, ($A$) et une valeur \'elev\'ee
de d\'epart
de la diff\'erence de potentiel entre l'anode et la cathode\footnote{Plus pr\'ecis\'ement, l'utilisateur doit imposer un potentiel nul
en cathode et le potentiel \var{DPOT} \`a l'anode, en utilisant explicitement,
dans le sous-programme utilisateur \fort{uselcl}, la variable \var{DPOT} qui
sera automatiquement recal\'ee au cours du calcul.}, \var{DPOT}, ($V$).
Le recalage est effectu\'e en fin de pas temps et permet de disposer, pour le pas
de temps suivant, de valeurs recal\'ees des forces de Laplace et de l'effet
Joule.
\begin{itemize}
\item Pour effectuer le recalage, \CS d\'etermine l'int\'egrale de l'effet Joule
estim\'e sur le domaine (en $W$) et en compare la
valeur au produit de l'intensit\'e \var{COUIMP} par la diff\'erence de
potentiel\footnote{\var{DPOT} est la diff\'erence de
potentiel impos\'ee entre l'anode et la cathode au
pas de temps qui s'ach\`eve. \var{DPOT} a conditionn\'e le champ \'electrique et
la densit\'e de courant utilis\'es pour le calcul de l'effet Joule.} \var{DPOT}.
Un coefficient multiplicatif de recalage \var{COEPOT} en
est d\'eduit (pour \'eviter des variations trop brusques,
on s'assure qu'il reste born\'e). % entre 0,75 et 1,5).
\item On multiplie alors par \var{COEPOT} la
diff\'erence de potentiel entre l'anode et la cathode, \var{DPOT}, et le vecteur $\vect{j}$. L'effet
Joule, produit de $\vect{j}$ par $\vect{E}$, est multipli\'e par
le carr\'e de \var{COEPOT}. Pour assurer la coh\'erence du post-traitement des
variables, le potentiel vecteur et le potentiel scalaire sont \'egalement
multipli\'es par \var{COEPOT}.
\item Le champ \'electrique n'\'etant pas explicitement
stock\'e, on ne le recale pas. Le potentiel vecteur et les forces de Laplace seront d\'eduits de la
densit\'e de courant et int\'egreront donc naturellement le recalage.
\end{itemize}

\bigskip
{\bf Les conditions aux limites sur le potentiel vecteur} sont des conditions de
Neumann homog\`ene sur toutes les fronti\`eres hormis sur une zone de bord
arbitrairement choisie (paroi par exemple) pour laquelle une condition de
Dirichlet est utilis\'ee afin que le syst\`eme soit inversible
(la valeur impos\'ee est la valeur du potentiel vecteur
calcul\'ee au pas de temps pr\'ec\'edent).



\subsection{Effet Joule}

\subsubsection{Conditions aux limites}

Seules les conditions aux limites pour les potentiels sont \`a pr\'eciser.

{\bf Les conditions aux limites sur le potentiel scalaire} sont \`a pr\'eciser
au cas par cas selon la configuration des \'electrodes. Ainsi, on dispose
classiquement de conditions de Neumann homog\`enes ou de Dirichlet (potentiel
impos\'e). On peut \'egalement avoir besoin d'imposer des conditions
d'antisym\'etrie (en utilisant des conditions de Dirichlet homog\`enes par exemple).
L'utilisateur peut \'egalement souhaiter qu'un recalage automatique du potentiel
soit effectu\'e pour atteindre une valeur pr\'ed\'etermin\'ee de la puissance
dissip\'ee par effet Joule.

Lorsque le recalage automatique est demand\'e (\var{IELCOR}=1), l'utilisateur doit fixer la
valeur cible de la puissance dissip\'ee dans le domaine, \var{PUISIM}, ($V.A$).
Il doit en outre, sur les fronti\`eres o\`u il
souhaite que le potentiel (r\'eel ou complexe) s'adapte automatiquement, fournir en
condition \`a la limite une valeur initiale du potentiel et la multiplier par
la variable \var{COEJOU} qui sera automatiquement recal\'ee au cours du calcul
(\var{COEJOU} vaut 1 au premier pas de temps).
Le recalage est effectu\'e en fin de pas temps et permet de disposer, pour le pas
de temps suivant, d'une valeur recal\'ee de l'effet
Joule.
\begin{itemize}
\item Pour effectuer le recalage, \CS d\'etermine l'int\'egrale de l'effet Joule
estim\'e sur le domaine (en $W$) et en compare la
valeur \`a la puissance cible. Un coefficient multiplicatif de recalage \var{COEPOT} en
est d\'eduit (pour \'eviter des variations trop brusques,
on s'assure qu'il reste born\'e entre 0,75 et 1,5).
\item On multiplie alors par
\var{COEPOT} le facteur multiplicatif \var{COEJOU} utilis\'e pour les conditions aux
limites. La puissance dissip\'ee par effet
Joule est multipli\'ee par
le carr\'e de \var{COEPOT}. Pour assurer la coh\'erence du post-traitement des
variables, le potentiel est \'egalement
multipli\'e par \var{COEPOT}.
\item Le champ \'electrique n'\'etant pas explicitement
stock\'e, on ne le recale pas.
\end{itemize}

\bigskip
On notera que la variable \var{DPOT} est \'egalement recal\'ee et qu'elle
peut donc \^etre utilis\'ee si besoin pour imposer les conditions aux limites.

%%%%%%%%%%%%%%%%%%%%%%%%%%%%%%%%%%
%%%%%%%%%%%%%%%%%%%%%%%%%%%%%%%%%%
\section{Mise en \oe uvre}
%%%%%%%%%%%%%%%%%%%%%%%%%%%%%%%%%%
%%%%%%%%%%%%%%%%%%%%%%%%%%%%%%%%%%


\subsection{Introduction}

Le module \'electrique est une ``physique particuli\`ere'' activ\'ee lorsque les
mots-cl\'es \var{IPPMOD(IELARC)} (arc \'electrique) ou \var{IPPMOD(IELJOU)}
(Joule) sont strictement positifs. Les d\'eveloppements concernant la conduction ionique
(mot-cl\'e \var{IPPMOD(IELION)}) ont \'et\'e pr\'evus dans le code mais restent \`a
r\'ealiser. Pour l'arc \'electrique, dans la version actuelle
de \CS, seule est op\'erationnelle l'option \var{IPPMOD(IELARC)}=2~: la version 2D axisym\'etrique qui permettrait de
        s'affranchir du potentiel vecteur (option \var{IPPMOD(IELARC)}=1) n'est pas
        activable.
Pour l'effet Joule, lorsqu'il n'est pas utile d'introduire un potentiel scalaire
complexe
(en courant continu ou alternatif monophas\'e), on utilise
\var{IPPMOD(IELJOU)}=1. Lorsqu'un potentiel scalaire complexe est indispensable (courant
alternatif triphas\'e, par exemple), on utilise
\var{IPPMOD(IELJOU)}=2.

Dans ce qui suit, on pr\'ecise les inconnues et les propri\'et\'es
principales utilis\'ees dans le module.
On fournit \'egalement un arbre d'appel simplifi\'e des sous-programmes du module
(initialisation avec \fort{initi1} puis \fort{inivar} et boucle en temps avec \fort{tridim}).
Les sous-programmes marqu\'es d'un ast\'erisque sont d\'etaill\'es ensuite.

\newpage

\subsection{Inconnues et propri\'et\'es}

Les d\'eveloppements ont \'et\'e r\'ealis\'es pour une unique phase (\var{NPHAS}=1).

Les \var{NSCAPP} inconnues scalaires associ\'ees \`a la physique
particuli\`ere sont d\'efinies dans \fort{elvarp} dans l'ordre
suivant (en particulier afin de limiter le stockage en m\'emoire lors de
la r\'esolution s\'equentielle des
scalaires par \fort{scalai})~:
\begin{itemize}
\item l'enthalpie \var{RTP(*,ISCA(IHM))},
\item un potentiel scalaire r\'eel \var{RTP(*,ISCA(IPOTR))},
\item un potentiel scalaire imaginaire \var{RTP(*,ISCA(IPOTI))} {\it ssi}
          \var{IPPMOD(IELJOU)}=2 (\'etudes Joule en courant alternatif non monophas\'e),
\item les trois composantes d'un potentiel vecteur r\'eel
          \var{RTP(*,ISCA(IPOTVA(i)))} (avec \var{i} variant de 1 \`a 3) {\it ssi}
        \var{IPPMOD(IELARC)}=2 (arc \'electrique),
\item \var{NGAZG}-1 fractions massiques \var{RTP(*,ISCA(IYCOEL(j)))}
        (avec \var{j} variant de 1 \`a \var{NGAZG}-1) pour un fluide \`a \var{NGAZG}
        constituants (avec \var{NGAZG} strictement sup\'erieur \`a 1).
        En arc \'electrique, la composition est fournie dans le fichier de donn\'ees
        \fort{dp\_ELE}. La fraction massique du
        dernier constituant n'est pas stock\'ee en m\'emoire. Elle est
        d\'etermin\'ee chaque fois que n\'ecessaire en calculant le compl\'ement \`a l'unit\'e
        des autres fractions massiques (et, en particulier, lorsque \fort{elthht} est
        utilis\'e pour le calcul des propri\'et\'es physiques).
\end{itemize}

\bigskip
Outre les propri\'et\'es associ\'ees en standard aux variables scalaires
identifi\'ees ci-dessus, le
tableau \var{PROPCE} contient \'egalement~:
 \begin{itemize}
\item la temp\'erature, \var{PROPCE(*,IPPROC(ITEMP))}. En th\'eorie, on
        pourrait \'eviter de stocker cette variable, mais l'utilisateur est
        presque toujours int\'eress\'e par sa valeur en post-traitement et
        les propri\'et\'es physiques sont souvent donn\'ees par des lois qui en
        d\'ependent explicitement.
        Son unit\'e (Kelvin ou Celsius) d\'epend des tables
        enthalpie-temp\'erature fournies par l'utilisateur.
\item la puissance \'electromagn\'etique dissip\'ee par effet Joule,
        \var{PROPCE(*,IPPROC(IEFJOU))} (terme source positif pour l'enthalpie),
\item les trois composantes des forces de Laplace,
        \var{PROPCE(*,IPPROC(ILAPLA(i)))} (avec \var{i} variant de 1 \`a 3)
        en arc \'electrique (\var{IPPMOD(IELARC)}=2).
\end{itemize}

\bigskip
La conductivit\'e \'electrique est {\it a priori} variable et
conserv\'ee dans le tableau de propri\'et\'es aux cellules
\var{PROPCE(*,IPPROC(IVISLS(IPOTR)))}. Elle intervient dans l'\'equation de
Poisson portant sur le potentiel scalaire. Lorsque le potentiel scalaire a une
partie imaginaire, la conductivit\'e n'est pas dupliqu\'ee~:
les entiers \var{IPPROC(IVISLS(IPOTI))} et \var{IPPROC(IVISLS(IPOTR))} pointent sur la
m\^eme case du tableau \var{PROPCE}. La conductivit\'e associ\'ee au potentiel
vecteur est uniforme et de valeur unit\'e (\var{VISLS0(IPOTVA(i))}=1.D0
avec \var{i} variant de 1 \`a 3).

Le champ \'electrique, la densit\'e de courant et le champ magn\'etique ne sont
stock\'es que de mani\`ere temporaire (voir \fort{elflux}).


\newpage

\subsection{Arbre d'appel simplifi\'e}

\begin{table}[htp]
\begin{center}
\begin{tabular}{llllp{10cm}}
\fort{usini1}         &                 &                &
        & Initialisation des mots-cl\'es utilisateur g\'en\'eraux et positionnement des variables\\
                &\fort{usppmo}         &                &
        & D\'efinition du module ``physique particuli\`ere'' employ\'e\\
                &\fort{varpos}         &                &
        & Positionnement des variables \\
                &                 & \fort{pplecd} &
        & Branchement des physiques particuli\`eres pour la lecture de fichier de donn\'ees \\
                &                 &                 & \fort{ellecd}*
        & Lecture du fichier de donn\'ees pour les arcs \'electriques  \fort{dp\_ELE} \\
                &                 & \fort{ppvarp} &
        & Branchement des physiques particuli\`eres pour le positionnement des inconnues \\
                &                 &                 & \fort{elvarp}*
        & Positionnement des inconnues (enthalpie, potentiels, fractions massiques) \\
                &                 & \fort{ppprop} &
        & Branchement des physiques particuli\`eres pour le positionnement des propri\'et\'es\\
                &                 &                 & \fort{elprop}*
        & Positionnement des propri\'et\'es (temp\'erature, effet Joule, forces de Laplace) \\
%
\fort{ppini1}         &                &                &
        & Branchement des physiques particuli\`eres pour l'initialisation des
mots-cl\'es sp\'ecifiques \\
                &\fort{elini1}         &                &
        & Initialisation des mots-cl\'es pour le module \'electrique\\
                &\fort{useli1}         &                &
        & Initialisation des mots-cl\'es utilisateur pour le module \'electrique\\
                &\fort{elveri}         &                &
        & V\'erification des mots-cl\'es pour le module \'electrique\\
\end{tabular}
\caption{Sous-programme \fort{initi1}~: initialisation des mots-cl\'es et
positionnement des variables}
\end{center}
\end{table}


\begin{table}[htp]
\begin{center}
\begin{tabular}{llllp{10cm}}
\fort{ppiniv}         &                &                &
        & Branchement des physiques particuli\`eres pour l'initialisation des variables \\
                & \fort{eliniv}*&                &
        & Initialisation des variables sp\'ecifiques au module \'electrique \\
                 &                 & \fort{elthht}*&
        & Transformation temp\'erature-enthalpie et enthalpie-temp\'erature par
                interpolation sur la base du fichier de donn\'ees \fort{dp\_ELE}
                (arc \'electrique uniquement) \\
                 &                 & \fort{useliv} &
        & Initialisation des variables par l'utilisateur  \\
                 &                 &                 & \fort{elthht}*
        & Transformation temp\'erature-enthalpie et enthalpie-temp\'erature par
                interpolation sur la base du fichier de donn\'ees \fort{dp\_ELE}
                (arc \'electrique uniquement) \\
\end{tabular}
\caption{Sous-programme \fort{inivar}~: initialisation des variables}
\end{center}
\end{table}


\begin{table}[htp]
\begin{center}
\begin{tabular}{llllp{10cm}}
\fort{phyvar}         &                &                &
        & Calcul des propri\'et\'es physiques variables \\
                & \fort{ppphyv} &                &
        & Branchement des physiques particuli\`eres pour le calcul des
                propri\'et\'es physiques variables \\
                & \fort{elphyv} &                &
        & Calcul des propri\'et\'es physiques variables pour le module
                \'electrique. En arc \'electrique, les propri\'et\'es sont
                calcul\'ees par interpolation \`a partir des tables fournies
                dans le fichier de donn\'ees \fort{dp\_ELE}\\
                 &                 & \fort{elthht}*&
        & Transformation temp\'erature-enthalpie et enthalpie-temp\'erature par
                interpolation sur la base du fichier de donn\'ees \fort{dp\_ELE}
                (arc \'electrique uniquement) \\
                 &                 & \fort{uselph} &
        & Calcul par l'utilisateur des propri\'et\'es physiques variables pour le module
                \'electrique. Pour les \'etudes Joule, en particulier, les propri\'et\'es
                doivent \^etre fournies ici sous forme de loi (des exemples sont
                disponibles)\\
                 &                 &                 & \fort{usthht}
        & Transformation temp\'erature-enthalpie et enthalpie-temp\'erature
                fournie par l'utilisateur (plus sp\'ecifiquement pour les
                \'etudes Joule, pour lesquelles on ne dispose pas d'un fichier
                de donn\'ees \`a partir duquel r\'ealiser des interpolations avec \fort{elthht}) \\
\end{tabular}
\caption{Sous-programme \fort{tridim}~: partie 1 (propri\'et\'es physiques)}
\end{center}
\end{table}

\begin{table}[htp]
\begin{center}
\begin{tabular}{llllp{10cm}}
\fort{ppclim}         &                  &                &
        & Branchement des physiques particuli\`eres pour les conditions aux limites\\
                & \fort{uselcl} &                &
        & Intervention de l'utilisateur pour les conditions aux limites (en lieu
                et place de \fort{usclim}, m\^eme pour les variables qui ne sont
                pas sp\'ecifiques au module \'electrique). Si un recalage
                automatique des potentiels est demand\'e (\var{IELCOR=1}), il
                doit \^etre pris en compte ici par le biais des variables
                \var{DPOT} ou \var{COEJOU} (voir la description des
                conditions aux limites).   \\
\fort{navsto}         &                  &                &
        & R\'esolution des \'equations de Navier-Stokes\\
                & \fort{preduv} &                &
        & Pr\'ediction de la vitesse~: prise en compte des forces de Laplace
                calcul\'ees dans \fort{elflux} au pas de temps pr\'ec\'edent\\
\fort{``turb''} &                  &                &
        & Turbulence : r\'esolution des \'equations pour les mod\`eles
                n\'ecessitant des \'equations de convection-diffusion\\
\fort{scalai}*         &                  &                &
        & R\'esolution des \'equations portant sur les scalaires associ\'es aux
                physiques particuli\`eres et des scalaires ``utilisateur''  \\
                & \fort{covofi}         &                &
        & R\'esolution successive de l'enthalpie, du potentiel scalaire
                r\'eel et, si \var{IPPMOD(IELJOU)=2}, de la partie imaginaire du
                potentiel scalaire (appels successifs \`a \fort{covofi} qui appelle
                \fort{eltssc} pour le calcul du terme d'effet Joule au second
                membre de l'\'equation de l'enthalpie)\\
                & \fort{elflux}* &                &
        & Calcul du champ \'electrique, de la densit\'e de courant et de l'effet
                Joule (premier de deux appels au cours du pas de temps courant) \\
                & \fort{uselrc}* &                &
        & Recalage automatique \'eventuel
                de
                la densit\'e de courant, de l'effet Joule, des potentiels et
                des coefficients \var{DPOT} et \var{COEJOU}.
                Ce recalage, s'il a \'et\'e demand\'e
                par l'utilisateur (\var{IELCOR}=1), est effectu\'e \`a partir
                du deuxi\`eme pas de temps. \\
                & \fort{covofi}         &                &
        & R\'esolution successive, si \var{IPPMOD(IELARC)=2}, des trois
                composantes du potentiel vecteur. On proc\`ede par
                appels successifs \`a \fort{covofi} qui appelle
                \fort{eltssc} pour le calcul du second membre de l'\'equation de
                Poisson portant sur chaque composante du potentiel. \\
                & \fort{covofi}         &                &
        & R\'esolution successive des \var{NGAZG-1} fractions massiques
                caract\'erisant la composition du fluide, s'il est
                multiconstituant.
                On proc\`ede par appels successifs \`a \fort{covofi}. \\
                & \fort{elflux}* &                &
        & En arc \'electrique, calcul du champ magn\'etique et
                des trois composantes des forces de
                Laplace (deuxi\`eme et dernier appel au cours du pas de temps courant)\\
                & \fort{covofi}         &                &
        & R\'esolution des scalaires ``utilisateur''\\
\end{tabular}
\caption{Sous-programme \fort{tridim}~: partie 2 (conditions aux limites,
Navier-Stokes, turbulence et scalaires)}
\end{center}
\end{table}

\newpage

\begin{table}[htp]
\begin{center}
\begin{tabular}{llllp{10cm}}
\fort{postlc}         &                  &                &
        & Post-traitement\\
                & \fort{ecrevo}         &                &
        & \'Ecriture des variables \`a post-traiter\\
                &                  & \fort{uselen}&
        & Ajout au post-traitement de variables calcul\'ees par
                l'utilisateur. En exemple activ\'e standard sont post-trait\'es,
                s'ils existent, l'oppos\'e du champ \'electrique ({\it i.e.} le
                gradient du potentiel scalaire, r\'eel ou complexe),  le vecteur
                densit\'e de courant imaginaire (en effet Joule), le champ
                magn\'etique (en arc \'electrique) et enfin le module et l'argument
                du potentiel (en effet Joule, avec \var{IPPMOD(IELJOU)=4})\\
\end{tabular}
\caption{Sous-programme \fort{tridim}~: partie 3 (post-traitement)}
\end{center}
\end{table}

\newpage

\subsection{Pr\'ecisions}

\etape{\fort{ellecd}}

Ce sous-programme r\'ealise la lecture du fichier de donn\'ees sp\'ecifique
aux arcs \'electriques. On donne ci-dessous, \`a titre d'exemple, l'ent\^ete
explicative et deux lignes de donn\'ees d'un fichier type. Ces valeurs sont interpol\'ees chaque
fois que n\'ecessaire par \fort{elthht} pour d\'eterminer les propri\'et\'es
physiques du fluide \`a une temp\'erature (une enthalpie) donn\'ee.

{\scriptsize
\begin{verbatim}
# Nb d'especes NGAZG et Nb de points NPO (le fichier contient NGAZG blocs de NPO lignes chacun)
# NGAZG NPO
   1   238
#
#  Proprietes
#      T           H          ROEL       CPEL           SIGEL        VISEL        XLABEL        XKABEL
#  Temperature  Enthalpie   Masse vol. Chaleur       Conductivite  Viscosite   Conductivite   Coefficient
#                           volumique  massique       electrique   dynamique     thermique   d'absorption
#       K         J/kg         kg/m3     J/(kg K)       Ohm/m        kg/(m s)     W/(m K)         -
#
   300.00       14000.       1.6225       520.33      0.13214E-03  0.34224E-04  0.26712E-01   0.0000
   400.00       65800.       1.2169       520.33      0.13214E-03  0.34224E-04  0.26712E-01   0.0000
\end{verbatim}
}


\etape{\fort{elvarp}}

Ce sous-programme permet de positionner les inconnues de calcul list\'ees
pr\'ec\'edemment. On y pr\'ecise \'egalement que la chaleur massique \`a
pression constante est variable, ainsi que la conductivit\'e de tous les
scalaires associ\'es au module \'electrique, hormis la conductivit\'e de
l'\'eventuel potentiel vecteur (celle-ci est uniforme et de valeur unit\'e).


\etape{\fort{elprop}}

C'est dans ce sous-programme que sont positionn\'ees les propri\'et\'es stock\'ees
dans le tableau \var{PROPCE}, et en particulier la temp\'erature, l'effet Joule
et les forces de Laplace.

\etape{\fort{eliniv}}

Ce sous-programme permet de r\'ealiser les initialisations par d\'efaut
sp\'ecifiques au module.

En particulier, en $k-\varepsilon$, les deux variables
turbulentes sont initialis\'ees \`a $10^{-10}$ (choix historique arbitraire,
mais r\'eput\'e, lors de tests non r\'ef\'erenc\'es, permettre le d\'emarrage de
certains calculs qui \'echouaient avec une initialisation classique).

Les potentiels sont initialis\'es \`a z\'ero, de m\^eme que l'effet Joule. En
arc \'electrique, les forces de Laplace sont initialis\'ees \`a z\'ero.

Le fluide est suppos\'e monoconstituant (seule est pr\'esente la premi\`ere
esp\`ece).

En arc \'electrique, l'enthalpie est initialis\'ee \`a la valeur de l'enthalpie du m\'elange
suppos\'e monoconstituant \`a la temp\'erature \var{T0(IPHAS)} donn\'ee
dans \fort{usini1}.  En effet Joule, l'enthalpie est initialis\'ee \`a z\'ero
(mais l'utilisateur peut fournir une valeur diff\'erente dans \fort{useliv}).

\etape{\fort{elthht}}

Ce sous-programme permet de r\'ealiser (en arc \'electrique) les interpolations
n\'ecessaires \`a la d\'e\-ter\-mi\-na\-tion des propri\'et\'es physiques du fluide, \`a
partir des tables fournies dans le fichier de donn\'ees \fort{dp\_ELE}.

On notera en particulier que ce sous-programme prend en argument le tableau
\var{YESP(NESP)} qui repr\'esente la fraction massique des \var{NGAZG}
constituants du fluide. Dans le code, on ne r\'esout que la fraction massique
des \var{NGAZG}-1 premiers constituants. Avant chaque appel \`a \fort{elthht},
la fraction massique du dernier constituant doit \^etre calcul\'ee comme le
compl\'ement \`a l'unit\'e des autres fractions massiques.

\etape{\fort{scalai}, \fort{elflux}, \fort{uselrc}}

Le sous-programme \fort{scalai} permet de calculer, dans l'ordre souhait\'e,
les \var{NSCAPP} scalaires ``physique particuli\`ere'' associ\'es au module
\'electrique, puis de calculer les grandeurs interm\'ediaires n\'e\-ces\-sai\-res et
enfin de
r\'ealiser les op\'erations qui permettent d'assurer le recalage automatique
des potentiels, lorsqu'il est requis par l'utilisateur ({\it i.e.} si \var{IELCOR=1}).

Les \var{NSCAPP} scalaires ``physique particuli\`ere'' sont calcul\'es successivement par un
appel \`a \fort{covofi} plac\'e dans une boucle portant sur les \var{NSCAPP}
scalaires. L'algorithme tire profit de l'ordre sp\'ecifique dans lequel ils sont d\'efinis et donc
r\'esolus (dans l'ordre~: enthalpie, potentiel scalaire, potentiel vecteur, fractions massiques).

Pour \'eviter des variations trop brutales en d\'ebut de calcul, le terme source
d'effet Joule n'est pris en compte dans l'\'equation de l'enthalpie qu'\`a
partir du troisi\`eme pas de temps.

Apr\`es la r\'esolution de l'enthalpie et du potentiel scalaire
(r\'eel ou complexe), le sous-programme \fort{elflux} permet de calculer
les trois composantes du champ \'electrique
(que l'on stocke dans des tableaux de travail), puis la densit\'e de courant
et enfin l'effet Joule, que l'on conserve dans le tableau \var{PROPCE(*,IPPROC(IEFJOU))}
pour le pas temps suivant (apr\`es recalage \'eventuel dans \fort{uselrc} comme
indiqu\'e ci-apr\`es).
Lorsque \var{IPPMOD(IELJOU)=2},  l'apport de la partie imaginaire est pris en
compte pour le calcul de l'effet Joule. Lorsque \var{IPPMOD(IELARC)=2}
(arc \'electrique), le vecteur densit\'e de courant est
conserv\'e dans \var{PROPCE}, en lieu et place des forces de Laplace
\var{PROPCE(*,IPPROC(ILAPLA(i)))}~: il est utilis\'e pour le calcul du potentiel vecteur dans le
second appel \`a \fort{elflux},
apr\`es recalage \'eventuel par \fort{uselrc} (en effet, il n'est plus
n\'ecessaire de conserver les forces de Laplace \`a ce stade puisque
la seule \'equation dans laquelle elles interviennent est l'\'equation de la
quantit\'e de mouvement et qu'elle a d\'ej\`a \'et\'e r\'esolue).

\`A la suite de \fort{elflux},
le sous-programme \fort{uselrc} effectue le recalage permettant d'adapter automatiquement
les conditions aux limites portant sur les potentiels, si l'utilisateur l'a
demand\'e ({\it i.e.} si \var{IELCOR=1}).
On se reportera au paragraphe relatif aux conditions aux limites. On pr\'ecise
ici que le coefficient de recalage \var{COEPOT} permet d'adapter l'effet Joule
\var{PROPCE(*,IPPROC(IEFJOU))} et la diff\'erence de potentiel \var{DPOT}
(utile pour les conditions aux limites portant sur le potentiel scalaire au pas
de temps suivant\footnote{{\it A priori}, \var{DPOT} n'est pas n\'ecessaire pour les
cas Joule.}).
Pour les cas d'arc \'electrique, \var{COEPOT} permet \'egalement de
recaler le vecteur densit\'e de courant que l'on vient
de stocker temporairement dans \var{PROPCE(*,IPPROC(ILAPLA(i)))} et qui va
servir imm\'ediatement \`a calculer le potentiel vecteur.
Pour les cas Joule,  on recale en outre le coefficient \var{COEJOU} (utile
pour les conditions aux limites portant sur le potentiel scalaire au pas de
temps suivant).

Pour les cas d'arc \'electrique (\var{IPPMOD(IELARC)=2}),
apr\`es \fort{elflux} et \fort{uselrc},
la r\'esolution s\'equentielle des inconnues scalaires se poursuit dans
\fort{scalai} avec le calcul des trois composantes du potentiel vecteur. Le second membre de
l'\'equation de Poisson consid\'er\'ee d\'epend de la densit\'e de courant qui,
dans \fort{elflux}, a \'et\'e temporairement stock\'ee dans le tableau
\var{PROPCE(*,IPPROC(ILAPLA(i)))} et qui,  dans \fort{uselrc}, vient d'\^etre
recal\'ee si \var{IELCOR=1}.
Les valeurs du potentiel vecteur obtenues int\`egrent donc  naturellement l'\'eventuel
recalage.

Pour les cas d'arc \'electrique (\var{IPPMOD(IELARC)=2}), un second appel \`a
\fort{elflux} permet alors de calculer le champ magn\'etique
que l'on stocke dans des tableaux de travail et les forces de Laplace que l'on stocke dans
\var{PROPCE(*,IPPROC(ILAPLA(i)))} pour le pas de temps suivant (la densit\'e de
courant, que l'on avait temporairement conserv\'ee dans ce tableau, ne servait
qu'\`a calculer le second membre de l'\'equation de Poisson portant sur le
potentiel vecteur~: il n'est donc plus n\'ecessaire de la conserver).

La r\'esolution s\'equentielle des inconnues scalaires sp\'ecifiques au module se poursuit dans
\fort{scalai}, avec le calcul des \var{NGAZG}-1 fractions massiques permettant
de d\'efinir la composition du fluide.

Pour terminer, \fort{scalai} permet la r\'esolution des scalaires
``utilisateurs'' (appel \`a \fort{covofi} dans une boucle portant sur les
\var{NSCAUS} scalaires utilisateur).

On peut remarquer pour finir que les termes sources des \'equations de la quantit\'e de
mouvement (forces de Laplace) et de l'enthalpie (effet Joule) sont disponibles
\`a la fin du pas de temps $n$ pour une utilisation au pas de temps $n+1$ (de ce
fait, pour permettre les reprises de calcul, ces termes sources sont stock\'es dans le fichier suite auxiliaire, ainsi que \var{DPOT}
et \var{COEJOU}).

\newpage
%%%%%%%%%%%%%%%%%%%%%%%%%%%%%%%%%%
%%%%%%%%%%%%%%%%%%%%%%%%%%%%%%%%%%
\section{Points \`a traiter}
%%%%%%%%%%%%%%%%%%%%%%%%%%%%%%%%%%
%%%%%%%%%%%%%%%%%%%%%%%%%%%%%%%%%%

\etape{Mobilit\'e ionique} Le module est \`a d\'evelopper.

\etape{Conditions aux limite en Joule} La prise en compte de conditions aux
limites coupl\'ees entre \'electrodes reste  \`a faire.

\etape{Compressible en arc \'electrique} Les  d\'eveloppements du module
compressible de \CS doivent \^etre rendus compatibles avec le module arc \'electrique.
