%-----------------------------------------------------------------------
%
%     This file is part of the Code_Saturne Kernel, element of the
%     Code_Saturne CFD tool.
%
%     Copyright (C) 1998-2008 EDF S.A., France
%
%     contact: saturne-support@edf.fr
%
%     The Code_Saturne Kernel is free software; you can redistribute it
%     and/or modify it under the terms of the GNU General Public License
%     as published by the Free Software Foundation; either version 2 of
%     the License, or (at your option) any later version.
%
%     The Code_Saturne Kernel is distributed in the hope that it will be
%     useful, but WITHOUT ANY WARRANTY; without even the implied warranty
%     of MERCHANTABILITY or FITNESS FOR A PARTICULAR PURPOSE.  See the
%     GNU General Public License for more details.
%
%     You should have received a copy of the GNU General Public License
%     along with the Code_Saturne Kernel; if not, write to the
%     Free Software Foundation, Inc.,
%     51 Franklin St, Fifth Floor,
%     Boston, MA  02110-1301  USA
%
%-----------------------------------------------------------------------
\documentclass[a4paper,10pt,twoside]{article}

%
%%%%%%%%%%%%%%%%%%%%%%%%%%%%%%%%%%%%%%%%%%%%%%%%%%%%%%%%%%%%%%%%%%%%%%
% PACKAGES OBLIGATOIRES
\usepackage{csdoc}
% MACROS SUPPLEMENTAIRES
\usepackage{csmacros}
%
%%%%%%%%%%%%%%%%%%%%%%%%%%%%%%%%%%%%%%%%%%%%%%%%%%%%%%%%%%%%%%%%%%%%%%

%
%%%%%%%%%%%%%%%%%%%%%%%%%%%%%%%%%%%%%%%%%%%%%%%%%%%%%%%%%%%%%%%%%%%%%%
% PACKAGES ET COMMANDES POUR LE DOCUMENTS PDF ET LES HYPERLIENS
\hypersetup{%
  pdftitle = {CodeSaturne autovnv},
  pdfauthor = {MFEE},
  pdfpagemode = UseOutlines
}
\pdfinfo{/CreationDate (D:20110704000000-01 00 )}
%
% To have thumbnails upon opening the document under ACROREAD
% pdfpagemode = UseThumbs
%
%%%%%%%%%%%%%%%%%%%%%%%%%%%%%%%%%%%%%%%%%%%%%%%%%%%%%%%%%%%%%%%%%%%%%%
%
%%%%%%%%%%%%%%%%%%%%%%%%%%%%%%%%%%%%%%%%%%%%%%%%%%%%%%%%%%%%%%%%%%%%%%
% INFO POUR PAGES DE GARDES
\titreCS{\CS version~\verscs: autovnv tool}

\docassociesCS{}
\resumeCS{This document presents the autovnv tool. The aim of this script
is to drive \CS's cases automatically, to compare checkpoint files and
to display results.

\CS version~\verscs.

\begin{center}
\large{WORK IN PROGRESS}
\end{center}
}
%
%%%%%%%%%%%%%%%%%%%%%%%%%%%%%%%%%%%%%%%%%%%%%%%%%%%%%%%%%%%%%%%%%%%%%%
% DEBUT DU DOCUMENT
\begin{document}

\def\contentsname{\textbf{\normalsize TABLE OF CONTENTS}\pdfbookmark[1]{Table of
contents}{contents}}

\pdfbookmark[1]{Flyleaf}{pdg}
\large
\makepdgCS
\normalsize

\passepage

\begin{center}\begin{singlespace}
\tableofcontents
\end{singlespace}\end{center}
%
\section{Introduction}

\textsc{Autovnv} is a small framework to automate the launch of \CS computations
and do some operations on new results.

The script needs a directory of previous \CS cases which are candidates to be duplicated.
This directory is called \textbf{repository}. The duplication is done in an new directory
which is called the \textbf{destination}.

For each duplicated case, \textsc{Autovnv} is able to compile the User Defined Functions,
to run the case, to compare the obtained checkpoint file with the previous one
from the \textbf{repository}, and to plot curves in order to illustrate the computations.

For all these steps, \textsc{Autovnv} generate two reports, a global report which summarize
the status of each case, and a detailed report which gives the differences between the new results
and the previous one in the \textbf{repository}, and display the defined plots.

In the \textbf{repository}, previous results of computations are required only for checkpoint
files comparison purpose. They can be also usfull, if the user needs to run a specific script.

\section{Installation and prerequisites}

\textsc{Autovnv} does not need a specific installation: the related files are installed
with other python scripts of the code. Additional prerequisites required are:
\begin{list}{$\bullet$}{}
\item \texttt{numpy},
\item \texttt{matplotlib}.
\end{list}

\section{Command line options}

The command line options can be found with the command: \texttt{code\_saturne autovnv -h}.

\begin{list}{$\bullet$}{}
\item \texttt{-f FILE, --file=FILE}: gives the file of parameters for \textsc{Autovnv}.
This file is mandatory, and therefore this option must be completed;
\item \texttt{-q, --quiet}: does not print status messages to stdout;
\item \texttt{-r, --run}: runs all cases;
\item \texttt{-c, --compare}: compares chekpoint files between \textbf{repository} and \textbf{destination};
\item \texttt{-p, --post}: postprocess results of computations;
\item \texttt{-m ADDRESS1 ADDRESS2 ..., --mail=ADDRESS1 ADDRESS2 ...}: addresses for sending the reports.
\end{list}

\underline{Examples:}

\begin{list}{$\bullet$}{}
\item \texttt{code\_saturne autovnv -f sample.xml}: duplicates all cases from the \textbf{repository}
int the \textbf{destination}, compile all User Defined Functions and exits;
\item \texttt{code\_saturne autovnv -f sample.xml -r}: as above, and run all cases if defined
in \texttt{sample.xml};
\item \texttt{code\_saturne autovnv -f sample.xml -r -c}: as above, and compares all new checkpoint files
with those from the \textbf{repository} if defined in \texttt{sample.xml};
\item \texttt{code\_saturne autovnv -f sample.xml -r -c -p}: as above, and plots results
if defined in \texttt{sample.xml};
\item \texttt{code\_saturne autovnv -f sample.xml -r -c -p -m "dupont@moulinsart.be dupond@moulinsart.be"}: as above,
and send the two reports.
\item \texttt{code\_saturne autovnv -f sample.xml -c -p}: compares and plots result in the \textbf{destination}
already computed.
\end{list}

\underline{Note:}

The detailed report is generated only if the options \texttt{-c, --compare} or \texttt{-p, --post} is present
in the command line.


\section{File of parameters}

The file of parameters is of XML formatted ascii file.

\subsection{Headers of the file of parameters}

\begin{verbatim}
<?xml version="1.0"?>
<autoverif version="2.0">
    <repository>/home/dupond/codesaturne/MyRepository</repository>
    <destination>/home/dupond/codesaturne/MyDestination</destination>
\end{verbatim}

This exemple shows the four mandatory first lines of the file of parameters.

The two last lines correspond to the definition of the \textbf{repository}
and \textbf{destination} directories.
Inside the markups \texttt{<repository>} and \texttt{<destination>} the user
must inform the related directories. If the \textbf{destination} does not exit,
the directory is created.

\subsection{Case creation and compilation fo the User Defined Functions}

When \textsc{Autovnv} is launched, the file of parameters if parsed in order to known
which studies and cases from the \textbf{repository} should be duplicated in
the \textbf{destination}. The selection is done with the markups \texttt{<study>}
and \texttt{<case>} as the following example:

\begin{verbatim}
<?xml version="1.0"?>
<autoverif version="2.0">
    <repository>/home/dupond/codesaturne/MyRepository</repository>
    <destination>/home/dupond/codesaturne/MyDestination</destination>

    <study label="MyStudy1" status="on">
        <case label="Grid1" status="on" compute="on" post="off"/>
        <case label="Grid2" status="off" compute="on" post="off"/>
    </study>
    <study label="MyStudy2" status="off">
        <case label="k-eps" status="on" compute="on" post="off"/>
        <case label="Rij-eps" status="on" compute="on" post="off"/>
    </study>
</autoverif>
\end{verbatim}

The attributes are:

\begin{list}{$\bullet$}{}
\item \texttt{label}: the name of the file of the script;
\item \texttt{status}: must ne equal to \texttt{on} or \texttt{off},
activate or desactivate the markup;
\item \texttt{compute}: must ne equal to \texttt{on} or \texttt{off},
activate or desactivate the computation of the case;
\item \texttt{post}: must ne equal to \texttt{on} or \texttt{off},
activate or desactivate the post-treatment of the case.
\end{list}

All these attributes are mandatory.

With the attribute \texttt{status}, a single case or a complete study can be switched off.
If the above example, only the case \texttt{Grid1} of the study \texttt{MyStudy1} is going
to be created.

After the creation of the directories in the \textbf{destination}, for each case, all User Defined Functions
are compiled. The \textsc{Autovnv} stops if a compilation error occurs: no computation, or
comparison, or plot will be performed, even if they are switched on.

\underline{Note:}

During the duplication, every files are copied, except mesh files, for which a symbolic link is used.


\subsection{Run a case}

The run of \CS is activated if the option \texttt{-r, --run} is present in the command line.

All cases described in the file of parameters with the attribute \texttt{compute="on"}
are taken into account.

\begin{verbatim}
<?xml version="1.0"?>
<autoverif version="2.0">
    <repository>/home/dupond/codesaturne/MyRepository</repository>
    <destination>/home/dupond/codesaturne/MyDestination</destination>

    <study label="MyStudy1" status="on">
        <case label="Grid1" status="on" compute="on" post="off"/>
        <case label="Grid2" status="on" compute="off" post="off"/>
    </study>
    <study label="MyStudy2" status="on">
        <case label="k-eps" status="on" compute="on" post="off"/>
        <case label="Rij-eps" status="on" compute="on" post="off"/>
    </study>
</autoverif>
\end{verbatim}

After the computation, if no error occurs, the attribute \texttt{compute} is set to \texttt{"off"}
in the copy of the file of parameters in the \textbf{destination}. It is allow a restart
of \textsc{Autovnv} without re-run successfull previous computations.

\subsection{Compare checkpoint files}

The comparison is activated if the option \texttt{-c, --compare} is present in the command line.

In order to compare two checkpoint files, the markup \texttt{<compare>} has to be added as
a child of the condidered case. In the following exemple, a checkpoint file comparison
is switched on for the case \textit{Grid1}, whereas no comparision is planed for the case
\textit{Grid2}.
The comparison is done by the external script \texttt{cs\_io\_dump} with the option \texttt{--diff}.

\begin{verbatim}
    <study label='MyStudy1' status='on'>
        <case label='Grid1' status='on' compute="on" post="off"/>
            <compare dest="" repo="" threshold="1.0e-5"/>
        <case label='Grid2' status='on' compute="off" post="off"/>
    </study>
\end{verbatim}

The attributes are:
\begin{list}{$\bullet$}{}

\item \texttt{repo}: id of the results directory in the \textbf{repository} for example \texttt{repo="20110704-1116"},
if there is a single results directory in the \texttt{RESU} directory of the case, the id can be ommitted: \texttt{repo=""};

\item \texttt{dest}: id of the results directory in the \textbf{destination}:
\begin{list}{$\rightarrow$}{}
\item if the id is not known already because the case has not yet run, just let the attribute empty \texttt{dest=""},
\item if \textsc{Autovnv} is restarted without the run step (with the command line \texttt{code\_saturne
autovnv -f sample.xml -c} for example), the id of the results directory in
the \textbf{destination} must be given (for example \texttt{dest="20110706-1523"}), but if there is a single results directory in the \texttt{RESU} directory
of the case, the id can be ommitted: \texttt{repo=""}, the id will be completed automatically;
\end{list}

\item \texttt{threshold}: real value above which a difference is considered significant (default: $1e-30$).

\end{list}

Only the attributes \texttt{repo} and \texttt{dest} are mandatory.

\subsection{Run an external additional script with arguments}

The launch of an external script is activated if the option \texttt{-p, --post} is present in the command line.

The markup \texttt{<script>} has to be added as a child of the condidered case.

\begin{verbatim}
    <study label='STUDY' status='on'>
        <case label='CASE1' status='on' compute="on" post="on">
            <script label="script_post.py" args="-n 1" dest="" repo="20110216-2147" status="on"/>
        </case>
    </study>
\end{verbatim}

The attributes are:

\begin{list}{$\bullet$}{}
\item \texttt{label}: the name of the file of the script;
\item \texttt{status}: mandatory must ne equal to \texttt{on} or \texttt{off},
activate or desactivate the markup.
\item \texttt{args}: the arguments to pass to the script;
\item \texttt{repo} and \texttt{dest}: id of the results directory in the \textbf{repository} or
in the \textbf{destination}; if there is a single results directory in the \texttt{RESU} directory
of the case, the id can be ommitted: \texttt{repo=""} or \texttt{dest=""}, the id will be completed automatically;

\end{list}

Only the attributes \texttt{label} and \texttt{status} are mandatory.

Several calls of the same script or to different scripts are permitted:

\begin{verbatim}
    <study label="STUDY" status="on">
        <case label="CASE1" status="on" compute="on" post="on">
            <script label="script_post.py" args="-n 1" status="on"/>
            <script label="script_post.py" args="-n 2" status="on"/>
            <script label="script_post.py" args="-n 3" status="on"/>
            <script label="another_script.py" status="on"/>
        </case>
    </study>
\end{verbatim}

Example of script, which searches printed informations in the listing:
\begin{verbatim}
#!/usr/bin/env python
# -*- coding: utf-8 -*-

import os, sys
import string
from optparse import OptionParser

def process_cmd_line(argv):
    """Processes the passed command line arguments."""
    parser = OptionParser(usage="usage: %prog [options]")

    parser.add_option("-r", "--repo", dest="repo", type="string",
                      help="Directory of the result in the repository")

    parser.add_option("-d", "--dest", dest="dest", type="string",
                      help="Directory of the result in the destination")

    (options, args) = parser.parse_args(argv)

    return options

def main(options):
    m = os.path.join(options.dest, "listing")
    f = open(m)
    lines = f.readlines()
    f.close()

    g = open(os.path.join(options.dest, "water_level.dat"), "w")
    g.write("# time,   h_sim,   h_th\n")
    for l in lines:
       if l.rfind("time, h_sim, h_th") == 0:
           d = l.split()
           g.write("%s  %s  %s\n" % (d[3], d[4], d[5]))
    g.close()

if __name__ == '__main__':
    options = process_cmd_line(sys.argv[1:])
    main(options)
\end{verbatim}

\subsection{Post-treatment: plotting curves}

The post-treatment is activated if the option \texttt{-p, --post} is present in the command line.

\subsubsection{Experimental data}

\begin{verbatim}
    <study label='MyStudy' status='on'>
        <measurement file='exp.dat' path ='/home/dupond/codesaturne/MyDestination/MyStudy/DATA_EXP'>
                <plot fig='1' xcol='1' ycol='2' legend='U Experimental data'/>
                <plot fig='2' xcol='3' ycol='4' legend='V Experimental data'/>
        </measurement>
        <case label='Grid1' status='on' compute="off" post="on">
            <data file="profile.dat" dest="">
                <plot fig="1" xcol="1" ycol="2"  legend="U computed" fmt='r-s'/>
                <plot fig="2" xcol="1" ycol="3"  legend="V computed" fmt='b-s'/>
            </data>
        </case>
    </study>
    <subplot fig="1" legstatus='on'  ylabel="U ($m/s$)" xlabel= "$r$ ($m$)" legpos ='0.05 0.1'/>
    <subplot fig="2" legstatus='off' ylabel="V ($m/s$)" xlabel= "$r$ ($m$)"/>
    <figure name="MyFigure" fig="1 2"  figsize="(4,4)" />
\end{verbatim}

\subsubsection{Curves with error bar}


\subsubsection{Legends}


\subsubsection{Matplotlib raw commands}


\section{Output and restart}

\textsc{Autovnv} produces several files in the \textbf{destination} directory:
\begin{list}{$\bullet$}{}
\item \texttt{report.txt}: standard output of the script;
\item \texttt{auto\_vnv.log}: log of the code and the \texttt{pdflatex} compilation;
\item \texttt{report\_global.pdf}: summary of the compilation, run, comparison, and plot steps;
\item \texttt{report\_detailed.pdf}: details the comparison and display the plot;
\item \texttt{sample.xml}: udpated file of parameters, useful for restart the script if an error occurs.
\end{list}

After the computation of a case, if no error occurs, the attribute \texttt{compute}
is set to \texttt{"off"} in the copy of the file of parameters in
the \textbf{destination}. It is allow a restart
of \textsc{Autovnv} without re-run successfull previous computations.

In the same manner, all empty attributes \texttt{repo=""} and \texttt{dest=""}
are completed in the udpated file of parameters.

%
\end{document}
%
%%%%%%%%%%%%%%%%%%%%%%%%%%%%%%%%%%%%%%%%%%%%%%%%%%%%%%%%%%%%%%%%%%%%%%
