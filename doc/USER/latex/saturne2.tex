%
%                      Code_Saturne version 1.3
%                      ------------------------
%
%     This file is part of the Code_Saturne Kernel, element of the
%     Code_Saturne CFD tool.
%
%     Copyright (C) 1998-2008 EDF S.A., France
%
%     contact: saturne-support@edf.fr
%
%     The Code_Saturne Kernel is free software; you can redistribute it
%     and/or modify it under the terms of the GNU General Public License
%     as published by the Free Software Foundation; either version 2 of
%     the License, or (at your option) any later version.
%
%     The Code_Saturne Kernel is distributed in the hope that it will be
%     useful, but WITHOUT ANY WARRANTY; without even the implied warranty
%     of MERCHANTABILITY or FITNESS FOR A PARTICULAR PURPOSE.  See the
%     GNU General Public License for more details.
%
%     You should have received a copy of the GNU General Public License
%     along with the Code_Saturne Kernel; if not, write to the
%     Free Software Foundation, Inc.,
%     51 Franklin St, Fifth Floor,
%     Boston, MA  02110-1301  USA
%
%-----------------------------------------------------------------------
%==================================
%==================================
\section{Key word list}
%==================================
%==================================
\label{prg_motscles}

The key words are classified under headings. For each key word of the Kernel of \CS,
the following data are given:

\noindent
\makebox[2.5cm][l]{Variable name}\makebox[1.3cm][l]{Type}%
\makebox[5cm][l]{Allowed values}%
\makebox[4.cm][l]{[Default]}O/C\hspace{1cm}Level\\
\hspace*{2.5cm}Description\\
\hspace*{2.5cm}Potential dependences


\begin{list}{$\bullet$}{}
\item \textbf{Variable name}: Name of the variable containing the key word.

\item \textbf{Type}: A (Array), I (Integer), R (Real number), C
      (Character string).

\item \textbf{Allowed values}: list or range of allowed values.

\item \textbf{Default}: value defined by the code before any user
      modification (every key word has one). In some cases, a
      non-allowed value is given (generally -999 or -1.D12), to force the
      user to specify a value. If he does not do it, the code may:
\begin{list}{-}{}
\item automatically use a recommended value (for instance, automatical
      choice of the variables for which chronological records will be
      generated).

\item stop, if the key word is essential (for instance, value of the
      time step).
\end{list}

\item \textbf{O/C}: Optional/Compulsory
\begin{list}{-}{}
\item O: optional key word, whose default value may be enough.

\item C: key word which must imperatively be specified (for instance,
      the time step).
\end{list}

\item \textbf{Level}: L1, L2 ou L3
\begin{list}{-}{}
\item L1 (level 1): the users will have to modify it in the framework of
      standard applications. The L1 key words are written in bold.

\item L2 (level 2): the users may have to modify it in the framework of
      advanced applications. The L2 key word are all optional.

\item L3 (level 3): the developers may have to modify it ; it keeps its
      default value in any other case. The L3 key word are all optional.
\end{list}

\item \textbf{Description}:  key word description, with its potential
      dependences.

\end{list}

The L1 key words can be modifed through the Graphical Use Interface or
in the \texttt{usini1} subroutine. L2 and L3 key words can only be modified through
the \texttt{usini1} subroutine, even if they do not appear in the version proposed
as example it the \texttt{FORT/USERS/base} directory.\\
It is however recommended not to modify the key words which do not belong to the L1
level.

The alphabetical key word list is displayed in the index, in the end of
this report.

\minititre{Notes}
$\bullet\ $The notation ``D'' refers to a double precision real. For
           instance, 1.8D-2 means 0.018. \\
$\bullet\ $The notation ``GRAND'' (which can be used in the code)
corresponds to 1.D12.

%==================================
\subsection{Inputs-outputs}
%==================================

\minititre{Notes}
\begin{list}{$\bullet$}{}
\item Two different files can have neither the same unit number nor the
      same name.
\item  ASCII files (also called ``formatted'' files, in opposition to
      ``binary'' files) are bigger, longer to write and to read, but can
      be used on every architecture (in particular, it is an asset for
      calculation restart files). However, \CS\ can automatically recognise and
      convert Big Endian/Little Endian files. It is therefore usually possible on
      a given architecture to use binary restart files generated on another
      architecture.
 \end{list}

%==================================
\subsubsection{''Calculation'' files}
%==================================


\minititre{General}

\motcle{IMPGEO}{I}{strictly positive integer}{10}{O}{L3}
{unit of the geometric file (if the Preprocessor is not used)\\
useful if and only if IFOENV = 0}

\motcle{FICGEO}{C}{string of 6 characters}{\tt geomet}{O}{L3}
{name of the geometric file (if the Preprocessor is not used)\\
useful if and only if IFOENV = 0}

\motcle{IMPAMO}{I}{strictly positive integer}{11}{O}{L3}
{unit of the upstream restart file\\
useful if and only if ISUITE = 1}

\motcle{FICAMO}{C}{string of 13 characters}{\tt suiamo}{O}{L3}
{name of the main upstream restart file. Its ``format'' (ASCII or binary)
is automatically determined by the code.\\
useful if and only if ISUITE = 1}

\motcle{FICAMX}{C}{string of 13 characters}{\tt suiamx}{O}{L3}
{name of the auxiliary upstream restart file. Its ``format'' (ASCII or binary)
is automatically determined by the code.\\
useful if and only if ISUITE = 1}

\motcle{IMPSTP}{I}{strictly positive integer}{12}{O}{L3}
{unit of the calculation interactive stop file\\
always useful (because of the interactive character)}

\motcle{FICSTP}{C}{string of 6 characters}{\tt ficstp}{O}{L3}
{name of the calculation interactive stop file (see p.\pageref{prg_ficstp})\\
always useful (because of the interactive characteristic)}

\motcle{IMPAVA}{I}{strictly positive integer}{20}{O}{L3}
{unit of the main downstream restart file\\
always useful}

\motcle{IMPAVX}{I}{strictly positive integer}{IMPAVA}{O}{L3}
{unit of the auxiliary downstream restart file\\
always useful}

\motcle{FICAVA}{C}{string of 13 characters}{\tt suiava}{O}{L3}
{name of the main downstream restart file\\
always useful}

\motcle{FICAVX}{C}{string of 13 characters}{\tt suiavx}{O}{L3}
{name of the auxiliary downstream restart file\\
always useful}

\motcle{IFOAVA}{I}{1 or 0}{0}{O}{L2}
{indicator (1: formatted, 0: binary main downstream restart file)\\
always useful}

\motcle{IFOAVX}{I}{1 or 0}{0}{O}{L2}
{indicator (1: formatted, 0: binary auxiliary downstream restart file)\\
always useful}


\minititre{1D wall thermal module}

\motcle{IMPMT1}{I}{strictly positive integer}{\tt IMPAMO}{O}{L3}
{unit of the upstream restart file for the 1D wall thermal module\\
useful if and only if ISUIT1 = 1 and NFPT1D$>$0}

\motcle{FICMT1}{C}{string of 13 characters}{\tt t1damo}{O}{L3}
{name of the upstream restart file for the 1D wall thermal module.
Its ``format'' (ASCII or binary) is  automatically determinedby the code.\\
useful if and only if ISUIT1 = 1 and NFPT1D$>$0}

\motcle{IMPVT1}{I}{strictly positive integer}{\tt IMPAVA}{O}{L3}
{unit of the downstream restart file for the 1D wall thermal module\\
useful if and only if NFPT1D$>$0}

\motcle{FICVT1}{C}{string of 13 characters}{\tt t1dava}{O}{L3}
{name of the upstream restart file for the 1D wall thermal module\\
useful if and only if NFPT1D$>$0}

\motcle{IFOVT1}{I}{1 or 0}{\tt IFOAVA}{O}{L2}
{indicator (1: formatted, 0: binary downstream restart file for the 1D wall thermal module)\\
useful if and only if NFPT1D$>$0}

\minititre{Vortex method for LES}

\motcle{IMPMVO}{I}{strictly positive integer}{\tt IMPAMO}{O}{L3}
{unit of the upstream restart file for the vortex method\\
useful if and only if ISUIVO = 1 et IVRTEX=1}

\motcle{FICMVO}{C}{string of 13 characters}{\tt voramo}{O}{L3}
{name of the upstream restart file for the vortex method\\
Its ``format'' is always ASCII (this file has a different structure from the
other restart files)\\
useful if and only if ISUIVO = 1 et IVRTEX=1}

\motcle{IMPVVO}{I}{strictly positive integer}{\tt IMPAVA}{O}{L3}
{unit of the downstream restart file for the vortex method\\
useful if and only if IVRTEX=1}

\motcle{FICVT1}{C}{string of 13 characters}{\tt vorava}{O}{L3}
{name of the upstream restart file for the vortex method\\
Its ``format'' is always ASCII (this file has a different structure from the
other restart files)\\
useful if and only if IVRTEX=1}

\motcle{IMPDVO}{I}{strictly positive integer}{\tt IMPAVA}{O}{L3}
{unit of the FICVOR data files for the vortex method. These files have an ASCII
format. Their number and names are specified by the user in the \texttt{usvort} subroutine.\\
(Although it corresponds to an ``upstream'' data file, IMPDVO is initialised to
IMPAVA because, in case of multiple vortex entries, it is opened at the same time
as the FICMVO upstream restart file, which already uses the IMPAMO unit)\\
useful if and only if IVRTEX=1}


\minititre{Radiation}

\motcle{IMPAMR}{I}{strictly positive integer}{\tt IMPAMO}{O}{L3}
{unit of the radiation upstream restart file\\
useful if and only if ISUIRD = 1}

\motcle{FICAMR}{C}{string of 13 characters}{\tt rayamo}{O}{L3}
{name of the radiation upstream restart file. Its ``format'' (ASCII or
binary) is automatically determined by the code.\\
useful if and only if ISUIRD = 1}

\motcle{IMPAVR}{I}{strictly positive integer}{\tt IMPAVA}{O}{L3}
{unit of the radiation downstream restart file\\
always useful in case of radiation modeling}

\motcle{FICAVR}{C}{string of 13 characters}{\tt rayava}{O}{L3}
{name of the radiation downstream restart file \\
always useful in case of radiation modeling}

\motcle{IFOAVR}{I}{1 or 0}{\tt IFOAVA}{O}{L2}
{indicator (1: formatted, 0:binary radiation downstream restart file)\\
always useful in case of radiation modeling}



\minititre{Thermochemistry}

\motcle{IMPFPP}{I}{strictly positive integer}{25}{O}{L3}
{unit of the thermochemical data file\\
useful in case of gas or pulverised coal combustion or electric arc}

\motcle{FICFPP}{C}{string of 6 characters}{\tt dp\_tch}{O}{L3}
{name of the thermochemical data file. The launch script is designed to copy the
user specified thermochemical data file in the temporary execution directory
under the name \texttt{dp\_tch}, for \CS\ to open it properly. Should the value
of FICFPP be changed, the launch script would have to be adapted.\\
useful in case of gas or pulverised coal combustion}

\motcle{IMPJNF}{I}{strictly positive integer}{\tt IMPFPP}{O}{L3}
{unit of the JANAF data file\\
useful in case of gas or pulverised coal combustion}

\motcle{FICJNF}{C}{string of 5 characters}{\tt JANAF}{O}{L3}
{name of the JANAF data file. The launch script is designed to copy the
user specified JANAF data file in the temporary execution directory
under the name \texttt{JANAF}, for \CS\ to open it properly. Should the value
of FICJNF be changed, the launch script would have to be adapted.\\
useful in case of gas or pulverised coal combustion}

\minititre{Lagrangian}

\motcle{IMPAML}{I}{strictly positive integer}{\tt IMPAMO}{O}{L3}
{unit of the upstream restart file in case of Lagrangian modeling\\
useful if and only if ISUILA = 1}

\motcle{FICAML}{C}{string of 6 characters}{\tt lagamo}{O}{L3}
{name of the upstream restart file in case of Lagrangian modeling.
Its ``format'' (ASCII or
binary) is automatically determined by the code.\\
useful if and only if ISUILA = 1}

\motcle{IMPMLS}{I}{strictly positive integer}{\tt IMPAMO}{O}{L3}
{unit of the upstream restart file for the statistics in case of
Lagrangian modeling\\
useful if and only if ISUIST = 1}

\motcle{FICMLS}{C}{string of 13 characters}{\tt lasamo}{O}{L3}
{name of the upstream restart file for the statistics in case of
Lagrangian modeling. Its ``format'' (ASCII or
binary) is automatically determined by the code.\\
useful if and only if ISUIST = 1}

\motcle{IMPAVL}{I}{strictly positive integer}{\tt IMPAVA}{O}{L3}
{unit of the downstream restart file in case of Lagrangian modeling\\
always useful in case of Lagrangian modeling}

\motcle{FICAVL}{C}{string of 13 characters}{\tt lagava}{O}{L3}
{name of the downstream restart file in case of Lagrangian modeling\\
always useful in case of Lagrangian modeling}

\motcle{IFOAVL}{I}{1 or 0}{\tt IFOAVA}{O}{L2}
{indicator (1: formatted, 0: binary Lagrangian downstream restart file)\\
always useful in case of Lagrangian modeling}

\motcle{IMPVLS}{I}{strictly positive integer}{\tt IMPAVA}{O}{L3}
{unit of the downstream restart file for the statistics in case of
Lagrangian modeling\\
useful in case of Lagrangian modeling with statistics}

\motcle{FICVLS}{C}{string of 6 characters}{\tt lasava}{O}{L3}
{name of the downstream restart file for the statistics in case of
Lagrangian modeling\\
useful in case of Lagrangian modeling with statistics}

\motcle{IFOVLS}{I}{1 or 0}{\tt IFOAVA}{O}{L2}
{indicator (1: formatted, 0: binary downstream restart file for the
statistics in case of Lagrangian modeling)\\
useful in case of Lagrangian modeling with statistics}

\motcle{IMPLA1}{I}{strictly positive integer}{50}{O}{L3}
{unit of a file specific to Lagrangian modeling\\
useful in case of Lagrangian modeling}

\motcle{IMPLA2}{I}{strictly positive integer}{51}{O}{L3}
{unit of a file specific to Lagrangian modeling\\
useful in case of Lagrangian modeling}

\motcle{IMPLA3}{I}{strictly positive integer}{52}{O}{L3}
{unit of a file specific to Lagrangian modeling\\
useful in case of Lagrangian modeling}

\motcle{IMPLA4}{I}{strictly positive integer}{53}{O}{L3}
{unit of a file specific to Lagrangian modeling\\
useful in case of Lagrangian modeling}

\motcle{IMPLA5}{IA}{strictly positive integer}{54 to 68}{O}{L3}
{units of files specific Lagrangian modeling, 15-dimension array\\
useful in case of Lagrangian modeling}


%==================================
\subsubsection{Post-processing for {\em EnSight} or other tools}
%==================================

\minititre{Notes}
$\bullet\quad$The format depends on the user choices.\\
$\bullet\quad$The post-processing files, directly generated by the Kernel
through the FVM library, can be of the following formats: {\em Ensight Gold},
{\em MED\_fichier} or {\em CGNS}. The use of the two latter formats depends on
the installation of the corresponding external libraries.\\
$\bullet\quad$For each quantity (problem unknow, preselected numerical
variable or preselected physical parameter), the user specifies if a
post-processing output is wanted. The output frequency can be set.\\


\motcle{ICHRVL}{I}{0 or 1}{1}{O}{L3}
{indicates whether post-processing outputs are wanted (=1) or not (=0)
on the 3D volume mesh\\
always useful}

\motcle{ICHRBO}{I}{0 or 1}{0}{O}{L2}
{indicates whether post-processing outputs are wanted (=1) or not (=0)
on the 2D boundary mesh\\
always useful}

 \motcle{ICHRSY}{I}{0 or 1}{0}{O}{L2}
 {indicates whether post-processing outputs are wanted (=1) or not (=0)
 on the 2D boundary mesh patches coupled with the Syrthes conjugate
 heat trabsfer code\\
 always useful}

 \motcle{ICHRMD}{I}{0, 1, 2, 10, 11 or 12}{0}{O}{L2}
 {indicates whether the post-processing geometry varies with time:\\
 \hspace*{1.3cm}= 0: time independent\\
 \hspace*{1.3cm}= 1: deforming or moving mesh\\
 \hspace*{1.3cm}= 2: changing vertex coordinates and topology\\
 \hspace*{1.3cm}= 10: time independent base, with time dependent nodal displacement field\\
 \hspace*{1.3cm}= 11: deforming or moving mesh, plus nodal displacement field\\
 \hspace*{1.3cm}= 12: changing vertex coordinates and topology, plus nodal displacement field}

 \motcleb{FMTCHR}{C}{string of less than 32 characters}{Ensight Gold}{O}{L1}
 {name of the output format, among the following:\\
 \hspace*{1.3cm}$\bullet\quad$ ``Ensight Gold''\\
 \hspace*{1.3cm}$\bullet\quad$ ``MED\_fichier'' (if available)\\
 \hspace*{1.3cm}$\bullet\quad$ ``CGNS'' (if available)}

 \motcle{OPTCHR}{C}{string of less than 96 characters}{binary}{O}{L2}
 {options associated to the selected output format. The string is given as a
 series of key words, separated by a comma (and optional spaces). The key words
 are among the following:\\
 \hspace*{1.3cm}$\bullet\quad${\em text} for a text format (for {\em EnSight})\\
 \hspace*{1.3cm}$\bullet\quad${\em binary} for a binary format (default choice)\\
 \hspace*{1.3cm}$\bullet\quad${\em big\_endian} to force outputs to be in {\em big-endian}
 mode; this can be useful when using {\em ParaView}, which uses this mode by default.\\
 \hspace*{1.3cm}$\bullet\quad${\em discard\_polygons} to prevent from exporting faces with more
 than four edges (which may not be recognised by some post-processing tools); such
 faces will therefore appear as ``holes'' in the post-processing mesh.\\
 \hspace*{1.3cm}$\bullet\quad${\em discard\_polyhedra} to prevent from exporting elements which
 are neither tetrahedra, prisms, pyramids nor hexahedra (which may not be recognised by
 some post-processing tools); such elements will therefore appear as ``holes'' in
 the post-processing mesh\\
 \hspace*{1.3cm}$\bullet\quad${\em divide\_polygons} to divide faces with more than four edges
 into triangles, so that any post-processing tool can recognise them\\
 \hspace*{1.3cm}$\bullet\quad${\em divide\_polyhedra} to divide elements which are neither
 tetrahedra, prisms, pyramids nor hexahedra into simpler elements (tetrahedra and
 pyramids), so that any post-processing tool can recognise them\\
 \hspace*{1.3cm}$\bullet\quad${\em split\_tensors} to export the components of a tensor
 variable as a series of independent variables (always the case for now)}


\motcleb{NTCHR}{I}{-1 or strictly positive integer}{-1}{O}{L1}
{output period for the post-processing\\
\hspace*{1.3cm}= -1: only at the end of the calculation\\
\hspace*{1.3cm}$>$ 0: period (every NTCHR time step)\\
always useful}


\motcleb{ICHRVR}{IA}{-999, 0 or 1}{-999}{O}{L1}
{for each quantity defined at the cell centers (physical or numerical
variable), indicator of whether it should be post-processed or not \\
\hspace*{1.3cm}= -999: not initialised. By default, the post-processed
quantities are the unknowns (pressure, velocity, $k$, $\varepsilon$,
$R_{ij}$, $\omega$, $\varphi$, $\overline{f}$, scalars), the density,
the turbulent viscosity and the time step if is not uniform\\
\hspace*{1.3cm}= 0: not post-processed\\
\hspace*{1.3cm}= 1: post-processed\\
useful if and only if the variable is defined at the cell centers:
calculation variable, physical property (time step, density,
viscosity, specific heat) or turbulent viscosity if ITURB(IPHAS)
$\geqslant$ 10}

\motcleb{IPSTDV}{I}{integer $\geqslant 1$: see below}{IPSTYP*IPSTCL*IPSTFT}{O}{L1}
{indicates the data to post-process on the boundary mesh (the boundary mesh must
have been activated with ICHRBO=1). The value of IPSTDV is
the product of the following integers, depending on the variables that should be
post-processed:\\
\hspace*{1.3cm}IPSTYP\index{IPSTYP}: $y^+$ at the boundary\\
\hspace*{1.3cm}IPSTCL\index{IPSTCL}: value of the variables at the
boundary (using the boundary conditions but without reconstruction)\\
\hspace*{1.3cm}IPSTFT\index{IPSTFT}: thermal flux at the boundary
($W\,m^{-2}$), if a thermal scalar has been defined (ISCALT)\\
For instance, with IPSTDV=IPSTYP*IPSTCL, $y^+$ and the variables will be
post-processed at the boundaries.\\
With IPSTDV=1, none of these data are post-processed at the boundaries.\\
always useful if ICHRBO=1}

%==================================
\subsubsection{Chronological records of the variables on specific points}
%==================================

\minititre{Standard use through Interface or \texttt{usini1}}
For each quantity (problem unknown, preselected numerical variable or
preselected physical parameter), the user indicates whether chronological records
should be generated, the output period and the position of the
probes. The code produces chronological records at the cell centers located
closest to the geometric points defined by the user by means of their
coordinates. For each quantity, the number of probes and their
index-numbers must be specified (it is not mandatory to generate all
the variables at all the probes).


\motcleb{NCAPT}{I}{positive or null integer}{0}{O}{L1}
{total number of probes (limited to NCAPTM=100)\\
always useful }

\motcleb{XYZCAP}{RA}{real numbers}{0.D0}{O}{L1}
{3D-coordinates of the probes\\
the coordinates are written: XYZCAP(I,J), with I = 1, 2 or 3 and J
$\leqslant$ NCAPT\\
useful if and only if NCAPT $>$ 0}

\motcleb{IHISVR}{IA}{-999, -1 or positive or null integer}{-999}{O}{L1}
{number IHISVR(N, 1) and index-numbers IHISVR(N, J$>$1) of the record
probes to be used for each variable, {\em i.e.} calculation variable
or physical property defined at the cell centers.
With IHISVR(N, 1)=-999 or -1, IHISVR(N, J$>$1) is useless.\\
\hspace*{.5cm} $\bullet$ IHISVR(N, 1): number of record probes to use
for the variable N\\
\hspace*{1.3cm}= -999: by default: chronogical records are generated on
all the probes if N is one of the main variables (pressure, velocity,
turbulence, scalars), the local time step or the turbulent
viscosity. For the other quantities, no chronological record is generated.\\
\hspace*{1.3cm}= -1: chronological records are produced on all the probes\\
\hspace*{1.3cm}= 0: no chronological record on any probe\\
\hspace*{1.3cm}$>0$: chronological record on IHISVR(N, 1) probes to be specified
with  IHISVR(N, J$>$1)\\
always useful, must be inferior or equal to NCAPT\\
\hspace*{.5cm} $\bullet$ IHISVR(N, J$>$1): index-numbers of the probes
used for the variable N\\
(with J$\leqslant$IHISVR(N,1)+1)\\
\hspace*{1.3cm}= -999: by default: if IHISVR(N, 1) $\ne$
-999, the  code stops. Otherwise, refer to the description of the case
IHISVR(N, 1)=-999\\
useful if and only if IHISVR(N, 1) $>$ 0 \\
The condition IHISVR(N, J) $\leqslant$NCAPT must be respected.\\
For an easier use, it is recommended to simply specify IHISVR(N,1)=-1 for
all the interesting variables.}

\motcle{IMPHIS}{IA}{strictly positive integer}{30 and 31}{O}{L3}
{working units for the production of chronological record files by the Kernel\\
useful if and only if chronological files are produced ({\em i.e.} there
is N for which IHISVR(N, 1) $\ne$ 0)}

\motcle{EMPHIS}{C}{string of less than 80 characters}{\tt ./}{O}{L3}
{directory in which the potential chronological record files generated by
the Kernel will be written (path related to the execution directory)\\
it is recommended to keep the default value and, if necessary, to modify the
launch script to copy the files in the alternate destination directory\\
useful if and only if chronological record files are generated ({\em
i.e.} there is N for which IHISVR(N, 1) $\ne$ 0)}

\motcle{EXTHIS}{C}{string of less than 80 characters}{hst}{O}{L3}
{extension of the chronological record files\\
useful if and only if chronological record files are generated ({\em
i.e.} there is N for which IHISVR(N, 1) $\ne$ 0)}

\motcleb{NTHIST}{I}{-1 or strictly positive integer}{1 or -1}{O}{L1}
{output period of the chronological record files\\
\hspace*{1.3cm}= -1: no output\\
\hspace*{1.3cm}$>$ 0: period  (every NTHIST time step)\\
The default value is -1 if there is no chronological record file to
generate (if there is no probe, NCAPT = 0, or if IHISVR(N, 1)=0 for
all the variables) and 1 otherwise\\
If chronological records are generated, it is usually wise to keep the default
value NTHIST=1, in order to avoid missing any high frequency evolution (unless
the total number of time steps is much too big)\\
useful if and only if chronological record files are generated ({\em
i.e.} there are probes (NCAPT$>$0) there is N for which IHISVR(N, 1)
$\ne$ 0)}

\motcle{NTHSAV}{I}{-1 or positive or null integer}{0}{O}{L3}
{saving period the chronological record files (they are first stored in a
temporary file and then saved every NTHSAV time step)\\
\hspace*{1.3cm}= 0: by default (4 times during a calculation)\\
\hspace*{1.3cm}= -1: saving at the end of the calculation\\
\hspace*{1.3cm}$>$ 0: period (every NTHSAV time step)\\
During the calculation, the user can read the chronological record files
in the execution directory when they have been saved, {\em i.e.} at the first
time step, at the tenth time step and when the time step number is a multiple of
NTHSAV (multiple of (NTMABS-NTPABS)/4 if NTHSAV=0)\\
{\em Note: using the \texttt{ficstp} file allows to update the value of
NTMABS. Hence, if the calculation is at the time step n, the saving of the
chronological record files can be forced by changing NTMABS to NTPABS+4(n+1)
using \texttt{ficstp}; after the files have been saved, NTMABS can be put back
to its original value, still using \texttt{ficstp}.}\\
useful if and only if chronological record files are generated ({\em
i.e.} there are probes (NCAPT$>$0) there is N for which IHISVR(N, 1)
$\ne$ 0)}

\minititre{Non-standard use through \texttt{ushist}}
(see p.\pageref{prg_ushist})

\motcle{IMPUSH}{IA}{strictly positive integer}{\tt 33 to 32+NUSHMX=49}{O}{L3}
{units of the user chronological record files\\
useful if and only if the subroutine \texttt{ushist} is used}

\motcle{FICUSH}{CA}{strings of 13 characters}{\texttt{ush*} or
\texttt{ush*.n\_*}}{O}{L2}
{names of the user chronological record files.
In the case of a non-parallel calculation, the suffix applied the file
name is a three digit number: \texttt{ush001}, \texttt{ush002},
\texttt{ush003}...
In the case of a parallel-running calculation, the processor
index-number is added to the suffix. For instance, for a calculation
running on two processors:  \texttt{ush001.n\_0001},
\texttt{ush002.n\_0001}, \texttt{ush003.n\_0001}... and
\texttt{ush001.n\_0002},
\texttt{ush002.n\_0002}, \texttt{ush003.n\_0002}...
The opening, closing, format and location of these files must be managed
by the user.\\
useful if and only if the subroutine \texttt{ushist} is used}


%==================================
\subsubsection{Time averages}
%==================================

The code allows the calculation of time averages of the type
$<f_1*f_2...*f_n>$. The variables $f_i$ (defined at the cell centers)
which may be taken into account are the followings:
\begin{list}{-}{}
\item the solved calculation variables (velocity, pressure ...),
\item the auxiliary variables from the array PROPCE (density and
      physical properties when they are variable in space).
\end{list}

The averages are treated like auxiliary variables defined at the cell
centers and stored in the PROPCE array. The standard post-processing
actions may therefore be activated, like the writing in the listing or
the output of result files (EnSight, MED, ...). However, if the user
wants to manipulate the averages in a more advanced way, it is
recommended to refer first to the user subroutines \texttt{usproj} and
\texttt{usvpst} which provide examples. Indeed, the PROPCE array
does not contain the time averages directly, but only the cumulated value
of the product $f_1*f_2...*f_n$ of the selected
variables $f_i$. The division by the cumulated duration is done only
before the writing of the results. See also page \pageref{prg_moyennes}.

To calculate $p$ time averages of the type $<f_1*f_2...*f_{n(IMOM)}>$,
the user must:
\begin{list}{-}{}
\item make sure that $p\leqslant$NBMOMX (do not overstep the maximum
      number of averages),
\item make sure that $n(IMOM)\leqslant$NDGMOX for every average IMOM
       (do not overstep the maximum degree, {\it i.e} the maximum number
       of variables which may compose an average),
\item define every average IMOM (1$\leqslant$IMOM$\leqslant p$, without
      skipping any index-number) by marking out the $n(IMOM)$ variables
      which form it by means of the array IDFMOM(II,IMOM) (with
      1$\leqslant$II$\leqslant$n(IMOM)),
\item define for each average IMOM the time step number at which the
      calculation of the cumulated value must begin, by means of the
      array NTDMOM(IMOM).
\end{list}

The total number of averages ($p$=NBMOMT) is automatically determined by
the code from the values of IDTMOM. The user must not specify specify it.

\motcleb{IDFMOM}{IA}{0, $\pm$ variable index-number}{0}{O}{L1}
{Index-number of the variables composing a time average of the type
$<f_1*f_2...*f_n>$. For every time average IMOM to calculate:\\
\hspace*{1.3cm} - if IDFMOM(II,IMOM) is positive, it refers to the
                  index-number of a solved variable (stored in the array
                  RTP), like for instance a velocity component
                  (IU(IPHAS), IV(IPHAS), IW(IPHAS)) or the pressure
                  (IPR(IPHAS))\\
\hspace*{1.3cm} - if IDFMOM(II,IMOM) is negative, it refers to the
                  index-number of an auxiliary variable (stored in
                  PROPCE), like for instance the density
                  (IDFMOM(II,IMOM)=-IROM(IPHAS))\\
useful if and only if the user wants to calculate time averages}

\motcleb{NTDMOM}{IA}{integer}{-1}{O}{L1}
{For every average IMOM to calculate, absolute time step number at which
the calculation should begin. The value -1 means ``never''. Every
strictly negative value (in particular -1) will considered an error and
cause the calculation to stop (because the user is supposed to want to
calculate the averages he has defined)\\
useful if and only if the user wants to calculate time averages}

\motcleb{IMOOLD}{IA}{-2, 1$\leqslant$ integer $\leqslant$ JBMOMT}{-2}{O}{L1}
{Correspondence table of the averages in the case of a calculation
restart. In this case, for every average IMOM in the current
calculation (1$\leqslant$IMOM$\leqslant$NBMOMX), IMOOLD(IMOM) gives the
index-number of the corresponding average in the previous calculation
(in which JBMOMT averages were calculated). \\
\hspace*{1.3cm} - if IMOOLD(IMOM) = -2, the user lets the code automatically
            determine the
                correspondence. By default, the average II in the
                current calculation will correspond to the average II in
                the previous calculation, if it existed.
                Otherwise, II will be a new average.\\
\hspace*{1.3cm} - if IMOOLD(IMOM) = -1, the average is reset to zero.\\
\hspace*{1.3cm} - if IMOOLD(IMOM) = KK, the average IMOM will correspond
to the average KK=IMOOLD(IMOM) in the previous calculation.\\
useful if and only if the user wants to calculate averages.
Allows to add or suppress some averages, to reset them, to change their
order, ...\\
{\em Warning : if the calculation is not a restart, IMOOLD must not be
specified (its value must remain -2)}}

%==================================
\subsubsection{Others}
%==================================

\motcle{IMPUSR}{IA}{strictly positive integer}{\tt 70 to 69+NUSRMX=79}{O}{L3}
{unit numbers for potential user specified files\\
useful if and only if the user needs files (therefore always useful, by security)}

\motcleb{FICUSR}{CA}{string of 13 characters}{\tt usrf* or usrf*.n\_*}{O}{L1}
{name of the potential user specified files. In the case of a non-parallel
calculation, the suffix applied the file name is a two digit number:
from \texttt{usrf01} to \texttt{usrf10}. In the case of a
parallel-running calculation, the four digit processor index-number is
added to the suffix. For instance, for a calculation running on two
processors: from \texttt{usrf01.n\_0001} to \texttt{usrf10.n\_0001} and
from \texttt{usrf01.n\_0002} to \texttt{usrf10.n\_0002}. The opening,
closing, format and location of these files must be managed by the user.\\
useful if and only if the user needs files (therefore always useful, by security)}

\motcleb{ILISVR}{IA}{-999, 1 or 0}{-999}{O}{L1}
{for every quantity (variable, physical or numerical property ...),
indicator concerning the writing in the execution report file \\
\hspace*{1.3cm}= -999: automatically converted into 1 if the concerned
quantity is one of the main variables (pressure, velocity, turbulence,
scalar), the density, the time step if IDTVAR $\ne$ 0 or the turbulent
viscosity. Otherwise converted into 0.\\
\hspace*{1.3cm}= 1: writing in the execution listing.\\
\hspace*{1.3cm}= 0: no writing.\\
always useful}

\motcleb{IWARNI}{IA}{integer}{0}{O}{L1}
{IWARNI(IVAR) characterises the level of detail of the outputs for the
variable IVAR (from 1 to NVAR). The quantity of information increases with
its value.\\
Impose the value 0 or 1 for a reasonable listing size. Impose the value 2
to get a maximum quantity of information, in case of problem during the
execution.\\
always useful}

\motcleb{NOMVAR}{CA}{string of less than 80 characters}{``''}{O}{L1}
{name of the variables (unknowns, physical properties ...): used in the
execution listing, in the post-processing files, etc.\\
{``''}: not initialised (the code chooses the manes by default)\\
It is recommended not to define variable names of more than 8
characters, to get a clear execution listing (some advanced writing
levels take into account only the first 8 characters).\\
always useful}

\motcleb{NTLIST}{I}{-1 or strictly positive integer}{1}{O}{L1}
{writing period in the execution report file\\
\hspace*{1.3cm}= -1: no writing\\
\hspace*{1.3cm}$>$ 0: period (every NTLIST time step)\\
The value of NTLIST must be adapted according to the number of iterations
carried out in the calculation. Keeping NTLIST to 1 will indeed provide
a maximum volume of information, but if the number of time steps is too large,
the execution report file might become too big and unusable
(problems with disk space, memory problems while opening the file with a text
editor, problems finding the desired information in the file, ...).\\
always useful}

\motcle{NTSUIT}{I}{-1, 0 or positive or null integer}{0}{O}{L3}
{saving period of the restart files\\
\hspace*{1.3cm}= -1: only at the end of the calculation\\
\hspace*{1.3cm}= 0: by default (four times during the calculation)\\
\hspace*{1.3cm}$>$ 0: period\\
always useful}


%==================================
\subsection{Numerical options}
%==================================
\subsubsection{Calculation management}
%==================================

\motcle{IECAUX}{I}{0 or 1}{1}{O}{L2}
{indicates the writing (=1) or not (=0) of the auxiliary calculation
restart file\\
always useful}

\motcle{ILEAUX}{I}{0 or 1}{1}{O}{L2}
{indicates the reading (=1) or not (=0) of the auxiliary
calculation restart file\\
useful if and only if ISUITE=1}

\motcleb{INPDT0}{I}{0 or 1}{0}{O}{L1}
{indicates the calculation mode: 1 for a zero time step control
calculation, {\em i.e.} without solving the transport equations,
and 0 for a standard calculation.\\
In case of a calculation using the control mode (INPDT0=1), when the
calculation is not a restart, the equations are not solved, but the
physical properties and the boundary conditions are calculated. When
the calculation is a restart, the physical properties and the boundary
conditions are those read from the restart file (note: in the case of a
second-order time scheme, the mass flow is modified as if a normal
time step was realised: the mass flow generated in an potential
post-processing is therefore not the mass flow read from the restart file).\\
In the control mode (INPDT0=1), the variable NTMABS is not used.\\
In the standard mode (INPDT0=0), the code solves the equations at least
once, even if NTMABS=0.\\
always useful}

\motcleb{ISUITE}{I}{0 or 1}{0}{C}{L1}
{indicator of a calculation restart (=1) or not (=0)\\
always useful}

\motcle{NTCABS}{I}{integer}{\tt NTPABS}{O}{L3}
{current time step number\\
always useful\\
NTCABS is initialised and updated automatically by the code, its value is not to
be modified by the user}

\motcleb{NTMABS}{I}{integer $>$ NTPABS}{10}{C}{L1}
{number of the last time step after which the calculation stops. It is
an absolute number: for the restart calculations, NTMABS takes into
account the number of time steps of the previous calculations. For
instance, after a first calculation of 3 time steps, a restart file of 2
time steps is realised by setting NTMABS=3+2=5\\
always useful}

\motcle{NTPABS}{I}{integer}{0, read}{O}{L3}
{number of the last time step in the previous calculation. In the case of
a restart calculation, NTPABS is read from the restart file. Otherwhise
it is initialised to 0\\
always useful\\
NTPABS is initialised automatically by the code, its value is not to
be modified by the user}

\motcle{TMARUS}{R}{-1D0 or strictly positive real}{-1D0}{O}{L3}
{margin in seconds on the remaining CPU time which is necessary to allow
the calculation to stop automatically and write all the required results
(for the machines having a queue manager)\\
\hspace*{1.3cm}= -1: calculated automatically\\
\hspace*{1.3cm}$>$ 0: margin defined by the user\\
always useful, but the default value should not be changed
unless absolutely necessary.}


\motcle{TTCABS}{R}{positive or null real number}{\tt TTPABS}{O}{L3}
{physical simulation time at the current time step. For the restart
calculations, \mbox{TTCABS} takes into account the physical time of the
previous calculations.\\
If the time step is uniform (IDTVAR=0 or 1), TTCABS increases of DT (value of
the time step) at each iteration. If the time step is non-uniform (IDTVAR=2),
TTCABS increases of DTREF at each time step.\\
always useful\\
TTCABS is initialised and updated automatically by the code, its value is not to
be modified by the user}

\motcle{TTPABS}{R}{positive or null real number}{0, read}{O}{L3}
{simulation physical time at the last time step of the previous
calculation. In the case of a restart calculation, TTPABS is read from
the restart file. Otherwhise it is initialised to 0.\\
always useful\\
TTCABS is initialised automatically by the code, its value is not to
be modified by the user}

%==================================
\subsubsection{Scalar unknowns}
%==================================


\motcleb{ISCOLD}{IA}{-999, 1$\leqslant$ integer $\leqslant$ JSCAL}{-999}{O}{L1}
{correspondence table of the scalars in the case of a calculation
restart. For a calculation restart with NSCAL scalars, ISCOLD(ISCAL)
gives, for every scalar ISCAL of the current calculation
(1$\leqslant$ISCAL$\leqslant$NSCAL), the index-number of the
corresponding scalar in the previous calculation (in which JSCAL scalars were
taken into account).\\
\hspace*{1.3cm} ISCOLD(ISCAL) = -999: the code automatically determines the
correspondence. By default, the following rules are applied:\\
\hspace*{2.cm} - the user scalar II of the current calculation is
initialised by the the user scalar II of the previous calculation, if
this scalar existed already (otherwise, II is a new scalar).\\
\hspace*{2.cm} - the particular physics scalar JJ is initialised by
the particular physics scalar JJ of the previous calculation if this
scalar existed already (otherwise, JJ is a new scalar).\\
\hspace*{1.3cm} ISCOLD(ISCAL) = KK: the scalar ISCAL (user or particular
physics scalar) is initialised by the scalar KK=ISCOLD(ISCAL) of the
previous calculation.\\
always useful. Allows to add or remove some scalars, to change the
solving order, to change the physics, ...}

\motcleb{NSCAUS}{I}{0$\leqslant$ integer $\leqslant$ NSCMAX}{0}{O}{L1}
{number of user scalars solutions of an advection equation\\
always useful}

\motcleb{ISCAVR}{IA}{0, 1 $\leqslant$ integer $\leqslant$ NSCAL}{0}{O}{L1}
{if the scalar ISCAL is the average of the square of the fluctuations of a
scalar KK, then \mbox{ISCAVR(ISCAL)=KK}.
Otherwise ISCAVR(ISCAL)=0. For ISCAL and KK, the user can only use index-numbers
refering to user scalars ($\leqslant$ NSCAUS). \\
always useful}

\motcle{IPHSCA}{IA}{1 $\leqslant$ integer $\leqslant$ NPHAS}{0}{O}{L3}
{for every scalar ISCAL, IPHSCA(ISCAL) is the index-number of the
associated phase\\
always useful}

\motcleb{ISCALT}{IA}{-1 or integer $>$ 0}{-1}{O}{L1}
{for every phase IPHAS, ISCALT(IPHAS) is the index-number of the scalar
representing the temperature or the enthalpy. If ISCALT(IPHAS)=-1, no
scalar represents the temperature nor the enthalpy. When a specific
physics module is activated (gas combustion, pulverised coal,
electricity or compressible),
the user must not modify ISCALT (the choice is made
automatically)\footnote{in the case of the compressible module, ISCALT does not
correspond to the temperature nor enthalpy but to the total energy}.\\
useful if and only if NSCAL $\geqslant$ 1}

\motcleb{ISCSTH}{IA}{-1, 0, 1, 2 or 3}{-10}{O}{L1}
{type of scalar\\
\hspace*{1.3cm}= -10: not specified. By default, the code chooses
ISCSTH(ISCAL)=0 for the scalars apart from ISCALT(IPHAS)\\
\hspace*{1.3cm}= -1: temperature in degrees Celsius (use only in case of
radiation modeling)\\
\hspace*{1.3cm}= 0: passive scalar\\
\hspace*{1.3cm}= 1: temperature (in Kelvin if the radiation modeling is
activated)\\
\hspace*{1.3cm}= 2: enthalpy\\
\hspace*{1.3cm}= 3: total energy (this value is automatically chosen by the code
when using the compressible module, it must never be used otherwise and must
never be specified by the user)\\
useful if and only if NSCAL $\geqslant$ 1. The distinction between
ISCSTH(ISCAL) = -1 or 1 (respectively degrees Celsius or Kelvin) is
useful only in case of radiation modeling. For calculations without
radiation modeling, use ISCSTH(ISCAL)=1 for the temperature. When a
particular physics module is activated (gas combustion, pulverised coal,
electricity or compressible), the user must not modify ISCSTH (the choice is made
automatically: the solved variable is the enthalpy or the total energy).\\
It is also reminded that, in the case of a coupling with
\syrthes, the solved thermal variable should be the temperature
(ISCSTH(ISCALT(IPHAS))=1 or -1).
More precisely, everything is designed in the code to allow for the
running of a calcualtion coupled with \syrthes\ with the enthalpy as thermal
variable (the correspondence and conversion is then specified by the user in
the subroutine \texttt{usthht}).
However this case has never been used in practice and has therefore not been
tested. With the compressible model, it is possible to carry out calculations
coupled with \syrthes, although the thermal scalar represents the total
energy and not the temperature.}

\motcle{ICLVFL}{IA}{-1, 0, 1 or 2}{-1}{O}{L3}
{for every scalar ISCAL representing the average of the square of the
fluctuations of another scalar II=ISCAVR(ISCAL) (noted $f$),
indicator of the clipping method\\
\hspace*{1.3cm}= -1: no clipping because the scalar does not represent
the average of the square of the fluctuations of another scalar\\
\hspace*{1.3cm}= 0: clipping to 0 for lower values\\
\hspace*{1.3cm}= 1: clipping to 0 for lower values and to
\mbox{$(f-f_{min})(f_{max}-f)$} for higher values, where $f$ is
the associated scalar, $f_{min}$ and $f_{max}$ its minimum and maximum
values specified by the user ({\em i.e.} SCAMIN(II) and SCAMAX(II)) \\
\hspace*{1.3cm}= 2: clipping to MAX(0,SCAMIN(ISCAL)) for lower
values and to SCAMAX(ISCAL) for higher values. SCAMIN and SCAMAX
are limits specified by the user\\
useful for the scalars ISCAL for which ISCAVR(ISCAL)$>$0.}

\motcle{ITBRRB}{I}{0 or 1}{0}{O}{L3}
{Reconstruction (=1) or not (=0) of the temperature, enthalpy or total energy
value in the boundary cells. Useful in the case of coupling with \syrthes\
and with radiation.}

\motcle {ICPSYR}{TI}{-999,0,1}{-999}{O}{L3}
{For each scalar ISCAL, ICPSYR(ISCAL) indicates if it is
coupled with \syrthes\ (=1) or not (=0).
There can be only one coupled scalar per calculation.\\
\hspace*{1.3cm}=-999: by default\\
\hspace*{2.cm} $\bullet$ ICPSYR(ISCAL)=1 for the thermal scalar
ISCAL=(ISCALT(IPHAS)) when a coupling with \syrthes\ has been specified in the
Interface or the launch script\\
\hspace*{2.cm} $\bullet$ ICPSYR(ISCAL)=0 otherwise\\
\hspace*{1.3cm}= 0: the scalar ISCAL is not coupled with \syrthes\\
\hspace*{1.3cm}= 1: the scalar ISCAL is coupled with \syrthes\\
useful in case of coupling with \syrthes}


%==================================
\subsubsection{Definition of the equations}
%==================================

\motcle{ISTAT}{IA}{0 or 1}{1 or 0}{O}{L2}
{for each unknown IVAR to calculate, indicates if
non-stationary terms are present (ISTAT(IVAR)=1) or not (0) in the matrices.\\
By default, ISTAT is set to 0 for the pressure (variable IVAR=IPR(IPHAS))
or $\overline{f}$ in v2f modeling (variable IVAR=IFB(IPHAS)) and set to
1 for the other unknowns.\\
useful for all the unknowns}

\motcle{ICONV}{IA}{0 or 1}{1 or 0}{O}{L2}
{for each unknown IVAR to calculate, indicates if the
convection is taken into account (ICONV(IVAR)=1) or not (0).\\
By default, ICONV is set to 0 for the pressure (variable IVAR=IPR(IPHAS))
or $\overline{f}$ in v2f modeling (variable IVAR=IFB(IPHAS)) and set to
1 for the other unknowns.\\
useful for all the unknowns}

\motcle{IDIFF}{IA}{0 or 1}{1}{O}{L2}
{for each unknown IVAR to calculate, indicates if the
diffusion is taken into account (IDIFF(IVAR)=1) or not (0)\\
useful for all the unknowns}

\motcle{IDIFFT}{IA}{0 or 1}{1}{O}{L3}
{for each unknown IVAR to calculate, when diffusion is taken
into account (IDIFF(IVAR)=1), IDIFFT(IVAR) indicates if the turbulent
diffusion is taken into account (IDIFFT(IVAR)=1) or not (0)\\
useful for all the unknowns }

\motcle{IDIRCL}{IA}{0 or 1}{1 or 0}{O}{L3}
{for each unknown IVAR to calculate, indicates whether the diagonal
of the matrix should be slightly shifted (IDIRCL(IVAR)=1) or not (0) if there
is no Dirichlet boundary condition and if ISTAT=0. Indeed, in such a case,
the matrix for the general advection/diffusion equation is singular. A slight
shift in the diagonal will make it invertable again.\\
By default, IDIRCL is set to 1 for all the unknowns, except $\overline{f}$ in
v2f modeling, since its equation contains another diagonal term that ensures
the regularity of the matrix.\\
useful for all the unknowns}

\motcle{IVISSE}{IA}{0 or 1}{1}{O}{L3}
{for each phase IPHAS, indicates whether the source terms in transposed gradient
and velocity divergence should be taken into account in the momentum
equation. In the compressible module, these terms also account for the volume
viscosity (cf. VISCV0 et IVISCV):\\
$\partial_i \left[(\kappa -2/3\,(\mu+\mu_t))\partial_k U_k  \right]
 +     \partial_j \left[ (\mu+\mu_t)\partial_i U_j \right]$ \\
\hspace*{1.3cm}= 0: not taken into account\\
\hspace*{1.3cm}= 1: taken into account\\
always useful}


%==================================
\subsubsection{Definition of the time advancement}
%==================================

\motcleb{IDTVAR}{I}{0, 1, 2}{0}{O}{L1}
{type of time step\\
\hspace*{1.3cm}= 0: constant in time and spatially uniform\\
\hspace*{1.3cm}= 1: variable in time and spatially uniform\\
\hspace*{1.3cm}= 2: variable in time and in space\\
If the numerical scheme is a second-order in time, only the option 0 is
allowed.\\
always useful}

\motcle{IPTLRO}{I}{0 or 1}{0}{O}{L2}
{when density gradients and gravity are present, a local thermal time
step can be calculated, based on the Brunt-Vaissala frequency. In numerical
simulations, it is usually wise for the time step to be lower than this limit,
otherwise numerical instabilities may appear\\
IPTLRO indicates whether the time step should be limited to the local thermal
time step (=1) or not (=0)\\
when IPTLRO=1, the listing shows the number of cells where the time step has
been clipped due to the thermal criterium, as well as the maximum ratio between
the time step and the maximum thermal time step. If IDTVAR=0, since the time
step is fixed and cannot be clipped, this ratio can be larger than
1\footnote{it is then the user's
choice to decide whether he should diminish DTREF or not}. When IDTVAR$>$0, this
ratio will be smaller than 1, except if the constraint DTMIN has prevented the
code from reaching a sufficiently low value for DT\\
useful when density gradients and gravity are present}

\motcleb{CDTVAR}{RA}{strictly positive real number}{1.D0}{O}{L1}
{multiplicative factor applied to the time step for each scalar\\
Hence, the time step used when solving the evolution equation for the
variable is the time step used for the dynamic equations (velocity/pressure)
multiplied by CDTVAR.\\
The size of the array CDTVAR is NVAR. For instance, the multiplicative
coefficient applied to the scalar 2 is CDTVAR(ISCA(2))). Yet, the value of CDTVAR
for the velocity components and the pressure is not used. Also, although it is
possible to change the value of CDTVAR for the turbulent variables, it is
highly unrecommended\\
useful if and only if NSCAL $\geqslant$ 1}


\motcle{COUMAX}{R}{strictly positive real number}{1D0}{O}{L1}
{target local or maximum Courant number in case of non-constant time step\\
useful if IDTVAR $\ne$ 0}

\motcle{FOUMAX}{R}{strictly positive real number}{10D0}{O}{L1}
{target local or maximum Fourier number in case of non-constant time step\\
useful if IDTVAR $\ne$ 0}

\motcleb{DTREF}{R}{strictly positive real number}{\tt -GRAND*10}{C}{L1}
{reference time step \\
always useful.\\
It is the time step value used in the case of a calculation run with a
uniform and constant time step, {\em i.e.} IDTVAR=0 (restart calculation
or not). It is the value used to initialise the time step in the case of
an initial calculation (ISUITE=0) run with a non-constant time step
(IDTVAR=1 or 2). It is also the value used to initialise the time step
in the case of a restart calculation (ISUITE=1) in which the type of
time step has been changed (for instance, IDTVAR=1 in the new
calculation and IDTVAR=0 or 2 in the previous calculation): see
\texttt{usiniv}}

\motcle{DTMIN}{R}{positive or null real number}{\tt 0.1D0*DTREF}{O}{L2}
{lower limit for the calculated time step when non-constant time step is activated\\
useful if IDTVAR $\ne$ 0}

\motcle{DTMAX}{R}{strictly positive real number}{\tt 1000*DTREF}{O}{L2}
{upper limit for the calculated time step when non-constant time step is activated\\
useful if IDTVAR $\ne$ 0}

\motcle{VARRDT}{R}{strictly positive real number}{0.1D0}{O}{L3}
{maximum allowed relative increase in the calculated time step value
between two succesive time steps (to ensure stability, any decrease in the time step
is immediate and without limit)\\
useful if IDTVAR $\ne$ 0}


\minititre{Non-constant time step}
The calculation of the time step uses a reference time step DTREF (at
the calculation beginning). Later, every time step, the time step value
is calculated by taking into account the different existing limits, in
the following order: \\
\hspace*{1.cm}$\bullet$ COUMAX, FOUMAX: the more restrictive limit between
both is used (in the compressible module, the acoustic limitation is added),\\
\hspace*{1.cm}$\bullet$ VARRDT:  progressive increase and immediate
decrease in the time step,\\
\hspace*{1.cm}$\bullet$ IPTLRO: limitation by the thermal time step,\\
\hspace*{1.cm}$\bullet$ DTMAX and DTMIN: clipping of the time step to
the maximum, then to the minimum limit.\\


%==================================
\subsubsection{Turbulence}
%==================================

\motcleb{ITURB}{IA}{0, 10, 20, 21, 30, 31, 40, 41, 50 or 60}{-999}{O}{L1}
{for each phase IPHAS, indicator of the turbulence model ITURB(IPHAS)\\
\hspace*{1.3cm}= -999: not initalised. This value is not allowed and
must be modified by the user\\
\hspace*{1.3cm}= 0: laminar\\
\hspace*{1.3cm}= 10: mixing length (not valided)\\
\hspace*{1.3cm}= 20: $k-\varepsilon$\\
\hspace*{1.3cm}= 21: $k-\varepsilon$ with linear production (Laurence \& Guimet)\\
\hspace*{1.3cm}= 30: $R_{ij}-\varepsilon$ ``standard'' LRR (Launder, Reece \& Rodi)\\
\hspace*{1.3cm}= 31: $R_{ij}-\varepsilon$ SSG (Speziale, Sarkar \& Gatski)\\
\hspace*{1.3cm}= 40: LES (Smagorinsky model) \\
\hspace*{1.3cm}= 41: LES (dynamic model) \\
\hspace*{1.3cm}= 50: v2-f, $\varphi$-model version\\
\hspace*{1.3cm}= 60: $k-\omega$, SST version \\
always useful}

The $k-\varepsilon$ (standard and linear production) and $R_{ij}-\varepsilon$
(LRR and SSG) turbulence
models implemented in \CS\ are ``High-Reynolds'' models. It is therefore
necessary to make sure that the thickness of the first cell neighboring
the wall is larger than the thickness of the viscous sublayer (at the
wall, $y^+>2.5$ is required as a minimum, and preferably between 30 and
100)\footnote{While creating the mesh, $y^+=\frac{yu*}{\nu}$ is
generally unknown. It can be roughly estimated as $\frac{yU}{10\nu}$, where
$U$ is the characteristic velocity, $\nu$ is the kinematic viscosity of the fluid
 and $y$ is the mid-height of the first cell near the wall.}. If the mesh does
 not respect this condition, the results may be biased
(particularly if thermal processes are involved). Using scalable wall-functions
(cf. key word IDEUCH) may help avoiding this problem.\\
The v2-f model is a ``Low-Reynolds'' model, it is therefore necessary to
make sure that the thickness of the first cell neighboring the wall is
smaller than the thickness of the viscous sublayer ($y^+<1$).\\
The $k-\omega$ SST model provides correct results whatever the thickness of the first cell.
Yet, it requires the knowledge of the distance to the wall in every
cell of the calculation domain. The user may refer to the key word
ICDPAR\index{ICDPAR} for more details about the potential limitations.\\
The $k-\varepsilon$ model with linear production allows to correct the
known flaw of the standard $k-\varepsilon$ model which overestimates the
turbulence level in case of strong velocity gradients (stopping point).\\
With LES, the wall functions are usually not greatly adapted. It is generally more advisable
(if possible) to refine the mesh towards the wall so that the first cell is in the
viscous sublayer, where the boundary conditions are simple natural no-slip conditions.\\
Concerning the LES model, the user may refer to the subroutine
\texttt{ussmag} for complements about the dynamic model. Its usage
and the interpretation of its results require particular attention.\\

\motcle{IDEUCH}{IA}{0, 1 or 2}{0 or 1}{O}{L2}
{for each phase IPHAS, indicates the type of wall function is used for the velocity
boundary conditions on a frictional wall.\\
\hspace*{1.3cm}= 0: one-scale model\\
\hspace*{1.3cm}= 1: two-scale model\\
\hspace*{1.3cm}= 2: scalable wall function\\
IDEUCH is initialised to 0 for ITURB(IPHAS)=0, 10, 40 or 41 (laminar,
mixing length, LES).\\
IDEUCH is initialised to 1 for ITURB(IPHAS)=20, 21, 30, 31 or 60
($k-\varepsilon$, $R_{ij}-\varepsilon$ LRR, $R_{ij}-\varepsilon$ SSG and
$k-\omega$ SST models).\\
The v2f model (ITURB(IPHAS)=50) is not designed to use wall functions
(the mesh must be ``low Reynolds'').\\
The value IDEUCH(IPHAS)=1 is not compatible with ITURB(IPHAS)=0, 10, 40
or 41 (laminar, mixing length and LES).\\
Concerning the $k-\varepsilon$ and $R_{ij}-\varepsilon$ models, the
two-scales model is usually at least as satisfactory as the one-scale
model.\\
The scalable wall function allows to virtually ``shift'' the wall when
necessary in order to be always in a logarithmic layer.
It is used to make up for
the problems related to the use of High-Reynolds models on very refined
meshes.\\
useful if ITURB(IPHAS) is different from 50}

\motcle{ILOGPO}{IA}{0 or 1}{1}{O}{L3}
{for each phase IPHAS, type of wall function used for the velocity: power law
(ILOGPO(IPHAS)=0) or logarithmic law (ILOGPO(IPHAS)=1)\\
always useful}

\motcle{YPLULI}{RA}{real number $>$ 0}{\tt 1/XKAPPA,10.88D0}{O}{L3}
{for each phase IPHAS, limit value of $y^+$ for the viscous sublayer\\
YPLULI depends on the chosen wall function: it is
initialised to 10.88D0 for the scalable wall function (IDEUCH(IPHAS)=2),
otherwise it is initialised to $1/\kappa\approx 2,38$\\
In LES, YPLULI is taken by default to be 10.88D0\\
always useful}

\minititre{ $k-\varepsilon$, $k-\varepsilon$ with linear production,
v2-f and $k-\omega$ SST}

\motcle{IGRAKE}{IA}{0 or 1}{1}{O}{L1}
{for each phase IPHAS, indicates if the terms related to gravity in the
equations of $k$ and $\varepsilon$ or $\omega$ are taken into account
(IGRAKE(IPHAS)=1) or not (0)\\
useful if and only if ITURB(IPHAS) = 20, 21, 50 or 60, (GX,GY,GZ)
$\ne$ (0,0,0) and the density is not uniform}

\motcle{IGRHOK}{IA}{0 or 1}{0}{O}{L2}
{for each phase IPHAS, indicates if the term $\frac{2}{3}\grad \rho k$
is taken into account\\ (IGRHOK(IPHAS)=1) or not (0) in the velocity
equation\\
useful if and only if ITURB(IPHAS) = 20, 21, 50 or 60.\\
This term may generate
non-physical velocities at the wall. When it is not explicitely taken into
account, it is implicitely included into the pressure.}

\motcle{IKECOU}{IA}{0 or 1}{0 or 1}{O}{L3}
{for each phase IPHAS, indicates if the coupling of the source terms of
$k$ and $\varepsilon$ or $k$ and $\omega$ is taken into account
(IKECOU(IPHAS)=1) or not (0)\\
if IKECOU=0 in $k-\varepsilon$ model, the term in $\varepsilon$ in the
equation of $k$ in made implicit\\
IKECOU(IPHAS) is initialised to 0 if ITURB(IPHAS) = 21 or 60, and to 1 if\\
ITURB(IPHAS)= 20\\
IKECOU(IPHAS)=1 is forbidden when using the v2f model
(ITURB(IPHAS)=50)\\
useful if and only if ITURB(IPHAS) = 20, 21 or 60 ($k-\varepsilon$ and
$k-\omega$ models)}

\motcle{RELAXK}{RA}{0.D0 $\leqslant$ real $\leqslant$ 1.D0}{0.7D0}{O}{L3}
{for each phase IPHAS, relaxation coefficient of the turbulent
variables ($k$ and $\varepsilon$ or $\omega$) when IKECOU(IPHAS) = 0. If
IKECOU(IPHAS)=1, RELAXK is not used, whatever its value may be.\\
useful if and only if ITURB(IPHAS) = 20, 21, 50 or 60 and IKECOU(IPHAS)=0
($k-\varepsilon$, v2f or $k-\omega$ models without coupling)}

\motcle{ICLKEP}{IA}{0 or 1}{0}{O}{L3}
{for each phase IPHAS, indicates the clipping method used for $k$ and
$\varepsilon$, for the $k-\varepsilon$ and v2f models\\
\hspace*{1.3cm}= 0: clipping in absolute value\\
\hspace*{1.3cm}= 1: clipping from physical relations\\
useful if and only if ITURB(IPHAS) = 20, 21 or 50 ($k-\varepsilon$ and
v2f models). The results obtained with the method corresponding to
ICLKEP(IPHAS)=1 showed in some cases a substantial sensitivity to the
values of the length scale ALMAX(IPHAS).\\
The option ICLKEP(IPHAS)=1 is therefore not recommended, and,
if chosen, must be used cautiously.}

\minititre{ $R_{ij}-\varepsilon$ (LRR and SSG)}

\motcle{ICLPTR}{IA}{0 or 1}{0}{O}{L3}
{for each phase IPHAS, indicates if $R_{ij}$ is made partially implicit
(ICLPTR(IPHAS)=1) or not (0) in the wall boundary conditions.\\
useful if and only if ITURB(IPHAS) = 30 or 31 ($R_{ij}-\varepsilon$ model)}

\motcle{ICLSYR}{IA}{0 or 1}{0}{O}{L3}
{for each phase IPHAS, indicates if $R_{ij}$ is made partially implicit
(ICLSYR(IPHAS)=1) or not (0) in the symmetry boundary conditions.\\
useful if and only if ITURB(IPHAS) = 30 or 31 ($R_{ij}-\varepsilon$ model)}

\motcle{IDIFRE}{IA}{0 or 1}{1}{O}{L3}
{for each phase IPHAS, complete (IDIFRE(IPHAS)=1) or simplified (0)
taking into account of the diagonals of the diffusion tensors of $R_{ij}$
and $\varepsilon$, for the LLR model.\\
useful if and only if ITURB(IPHAS) = 30 (LLR $R_{ij}-\varepsilon$ model)}

\motcle{IGRARI}{IA}{0 or 1}{1}{O}{L1}
{for each phase IPHAS, indicates if the terms related to gravity are
taken into account (IGRARI(IPHAS)=1) or not (0) in the equations of
$R_{ij}-\varepsilon$. \\
useful if and only if ITURB(IPHAS) = 30 or 31 and (GX,GY,GZ) $\ne$
(0,0,0) ($R_{ij}-\varepsilon$ model with gravity) and the density is not uniform}

\motcle{IRIJEC}{IA}{0 or 1}{0}{O}{L2}
{for each phase IPHAS, indicates if the wall echo terms in
$R_{ij}-\varepsilon$ LRR model are
taken into account (IRIJEC(IPHAS)=1) or not (0).\\
useful if and only if ITURB(IPHAS) = 30 ($R_{ij}-\varepsilon$ LRR).\\
It is not recommended to take these terms into account:
they have an influence only near the walls, their expression is hardly
justifiable according to some authors and, in the configurations
studied with \CS\, they did not bring any improvement in the results.\\
In addition, their use induces an increase in the calculation
time.\\
The wall echo terms imply the calculation of the distance to the wall
for every cell in the domain. See ICDPAR for potential restrictions due
to this.}

\motcle{IRIJNU}{IA}{0 or 1}{0}{O}{L3}
{for each phase IPHAS, addition (IRIJNU(IPHAS)=1) or not (0) of a
turbulent viscosity in the matrix of the incermental system solved
for the velocity in $R_{ij}-\varepsilon$ models. The goal is to improve
the stability of the calculation. The usefulness of IRIJNU(IPHAS)=1 has
however not been clearly demonstrated.\\
Since the system is solved in incremental form, this extra turbulent
viscosity does not change the final solution for steady flows. However,
for unsteady flows, the parameter NSWRSM should be increased.\\
useful if and only if ITURB(IPHAS) = 30 or 31 ($R_{ij}-\varepsilon$ model).}

\motcle{IRIJRB}{IA}{0 or 1}{0}{O}{L3}
{for each phase IPHAS, reconstruction (IRIJRB(IPHAS)=1) or not (0) of the
boundary conditions at the walls for $R_{ij}$ and  $\varepsilon$.\\
useful if and only if ITURB(IPHAS) = 30 or 31 ($R_{ij}-\varepsilon$ model)}


\minititre{$LES$}

\motcle{IVRTEX}{I}{0 or 1}{0}{O}{L1}
{activates (=1) or not (=0) the generation of synthetic turbulence at the
different inlet boundaries with the LES model (generation of unsteady synthetic
eddies)\\
useful if ITURB(IPHAS)=40 or 41\\
this key word requires the completion of the routine
\texttt{usvort}}

\motcle{ISUIVO}{I}{0 or 1}{ISUITE}{O}{L1}
{for the vortex method, indicates whether the synthetic vortices at the inlet
should be initialised (=0) or read form the restart file FICMVO.\\
useful if ITURB(IPHAS)=40 or 41 and IVRTEX=1}


\motcle{IDRIES}{IA}{0 or 1}{0,1}{O}{L2}
{for each phase IPHAS, IDRIES(IPHAS) activates (1) or not (0) the van
Driest wall-damping for the Smagorinsky constant (the Smagorinsky
constant is multiplied by the damping function
$1-e^{-y^+/CDRIES(IPHAS)}$, where $y^+$ designates the adimensional
distance to the nearest wall). The default value is 1 for the
Smagorinsky model and 0 for the dynamic model.\\
the van Driest wall-damping requires the knowledge of the distance to the
nearest wall for each cell in the domain. Refer to key word ICDPAR for potential
limitations\\
useful if and only if ITURB(IPHAS) = 40 or 41}

\motcle{CDRIES}{RA}{real number $>$ 0}{26.D0}{O}{L3}
{for each phase IPHAS, CDRIES(IPHAS) is the constant appearing in the van Driest
damping function applied to the Smagorinsky constant: $1-e^{-y^+/CDRIES(IPHAS)}$\\
useful if and only if ITURB(IPHAS) = 40 or 41}

\motcle{CSMAGO}{RA}{real number $>$ 0}{0.065D0}{O}{L2}
{for each phase IPHAS, CSMAGO(IPHAS) is the Smagorinsky constant used in
the Smagorinsky model for LES\\
the sub-grid scale viscosity is calculated by
 $\displaystyle\mu_{sg}=\rho C_{smago}^2\bar{\Delta}^2\sqrt{2\bar{S}_{ij}\bar{S}_{ij}}$
where $\bar{\Delta}$ is the width of the filter and $\bar{S}_{ij}$ the filtered
strain rate\\
useful if and only if ITURB(IPHAS) = 40}

\motcle{SMAGMX}{RA}{real number $>$ 0}{10.D0*CSMAGO}{O}{L3}
{for each phase IPHAS, SMAGMX(IPHAS)**2 is
the maximum allowed value for the variable $C$ appearing in the LES dynamic
model (the ``square'' comes from the fact that the
variable of the dynamic model corresponds to the square of the
constant of the Smagorinsky model). Any larger value yielded by the calculation
procedure of the dynamic model will be clipped to SMAGMX(IPHAS)**2\\
useful if and only if ITURB(IPHAS) = 41}

\motcle{XLESFL}{RA}{real number $>$ 0}{2.D0}{O}{L3}
{for each phase IPHAS, XLESFL(IPHAS) is a constant used to define, for
each cell $\Omega_i$, the width of the (implicit) filter:\\
$\overline{\Delta}=XLESFL(IPHAS)(ALES(IPHAS)*|\Omega_i|)^{BLES(IPHAS)}$\\
useful if and only if ITURB(IPHAS) = 40 or 41}

\motcle{ALES}{RA}{real number $>$ 0}{1.D0}{O}{L3}
{for each phase IPHAS, ALES(IPHAS) is a constant used to define, for
each cell $\Omega_i$, the width of the (implicit) filter:\\
$\overline{\Delta}=XLESFL(IPHAS)(ALES(IPHAS)*|\Omega_i|)^{BLES(IPHAS)}$\\
useful if and only if ITURB(IPHAS) = 40 or 41}

\motcle{BLES}{RA}{real number $>$ 0}{1.D0/3.D0}{O}{L3}
{for each phase IPHAS, BLES(IPHAS) is a constant used to define, for
each cell $\Omega_i$, the width of the (implicit) filter:\\
$\overline{\Delta}=XLESFL(IPHAS)(ALES(IPHAS)*|\Omega_i|)^{BLES(IPHAS)}$\\
useful if and only if ITURB(IPHAS) = 40 or 41}

\motcle{XLESFD}{RA}{real number $>$ 0}{1.5D0}{O}{L3}
{for each phase IPHAS, XLESFD(IPHAS) is the constant used to define, for
each cell $\Omega_i$, the width of the explicit filter used in the framework of
the LES dynamic model:\\
$\widetilde{\overline{\Delta}}=XLESFD(IPHAS)\overline{\Delta}$\\
useful if and only if ITURB(IPHAS) = 41}



%==================================
\subsubsection{Time scheme}
%==================================

By default, the standard time scheme is a first-order.
A second-order scheme is activated automatically with LES modeling.
On the other hand, when ``specific physics'' (gas combustion, pulverised coal,
compressible module) are activated, the second-order scheme is not allowed.

In the current version, the second-order time scheme is not compatible
with the estimators (IESCAL), the velocity-pressure coupling (IPUCOU),
the modeling of hydrostatic pressure (ICALHY and IPHYDR) and the time- or
space-variable time step (IDTVAR).

Also, in the case of a rotation periodicity, a proper second-order is not
ensured for the velocity, but calculations remain possible.

It is recommended to keep the default values of the variables listed
below. Hence, in standard cases, the user does not need to specify these
options.


\motcle{ISCHTP}{IA}{1 or 2}{1 or 2}{O}{L2}
{for each phase IPHAS, ISCHTP(IPHAS) indicates the order of the activated
time scheme (this indicator allows the code to automatically complete
the other indicators related to the time scheme)\\
\hspace*{1.3cm}= 1: first-order \\
\hspace*{1.3cm}= 2: second-order \\
when ISCHTP(IPHAS)=2, the physical properties are by default not
second-order. It it possible to modify this by means of the
following indicators.\\
due to specific coupling between certain variables, the source terms in the
turbulence equations (except convection and diffusion) cannot be second order,
except with the $R_{ij}$ models (cf. key word ISTO2T)\\
by default, ISCHTP(IPHAS) is initialised to 2 with the LES model and 1
otherwise\\
always useful}

\motcle{ISTMPF}{IA}{0, 1 or 2}{0 or 1}{O}{L3}
{for each phase IPHAS, ISTMPF(IPHAS) specifies the time scheme activated
for the mass flow. The chosen value for ISTMPF(IPHAS) will automatically
determine the value given to the variable THETFL(IPHAS)\\
\hspace*{1.3cm}= 0: ''explicit'' first-order: the mass flow calculated
at the previous time step (``n'') is used in the convective terms of all
the equations (momentum, turbulence and scalars\\
\hspace*{1.3cm}= 1: ``standard'' first-order: the mass flow calculated
at the previous time step (``n'') is used in the convective terms of the
momentum equation, and the updated mass flow (time ``n+1'') is used in the
equations of turbulence and scalars\\
\hspace*{1.3cm}= 2: second-order: the mass flow used in the momentum equations
is extrapolated at ``n+THETFL'' (=n+1/2) from the values at the two former time
steps (Adams Bashforth); the mass flow used in the equations for turbulence and
scalars is interpolated at time ``n+THETFL'' (=n+1/2) from the values at the
former time step and at the newly calculated ``n+1'' time step.\\
by default, ISTMPF(IPHAS)=2 is used in the case of a second-order time
scheme (if ISCHTP(IPHAS)=2) and ISTMPF(IPHAS)=1 otherwise\\
always useful}

\motcle{ISNO2T}{IA}{0, 1 or 2}{0 or 1}{O}{L3}
{for each phase IPHAS, ISNO2T(IPHAS) specifies the time scheme activated
for the source terms of the momentum equation, apart from convection and
diffusion (for instance: head loss, transposed gradient, ...).\\
\hspace*{1.3cm}= 0: ''standard'' first-order: the terms which are linear
functions of the solved variable are implicit and the others are explicit\\
\hspace*{1.3cm}= 1: second-order: the terms of the form $S_i\phi$ which are
linear functions of the solved variable
$\phi$ are expressed as second-order terms by interpolation (according
to the formula
$(S_i\phi)^{n+\theta}=S_i^n[(1-\theta)\phi^n+\theta\phi^{n+1}]$, $\theta$
being given by the value of THETAV associated with the variable $\phi$)
; the other terms $S_e$ are expressed as second-order terms by
extrapolation (according to the formula
$(S_e)^{n+\theta}=[(1+\theta)S_e^n-\theta S_e^{n-1}]$, $\theta$ being
given by the value of THETSN(IPHAS)=0.5D0)\\
\hspace*{1.3cm}= 2: the linear terms $S_i\phi$ are treated in the same
way as when ISNO2T=1;
the other terms $S_e$ are extrapolated according to the same formula
as when ISNO2T=1, but with $\theta$=THETSN(IPHAS)=1.D0\\
by default, ISNO2T(IPHAS) is initialised to 1 (second-order) when the selected
time scheme is second-order (ISCHTP=2), otherwise to 0.\\
always useful}

\motcle{ISTO2T}{IA}{0, 1 or 2}{0}{O}{L3}
{for each phase IPHAS, ISTO2T(IPHAS) specifies the time scheme activated
for the source terms of the turbulence equations (related to $k$,
$R_{ij}$, $\varepsilon$, $\omega$, $\varphi$, $\overline{f}$), apart
from convection and diffusion.\\
\hspace*{1.3cm}= 0: ''standard'' first-order: the terms which are linear
functions of the solved variable are implicit and the others are explicit\\
\hspace*{1.3cm}= 1: second-order: the terms of the form $S_i\phi$ which are
linear functions of the solved variable
$\phi$ are expressed as second-order terms by interpolation (according
to the formula
$(S_i\phi)^{n+\theta}=S_i^n[(1-\theta)\phi^n+\theta\phi^{n+1}]$, $\theta$
being given by the value of THETAV associated with the variable $\phi$)
; the other terms $S_e$ are expressed as second-order terms by
extrapolation (according to the formula
$(S_e)^{n+\theta}=[(1+\theta)S_e^n-\theta S_e^{n-1}]$, $\theta$ being
given by the value of THETST(IPHAS)=0.5D0)\\
\hspace*{1.3cm}= 2: the linear terms $S_i\phi$ are treated in the same
way as when ISTO2T=1;
the other terms $S_e$ are extrapolated according to the same formula
as when ISTO2T=1, but with $\theta$=THETST(IPHAS)=1.D0\\
due to certain specific couplings between the turbulence equations,
ISTO2T(IPHAS) is allowed the value 1 or 2 only for the $R_{ij}$ models
(ITURB(IPHAS)=30 or 31); hence, it is always initialised to 0.\\
always useful}


\motcle{ISSO2T}{IA}{0, 1 or 2}{0 or 1}{O}{L3}
{for each scalar ISCAL, ISSO2T(ISCAL) specifies the time scheme activated
for the source terms of the equation for the scalar, apart from convection and
diffusion (for instance: variance production, user-specified terms, ...).\\
\hspace*{1.3cm}= 0: ''standard'' first-order: the terms which are linear
functions of the solved variable are implicit and the others are explicit\\
\hspace*{1.3cm}= 1: second-order: the terms of the form $S_i\phi$ which are
linear functions of the solved variable
$\phi$ are expressed as second-order terms by interpolation (according
to the formula
$(S_i\phi)^{n+\theta}=S_i^n[(1-\theta)\phi^n+\theta\phi^{n+1}]$, $\theta$
being given by the value of THETAV associated with the variable $\phi$)
; the other terms $S_e$ are expressed as second-order terms by
extrapolation (according to the formula
$(S_e)^{n+\theta}=[(1+\theta)S_e^n-\theta S_e^{n-1}]$, $\theta$ being
given by the value of THETSS(ISCAL)=0.5D0)\\
\hspace*{1.3cm}= 2: the linear terms $S_i\phi$ are treated in the same
way as when ISSO2T=1;
the other terms $S_e$ are extrapolated according to the same formula
as when ISSO2T=1, but with $\theta$=THETSS(ISCAL)=1.D0\\
by default, ISSO2T(ISCAL) is initialised to 1 (second-order) when the selected
time scheme is second-order (ISCHTP=2), otherwise to 0.\\
always useful}


\motcle{IROEXT}{IA}{0, 1 or 2}{0}{O}{L3}
{for each phase IPHAS, IROEXT(IPHAS) specifies the time scheme activated
for the physical property $\phi$ ``density''.\\
\hspace*{1.3cm}= 0: ''standard'' first-order: the value calculated at
the beginning of the current time step (from the
variables known at the end of the previous time step) is used \\
\hspace*{1.3cm}= 1: second-order: the physical property $\phi$ is
extrapolated according to the formula
$\phi^{n+\theta}=[(1+\theta)\phi^n-\theta \phi^{n-1}]$, $\theta$ being
given by the value of THETRO(IPHAS)=0.5D0 \\
\hspace*{1.3cm}= 2: first-order: the physical property $\phi$ is
extrapolated at $n+1$ according to the
same formula as when IROEXT=1 but with $\theta$=THETRO(IPHAS)=1.D0\\
always useful}

\motcle{IVIEXT}{IA}{0, 1 or 2}{0}{O}{L3}
{for each phase IPHAS, IVIEXT(IPHAS) specifies the time scheme activated
for the physical property $\phi$ ``total viscosity''
(molecular+turbulent or sub-grid viscosities).\\
\hspace*{1.3cm}= 0: ''standard'' first-order: the value calculated at
the beginning of the current time step (from the
variables known at the end of the previous time step) is used \\
\hspace*{1.3cm}= 1: second-order: the physical property $\phi$ is
extrapolated according to the formula
$\phi^{n+\theta}=[(1+\theta)\phi^n-\theta \phi^{n-1}]$, $\theta$ being
given by the value of THETVI(IPHAS)=0.5D0 \\
\hspace*{1.3cm}= 2: first-order: the physical property $\phi$ is
extrapolated at $n+1$ according to the
same formula as when IVIEXT=1, but with $\theta$=THETVI(IPHAS)=1.D0\\
always useful}

\motcle{ICPEXT}{IA}{0, 1 or 2}{0}{O}{L3}
{for each phase IPHAS, ICPEXT(IPHAS) specifies the time scheme activated
for the physical property $\phi$ ``specific heat''.\\
\hspace*{1.3cm}= 0: ''standard'' first-order: the value calculated at
the beginning of the current time step (from the
variables known at the end of the previous time step) is used \\
\hspace*{1.3cm}= 1: second-order: the physical property $\phi$ is
extrapolated according to the formula
$\phi^{n+\theta}=[(1+\theta)\phi^n-\theta \phi^{n-1}]$, $\theta$ being
given by the value of THETCP(IPHAS)=0.5D0 \\
\hspace*{1.3cm}= 2: first-order: the physical property $\phi$ is
extrapolated at $n+1$ according to the
same formula as when ICPEXT=1, but with $\theta$=THETCP(IPHAS)=1.D0\\
always useful}

\motcle{IVSEXT}{IA}{0, 1 ou 2}{0}{O}{L3}
{for each scalar ISCAL, IVSEXT(ISCAL) specifies the time scheme activated
for the physical property $\phi$ ``diffusivity''.\\
\hspace*{1.3cm}= 0: ''standard'' first-order: the value calculated at
the beginning of the current time step (from the
variables known at the end of the previous time step) is used \\
\hspace*{1.3cm}= 1: second-order: the physical property $\phi$ is
extrapolated according to the formula
$\phi^{n+\theta}=[(1+\theta)\phi^n-\theta \phi^{n-1}]$, $\theta$ being
given by the value of THETVS(ISCAL)=0.5D0 \\
\hspace*{1.3cm}= 2: first-order: the physical property $\phi$ is
extrapolated at $n+1$ according to the
same formula as when IVSEXT=1, but with $\theta$=THETVS(ISCAL)=1.D0\\
always useful}

\motcle{THETAV}{RA}{0.D0 $\leqslant$ real $\leqslant$1.D0}{1.D0 or 0.5D0}{O}{L3}
{for each variable IVAR, THETAV(IVAR) is the value of $\theta$ used to
express at the second-order the terms of convection, diffusion and the
source terms which are linear functions of the solved variable
(according to the formula
$\phi^{n+\theta}=(1-\theta)\phi^n+\theta\phi^{n+1}$). Generally,
only the values 1.0D0 and 0.5D0 are used. The user is not allowed to modify
this variable.\\
\hspace*{1.3cm}= 1.D0: first-order \\
\hspace*{1.3cm}= 0.5D0: second-order \\
Concerning the pressure, the value of THETAV is always 1.0D0. Concerning
the other variables, the value THETAV=0.5D0 is used when the
second-order time scheme is activated by ISCHTP=2 (standard value for
LES calculations), otherwise THETAV is set to 1.0D0.\\
always useful}

\motcle{THETFL}{RA}{0.D0 $\leqslant$ real $\leqslant$1.D0}{ 0.D0 or 0.5D0}{O}{L3}
{for each phase IPHAS, THETFL(IPHAS) is the value of $\theta$ used to
interpolate the convective fluxes of the variables when a second-order time
scheme has been activated for the mass flow (see ISTMPF)\\
generally, only the value 0.5D0 is used. The user is not allowed to
modify this variable.\\
\hspace*{1.3cm}= 0.0D0: ``explicit'' first-order (corresponds to
ISTMPF(IPHAS)=0 or 1)\\
\hspace*{1.3cm}= 0.5D0: second-order (corresponds to ISTMPF(IPHAS)=2). The mass
flux will be interpolated according to the formula
$Q^{n+\theta}=\frac{1}{2-\theta}Q^{n+1}+\frac{1-\theta}{2-\theta}Q^{n+1-\theta}$).\\
always useful}

\motcle{THETSN}{RA}{0.D0 $\leqslant$ real $\leqslant$1.D0}{0.0D0, 0.5D0 or 1.D0}{O}{L3}
{for each phase IPHAS, THETSN(IPHAS) is the value of $\theta$ used to
extrapolate the non linear explicit source terms $S_e$ of the momentum equation,
when the source term extrapolation has been activated (see ISNO2T),
following the formula\\
$(S_e)^{n+\theta}=(1+\theta)S_e^n-\theta S_e^{n-1}$\\
the value
of $\theta$=THETSN(IPHAS) is deduced from the value chosen for
ISNO2T(IPHAS). Generally, only the value 0.5D0 is used. The user is not
allowed to modify this variable.\\
\hspace*{1.3cm}= 0.0D0: first-order (unused, corresponds to ISNO2T(IPHAS)=0) \\
\hspace*{1.3cm}= 0.5D0: second-order (used when ISNO2T(IPHAS)=1) \\
\hspace*{1.3cm}= 1.0D0: first-order (used when ISNO2T(IPHAS)=2) \\
always useful}

\motcle{THETST}{RA}{0.D0 $\leqslant$ real $\leqslant$1.D0}{0.0D0, 0.5D0 or 1.D0}{O}{L3}
{for each phase IPHAS, THETST(IPHAS) is the value of $\theta$ used to
extrapolate the non linear explicit source terms $S_e$ of the turbulence equations,
when the source term extrapolation has been activated (see ISTO2T),
following the formula\\
$(S_e)^{n+\theta}=(1+\theta)S_e^n-\theta S_e^{n-1}$\\
the value
of $\theta$=THETSN(IPHAS) is deduced from the value chosen for
ISTO2T(IPHAS). Generally, only the value 0.5D0 is used. The user is not
allowed to modify this variable.\\
\hspace*{1.3cm}= 0.0D0: first-order (unused, corresponds to ISTO2T(IPHAS)=0) \\
\hspace*{1.3cm}= 0.5D0: second-order (used when ISTO2T(IPHAS)=1) \\
\hspace*{1.3cm}= 1.0D0: first-order (used when ISTO2T(IPHAS)=2) \\
always useful}

\motcle{THETSS}{RA}{0.D0 $\leqslant$ real $\leqslant$1.D0}{0.0D0, 0.5D0 or 1.D0}{O}{L3}
{for each scalar ISCAL, THETSS(ISCAL) is the value of $\theta$ used to
extrapolate the non linear explicit source terms $S_e$ of the scalar equation,
when the source term extrapolation has been activated (see ISSO2T),
following the formula\\
$(S_e)^{n+\theta}=(1+\theta)S_e^n-\theta S_e^{n-1}$\\
the value
of $\theta$=THETSS(ISCAL) is deduced from the value chosen for
ISSO2T(ISCAL). Generally, only the value 0.5D0 is used. The user is not
allowed to modify this variable.\\
\hspace*{1.3cm}= 0.0D0: first-order (unused, corresponds to ISSO2T(ISCAL)=0) \\
\hspace*{1.3cm}= 0.5D0: second-order (used when ISSO2T(ISCAL)=1) \\
\hspace*{1.3cm}= 1.0D0: first-order (used when ISSO2T(ISCAL)=2) \\
useful if NSCAL$>$1}


\motcle{THETRO}{RA}{0.D0 $\leqslant$ real $\leqslant$1.D0}{0.0D0, 0.5D0 or 1.D0}{O}{L3}
{for each phase IPHAS, THETRO(IPHAS) is the value of $\theta$ used to
extrapolate the physical property $\phi$ ``density'' when the extrapolation has
been activated (see IROEXT),according to the
formula $\phi^{n+\theta}=(1+\theta)\phi^n-\theta \phi^{n-1}$\\
the value of $\theta$=THETRO(IPHAS) is deduced from the value chosen for
IROEXT(IPHAS). Generally, only the value 0.5D0 is used. The user is not
allowed to modify this variable.\\
\hspace*{1.3cm}= 0.0D0: first-order (unused, corresponds to
IROEXT(IPHAS)=0)\\
\hspace*{1.3cm}= 0.5D0: second-order (corresponds to IROEXT(IPHAS)=1) \\
\hspace*{1.3cm}= 1.0D0: first-order (corresponds to IROEXT(IPHAS)=2) \\
always useful}

\motcle{THETVI}{RA}{0.D0 $\leqslant$ real $\leqslant$1.D0}{0.0D0, 0.5D0 or 1.D0}{O}{L3}
{for each phase IPHAS, THETVI(IPHAS) is the value of $\theta$ used to
extrapolate the physical property $\phi$ ``total viscosity'' when the extrapolation has
been activated (see IVIEXT),according to the
formula $\phi^{n+\theta}=(1+\theta)\phi^n-\theta \phi^{n-1}$\\
the value of $\theta$=THETVI(IPHAS) is deduced from the value chosen for
IVIEXT(IPHAS). Generally, only the value 0.5D0 is used. The user is not
allowed to modify this variable.\\
\hspace*{1.3cm}= 0.0D0: first-order (unused, corresponds to
IVIEXT(IPHAS)=0)\\
\hspace*{1.3cm}= 0.5D0: second-order (corresponds to IVIEXT(IPHAS)=1) \\
\hspace*{1.3cm}= 1.0D0: first-order (corresponds to IVIEXT(IPHAS)=2) \\
always useful}

\motcle{THETCP}{RA}{0.D0 $\leqslant$ real $\leqslant$1.D0}{0.0D0, 0.5D0 or 1.D0}{O}{L3}
{for each phase IPHAS, THETCP(IPHAS) is the value of $\theta$ used to
extrapolate the physical property $\phi$ ``specific heat'' when the extrapolation has
been activated (see ICPEXT),according to the
formula $\phi^{n+\theta}=(1+\theta)\phi^n-\theta \phi^{n-1}$\\
the value of $\theta$=THETCP(IPHAS) is deduced from the value chosen for
ICPEXT(IPHAS). Generally, only the value 0.5D0 is used. The user is not
allowed to modify this variable.\\
\hspace*{1.3cm}= 0.0D0: first-order (unused, corresponds to
ICPEXT(IPHAS)=0)\\
\hspace*{1.3cm}= 0.5D0: second-order (corresponds to ICPEXT(IPHAS)=1) \\
\hspace*{1.3cm}= 1.0D0: first-order (corresponds to ICPEXT(IPHAS)=2) \\
always useful}

\motcle{THETVS}{RA}{0.D0 $\leqslant$ real $\leqslant$1.D0}{0.0D0, 0.5D0 or 1.D0}{O}{L3}
{for each scalar ISCAL, THETVS(ISCAL) is the value of $\theta$ used to
extrapolate the physical property $\phi$ ``diffusivity'' when the extrapolation has
been activated (see IVSEXT),according to the
formula $\phi^{n+\theta}=(1+\theta)\phi^n-\theta \phi^{n-1}$\\
the value of $\theta$=THETVS(ISCAL) is deduced from the value chosen for
IVSEXT(ISCAL). Generally, only the value 0.5D0 is used. The user is not
allowed to modify this variable.\\
\hspace*{1.3cm}= 0.0D0: first-order (unused, corresponds to
IVSEXT(ISCAL)=0)\\
\hspace*{1.3cm}= 0.5D0: second-order (corresponds to IVSEXT(ISCAL)=1) \\
\hspace*{1.3cm}= 1.0D0: first-order (corresponds to IVSEXT(ISCAL)=2) \\
useful if NSCAL$>$1}


%==================================
\subsubsection{Gradient reconstruction}
%==================================

\motcle{IMRGRA}{I}{0, 1, 2, 3 or 4}{0}{O}{L2}
{indicates the type of gradient reconstruction (one method for all the
variables)\\
\hspace*{1.3cm}= 0: iterative reconstruction of the non-orthogonalities\\
\hspace*{1.3cm}= 1: least squares method based on the first neighbor cells
(cells which share a face with the treated cell)\\
\hspace*{1.3cm}= 2: least squares method based on the extended neighborhood
(cells which share a node with the treated cell)\\
\hspace*{1.3cm}= 3: least squares method based on a partial extended
neighborhood (all first neighbors plus the extended neighborhood cells that
are connected to a face where the non-orthogonality angle is larger than
parameter ANOMAX)\\
\hspace*{1.3cm}= 4: iterative reconstruction with initialisation using the least
squares method (first neighbors)\\
if IMRGRA fails due to probable mesh quality problems, it is usually effective
to use IMRGRA=3. Moreover, IMRGRA=3 is usually faster than IMRGRA=0 (but with
less feedback on its use).\\
it should be noted that IMRGRA=1, 2 or 3 automatically triggers a gradient
limitation procedure. See IMLIGR.\\
useful if and only if there is N so that NSWRGR(N) $>$ 1}

\motcle{NSWRGR}{IA}{positive integer}{100}{O}{L3}
{for each unknown IVAR, NSWRGR(IVAR) $\leqslant$ 1 indicates that the
gradients are not reconstructed\\
\hspace*{1.3cm}if IMRGRA = 0 or 4, NSWRGR(IVAR) is the number of
iterations for the gradient reconstruction\\
\hspace*{1.3cm}if IMRGRA = 1, 2 or 3, NSWRGR(IVAR) $>$ 1 indicates that
the gradients are reconstructed (but the method is not iterative, so any value
larger than 1 for NSWRGR yields the same result)\\
useful for all the unknowns}

\motcle{EPSRGR}{RA}{real number $>$ 0}{1.D-5}{O}{L3}
{for each unknown IVAR, relative precision for the iterative gradient
reconstruction: EPSRGR(IVAR)\\
useful for all the unknowns when IMRGRA = 0 or 4}

\motcle{IMLIGR}{IA}{-1, 0 or 1}{-1 or 1}{O}{L3}
{for each unknown IVAR, indicates the type of gradient limitation:
IMLIGR(IVAR)\\
\hspace*{1.3cm}=-1: no limitation\\
\hspace*{1.3cm}= 0: based on the neighbors\\
\hspace*{1.3cm}= 1: superior order\\
for all the unknowns, IMLIGR is initialised to -1 if IMRGRA=0 or 4 and to 1 if
\mbox{IMRGRA = 1, 2 or 3}\\
useful for all the unknowns}

\motcle{CLIMGR}{RA}{real number $>$ 0}{1.5D0}{O}{L3}
{for each unknown IVAR, factor of gradient limitation: CLIMGR(IVAR) (high
value means little limitation)\\
useful for all the unknowns IVAR for which IMLIGR(IVAR) $\ne$ -1}

\motcle{EXTRAG}{RA}{0.D0, 0.5D0 or 1.D0}{0.D0}{O}{L3}
{for the variable ``pressure'' IVAR=IPR(IPHAS), extrapolation coefficient
of the gradients at the boundaries. It affects only the Neumann conditions.
The only possible values of EXTRAG(IPR(IPHAS)) are:\\
\hspace*{1.3cm}= 0.D0: homogeneous Neumann calculated at first-order\\
\hspace*{1.3cm}= 0.5D0: improved homogeneous Neumann, calculated at
second-order in the case of an orthogonal mesh and at first-order otherwise\\
\hspace*{1.3cm}= 1.D0: gradient extrapolation (gradient at the boundary face
equal to the gradient in the neighbor cell), calculated at
second-order in the case of an orthogonal mesh and at first-order otherwise\\
EXTRAG often allows to correct the non-physical velocities that
appear on horizontal walls when density is variable and there is gravity.
It is strongly advised to keep EXTRAG=0 for the variables apart from
pressure. See also IPHYDR.\\
In practice, only the values 0.D0 and 1.D0 are allowed. The
value 0.5D0 isn't allowed by default (but the lock can be overridden if
necessary, contact the development team).\\
always useful}

\motcle{ANOMAX}{R}{0.D0 $\leqslant$ real $\leqslant\pi/2$}{$\pi/4$}{O}{L3}
{limit non-orthogonality angle used to restrict the extended neighborhood for
the gradient calculation with IMRGRA=3.\\
ANOMAX=0 will yield the same result as IMRGRA=2 (full extended
neighborhood). ANOMAX=$\pi/2$ will yield the same result as IMRGRA=2 (first
neighbors only)\footnote{except for pathological cases where the
non-orthogonality angle of a face would be larger than $\pi/2$}\\
useful if and only if IMRGRA=3}

%==================================
\subsubsection{Solution of the linear systems}
%==================================

\motcle{IRESOL}{IA}{-1, 1000*IPOL+J}{-1}{O}{L3}
{for each unknown IVAR, IRESOL(IVAR) indicates the method used for the solution of
the linear system\\
\hspace*{1.3cm}= -1: automatically managed by the code (conjugate
gradient for the pressure IVAR=IPR(IPHAS) or any variable which is not
convected, Jacobi for the others. Diagonal preconditioning with
conjugate gradient).\\
\hspace*{1.3cm}= IPOL*1000+J with           J= 0: conjugate gradient\\
\hspace*{1.3cm}\phantom{= IPOL*1000+J with} J= 1: Jacobi\\
\hspace*{1.3cm}\phantom{= IPOL*1000+J with} J= 2: stabilised bi-conjugate
gradient (BI-CGSTAB)\\
\hspace*{1.3cm}IPOL is the degree of the Neumann polynomial used for the
preconditioning\footnote{$D$ being the diagonal part of $A$ and $X$ its
extra-diagonal part, it can be written
$A=D(Id+D^{-1}X)$. Therefore \mbox{$A^{-1}=(Id+D^{-1}X)^{-1}D^{-1}$}. A series
development of $Id+D^{-1}X$ can then be used which yields, symbolically,
\mbox{$Id+\sum\limits_{I=1}^{IPOL}\left(-D^{-1}X\right)^{I}$}.}.\\
IPOL is necessarily null with the Jacobi algorithm.\\
Concerning the computational time, the performance depends on the
case. If a preconditioning method different from
the diagonal preconditioning is to be used, it seems to be better to restrict
to a first-order preconditioning (IPOL=1). This preconditioning may save up to 10\%
of time in some cases but in the others it may also increase the computational
time by a few percents\\
always useful}

\motcle{NITMAX}{IA}{integer $>$ 0}{10000}{O}{L3}
{for each unknown IVAR, maximum number of iterations for the solution of
the linear systems: NITMAX(IVAR)\\
when the algebraic multigrid option is activated for the variable IVAR
(IMGR(IVAR)=1), NITMAX(IVAR) is the maximum number of iterations for the
solution on the coarsest mesh\\
always useful}

\motcle{EPSILO}{RA}{real number $>$ 0}{1.D-8,1.D-5}{O}{L3}
{for each unknown IVAR, relative precision for the solution of the linear
system. The default value is EPSILO(IVAR)=1.D-8. This value is set low
on purpose. When there are enough iterations on the reconstruction of the
right-hand side of the equation,
the value may be increased (by default, in case of second-order in time,
with NSWRSM = 5 or 10, EPSILO is increased to 1.D-5).\\
always useful}

\motcle{IMGR}{IA}{0 or 1}{0}{O}{L3}
{for each unknown IVAR, indicates the use (IMGR(IVAR)=1) or not (=0) of the
algebraic multigrid method for the solution of the linear systems\\
IMGR(IVAR) can be set independently for every variable\\
always useful. Generally, its use is designed for the variable ``pressure'' in
case of meshes with strongly stretched cells. It is recommended not to modify
IMGR}

\motcle{NCEGRM}{I}{integer $>$ 0}{30}{O}{L3}
{for the multigrid method, maximum number of cells on the coarsest grid\\
useful if and only if IMGR(IVAR) = 1 for at least one variable IVAR}

\motcle{NCYMAX}{IA}{integer $>$ 0}{100}{O}{L3}
{for each unknown IVAR, NCYMAX(IVAR) is the maximum number of cycles when using
the multigrid method.\\
useful if and only if IMGR(IVAR) = 1}

\motcle{NGRMAX}{I}{1$\leqslant$ integer $\leqslant$NGRMMX }{NGRMMX}{O}{L3}
{when using the multigrid method, maximum number of grid levels\\
useful if and only if IMGR(IVAR) = 1 for at least one variable IVAR}

\motcle{NITMGF}{IA}{integer $>$ 0}{10}{O}{L3}
{for each unknown IVAR, NITMGF(IVAR) is the maximum number of iterations on the intermediary
grids when the multigrid method is used\\
useful if and only if IMGR(IVAR) = 1}

\minititre{Warning}
The algebraic multigrid method does not work when the mesh contains
subdivided faces (or more generally when two different faces have the
same pair of neighbors). In addition, it has been validated up to now only for
the variable ``pressure'' (IMGR(IPR(IPHAS))=1) and with non-parallel computations.

%==================================
\subsubsection{Convective scheme}
%==================================

\motcleb{BLENCV}{RA}{0 $\leqslant$ real $\leqslant$ 1}{0.D0 or 1.D0}{O}{L1}
{for each unknown IVAR to calculate, BLENCV(IVAR) indicates the proportion of
second-order convective scheme (0.D0 corresponds to an
``upwind'' first-order scheme) ; in case of LES calculation, a
second-order scheme is recommended and activated by default (BLENCV=1.D0)\\
useful for all the unknowns IVAR for which ICONV(IVAR) = 1}

\motcle{ISCHCV}{IA}{0 or 1}{1}{O}{L2}
{for each unknown IVAR to calculate, ISCHCV(IVAR) indicates the type of second-order
convective scheme\\
\hspace*{1.3cm}= 0: Second Order Linear Upwind\\
\hspace*{1.3cm}= 1: Centered\\
useful for all the unknowns IVAR which are convected (ICONV(IVAR)=1) and
for which a second-order scheme is used (BLENCV(IVAR) $>$ 0)}

\motcle{ISSTPC}{IA}{0 or 1}{0}{O}{L2}
{for each unknown IVAR to calculate, ISSTPC(IVAR)
indicates whether a ``slope test'' should
be used to switch from a second-order to an ``upwind'' convective
scheme under certain conditions, to ensure stability.\\
\hspace*{1.3cm}= 0: ``slope test'' activated for the considered unknown\\
\hspace*{1.3cm}= 1: ``slope test'' deactivated for the considered unknown\\
useful for all the unknowns IVAR  which are convected (ICONV(IVAR)=1) and
for which a second-order scheme is used (BLENCV(IVAR) $>$ 0).\\
the use of the ``slope test'' stabilises the calculation but may bring
the order in space to decrease quickly.}

%==================================
\subsubsection{Pressure-continuity step}
%==================================


\motcle{IPRCO}{I}{0 or 1}{1}{O}{L3}
{indicates if the pressure-continuity step is taken into account (1) or
not (0)\\
always useful}

\motcle{ARAK}{RA}{0 $<$ real $\leqslant$ 1}{1.D0}{O}{L3}
{for each phase IPHAS, ARAK(IPHAS) is the Arakawa coefficient before the
Rhie\& Chow filter\\
always useful}

\motcle{RELAXP}{RA}{0 $<$ real $\leqslant$ 1}{1.D0}{O}{L2}
{for each phase IPHAS, relaxation of the pressure increment during the
solution of the system (RELAXP(IPHAS)=1: no relaxation)\\
can improve the convergence in case of
meshes of insufficient quality\\
always useful}

\motcle{IREVMC}{IA}{0, 1 or 2}{0}{O}{L3}
{for each phase IPHAS, method used to update the velocity after the pressure
correction:\\
\hspace*{0,5cm}- standard gradient of pressure increment
(IREVMC(IPHAS)=0)\\
\hspace*{0,5cm}- least squares on the pressure increment
(IREVMC(IPHAS)=1)\\
\hspace*{0,5cm}-``RT0'' {\em i.e.} least squares on the updated mass flux
(IREVMC(IPHAS)=2)\\
the method IREVMC(IPHAS)=2 is generally not recommended\\
always useful}

\motcle{IPHYDR}{I}{0 or 1}{0}{O}{L2}
{method for taking into account the balance between the pressure gradient and
the source terms (gravity and head losses): by extension it will be
referenced as ``taking into account of the hydrostatic pressure''\\
\hspace*{1.3cm}= 0: standard algorithm\\
\hspace*{1.3cm}= 1: improved algorithm\\
always useful\\
When the density effects are important, the choice of
IPHYDR=1 allows to improve the interpolation of the pressure and correct the
non-physical velocities which may appear in highly
stratified areas or near horizontal walls (thus
avoiding the use of EXTRAG if the non-physical velocities are due only to
gravity effects).\\
The improved algorithm also allows to eradicate the velocity oscillations
which tend to appear at the frontiers of areas with high head losses.\\
In the case of a stratified flow, the calculation cost is higher when the
improved algorithm is used (about 30\% depending on the case) because
the hydrostatic pressure has to be recalculated at the outlet boundary
conditions: see ICALHY.\\
On meshes of insufficient quality, in order to
improve the convergence, it may be useful to increase the number of
iterations for the reconstruction of the pressure right-hand member,
{\em i.e.} \mbox{NSWRSM(IPR(IPHAS))}.\\
If head losses are present just along an outlet boundary, it is necessary to
specify ICALHY=0 in order to deactivate the recalculation of the hydrostatic
pressure at the boundary, which may otherwise cause instabilities.}

\motcle{ICALHY}{I}{0 or 1}{0 or 1}{O}{L3}
{activates the calculation of hydrostatic pressure boundary conditions at outlet
boundaries\\
\hspace*{1.3cm}= 0: no calculation of the hydrostatic pressure at the outlet boundary\\
\hspace*{1.3cm}= 1: calculation of the hydrostatic pressure at the outlet boundary\\
always useful\\
This option is automatically specified depending on the
choice of IPHYDR and the value of gravity
(ICALHY=1 if IPHYDR=1 and gravity is different from 0; otherwise
ICALHY=0). The activation of this option generates an additional
calculation cost (about 30\% depending on the case).\\
If head losses are present just along an outlet boundary, it is necessary to
specify ICALHY=0 in order to deactivate the recalculation of the hydrostatic
pressure at the boundary, which may otherwise cause instabilities}



%==================================
\subsubsection{Error estimators for Navier-Stokes}
%==================================

There are currently NESTMX\index{NESTMX}=4 types of local estimators
provided at every time step, with two possible definitions for
each\footnote{choice made by the user}. These scalars indicate the areas
(cells) in which some error types may be important. They are
stored in the array PROPCE containing the properties at the cells (see
IESTIM\index{IESTIM}). For each estimator, the code writes the minimum and
maximum values in the listing and generates post-processing outputs along with
the other variables.

The additional memory cost is about one real number per cell and per
estimator. The additional calculation cost is variable. For instance, on a
simple test case, the total estimator IESTOT generates an additional cost
of 15 to 20 $\%$ on the CPU time\footnote{indeed, all the first-order in
space differential terms have to be recalculated at the time $t^{\,n+1}$} ;
the cost of the three others may be neglected. If the user wants to
avoid the calculation of the estimators during the computation, it is
possible to run a calculation without estimators first, and then activate them on
a restart of one or two time steps.

It is recommended to use the estimators only for visual and qualitative
analysis. Also, their use is compatible neither with a second-order time scheme
nor with a calculation with a frozen velocity field.

{\bf IEST = IESPRE\index{IESPRE}: prediction} (default name: EsPre).
After the velocity prediction step (yielding $\vect{u}^*$), the
estimator $\eta^{\,pred}_{\,i,k}(\vect{u}^*)$, local variable calculated
at every cell $\Omega_i$, is created from $\vect{\mathcal
R}^{\,pred}(\vect{u}^*)$, which represents the residual of the equation
solved during this step:
\begin{eqnarray*}
\vect{\mathcal R}^{\,pred}(\vect{u}^*)&= & \rho^n \frac{\vect{u}^*-\vect{u}^n}{\Delta t}
              + \rho^n \vect{u}^n \cdot \tens{grad}(\vect{u}^*)
              - div \left((\mu+\mu_t)^n \tens{grad}(\vect{u}^*) \right)
              + \grad(P^n)     \\
              &- &\text{rest of the right-hand member}
                        \,(\vect{u}^n, P^n, \text{other variables}^n)
\end{eqnarray*}
By definition:
$$ \eta^{\,pred}_{\,i,k}(\vect{u}^*)= {|\Omega_i|}^{\,(k-2)/2}\ ||\vect{\mathcal R}^{\,pred}(\vect{u}^*)||
_{{I\hspace{-.25em}L}^{2}(\Omega_i)}$$
\hspace*{0.5cm}$\bullet$ The first family, $k=1$, suppresses the
volume $|\Omega_i|$ which intrinsicly appears  with the norm
${{I\hspace{-.25em}L}^{2}(\Omega_i)}$.\\
\hspace*{0.5cm}$\bullet$ The second family, $k=2$, exactly represents the norm
${{I\hspace{-.25em}L}^{2}(\Omega_i)}$. The size of the cell therefore
appears in its calculation and induces a weighting effect.\\
$ \eta^{\,pred}_{\,i,k}(\vect{u}^*)$  is ideally equal to zero when the
reconstruction methods are perfect and the associated system is
solved exactly.

{\bf IEST = IESDER\index{IESDER}: drift}  (default name: EsDer).
The estimator $\eta^{\,der}_{\,i,k}(\vect{u}^{\,n+1})$ is based on the
following quantity (intrinsic to the code):
\begin{equation}
\begin{array}{lll}
 \eta^{\,der}_{\,i,k}(\vect{u}^{\,n+1})
&=& {|\Omega_i|}^{\,(k-2)/2}
||div (\text{corrected mass flow after the pressure step})
                                              -\ \Gamma||_{{L}^{2}(\Omega_i)} \\
&=& {|\Omega_i|}^{\,(1-k)/2}
|div (\text{corrected mass flow after the pressure step})-\ \Gamma|
\end{array}
\end{equation}
Ideally, it is equal to zero when the Poisson equation related to the pressure is
solved exactly.

{\bf IEST = IESCOR\index{IESCOR}: correction}  (default name: EsCor).
The estimator $ \eta^{\,corr}_{\,i,k}(\vect{u}^{\,n+1})$ comes directly
from the mass flow calculated with the updated velocity field:
\begin{eqnarray*}
            \eta^{\,corr}_{\,i,k}(\vect{u}^{\,n+1})=
|\Omega_i|^{\,\delta_{\,2,k}}\ |div (\rho^n \vect{u}^{n+1}) -\ \Gamma|
\end{eqnarray*}
The velocities $\vect{u}^{n+1}$ are taken at the cell centers,
the divergence is calculated after projection on the faces.\\
            $ \,\delta_{\,2,k}$ represents the Kronecker symbol.\\
\hspace*{0.5cm}$\bullet$ The first family, $k=1$, is the absolute raw
value of the divergence of the mass flow minus the mass source term.\\
\hspace*{0.5cm}$\bullet$ The second family, $k=2$, represents a physical
property and allows to evaluate the difference in $kg.s^{\,-1}$.\\
Ideally, it is equal to zero when the Poisson equation is solved exactly and
the projection from the mass flux at the faces to the velocity at the cell
centers is made in a set of  functions with null divergence.

{\bf IEST = IESTOT\index{IESTOT}: total} (default name: EsTot).
The estimator $ \eta^{\,tot}_{\,i,k}(\vect{u}^{\,n+1})$, local variable
calculated at every cell $\Omega_i$, is based on the quantity
$\vect{\mathcal R}^{\,tot}(\vect{u}^{\,n+1})$, which represents the
residual of the equation using the updated values of
$\vect{u}$ and $P$:
\begin{eqnarray*}
\vect{\mathcal R}^{\,tot}(\vect{u}^{\,n+1})&= & \rho^n \frac{\vect{u}^{\,n+1}-\vect{u}^n}{\Delta t}
              + \rho^n \vect{u}^{\,n+1} \cdot \tens{grad}(\vect{u}^{\,n+1})
              - div \left((\mu+\mu_t)^n \tens{grad}(\vect{u}^{\,n+1}) \right)
              + \grad(P^{\,n+1})     \\
              &- &\text{rest of the right-hand member}
                        \,(\vect{u}^{\,n+1}, P^{\,n+1}, \text{other variables}^n)
\end{eqnarray*}
By definition:
$$ \eta^{\,tot}_{\,i,k}(\vect{u}^{\,n+1})= {|\Omega_i|}^{\,(k-2)/2}\ ||\vect{\mathcal R}^{\,tot}(\vect{u}^{\,n+1})||
_{{I\hspace{-.25em}L}^{2}(\Omega_i)}$$

The mass flux in the convective term is recalculated from $\vect{u}^{n+1}$
expressed at the cell centers (and not taken from the updated mass flow at the
faces).\\

As for the prediction estimator:\\
\hspace*{0.5cm}$\bullet$ The first family, $k=1$, suppresses the
volume $|\Omega_i|$ which intrinsicly appears  with the norm
${{I\hspace{-.25em}L}^{2}(\Omega_i)}$.\\
\hspace*{0.5cm}$\bullet$ The second family, $k=2$, exactly represents the norm
${{I\hspace{-.25em}L}^{2}(\Omega_i)}$. The size of the cell therefore
appears in its calculation and induces a weighting effect.\\


The estimators are evaluated depending on the values of IESCAL.


\motcleb{IESCAL}{IA}{0, 1 or 2}{0}{O}{L1}
{for each phase IPHAS, IESCAL(IEST,IPHAS) indicates the calculation mode
for the error estimator IEST (IESPRE, IESDER, IESCOR or IESTOT), for
the Navier-Stokes equation:\\
 IESCAL = 0: estimator not calculated, \\
 IESCAL = 1: the estimator $ \eta^{\,* }_{\,i,1}$ is calculated,
               without contribution of the volume, \\
 IESCAL = 2: the estimator $ \eta^{\,* }_{\,i,2}$ is calculated,
               with contribution of the volume ("norm $L^2$"),
               except for IESCOR, for which
               $|\Omega_i|\ \eta^{\,corr}_{\,i,1}\ $
               is calculated.

The name of the estimators appearing in the listing and the post-processing is
made up of the default name (given before), followed first by the value of
IESCAL, then by the phase number. For
instance, EsPre201 is the estimator IESPRE calculated with IESCAL=2 for
the phase 01.\\
always useful}



%==================================
\subsubsection{Calculation of the distance to the wall}
%==================================

\motcle{ICDPAR}{I}{-1, 1, -2 or 2}{-1}{O}{L2}
{specifies the method used to calculate the distance to the wall $y$ and the
adimensional distance $y^+$ for all the
cells of the calculation domain (when necessary):\\
\hspace*{1.3cm}= 1: standard algorithm (based on a Poisson equation for $y$ and
convection equation for $y^+$),
with reading of the distance to the wall from the restart file
if possible\\
\hspace*{1.3cm}=-1: standard algorithm (based on a Poisson equation for $y$ and
convection equation for $y^+$),
with systematic recalculation of the distance to the wall in case of
calculation restart\\
\hspace*{1.3cm}= 2: former algorithm (based on geometrical
considerations),
with reading of the distance to the wall from the restart file
if possible\\
\hspace*{1.3cm}=-2: former algorithm (based on geometrical
considerations) with systematic recalculation of the distance to the
wall in case of calculation restart\\
In case of restart calculation, if the position of the walls haven't changed,
reading the distance to the wall from the restart file can save a fair amount of
CPU time.\\
Useful in $R_{ij}-\varepsilon$ model with wall echo (ITURB(IPHAS)=30 and IRIJEC=1),
in LES with van Driest damping (ITURB(IPHAS)=40 and IDRIES(IPHAS)=1) and
in $k-\omega$ SST (ITURB(IPHAS)=60). \\
By default, ICDPAR is initialied to -1, in case there has been a change in the
definition of the boundary conditions between two computations (change in the
number or the positions of the walls). Yet, with the $k-\omega$ SST model, the
distance to the wall is needed to calculate the turbulent viscosity, which is
done before the calculation of the distance to the wall. Hence, when this model
is used (and only in that case), ICDPAR is set to 1 by default, to ensure total
continuity of the calculation at restart.\\
{\bf As a consequence, with the \boldmath$k-\omega$\unboldmath\ SST model, if
the number and positions of the walls are changed at a calculation restart, it
is mandatory for the user to set ICDPAR explicitly to -1}, otherwise the
distance to the wall used will not correspond to the actual position of the
walls.\\
The former algorithm is not compatible with parallelism nor periodicity. Also,
whatever the value chosen for ICDPAR, the calculation of the distance to the
wall is made at the most once for all a the beginning of the calculation. It is
therefore not compatible with moving walls. Please contact the development team
if you need to override this limitation.}


The following options are related to ICDPAR=1 or -1. The options of
level 2 are described first. Some options are used only in the case of
the calculation of the adimensional distance to the wall $y^+$ (LES model with
van Driest damping). Most of these key words are simple copies of the
key words for the numerical options of the general equations, with a potentially
specific value in the case of the calculation of the distance to the wall.\\

\motcle{IWARNY}{I}{integer}{0}{O}{L2}
{specifies the level of the output writing concerning the calculation of the
distance to the wall with ICDPAR=1 or -1. The higher the value, the more
detailled the outputs\\
useful when ICDPAR=1 or -1}

\motcle{NTCMXY}{I}{positive integer}{1000}{O}{L2}
{number of pseudo-time iterations for the calculation of the adimensional
distance to the wall $y^+$\\
useful when ICDPAR=1 or -1 for the calculation of $y^+$}

\motcle{NITMAY}{I}{integer  $>$ 0}{10000}{O}{L3}
{maximum number of iterations for the solution of the linear systems\\
useful when ICDPAR=1 or -1}

\motcle{NSWRSY}{I}{positive integer}{1}{O}{L3}
{number of iterations for the reconstruction of the right-hand members:
corresponds to NSWRSM\\
useful when ICDPAR=1 or -1}

\motcle{NSWRGY}{I}{positive integer}{100}{O}{L3}
{number of iterations for the gradient reconstruction: corresponds to NSWRGR\\
useful when ICDPAR=1 or -1}

\motcle{IMLIGY}{I}{-1, 0 ou 1}{-1 or 1}{O}{L3}
{type of gradient limitation: corresponds to IMLIGR\\
useful when ICDPAR=1 or -1}

\motcle{IRCFLY}{I}{0 or 1}{1}{O}{L3}
{indicates the reconstruction of the convective and diffusive fluxes at
the faces: corresponds to IRCFLU\\
useful when ICDPAR=1 or -1}

\motcle{ISCHCY}{I}{0 or 1}{1}{O}{L3}
{type of second-order convective scheme: corresponds to ISCHCV\\
useful when ICDPAR=1 or -1 for the calculation of $y^+$}

\motcle{ISSTPY}{I}{0 or 1}{0}{O}{L3}
{indicates if a ``slope test'' should be used for a second-order convective
scheme: corresponds to ISSTPC\\
useful when ICDPAR=1 or -1 for the calculation of $y^+$}

\motcle{IMGRPY}{I}{0 or 1}{0}{O}{L3}
{indicates whether the algebraic
multigrid method should be used (IMGR(IVAR)=1) or not (0): corresponds to IMGR\\
useful when ICDPAR=1 or -1}

\motcle{BLENCY}{R}{0 $\leqslant$ real $\leqslant$ 1}{0.D0}{O}{L3}
{proportion of second-order convective scheme: corresponds to BLENCV\\
useful when ICDPAR=1 or -1 for the calculation of $y^+$}

\motcle{EPSILY}{R}{real number $>$ 0}{1.D-8}{O}{L3}
{relative precision for the solution of the linear systems: corresponds to EPSILO\\
useful when ICDPAR=1 or -1}

\motcle{EPSRGY}{R}{real number $>$ 0}{1.D-5}{O}{L3}
{relative precision for the iterative gradient reconstruction: corresponds to EPSRGR\\
useful when ICDPAR=1 or -1}

\motcle{CLIMGY}{R}{real number $>$ 0}{1.5D0}{O}{L3}
{limitation factor of the gradients: corresponds to CLIMGR\\
useful when ICDPAR=1 or -1}

\motcle{EXTRAY}{R}{0.D0, 0.5D0 or 1.D0}{0.D0}{O}{L3}
{extrapolation coefficient of the gradients at the boundaries: corresponds to EXTRAG\\
useful when ICDPAR=1 or -1}

\motcle{COUMXY}{R}{strictly positive real number}{5000.D0}{O}{L3}
{Target Courant number for the calculation of the adimensional distance
to the wall\\
useful when ICDPAR=1 or -1 for the calculation of $y^+$}

\motcle{EPSCVY}{R}{strictly positive real number}{1.D-8}{O}{L3}
{relative precision for the convergence of the pseudo-transient regime
for the calculation of the adimensional distance to the wall\\
useful when ICDPAR=1 or -1  for the calculation of $y^+$}

\motcle{YPLMXY}{R}{real number}{200.D0}{O}{L3}
{value of the adimensional distance to the wall above which the
calculation of the distance is not necessary (for the damping)\\
useful when ICDPAR=1 or -1 for the calculation of $y^+$}


%==================================
\subsubsection{Others}
%==================================

\motcleb{ICCVFG}{I}{0 or 1}{0}{O}{L1}
{indicates whether the dynamic field should be frozen (1) or not (0)\\
in such a case, the values of velocity,
pressure and the variables related to the potential turbulence model
($k$, $R_{ij}$, $\varepsilon$, $\varphi$, $\bar{f}$, $\omega$, turbulent viscosity) are kept
constant over time and only the equations for the scalars are solved\\
also, if ICCVFG=1, the physical properties modified in \texttt{usphyv} will keep
being updated. Beware of non-consistencies if these properties would normally
affect the dynamic field (modification of density for instance)\\
useful if and only if NSCAL $>$ 0 and ISUITE=1}

\motcleb{IPUCOU}{I}{0 or 1}{0}{O}{L1}
{indicates the algorithm for velocity/pressure coupling\\
\hspace*{1.3cm}= 0: standard algorithm\\
\hspace*{1.3cm}= 1: reinforced coupling in case calculation with long
time steps\\
always useful (it is seldom advised, but it can prove very useful, for instance,
in case of flows with weak convection effects and highly variable viscosity)}

\motcleb{ISUIT1}{I}{0 or 1}{0}{O}{L1}
{for the  1D wall thermal module, activation (1) or not(0) of the reading
of the mesh and of the wall temperature from the FICMT1 restart file\\
useful if NFPT1D$>$0.}

\motcle{IMVISF}{I}{0 or 1}{0}{O}{L3}
{indicates the interpolation method used to project variables from the cell
centers to the faces\\
\hspace*{1.3cm}= 0: linear\\
\hspace*{1.3cm}= 1: harmonic\\
always useful}

\motcle{IRCFLU}{IA}{0 or 1}{1}{O}{L2}
{for each unknown IVAR, IRCFLU(IVAR) indicates whether the convective
and diffusive fluxes at the faces should be reconstructed: \\
\hspace*{1.3cm}= 0: no reconstruction\\
\hspace*{1.3cm}= 1: reconstruction\\
deactivating the reconstruction of the fluxes can have a stabilising effect on
the calculation. It is sometimes useful with the $k-\varepsilon$ model, if the
mesh is strongly non-orthogonal in the near-wall region, where the gradients of
$k$ and $\varepsilon$ are strong. In such a case, setting IRCFLU(IK(IPHAS))=0
and IRCFLU(IEP(IPHAS))=0 will probably help (switching to a first order
convective scheme, BLENCV=0.D0, for $k$ and $\varepsilon$ might also help in
that case)\\
always useful}

\motcle{NSWRSM}{IA}{positive integer}{1, 2, 5 or 10}{O}{L3}
{for each unknown IVAR, NSWRSM(IVAR) indicates the number of iterations for the
reconstruction of the right-hand members of the equations\\
with a first-order scheme in time
(standard case), the default values are 2 for pressure and 1 for the
other variables. With a second-order scheme in time (ISCHTP=2) or LES, the
default values are 5 for pressure and 10 for the other variables.\\
useful for all the unknowns}


%====================================================================================
\subsection{Numerical, physical and modeling parameters}
%=============================================================================
\subsubsection{Numeric Parameters}
%================================

These parameters correspond to numeric reference values in the code.
They can be used but shall not be modified (they are defined as PARAMETER).

\motcle{ZERO}{R}{0.D0}{O.D0}{O}{L3}
{Parameter containing the value 0}

\motcle{EPZERO}{R}{1.D-12}{1.D-12}{O}{L3}
{``Small'' real parameter, used for the comparisons of real numbers (absolute
value of the difference lower than EPZERO)}

\motcle{PI}{R}{3.141592653589793D0}{3.141592653589793D0}{O}{L3}
{Parameter containing the value of $\pi$}

\motcle{GRAND}{R}{1.D12}{1.D12}{O}{L3}
{``Large'' real parameter, generally used by default as a non physical value for
the initialisations of variables which have to be modified by the user}

\motcle{RINFIN}{RR}{1.D30}{1.D30}{O}{L3}
{Real parameter used to represent the ``infinite''}

%==================================
\subsubsection{Physical parameters}
%==================================
These parameters correspond to physical reference values in the code.
They can be used but shall not be modified (they are defined as PARAMETER).

\motcle{TKELVI}{R}{273.15D0}{273.15D0}{O}{L3}
{Temperature in Kelvin correponding to 0 degrees Celsius.}

\motcle{TKELVI}{R}{-273.15D0}{-273.15D0}{O}{L3}
{Temperature in degrees Celsius corresponding to 0 Kelvin.}

\motcle{RR}{R}{8.31434D0}{8.31434D0}{O}{L3}
{Perfect gas constant in $J/mol/K$}

\motcle{TREFTH}{R}{25.D0 + TKELVI}{25.D0 + TKELVI}{O}{L3}
{Reference temperature for the specific physics, in $K$}

\motcle{PREFTH}{R}{1.01325D5}{1.01325D5}{O}{L3}
{Reference pressure for the specific physics, in $Pa$}

\motcle{VOLMOL}{R}{22.41D-3}{22.41D-3}{O}{L3}
{Molar volume under normal pressure and temperature conditions (1 atmosphere,
0\degresC) in $m^{-3}$}

\motcle{STEPHN}{R}{5.6703D-8}{5.6703D-8}{O}{L3}
{Stephan constant for the radiative module $\sigma$ in $W.m^{-2}.K^{-4}$}

\motcle{PERMVI}{R}{1.2566D-6}{1.2566D-6}{O}{L3}
{Vacuum magnetic permeability $\mu_0$ (=$4\pi.10^{-7}$) in $kg.m.A^{-2}.s^{-2}$}

\motcle{EPSZER}{R}{8.854D-12}{8.854D-12}{O}{L3}
{Vacuum permittivity $\varepsilon_0$ in $F.m^{-1}$}



%==================================
\subsubsection{Physical variables}
%==================================

\motcleb{GX,GY,GZ}{R}{3 real numbers}{0.D0,0.D0,0.D0}{O}{L1}
{gravity components\\
always useful }

\motcleb{IROVAR}{IA}{0 or 1}{\tt -1}{C}{L1}
{for each phase IPHAS, IROVAR(IPHAS)=0 indicates that the density is
constant. Its value is the reference density RO0(IPHAS).\\
IROVAR(IPHAS)=1 indicates that the density is variable: its variation
law must be given in the user subroutine \texttt{usphyv}\\
negative value: not initialised\\
always useful}

\motcleb{IVIVAR}{IA}{0 or 1}{\tt -1}{C}{L1}
{for each phase IPHAS, IVIVAR(IPHAS)=0 indicates that the molecular
dynamic viscosity is constant. Its value is the reference molecular
dynamic viscosity VISCL0(IPHAS).\\
IVIVAR(IPHAS)=1 indicates that the molecular dynamic viscosity is
variable: its variation law must be given in the user subroutine
\texttt{usphyv}\\
negative value: not initialised\\
always useful}

\motcleb{RO0}{RA}{real number $\geqslant$ 0}{\tt -GRAND*10}{C}{L1}
{for each phase IPHAS, RO0(IPHAS) is the reference density\\
negative value: not initialised\\
its value is not used in gas or coal combustion modeling (it
will be calculated following the perfect gas law, with P0 and T0). With the
compressible module, it is also not used by the code, but it may be (and often
is) referenced by the user in user subroutines; it is therefore better to
specify its value.\\
always useful otherwise, even if a law defining the density is given by
the user subroutine \texttt{usphyv} or \texttt{uselph}\\
indeed, except with the
compressible module, \CS\ does not
use the total pressure $P$ when solving the Navier-Stokes equation, but a
reduced pressure \\
$P^*=P-\rho_0\vect{g}.(\vect{x}-\vect{x}_0)+P^*_0-P_0$\\
where
$\vect{x_0}$ is a reference point (see XYZP0) and $P^*_0$ and $P_0$ are
reference values (see PRED0 and P0). Hence, the term
$-\grad{P}+\rho\vect{g}$ in the equation is treated as
$-\grad{P^*}+(\rho-\rho_0)\vect{g}$. The closer RO0 is to the value of $\rho$,
the more $P^*$ will tend to represent only the dynamic part of the pressure and
the faster and more precise its solution will be. Whatever the value of RO0,
both $P$ and $P^*$ appear in the listing and the post-processing outputs.\\
with the compressible module, the calculation is made directly on the total
pressure}

\motcleb{VISCL0}{RA}{real number $>$ 0}{\tt -GRAND*00}{C}{L1}
{for each phase IPHAS,  VISCL0(IPHAS) is the reference molecular dynamic
viscosity\\
negative value: not initialised\\
always useful, it is the used value unless the user specifies the
viscosity in the subroutine \texttt{usphyv}}

\motcleb{SRROM}{R}{$0 \leqslant \text{r\'eel} < 1$}{-GRAND ou 0}{C or O}{L1}
{With gas combustion, pulversied coal or the electric module, SRROM
 is the sub-relaxation coefficient for the density, following the formula:\\
$\rho^{n+1}$\,=\,SRROM\,$\rho^n$+(1-SRROM)\,$\rho^{n+1}$\\
hence, with a zero value, there is no sub-relaxation.
With combustion and pulversied coal, SRROM is initialised to -GRAND
and the user must specify a proper value through the Interface or the
initialisation subroutines (\texttt{usd3p1},
 \texttt{usebu1}, \texttt{uslwc1}, \texttt{uscpi1} or
\texttt{uscpl1}).With the electric module, SRROM is initialised in to 0
and may be modified by the user in \texttt{useli1}.\\
With gas combustion, pulverised coal or electric arc, SSROM is
automatically used after the second time-step. With Joule effect,
the user decides whether or not it will be used in \texttt{uselph}
from the coding law giving the density.}\\
always useful with gas combustion, pulversized coal or the electric module.

\motcleb{P0}{RA}{real number}{1.013D5}{O}{L1}
{for each phase IPHAS, P0(IPHAS) is the reference pressure for the total
pressure\\
except with the compressible module, the total pressure $P$ is evaluated
from the reduced pressure $P^*$ so that $P$
is equal to P0 at the reference position $\vect{x}_0$ (given by XYZP0)\\
with the compressible module, the total pressure is solved directly\\
always useful}

\motcle{PRED0}{RA}{real number}{0.D0}{O}{L3}
{for each phase IPHAS, PRED0(IPHAS) is the reference value for the reduced
pressure $P^*$ (see RO0)\\
it is especially used to initialise the reduced pressure and as a reference
value for the outlet boundary conditions\\
for an optimised precision in the resolution of $P^*$, it is wiser to keep PRED0
to 0\\
with the compressible module, the ``pressure'' variable appearing in the
equations directly represents the total pressure. It is therefore initialised
to P0 and not PRED0 (see RO0)\\
always useful, except with the compressible module}

\motcleb{XYZP0}{RA}{3 real numbers}{0.D0,0.D0,0.D0}{O}{L1}
{for each phase IPHAS, XYZP0(II,IPHAS) is the II coordinate
(1$\leqslant$II$\leqslant$3) of the reference point $\vect{x}_0$ for the
total pressure\\
when there are no Dirichlet conditions for the pressure (closed domain), XYZP0
does not need to be specified (unless the total pressure has a clear physical
meaning in the configuration treated)\\
when Dirichlet conditions on the pressure are specified but only through stantard
outlet conditions (as it is in most configurations),
XYZP0 does not need to be specified by the user, since it will be set to the
coordinates of the reference outlet face ({\em i.e.} the code will automatically
select a
reference outlet boundary face and set XYZP0 so that $P$ equals P0 at this
face). Nontheless, if XYZP0 is pecified by the user, the calculation will remain
correct\\
when direct Dirichlet conditions are specified by the user (specific value set
on specific boundary faces), it is better to specify the corresponding reference
point ({\em i.e.} specifiy where the total pressure is P0). This way, the
boundary conditions for the reduced pressure will be close to PRED0, ensuring an
optimal precision in the resolution. If XYZP0 is not specified, the reduced
pressure will be shifted, but the calculations will remain correct.\\
with the compressible module, the ``pressure'' variable appearing in the
equations directly represents the total pressure. XYZP0 is therefore not used.\\
always useful, except with the compressible module}

\motcleb{T0}{RA}{real number}{0.D0}{O}{L1}
{for each phase IPHAS, T0(IPHAS) is the reference temperature \\
useful for the specific physics gas or coal combustion (initialisation
of the density), for the electricity modules to initialise the domain
temperature and for the comperssible module (initialisations). It must be given
in Kelvin.}

\motcleb{CP0}{RA}{real number $>$ 0}{\tt -GRAND*10}{O}{L1}
{for each phase IPHAS, CP0(IPHAS) is the reference specific heat\\
useful if there is 1$\leqslant$N$\leqslant$NSCAUS\footnote{none of the scalars
from the specific physics is a temperature} so that ISCSTH(N)=1
(there is a scalar ``temperature''), unless the user specifies the
specific heat in the user subroutine \texttt{usphyv}\footnote{when using the
Graphical Interface, CP0 is also used to calculate the diffusivity of the
thermal scalars, based on their conductivity; it is therefore needed, unless the
diffusivity is also specified in \texttt{usphyv}} (ICP(IPHAS) $>$ 0)\\
with the compressible module or coal combustion, CP0 is also needed even when
there is no user scalar}

\motcleb{ICP}{IA}{0 or 1}{0}{O}{L1}
{for each phase IPHAS, indicates if the specific heat $C_p$ is variable
(ICP(IPHAS)=1) or not (0)\\
When gas or coal combustion is activated, ICP is automatically set to 0
(constant $C_p$). With the electric module, it is automatically set to 1.
The user is not allowed to modify these default choices.\\
When ICP(IPHAS)=1 is specified, the code automatically modifies this value to
make ICP(IPHAS) designate the effective index-number of the property
``specific heat of the phase IPHAS''. For each cell IEL, the value of
$C_p$ is then specified by the user in the appropriate subroutine
(\texttt{usphyv} for the standard physics) and stored in the array\\
PROPCE(IEL,IPPROC(ICP(IPHAS)))
({\em see p.\pageref{prg_propvar} for specific conditions of use})\\
useful if there is 1$\leqslant$N$\leqslant$NSCAL so that ISCSTH(N)=1
(there is a scalar ``temperature'') or with the compressible module for non
perfect gases}

\motcleb{VISLS0}{RA}{real number $>$ 0}{\tt -GRAND*10}{C}{L1}
{VISLS0(J): reference molecular diffusivity related to the scalar J
($kg.m^{-1}.s^{-1}$)\\
negative value: not initialised\\
useful if 1$\leqslant$J$\leqslant$ NSCAL, unless the user specifies the
molecular diffusivity in the appropriate user subroutine (\texttt{usphyv} for
the standard physics) (IVISLS(ISCAL) $>$ 0)\\
{\em Warning : VISLS0 corresponds to the diffusivity. For the temperature, it is
therefore defined as $\lambda/C_p$ where $\lambda$ and $C_p$ are the
conductivity and specific heat. When using the Graphical Inteface, $\lambda$ and
$C_p$ are specified separately, and VISLS0 is calculated automatically\\
With the compressible module, VISLS0 (given in \texttt{uscfxi2}) is directly the
thermal conductivity $W.m^{-1}.K^{-1}$\\
With gas or coal combustion, the molecular diffusivity of the enthalpy
($kg.m^{-1}.s^{-1}$) must be specified by the user in the variable DIFTL0 (\texttt{usebu1},
\texttt{usd3p1}, \texttt{uslwc1}, \texttt{uscpi1}, \texttt{uscpl1})\\
With the electric module, for the Joule effect, the diffusivity is specified by
the user in \texttt{uselph} (even if it is constant). For the electric arc, it
is calculated from the thermochemical data file}}

\motcleb{IVISLS}{IA}{positive or zero integer}{0}{O}{L1}
{indicates if the viscosity related to the scalar ISCAL is variable
(IVISLS(ISCAL)=1) or not (0). The user must specify IVISLS only for the
user scalars (ISCAL $\leqslant$ NSCAUS).\\
When IVISLS(ISCAL)=1 is specified, the code automatically modifies this value to
make IVISLS(ISCAL) designate the effective index-number of the property
``diffusivity of the scalar ISCAL''. For each cell IEL, the value
is then specified by the user in the appropriate subroutine
(\texttt{usphyv} for the standard physics) and stored in the array\\
PROPCE(IEL,IPPROC(IVISLS(IPHAS)))
({\em see p.\pageref{prg_propvar} for specific conditions of use})\\
useful if 1$\leqslant$N$\leqslant$NSCAL}


\motcleb{DIFTL0}{R}{real number $>$ 0}{\tt -GRAND}{C}{L1}
{molecular diffusivity for the enthalpy ($kg.m^{-1}.s^{-1}$) for gas or coal
combustion (the code then automatically sets VISLS0 to DIFTL0 for the scalar
representing the enthalpy)\\
always useflu for gas or coal combustion}

\motcleb{SCAMIN}{RA}{real number}{\tt GRAND}{O}{L1}
{SCAMIN(ISCAL) is the lower limit value for the scalar ISCAL. At each time step,
in every cell where the calculated value for RTP(IEL,ISCA(ISCAL)) is lower than
\mbox{SCAMIN(ISCAL)}, RTP(IEL,ISCA(ISCAL)) will be reset to
\mbox{SCAMIN(ISCAL)}\\
there is no limitation if SCAMIN(ISCAL)$>$SCAMAX(ISCAL)\\
SCAMIN shall not be specified for non-user scalars (specific physics) or for
scalar variances\\
useful if and only if 1$\leqslant$ISCAL$\leqslant$ NSCAUS}

\motcleb{SCAMAX}{RA}{real number}{\tt -GRAND}{O}{L1}
{SCAMAX(ISCAL) is the higher limit value for the scalar ISCAL. At each time step,
in every cell where the calculated value for RTP(IEL,ISCA(ISCAL)) is higher than
\mbox{SCAMAX(ISCAL)}, RTP(IEL,ISCA(ISCAL)) will be reset to
\mbox{SCAMAX(ISCAL)}\\
there is no limitation if SCAMIN(ISCAL)$>$SCAMAX(ISCAL)\\
SCAMAX shall not be specified for non-user scalars (specific physics) or for
scalar variances\\
useful if and only if 1$\leqslant$ISCAL$\leqslant$ NSCAUS}

\motcle{SIGMAS}{RA}{real number $>$ 0}{1D0}{O}{L2}
{SIGMAS(ISCAL): turbulent Prandtl (or Schmidt) number for the scalar ISCAL\\
useful if and only if  1$\leqslant$ISCAL$\leqslant$ NSCAUS}

\motcle{RVARFL}{RA}{real number $>$ 0}{0.8D0}{O}{L2}
{when ISCAVR(ISCAL)$>$0, RVARFL(ISCAL) is the coefficient $R_f$ in the
dissipation term $\displaystyle -\frac{\rho}{R_f}\frac{\varepsilon}{k}$
of the equation concerning the scalar ISCAL,
which represents the root mean square of the
fluctuations of the scalar ISCAVR(ISCAL)\\
useful if and only if there is 1$\leqslant$ISCAL$\leqslant$ NSCAL such as
 ISCAVR(ISCAL)$>$0}


%==================================
\subsubsection{Modeling parameters}
%==================================

\motcleb{XLOMLG}{RA}{real number $>$ 0}{\tt -GRAND*10}{O}{L1}
{for each phase IPHAS, XLOMLG(IPHAS) is the mixing length\\
useful if and only if there is a phase IPHAS so that ITURB(IPHAS)= 10
(mixing length)}

\motcle{ALMAX}{RA}{\texttt{-GRAND}, real number $>$ 0}{\tt -GRAND*10}{O}{L2}
{for each phase IPHAS, ALMAX(IPHAS) is a characteristic macroscopic
length of the domain, used for the initialisation of the turbulence and
the potential clipping (with ICLKEP(IPHAS)=1)\\
negative value: not initialised (the code then uses the cubic root of
the domain volume)\\
useful if and only if there is a phase IPHAS such as
TURB(IPHAS)= 20, 21, 30, 31, 50 or 60 (RANS models)}

\motcleb{UREF}{RA}{real number $>$ 0}{\tt -GRAND*10}{C}{L1}
{for each phase IPHAS, UREF(IPHAS) is the characteristic flow velocity,
used for the initialisation of the turbulence\\
negative value: not initialised\\
useful if and only if there is a phase IPHAS such as
ITURB(IPHAS)= 20, 21, 30, 31, 50 ou 60 (RANS model)
and the turbulence is not initialised somewhere
else (restart file or subroutine \texttt{usiniv})}


\minititre{Basic constants of the $k-\varepsilon$ and the other RANS models}

\motcle{XKAPPA}{R}{real number $>$ 0}{0.42D0}{O}{L3}
{K\'arm\'an constant\\
useful if and only if there is a phase IPHAS such as ITURB(IPHAS)$\geqslant$10
(mixing length, $k-\varepsilon$, $R_{ij}-\varepsilon$, LES, v2f or $k-\omega$)}

\motcle{CSTLOG}{R}{real number $>$ 0}{5.2D0}{O}{L3}
{constant of the logarithmic wall function\\
useful if and only if there is a phase IPHAS such as ITURB(IPHAS)$\geqslant$10
(mixing length, $k-\varepsilon$, $R_{ij}-\varepsilon$, LES, v2f or $k-\omega$)}

\motcle{CMU}{R}{real number $>$ 0}{0.09D0}{O}{L3}
{constant $C_\mu$ for all the RANS turbulence models except for the v2f model
(see CV2FMU\index{CV2FMU} for the value of $C_\mu$ in case of v2f modeling)\\
useful if and only if there is a phase IPHAS such as ITURB(IPHAS)= 20,
21, 30, 31 or 60 ($k-\varepsilon$, $R_{ij}-\varepsilon$ or $k-\omega$)}

\motcle{CE1}{R}{real number $>$ 0}{1.44D0}{O}{L3}
{constant $C_{\varepsilon 1}$ for all the RANS turbulence models except
for the v2f and the $k-\omega$ models\\
useful if and only if there is a phase IPHAS such as ITURB(IPHAS)= 20,
21, 30 or 31 ($k-\varepsilon$ or $R_{ij}-\varepsilon$)}

\motcle{CE2}{R}{real number $>$ 0}{1.92D0}{O}{L3}
{constant $C_{\varepsilon 2}$ for the $k-\varepsilon$ and
$R_{ij}-\varepsilon$ LRR models\\
useful if and only if there is a phase IPHAS such as ITURB(IPHAS)= 20, 21 or 30
($k-\varepsilon$ or $R_{ij}-\varepsilon$ LRR)}

%\motcle{CE3}{R}{real number $>$ 0}{1.D0}{O}{L3}
%{constante $C_{\varepsilon 3}$\\
%ne sert pas dans la version consid\'er\'ee}

\motcle{CE4}{R}{real number $>$ 0}{1.2D0}{O}{L3}
{constant $C_{\varepsilon 4}$ for the interfacial term (Lagrangian module) in
case of two-way coupling\\
useful in case of Lagrangian modeling, in $k-\varepsilon$ and $R_{ij}-\varepsilon$
with two-way coupling}
%c'est aussi implante en k-omega mais en ``imitant''le k-eps et sans aucune verif
% ... et d'ailleurs ca ne marche pas !!

\motcle{SIGMAK}{R}{real number $>$ 0}{1.00D0}{O}{L3}
{Prandtl number for $k$ with $k-\varepsilon$ and v2f models\\
useful if and only if there is a phase IPHAS such as ITURB(IPHAS)=20, 21 or 50
($k-\varepsilon$ or v2f)}

\motcle{SIGMAE}{R}{real number $>$ 0}{1.30D0}{O}{L3}
{Prandtl number for $\varepsilon$\\
useful if and only if there is a phase IPHAS such as ITURB(IPHAS)= 20,
21, 30, 31 or 50 ($k-\varepsilon$, $R_{ij}-\varepsilon$ or v2f)}

\minititre{Constants specific to the $R_{ij}-\varepsilon$ LRR model (ITURB=30)}

\motcle{CRIJ1}{R}{real number $>$ 0}{1.8D0}{O}{L3}
{constant $C_1$ for the $R_{ij}-\varepsilon$ LRR model\\
useful if and only if there is a phase IPHAS such as ITURB(IPHAS)=30
($R_{ij}-\varepsilon$ LRR)}

\motcle{CRIJ2}{R}{real number $>$ 0}{0.6D0}{O}{L3}
{constant $C_2$ for the $R_{ij}-\varepsilon$ LRR model\\
useful if and only if there is a phase IPHAS such as ITURB(IPHAS)=30
($R_{ij}-\varepsilon$ LRR)}

\motcle{CRIJ3}{R}{real number $>$ 0}{0.55D0}{O}{L3}
{constant $C_3$ for the $R_{ij}-\varepsilon$ LRR model\\
useful if and only if there is a phase IPHAS such as ITURB(IPHAS)=30
($R_{ij}-\varepsilon$ LRR)}

\motcle{CRIJEP}{R}{real number $>$ 0}{0.18D0}{O}{L3}
{constant $C_\varepsilon$ for the $R_{ij}-\varepsilon$ LRR model\\
useful if and only if there is a phase IPHAS such as ITURB(IPHAS)=30
($R_{ij}-\varepsilon$ LRR)}

\motcle{CSRIJ}{R}{real number $>$ 0}{0.22D0}{O}{L3}
{constant $C_s$ for the $R_{ij}-\varepsilon$ LRR model\\
useful if and only if there is a phase IPHAS such as ITURB(IPHAS)=30
($R_{ij}-\varepsilon$ LRR)}

\motcle{CRIJP1}{R}{real number $>$ 0}{0.5D0}{O}{L3}
{constant $C_1^\prime$ for the $R_{ij}-\varepsilon$ LRR model, corresponding to
the wall echo terms\\
useful if and only if there is a phase IPHAS such as ITURB(IPHAS)=30 and IRIJEC(IPHAS)=1
($R_{ij}-\varepsilon$ LRR)}

\motcle{CRIJP2}{R}{real number $>$ 0}{0.3D0}{O}{L3}
{constant $C_2^\prime$ for the $R_{ij}-\varepsilon$ LRR model, corresponding to
the wall echo terms\\
useful if and only if there is a phase IPHAS such as ITURB(IPHAS)=30 and IRIJEC(IPHAS)=1
($R_{ij}-\varepsilon$ LRR)}

\minititre{Constants specific to the $R_{ij}-\varepsilon$ SSG model}

\motcle{CSSGS1}{R}{real number $>$ 0}{1.7D0}{O}{L3}
{constant $C_{s1}$ for the $R_{ij}-\varepsilon$ SSG model\\
useful if and only if there is a phase IPHAS such as ITURB(IPHAS)=31
($R_{ij}-\varepsilon$ SSG)}

\motcle{CSSGS2}{R}{real number $>$ 0}{-1.05D0}{O}{L3}
{constant $C_{s2}$ for the $R_{ij}-\varepsilon$ SSG model\\
useful if and only if there is a phase IPHAS such as ITURB(IPHAS)=31
($R_{ij}-\varepsilon$ SSG)}

\motcle{CSSGR1}{R}{real number $>$ 0}{0.9D0}{O}{L3}
{constant $C_{r1}$ for the $R_{ij}-\varepsilon$ SSG model\\
useful if and only if there is a phase IPHAS such as ITURB(IPHAS)=31
($R_{ij}-\varepsilon$ SSG)}

\motcle{CSSGR2}{R}{real number $>$ 0}{0.8D0}{O}{L3}
{constant $C_{r2}$ for the $R_{ij}-\varepsilon$ SSG model\\
useful if and only if there is a phase IPHAS such as ITURB(IPHAS)=31
($R_{ij}-\varepsilon$ SSG)}

\motcle{CSSGR3}{R}{real number $>$ 0}{0.65D0}{O}{L3}
{constant $C_{r3}$ for the $R_{ij}-\varepsilon$ SSG model\\
useful if and only if there is a phase IPHAS such as ITURB(IPHAS)=31
($R_{ij}-\varepsilon$ SSG)}

\motcle{CSSGR4}{R}{real number $>$ 0}{0.625D0}{O}{L3}
{constant $C_{r4}$ for the $R_{ij}-\varepsilon$ SSG model\\
useful if and only if there is a phase IPHAS such as ITURB(IPHAS)=31
($R_{ij}-\varepsilon$ SSG)}

\motcle{CSSGR5}{R}{real number $>$ 0}{0.2D0}{O}{L3}
{constant $C_{r1}$ for the $R_{ij}-\varepsilon$ SSG model\\
useful if and only if there is a phase IPHAS such as ITURB(IPHAS)=31
($R_{ij}-\varepsilon$ SSG)}

\motcle{CSSGE2}{R}{real number $>$ 0}{1.83D0}{O}{L3}
{constant $C_{\varepsilon 2}$ for the $R_{ij}-\varepsilon$ SSG model\\
useful if and only if there is a phase IPHAS such as ITURB(IPHAS)=31
($R_{ij}-\varepsilon$ SSG)}

\minititre{Constants specific to the v2f $\varphi$-model}

\motcle{CV2FA1}{R}{real number $>$ 0}{0.05D0}{O}{L3}
{constant $a_1$ for the v2f $\varphi$-model\\
useful if and only if there is a phase IPHAS such as ITURB(IPHAS)=50
(v2f $\varphi$-model)}

\motcle{CV2FE2}{R}{real number $>$ 0}{1.85D0}{O}{L3}
{constant $C_{\varepsilon 2}$ for the v2f $\varphi$-model\\
useful if and only if there is a phase IPHAS such as ITURB(IPHAS)=50
(v2f $\varphi$-model)}

\motcle{CV2FMU}{R}{real number $>$ 0}{0.22D0}{O}{L3}
{constant $C_\mu$ for the v2f $\varphi$-model\\
useful if and only if there is a phase IPHAS such as ITURB(IPHAS)=50
(v2f $\varphi$-model)}

\motcle{CV2FC1}{R}{real number $>$ 0}{1.4D0}{O}{L3}
{constant $C_1$ for the v2f $\varphi$-model\\
useful if and only if there is a phase IPHAS such as ITURB(IPHAS)=50
(v2f $\varphi$-model)}

\motcle{CV2FC2}{R}{real number $>$ 0}{0.3D0}{O}{L3}
{constant $C_2$ for the v2f $\varphi$-model\\
useful if and only if there is a phase IPHAS such as ITURB(IPHAS)=50
(v2f $\varphi$-model)}

\motcle{CV2FCT}{R}{real number $>$ 0}{6.D0}{O}{L3}
{constant $C_T$ for the v2f $\varphi$-model\\
useful if and only if there is a phase IPHAS such as ITURB(IPHAS)=50
(v2f $\varphi$-model)}

\motcle{CV2FCL}{R}{real number $>$ 0}{0.25D0}{O}{L3}
{constant $C_L$ for the v2f $\varphi$-model\\
useful if and only if there is a phase IPHAS such as ITURB(IPHAS)=50
(v2f $\varphi$-model)}

\motcle{CV2FET}{R}{real number $>$ 0}{110.D0}{O}{L3}
{constant $C_\eta$ for the v2f $\varphi$-model\\
useful if and only if there is a phase IPHAS such as ITURB(IPHAS)=50
(v2f $\varphi$-model)}

\minititre{Constants specific to the $k-\omega$ SST model}

\motcle{CKWSK1}{R}{real number $>$ 0}{1.D0/0.85D0}{O}{L3}
{constant $\sigma_{k1}$ for the $k-\omega$ SST model\\
useful if and only if there is a phase IPHAS such as ITURB(IPHAS)=60
($k-\omega$ SST)}

\motcle{CKWSK2}{R}{real number $>$ 0}{2.D0}{O}{L3}
{constant $\sigma_{k2}$ for the $k-\omega$ SST model\\
useful if and only if there is a phase IPHAS such as ITURB(IPHAS)=60
($k-\omega$ SST)}

\motcle{CKWSW1}{R}{real number $>$ 0}{2.D0}{O}{L3}
{constant $\sigma_{\omega 1}$ for the $k-\omega$ SST model\\
useful if and only if there is a phase IPHAS such as ITURB(IPHAS)=60
($k-\omega$ SST)}

\motcle{CKWSW2}{R}{real number $>$ 0}{1.D0/0.856D0}{O}{L3}
{constant $\sigma_{\omega 2}$ for the $k-\omega$ SST model\\
useful if and only if there is a phase IPHAS such as ITURB(IPHAS)=60
($k-\omega$ SST)}

\motcle{CKWBT1}{R}{real number $>$ 0}{0.075D0}{O}{L3}
{constant $\beta_1$ for the $k-\omega$ SST model\\
useful if and only if there is a phase IPHAS such as ITURB(IPHAS)=60
($k-\omega$ SST)}

\motcle{CKWBT2}{R}{real number $>$ 0}{0.0828D0}{O}{L3}
{constant $\beta_2$ for the $k-\omega$ SST model\\
useful if and only if there is a phase IPHAS such as ITURB(IPHAS)=60
($k-\omega$ SST)}

\motcle{CKWGM1}{R}{real number $>$ 0}
{$\frac{\beta_1}{C_\mu}-\frac{\kappa^2}{\sqrt{C_\mu}\sigma_{\omega 1}}$}{O}{L3}
{constant $\gamma_1$ for the $k-\omega$ SST model\\
useful if and only if there is a phase IPHAS such as ITURB(IPHAS)=60
($k-\omega$ SST)\\
{\em Warning: $\gamma_1$ is calculated before the call to
\texttt{usini1}. Hence, if $\beta_1$, $C_\mu$, $\kappa$ or $\sigma_{\omega 1}$
is modified in \texttt{usini1}, CKWGM1 must also be modified in accordance}}

\motcle{CKWGM2}{R}{real number $>$ 0}
{$\frac{\beta_2}{C_\mu}-\frac{\kappa^2}{\sqrt{C_\mu}\sigma_{\omega 2}}$}{O}{L3}
{constant $\gamma_2$ for the $k-\omega$ SST model\\
useful if and only if there is a phase IPHAS such as ITURB(IPHAS)=60
($k-\omega$ SST)\\
{\em Warning: $\gamma_2$ is calculated before the call to
\texttt{usini1}. Hence, if $\beta_2$, $C_\mu$, $\kappa$ or $\sigma_{\omega 2}$
is modified in \texttt{usini1}, CKWGM2 must also be modified in accordance}}

\motcle{CKWA1}{R}{real number $>$ 0}{0.31D0}{O}{L3}
{constant $a_1$ for the $k-\omega$ SST model\\
useful if and only if there is a phase IPHAS such as ITURB(IPHAS)=60
($k-\omega$ SST)}

\motcle{CKWC1}{R}{real number $>$ 0}{10.D0}{O}{L3}
{constant $c_1$ for the $k-\omega$ SST model\\
useful if and only if there is a phase IPHAS such as ITURB(IPHAS)=60
($k-\omega$ SST)}


%==================================
\subsection{Thermal radiative transfers: global settings}
%==================================

All the following key words may be modified in the user subroutines
\texttt{usray*} (or, for some of them, by through the thermochemical data files).
It is however not recommended to modify those which do not belong to
level L1.

\motcleb{IRAYON}{IA}{0, 1, 2}{0}{O}{L1}
{for each phase IPHAS, IRAYON(IPHAS) activates ($>0$) or deactivates
(=0) the radiation module\\
if a specific physics is activated (in that case, NSCAPP$>$0),
IRAYON(IPHAS) must be kept to 0 (see IRAYPP)\\
The different values correspond to the following modelings:\\
\hspace*{1.3cm} = 1 discrete ordinates
(standard option for radiation in semi-transparent media)\\
\hspace*{1.3cm} = 2  ``P-1'' model\\
{\em Warning: the P-1 model allows faster computations, but it
may only be applied to media with uniform large optical thickness, such as
some cases of pulverised coal combustion}}

\motcleb{IRAYPP}{I}{0, 1, 2, 3 or 4}{0}{O}{L1}
{when a specific physics is activated\footnote{except with the compressible
module, which is not compatible with radiation}
(NPHAS=1, compulsory) IRAYPP
indicates if thermal radiative transfers are calculated ($>0$) or not
(=0).\\
The value of IRAYPP is given {\em via} a data file (gas combustion:
{\em dp\_C3P}, {\em dp\_C3PSJ}, or {\em dp\_C4P};
pulverised coal combustion: {\em dp\_FCP}; electric module: {\em dp\_ELE})\\
IRAYPP allows to choose between the discrete ordinates method and the
P-1 method (see IRAYON) and to choose the method used to calculate the
absorption coefficient. The absorption coefficient may be set by the
user in the data file (then, \mbox{IMODAK=0}) or calculated using
``Modak\footnote{for details about the calculation of the absorption
coefficient, please refer to MODAK A.T., ``Radiation from products
of combustion''}'' (then, IMODAK=1).
The options are the followings:\\
\hspace*{1.3cm} = 1 discrete ordinates method with the absorption coefficient
given by the user in the data file (IMODAK=0)\\
\hspace*{1.3cm} = 2 discrete ordinates method using Modak for the
calculation of the absorption coefficient (IMODAK=1)\\
\hspace*{1.3cm} = 3 ``P-1'' model  with the absorption coefficient
given by the user in the data file (IMODAK=0)\\
\hspace*{1.3cm} = 4 ``P-1'' model using Modak for the
calculation of the absorption coefficient (IMODAK=1)\\
For the electric module, IRAYPP is not set directly in the data file, but
deduced from the type of XKABEL specified in the file (given by IXKABE).
In that case, IRAYPP can only be equal to 0 (IXKABE=0 or 2) or 1 (IXKABE=1)}

\motcle{IMODAK}{I}{0 or 1}{0}{O}{L3}
{when gas or coal combustion is activated, IMODAK indicates whether the
absorption coefficient shall be calculated ``automatically'' (=1) or read from
the data file (=0) (see IRAYPP)\\
useful if the radiation module is activated; IMODAK is then automatically set
from the value of IRAYPP, without intervention of the user}

\motcleb{ISUIRD}{I}{0 or 1}{ISUITE}{C}{L1}
{indicates whether the radiation variables should be initialised (=0) or read
from a restart file (=1)\\
useful if and only if the radiation module is activated (in this case, a
restart file {\em rayamo} must be available)}

\motcleb{NFREQR}{I}{strictly positive integer}{1}{O}{L1}
{period of the radiation module\\
the radiation module is called every NFREQR time steps (more precisely, every
time NTCABS is a multiple of NFREQR). Also, in order to have proper
initialisation of the variables, whatever the value of NFREQR, the radiation
module is called at the first time step of a calculation (restart or not)\\
useful if and only if the radiation module is activated}

\motcleb{NDIREC}{I}{32 ou 128}{32}{O}{L1}
{number of directions for the angular discretisation of the radiation
propagation with the DOM model (IRAYON=1)\\
no other possible value, because of the way the directions are calculated\\
the calculation with 32 directions may break the symmetry of
physically axisymmetric cases (but the cost in CPU time is much lower
 than with 128 directions)\\
useful if and only if the radiation module is activated with the DOM method}

\motcle{XNP1MX}{R}{real number}{10}{O}{L3}
{with the P-1 model (IRAYON=2), XNP1MX is the percentage of cells of
the calculation domain for which it is acceptable that the optical
thickness is lower than unity\footnote{more precisely, where $KL$ is lower than
1, where $K$ is the absorption coefficient of the medium and $L$ is a
characteristic length of the domain}, although it is not to be desired\\
useful if and only if the radiation module is activated with the P-1 method}

\motcleb{IDIVER}{I}{0, 1 or 2}{2}{C}{L1}
{indicates the method used to calculate the radiative source term:\\
\hspace*{1.3cm}= 0: semi-analytic calculation (compulsory with transparent
media)\\
\hspace*{1.3cm}= 1: conservative calculation\\
\hspace*{1.3cm}= 2: semi-analytic calculation corrected in order to be
globally consevative\\
useful if and only if the radiation module is activated\\
{\em Note: if the medium is transparent, the choice has no effect on the calculation}}

\motcleb{IIMPAR}{I}{0, 1 or 2}{1}{O}{L1}
{choice of the display level in the listing concerning the calculation of
the wall temperatures:\\
\hspace*{1.3cm}= 0: no display\\
\hspace*{1.3cm}= 1: standard\\
\hspace*{1.3cm}= 2: complete\\
useful if and only if the radiation module is activated}

\motcleb{IIMLUM}{I}{0, 1 or 2}{1}{O}{L1}
{choice of the display level in the listing concerning the solution of
the radiative transfer equation:\\
\hspace*{1.3cm}= 0: no display\\
\hspace*{1.3cm}= 1: standard\\
\hspace*{1.3cm}= 2: complete\\
useful if and only if the radiation module is activated}

\motcleb{NBRVAP}{CA}{string of less than 80 characters}{name\_IPHAS}{O}{L1}
{name associated for the post-processing to each of the following variables,
defined at the cell centers ({\it see}
\cite{Douce02} for more details concerning their definitions):\\
\hspace*{1.3cm} NBRVAP(ITSRAY,IPHAS): radiative source term ($W/m^3$)\\
\hspace*{1.3cm} NBRVAP(IQRAYP,IPHAS): radiative flux density vector ($W/m^2$)\\
\hspace*{1.3cm} NBRVAP( IABSP,IPHAS): absorption part in the source term
($W/m^3$)\\
\hspace*{1.3cm} NBRVAP( IEMIP,IPHAS): emission part in the source term
($W/m^3$)\\
\hspace*{1.3cm} NBRVAP( ICAKP,IPHAS): absorption coefficient of the
medium ($m^{-1}$)\\
the default values are: \\
\hspace*{1.3cm} NBRVAP(ITSRAY,IPHAS) = Srad\_IPHAS \\
\hspace*{1.3cm} NBRVAP(IQRAYP,IPHAS) = Qrad\_IPHAS \\
\hspace*{1.3cm} NBRVAP( IABSP,IPHAS) = Absorp\_IPHAS \\
\hspace*{1.3cm} NBRVAP( IEMIP,IPHAS) = Emiss\_IPHAS \\
\hspace*{1.3cm} NBRVAP( ICAKP,IPHAS) = CoefAb\_IPHAS \\
useful if and only if the radiation module is activated}

\motcleb{IRAYVP}{IA}{-1 or 1}{-1}{O}{L1}
{activates (=1) or deactivates (=-1) the post-processing for the each of the
followiing variables defined at the cell centers:\\
\hspace*{1.3cm} IRAYVP(ITSRAY,IPHAS): radiative source term ($W/m^3$)\\
\hspace*{1.3cm} IRAYVP(IQRAYP,IPHAS): radiative flux density vector ($W/m^2$)\\
\hspace*{1.3cm} IRAYVP(IABSP,IPHAS): absorption part in the source term
($W/m^3$)\\
\hspace*{1.3cm} IRAYVP(IEMIP,IPHAS): emission part in the source term
($W/m^3$)\\
\hspace*{1.3cm} IRAYVP(ICAKP,IPHAS): absorption coefficient of the
medium ($m^{-1}$)\\
useful if and only if the radiation module is activated}

\motcleb{NBRVAF}{CA}{string of less than 80 characters}{name\_IPHAS}{O}{L1}
{name associated for the post-processing to each of the following variables,
defined at the boundary faces ({\it see}
\cite{Douce02} for more details concerning their definitions):\\
\hspace*{1.3cm} NBRVAF(ITPARP,IPHAS): wall temperature at the boundary
faces ($K$)\\
\hspace*{1.3cm} NBRVAF(IQINCP,IPHAS): radiative incident flux density
($W/m^2$)\\
\hspace*{1.3cm} NBRVAF(IXLAMP,IPHAS): thermal conductivity of the
boundary faces ($W/m/K$)\\
\hspace*{1.3cm} NBRVAF(IEPAP,IPHAS): wall thickness ($m$)\\
\hspace*{1.3cm} NBRVAF(IEPSP,IPHAS): wall emissivity \\
\hspace*{1.3cm} NBRVAF(IFNETP,IPHAS): net radiative flux density ($W/m^2$)\\
\hspace*{1.3cm} NBRVAF(IFCONP,IPHAS): convective flux density ($W/m^2$)\\
\hspace*{1.3cm} NBRVAF(IHCONP,IPHAS): convective exchange coefficient
($W/m^2/K$)\\
The default values are: \\
\hspace*{1.3cm} NBRVAF(ITPARP,IPHAS) = Temp\_paroi\_IPHAS \\
\hspace*{1.3cm} NBRVAF(IQINCP,IPHAS) = Flux\_incident\_IPHAS \\
\hspace*{1.3cm} NBRVAF(IXLAMP,IPHAS) = Conductivite\_th\_IPHAS \\
\hspace*{1.3cm} NBRVAF(IEPAP,IPHAS)  = Epaisseur\_IPHAS \\
\hspace*{1.3cm} NBRVAF(IEPSP,IPHAS)  = Emissivite\_IPHAS \\
\hspace*{1.3cm} NBRVAF(IFNETP,IPHAS) = Flux\_net\_IPHAS \\
\hspace*{1.3cm} NBRVAF(IFCONP,IPHAS) = Flux\_convectif\_IPHAS \\
\hspace*{1.3cm} NBRVAF(IHCONP,IPHAS) = Coef\_ech\_convectif\_IPHAS \\
useful if and only if the radiation module is activated}

\motcleb{IRAYVF}{IA}{-1 or 1}{-1}{O}{L1}
{activates (=1) or deactivates (=-1) the post-processing for each of the
followiing variables defined at the boundary faces:\\
\hspace*{1.3cm} IRAYVF(ITPARP,IPHAS): wall temperature at the boundary
faces ($K$)\\
\hspace*{1.3cm} IRAYVF(IQINCP,IPHAS): radiative incident flux density
($W/m^2$)\\
\hspace*{1.3cm} IRAYVF(IXLAMP,IPHAS): thermal conductivity of the
boundary faces ($W/m/K$)\\
\hspace*{1.3cm} IRAYVF( IEPAP,IPHAS): wall thickness ($m$)\\
\hspace*{1.3cm} IRAYVF( IEPSP,IPHAS): wall emissivity \\
\hspace*{1.3cm} IRAYVF(IFNETP,IPHAS): net radiative flux density ($W/m^2$)\\
\hspace*{1.3cm} IRAYVF(IFCONP,IPHAS): convective flux density ($W/m^2$)\\
\hspace*{1.3cm} IRAYVF(IHCONP,IPHAS): convective exchange coefficient
($W/m^2/K$)\\
useful if and only if the radiation module is activated}

\motcle{TMIN}{R}{real number positif}{0.D0}{O}{L3}
{minimum allowed value for the wall temperatures in Kelvin\\
useful if and only if the radiation module is activated}

\motcle{TMAX}{R}{real number positif}{GRAND + 273.15D0}{O}{L3}
{maximum allowed value for the wall temperatures in Kelvin\\
useful if and only if the radiation module is activated}

%==================================
\subsection{Electric module (Joule effect and electric arc): specificities}
%==================================

The electric module is composed of a Joule effect module
(IPPMOD(IELJOU)\index{IELJOU}) and an electric arc module
(IPPMOD(IELARC)\index{IELARC}).

The Joule effect module is designed to take into account the Joule effect
(for instance in glass furnaces) with real or complex potential in the
enthalpy equation. The Laplace forces are not taken into account in the
impluse momentum equation. Specific boundary conditions can be applied to
account for the coupled effect of transformers (offset) in glass furnaces.

The electric arc module is designed to take into account the Joule effect
(only with real potential) in the enthalpy equation. The Laplace forces
are taken into account in the impulse momentum equation.

The key words used in the global settings are quite few. They are
found in the subroutine \texttt{useli1} (see the description of this
user subroutine \S\ref{prg_useli1}).

\motcleb{IELCOR}{I}{0, 1}{0}{O}{L1}
{when IELCOR=1, the boundary conditions for the potential will be tuned at each
time step in order to reach a user-specified target dissipated power PUISIM
(Joule effect) or a user-specified target current intensity COUIMP (electric
arc)\\
the boundary condition tuning is controlled by the subroutine \texttt{uselrc}\\
alway useful}

\motcleb{COUIMP}{R}{real number $\geqslant 0$}{0}{O}{L1}
{with the electric arc module, COUIMP is the target current intensity ($A$)
for the calculations with boundary condition tuning for the potential\\
the target intensity will be reached if the boundary conditions are expressed
using the variable DPOT or if the initial boundary conditions are multiplied by
the variable COEJOU\\
useful with the electric arc module if IELCOR=1}

\motcleb{PUISIM}{R}{real number $\geqslant 0$}{0}{O}{L1}
{with the Joule effect module, PUISIM is the target dissipated power ($W$)
for the calculations with boundary condition tuning for the potential\\
the target power will be reached if the boundary conditions are expressed
using the variable DPOT or if the initial boundary conditions are multiplied by
the variable COEJOU\\
useful with the Joule effect module if IELCOR=1}

\motcleb{DPOT}{R}{real number $\geqslant 0$}{0}{O}{L1}
{DPOT is the potential difference ($V$) which generates the current (and the Joule effect) for the
calculations with boundary conditions tuning for the potential. This value is
initialised set by the user (\texttt{useli1}). It is then automatically tuned
depending on the value of dissipated power (Joule effect module) or the
intensity of current (electric arc module).
In order for the correct power or intensity to be reached, the boundary
conditions for the potential must be expressed with DPOT (\texttt{uselcl}). The
tuning can be controlled in \texttt{uselrc}\\
useful if IELCOR=1}

\motcle{COEJOU}{R}{real number $\geqslant 0$}{1}{O}{L2}
{only with the Joule effect, COEJOU can be used if the user does not
wish to use DPOT. COEJOU is the coefficient to be applied to the initial
potential difference to reach the target dissipated power. Its value is
automatically initialised to 1 and is updated during the calculation.
In order for the correct power to be reached, the boundary
conditions for the potential must be expressed with COEJOU (\texttt{uselcl}). The
tuning can be controlled in \texttt{uselrc}\\
Useful if IELCOR=1}

%=================================================
\subsection{Compressible module: specificities}
%==================================

The key words used in the global settings are quite few. They are
found in the subroutines \texttt{uscfx1} and \texttt{uscfx2} (see the
description of these user subroutines, \S\ref{prg_uscfx12}).

\motcleb{ICFGRP}{IA}{0 or 1}{1}{C}{L1}
{for each phase IPHAS, ICFGRP(IPHAS) indicates if the boundary
conditions should take into account (=1) or not (=0) the hydrostatic balance.\\
always useful.\\
In the cases where gravity is predominant, taking into account the hydrostatic
pressure allows to get rid of the disturbances which may appear near the
horizontal walls when the flow is little convective.\\
%
Otherwise, when ICFGRP=0, the pressure condition is calculated
from the solution of the unidimensional Euler equations for a perfect
gas near a wall, for the variables ``normal velocity'', ``density'' and
``pressure'':
\vspace{0.2cm}
\begin{list}{}{}
\item Case of an expansion ($M \leqslant 0$):\\
$$
\begin{array}{l}
\left\{\begin{array}{lll}
P_p=0 & \text{if} & 1 + \displaystyle\frac{\gamma-1}{2}M<0\\
P_p = P_i \left(1 + \displaystyle\frac{\gamma-1}{2}M\right)
^{\frac{2\gamma}{\gamma-1}} & \text{otherwise}\\
\end{array}\right.\\
\end{array}
$$

\item Case of a shock ($M > 0$):\\
$$
\begin{array}{l}
P_p = P_i \left(1 + \displaystyle\frac{\gamma(\gamma+1)}{4}M^2
+\gamma M \displaystyle\sqrt{1+\displaystyle\frac{(\gamma+1)^2}{16}M^2}\right)
\end{array}
$$

with $M = \displaystyle\frac{\vect{u}_i \cdot \vect{n}}{c_i}$, internal
      Mach number calculated with the variables taken in the cell
      adjacent to the wall.\\

\end{list}
}

\motcleb{IVISCV}{IA}{0 or 1}{\tt 0}{C}{L1}
{for each phase IPHAS, IVISCV(IPHAS)=0 indicates that the volume
viscosity is constant and equal to the reference volume viscosity
VISCV0(IPHAS).\\
IVISCV(IPHAS)=1 indicates that the volume viscosity is variable: its
variation law must be specified in the user subroutine \texttt{uscfpv}.\\
always useful\\
The volume viscosity $\kappa$ is defined by the formula expressing the stress:
\begin{equation}
   \tens{\sigma} = -P\,\tens{Id} + \mu (\grad\,\vect{u} +  \ ^{t}\ggrad\,\vect{u})
           +(\kappa-\frac{2}{3}\mu)\,\dive(\vect{u})\,\tens{Id}
\end{equation}
}

\motcleb{VISCV0}{RA}{real number $\ge$ 0}{0.D0}{O}{L1}
{for each phase IPHAS, VISCV0(IPHAS) is the reference volume viscosity
(noted $\kappa$ in the equation expressing $\tens{\sigma}$ in the
paragraph dedicated to IVISCV)\\
always useful, it is the used value, unless the user specifies the volume
viscosity in the user subroutine \texttt{uscfpv}}

\motcle{IGRDPP}{I}{0 ou 1}{0}{O}{L3}
{indicates whether the pressure should be updated (=1) or not (=0) after the
solution of the acoustic equation\\
always useful}

%==================================================
\subsection{Lagrangian multiphase flows}
%==================================================
\label{prg_motscles_lagr}

Most of these key words may be modified in the user subroutines
\texttt{uslag1}, \texttt{uslag2}, \texttt{uslabo}, \texttt{uslaen},
\texttt{uslast} and \texttt{uslaed}. It is however strongly recommended
not to modify those belonging to the level L3.

First of all, it should be noted that the Lagrangian module is compliant with
all the RANS turbulence models and with laminar flows. However, the particule
turbulent diffusion is not specially adapted to the second order
$R_{ij}-\varepsilon$ models. The same isotropic model is used as in the
$k-\varepsilon$ models, with $k$ calculated from the trace of $R_{ij}$. Also,
two-way coupling is not compatible with the $k-\omega$ SST model.


\subsubsection{Global settings}

\motcleb{IILAGR}{I}{0, 1, 2, 3}{0}{C}{L1}
{activates ($>$0) or deactivates (=0) the Lagrangian module\\
the different values correspond to the following modelings: \\
\hspace*{1.3cm} = 1 Lagrangian two-phase flow in one-way coupling (no influence of
the particles on the continuous phase)\\
\hspace*{1.3cm} = 2 Lagrangian two-phase flow with two-way coupling (influence of
the particles on the dynamics of the continuous phase).
It must be noted that the two-way coupling is taken
into account only for the first eulerian phase. Dynamics,
temperature and mass may be coupled independently\\
\hspace*{1.3cm} = 3 Lagrangian two-phase flow on frozen continuous phase. This option can
only be used in case of a calculation restart (ISUITE = 1). All the eulerian
fields are frozen (including the scalar fields). This option automatically
implies ICCVFG = 1\\
always useful}

\motcleb{ISUILA}{I}{0, 1}{0}{C}{L1}
{activation (=1) or not (=0) of a Lagrangian calculation restart.
The calculation restart file read when this option is activated (FICAML)
only contains the data related to the particles (see also ISUIST)\\
the global calculation must also be a restart calculation (ISUITE=1)\\
always useful}

\motcleb{ISUIST}{I}{0, 1}{0}{C}{L1}
{during a Lagrangian calculation restart, indicates whether the particle
statistics (volume and boundary) and two-way coupling terms are to be read from
a restart file (=1) or reinitialised (=0). The file to be read is FICMLS\\
useful if ISUILA = 1}

\motcleb{NBPMAX}{I}{positive or null integer}{1000}{C}{L1}
{maximum number of particles allowed simultaneously in the calculation
domain. It must be reminded that the required memory evolves accordingly}

\motcle{NBPART}{I}{positive or null integer}{0}{O}{L3}
{number of particles treated during one Lagrangian time step\\
NBPART must always be lower than NBPMAX\\
always useful, but initialised and updated without intervention of the user}

\motcle{NVLS}{I}{integer between 0 and 10}{0}{O}{L2}
{number of additional variables related to the particles\\
the additional variables can be accessed in the arrays
ETTP\index{ETTP} and ETTPA\index{ETTPA} by means of the pointer
JVLS\index{JVLS}: ETTP(NBPT,JVLS(II)) and ETTPA(NBPT,JVLS(II)) (NBPT is
the index-number of the treated particle, and II an integer between 1 and NVLS)}

\motcleb{ISTTIO}{I}{0, 1}{0}{C}{L1}
{indicates the steady (=1) or unsteady (=0) state of the
continuous phase flow\\
in particular, ISTTIO = 1 is needed in order to: \\
\hspace*{1cm}calculate stationary statistics in the volume or at the boundaries
(starting respectively from the Lagrangian iterations NSTIST and NSTBOR) \\
\hspace*{1cm}calculate time-averaged two-way coupling source terms (from the
Lagrangian iteration NSTITS) \\
useful if IILAGR=1 or IILAGR=2 (if IILAGR=3, then ISTTIO=1 automatically)}

\motcleb{INJCON}{I}{0, 1}{0}{O}{L1}
{activates (=1) or not (=0) the continuous injection of particles\\
this option allows to inject particles continuously during the duration of the
Lagrangian time step DTP\index{DTP} rather than only once at the
beginning of the Lagrangian iteration. It helps avoiding the fractioning of
the particle cloud close to the injection areas}

\motcleb{IROULE}{I}{0, 1}{0}{O}{L1}
{activates (=1) or not (=0) of the particle cloning/fusion technique
(option also called ``Russian roulette'')\\
when IROULE = 1, the importance function must be specified {\em via} the
array CROULE\index{CROULE} in the user subroutine \texttt{uslaru}}

\motcle{ISUIVI}{I}{0, 1}{0 or 1}{O}{L2}
{specifies if a particle should be followed (=1) or will disappear from the
domain (=0) after an interaction with a boundary:\\
\hspace*{1.3cm} = 0: the particle must not be followed in the
calculation domain after an iteraction between its trajectory and a
boundary face, for instance entry (IENTRL), outlet (ISORTL), definitive
deposition on a wall (IDEPO1, IDEPO2)\\
\hspace*{1.3cm} = 1: the particle must still be followed in the
calculation domain after an iteraction between its trajectory and a
boundary face, for instance rebound (IREBOL), deposition with potential
resuspension (IDEPO3)\\
the value of ISUIVI (ISUIVI = 0 or ISUIVI = 1) for a type of interaction
can be defined as a function of the particle behaviour or properties. It
is for example the default case for the fouling interaction type (IENCRL)\\
always useful}

\motcle{TTCLAG}{R}{positive real number}{0}{O}{L3}
{physical time of the Lagrangian simulation \\
always useful}

\motcle{IPLAS}{I}{integer $>$ 0}{1}{O}{L3}
{absolute iteration number (including the restarts) in the Lagrangian
module ({\em i.e.} Lagrangian time step number) \\
always useful}


%==================================================
\subsubsection{Specific physics models associated with the particles}
%==================================================

\motcleb{IPHYLA}{I}{0, 1, 2}{0}{C}{L1}
{activates ($>$0) or deactivates (=0) the physical models associated to the
particles:\\
\hspace*{1.3cm} = 1: allows to associate with the particles evolution
equations on their temperature (in degrees Celsius), their diameter and
their mass\\
\hspace*{1.3cm} = 2: the particles are pulverised coal particles.
Evolution equations on temperature (in degree Celsius), mass of
reactive coal, mass of char and diameter of the shrinking core are
associated with the particles. This option is available only if the
continuous phase represents a pulverised coal flame\\
always useful}

\motcleb{IDPVAR}{I}{0, 1}{0}{O}{L1}
{activation (=1) or not (=0) of an evolution equation on the particle
diameter\\
useful if IPHYLA = 1}

\motcleb{ITPVAR}{I}{0, 1}{0}{O}{L1}
{activation (=1) or not (=0) of an evolution equation on the particle
temperature (in degrees Celsius)\\
useful if IPHYLA = 1 and if there is a thermal scalar associated with
the continuous phase}

\motcleb{IMPVAR}{I}{0, 1}{0}{O}{L1}
{activation (=1) or not (=0) of an evolution equation on the
particle mass\\
useful if si IPHYLA = 1}

\motcleb{TPART}{R}{real number $>$ TKELVN}{700.D0}{O}{L1}
{initialisation temperature (in degree Celsius) for the particles already
present in the calculation domain when an evolution equation on
the particle temperature is activated during a calculation (IPHYLA =
1 and ITPVAR = 1)\\
useful if ISUILA = 1 and ITPVAR = 0 in the previous calculation}

\motcleb{CPPART}{R}{positive real number}{5200.D0}{O}{L1}
{initialisation value for the specific heat ($J.kg^{-1}.K^{-1}$) of the
particles already present in the calculation domain when an evolution equation
on the particle temperature is activated during a calculation (IPHYLA =
1 and ITPVAR = 1)\\
useful if ISUILA = 1 and ITPVAR = 0 in the previous calculation}

\motcleb{IENCRA}{I}{0, 1}{0}{O}{L1}
{activates (=1) or not (=0) the option of coal particle
fouling. It then is necessary to specify the domain boundaries
on which fouling may take place.\\
useful if IPHYLA = 2}

\motcleb{TPRENC}{R}{real number $>$ TKELVN}{600.D0}{O}{L1}
{limit temperature (in degree Celsius) below which the coal particles do
not cause any fouling (if the fouling model is activated)\\
useful if IPHYLA = 2 and IENCRA = 1}

\motcleb{VISREF}{R}{positive real number}{10000.D0}{O}{L1}
{ash critical viscosity in $kg.m^{-1}.s^{-1}$, in the fouling model
\footnote{J.D. Watt
et T. Fereday (\textit{J.Inst.Fuel}, Vol.42-p99)}\\
useful if IPHYLA = 2 and IENCRA = 1}

%==================================================
\subsubsection{Options for two-way coupling}
%==================================================

\motcleb{NSTITS}{I}{strictly positive integer}{1}{O}{L1}
{number of absolute Lagrangian iterations (including the restarts)
after which a time-average of the two-way coupling source terms is
calculated\\
indeed, if the flow is steady (ISTTIO=1), the average quantities that appear in
the two-way coupling source terms can be calculated over different time steps,
in order to get a better precision\\
if the number of absolute Lagrangian iterations is strictly inferior to
NSTITS, the code considers that the flow has not yet reached its steady state
(transition period) and the averages appearing in the source terms are
reinitialised at each time step, as it is the case for unsteady flows (ISTTIO=0)\\
useful if IILAGR = 2 and ISTTIO = 1}

\motcleb{LTSDYN}{I}{0, 1}{0}{O}{L1}
{activation (=1) or not (=0) of the two-way coupling on the dynamics of the
continuous phase\\
useful if IILAGR = 2 and ICCVFG = 0}

\motcleb{LTSMAS}{I}{0, 1}{0}{O}{L1}
{activation (=1) or not (=0) of the two-way coupling on the mass\\
useful if IILAGR = 2, IPHYLA = 1 and IMPVAR = 1}

\motcleb{LTSTHE}{I}{0, 1}{0}{O}{L1}
{if IPHYLA = 1 and ITPVAR = 1, LTSTHE activates (=1) or not (=0) the
two-way coupling on temperature\\
if IPHYLA = 2, LTSTHE activates (=1) or not (=0) the two-way coupling
on the eulerian variables related to pulverised coal combustion\\
useful if IILAGR = 2}

%==================================================
\subsubsection{Numerical modeling}
%==================================================

\motcle{NORDRE}{I}{1, 2}{2}{O}{L2}
{order of integration for the stochastic differential equations\\
\hspace*{1.3cm} = 1 integration using a first-order scheme\\
\hspace*{1.3cm} = 2 integration using a second-order scheme\\
always useful}

\motcle{ILAPOI}{I}{0, 1}{0}{O}{L3}
{activation (=1) or not (=0) of the solution of a Poisson's equation for
the correction of the particle instantaneous velocities (in order to obtain a
null divergence)\\
this option is not validated and reserved to the development team. Do not change
the default value}

\motcle{IDISTU}{I}{0, 1}{1}{O}{L3}
{activation (=1) or not (=0) of the particle turbulent dispersion\\
the turbulent dispersion is compatible only with the RANS turbulent models
($k-\varepsilon$, $R_{ij}-\varepsilon$, v2f or $k-\omega$)\\
(ITURB(IPHAS)=20, 21, 30, 31, 50 or 60 with IPHAS = 1)\\
always useful}

\motcle{IDIFFL}{I}{0, 1}{0}{O}{L3}
{IDIFFL=1 suppresses the crossing trajectory effect, making turbulent dispersion
for the particles identical to the turbulent diffusion of fluid particles\\
useful if IDISTU=1}

\motcleb{MODCPL}{I}{positive integer}{0}{O}{L1}
{activates ($>$0) or not (=0) the complete turbulent dispersion model\\
when MODCPL is strictly positive, its value is interpreted as the absolute
Lagrangian time step number (including restarts) after which the complete model
is applied\\
since the complete model uses volume statistics, MODCPL must either be 0 or be
larger than IDSTNT\\
useful if ISTALA = 1}

\motcleb{IDIRLA}{I}{1, 2, 3}{1}{O}{L1}
{$x$, $y$ or $z$ direction of the complete model\\
it corresponds to the main directions of the flow\\
useful if MODCPL $>$ 0}

%==================================================
\subsubsection{Volume statistics}
%==================================================

\motcleb{ISTALA}{I}{0, 1}{0}{C}{L1}
{activation (=1) or not (=0) of the calculation of the volume
statistics related to the dispersed phase\\
if ISTALA = 1, the calculation of the statistics is activated starting from the
absolute iteration (including the restarts) IDSTNT\\
by default, the statistics are not stationary (reset to zero at every
Lagrangian iteration). But if ISTTIO=1, since the flow is steady, the statistics
will be averaged overt he different time steps\\
the statistics represent the significant results on the particle cloud\\
always useful}

\motcleb{SEUIL}{R}{positive real number}{0.D0}{O}{L1}
{every cell of the calculation domain contains a certain quantity of
particles, representing a certain statistical weight (sum of the
statistical weights of all the particles present in the cell). SEUIL is
the limit statistical weight value, below which the contribution of the
cell in term of statistical weight is not taken into account in the volume
statistics (for the complete turbulent dispersion model, in the
Poisson's equation used to correct the mean velocities or in the listing and
post-processing outputs)\\
useful if ISTALA = 1}

\motcleb{IDSTNT}{I}{strictly positive integer}{1}{C}{L1}
{absolute Lagrangian iteration number (includings the restarts) after
which the calculation of the volume statistics is activated\\
useful if ISTALA = 1}

\motcleb{NSTIST}{I}{integer $\geqslant$ IDSTNT}{IDSTNT}{O}{L1}
{absolute Lagrangian iteration number (includings the restarts) after
which the volume statistics are cumulated over time (they are then said to be
stationary)\\
if the absolute Lagrangian iteration number is lower than NSTIST, or if the flow
is unsteady (ISTTIO=0), the statistics are reset to zero at every Lagrangian
iteration (the volume statistics are then said to be non-stationary)\\
useful if ISTALA=1 and ISTTIO=1}

\motcleb{NOMLAG}{CA}{string of less than 50 characters}{VarLagXXXX}{O}{L1}
{name of the volumetric statistics, displayed in the listing
and the post-processing files. The default value is given above, with ``XXXX''
representing a four digit number (for instance 0001, 0011 ...)\\
useful if ISTALA = 1\\
{\em Warning: this name is also used to reference information in the restart file
\mbox{(ISUIST =1)}. If the name of a variable is changed between two
calculations, it will not be possible to read its value from the restart file}}

\motcleb{NVLSTS}{I}{0 $\leqslant$ integer $\leqslant$ NUSSTA=20}{0}{O}{L1}
{number of additional user volume statistics\\
the additional statistics (or their cumulated value in the stationary
case) can be accessed in the array STATIS by means of the pointer
ILVU\index{ILVU}: STATIS(IEL,ILVU(II)\index{STATIS}) (IEL is the cell
index-number and II an integer between 1 and NVLSTS)\\
useful if ISTALA = 1}

\motcle{NPST}{I}{positive integer}{0}{O}{L3}
{number of iterations during which stationary volume statistics have
been cumulated\\
useful if ISTALA=1, ISTTIO=1 and if NSTIST is inferior or equal to
the current Lagrangian iteration\\
NPST is initialised and updated automatically by the code, its value is not to
be modified by the user}

\motcle{NPSTT}{I}{positive integer}{0}{O}{L3}
{number of iterations during which volume statistics have been
calculated (the potential iterations during which non-stationary
statistics have been calculated are counted in NPSTT)\\
useful if ISTALA=1\\
NPSTT is initialised and updated automatically by the code, its value is not to
be modified by the user}

\motcle{TSTAT}{R}{positive real number}{DTP}{O}{L3}
{if the volume statistics are calculated in a stationary way, TSTAT
represents the physical time during which the statistics have been cumulated\\
if the volume statistics are calculated in a non-stationary way,
then TSTAT=DTP\index{DTP} (it is the Lagrangian time step, because the
statistics are reset to zero at every iteration)\\
useful if ISTALA=1\\
TSTAT is initialised and updated automatically by the code, its value is not to
be modified by the user}

%==================================================
\subsubsection{Display of trajectories and particle movements}
%==================================================

\motcleb{IENSI1}{I}{0, 1}{0}{O}{L1}
{activation (=1) or not (=0) of the post-processing in trajectory mode\\
this option generates files allowing to display the trajectory of
some pre-selected particles in the \textit{EnSight6} format\\
always useful\\
{\em Warning: this option very expensive with regards to CPU time and may
generate very large files}}

\motcleb{IENSI2}{I}{0, 1}{0}{O}{L1}
{activation (=1) or not (=0) of the post-processing in movement
mode\\
This option generates files allowing to display the movement of
some pre-selected particles in the \textit{EnSight6} format\\
always useful\\
{\em Warning: this option very expensive with regards to CPU time and may
generate very large files}}

\motcleb{NBVIS}{I}{positive integer}{NLISTE}{O}{L1}
{number of particles selected for post-processing display in trajectory or
movement mode\\
NBVIS must be lower than NBPMAX and NLISTE (set to 500 in \texttt{lagpar.h} and
not to be modified)\\
useful if IENSI1 = 1 or IENSI2 = 1}

\motcleb{NVISLA}{I}{strictly positive integer}{1}{O}{L1}
{output period for the post-processing in trajectory or
movement mode\\
may be useful to diminish the size of the post-processing files\\
useful if IENSI1 = 1 or IENSI2 = 1}

\motcleb{LISTE}{IA}{positive integers}{between 1 and 500}{O}{L1}
{contains the index-numbers of the particles selected for the display in
trajectory or movement mode\\
useful if IENSI1 = 1 or IENSI2 = 1}

\motcleb{IVISV1}{I}{0, 1}{0}{O}{L1}
{associates (=1) or not (=0) the variable ``velocity of the locally
undisturbed fluid flow field'' with the display in trajectory or
movement mode\\
useful if IENSI1 = 1 or IENSI2 = 1}

\motcleb{IVISV2}{I}{0, 1}{0}{O}{L1}
{associates (=1) or not (=0) the variable ``particle velocity''
with the display in trajectory or movement mode\\
useful if IENSI1 = 1 ou IENSI2 = 1}

\motcleb{IVISTP}{I}{0, 1}{0}{O}{L1}
{associates (=1) or not (=0) the variable ``residence time''
with the display in trajectory or movement mode\\
useful if IENSI1 = 1 or IENSI2 = 1}

\motcleb{IVISDM}{I}{0, 1}{0}{O}{L1}
{associates (=1) or not (=0) the variable ``particle diameter''
with the display in trajectory or movement mode\\
useful if IENSI1 = 1 or IENSI2 = 1}

\motcleb{IVISTE}{I}{0, 1}{0}{O}{L1}
{associates (=1) or not (=0) the variable ``particle temperature''
with the display in trajectory or movement mode\\
useful if IENSI1 = 1 or IENSI2 = 1}

\motcleb{IVISMP}{I}{0, 1}{0}{O}{L1}
{associates (=1) or not (=0) the variable ``particle mass''
with the display in trajectory or movement mode\\
useful if IENSI1 = 1 or IENSI2 = 1}

\motcleb{IVISHP}{I}{0, 1}{0}{O}{L1}
{associates (=1) or not (=0) the variable ``temperature of the coal particles''
with the display in trajectory or movement mode\\
useful if IENSI1 = 1 or IENSI2 = 1, if and only if IPHYLA = 2}

\motcleb{IVISDK}{I}{0, 1}{0}{O}{L1}
{associates (=1) or not (=0) the variable ``shrinking core diameter of
the coal particles'' with the display in trajectory or movement mode\\
useful if IENSI1 = 1 or IENSI2 = 1, if and only if IPHYLA = 2}

\motcleb{IVISCH}{I}{0, 1}{0}{O}{L1}
{associates (=1) or not (=0) the variable ``mass of reactive coal of the
coal particles'' with the display in trajectory or movement mode\\
useful if IENSI1 = 1 or IENSI2 = 1, if and only if IPHYLA = 2}

\motcleb{IVISCK}{I}{0, 1}{0}{O}{L1}
{associates (=1) or not (=0) the variable ``mass of char of the
coal particles'' with the display in trajectory or movement mode\\
useful if IENSI1 = 1 or IENSI2 = 1, if and only if IPHYLA = 2}

%==================================================
\subsubsection{Display of the particle/boundary interactions and the statistics at the boundaries}
%==================================================

\motcleb{IENSI3}{I}{0, 1}{0}{C}{L1}
{activation (=1) or not (=0) of the recording of the particle/boundary
interactions in  PARBOR\index{PARBOR}, and of the calculation of the
statistics at the corresponding boundaries, for post-processing
(\textit{EnSight6} format)\\
By default, the statistics are non-stationary (reset to zero at every
Lagrangian iteration). They may be stationary if ISTTIO=1 ({\em i.e.}
calculation of a cumulated value over time, and then calculation of an
average over time or over the number of interactions with the boundary)\\
always useful}

\motcleb{NSTBOR}{I}{strictly positive integer}{1}{O}{L1}
{number of absolute Lagrangian iterations (including the restarts)
after which the statistics at the boundaries are considered stationary and are
averaged (over time or over the number of interactions)\\
If the number of absolute Lagrangian iterations is lower than NSTBOR, or if
ISTTIO=0, the statistics are reset to zero at every Lagrangian iteration
(non-stationary statistics)\\
useful if IENSI3=1 and ISTTIO=1}

\motcleb{SEUILF}{R}{positive real number}{0.D0}{O}{L1}
{every boundary face of the mesh undergoes a certain number of
interactions with particles, expressed in term of statistical weight
(sum of the statistical weights of all the particles which have
interacted with the boundary face). SEUILF is
the limit statistical weight value, below which the contribution of the
face is not taken into account in the
statistics at the boundaries for post-processing\\
useful if IENSI3=1}

\motcleb{INBRBD}{I}{0, 1}{1}{O}{L1}
{activation (=1) or not (=0) of the recording of the number of particle/boundary
interactions, and of the calculation of the associated boundary statistics.\\
INBRD = 1 is a compulsory condition to use the particulate average
IMOYBR = 2\\
the selection of the type of interactions that are to be recorded is specified
in the subroutine \texttt{uslabo}\\
useful if IENSI3=1}

\motcleb{IFLMBD}{I}{0, 1}{0}{O}{L1}
{activation (=1) or not (=0) of the recording of the particulate mass flow
related to the particle/boundary interactions, and of the calculation of
the associated boundary statistics\\
the selection of the type of interactions that are to be recorded is specified
in the subroutine \texttt{uslabo}\\
INBRD = 1 is a compulsory condition to use IFLMBD=1\\
useful if IENSI3=1 and INBRBD=1}


\motcleb{IANGBD}{I}{0, 1}{0}{O}{L1}
{activation (=1) or not (=0) of the recording of the angle between a
particle trajectory and a boundary face involved in a particle/boundary
interaction, and of the calculation of the associated boundary statistics\\
the selection of the type of interactions that are to be recorded is specified
in the subroutine \texttt{uslabo}\\
useful if IENSI3=1}

\motcleb{IVITBD}{I}{0, 1}{0}{O}{L1}
{activation (=1) or not (=0) of the recording of the velocity of a particle
involved in a particle/boundary interaction, and of the calculation of
the associated boundary statistics\\
the selection of the type of interactions that are to be recorded is specified
in the subroutine \texttt{uslabo}\\
useful if IENSI3=1}

\motcleb{IENCBD}{I}{0, 1}{0}{O}{L1}
{activation (=1) or not (=0) of the recording of the mass of coal particles
stuck to the wall due to fouling, on the boundary faces of the IENCRL
interaction type\\
useful if IENSI3=1, IPHYLA=2,
IENCRA=1, and if there is at least one boundary face of the IENCRL interaction
type}

\motcleb{NUSBOR}{I}{positive integer}{0}{O}{L1}
{number additional user data to record for the calculation
of additional boundary statistics in PARBOR\index{PARBOR}\\
useful if IENSI3=1}

\motcleb{NOMBRD}{CA}{string of less than 50 characters}{see
\texttt{uslag1}}{O}{L1}
{name of the boundary statistics, displayed in the listing
and the post-processing files\\
useful if IENSI3=1\\
{\em Warning: this name is also used to reference information in the restart file
\mbox{(ISUIST =1)}. If the name of a variable is changed between two
calculations, it will not be possible to read its value from the restart file}}

\motcleb{IMOYBR}{IA}{0, 1, 2}{0 , 1 or 2}{O}{L1}
{the recordings in PARBOR at every particle/boundary interaction are
cumulated values (possibly reset to zero at every iteration in the
non-stationary case). They must therefore be divided by a quantity to
get boundary statistics. The user can choose between two average types:\\
\hspace*{1.3cm} = 0: no average is applied to the recorded cumulated values\\
\hspace*{1.3cm} = 1: a time-average is calculated. The cumulated value
is divided by the physical duration in the case of stationary
averages (ISTTIO=1). The cumulated value is divided by the value of
the last time step in the case of non-stationary averages (ISTTIO=0),
and also in the case of stationary averages while the
absolute Lagrangian iteration number is inferior to NSTBOR\\
\hspace*{1.3cm} = 2: a particulate average is calculated. The cumulated
value is divided by the number of particle/boundary interactions (in term of
statistical weight) recorded in PARBOR(NFABOR,INBR). This average can only
be calculated when INBRBD=1. The average is calculated if the number
of interactions (in statistical weight) of the considered boundary face
is strictly higher than SEUILF, otherwise the average at the face is set
to zero\\
only the cumulated value is recorded in the restart file\\
useful if IENSI3=1}

\motcle{NPSTF}{I}{positive integer}{0}{O}{L3}
{number of iterations during which stationary boundary statistics have
been cumulated\\
useful if IENSI3=1, ISTTIO=1 and NSTBOR inferior or equal to the
current Lagrangian iteration\\
NPSTF is initialised and updated automatically by the code, its value is not to
be modified by the user}

\motcle{NPSTFT}{I}{positive integer}{0}{O}{L3}
{number of iterations during which boundary statistics have
been calculated
(the potential iterations during which non-stationary
statistics have been calculated are counted in NPSTFT)\\
useful if IENSI3=1\\
NPSTFT is initialised and updated automatically by the code, its value is not to
be modified by the user}

\motcle{TSTATP}{R}{positive real number}{DTP}{O}{L3}
{if the recording of the boundary statistics is stationary, TSTATP contains
the cumulated physical duration of the recording of the boundary statistics\\
if the recording of the boundary statisticss is non-stationary, then
TSTAT=DTP\index{DTP} (it is the Lagrangian time step, because the
statistics are reset to zero at every time step)\\
useful if IENSI3=1}
