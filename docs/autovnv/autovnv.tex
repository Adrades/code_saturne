%-------------------------------------------------------------------------------

% This file is part of Code_Saturne, a general-purpose CFD tool.
%
% Copyright (C) 1998-2012 EDF S.A.
%
% This program is free software; you can redistribute it and/or modify it under
% the terms of the GNU General Public License as published by the Free Software
% Foundation; either version 2 of the License, or (at your option) any later
% version.
%
% This program is distributed in the hope that it will be useful, but WITHOUT
% ANY WARRANTY; without even the implied warranty of MERCHANTABILITY or FITNESS
% FOR A PARTICULAR PURPOSE.  See the GNU General Public License for more
% details.
%
% You should have received a copy of the GNU General Public License along with
% this program; if not, write to the Free Software Foundation, Inc., 51 Franklin
% Street, Fifth Floor, Boston, MA 02110-1301, USA.

%-------------------------------------------------------------------------------

%%%%%%%%%%%%%%%%%%%%%%%%%%%%%%%%%%%%%%%%%%%%%%%%%%%%%%%%%%%%%%%%%%%%%%
% Short doc CS class corresponding to article
\documentclass[a4paper,10pt,twoside]{csshortdoc}
% MACROS SUPPLEMENTAIRES
\usepackage{csmacros}
\usepackage{longtable}
%
%%%%%%%%%%%%%%%%%%%%%%%%%%%%%%%%%%%%%%%%%%%%%%%%%%%%%%%%%%%%%%%%%%%%%%

%
%%%%%%%%%%%%%%%%%%%%%%%%%%%%%%%%%%%%%%%%%%%%%%%%%%%%%%%%%%%%%%%%%%%%%%
% PACKAGES ET COMMANDES POUR LE DOCUMENTS PDF ET LES HYPERLIENS
\hypersetup{%
  pdftitle = {CodeSaturne autovnv},
  pdfauthor = {MFEE},
  pdfpagemode = UseOutlines
}
\pdfinfo{/CreationDate (D:20110704000000-01 00 )}
%
% To have thumbnails upon opening the document under ACROREAD
% pdfpagemode = UseThumbs
%
%%%%%%%%%%%%%%%%%%%%%%%%%%%%%%%%%%%%%%%%%%%%%%%%%%%%%%%%%%%%%%%%%%%%%%
%
%%%%%%%%%%%%%%%%%%%%%%%%%%%%%%%%%%%%%%%%%%%%%%%%%%%%%%%%%%%%%%%%%%%%%%
% INFO POUR PAGES DE GARDES
\titreCS{\CS version~\verscs: autovnv tool}

\docassociesCS{}
\resumeCS{This document presents the autovnv tool. The aim of this script
is to drive \CS's cases automatically, to compare checkpoint files and
to display results.

\CS version~\verscs.

\begin{center}
\large{WORK IN PROGRESS}
\end{center}
}
%
%%%%%%%%%%%%%%%%%%%%%%%%%%%%%%%%%%%%%%%%%%%%%%%%%%%%%%%%%%%%%%%%%%%%%%
% DEBUT DU DOCUMENT
\begin{document}

\def\contentsname{\textbf{\normalsize TABLE OF CONTENTS}\pdfbookmark[1]{Table of
contents}{contents}}

\pdfbookmark[1]{Flyleaf}{pdg}
\large
\makepdgCS
\normalsize

\passepage

\begin{center}\begin{singlespace}
\tableofcontents
\end{singlespace}\end{center}
%
\section{Introduction}

\textsc{Autovnv} is a small framework to automate the launch of \CS computations
and do some operations on new results.

The script needs a directory of previous \CS cases which are candidates to be
duplicated. This directory is called \textbf{repository}. The duplication is
done in a new directory which is called the \textbf{destination}.

For each duplicated case, \textsc{Autovnv} is able to compile the user files, to
run the case, to compare the obtained checkpoint file with the previous one from
the \textbf{repository}, and to plot curves in order to illustrate the
computations.

For all these steps, \textsc{Autovnv} generate two reports, a global report
which summarizes the status of each case, and a detailed report which gives the
differences between the new results and the previous ones in the
\textbf{repository}, and display the defined plots.

In the \textbf{repository}, previous results of computations are required only
for checkpoint files comparison purpose. They can be also usfull, if the user
needs to run specific scripts.

\section{Installation and prerequisites}

\textsc{Autovnv} does not need a specific installation: the related files are
installed with the other Python scripts of \CS. Nevertheless, additional
prerequisites required are:
\begin{list}{$\bullet$}{}
\item \texttt{numpy},
\item \texttt{matplotlib},
\item \texttt{python-vtk}.
\end{list}

\section{Command line options}

The command line options can be found with the command: \texttt{code\_saturne
autovnv -h}.

\begin{list}{$\bullet$}{}
\item \texttt{-f FILE, --file=FILE}: gives the file of parameters for
\textsc{Autovnv}. This file is mandatory, and therefore this option must be
completed;
\item \texttt{-q, --quiet}: does not print status messages to stdout;
\item \texttt{-u, --update}: update installation pathes in scripts (i.e. \texttt{SaturneGUI} and
\texttt{runcase}) only in the repository, reinitialize xml files of parameters and compile;
\item \texttt{-r, --run}: runs all cases;
\item \texttt{-c, --compare}: compares chekpoint files between
\textbf{repository} and \textbf{destination};
\item \texttt{-p, --post}: postprocess results of computations;
\item \texttt{-m ADDRESS1 ADDRESS2 ..., --mail=ADDRESS1 ADDRESS2 ...}: addresses
for sending the reports.
\end{list}

\underline{Examples:}

\begin{list}{$\bullet$}{}
\item \texttt{code\_saturne autovnv -f sample.xml}: duplicates all cases from
the \textbf{repository} in the \textbf{destination}, compile all user files
and exits;
\item \texttt{code\_saturne autovnv -f sample.xml -r}: as above, and run all
cases if defined in \texttt{sample.xml};
\item \texttt{code\_saturne autovnv -f sample.xml -r -c}: as above, and compares
all new checkpoint files with those from the \textbf{repository} if defined
in \texttt{sample.xml};
\item \texttt{code\_saturne autovnv -f sample.xml -rcp}: as above, and plots
results if defined in \texttt{sample.xml};
\item \texttt{code\_saturne autovnv -f sample.xml -r -c -p -m
"dt@moulinsart.be dd@moulinsart.be"}: as above, and send the two
reports.
\item \texttt{code\_saturne autovnv -f sample.xml -c -p}: compares and plots
results in the \textbf{destination} already computed.
\end{list}

\underline{Note:}

The detailed report is generated only if the options \texttt{-c, --compare}
or \texttt{-p, --post} is present in the command line.

\section{File of parameters}

The file of parameters is a XML formatted ascii file.

\subsection{Begin and end of the file of parameters}

This example shows the four mandatory first lines of the file of parameters.

\small
\begin{verbatim}
<?xml version="1.0"?>
<autovnv>
    <repository>/home/dupond/codesaturne/MyRepository</repository>
    <destination>/home/dupond/codesaturne/MyDestination</destination>
\end{verbatim}
\normalsize

The third and fourth lines correspond to the definition of the
\textbf{repository} and \textbf{destination} directories.
Inside the markups \texttt{<repository>} and \texttt{<destination>} the user
must inform the related directories. If the \textbf{destination} does not exit,
the directory is created.

The last line of the file of parameters must be:

\small
\begin{verbatim}
</autovnv>
\end{verbatim}
\normalsize

\subsection{Case creation and compilation fo the user files}

When \textsc{Autovnv} is launched, the file of parameters is parsed in order to
known which studies and cases from the \textbf{repository} should be duplicated
in the \textbf{destination}. The selection is done with the markups
\texttt{<study>} and \texttt{<case>} as the following example:

\small
\begin{verbatim}
<?xml version="1.0"?>
<autovnv>
    <repository>/home/dupond/codesaturne/MyRepository</repository>
    <destination>/home/dupond/codesaturne/MyDestination</destination>

    <study label="MyStudy1" status="on">
        <case label="Grid1" status="on" compute="on" post="off"/>
        <case label="Grid2" status="off" compute="on" post="off"/>
    </study>
    <study label="MyStudy2" status="off">
        <case label="k-eps" status="on" compute="on" post="off"/>
        <case label="Rij-eps" status="on" compute="on" post="off"/>
    </study>
</autovnv>
\end{verbatim}
\normalsize

The attributes are:
\begin{list}{$\bullet$}{}
\item \texttt{label}: the name of the file of the script;
\item \texttt{status}: must be equal to \texttt{on} or \texttt{off},
activate or desactivate the markup;
\item \texttt{compute}: must be equal to \texttt{on} or \texttt{off},
activate or desactivate the computation of the case;
\item \texttt{post}: must nb equal to \texttt{on} or \texttt{off},
activate or desactivate the post-processing of the case.
\end{list}

All these attributes are mandatory.

With the attribute \texttt{status}, a single case or a complete study can be
switched off. In the above example, only the case \texttt{Grid1} of the study
\texttt{MyStudy1} is going to be created.

After the creation of the directories in the \textbf{destination}, for each
case, all user files are compiled. The \textsc{Autovnv} stops if a compilation
error occurs: neither computation nor comparison nor plot will be performed,
even if they are switched on.

\underline{Notes:}

\begin{list}{$\bullet$}{}
\item During the duplication, every files are copied, except mesh files, for
which a symbolic link is used.
\item During the duplication, if a file already exists in the
\textbf{destination}, this file is not overwritten by \textsc{Autovnv}.
\end{list}


\subsection{Run cases}

The computations are activated if the option \texttt{-r, --run} is present in
the command line.

All cases described in the file of parameters with the attribute
\texttt{compute="on"} are taken into account.

\small
\begin{verbatim}
<?xml version="1.0"?>
<autovnv>
    <repository>/home/dupond/codesaturne/MyRepository</repository>
    <destination>/home/dupond/codesaturne/MyDestination</destination>

    <study label="MyStudy1" status="on">
        <case label="Grid1" status="on" compute="on" post="off"/>
        <case label="Grid2" status="on" compute="off" post="off"/>
    </study>
    <study label="MyStudy2" status="on">
        <case label="k-eps" status="on" compute="on" post="off"/>
        <case label="Rij-eps" status="on" compute="on" post="off"/>
    </study>
</autovnv>
\end{verbatim}
\normalsize

After the computation, if no error occurs, the attribute \texttt{compute} is set
to \texttt{"off"} in the copy of the file of parameters in the
\textbf{destination}. It is allow to restart \textsc{Autovnv} without re-run
successfull previous computations.

\subsection{Compare checkpoint files}

The comparison is activated if the option \texttt{-c, --compare} is present in
the command line.

In order to compare two checkpoint files, markups \texttt{<compare>} have to
be added as a child of the condidered case. In the following exemple, a
checkpoint file comparison is switched on for the case \textit{Grid1} (for all
variables, with the default threshold), whereas no comparison is planed for
the case \textit{Grid2}. The comparison is done by the external
script \texttt{cs\_io\_dump} with the option \texttt{--diff}.

\small
\begin{verbatim}
<study label='MyStudy1' status='on'>
    <case label='Grid1' status='on' compute="on" post="off">
        <compare dest="" repo="" status="on"/>
    </case>
    <case label='Grid2' status='on' compute="off" post="off"/>
</study>
\end{verbatim}
\normalsize

The attributes are:
\begin{list}{$\bullet$}{}

\item \texttt{repo}: id of the results directory in the \textbf{repository} for
example \texttt{repo="20110704-1116"}, if there is a single results directory
in the \texttt{RESU} directory of the case, the id can be ommitted:
\texttt{repo=""};

\item \texttt{dest}: id of the results directory in the \textbf{destination}:
\begin{list}{$\rightarrow$}{}
\item if the id is not known already because the case has not yet run, just let
the attribute empty \texttt{dest=""}, the value will be updated after the run
step in the \textbf{destination} directory (see section \ref{sec:restart});
\item if \textsc{Autovnv} is restarted without the run step (with the command
line \texttt{code\_saturne autovnv -f sample.xml -c} for example), the id of
the results directory in the \textbf{destination} must be given (for example
\texttt{dest="20110706-1523"}), but if there is a single results directory in
the \texttt{RESU} directory of the case, the id can be ommitted:
\texttt{dest=""}, the id will be completed automatically;
\end{list}

\item \texttt{args}: additional options for the script \texttt{cs\_io\_dump}
\begin{list}{$\diamond$}{}
\item \texttt{--section}: name of a particular variable;
\item \texttt{--threshold}: real value above which a difference is considered
significant (default: $1e-30$ for all variables);
\end{list}
\item \texttt{status}: must be equal to \texttt{on} or \texttt{off}:
activate or desactivate the markup.
\end{list}

Only the attributes \texttt{repo}, \texttt{dest} and \texttt{status} are
mandatory.

Several comparisons with different options are permitted:
\small
\begin{verbatim}
<study label='MyStudy1' status='on'>
    <case label='Grid1' status='on' compute="on" post="off">
        <compare dest="" repo="" args="--section Pressure --threshold=1000" status="on"/>
        <compare dest="" repo="" args="--section VelocityX --threshold=1e-5" status="on"/>
        <compare dest="" repo="" args="--section VelocityY --threshold=1e-3" status="on"/>
    </case>
</study>
\end{verbatim}
\normalsize

Comparisons results will be sumarized in a table in the file
\texttt{report\_detailed.pdf}  (see \ref{sec:restart}):

\begin{center}
\begin{longtable}{|l|l|l|l|}
\hline
\textbf{Variable Name} &\textbf{Diff. Max} &\textbf{Diff. Mean} &\textbf{Threshold} \\
\hline
\hline
VelocityX &0.102701 &0.00307058 &1.0e-5 \\
\hline
VelocityY &0.364351 &0.00764912 &1.0e-3 \\
\hline
\end{longtable}
\end{center}

\subsection{Run external additional preprocessing scripts with arguments}

The markup \texttt{<prepro>} has to be added as a child of the condidered case.

\small
\begin{verbatim}
<study label='STUDY' status='on'>
    <case label='CASE1' status='on' compute="on" post="on">
        <prepro label="mesh_coarse.py" args="-n 1" status="on"/>
    </case>
</study>
\end{verbatim}
\normalsize

The attributes are:
\begin{list}{$\bullet$}{}
\item \texttt{label}: the name of the file of the considered script;
\item \texttt{status}: must be equal to \texttt{on} or \texttt{off}:
activate or desactivate the markup;
\item \texttt{args}: the command line to pass to the script.
\end{list}

Only the attributes \texttt{label} and \texttt{status} are mandatory.

Several calls of the same script or to different scripts are permitted:
\small
\begin{verbatim}
<study label="STUDY" status="on">
    <case label="CASE1" status="on" compute="on" post="on">
        <prepro label="script_pre1.py" args="-n 1" status="on"/>
        <prepro label="script_pre2.py" args="-n 2" status="on"/>
    </case>
</study>
\end{verbatim}
\normalsize

All preprocessing scripts must be in the \texttt{MESH} directory from
the current study in the \textbf{repository}.
The main objectif of running such external scripts is to create or modify
meshes.

The flollowing example shows how to create a mesh with a SALOME command file:
\small
\begin{verbatim}
<study label="STUDY" status="on">
    <case label="CASE1" status="on" compute="on" post="on">
        <prepro label="salome.sh" args="-t -u my_mesh.py" status="on"/>
    </case>
</study>
\end{verbatim}
\normalsize

with the script \texttt{salome.sh} (depending of the local installation of
SALOME):
\small
\begin{verbatim}
#!/bin/bash

export ROOT_SALOME=/home/salome/salome-640/Salome-V6_4_0-c7-v2
source /home/salome/salome-640/Salome-V6_4_0-c7-v2/salome_prerequisites_V6_4_0_appli.sh
source /home/salome/salome-640/Salome-V6_4_0-c7-v2/salome_modules_V6_4_0.sh

/home/salome/salome-640/appli_V6_4_0/bin/salome/runSalome $*
\end{verbatim}
\normalsize

and the script of SALOME commands \texttt{my\_mesh.py}:
\small
\begin{verbatim}
#!/usr/bin/env python
# -*- coding: utf-8 -*-

import geompy
import smesh

# create a box
box = geompy.MakeBox(0., 0., 0., 100., 200., 300.)
idbox = geompy.addToStudy(box, "box")

# create a mesh
tetra = smesh.Mesh(box, "MeshBox")

algo1D = tetra.Segment()
algo1D.NumberOfSegments(7)

algo2D = tetra.Triangle()
algo2D.MaxElementArea(800.)

algo3D = tetra.Tetrahedron(smesh.NETGEN)
algo3D.MaxElementVolume(900.)

# compute the mesh
tetra.Compute()

# export the mesh in a MED file
tetra.ExportMED("./my_mesh.med")
\end{verbatim}
\normalsize

\subsection{Run external additional postprocessing scripts with arguments}

The launch of external scripts is activated if the option \texttt{-p, --post}
is present in the command line.

The markup \texttt{<script>} has to be added as a child of the condidered case.

\small
\begin{verbatim}
<study label='STUDY' status='on'>
    <case label='CASE1' status='on' compute="on" post="on">
        <script label="script_post.py" args="-n 1" dest="" repo="20110216-2147" status="on"/>
    </case>
</study>
\end{verbatim}
\normalsize

The attributes are:
\begin{list}{$\bullet$}{}
\item \texttt{label}: the name of the file of the considered script;
\item \texttt{status}: must be equal to \texttt{on} or \texttt{off}:
activate or desactivate the markup;
\item \texttt{args}: the arguments to pass to the script;
\item \texttt{repo} and \texttt{dest}: id of the results directory in the
\textbf{repository} or in the \textbf{destination};
\begin{list}{$\rightarrow$}{}
\item if the id is not known already because the case has not yet run, just let
the attribute empty \texttt{dest=""}, the value will be updated after the run
step in the \textbf{destination} directory (see section \ref{sec:restart});
\item if there is a single results directory in the \texttt{RESU} directory
(either in the \textbf{repository} or in the \textbf{destination}) of the case,
the id can be ommitted: \texttt{repo=""} or \texttt{dest=""}, the id will be
completed automatically.
\end{list}
If attributes \texttt{repo} and \texttt{dest} exist, their associated value
will be passed to the script as arguments, with options \texttt{"-r"} and
\texttt{"-d"} respectively.
\end{list}

Only the attributes \texttt{label} and \texttt{status} are mandatory.

Several calls of the same script or to different scripts are permitted:
\small
\begin{verbatim}
<study label="STUDY" status="on">
    <case label="CASE1" status="on" compute="on" post="on">
        <script label="script_post.py" args="-n 1" status="on"/>
        <script label="script_post.py" args="-n 2" status="on"/>
        <script label="script_post.py" args="-n 3" status="on"/>
        <script label="another_script.py" status="on"/>
    </case>
</study>
\end{verbatim}
\normalsize

All postprocessing scripts must be in the \texttt{POST} directory from
the current study in the \textbf{repository}.
The main objectif of running external scripts is to create or modify
results in order to plot them.

Example of script, which searches printed informations in the listing,
note the function to process the passed command line arguments:
\small
\begin{verbatim}
#!/usr/bin/env python
# -*- coding: utf-8 -*-

import os, sys
import string
from optparse import OptionParser

def process_cmd_line(argv):
    """Processes the passed command line arguments."""
    parser = OptionParser(usage="usage: %prog [options]")

    parser.add_option("-r", "--repo", dest="repo", type="string",
                      help="Directory of the result in the repository")

    parser.add_option("-d", "--dest", dest="dest", type="string",
                      help="Directory of the result in the destination")

    (options, args) = parser.parse_args(argv)
    return options

def main(options):
    m = os.path.join(options.dest, "listing")
    f = open(m)
    lines = f.readlines()
    f.close()

    g = open(os.path.join(options.dest, "water_level.dat"), "w")
    g.write("# time,   h_sim,   h_th\n")
    for l in lines:
       if l.rfind("time, h_sim, h_th") == 0:
           d = l.split()
           g.write("%s  %s  %s\n" % (d[3], d[4], d[5]))
    g.close()

if __name__ == '__main__':
    options = process_cmd_line(sys.argv[1:])
    main(options)
\end{verbatim}
\normalsize

\subsection{Post-processing: curves}

The post-processing is activated if the option \texttt{-p, --post} is present
in the command line.

The following example shows the drawing of four curves (or plots, or 2D lines)
from two files of data (which have the same name \texttt{profile.dat}). There
are two subsets of curves (i.e. frames with axis and 2D lines), in a single
figure. The figure will be saved on the disk in a \textbf{pdf} (default)
or \textbf{png} format, in the \texttt{POST} directory of the related study
in the \textbf{destination}. Each drawing of a single curve is defined as a
markup child of a file of data inside a case. Subsets and figures are defined
as markup childs of \texttt{<study>}.

\small
\begin{verbatim}
<study label='Study' status='on'>
    <case label='Grid1' status='on' compute="off" post="on">
        <data file="profile.dat" dest="">
            <plot fig="1" xcol="1" ycol="2" legend="Grid level 1" fmt='r-s'/>
            <plot fig="2" xcol="1" ycol="3" legend="Grid level 1" fmt='r-s'/>
        </data>
    </case>
    <case label='Grid2' status='on' compute="off" post="on">
        <data file="profile.dat" dest="">
            <plot fig="1" xcol="1" ycol="2" legend="Grid level 2" fmt='b-p'/>
            <plot fig="2" xcol="1" ycol="3" legend="Grid level 2" fmt='b-p'/>
        </data>
    </case>
    <subplot id="1" legstatus='on' legpos ='0.95 0.95' ylabel="U ($m/s$)" xlabel="Time ($s$)"/>
    <subplot id="2" legstatus='on' legpos ='0.95 0.95' ylabel="U ($m/s$)" xlabel="Time ($s$)"/>
    <figure name="velocity" idlist="1 2" figsize="(4,5)" format="png"/>
</study>
\end{verbatim}
\normalsize

\subsubsection{Define curves}

The curves of computational data are build from data files. These data must be
ordered as column and the files should be in results directory in the
\texttt{RESU} directory (either in the \textbf{repository} or in the
\textbf{destination}). Commentaries are allowed in the file, the head of every
commentary line must start with character \texttt{\#}.

In the file of parameters, curves are defined with two markups:
\texttt{<data>} and \texttt{<plot>}:

\begin{list}{$\bullet$}{}
\item \texttt{<data>}: child of markup \texttt{<case>}, defines a file of data;
\begin{list}{$\rightarrow$}{}
\item \texttt{file}: name of the file of data
\item \texttt{repo} or \texttt{dest}: id of the results directory either in the
\textbf{repository} or in the \textbf{destination};
\begin{list}{$\Rightarrow$}{}
\item if the id is not known already because the case has not yet run, just let
the attribute empty \texttt{dest=""}, the value will be updated after the run
step in the \textbf{destination} directory (see section \ref{sec:restart});
\item if there is a single results directory in the \texttt{RESU} directory
(either in the \textbf{repository} or in the \textbf{destination}) of the case,
the id can be ommitted: \texttt{repo=""} or \texttt{dest=""}, the id will be
completed automatically.
\end{list}
\end{list}
The attribute \texttt{file} is mandatory, and either \texttt{repo} or
\texttt{dest} must be present (but not the both) even if it is empty.

\item \texttt{<plot>}: child of markup \texttt{<data>}, defines a single
curve; the attributes are:
\begin{list}{$\rightarrow$}{}
\item \texttt{fig}: id of the subset of curves (i.e. markup \texttt{<subplot>})
where the current curve should be plotted;
\item \texttt{xcol}: number of the column in the file of data for the abscisse;
\item \texttt{ycol}: number of the column in the file of data for the ordinate;
\item \texttt{legend}: add a label to a curve;
\item \texttt{fmt}: format of the line, composed from a symbol, a color and a
linestyle, for example \texttt{fmt="r--"} for a dashed red line;
\item \texttt{xplus}: real to add to all values of the column \texttt{xcol};
\item \texttt{yplus}: real to add to all values of the column \texttt{ycol};
\item \texttt{xfois}: real to multiply to all values of the column
\texttt{xcol};
\item \texttt{yfois}: real to multiply to all values of the column
\texttt{ycol};
\item \texttt{xerr}: draw horizontal error bar (see section \ref{sec:err});
\item \texttt{yerr}: draw vertical error bar (see section \ref{sec:err});
\item some standard options of 2D lines can be added, for example
\texttt{markevery="2"} or \texttt{markersize="3.5"}. These options
are summarized in the table \ref{table:curves}. Note that the options
which are string of characters must be overquoted likes this: 
\texttt{color="'g'"}.

\begin{table}[htbp]
\begin{center}
\begin{tabular}{|l|l|}
\hline
\textbf{Property} & \textbf{Value Type} \\
\hline
alpha & float (0.0 transparent through 1.0 opaque) \\
antialiased or aa & \texttt{True} or \texttt{False} \\
color or c & any matplotlib color \\
dash\_capstyle & \texttt{butt}; \texttt{round}; \texttt{projecting} \\
dash\_joinstyle & \texttt{miter}; \texttt{round}; \texttt{bevel} \\
dashes & sequence of on/off ink in points ex: \texttt{dashes="(5,3)"} \\
label & any string, same as legend\\
linestyle or ls &  \texttt{-}; \texttt{--}; \texttt{-.}; \texttt{:}; \texttt{steps}; ... \\
linewidth or lw & float value in points \\
marker &  \texttt{+}; \texttt{,}; \texttt{.}; \texttt{1}; \texttt{2}; \texttt{3}; \texttt{4}; ... \\
markeredgecolor or mec & any matplotlib color \\
markeredgewidth or mew & float value in points \\
markerfacecolor or mfc & any matplotlib color \\
markersize or ms & float \\
markevery & \texttt{None}; integer; (startind, stride) \\
solid\_capstyle & \texttt{butt}; \texttt{round}; \texttt{projecting} \\
solid\_joinstyle & \texttt{miter}; \texttt{round}; \texttt{bevel} \\
zorder & any number \\
\hline
\end{tabular}
\end{center}
\caption{Options authorized as attributes of the markup \texttt{plot}.}
\label{table:curves}
\end{table}

\end{list}
\end{list}

The attributes \texttt{fig} and \texttt{ycol} are mandatory.

Details on 2D lines properties can be found in the \texttt{matplotlib}
documentation. For more advanced options see section \ref{sec:raw}.

\subsubsection{Define subsets of curves}

A subset of curves is a frame with two axis, axis labels, legend, title and
drawing of curves inside. Such subset is called subplot in the nomenclature
of \texttt{matplotlib}.

\texttt{<subplot>}: child of markup \texttt{<study>}, defines a frame with
severals curves; the attributes are:
\begin{list}{$\rightarrow$}{}
\item \texttt{id}: id of the subplot, should be an integer;
\item \texttt{legstatus}: if \texttt{"on"} display the frame of the legend;
\item \texttt{legpos}: sequence of the relative coordinates of the center of
the legend, it is possible to draw the legend outside the axis;
\item \texttt{title}: set title of the subplot;
\item \texttt{xlabel}: set label for the x axis;
\item \texttt{ylabel}: set label for the y axis;
\item \texttt{xlim}: set range for the x axis;
\item \texttt{ylim}: set range for the y axis.
\end{list}

The attributes \texttt{fig} and \texttt{ycol} are mandatory.

For more advanced options see section \ref{sec:raw}.

\subsubsection{Define figures}

Figure is a compound of subset of curves.

\texttt{<figure>}: child of markup \texttt{<study>}, defines a pictures
with a layout of frames; the attributes are:
\begin{list}{$\rightarrow$}{}
\item \texttt{name}: name of the file to be written on the disk;
\item \texttt{idlist}: list of the subplot to be displayed in the figure;
\item \texttt{title}: add a title on the top of the figure;
\item \texttt{nbrow}: impose a number of row of the layout of the subplots;
\item \texttt{nbcol}: impose a number of column of the layout of the subplots;
\item \texttt{format}: format of the file to be written on the disk,
\texttt{"pdf"} (default) or \texttt{"png"} \footnote{Other format could
be choosen (eps, ps, svg,...), but the pdf generation with pdflatex will failed.};
\item standard options of figure can be added (table \ref{table:fig}),
for example \texttt{figsize="(3,4)"}.

\begin{table}[htbp]
\begin{center}
\begin{tabular}{|l|l|}
\hline
\textbf{Property} & \textbf{Value Type} \\
\hline
figsize   & width x height in inches; defaults to (4,4) \\
dpi       & resolution; defaults to 200 \\
\hline
\end{tabular}
\end{center}
\caption{Options authorized as attributes of the markup \texttt{figure}.}
\label{table:fig}
\end{table}

Details can be found in the matplotlib documentation.
For more advanced options see section \ref{sec:raw}.

\end{list}

The attributes \texttt{name} and \texttt{idlist} are mandatory.

\subsubsection{Experimental or analytical data}

A particular markup is provided for curves of experimental or analytical data:
\texttt{<measurement>}; the attributes are:
\begin{list}{$\rightarrow$}{}
\item \texttt{file}: name of the file to be read on the disk;
\item \texttt{path}: path of the directory where the file of data
is. the path could be ommitted (\texttt{path=""}), and in this case, the file
will be searched recursively in the directories of the considered study.
\end{list}

The attributes \texttt{file} and \texttt{path} are mandatory.

In order to draw curves of experimental or analytical data, the markup \texttt{<measurement>}
should be used with the markup \texttt{<plot>} as illustrated below:

\small
\begin{verbatim}
<study label='MyStudy' status='on'>
    <measurement file='exp1.dat' path=''>
            <plot fig='1' xcol='1' ycol='2' legend='U Experimental data'/>
            <plot fig='2' xcol='3' ycol='4' legend='V Experimental data'/>
    </measurement>
    <measurement file='exp2.dat' path =''>
            <plot fig='1' xcol='1' ycol='2' legend='U Experimental data'/>
            <plot fig='2' xcol='1' ycol='3' legend='V Experimental data'/>
    </measurement>
    <case label='Grid1' status='on' compute="off" post="on">
        <data file="profile.dat" dest="">
            <plot fig="1" xcol="1" ycol="2" legend="U computed" fmt='r-s'/>
            <plot fig="2" xcol="1" ycol="3" legend="V computed" fmt='b-s'/>
        </data>
    </case>
</study>
<subplot id="1" legstatus='on'  ylabel="U ($m/s$)" xlabel= "$r$ ($m$)" legpos ='0.05 0.1'/>
<subplot id="2" legstatus='off' ylabel="V ($m/s$)" xlabel= "$r$ ($m$)"/>
<figure name="MyFigure" idlist="1 2"  figsize="(4,4)" />
\end{verbatim}
\normalsize

\subsubsection{Curves with error bar}\label{sec:err}

In order to draw horizontal and vertical error bar, it is possible to
specify to the markup \texttt{<plot>} the attributes \texttt{xerr} and
\texttt{yerr} respectively. The value of theses attributes could be:
\begin{list}{$\bullet$}{}
\item a single column number where the average uncertainty is in the file of
data:
\small
\begin{verbatim}
<measurement file='axis.dat' path =''>
    <plot fig='1' xcol='1' ycol='3' legend='Experimental data' xerr='2' />
</measurement>
\end{verbatim}
\normalsize
\item a sequence of two column numbers where the minimum and the maximum
of the uncertainty are in the file of data:
\small
\begin{verbatim}
<data file='profile.dat' dest="">
    <plot fig='1' xcol='1' ycol='2' legend='computation' yerr='3 4' />
</data>
\end{verbatim}
\normalsize
\end{list}

\underline{Notes:}

The attributes \texttt{xerr} and \texttt{yerr} could not be both at the same
time in the markup \texttt{plot}. If horizontal and vertical bars has be
drawn together, repeat the markup \texttt{plot} like this:
\small
\begin{verbatim}
<measurement file='axis.dat' path =''>
    <plot fig='1' xcol='1' ycol='3' legend='Experimental data' xerr='2' />
    <plot fig='1' xcol='1' ycol='3' yerr='4' />
</measurement>
\end{verbatim}
\normalsize

\subsubsection{Monitoring points or probes}

A particular markup is provided for curves of probes data:
\texttt{<probes>}; the attributes are:

\begin{list}{$\bullet$}{}
\item \texttt{file}: name of the file to be read on the disk;
\item \texttt{fig}: id of the subset of curves (i.e. markup \texttt{<subplot>})
where the current curve should be plotted;
\item \texttt{dest}: id of the results directory in the \textbf{destination}:
\begin{list}{$\rightarrow$}{}
\item if the id is not known already because the case has not yet run, just let
the attribute empty \texttt{dest=""}, the value will be updated after the run
step in the \textbf{destination} directory (see section \ref{sec:restart});
\item if \textsc{Autovnv} is restarted without the run step (with the command
line \texttt{code\_saturne autovnv -f sample.xml -c} for example), the id of
the results directory in the \textbf{destination} must be given (for example
\texttt{dest="20110706-1523"}), but if there is a single results directory in
the \texttt{RESU} directory of the case, the id can be ommitted:
\texttt{dest=""}, the id will be completed automatically;
\end{list}
\end{list}

The attributes \texttt{file}, \texttt{fig} and \texttt{dest} are mandatory.

In order to draw curves of probes data, the markup \texttt{<probes>}
should be used as a child of a markup \texttt{<case>} as illustrated below:

\small
\begin{verbatim}
<study label='MyStudy' status='on'>
    <measurement file='exp1.dat' path=''>
        <plot fig='1' xcol='1' ycol='2' legend='U Experimental data'/>
    </measurement>
    <case label='Grid1' status='on' compute="off" post="on">
        <probes file="probes_U.dat" fig ="2" dest="">
        <data file="profile.dat" dest="">
            <plot fig="1" xcol="1" ycol="2" legend="U computed" fmt='r-s'/>
        </data>
    </case>
</study>
<subplot id="1" legstatus='on'  ylabel="U ($m/s$)" xlabel= "$r$ ($m$)" legpos ='0.05 0.1'/>
<subplot id="2" legstatus='on'  ylabel="U ($m/s$)" xlabel= "$time$ ($s$)" legpos ='0.05 0.1'/>
<figure title="Results" name="MyFigure" idlist="1"/>
<figure title="Grid1: probes for velocity"  name="MyProbes" idlist="2"/>
\end{verbatim}
\normalsize

\subsubsection{Matplotlib raw commands}\label{sec:raw}

The file of parameters allows to execute additional matplotlib commands (i.e
Python commands), for curves (2D lines), or subplot, or figure. For every object
drawn, \texttt{Autovnv} associate a name to this object that can be reused in
standard matplotlib commands. Therefore, childs markup \texttt{<plt\_command>}
could be added to \texttt{<plot>}, \texttt{<subplot>} or \texttt{<figure>}.

It is possible to add commands with \textbf{Matlab style} or \textbf{Python
style}. For the Matlab style, commands are called as methods of the module
\texttt{plt}, and for Python style commands or called as methods of the instance
of the graphical object.

Matlab style and Python style commands can be mixed.

\begin{list}{$\bullet$}{}

\item curves or 2D lines: when a curve is drawn, the associated name
are \texttt{line} and \texttt{lines} (with \texttt{line = lines[0]}).

\small
\begin{verbatim}
<plot fig="1" xcol="1" ycol="2" fmt='g^' legend="Simulated water level">
    <plt_command>plt.setp(line, color="blue")</plt_command>
    <plt_command>line.set_alpha(0.5)</plt_command>
</plot>
\end{verbatim}
\normalsize

\item subset of curves (subplot): for each subset, the associated name is \texttt{ax}:

\small
\begin{verbatim}
<subplot id="1" legend='Yes' legpos ='0.2 0.95'>
    <plt_command>plt.grid(True)</plt_command>
    <plt_command>plt.xlim(0, 20)</plt_command>
    <plt_command>ax.set_ylim(1, 3)</plt_command>
    <plt_command>plt.xlabel(r"Time ($s$)", fontsize=8)</plt_command>
    <plt_command>ax.set_ylabel(r"Level ($m$)", fontsize=8)</plt_command>
    <plt_command>for l in ax.xaxis.get_ticklabels(): l.set_fontsize(8)</plt_command>
    <plt_command>for l in ax.yaxis.get_ticklabels(): l.set_fontsize(8)</plt_command>
    <plt_command>plt.axis([-0.05, 1.6, 0.0, 0.15])</plt_command>
    <plt_command>plt.xticks([-3, -2, -1, 0, 1])</plt_command>
</subplot>
\end{verbatim}
\normalsize


\end{list}

\subsection{Post-processing: scalar map}

The post-processing is activated if the option \texttt{-p, --post} is present
in the command line.

\textsc{Autovnv} is able to draw colored maps from results on cut planes of the
computational domain. The results must be a set of ensight files. The following
example shows the drawing of a scalar map using the master file of result
\texttt{RESULTS.case}. The figure will be saved on the disk in a
\textbf{png} format, in the \texttt{POST} directory of the related study in the
\textbf{destination}. Each drawing of a single scalar map is defined as a markup
child of a file of data inside a case.

\small
\begin{verbatim}
<case label='Grid1' status='on' compute="on" post="on">
    <resu file="RESULTS.case" dest="">
        <scalar name="alpha1" variable="alpha1" normal="(0,0,1)">
            <scale legend="Water fraction" levels="10" />
            <title label="Grid level 1" fontsize="15"/>
            <axes/>
        </scalar>
        <scalar name="pressure" variable="Pressure"/>
    </resu>
</case>
\end{verbatim}
\normalsize

In the file of parameters, scalar map are defined with two markups:
\texttt{<resu>} and \texttt{<scalar>}. Three other markups can be used:
\texttt{<scale>}, \texttt{<title>} and \texttt{<axes>} as childs of
the markups \texttt{<scalar>}.

\begin{list}{$\bullet$}{}
\item \texttt{<resu>}: child of markup \texttt{<case>}, defines the master file
of set of ensight files;
\begin{list}{$\rightarrow$}{}
\item \texttt{file}: name of the master ensight file
\item \texttt{repo} or \texttt{dest}: id of the results directory either in the
\textbf{repository} or in the \textbf{destination};
\begin{list}{$\Rightarrow$}{}
\item if the id is not known already because the case has not yet run, just let
the attribute empty \texttt{dest=""}, the value will be updated after the run
step in the \textbf{destination} directory (see section \ref{sec:restart});
\item if there is a single results directory in the \texttt{RESU} directory
(either in the \textbf{repository} or in the \textbf{destination}) of the case,
the id can be ommitted: \texttt{repo=""} or \texttt{dest=""}, the id will be
completed automatically.
\end{list}
\end{list}
The attribute \texttt{file} is mandatory, and either \texttt{repo} or
\texttt{dest} must be present (but not the both) even if it is empty.

\item \texttt{<scalar>}: child of markup \texttt{<resu>}, defines a scalar map;
\begin{list}{$\rightarrow$}{}
\item \texttt{name}: name of the image file on the disk
\item \texttt{variable}: name of the variable to draw (must be present in the master ensight file)
\item \texttt{normal}: tuple to set the normal of the cut plane of the computational domain
\item \texttt{center}: tuple to set the center of the cut plane of the computational domain
\item \texttt{dilatation}: tuple to transform the view of the cut plane
\item \texttt{time-step}: select the physical time of the value to display
\item \texttt{size}: size of the png image
\item \texttt{zoom}: adjust the zoom for the point of view of the cut plane
\item \texttt{wireframe}: switch on or off the mapping on the grid\footnote{But the view might not be the true computational grid.}
\end{list}

See default values for these attributes in the table \ref{table:scalar}.

\begin{table}[htbp]
\begin{center}
\begin{tabular}{|l|l|}
\hline
\textbf{Attributes} & \textbf{Default values} \\
\hline
normal & (0,0,1) \\
center & (0,0,0) \\
dilatation & (1,1,1) \\
time-step & -1 \textit{i.e. the last record} \\
size & (500,400) \\
zoom & 1.0 \\
wireframe & \texttt{off} \\
\hline
\end{tabular}
\end{center}
\caption{Default attributes values of the markup \texttt{scalar}.}
\label{table:scalar}
\end{table}

The attributes \texttt{name} and \texttt{variable} are mandatory.


\item \texttt{<scale>}: child of markup \texttt{<scalar>}, defines a color bar;
\begin{list}{$\rightarrow$}{}
\item \texttt{color}: set a predefined palette in "hsv", "gray", "hot", "flag", "jet", "blue\_to\_yellow", "spring", "summer", "winter", "autumn"
\item \texttt{range}: tuple to set the range of values to display
\item \texttt{coord}: tuple to set the normalized coordinates of the color bar
\item \texttt{levels}: number of levels for the color bar
\item \texttt{height}: set the height  of the color bar
\item \texttt{width}: set the width  of the color bar
\item \texttt{position}: simple predefined position for the color bar in "North", "South", "West", or "East" 
\item \texttt{legend}: legend of the color bar
\item \texttt{fontsize}: font size for the legend
\item \texttt{format}: float format to display the value of the levels of the color bar
\end{list}

See default values for these attributes in the table \ref{table:scale}.

\begin{table}[htbp]
\begin{center}
\begin{tabular}{|l|l|}
\hline
\textbf{Attributes} & \textbf{Default values} \\
\hline
color & \textit{default vtk palette} \\
range & \textit{all values} \\
coord & (0.9,0.09) \\
levels & 10 \\
height & 0.8 \\
width & 0.1 \\
position & East \\
legend & \textit{no legend} \\
fontsize & 20 \\
format & \%4.4f \\
\hline
\end{tabular}
\end{center}
\caption{Default attributes values of the markup \texttt{scale}.}
\label{table:scale}
\end{table}

No attribute is mandatory.


\item \texttt{<title>}: child of markup \texttt{<scalar>}, defines a title for the scalarmap;
\begin{list}{$\rightarrow$}{}
\item \texttt{label}: text for the title
\item \texttt{fontsize}: font size for the title
\item \texttt{coord}: tuple to set the normalized coordinates of the title
\end{list}

See default values for these attributes in the table \ref{table:title}.

\begin{table}[htbp]
\begin{center}
\begin{tabular}{|l|l|}
\hline
\textbf{Attributes} & \textbf{Default values} \\
\hline
label & \textit{no title} \\
fontsize & 20 \\
coord & (0.5,0.99) \\
\hline
\end{tabular}
\end{center}
\caption{Default attributes values of the markup \texttt{title}.}
\label{table:title}
\end{table}

No attribute is mandatory.

\item \texttt{<axes>}:  child of markup \texttt{<scalar>}, defines axes bars around the scalarmap;
\begin{list}{$\rightarrow$}{}
\item \texttt{fontsize}: font size for the ticks of the axes
\item \texttt{format}: float format to display the ticks of the axes
\item \texttt{levels}: number of ticks of all axes
\item \texttt{xlabel}: text of the X axe
\item \texttt{ylabel}: text of the Y axe
\item \texttt{zlabel}: text of the Z axe
\end{list}

See default values for these attributes in the table \ref{table:axes}.

\begin{table}[htbp]
\begin{center}
\begin{tabular}{|l|l|}
\hline
\textbf{Attributes} & \textbf{Default values} \\
\hline
fontsize & 20 \\
format & \%6.3g \\
levels & 3 \\
xlabel & \textit{no text} \\
ylabel & \textit{no text} \\
zlabel & \textit{no text} \\
\hline
\end{tabular}
\end{center}
\caption{Default attributes values of the markup \texttt{axes}.}
\label{table:axes}
\end{table}

No attribute is mandatory.

\end{list}

\subsubsection{VTK raw commands}\label{sec:raw_vtk}

The file of parameters allows to execute additional vtk commands (i.e Python
commands). This possibility is mainly useful for interaction with actors and the
camera. For each actor drawn, \texttt{Autovnv} associate a name to this object
that can be reused in standard vtk commands. Therefore, childs markup
\texttt{<vtk\_command>} could be added to \texttt{<scalar>}.

The defined actors and camera are:
\begin{list}{$\rightarrow$}{}
\item \texttt{grid}: the scalar map,
\item \texttt{axes} (if \texttt{<axes>} exits),
\item \texttt{legend} (if \texttt{<scale>} exits),
\item \texttt{title} (if \texttt{<title>} exits),
\item \texttt{cam}: the parameters of the camera could be very important,
associated to the zoom to display correctly the scalar map.
\end{list}

\small
\begin{verbatim}
<resu file="RESULTS.case" dest="">
    <scalar name="alpha1_case1" variable="alpha1" normal="(0,1,0)" zoom="0.3">
        <scale legend="Water fraction" levels="5" fontsize="20"/>
        <title label="Grid level 1"/>
        <axes/>
        <vtk_command>cam.SetViewUp(1, 0, 1)</vtk_command>
        <vtk_command>axes.ZAxisVisibilityOff()</vtk_command>
        <vtk_command>legend.GetTitleTextProperty().BoldOn()</vtk_command>
        <vtk_command>legend.SetOrientationToHorizontal()</vtk_command>
    </scalar>
</resu>
\end{verbatim}
\normalsize

\subsection{Post-processing: input files}

The post-processing is activated if the option \texttt{-p, --post} is present
in the command line.

\textsc{Autovnv} is able to include files into the final detailed report. These
files must be in the directory of results either in the \textbf{destination} or
in the \textbf{repository}. The following example shows the inclusion of three
files: \texttt{performance.log} and \texttt{setup.log} from the
\textbf{destination}, and a \texttt{performance.log} from the \textbf{repository}:

\small
\begin{verbatim}
<case label='Grid1' status='on' compute="on" post="on">
    <input dest="" file="performance.log"/>
    <input dest="" file="setup.log"/>
    <input repo="" file="performance.log"/>
</case>
\end{verbatim}
\normalsize

In the file of parameters, input files are defined with markups \texttt{<input>}
as childs of a single markup \texttt{<case>}.
The attributes of \texttt{<input>} are:
\begin{list}{$\rightarrow$}{}
\item \texttt{file}: name of the file to be included
\item \texttt{repo} or \texttt{dest}: id of the results directory either in the
\textbf{repository} or in the \textbf{destination};
\begin{list}{$\Rightarrow$}{}
\item if the id is not known already because the case has not yet run, just let
the attribute empty \texttt{dest=""}, the value will be updated after the run
step in the \textbf{destination} directory (see section \ref{sec:restart});
\item if there is a single results directory in the \texttt{RESU} directory
(either in the \textbf{repository} or in the \textbf{destination}) of the case,
the id can be ommitted: \texttt{repo=""} or \texttt{dest=""}, the id will be
completed automatically.
\end{list}
\end{list}
The attribute \texttt{file} is mandatory, and either \texttt{repo} or
\texttt{dest} must be present (but not the both) even if it is empty.

\section{Output and restart}\label{sec:restart}

\textsc{Autovnv} produces several files in the \textbf{destination} directory:
\begin{list}{$\bullet$}{}
\item \texttt{report.txt}: standard output of the script;
\item \texttt{auto\_vnv.log}: log of the code and the \texttt{pdflatex}
compilation;
\item \texttt{report\_global.pdf}: summary of the compilation, run, comparison,
and plot steps;
\item \texttt{report\_detailed.pdf}: details the comparison and display the
plot;
\item \texttt{sample.xml}: udpated file of parameters, useful for restart the
script if an error occurs.
\end{list}

After the computation of a case, if no error occurs, the attribute
\texttt{compute} is set to \texttt{"off"} in the copy of the file of parameters
in the \textbf{destination}. It is allow a restart of \textsc{Autovnv} without
re-run successfull previous computations.
In the same manner, all empty attributes \texttt{repo=""} and \texttt{dest=""}
are completed in the udpated file of parameters.

\section{Tricks}\label{sec:tricks}
\begin{list}{$\bullet$}{}
\item How to comment markups in the file of parameter ?

The opening and closing signs for commantaries are \texttt{<!--} and
\texttt{-->}. In the following example, nothing from the study
\texttt{MyStudy2} will be read:
\small
\begin{verbatim}
<?xml version="1.0"?>
<autovnv>
    <repository>/home/dupond/codesaturne/MyRepository</repository>
    <destination>/home/dupond/codesaturne/MyDestination</destination>

    <study label="MyStudy1" status="on">
        <case label="Grid1" status="on" compute="on" post="on"/>
        <case label="Grid2" status="on" compute="off" post="on"/>
    </study>
    <!--
    <study label="MyStudy2" status="on">
        <case label="k-eps" status="on" compute="on" post="on"/>
        <case label="Rij-eps" status="on" compute="on" post="on"/>
    </study>
    -->
</autovnv>
\end{verbatim}
\normalsize

\item How to add text in a figure ?

It is possible to use raw commands:
\small
\begin{verbatim}
<subplot id='301' ylabel ='Location ($m$)' title='Before jet -0.885' legstatus='off'>
    <plt_command>plt.text(-4.2, 0.113, 'jet')</plt_command>
    <plt_command>plt.text(-4.6, 0.11, r'$\downarrow$', fontsize=15)</plt_command>
</subplot>
 \end{verbatim}
\normalsize

\item Adjust margins for layout of subplots in a figure.

You have to use the raw command \texttt{subplots\_adjust}:

\small
\begin{verbatim}
<subplot id="1" legend='Yes' legpos ='0.2 0.95'>
    <plt_command>plt.subplots_adjust(hspace=0.4, wspace=0.4, right=0.9,
                 left=0.15, bottom=0.2, top=0.9)</plt_command>
</subplot>
\end{verbatim}
\normalsize

\end{list}

%
\end{document}
%
%%%%%%%%%%%%%%%%%%%%%%%%%%%%%%%%%%%%%%%%%%%%%%%%%%%%%%%%%%%%%%%%%%%%%%
