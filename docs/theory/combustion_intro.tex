%-------------------------------------------------------------------------------

% This file is part of Code_Saturne, a general-purpose CFD tool.
%
% Copyright (C) 1998-2019 EDF S.A.
%
% This program is free software; you can redistribute it and/or modify it under
% the terms of the GNU General Public License as published by the Free Software
% Foundation; either version 2 of the License, or (at your option) any later
% version.
%
% This program is distributed in the hope that it will be useful, but WITHOUT
% ANY WARRANTY; without even the implied warranty of MERCHANTABILITY or FITNESS
% FOR A PARTICULAR PURPOSE.  See the GNU General Public License for more
% details.
%
% You should have received a copy of the GNU General Public License along with
% this program; if not, write to the Free Software Foundation, Inc., 51 Franklin
% Street, Fifth Floor, Boston, MA 02110-1301, USA.

%-------------------------------------------------------------------------------

%===============================================
\section{Introduction}\label{sec:combbase}
%===============================================

%===============================================
\subsection{Use \& call}
%===============================================


From a CFD point of view, combustion is a (sometimes very) complicated way to
determine $\rho$, the density.

Depending on the presence of a match or not, two solutions exist, known as
ignited and extinguished. From a numerical point of view, it means that these
two solutions have two attraction basin; the more representative the model, the more difficult the stabilisation of the combustion (may be difficult to
ignite).

However, combustion models needs few extra fields of scalar with regular
transport equations, some of them with a reactive or interfacial source term.

This version of \CS focuses on steady industrial combustion processes;
propagating fires are out of the present range (but  in the short coming release).

In \CS~ modelling of combustion is able to deal with gas phase combustion
(diffusion, premix, partial premix), and with solid or liquid finely dispersed
fuels (fixed and fluidised beds are out of range).

Combustion of condensed fuels involves one-way interfacial flux due to phenomena in the condensed phase (evaporation or pyrolisis) and reciprocal ones
(heterogeneous combustion and gasification). Many of the species injected in the gas phase are afterwards involved in
gas phase combustion.

That is the reason why many modules are similar for gas, coal and fuel oil
combustion modelling. Obviously, the thermodynamical description of gas
species is similar in every version as close as possible to the JANAF rules.

All models are developed in both adiabatic and unadiabatic (permeatic : heat
loss, \emph{e.g.} by radiation) version, beyond the standard (\fort{-1, 0, 1}), the rule to call models in \fort{usppmo} is:
%FIXME
\begin{eqnarray}
\fort{ippmod(index ; model)}  &=&     -1   ~~\quad \text{unused}                     \nonumber \\
\fort{ippmod(index ; model)}  &=& ~~   0  ~~~\quad \text{simplest adiabatic version} \nonumber \\
\fort{ippmod(index ; model)}  &=& ~~  1  ~~~\quad \text{simplest permeatic version}  \label{Eqs_00A}\\
\text{and possibly:}\qquad\qquad\qquad\qquad\qquad\qquad\qquad \nonumber & &\\
\fort{ippmod(index ; model)}  &=&  2.p    \qquad \text{p} ~\text{adiabatic version}  \nonumber \\
\fort{ippmod(index ; model)}  &=&  2.p+1 ~       \text{p} ~\text{permeatic version}  \nonumber
\end{eqnarray}


Every permeatic version involves the transport of enthalpy (one more variable).

%===============================================
\subsection{Gas combustion modelling}
%===============================================

Gas combustion is limited by disponibility (in the same fluid particle) of both
fuel and oxidizer and by kinetic effects (a lot of chemical reactions involved
can be described using an Arrhenius law with high activation energy). The mixing of mass (atoms) incoming with fuel and oxydizer is described by a mixture
fraction (mass fraction of matter incoming with fuel), this
variable is not affected by combustion.

A progress variable is used to describe the transformation of the mixture from
fuel and oxydant to products (carbon dioxyde and so on).
Combustion of gas is, traditionnaly, splitted in premix and diffusion regimes.

In premixed combustion process a first stage of mixing has been realised
(without blast ...) and the mixture is introduced in the boiler (or combustor
can). In common industrial conditions the combustion is mainly limited by the
mixing of fresh gases (frozen) and burnt gases (exhausted) resulting in the
inflammation of the first and their conversion to burnt ones ; so an assumption
of chemistry much faster than mixing (characteristic time for chemistry much
smaller than characteristic time for turbulent mixing) induces an intermittent
regime. The gas flow is constituted of totally fresh and totally burnt gases
(the flamme containing the gases during their transformation is extremely
thin). With this previous assumptions, Spalding \cite{Spalding:1971a} established the "Eddy
Break Up" model, which allows a complete description of the combustion process
with only one progress variable (mixture fraction is both constant
- in time - and homogeneous - in space).

In diffusion flames the fuel and the oxydant are introduced by, at least, two
inlets. In ordinary industrial conditions, their mixing is the main limitation
and the mixture fraction is enough to qualify a fluid particle, but in turbulent
flows a {\em P}robability {\em D}ensity {\em F}unction of the mixture fraction
is needed to qualify the thermodynamical state of the bulk. So, at least, both
the mean and the variance of the mixture fraction are needed (two variables) to
fit parameters of the pdf (the shape of whose is presumed).

Real world's chemistry is not so fast and, unfortunately, the mixing can not be
as homogeneous as wished. The main part of industrial combustion occurs in
partial premix regime. Partial premix occurs if mixing is not finished ( at
molecular level) when the mixture is introduced, or if air or fuel, are
staggered, or if a diffusion flame is blown off. For these situations, and
specifically for lean premix gas turbines \cite{Libby:2000a} developed a model allowing a
description of both mixing and chemical limitations. A collaboration between the
LCD Poitiers \cite{Ribert:2004a} and EDF R\&D has produced a simpler version of their
algorithm. Not only the mean and the variance of both mixture fraction and
progress variable are needed but also their covariance (five variables).


%===============================================
\subsection{Two-phase combustion modelling}
%===============================================

Coal combustion is the main way to produce electricity in the world. Heavy fuel
oil combustion have been hugely used to produce electrical energy.
Biomass is a promising fuel to be used alone or in blend.

Advanced combustion process may include exhaust gases recycling, pure oxygen or
steam injection, so this release of \CS ~takes into account
three oxidizers (tracked by three mixture fractions).

Coal is a natural product with a very complex composition. During the industrial
process of milling, the raw coal is broken in tiny particles of different
sizes. After its introduction in the boiler, coal particles undergoes drying,
devolatilisation (heating of coal turn it in a mixture of char and gases),
heterogenous combustion (of char by oxygen in carbon monoxide), gasification (of
char by carbon
dioxide or by water steam in carbon monoxide), leaving ash particles.

The description of fuel evaporation is done with respect to its heaviness :
after a minimum temperature is reached, the gain of enthalpy is splitted between
heating and evaporation. This way the evaporation takes place on a range of
temperature (which can be large). The total evaporation is common for light
(domestic) oil but impossible for heavy ones : at high temperature, during the
last evaporation, a craking reaction appears : so a particle similar to the char
is leaved. The heterogeneous oxydation of this char particle is very similar to
coal char ones.

Each of these phenomena are taken into account for some classes of particles : a
solid class is caracterised by a coal (it is useful to burn mixture of coals
with differents ranks or mixture of coal with biomass ...) and an initial
diameter, for heavy fuel oil, liquid classes refer to initial diameter (neither
possibility of blending after injection nor cofiring with oil and coal). \CS~
computes the number, the mass and the enthalpy for each class of particles by
unit of mass of mixture; allowing the determination of local diameter and
temperature (for each class;
\emph{e.g.} the finest will be be heated the fastest).

The main assumption is to solve only one velocity (and pressure) field: it
means that the discrepancy of velocity between coal particles and
gases is assumed to be negligible.

Due to the radiation, evaporation and heterogeneous combustion, temperature can
be different for gas and different size particles : so the specific enthalpy of
each particle class is solved.

The description of coal pyrolysis proposed by \cite{Kobayashi:1976} is used, leading to two
source terms for light and heavy volatile matters (the moderate temperature
reaction produces gases with low molecular mass, the high temperature reaction
produces heavier gases and less char) represented by two passive scalars :
mixture fractions.  The description of the heterogeneous reaction (which produce
carbon monoxide) produces a source term for the carbon: the corresponding
mixture fraction is bounded far below one (the carbon can't be free, it is
always in carbon monoxide form, mixed with nitrogen or other).

The retained model for the gas phase combustion is the assumption of diffusion
flammelets surrounding particle (for a single paticvle or a cloud), this
diffusion flame establish itself between a mixing of the previous gaseous fuels
issued from fast phenomenon (pyrolysis or fuel evaporation) mixed in a local
mean fuel and the mixing of oxidizers, water vapor (issued from drying) and
carbon monoxide issued from slow phenomenon (heterogeneous oxydation and
gasification of char). The PDF diffusion approcah is used to describe the
conversion of hydrocarbon to carbon monoxide (hydrocarbon conversion is assumed
fast vs. mixing); the further conversion of carbon monoxide to carbon dioxyde
was (in previous release, still existing for fast first evaluation of carbon
dioxide usefull to initialize the kinetic model) ruled by mixing or is (now
recommended for better prediction of carbon monoxide at outlet and corrosion
risks) kineticaly ruled with respect to the mean mass fraction and temperature
(reach of equilibrium assumed slow vs. mixing). Special attention is paid to
pollutant formation (conversion of $H_{2}S$ to $SO_{2}$ involved in soot
agglomeration, NOx formation).

\newpage
