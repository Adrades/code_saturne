%-------------------------------------------------------------------------------

% This file is part of Code_Saturne, a general-purpose CFD tool.
%
% Copyright (C) 1998-2015 EDF S.A.
%
% This program is free software; you can redistribute it and/or modify it under
% the terms of the GNU General Public License as published by the Free Software
% Foundation; either version 2 of the License, or (at your option) any later
% version.
%
% This program is distributed in the hope that it will be useful, but WITHOUT
% ANY WARRANTY; without even the implied warranty of MERCHANTABILITY or FITNESS
% FOR A PARTICULAR PURPOSE.  See the GNU General Public License for more
% details.
%
% You should have received a copy of the GNU General Public License along with
% this program; if not, write to the Free Software Foundation, Inc., 51 Franklin
% Street, Fifth Floor, Boston, MA 02110-1301, USA.

%-------------------------------------------------------------------------------

\programme{cfener}
%
\vspace{1cm}
%%%%%%%%%%%%%%%%%%%%%%%%%%%%%%%%%%
%%%%%%%%%%%%%%%%%%%%%%%%%%%%%%%%%%
\section*{Fonction}
%%%%%%%%%%%%%%%%%%%%%%%%%%%%%%%%%%
%%%%%%%%%%%%%%%%%%%%%%%%%%%%%%%%%%

Pour les notations et l'algorithme dans son ensemble,
on se reportera \`a \fort{cfbase}.

Apr\`es masse (acoustique) et quantit\'e de mouvement,
on consid\`ere un dernier pas fractionnaire (de $t^{**}$ \`a $t^{***}$)
au cours duquel seule varie l'\'energie totale $E = \rho e$.

\begin{equation}\label{Cfbl_Cfener_eq_energie_cfener}
\left\{\begin{array}{l}
\rho^{***}=\rho^{**}=\rho^{n+1}\\
\\
\vect{Q}^{***}=\vect{Q}^{**}=\vect{Q}^{n+1}\\
\\
\displaystyle\frac{\partial \rho e}{\partial t}
+ \divs\left( \vect{Q}_{ac} \left(e+\displaystyle\frac{P}{\rho}\right) \right)
= \rho\vect{f}_v\cdot\vect{u}
+ \divs(\tens{\Sigma}^v \vect{u})
- \divs{\,\vect{\Phi}_s} + \rho\Phi_v
\end{array}\right.
\end{equation}

Pour conserver la positivit\'e de l'\'energie, il est indispensable ici,
comme pour les scalaires, d'utiliser le flux de masse convectif acoustique
$\vect{Q}_{ac}^{n+1}$ compatible avec l'\'equation de la masse.
De plus, pour obtenir des propri\'et\'es de positivit\'e sur les scalaires,
un sch\'ema upwind pour le terme convectif doit \^etre utilis\'e
(mais les termes sources introduisent des contraintes suppl\'ementaires
qui peuvent \^etre pr\'epond\'erantes et g\^enantes).

\vspace{0.5cm}

\`A la fin de cette \'etape, on actualise �ventuellement
(mais par d�faut non)
une deuxi\`eme et derni\`ere fois la pression
en utilisant la loi d'\'etat pour obtenir la pression finale~:
\begin{equation}
\displaystyle P^{n+1}=P(\rho^{n+1},\varepsilon^{n+1})
\end{equation}

%%%%%%%%%%%%%%%%%%%%%%%%%%%%%%%%%
%%%%%%%%%%%%%%%%%%%%%%%%%%%%%%%%%%
\section*{Discr\'etisation}
%%%%%%%%%%%%%%%%%%%%%%%%%%%%%%%%%%
%%%%%%%%%%%%%%%%%%%%%%%%%%%%%%%%%%

%---------------------------------
\subsection*{Discr\'etisation en temps}
%---------------------------------

La mod\'elisation des flux de chaleur choisie jusqu'\`a pr\'esent est de la
forme $-\divs(\,\vect{\Phi}_s) = \divs(\lambda \gradv{T})$.

Pour faire appara\^itre un terme diffusif stabilisant dans la
matrice de r\'esolution, on cherche \`a exprimer le flux diffusif de chaleur
($-\divs(\,\vect{\Phi}_s)$)
en fonction de la variable r\'esolue (l'\'energie totale).

Avec $\varepsilon_{sup}(P,\rho)$
d\'ependant de la loi d'\'etat, on exprime l'\'energie totale de la fa\c con suivante~:
\begin{equation}
e = \varepsilon + \frac{1}{2} u^2
= (C_v T + \varepsilon_{sup}) + \frac{1}{2} u^2
\end{equation}

En supposant $C_v$ constant\footnote{Pour $C_v$ non constant, les
d\'eveloppements restent \`a faire~: on pourra se
reporter \`a  P. Mathon, F. Archambeau, J.-M. H�rard : "Implantation d'un
algorithme compressible dans \CS", HI-83/03/016/A}, on a alors~:
\begin{equation}\label{Cfbl_Cfener_eq_flux_thermique_cfener}
-\divs(\,\vect{\Phi}_s)
= \divs(K \gradv(e - \frac{1}{2} u^2 - \varepsilon_{sup}))\qquad
\text{avec } K=\lambda / C_v
\end{equation}

Lorsqu'un mod\`ele de turbulence est activ\'e, on conserve la m\^eme forme
de mod\'elisation pour les flux thermiques et $K$ int\`egre alors
la diffusivit\'e turbulente. On pourra se reporter \`a
la documentation de \fort{cfxtcl} \`a ce sujet.

Avec la formulation~(\ref{Cfbl_Cfener_eq_flux_thermique_cfener}),
on peut donc impliciter le terme en $\gradv{e}$.

\bigskip
De plus, puisque la vitesse a
d\'ej\`a \'et\'e r\'esolue, on implicite \'egalement le terme en
$\gradv{\frac{1}{2} u^2}$. L'exposant $n+\frac{1}{2}$ de $\varepsilon_{sup}$
indique que l'implicitation de ce terme est partielle (elle d\'epend de la forme
de la loi d'\'etat).

Par ailleurs, on implicite le terme de convection, le terme de puissance
des forces volumiques, �ventuellement le terme de puissance des forces de
pression (suivant la valeur de \var{IGRDPP}, on utilise la pr\'ediction de
pression obtenue apr\`es r\'esolution de l'\'equation portant sur la masse
volumique ou bien la pression du pas de temps pr�c�dent)
et le terme de puissance des forces visqueuses. On implicite le terme de puissance
volumique en utilisant $\rho^{n+1}$.

\bigskip
On obtient alors l'\'equation discr\`ete portant sur $e$~:
\begin{equation}\label{Cfbl_Cfener_eq_energie_totale_cfener}
\begin{array}{l}
\displaystyle\frac{(\rho e)^{n+1} - (\rho e)^n}{\Delta t^n}
+ \divs(\vect{Q}_{ac}^{n+1} e^{n+1}) - \divs(K^n \gradv{e^{n+1}})
= \rho^{n+1} \vect{f}_v \cdot \vect{u}^{n+1}
- \divs(\vect{Q}_{ac}^{n+1} \displaystyle\frac{\widetilde{P}}{\rho^{n+1}} )\\
\text{\ \ \ \ }+ \divs((\tens{\Sigma}^v)^{n+1} \vect{u}^{n+1})
- \divs(K^n \gradv(\frac{1}{2} (u^2)^{n+1}
+ \varepsilon_{sup}^{n+\frac{1}{2}}))
+ \rho^{n+1}\Phi_v\\
\end{array}
\end{equation}
avec $\widetilde{P}=P^{Pred}\text{ ou }P^n$ suivant la valeur de \var{IGRDPP}
($P^n$ par d�faut).

En pratique, dans \CS, on r\'esout cette \'equation en faisant appara\^itre \`a
gauche l'\'ecart $e^{n+1} - e^n$. Pour cela, on \'ecrit la d\'eriv\'ee
en temps discr\`ete sous la forme suivante~:

\begin{equation}
\begin{array}{ll}
\displaystyle
\frac{(\rho e)^{n+1} - (\rho e)^n}{\Delta t^n}
& =
\displaystyle
\frac{\rho^{n+1}\, e^{n+1} - \rho^n\, e^n}{\Delta t^n}\\
& =
\displaystyle
\frac{\rho^{n}\, e^{n+1} - \rho^n\, e^n}{\Delta t^n}+
\frac{\rho^{n+1}\, e^{n+1} - \rho^n\, e^{n+1}}{\Delta t^n}\\
& =
\displaystyle
\frac{\rho^{n}}{\Delta t^n}\left(e^{n+1} - e^n\right)+
e^{n+1}\frac{\rho^{n+1} - \rho^n}{\Delta t^n}
\end{array}
\end{equation}

et l'on utilise l'\'equation de la masse discr\`ete pour \'ecrire~:
\begin{equation}
\displaystyle
\frac{(\rho e)^{n+1} - (\rho e)^n}{\Delta t^n}
=
\frac{\rho^{n}}{\Delta t^n}\left(e^{n+1} - e^n\right)-
e^{n+1}\dive\,\vect{Q}_{ac}^{n+1}
\end{equation}



%---------------------------------
\subsection*{Discr\'etisation en espace}
%---------------------------------


%.................................
\subsubsection*{Introduction}
%.................................

On int\`egre l'\'equation (\ref{Cfbl_Cfener_eq_energie_totale_cfener})
sur la cellule $i$ de volume $\Omega_i$ et l'on proc\`ede comme
pour l'\'equation de la masse et de la quantit\'e de mouvement.

On obtient alors l'\'equation discr\`ete
suivante~:
\begin{equation}\label{Cfbl_Cfener_eq_energie_totale_discrete_cfener}
\begin{array}{l}
\displaystyle\frac{\Omega_i}{\Delta t^n}
(\rho_i^{n+1} e_i^{n+1}-\rho_i^n e_i^n)
+ \displaystyle\sum\limits_{j\in V(i)}
\left(e^{n+1} \vect{Q}_{ac}^{n+1}\right)_{ij} \cdot \vect{S}_{ij}
- \displaystyle\sum\limits_{j\in V(i)}
\left(K^n\gradv(e^{n+1})\right)_{ij}\cdot\vect{S}_{ij}\\
\\
\text{\ \ \ \ } = \Omega_i\rho_i^{n+1} {\vect{f}_v}_i \cdot \vect{u}_i^{n+1}
- \displaystyle\sum\limits_{j\in V(i)}
\left(\displaystyle\frac{P^{Pred}}{\rho^{n+1}}\
\vect{Q}_{ac}^{n+1}\right)_{ij} \cdot \vect{S}_{ij}
+ \displaystyle\sum\limits_{j\in V(i)}
\left((\tens{\Sigma}^v)^{n+1} \vect{u}^{n+1} \right)_{ij}\cdot \vect{S}_{ij}\\
\\
\text{\ \ \ \ } - \displaystyle\sum\limits_{j\in V(i)}
\left(K^n \gradv\left(\frac{1}{2}(u^2)^{n+1}
+ \varepsilon_{sup}^{n+\frac{1}{2}}\right)\right)_{ij}\cdot\vect{S}_{ij}
+ \Omega_i\rho_i^{n+1}{\Phi_v}_i\\
\end{array}
\end{equation}


%.................................
\subsubsection*{Discr\'etisation de la partie ``convective''}
%.................................

La valeur \`a la face s'\'ecrit~:
\begin{equation}
\left(e^{n+1} \vect{Q}_{ac}^{n+1}\right)_{ij} \cdot \vect{S}_{ij}
= e_{ij}^{n+1}(\vect{Q}_{ac}^{n+1})_{ij} \cdot \vect{S}_{ij}
\end{equation}
avec un d\'ecentrement sur la valeur de $e^{n+1}$ aux faces~:
\begin{equation}
\begin{array}{lllll}
e_{ij}^{n+1}
& = & e_i^{n+1}
& \text{si\ } & (\vect{Q}_{ac}^{n+1})_{ij} \cdot \vect{S}_{ij} \geqslant 0 \\
& = & e_j^{n+1}
& \text{si\ } & (\vect{Q}_{ac}^{n+1})_{ij} \cdot \vect{S}_{ij} < 0 \\
\end{array}
\end{equation}
que l'on peut noter~:
\begin{equation}
 e_{ij}^{n+1}
 = \beta_{ij}e_i^{n+1} + (1-\beta_{ij})e_j^{n+1}
\end{equation}
avec
\begin{equation}
\left\{\begin{array}{lll}
\beta_{ij} = 1 & \text{si\ }
& (\vect{Q}_{ac}^{n+1})_{ij} \cdot \vect{S}_{ij} \geqslant 0 \\
\beta_{ij} = 0 & \text{si\ }
& (\vect{Q}_{ac}^{n+1})_{ij} \cdot \vect{S}_{ij} < 0 \\
\end{array}\right.
\end{equation}


%.................................
\subsubsection*{Discr\'etisation de la partie ``diffusive''}
%.................................

La valeur \`a la face s'\'ecrit~:
\begin{equation}
\begin{array}{c}
\left(K^n\gradv(e^{n+1})\right)_{ij}\cdot\vect{S}_{ij}
= K_{ij}^n
\displaystyle \left( \frac{\partial e}{\partial n} \right)^{n+1}_{ij}S_{ij}\\
\text{et}\\
\left(K^n \gradv\left(\frac{1}{2}(u^2)^{n+1}
+ \varepsilon_{sup}^{n+\frac{1}{2}}\right)\right)_{ij}\cdot\vect{S}_{ij}
= K_{ij}^n
\displaystyle \left( \frac{\partial \left(\frac{1}{2} u^2
+ \varepsilon_{sup}\right)}{\partial n} \right)^{n+\frac{1}{2}}_{ij}S_{ij}
\end{array}
\end{equation}
avec une interpolation lin\'eaire pour
$K^n$ aux faces (et en pratique, $\alpha_{ij}=\frac{1}{2}$)~:
\begin{equation}
K_{ij}^n
= \alpha_{ij}K_{i}^n+(1-\alpha_{ij})K_{j}^n
\end{equation}
et un sch\'ema centr\'e avec reconstruction pour le gradient normal aux faces~:
\begin{equation}
\displaystyle \left( \frac{\partial e}{\partial n} \right)^{n+1}_{ij}
= \displaystyle\frac{e_{J'}^{n+1} - e_{I'}^{n+1}}{\overline{I'J'}}
\quad \text{et} \quad
\displaystyle \left( \frac{\partial \left(\frac{1}{2} u^2
+ \varepsilon_{sup}\right)}{\partial n} \right)^{n+\frac{1}{2}}_{ij}
= \displaystyle\frac{(\frac{1}{2} u^2
+ \varepsilon_{sup})_{J'}^{n+\frac{1}{2}} - (\frac{1}{2} u^2
+ \varepsilon_{sup})_{I'}^{n+\frac{1}{2}}}{\overline{I'J'}}
\end{equation}



%.................................
\subsubsection*{Discr\'etisation de la puissance des forces de pression}
%.................................

Ce terme
est issu du terme convectif, on le discr\'etise donc de la m\^eme fa\c con.

\begin{equation}
\left(\displaystyle\frac{\widetilde{P}}{\rho^{n+1}}\
\vect{Q}_{ac}^{n+1}\right)_{ij} \cdot \vect{S}_{ij}
= \left(\displaystyle\frac{\widetilde{P}}{\rho^{n+1}}\right)_{ij}
(\vect{Q}_{ac}^{n+1})_{ij} \cdot \vect{S}_{ij}
\end{equation}

avec un d\'ecentrement sur la valeur de
$\displaystyle\frac{P}{\rho}$ aux faces~:
\begin{equation}
\begin{array}{lll}
\left(\displaystyle\frac{\widetilde{P}}{\rho^{n+1}}\right)_{ij}
 = \beta_{ij}\displaystyle\frac{\widetilde{P}_i}{\rho^{n+1}_i}
+ (1-\beta_{ij})\displaystyle\frac{\widetilde{P}_j}{\rho^{n+1}_j}
& \text{avec}
& \left\{\begin{array}{lll}
\beta_{ij} = 1 & \text{si\ }
& (\vect{Q}_{ac}^{n+1})_{ij} \cdot \vect{S}_{ij} \geqslant 0 \\
\beta_{ij} = 0 & \text{si\ }
& (\vect{Q}_{ac}^{n+1})_{ij} \cdot \vect{S}_{ij} < 0 \\
\end{array}\right.
\end{array}
\end{equation}



%.................................
\subsubsection*{Discr\'etisation de la puissance des forces visqueuses}
%.................................

On calcule les termes dans les cellules puis on utilise une
interpolation lin\'eaire (on utilise
$\alpha_{ij}=\frac{1}{2}$ dans la relation ci-dessous)~:
\begin{equation}
\left((\tens{\Sigma}^v)^{n+1} \vect{u}^{n+1} \right)_{ij}\cdot \vect{S}_{ij}
= \left\{\alpha_{ij} \left((\tens{\Sigma}^v)^{n+1} \vect{u}^{n+1}\right)_i
+ (1-\alpha_{ij}) \left((\tens{\Sigma}^v)^{n+1} \vect{u}^{n+1}\right)_j
\right\} \cdot \vect{S}_{ij}
\end{equation}


%.................................
\subsubsection*{Remarques}
%.................................


Les termes ``convectifs'' associ\'es \`a
$\displaystyle\dive\left(\left(e^{n+1}+\frac{\widetilde{P}}{\rho^{n+1}}\right)\,
\vect{Q}_{ac}^{n+1}\right)$ sont calcul\'es avec un d\'ecentrement amont
(consistant, d'ordre 1 en espace). Les valeurs utilis\'ees sont bien prises au
centre de la cellule amont ($e_i$, $P_i$, $\rho_i$) et non pas au projet\'e $I'$
du centre de la cellule sur la normale \`a la face passant par son centre de
gravit\'e (sur un cas test en triangles, l'utilisation de $P_I'$ et de $\rho_I'$
pour le terme de transport de pression a conduit \`a un r\'esultat
insatisfaisant, mais des corrections ont \'et\'e apport\'ees aux sources depuis
et il serait utile de v\'erifier que cette conclusion n'est pas remise en question).

Les termes diffusifs associ\'es \`a
$\displaystyle\dive\left(K\,\grad\left(e+\frac{1}{2} u^2 +
\varepsilon_{sup}\right)\right)$ sont calcul\'es en utilisant des valeurs aux
faces reconstruites pour s'assurer de la consistance du sch\'ema.

%%%%%%%%%%%%%%%%%%%%%%%%%%%%%%%%%%
%%%%%%%%%%%%%%%%%%%%%%%%%%%%%%%%%%
\section*{Mise en \oe uvre}
%%%%%%%%%%%%%%%%%%%%%%%%%%%%%%%%%%
%%%%%%%%%%%%%%%%%%%%%%%%%%%%%%%%%%


Apr\`es une \'etape de gestion de la m\'emoire (\fort{memcfe}), on calcule les
diff\'erents termes sources (au centre des cellules)~:
\begin{itemize}
\item source volumique de chaleur (\fort{ustssc}),
\item source associ\'ee aux sources de masse (\fort{catsma}),
\item source associ\'ee \`a l'accumulation de masse $\dive\,\vect{Q}_{ac}$ (directement dans \fort{cfener}),
\item dissipation visqueuse (\fort{cfdivs}),
\item transport de pression (directement dans \fort{cfener}),
\item puissance de la pesanteur (directement dans \fort{cfener}),
\item termes diffusifs en $\displaystyle\dive\left(K\,\grad\left(\frac{1}{2} u^2 +
\varepsilon_{sup}\right)\right)$ (calcul de $\varepsilon_{sup}$ par
\fort{uscfth}, puis calcul du terme diffusif directement dans \fort{cfener}).
\end{itemize}

\bigskip
Le syst\`eme (\ref{Cfbl_Cfener_eq_energie_totale_discrete_cfener}) est r\'esolu par une m\'ethode
d'incr\'ement et r\'esidu  en utilisant une m\'ethode de Jacobi (\fort{cfcdts}).

L'impression des bornes et
la limitation \'eventuelle de l'\'energie sont ensuite effectu\'ees par
\fort{clpsca} suivi de \fort{uscfth} (intervention utilisateur optionnelle).

On actualise enfin la pression et on calcule la
temp\'erature (\fort{uscfth}).

Pour
terminer, en parall\`ele ou en p\'eriodique, on \'echange les variables
pression, \'energie et temp\'erature.



%%%%%%%%%%%%%%%%%%%%%%%%%%%%%%%%%%
%%%%%%%%%%%%%%%%%%%%%%%%%%%%%%%%%%
\section*{Points \`a traiter}
%%%%%%%%%%%%%%%%%%%%%%%%%%%%%%%%%%
%%%%%%%%%%%%%%%%%%%%%%%%%%%%%%%%%%

% propose en patch 1.2.1

%Corriger \fort{cfener} dans lequel \var{W1} produit par \fort{uscfth} est
%\'ecras\'e par \fort{grdcel}, causant probablement des d\'eg\^ats
%dans les cas o\`u le gradient de l'\'energie cin\'etique dans la direction $x$
%est sensiblement non nul sur des faces de bord dont la normale a une
%composante en $x$ et lorsque la conductivit\'e n'est pas n\'egligeable.


\etape{Choix de $\widetilde{P}$}
En standard, on utilise $\widetilde{P}=P^n$, mais ce n'est pas le seul choix
possible. On pourrait �tudier le comportement de l'algorithme avec $P^{Pred}$ et
$P^{n+1}$ (avec $P^{n+1}$, en particulier,
$\displaystyle\frac{\widetilde{P}}{\rho^{n+1}}$
est �valu� avec la masse volumique et l'�nergie prises au m�me instant).

\etape{Terme source dans l'�quation de l'�nergie}
La pr�sence d'un terme source externe dans l'�quation de l'�nergie g�n�re des
oscillations de vitesse qu'il est important d'analyser et de comprendre.
