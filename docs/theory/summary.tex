%-------------------------------------------------------------------------------

% This file is part of Code_Saturne, a general-purpose CFD tool.
%
% Copyright (C) 1998-2011 EDF S.A.
%
% This program is free software; you can redistribute it and/or modify it under
% the terms of the GNU General Public License as published by the Free Software
% Foundation; either version 2 of the License, or (at your option) any later
% version.
%
% This program is distributed in the hope that it will be useful, but WITHOUT
% ANY WARRANTY; without even the implied warranty of MERCHANTABILITY or FITNESS
% FOR A PARTICULAR PURPOSE.  See the GNU General Public License for more
% details.
%
% You should have received a copy of the GNU General Public License along with
% this program; if not, write to the Free Software Foundation, Inc., 51 Franklin
% Street, Fifth Floor, Boston, MA 02110-1301, USA.

%-------------------------------------------------------------------------------

%%%%%%%%%%%%%%%%%%%%%%%%%%%%%%%%%%%%%%%%%%%%%%%%%%%%%%%%%%%%%%%%%%%%%%
% ABSTRACT
\vspace*{0.1cm}
\begin{center}
\medskip
\textit{ABSTRACT}
\end{center}
\vspace*{1cm}
\pdfbookmark[1]{Summary}{summary}

\CS solves the Navier-Stokes equations for 2D, 2D axisymmetric, or 3D,
steady or unsteady, laminar or turbulent, incompressible or dilatable
flows, with or without heat transfer, and with possible scalar
fluctuations. The code also includes a Lagrangian module, a
semi-transparent radiation module, a gas combustion module, a coal
combustion module, an electric module (Joule effect and electric arc)
and a compressible module. In the present document, the "gas
combustion", "coal combustion", "electric" and "compressible"
capabilities of the code will be referred to as "particular
physics". The code uses a finite volume discretization. A wide range
of unstructured meshes, either hybrid (containing elements of
different types) and/or non-conform, can be used.

This document constitutes the theory guide
associated with the kernel of \CS.
The system of equations considered consists of the
Navier-Stokes equations, with turbulence and passive scalars. Firstly, the
continuous equations for mass, momentum, turbulence and passive scalars are
presented. Secondly, information related to the time scheme is supplied.
Thirdly, the spatial discretisation is detailed: it is based on a co-located%
\footnote{%
All the variables are located at the centres of the cells.} finite volume
scheme for unstructured meshes. Fourthly, the different source terms are 
described. Fithly, boundary conditions are detailed. And finally, some algebrae
such as how to solve a non-linear convection diffusion equation and some 
linear algebrae algorithms are presented.

In a seconde part, advanced modellings such as Combustion, electric and compressible flows
 are presented with their perticular treatments.

To make the documentation suitable to the developers' needs, the appendix
has been organized into sub-sections corresponding to the major steps of the
algorithm and to some important subroutines of the code.

During the development process of the code, the documentation is naturally
updated as and when required by the evolution of the source code itself.
Suggestions for improvement are \textbf{more than} welcome. In particular, 
it will be necessary to deal with some transverse subjects 
(such as parallelism, periodicity) which were voluntarily left out of 
the first versions, to focus on the algorithms and their implementation.

To make it easier for the developers to keep the documentation up to date
during the development process, the choice is made to not based this document
on the implementation (except in the appendix) but to keep as much as possible 
a general formulation. For developpers who are interested in the way theory is
implemented, please refer to the \texttt{doxygen} documentation. 
A special effort will be made to link this theory guide to the \texttt{doxygen}
documentation.

\CS is free software; you can redistribute it
and/or modify it under the terms of the GNU General Public License
as published by the Free Software Foundation; either version 2 of
the License, or (at your option) any later version.
\CS is distributed in the hope that it will be
useful, but WITHOUT ANY WARRANTY; without even the implied warranty
of MERCHANTABILITY or FITNESS FOR A PARTICULAR PURPOSE.  See the
GNU General Public License for more details.

%
%%%%%%%%%%%%%%%%%%%%%%%%%%%%%%%%%%%%%%%%%%%%%%%%%%%%%%%%%%%%%%%%%%%%%%

%
%%%%%%%%%%%%%%%%%%%%%%%%%%%%%%%%%%%%%%%%%%%%%%%%%%%%%%%%%%%%%%%%%%%%%%
% EXECUTIVE SUMMARY
%\passepage
%\vspace*{0.1cm}
%\begin{center}
%\textbf{\large \titreang}\\
%\medskip
%\textit{EXECUTIVE SUMMARY}
%\end{center}
%\vspace*{1cm}
%\pdfbookmark[1]{Excutive summary}{executive summary}
%
%This is the executive summary of the report.
%
%
%%%%%%%%%%%%%%%%%%%%%%%%%%%%%%%%%%%%%%%%%%%%%%%%%%%%%%%%%%%%%%%%%%%%%%
