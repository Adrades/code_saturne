%===============================================
\section{Atmospheric flow modelling}
%===============================================

%=====================================
\subsection{Equations of atmospheric motion}
%=====================================

\subsubsection{Atmospheric specific features}

The following table illustrates the vertical
variations observed in the standard atmosphere:


\begin{table}[htbp]

\begin{center}

\caption[Variation]{vertical variation of different variables}

\begin{tabular}{|l|l|l|l|}

\hline

altitude \newline
(m)&
Temperature (\textdegree C)&
Pressure \newline
(Pa)&
Density \newline
(kg/m$^{3})$ \\

\hline
0.&
15.0&
101325.&
1.225 \\
\hline
1000.&
8.5&
89869.&
1.112 \\
\hline
2000.&
2.0&
79485.&
1.007 \\
\hline
3000.&
-4.5&
70095.&
.909 \\
\hline
4000.&
-11.0&
61624.&
.819 \\
\hline
5000.&
-17.5&
54002.&
.736 \\
\hline
6000.&
-24.0&
47162.&
.660 \\
\hline
7000.&
-30.5&
41041.&
.589 \\
\hline
8000.&
-37.0&
35580.&
.525 \\
\hline
9000.&
-43.5&
30723.&
.466 \\
\hline
10000.&
-50.0&
26418.&
.412 \\
\hline
11000.&
-56.5&
22614.&
.364 \\
\hline


\end{tabular}
\label{tab:atmo:tab1}
\end{center}
\end{table}

We therefore see that for motions with vertical scales that are of the order
of 100m one can approximately consider the density as constant, except when
thermal effects are important. Alternatively we have to take into account
these variations with height as described below.

\subsubsection{Anelastic approximation}
Atmospheric flows can be both considered incompressible in the sense that
their Mach number is much less than one but also flows in which the density
is a function of pressure (as seen in \tablename{} \ref{tab:atmo:tab1}).

These can be reconcilied with the anelastic approximation (see \cite{Pielke:1984})
 in which the time derivative term $\frac{\partial \rho }{\partial
t}$ is neglected in front of the other terms. The continuity equation
therefore becomes:
\begin{equation}
\frac{\partial \rho u_{i} }{\partial x_{i} }=0
\end{equation}
This form of the continuity equation eliminates the acoustic waves, which
have a very high propagation velocity, from the possible solutions.

\subsubsubsection{Potential temperature}
Potential temperature is derived from the temperature but is a conserved
quantity through adiabatic transformation (compression/expansion). Its
expression is (see \cite{Holton:1979} and \cite{Stull:1988}):
\begin{equation}
\label{eq1}
\theta =T\left( {\frac{p_{s}}{p}} \right)^{\left( {\frac{R^{\star }}{C_{p}}} \right)}
\end{equation}
where $p_{s}$ is a constant pressure conventionally taken as 1000 mb
and $R^{\star}$ is the gaz constant for dry air (=287 $Jkg^{-1}K^{-1}$).
With this variable the thermal equation becomes:

\begin{equation}
\rho \left( {\frac{\partial \theta }{\partial t}+u_{j} \frac{\partial \theta
}{\partial x_{j} }} \right)\approx\left( {\frac{\partial }{\partial x_{j}
}\left( {\frac{\lambda_{t} }{C_{p} }\frac{\partial \theta}{\partial x_{j} }}
\right)+\Phi } \right)
\end{equation}
Where $\Phi $ is the heating/cooling source term and $\lambda_{t}$ is the thermal conductivity.

During their motion air parcels keep their potential temperature unless
diabatic effect are present (such as condensation/evaporation of water as we
will see in the micro-physics section)

Potential temperature is also used to easily characterize the local static
stability of the atmosphere:
\begin{itemize}
  \item if $\frac{\partial \theta }{\partial z}_{\, \, \, }$\textgreater 0 , the
    thermal stratification is stable,
  \item if$_{\, \, \, }\frac{\partial \theta }{\partial z}_{\, }=$ 0 , the
    thermal stratification is neutral,
  \item if$_{\, \, \, }\frac{\partial \theta }{\partial z}_{\, }$\textless 0 ,
    the thermal stratification is unstable.
\end{itemize}

The distinction between the temperature and the potential temperature is
important. For example in the table above the temperature is decreasing with
height (by approximately 6.5 K/km) but the standard atmosphere is stably
stratified with a potential temperature increasing with height.


\subsubsection{Turbulence production by buoyancy}
In standard \CS with the $k-\varepsilon$ turbulence model, the production or destruction rate of the turbulent kinetic energy due to buoyancy is given by:
\begin{equation}
\label{eq:atmo:buoyancy_prod}
G=-g \frac{\mu_{t} }{\sigma_{t} }\frac{1}{\rho }\frac{\partial \rho
}{\partial z}
\end{equation}
where $\sigma_{t}$ is the turbulent Prandtl (Schmidt) number and $\mu_{t}$ is the turbulent viscosity.
However for the neutral (adiabatic) atmosphere for which the buoyancy
production of turbulence is zero, we have a diminution of the density with
height and formula \eqref{eq:atmo:buoyancy_prod} is no longer valid and for atmospheric motions should
be replaced by \cite{Stull:1988}:
\begin{equation}
\label{eq:atmo:eq4}
G=-g \frac{\mu_{t}}{\sigma_{t} }\frac{1}{\theta }\frac{\partial \theta
}{\partial z}
\end{equation}

\subsubsection{ Momentum equations }
The momentum equations of the standard \CS are not
modified for small scale atmospheric flows, and in particular the Coriolis
term should not be used as the Rossby number is almost always much larger
than one. In very specific cases the Coriolis term and the associated
pressure gradient can be introduced as user source terms.

\section{ Method of imbrication in \CS Atmospheric physics (MICA)}
We explain in this section how meteorological profiles are used in
atmospheric flow calculations for defining boundary values.

A first option is to use only one profile giving the wind (horizontal components U and V),
the turbulent kinetic energy (TKE), the dissipation rate of TKE ($\epsilon$),
temperature (and optionally specific humidity and water droplet density for
humid atmosphere physics) as function of the altitude. This option does not allow to take into account the horizontal gradients present in the
fields given by a larger scale model, as the values given by the
profile are imposed at all the boundary faces of the computational domain
marked as inlet.

Generating a physically consistent marking is already a
challenging task since the wind direction may vary with the altitude,
justifying the introduction of the ``automatic boundary conditions'' flag:
for each boundary face so flagged the scalar product of the wind given by
the profile and the normal vector to the face is computed and that face is
then deemed inlet or free outlet according to the sign of the result (the
limit case 0 is considered inlet too). But if the nature (inlet or outlet)
is diagnosed face by face at each time step one is not restricted to
horizontally homogeneous profiles: this flag permits the use of large scale
fields coming from another meteorological model , WRF (the Weather
Research and Forecasting Model: www.wrf-model.org) for
instance, to specify boundary values.

We describe now the method of interpolation of the large scale profiles on the ``automatic boundary
conditions'' faces of the computational domain: the so called Cressman
method.

This method has a rather long history of use in meteorological
modelisation: the first publication (see \cite{Cressman:1959}) appeared more than 50 years ago.
Its formulation is simple: in order to estimate the value of a physical variable (\emph{e.g.} wind components, temperature)
at a grid point from scattered data points one uses a weighted average of the values.
The weights of the different data are function of the displacement between
the data points and the estimation point, very often function of their
distance only. In the original publication the function was a clipped
rational function of the distance while the method used in \CS
atmospheric flows uses a Gaussian kernel like in \cite{Barnes:1964}.
If we denote the Cartesian coordinates of the scattered data points by
$(x_{i}y_{i}z_{i})$, by (x, y, z) the coordinates of the generic estimation
point and by Vi the values of the variable of interest the estimate reads:
\begin{equation}
\tilde{V}\, =\frac{\sum\nolimits_i {V_{i\, }f(x-x_{i},y-y_{i},z-z_{i})}
}{\sum\nolimits_i {f(x-x_{i},y-y_{i},z-z_{i})} }
\end{equation}
This will be a well-defined weighted average as long as the weight function
f is positive and the sum appearing as denominator strictly positive. For
application in \CS the following form of f has been chosen:

\begin{equation}
f\left( x,y,z \right)=
\mathrm{exp}(-\frac{x^{2}+y^{2}}{4R_{H}^{2}}-\frac{z^{2}}{4R_{V}^{2}})
\end{equation}

The radius $R_{H}$ ($R_{V})$ is the so called horizontal (vertical) range
of the method. Their influence on the estimates is easy to describe. When
the horizontal (vertical) distance of a data point to the estimation point
is much larger than the corresponding range this data will see its
contribution to the estimate diminish. Only points relatively close to the
estimation point will yield appreciable contributions to this estimate.
Please notice also that even at the data point the formula will give an
estimate different from the data value. The estimation is not an exact
interpolator, but a smoothing estimator.

\section{Radiation parameterizations for atmosphere}

For these parameterizations we distinguish two spectral domains: visible
(0.2 $\mu$m -- 5 $\mu$m) for solar radiation and infrared (5 $\mu$m
-- 100 $\mu$m) for thermal radiation.

\subsection{Thermal infrared radiation}
The radiation transfer equation applied to a plan parallel atmosphere (1-D)
using the two stream approximation for a non-scattering medium with the
broadband emissivity approximation for spectral integration (see \cite{Liou:2002}
and \cite{Makke:2016}) leads to the expressions of the upward
and downward fluxes:
\begin{equation}
F\uparrow =\sigma T_{g}^{4}\left( 1-\epsilon \left( z,0 \right)
\right)-\int_0^z \sigma T^{4}\left( z' \right)\frac{d\epsilon
}{dz'}\left( z,z' \right)dz'
\end{equation}

\begin{equation}
F\downarrow =\int_z^\infty \sigma T^{4}\left( z' \right)\frac{d\epsilon
}{dz'}\left( z,z' \right)dz'
\end{equation}
Where T is the fluid temperature in K, T$_{g}$ the ground surface temperature in K,

$\sigma $ the Stefan Boltzman constant and $\varepsilon$ the atmosphere
emissivity defines by:
\begin{equation}
\varepsilon \left( z,z' \right)
=\frac{1}{\sigma T^{4}}\int_0^\infty {\left(1-I_{\lambda}\left(z,z'
\right) \right)\pi I_{\lambda}^{0}} \left( z,z' \right)d\lambda
\end{equation}
\newline
Where
$I_{\lambda }^{0}=
{2hc^{2}\lambda^{-5}}
{(e^{\frac{{hc}}{k_{B}\lambda T}}-1)}$ is the Planck function,
\newline
$I_{\lambda}$ the transmittance and $k_{B}$ is the Boltzman constant.
\newline
\newline
If the ground surface is not considered as a black body, with a ground
emissivity $\varepsilon_{g}$ the upward flux can be written (see \cite{Ponnulakshmi:2012}):
\newline
\begin{equation}
F\uparrow =\varepsilon_{g}\sigma T_{g}^{4}\left( 1-\varepsilon \left( z,0
\right) \right)+\left( 1-\varepsilon_{g} \right)\int_0^\infty {\sigma
T^{4}} \left( z' \right)\varepsilon \left( -z,z' \right)dz'-\int_0^z
{\sigma T^{4}\left( z' \right)\frac{d\varepsilon }{dz'}\left(z',z
\right)dz'}
\end{equation}
\newline
In this expression, the second term can be considered as a downward
radiation along --z and infinite. Integration by parts and supposing
isothermal atmospheric layers above zt $=$ 11000 m leads to the following
expressions for upward and downward fluxes:
\newline
\begin{equation}
F\downarrow =\varepsilon \left( \infty ,0 \right)\sigma T^{4}\left( zt
\right)\, -\, \int_z^{zt} {\varepsilon (z',z)\frac{d \sigma
T^{4}(z')}{dz'}dz'}
\end{equation}

\begin{equation}
F\uparrow =\varepsilon_{g}T_{g}^{4}+\int_0^z {\frac{d\left( \sigma
T^{4}\left( z' \right) \right)}{dz'}\, \varepsilon \left( z',z
\right)dz'+\left( 1-\varepsilon_{g} \right)[\sigma T^{4}} \left( zt
\right)\, \varepsilon \left( \infty ,-z \right)-\int_0^{zt} \frac{d\sigma
T^{4}\left( z' \right)}{dz'} \, \varepsilon (-z,z')dz']
\end{equation}

The heating/cooling for atmospheric layers is deduced from the divergence of
the net Flux:
\begin{equation}
F_{n}=F\uparrow -F\downarrow .
\end{equation}

\begin{multline}
\frac{dF_{n}}{dz}=\int_0^z {\frac{d\left( \sigma T^{4}\left( z' \right)
\right)}{dz'}\, \frac{d\varepsilon \left( z',z
\right)}{dz'}dz'
+\left( 1-\varepsilon_{g} \right)[\sigma T^{4}} \left(zt \right)\frac{\, d\varepsilon \left( \infty ,-z \right)}{dz'}\\
-\int_0^{zt} \frac{d\sigma T^{4}\left( z' \right)}{dz'} \frac{\, d\varepsilon \left( -z,z' \right)}{dz'}dz']
-\, \frac{d\varepsilon \left( \infty ,0 \right)}{dz}\sigma T^{4}\left( zt \right)
+\, \int_z^{zt} {\frac{d\varepsilon (z',z)}{dz'}\frac{d\sigma T^{4}(z')}{dz'}dz'}
\end{multline}

\subsubsection{Clear sky}
To determine emissivity it is common to introduce the corrected path length
$u_{g}(z,z')$ for major absorbing gases: water vapor, carbon dioxide and ozone.

\begin{equation}
u_{g}\left( z,z' \right)=\int_{z'}^{z} \rho_{g} \left( z"
\right)\mathrm{\Psi }\left( T_{0},P_{0},T,P \right)\, dz"
\end{equation}
where $\mathrm{\Psi }$ is a scaling function eliminating the dependence of
the absorption coefficient on pressure and temperature. These corrections,
known as the one-parameter scaling approximation, are attributed to \cite{Chou:1980}.
The emissivity $\varepsilon(z,z')$ for each gas is now a
function of the scaled path $\varepsilon (u_{g}(z,z'))$. The total emissivity for
the three dominant atmospheric gases in the infrared domain is:

\begin{multline}
\varepsilon \left( z,z' \right)=
\, \varepsilon_{H_{2}O}(u_{H_{2}O}\left(z,z' \right)
+\varepsilon_{O_{3}}(u_{O_{3}}\left( z,z' \right)\\
+\varepsilon_{CO_{2}}(u_{CO_{2}}\left( z,z' \right)T_{15\mu }\left( u_{H_{2}O}\left( z,z'
\right) \right)+\varepsilon_{dim}(u_{dim}\left( z,z' \right)T_{w}\left(
u_{H_{2}O}\left( z,z' \right) \right)
\end{multline}

The optical thickness are:
\begin{equation}
u_{H_{2}O}\left( z,z' \right)=\int_{z'}^{z} \rho_{H_{2}O} \left( z" \right)\,
\left( \frac{P(z")}{P_{0}}\sqrt \frac{T_{0}}{T(z")} \right)^{n}dz"
\end{equation}

\begin{equation}
u_{CO_{2}}\left( z,z' \right)=\int_{z'}^{z} \rho_{CO_{2}} \left( z" \right)\,
\left( \frac{P(z")}{P_{0}} \right)^{3/4}dz"
\end{equation}

\begin{equation}
u_{O_{3}}\left( z,z' \right)=\left| f_{Green}\left( z
\right)-f_{Green}(z') \right|
\end{equation}

\begin{equation}
u_{dim}\left( z,z' \right)=\int_{z'}^{z} \rho_{H_{2}O} \left( z" \right)\,
e(z")dz"f(T(z"),T_{0})dz"
\end{equation}
Where $f\left( T\left( z \right)\mathrm{,}T_{\mathrm{0}}
\right)\mathrm{=exp}\left[ \mathrm{1800}\left(
\frac{\mathrm{1}}{T\mathrm{(}z\mathrm{)}}\mathrm{-}\frac{\mathrm{1}}{\mathrm{296.0}}
\right) \right]$ and $f_{Green}\left( z
\right)\mathrm{=}a\frac{\mathrm{1+}e^{\mathrm{-}b\mathrm{/}c}}{\mathrm{1+}e^{\mathrm{(}z\mathrm{-}b\mathrm{)/}c}}$
\newline
with a$=$0.1 cm$/$STP (Standard Temperature and Pressure), b$=$ 20km,
\newline
c chosen as b/c$=$5 (see \cite{Green:1964})

The associated emissivities are:

\cite{Sasamori:1968} formula for water vapor, carbon dioxide and ozone:
\begin{equation}
\varepsilon_{H_{2}O}\left( u_{H_{2}O} \right)=\left\{ {\begin{array}{l}
 0.846\ast \left( 3.59\times {10}^{-5}+\frac{u_{H_{2}O}}{10.} \right)^{0.243}\,
\, for\, u_{H_{2}O}<0.1\, kg.m^{-2} \\
 0.24\, {log}_{10}\left( 0.01+\frac{u_{H_{2}O}}{10.} \right)+0.622\, \, \,
for\, u_{H_{2}O}\ge 0.1\, kg.m^{-2} \\
 \end{array}} \right.
\end{equation}

\begin{equation}
\varepsilon_{O_{3}}\left( u_{O_{3}} \right)=\left\{ {\begin{array}{l}
 0.209\ast \left( u_{O_{3}}+7.0\times {10}^{-5} \right)^{0.436}-3.21\times
{10}^{-3}\, for\, u_{O_{3}}<0.01\, cm/STP \\
 7.49\times {10}^{-2}+2.12\times {10}^{-2}\, {log}_{10}\left( u_{O_{3}}
\right)\, \, \, \, \, \, \, \, \, \, \, \, \, \, \, \, \, \, \, for\,
{u_{O_{3}}\ge 0.01\, cm/STP} \\
 \end{array}} \right.
\end{equation}

\begin{equation}
\varepsilon_{CO_{2}}\left( u_{CO_{2}} \right)=\left\{ {\begin{array}{l}
 0.0676\ast \left( 0.01022+u_{C02} \right)^{0.421}-0.00982\, \, for\,
u_{CO_{2}}<1\, cm/STP \\
 0.24\, {log}_{10}\left( 0.0676+u_{CO_{2}} \right)+0.622\, \, \, \, \, \, \, \,
\, \, \, \, \, \, for\, u_{CO_{2}}\ge 1\, cm/STP \\
 \end{array}} \right.
\end{equation}

\begin{equation}
T_{15\mu }\left( u_{H_{2}O} \right)=\left\{ {\begin{array}{l}
 1.33-0.832\left( 0.0286+\frac{u_{H_{2}O}}{10.} \right)^{0.26}\, for\, {\,
u}_{H_{2}O}<20\, kg.m^{-2} \\
 0.33-0.2754\, \left( {log}_{10}\left( \frac{u_{H_{2}O}}{10.} \right)-0.3011
\right)\, for\, {\, u}_{H_{2}O}\ge 20\, kg.m^{-2} \\
 \end{array}} \right.
\end{equation}
(see \cite{Veyre:1980}) for water vapor dimer:
\newline%FIXME
$\varepsilon_{dim}\left( u_{dim} \right)=\left\{ {\begin{array}{l}
 0.4614\left[ 1-\left( \sum\limits_{i=0}^2 {a_{i}u_{dim}^{i}} \right)/\left(
\sum\limits_{j=0}^3 {b_{j}u_{dim}^{j}} \right) \right]\, \, \, for{\,
u}_{dim}\le 0.5\, g.{cm}^{-2} \\
 0.4614\, \, \, \, \, \, \, \, \, \, \, \, \, \, \, \, \, \, \, \, \, \, \,
\, \, \, \, \, \, \, \, \, \, \, \, \, \, \, \, \, \, \, \, \, \, \, \, \,
\, \, \, \, \, \, \, \, \, \, \, \, \, \, \, \, \, \, \, \, \, for\, {\,
u}_{dim}>0.5\, g.{cm}^{-2} \\
 \end{array}} \right.$

with
\begin{equation}
a_{0}=0.015075,\, a_{1}=\, -0.036185,\, a_{2}=0.0019245,\,
b_{0}=a_{0},b_{1}=0.19547,\, b_{2}=0.75271,\, b_{3}=1
\end{equation}

$T_{w}\left( u_{H_{2}O} \right)=\left( \sum\limits_{i=0}^4
{a_{i}^{'}u_{dim}^{i}} \right)/\left( \sum\limits_{j=0}^5
{b_{j}^{'}u_{dim}^{j}} \right)$
\newline
with
\newline
\begin{equation}
a_{0}^{'}=7.76192\times {10}^{-7},\, a_{1}^{'}=1.33836\times {10}^{-3},\,
a_{2}^{'}=0.166649,\, a_{3}^{'}=2.17686,\, a_{4}^{'}=2.6902\,
\end{equation}

\begin{equation}
b_{0}^{'}=7.79097\times {10}^{-7},\, b_{1}^{'}=1.36832\times {10}^{-3},\,
b_{2}^{'}=0.179601,\, b_{3}^{'}=2.70573,\, b_{4}^{'}=5.15119,\, b_{5}^{'}=1
\end{equation}

\subsubsection{Cloudy sky}
The transmittance by cloud droplets can be considered as grey body
transmittance and overlapping by gas absorption is taken into account as
in \cite{Sasamori:1972}.
\begin{equation}
\varepsilon \left( z',z \right)=1-\left( 1-\varepsilon_{gas}\left( z',z
\right) \right)\tau_{l}\left( z',z \right)
\end{equation}
with $\tau_{l}\mathrm{(}z^{\mathrm{'}}z\mathrm{)=1-}\varepsilon
_{l}\mathrm{(z',z)=exp(-}K_{l}\int_{z\mathrm{'}}^z
{\rho \mathrm{(}z^{\mathrm{''}}\mathrm{)}q_{l}\left( z^{\mathrm{''}}
\right)dz\mathrm{''}} $ )

and K$_{l}=$1.66 3/4r$_{e}$ where r$_{e}$ is the
effective radius of the cloud droplet.

In case of partial cloudiness N$_{max}$, the transmittance for liquid water
can be modified following \cite{Bougeault:1985}.
\begin{equation}
\tau_{l}\left( z',z \right)=1-N_{max}\left( z',z
\right)+N_{max}(z',z)\mathrm{exp}(-K_{l}\int_{z'}^z {\rho
(z^{''})q_{l}\left( z^{''} \right)dz'')}
\end{equation}


\subsection{Solar radiation}
The first step consists to determine astronomic factors linked to earth-sun
position

\subsubsection{Determination of zenithal angle and correction factors for solar
constant (see \cite{Paltridge:1974})}

\begin{equation}
\mu_{0}=\cos \theta =\sin \delta \sin \phi +\cos \delta \cos \phi \cos
{th}
\end{equation}
Where $\varphi $ is the latitude, $\delta $ the inclination which depends on
the day of year:
\begin{multline}
\delta =
0.006918
-0.399912\cos \theta_{0}
+0.070257\sin \theta_{0}
-0.006758\cos 2\theta_{0}\\
+0.000907\sin 2\theta_{0}
-0.002697\cos 3\theta_{0}
+0.001480\sin 3\theta_{0}
\end{multline}

with $\theta_{0}=2\pi J/365\\$

The solar local time th$= $ local time $+$ longitude correction $+$ Eq (Time
equation).

\begin{equation}
Eq=0.000075+0.001868\cos {\theta_{0}-0.032077\sin {\theta_{0}-0.014615\cos
{2\theta_{0}}}}-0.040849\sin {2\theta_{0}}
\end{equation}
A correction for sun-earth distance is taken into account

\begin{equation}
F=F_{0} (1.00011+0.034221\cos {\theta_{0}+0.001280\sin {\theta
_{0}+0.000719\cos {2\theta_{0}}}}+0.000077\sin {2\theta_{0}})
\end{equation}
with F$_{0}=$1367 W m$^{-2}$.

In order to take into account earth curvature a correcting term is
added:
\begin{equation}
\mu_{0}=\frac{r_{1}}{\sqrt {\mu_{0}^{2}+r_{1}(r_{1}+2)} -\mu_{0}}\, \,
with\, r_{1=}\, \frac{H_{atmo}}{R_{earth}}=\frac{8\, km}{6371\, km}
\end{equation}

\subsubsection{Solar radiation in clear sky without aerosols}
The radiation transfer equation applied to a plan parallel atmosphere (1D)
using the two stream approximation in a non-scattering medium with an
external radiation source term (the sun) for the global radiation (direct
$+$ diffuse) leads to the expression of downward and upward fluxes for a
spectral band $\Delta \nu $ (see \cite{Lacis:1974}):
\begin{equation}
S\downarrow =\mu_{0} F_{0} \tau_{\Delta \nu}\, u_{gas}\left( z,\infty
\right)\,
\end{equation}

\begin{equation}
S\uparrow =\mu_{0} F_{0} R_{g}\tau_{\Delta \nu}\, (u_{gas}(
0,\infty), u_{gas}(0,z))
\end{equation}
Where F$_{0}$ is the solar constant, $\mu_{0}$ is the corrected cosines of the
zenithal angle, $\tau_{\Delta \nu}$ the transmittance and u$_{gas}$ the
optical thickness for the considered gas.

The solar spectrum is divided in two bands: One for ozone (0.2-0.7 $\mu$m)
where absorption occurs in the upper layers of the atmosphere, a second one
for water vapor (0.7-5 $\mu$m) part where absorption occurs in the lower
part of the atmosphere.

\subsubsection{Ozone Band}

For ozone band, the Rayleigh diffusion is parameterized as an albedo for the
lower part of the atmosphere combined with earth surface albedo. With these
approximations the heating rate due to ozone in the layer (l, l$+$1) is
(LH74):
\begin{equation}
F_{abs}^{O_{3}}=\mu_{0}F\, [\, A_{O_{3}}\left( x_{l+1} \right)-A_{O_{3}}\left( x_{l}
\right)-\bar{R}(\mu_{0})\left( A_{O_{3}}\left( x_{l+1}^{\star }
\right)-A_{O_{3}}\left( x_{l}^{\star } \right) \right)]
\end{equation}
With $x=Mu_{O_{3}}\left( \infty ,z \right),\, \, \, \, x^{\star }=Mu_{O_{3}}\left(
\infty ,0 \right)+\bar{M}(u_{O_{3}}\left( \infty ,0 \right)-u_{O_{3}}\left( \infty
,z \right))$ \\
where $\bar{M}=1.9$ and $\, \, \, M=35/{(1224\mu
_{0}^{2}+1)}^{1/2}$.

$\bar{R}(\mu_{0})$ is the composite albedo including Rayleigh diffusion
effect and ground reflection

$\bar{R}\left( \mu_{0} \right)=\overline{R_{a}}\left( \mu_{0}
\right)+(1-\overline{R_{a}}\left( \mu_{0} \right))(1-\overline{R_{a}^{\star
}}{)R}_{g}/(1-\overline{R_{a}^{\star }}R_{g})$

with $\overline{R_{a}}\left( \mu_{0}
\right)=0.213/(1+0.816\mu_{0})$ and $\overline{R_{a}^{\star }}=0.144$.

The ozone absorption is parameterized as LH74 for both Chappuis and
ultraviolet bands:

$A_{O_{3}}\left( x
\right)=\frac{0.02118x}{1+0.042x+0.000323x^{2}}+\frac{1.082x}{{(1+138.6x)}^{0.805}}+\frac{0.0658x}{1+{(103.6x)}^{3}}$
where x is in cm/STP.


The global downward flux in the O3 band is (LH74):

${F\downarrow}_{g}^{O_{3}}=\mu_{0}F(0.647-A_{O_{3}}\left( x
\right)-\overline{R_{r}}\left( \mu_{0} \right))(1-
\overline{\overline{R_{r}^{\star }}}
R_{g})$
\newline
with $\overline{R_{r}} \mu_{0}=0.28/(1+6.43\mu_{0})$
\newline
and $\overline{\overline{R_{r}^{\star }}}=0.0685$
\newline
\newline
The direct downward flux is (see \cite{Atwater:1978}):
\begin{equation}
{F\downarrow }_{d}^{O_{3}}=\mu_{0}F\left( 0.647-A_{d}^{O_{3}}\left( z \right)
\right)with
A_{d}^{O_{3}}\left( z \right)=1-(1.041-0.16\ast \sqrt {M\left( 0.949\,
{10}^{-3}P\left( z \right)+0.051 \right)} )
\end{equation}
\subsubsection{Vapor water band}

For water vapor band (LH74):
\begin{equation}
F_{abs}^{H_{2}O}=\mu_{0}F\, [\, A_{H_{2}O}\left( y_{l+1} \right)-A_{H_{2}O}\left(
y_{l} \right)-R_{g}\left( A_{H_{2}O}\left( y_{l+1}^{\star }
\right)-A_{H_{2}O}\left( y_{l}^{\star } \right) \right)]
\end{equation}
\newline
With $y=Mu_{H_{2}O}\left( \infty ,z \right),\, \, \, \, y^{\star
}=Mu_{H_{2}O}\left( \infty ,0 \right)+\frac{5}{3}(u_{H_{2}O}\left( \infty ,0
\right)-u_{H_{2}O}\left( \infty ,z \right))$

The direct downward flux is equal to the global downward flux:
\begin{equation}
{F\downarrow }_{d}^{H_{2}O}={F\downarrow }_{g}^{H_{2}O}=\mu_{0}F\left(
0.353-A_{H_{2}O}\left( y_{l} \right) \right)
\end{equation}
The water vapor absorption is parameterized as in \cite{Yamamoto:1962}

$A_{H_{2}O}\left( y \right)=\frac{0.29}{{(1.+14.15y)}^{0.635}+0.5925y}$ for y in
kg m$^{-2}$

\subsubsection{Solar radiation for cloudy atmosphere or in presence of aerosols}
\subsubsection{Ozone Band}

For ozone band, a similar expression is used to clear sky formulation by
replacing Rayleigh diffusion by cloud droplets diffusion in the composite
albedo including ground reflection. That leads to the following expression
for solar heating:
\begin{equation}
F_{abs}^{c,a,O3}=\mu_{0}F\, [\, A_{O_{3}}\left( x_{l+1} \right)-A_{O_{3}}\left(
x_{l} \right)-\bar{R}(\mu_{0})\left( A_{O_{3}}\left( x_{l+1}^{\star }
\right)-A_{O_{3}}\left( x_{l}^{\star } \right) \right)]
\end{equation}

$\bar{R}(\mu_{0})$ is the composite albedo including cloud droplet (c) or
aerosol (a) diffusion effect and ground reflection

\begin{equation}
\bar{R}\left( \mu_{0} \right)=\overline{R_{c,a}}\left( \mu_{0}
\right)+(1-\overline{R_{c,a}}\left( \mu_{0} \right))(1-\overline{R_{c,a}^{\star
}}{)R}_{g}/(1-\overline{R_{c,a}^{\star }}R_{g})
\end{equation}
With $\overline{R_{c,a}}\left( \mu_{0} \right)=\overline{R_{c,a}^{\star }}=\sqrt 3
\left( 1-g_{c,a} \right)\tau_{c,a}/(2+\sqrt 3 \left( 1-g_{c,a} \right)\tau
_{c,a})$
Where $g_{c,a}$ is the asymmetry factor of the cloud or aerosols
particles phase function, \\
g$_{c}=$0.85, g$_{a}=$ 0.66, $\tau_{a,c}$ is the total visual optical thickness
for cloud liquid water and aerosol concentration:

$\tau_{c}=1.5\int_0^{zt} \frac{\rho q_{l}}{r_{e}} dz$ and $\tau
_{a}=1.5\int_0^{zt} \frac{\rho C_{aero}}{r_{aero}} dz$ (see \cite{Stephens:1984}).

In case of partial cloudiness $N_{max}$, the heating rate and the downward
flux can be weighted between clear and cloudy sky as follows:

\begin{equation}
F_{abs}^{t,O3}={N_{max}F}_{abs}^{c,O3}+(1-N_{max})F_{abs}^{O_{3}}
\end{equation}
The effect of aerosol particles is only taken into account in the clear sky
layers with aerosol cloud fraction equal to unity and the absorption in that
case is $F_{abs}^{a,O3}$.

The global downward and direct fluxes are:
\begin{equation}
{F\downarrow }_{g}^{c,a,O3}=\mu_{0}F(0.647-A_{O_{3}}\left( x
\right)-\overline{R_{r}}\left( \mu_{0} \right))(1-\overline{R_{c,a}}\left( \mu_{0}
\right)/(1-\overline{R_{c,a}^{\star }}R_{g}),\quad
{F\downarrow }_{d}^{c,a,O3}=0.
\end{equation}

That gives if partial cloudiness is taken into account:
\begin{equation}
{F\downarrow }_{g}^{t,O3}={N_{max}F\downarrow
}_{g}^{c,a,O3}+(1-N_{max}){F\downarrow }_{g}^{O_{3}}
\end{equation}

\begin{equation}
{F\downarrow }_{d}^{t,O3}=(1-N_{max}){F\downarrow }_{d}^{O_{3}}
\end{equation}
\subsubsection{Vapor water band}

For water vapor, the diffusion processes are directly modelled using the
adding method with k distribution method for overlapping between liquid and
vapor water (LH74).

For each cloud layer l in the frequency interval n, the optical thickness $\tau
_{l,n}$ , the single scattering albedo $\omega_{l,n}$ and the asymmetry
factor $g$ are:
\begin{equation}
\tau_{l,n}^{'}=\tau_{c}+\tau_{a}+k_{n}u_{H_{2}O}
\end{equation}

\begin{equation}
\omega_{l,n}^{'}=(\omega_{c}\tau_{c}+\omega_{a}\tau_{a})/\tau
_{l,n}^{'}
\end{equation}

\begin{equation}
g^{'}=(\omega_{c}\tau_{c}g_{c}+\omega_{a}\tau_{a}g_{a})/{\omega
'}_{l,n}\tau_{l,n}^{'}
\end{equation}

\cite{Joseph:1976} corrections are used in order to describe highly forward
scattering for cloud droplets. That gives with $f=g'^{2}$ (see \cite{Stephens:1984}):

\begin{equation}
{\tau_{l,n}=\tau }_{l,n}^{'}(1-f\omega_{l,n}^{'})
\end{equation}

\begin{equation}
{\omega_{l,n}=\omega }_{l,n}^{'}(1-f)/(1-f\tau_{l,n}^{'})
\end{equation}

\begin{equation}
g=(g'-f)/(1-f)
\end{equation}

The transmission and reflection functions for global radiation are:

For clear layer:

$R_{l}=R_{l}^{\star }=0\, ,
\quad
T_{l}=T_{l}^{\star }=\mathrm{exp}(-\frac{5}{3}\tau_{l,n})$ except for a
clear layer above the highest cloud layer for which
$T_{l}=\mathrm{exp}(-M\tau_{l,n})$

For cloudy layer:
\begin{equation}
R_{l}=\frac{(u+1)(u-1)\left( e^{t}-e^{-t} \right)}{\left( u+1
\right)^{2}e^{t}-\left( u-1 \right)^{2}e^{-t}}\, ,\, T_{l}=\frac{4u}{\left(
u+1 \right)^{2}e^{t}-\left( u-1 \right)^{2}e^{-t}}
\end{equation}
with
\newline
\newline
$u=\sqrt {\frac{1-g\omega_{l,n}}{1-\omega_{l,n}}\, } $ and $t=\tau
_{l,n}\sqrt {3(1-\omega_{l,n})(1-g\omega_{l,n})} $
\newline
The transmission function for direct solar radiation is
\newline
\newline
$T_{l}=\mathrm{exp}(-\frac{5}{3}\tau_{l,n})$ for clear layers and
\newline
$T_{l}=\mathrm{exp}(-M\tau_{l,n})$ for cloudy layers
\newline
\newline
In the case where cloud fraction is taken into account the reflection and
transmission functions can be weighted by the cloud fraction
\newline
\newline
 $N_{l} : T_{l}=N_{l} T_{l,cloud}+$\textit{ (1-N}$_{l})T_{l,clear}$\textit{ and R}$_{l}=N_{l}R_{l,cloud}$
\newline
\newline
The following five steps are carried out for each value of k$_{n}$ which can
yield significant absorption, \emph{e.g.}, n$=$1-8 for the discrete distribution
given in the following table:

\begin{table}[htbp]
\begin{center}
\caption[absorption]{Absorption}
\begin{tabular}{|p{76pt}|l|l|l|l|l|l|l|l|}

\hline

\hline

$k_{n};n\in \left[ 1,8 \right]$&
4.10$^{-6}$&
2.10$^{-4}$&
0.0035&
0.0377&
0.195&
0.94&
4.46&
19 \\
\hline
$p\left( k_{n} \right)$&
0.6470&
0.0698&
0.1443&
0.0584&
0.0335&
0.0225&
0.0158&
0.0087\\
\hline

\end{tabular}
\label{tab2}
\end{center}
\end{table}


\begin{enumerate}
\item $R_{l} $ and $T_{l}$ for l $=$ 1, L are computed for each layer
\item The layers are added, going down, to obtain $R_{1,l}$ and $T_{1,l}$ as follow:

\begin{equation}
R_{1,l}=R_{1,l-1}+\frac{T_{1,l-1}R_{l}T_{1,l-1}^{\star }}{1-R_{1,l-1}^{\star
}R_{l}}\, \, ,\, \, T_{1,l}=\frac{T_{1,l-1}T_{l}}{1-R_{1,l-1}^{\star }R_{l}}
\end{equation}
With similar expressions for $R*$ and $T*$
\item Layers are added one at a time, going up, to obtain $R_{L+1,l}$ and $T_{L+1,l}$
\item As two composite layers, say 1,l and l$+$1, L$+$1 are added, the upward and downward fluxes boundary between the two layers are determined:
\begin{equation}
U_{l}=\frac{T_{1,l}{\, R}_{L+1,l}}{1-R_{1,l}^{\star }R_{L+1,l+1}}\, \, ,\, \,
D_{l}=\frac{T_{1,l}}{1-R_{1,l}^{\star }R_{L+1,l+1}}
\end{equation}
The fraction of the total incident flux absorbed in the upper composite
layer is:
\begin{equation}
A_{1,l(n)}=p\left( k_{n} \right)\left( 1-R_{1,L+1(n)}
\right)+U_{l(n)}-D_{l(n)}
\end{equation}
The total absorption in each layer l is found by differencing, \emph{e.g.},
\begin{equation}
A_{l(n)}=A_{1,l(n)}-A_{1,l-1(n)}
\end{equation}
The total absorption in each layer l is found by summing over the values of
n for which k$_{n}$ is significant
\begin{equation}
F_{abs}^{H_{2}O}=\mu_{0}F\sum\limits_{n=1}^{n=8} A_{l(n)}
\end{equation}
The downward flux for global radiation is:
\begin{equation}
{F\downarrow }_{g}^{H_{2}O}=\mu_{0}F\sum\limits_{n=1}^{n=8} {D_{l}\left( n
\right)p\left( k_{n} \right)}
\end{equation}
\end{enumerate}
For the downward direct radiation the layers are added as in the step 2) but
only for the transmission function and with reflection equal to zero:
$T_{1,l}=T_{1,l-1}T_{l}$ Since the downward direct radiation is:
\begin{equation}
{F\downarrow }_{d}^{H_{2}O}=\mu_{0}F\, \sum\limits_{n=1}^{n=8} T_{1,l} \left(
n \right)p(k_{n})
\end{equation}

\section{Warm cloud parameterization}
In order to represent cloud formation new thermodynamic variables are used
to be conservative through condensation/evaporation processes. These
variables are (see \cite{Betts:1973}):
\begin{equation}
\theta_{l} =\theta -\frac{L_{v} \theta }{C_{p} T }q_{l},
\quad
q_{w} =q_{v} +q_{l}
\end{equation}
\newline
where $\theta_{l}$ (K) is the liquid-water potential temperature,
\newline
$\theta $(K) the potential temperature,$T $(K) the temperature,
\newline
$q_{w}$ (kg kg$^{-1})$ the total water specific humidity (sum of the specific humidity
for water vapor $q_{v}$ (kg kg$^{-1})$ and liquid water $q_{l}$ (kg
kg$^{-1}))$,
\newline
$ L_{v\, }=$2.5 (MJ kg$^{-1})$ and $C_{p\, }=$ 1005 (J kg$^{-1}$ K$^{-1})$.
\newline
The ensemble mean of each quantity is denoted by an
overbar, whereas primes denotes fluctuating quantities (\emph{e.g.} $q=\bar{q}+q')$.

The energy and water conservation equations become with these variables:
\newline
\newline
\begin{equation}
\left( \frac{\partial }{\partial t}+\overline u_{i} \frac{\partial }{x_{i}}
\right)\overline \theta_{l} =\frac{\partial }{x_{i}}\left[ \left(
\frac{\lambda_{c}}{C_{p}}+\frac{\mu_{t}}{\sigma_{t}}
\right)\frac{\partial \overline \theta_{l} }{\partial x_{i}}
\right]+\frac{\overline \theta }{\overline T }\frac{1}{C_{p}}\frac{\partial
F_{R}}{\partial z}-\rho \frac{L}{C_{p}}\frac{\overline \theta }{\overline T}\left(
\frac{\partial \overline q_{l} }{\partial t} \right)_{sed}
\end{equation}

\begin{equation}
\left( \frac{\partial }{\partial t}+\overline u_{i} \frac{\partial }{x_{i}}
\right)\overline {q_{w}} =\frac{\partial }{x_{i}}\left[ \left( \frac{\lambda
_{c}}{C_{p}}+\frac{\mu_{t}}{\sigma_{t}} \right)\frac{\partial \overline
q_{w} }{\partial x_{i}} \right]+\rho \left( \frac{\partial
\bar{q_{l}}}{\partial t} \right)_{sed}
\end{equation}
Where $\rho $ is the air density, u$_{i}$ the wind components, $\lambda_{c}$ the thermal diffusivity, $\mu_{t}$ the turbulent viscosity,
$\sigma_{t}$ the turbulent Prandtl number and Fr the net radiative flux.

To be comprehensive, we have included the settling term $\left(
\frac{\partial \overline{q_{l} } } {\partial t} \right)_{sed}$ that can play an
important role in some cases, as shown for example for fog in \cite{Musson-Genon:1987}.

With the set of variables used, the liquid water content $q_{l}$ must be
diagnosed. This is done by using a subgrid condensation scheme described in
details in \cite{Bouzereau:2007}.

\subsection{Condensation processes in $\theta_{l}$ and $q_{w}$ variables}

$q_{l}$ is diagnosed from the predicted value of $q_{w}$ by using a subgrid
condensation scheme (see \cite{Bouzereau:2007}) which can take into
account temperature and specific humidity turbulent fluctuations inside the
resolved mesh:
\begin{flalign}
&q_{l}=q_{w}-q_{s}\\
&\! q_{w}-q_{s}=A_{l}
\left( \overline {q_{w}}-\overline{q_{sl}} \right)+A_{l}\left(
q_{w}^{'}-\alpha_{l}\theta_{l}^{'} \right)
\end{flalign}
with
$A_{l}=\left(1+\frac{{L_{v}}^{2}\overline{q_{sl}}}{C_{p}R_{v}\overline{T}_{l}^{2}} \right)^{-1}$
and
$\alpha_{l}=\frac{L_{v}\overline{q_{sl}}}{R_{v}\overline{T}_{l}^{2}}\frac{\overline{T}}{\bar{\theta}}$

where $q_{s}$ (kg kg$^{-1})$ is water vapor specific humidity at saturation
level and R$_{v}$ is the specific gas constant for water vapor.

$with\, q_{s}=\frac{0.622\, e_{sat}}{P-0.378e_{sat}}\, ,\, \,
e_{sat}=610.78\, \mathrm{exp}(17.269\ast \frac{\overline {T}-273.15}{\overline T
-35.86})$ \textit{and} $\overline{q_{sl}}=q_{s}\left( \overline{T_{l}} \right)$

where $e_{sat} $ is saturation vapor pressure and $P$ atmospheric pressure.

We assume a bivariate normal distribution function $\tilde{G}(\theta
_{l}q_{w})$ in order to represent the subgrid-scale fluctuations of the
variables $\theta_{l}$ and $q_{w}$:
\begin{equation}
R=\int_{-\infty}^{ +\infty}\int_{q_{s}}^{+\infty}\tilde{G}\left(\theta_{l},q_{w} \right)dq_{w}\, d\theta_{l}
\end{equation}
\begin{equation}
\overline {q_{w}}=
\int_{-\infty}^{ +\infty}\int_{q_{s}}^{+\infty}
(q_{w}-q_{s})\tilde{G}\left( \theta_{l},q_{w} \right)\, dq_{w\, \,
}d\theta_{l}
\end{equation}
We reduce the integration to a single variable $s=\bar{s}+s^{'}\, with\,
s=\frac{A_{l}\left( \overline {q_{w}}-\overline{q_{sl}}
\right)}{2}, s'=A_{l}(q_{w}^{'}-\alpha_{l}\theta_{l}^{'})/2$

and introduce $Q_{1}=\bar{s}/s'$ a dimensionless measure of the deviation of
the mean state from saturation and $t=s^{'}/\sigma_{s'}$ with $\sigma
_{s'}=(\frac{A_{l}}{2})(\overline{q_{w}^{'2}}+\alpha_{l}^{2}\overline{\theta
_{l}^{'2}}-2\alpha_{l}\overline{\theta_{l}^{'}q_{w}^{'}})$.

In that condition, partial cloudiness R, cloud water specific humidity, and
second order momentum can be estimated for different subgrid distribution
(all or nothing and gaussian) and are given in the following table:

\begin{table}[htbp]
\begin{center}
\caption[distribution]{}
\begin{tabular}{|p{99pt}|p{177pt}|p{184pt}|}
\hline
Subgrid distribution&
``All or nothing'' distribution&
Gaussian distribution \\
\hline
G(t)&
$\delta (t)$ Dirac distribution&
$\frac{1}{\sqrt {2\pi } }e^{-t^{2}/2}$ \\
\hline
R&
$Q_{1}\ge 0,\quad 1$ \par $Q_{1}<0,\quad 0$&
$\frac{1}{2}\left(1+erf\left(Q_{1}/\sqrt 2 \right) \right)$ \\
\hline
$\frac{\overline{q_{l}}}{2\sigma_{s'}}$&
$Q_{1}\ge 0,\quad Q_{1}$ \par $Q_{1}<0,\quad 0$&
$RQ_{1}+\frac{e^{-Q_{1}^{2}/2}}{\sqrt {2\pi } }$ \\
\hline
$
\frac{\overline{s'q_{l}^{'}}}{2\sigma_{s'}^{2}}
$
&
0&
$R-\frac{\bar{q_{l}}}{2\sigma_{s'}}\frac{e^{-Q_{1}^{2}/2}}{\sqrt {2\pi } }$\\
\hline
\end{tabular}
\label{tab3}
\end{center}
\end{table}

We have now to determine$\, \overline{q_{w}^{'2}}$,$\, \overline{\theta_{l}^{'2}}$
and $\overline{\theta_{l}^{'}q_{w}^{'}}$. This can be done by stationarising
(balance between production and dissipation) the equation of evolution of
the second order moments. That gives (see \cite{Musson-Genon:1995}):
\begin{equation}
\overline{q_{w}^{'2}}=\frac{2}{c_{2}}\frac{e}{\varepsilon }K_{\theta }\left(
\frac{\partial \overline {q_{w}}}{\partial z} \right)^{2}{,\quad
\overline{{\theta'_{l}}^{2}}}=\frac{2}{c_{2}}\frac{e}{\varepsilon }K_{\theta }\left(
\frac{\partial \overline{\theta_{l}}}{\partial z} \right)^{2},\quad
\overline{\theta_{l}^{'}q_{w}^{'}}=-\frac{2}{c_{2}}\frac{e}{\varepsilon }K_{\theta }{\left(
\frac{\partial \overline {q_{w}}}{\partial z} \right)\left( \frac{\partial
\overline{\theta_{l}}}{\partial z} \right)}
\end{equation}
With $c_{2\, }=$2.3 and $K_{\theta }=c_{\mu }\frac{e^{2}}{\varepsilon }$
for the e-$\varepsilon $ turbulent closure

In addition, the buoyancy term in the e-$\varepsilon $ turbulent closure has
to be expressed in terms of $\theta_{l\, }$ and $ q_{w}$ variables. This
leads to the following expression (see \cite{Bouzereau:2007}).
\begin{equation}
\overline{w^{'}\theta_{v}^{'}}=E_{\theta }\overline{w^{'}\theta
_{l}^{'}}+E_{q}\overline{w'q_{w}^{'}}
\end{equation}
where

\begin{equation}
\end{equation}
$E_{\theta }=\tau -NA_{l}\alpha_{l}D_{q\, }and\, E_{q}=0.608 \, \theta
+NA_{l}D_{q}$ and $\tau =\left( 1+0.608 \, \overline {q_{w}}-1.608 \, \bar{q_{l}}
\right)$,
$D_{q}=\frac{g}{T_{0}}-1.608 \, \bar{\theta }$ and
$N=\overline{s^{'}q_{l}^{'}}/2\sigma_{s'}^{2}$
T$_{0}$ being the mean temperature
of the atmospheric boundary-layer.

\subsection{ Evolution of the droplet spectrum }
For computing the evolution of the droplet spectrum, we have chosen a
semi-spectral method for its simplicity and its low computational cost.

The prognostic equation for the number of droplet $N_{d}$ is (here N$_{d}$ is
the ensemble mean of droplet number):
\begin{equation}
\left( \frac{\partial }{\partial t}+\bar{u_{i}}\frac{\partial }{\partial
x_{i}} \right)N_{d}=\frac{1}{\rho }\frac{\partial }{x_{i}}\left[ \left(
\frac{\lambda_{c}}{C_{p}}+\frac{\mu_{t}}{\sigma_{t}}
\right)\frac{\partial \overline {q_{w}}}{\partial x_{i}} \right]+\left(
\frac{\partial \bar{q_{l}}}{\partial t} \right)_{E/C}+\left( \frac{\partial
\bar{q_{l}}}{\partial t} \right)_{nuc}+\left( \frac{\partial
\bar{q_{l}}}{\partial t} \right)_{sed}
\end{equation}
where the last three terms of the right-hand side are respectively the
evaporation and condensation term, the nucleation term and the droplet
settling term. Besides, this equation is associated with a lognormal
distribution law defined as follows:

$N_{d}\left( r \right)=\frac{N_{d}}{r\sigma_{c}\sqrt {2\pi } }exp\left[
\frac{-1}{2\sigma_{c}^{2}}\left( ln\frac{r}{r_{0}} \right)^{2} \right]$

where r is the droplet radius, r$_{0}$ is the distribution median value and
$\sigma_{c}$ the logarithmic standard deviation.

With this law, the relationship between liquid water content q$_{l}$ and
N$_{d}$ is:
\begin{equation}
\rho \bar{q_{l}}=\frac{4}{3}\pi \bar{r_{3}}^{3}\rho_{w}N_{d}
\end{equation}
Where$\, \, \, \, \bar{r_{3}}=r_{0}exp\left( \frac{3}{2}\sigma_{c}^{2}
\right)$ is the mean volume (or mass) radius and $\rho
_{w}=$1000kg.m$^{-3}$ the density of liquid water.

We will now describe the different processes which are playing a role in the
evolution equation for $N_{d}$.

\subsubsection{Nucleation}
The nucleation is described by the equations of the heterogeneous nucleation
occurring in the atmosphere and summarized in
\cite{Pruppacher:2000}. It depends on the characteristics of the cloud condensation nuclei
(CNN). We have:
\begin{equation}
\left( \frac{\partial N_{d}}{\partial t} \right)_{Nuc}=\frac{1.}{\Delta
t}max\left( \left( N_{CCN} \right)_{max}-N_{d},\, 0. \right)
\end{equation}
where N$_{CCN}$ is the Cloud Condensation Nuclei density and $\Delta $t is
the numerical time step.

The number of droplet formed through nucleation is strongly correlated to
the super-saturation
\newline
s $=$ e/e$_{sat}$ -1

 Supersaturation is characterized by s\textgreater 0
(e being the partial pressure of water vapour in moist
air and e$_{sat}$ the saturation vapour pressure).

A simple expression taking into account all cooling processes for s is (see
\cite{Zhang:2014}):
\begin{equation}
\frac{ds}{dt}=\left( \frac{\xi L_{v}g}{R_{a}T^{2}C_{p}}-\frac{g}{R_{a}T}
\right)\bar{w}-\left( \frac{R_{a}T}{\xi e_{s}}+\frac{\xi L_{v}^{2}}{pTC_{p}}
\right)\frac{d\bar{q_{l}}}{dt}+\frac{\xi L_{v}g}{{\rho
R}_{a}T^{2}C_{p}}\frac{\partial F_{rad}}{\partial z}
\end{equation}
\begin{center}
lifting \quad	\quad \quad \quad \quad \quad \quad \quad \quad latent	\quad heat \quad	release \quad	radiation
\end{center}
\begin{equation}
=A_{1}w-A_{2}\frac{d\bar{q_{l}}}{dt}+A_{4}\frac{\partial F_{rad}}{\partial
z}
\end{equation}
where $\xi =$ 0.622 (molecular weight of water/molecular
weight of air), $g $(m s$^{-2})$ is the gravity, $R_{a}$ is the gas constant for
dry air, $\bar{w}$ (m s$^{-1})$ is the vertical air velocity, $e_{s}$ (Pa) is
the saturation vapor pressure water and $F_{rad}$ the net radiation flux (W
m$^{-2})$.

\subsubsection{Abdul-Razzack nucleation scheme}

As the primary source of cloud droplets, the nucleation process depends on
many factors including the characteristics (size and chemical composition)
of the aerosols. In order to take into account size distribution and
chemical composition if they are known, the Abdul-Razzak (AR) scheme
for aerosol activation is used \cite{Abdul-Razzak:2000}. With the superposition of three
lognormal aerosol distributions, as proposed by AR, the droplet number
concentration at the maximum super-saturation $s_{max}$, is given by:
\begin{equation}
N_{d}\left( s_{max} \right)=\frac{1}{2}\sum\limits_{i=1}^3 {N_{ai}\left(
1-erf\left[ \frac{2\, ln\left( s_{i}/s_{max} \right)}{3\sqrt {2\, } ln\left(
\sigma_{ai} \right)} \right] \right)} ,
\end{equation}
where $N_{ai} $ is the total aerosol number concentration of mode $i$,
$s_{i}$ the critical super-saturation of a particle with the diameter $r_{ai}$ and
the geometric mean diameter of the aerosol mode $i$.

It can be calculated by using K\"{o}hler's theory.

The maximum supersaturation $s_{max} $ is given by:
\begin{equation}
s_{max}=\sum\limits_{i=1}^3 \frac{1}{s_{i}^{2}} \left[ f_{i}\left(
\frac{\varsigma }{\eta_{i}} \right)^{3/2}+g_{i}\left( \frac{s_{i}^{2}}{\eta
_{i}+3\varsigma } \right)^{3/4} \right],
\end{equation}
With $f_{i}=0.5\exp \left( 2.5\, {ln}^{2}\sigma_{i\, } \right),\, g_{i}=1+0.25\, ln\sigma_{i}$,
$s_{i}=\frac{2}{\sqrt B_{i} }\left(\frac{A}{3r_{a}} \right)^{3/2}$
\newline
and $\varsigma =\frac{2A}{3}\left( \frac{A_{1}w+A_{4}\partial
F_{rad}/\partial z}{A_{3}} \right)^{1/2}$, $\eta_{i}=\frac{\left[ \left(
A_{1}w+A_{4}\partial F_{rad}/\partial z \right)/A_{3} \right]^{3/2}}{2\pi
\rho_{w}A_{2}N_{ai}}$
\newline
where $\rho_{w}$ (kg m$^{-3})$ is the water density, $A_{1}$, $A_{2}$,
$A_{4}$ are the constants defined above,
and $A_{3}$, $A$, $B_{i}$ can be found in
 \cite{Abdul-Razzak:2000}.

\begin{equation}
A_{3}=\left( \frac{\rho_{w}R_{v}T}{e_{s}D_{v}^{\star }}-\frac{L_{v}\rho
_{w}}{k_{a}^{\star }T}\left( 1-\frac{L_{v}}{TR_{v}} \right) \right)^{-1},\,
\, A=\frac{2\sigma_{vw}}{R_{v}T\rho_{w}},
B_{i}=\frac{\nu \phi_{s}\epsilon_{m}M_{w}\rho_{ai}}{M_{ai}\rho_{w}}
\end{equation}
where $k_{a}^{\star }=4.187\, {10}^{-3}\left( 5.69+0.017\, T\left( in\,
^{\circ}C \right) \right)$ Jm$^{-1}$s$^{-1}$K$^{-1}$ is the dry air
conductivity and

$D_{v}^{\star }=0.211\, {10}^{-4}\left( \frac{T}{273.15} \right)^{1.94}\left(
\frac{1.01325\, {10}^{5}}{P} \right)$ m$^{2}$s$^{-1}$ the water vapour
diffusivity.

$M_{w}$ is the molecular weight of water, $\rho_{w}$ water density,
$\sigma_{vw}$ is the partial pressure at air/water interface,
\newline
$\nu $ the number of ions the salt dissociates into within water,
\newline
$\varphi $ the osmotic coefficient, $\rho_{ai}$ the density of aerosols material,
\newline
$M_{ai}$ the molecular weight of the aerosol material.
\newline
When not any information is available on the characteristic of the aerosols,
a simpler model can be used deduced from typical air mass aerosol
concentrations.

\subsubsection{Cohard and Pinty nucleation scheme}

The Cloud Condensation Nuclei (CCN) density is given by:
\newline
$N_{ccn}\left( {cm}^{-3} \right)=\tilde{C}\tilde{s}^{k}=Cs^{k}\, \, \, \, \,
$when s\textgreater 0
\newline
where the deviation from saturation s is defined by s $=$ e/e$_{sat}$ -1 and
\newline
when s\textgreater 0 is called the supersaturation (e being the partial
pressure of water vapour in moist air and e$_{sat}$ the saturation vapor
pressure). Note that the nucleation is not possible for negative value of s.
The parameters C and $\tilde{C}$ are constants of the nucleation
process.

Due to different usage in the literature, it is important to be aware of the
difference between $\tilde{C}$ and $C=10^{2k}\tilde{C}$ (C is in cm$^{-3})$ and
between $\tilde{s}$, the super-saturation in percentage, and s
($\tilde{s}=100\mbox{s})$.
\newline
\newline
In our simulation, we choose the model and methodology of
\cite{Cohard:1998} and \cite{Cohard:2000}, that is: $N_{ccn}\left(
{cm}^{-3} \right)=\tilde{C}\tilde{s}^{k}F\left( \mu
,\frac{k}{2};\frac{k}{2}+1;-\beta \tilde{s}^{2} \right)$
\newline
F is the hyper-geometric function adjusted to give rather good results from
weak (s\textless 0.02{\%}) to significant (s\textgreater 1{\%})
super-saturation, as it could occur in the atmosphere. $\mu$, $\beta $ and $k$ are
some parameters characterizing the aerosols distribution of the air mass.

\subsubsection{ Sedimentation and deposition}
Cloud water sedimentation is included by assuming a lognormal size
distribution of droplets falling in a Stokes regime, in which the changes to
N$_{d}$ and q$_{l\, }$are given by:

\begin{equation}
\left( \frac{\partial N_{d}}{\partial t} \right)_{sed}=\frac{\partial
}{\partial z}\int\limits_0^\infty {V_{g}\left( r \right)n_{d}\left( r
\right)dr} =\frac{\partial }{\partial z}\left( N_{d}V_{g}\left( r_{mv}
\right)\exp \left( -\sigma_{d}^{2} \right) \right)
\end{equation}

\begin{equation}
\left( \frac{\partial \bar{q_{l}}}{\partial t} \right)_{sed}=\frac{1}{\rho
}\frac{\partial }{\partial z}\int\limits_0^\infty {V_{g}\left( r
\right)\frac{4\pi }{3}r^{3}n_{d}\left( r \right)dr} =\frac{1}{\rho
}\frac{\partial }{\partial z}\left( \rho \bar{q_{l}}V_{g}\left( r_{mv}
\right)\exp \left( 5\sigma_{d}^{2} \right) \right)
\end{equation}
\newline
where$ V_{g}$ is droplet fall velocity.
\newline
In this parametrization $V_{g}$(r) is calculated as a function of droplet
size radius:
\begin{equation}
V_{g}\left( r \right)=\rho gC_{c}r^{2}\left( 4.5\mu_{air} \right)^{-1}
\end{equation}
where $C_{c} =$\textit{ 1. }$+ l_{air}$\textit{ (1.257 }$+$\textit{ 0.4 exp(1.1 r/l}$_{air}))/r$ is the slip correction factor to
account for non-continuum effects for small droplets and which depends on
the mean free path of air molecules l$_{air}$, $\mu_{air}$ is the
viscosity of the air.

Fog deposition process over vegetation has long been recognized as an
important factor in the water balance of ecosystems. This process results
from the turbulence exchange of fog water and collection from the surface.
The deposition flux of fog water, $F_{dep}$, is predicted from the simple
inferential model equation of the type:

\begin{equation}
F_{dep}=\bar{q_{l}}\frac{1}{R_{t}}=\bar{q_{l}}V_{dep}
\end{equation}
where $V_{dep}$ is the deposition velocity and $R_{t}$ is the total resistance
against deposition and computed as a combination (parallel and serial
arrangements) of aerodynamic ($R_{aero})$ and surface ($R_{surf})$ resistances
within the first layer (between the ground surface and the first grid level):
\newline
$R_{aero}=\frac{1}{2}\frac{\ln \left( \frac{z_{1}}{z_{0}}
\right)-\mathrm{\Psi }_{h}}{\kappa u_{\star }}$ and
$R_{surf}=\frac{1}{\varepsilon_{0}u_{\star }\left( E_{imp}+E_{int} \right)}$
\newline
where $z_{1}$ is the height at which the deposition velocity is evaluated,
$z_{0\, }$ is the roughness height, $\psi_{h}$ is the stability function,
$\kappa $ is the von Karman constant, $u_{\star }$ is the friction velocity,
$\varepsilon_{0}=$ 3 is an empirical constant for all types of land.
$E_{imp}$, $E_{int}$ are collection efficiency from impaction and interception
respectively (see \cite{Zhang:2001} ):
\newline
\newline
$E_{imp}=\left( \frac{St}{St+1.5} \right)^{2}$ with $St=\frac{V_{g}u_{\star
}}{0.01g}$ , $E_{int}=2\frac{r_{mv}}{0.01}$
\newline
\newline
The collection by Brownian motion of fog droplet is neglected because its
effect is only significant for very small particle diameter (\textless 0.1
$\mu$m).

If a deposition process is taken into account in the model, the new bulk
velocity that is estimated for liquid water will be calculated as the sum of
$V_{g}$ and $V_{dep\, }$at the$_{\, }$lowest level and this will be used to
estimate $F_{dep}$.

\subsubsection{Condensation-evaporation}
The condensation processes do not have any effect on $N_{d}$, the nucleation
being treated independently, thus we can write:
\begin{equation}
\left( \frac{\partial N_{d}}{\partial t} \right)_{C}=0.
\end{equation}
For the evaporation, we distinguish two regimes:
\begin{itemize}
\item the total evaporation of all droplets (s\textless 0 and
 $\bar{q_{l}}=$0) is simply described by $N_{d}=$\textit{0} (see \cite{Cohard:2000}).
\item the partial evaporation (s\textless 0 and
 $\bar{q_{l}}$\textgreater 0, i.e. a total evaporation only for the
 smallest droplets), we choose to follow \cite{Chaumerliac:1987} and to force
 the droplets to disappear for radius less than a critical value r$_{crit}$:
\end{itemize}
\begin{equation}
r_{crit}=\sqrt {-2A_{3}s\mathrm{\Delta }t}
\end{equation}

Finally, the expression for partial evaporation is:
\begin{equation}
\left( \frac{\partial N_{d}}{\partial t}
\right)_{E/C}=\frac{1}{\mathrm{\Delta t}}\int_{0}^{r_{crit}}
\frac{N_{d}}{r\sigma_{c}\sqrt {2\pi } } exp\left( -\frac{1}{2\sigma
_{c}^{2}}\left( ln\frac{r}{r_{0}} \right)^{2}\, \right)dr
\end{equation}
For $\overline{{\, q}_{l}}$, condensation and evaporation are not treated
explicitly but implicitly through the subgrid scheme (see \cite{Bouzereau:2007}).

\section{Earth-atmosphere interactions}
In \CS different options can be used in order to
determine surface boundary conditions.

\subsection{Direct estimation of turbulent fluxes}
The first one is to directly impose the fluxes for momentum (surface shear stress), temperature and
humidity (eventually any scalar). That consists in giving data for
${u^{\star}} ^2 =\sqrt {\overline {u'w'}^{2}+\overline {v'w'}^{2}} $ where
$u',\,\,v',\,\,w'$ are the turbulent fluctuations of
the three components of the wind speed,
\newline
\newline
$Q_{0} =\overline{w'\theta'}$ the kinematic flux of sensible heat where
$\theta '$ is the turbulent fluctuation of the potential temperature,
\newline
\newline
$E_{0} =\overline {w'q'} $ being the kinematic
evaporative flux and $q'$ the turbulent fluctuation of the specific
humidity.

These data could be obtained, for example, by experimental
measurements with sonic anemometer.

\subsection{Impose temperature and humidity from experimental estimation}
In that option fluxes are computed from difference of temperature and
humidity between their values at ground level (z$_{1}=$z$_{0}$ roughness
height for the wind, z$_{1}=$z$_{ot}$ thermic roughness height for
temperature and humidity) and at the first level z$_{2}$ in the air.
It needs to determine turbulent fluxes from gradients (see \cite{Musson_Genon:2007}).

Monin-Obukhov's similarity theory gives relations between the fluxes of
sensible heat, latent heat and momentum by means of universal functions:
\begin{equation}
\label{eq:atmo:eq5}
\frac{\partial \left| {\vect{u}} \right|}{\partial z} =
\left( {\frac{{u}_{{\star }} }{\kappa z}} \right) \, \phi_{m} \left(
{\dfrac{z}{L}} \right),
\end{equation}

\begin{equation}
\label{eq:atmo:eq6}
\frac{\partial {\theta }}{\partial z}=\left( {\frac{\theta_{\star}}{\kappa z}} \right)\phi_{h} \left({\frac{z}{L}} \right)
\end{equation}

\begin{equation}
\label{eq:atmo:eq7}
\frac{\partial q}{\partial z}=\left(
{\frac{q_{\star} }{\kappa z}} \right) \,
\phi_{q} \left(
{\frac{z}{L}} \right),
\end{equation}


where $\left| {\vect{u}} \right|$ is the modulus of the horizontal mean wind speed,
\newline
$u_\star $ is the friction velocity with $u_{\star}^{2} =\sqrt{(\overline{u'w'}^{2}+\overline{v'w'}^{2})}$
\newline
$u',\,\,v',\,\,w'$ are the turbulent fluctuations of the three components of the wind speed,
\newline
where $\theta$ is the mean potential temperature and q is the mean specific humidity:
\begin{equation}
\theta_{\star}=\frac{-Q_{0}}{u_{\star}}
\end{equation}

\begin{equation}
q_{\star} =-E_{0} \mbox{/u}_{\star}
\end{equation}
\newline
Here, $\kappa$ denotes Von Karman's constant, $\phi_{m} \, \phi_{h} \,\,\mbox{and}\,\,\phi_{q} $
are the universal functions for wind, temperature and humidity, z is the
altitude and L is the Obukhov length:
\begin{equation}
\label{eq:atmo:eq8}
L = \dfrac{-u_{\star}^{3}}{
  \kappa \dfrac{g}{T_{0}} \, Q_{0}
  +0.61T_{0} E_{0}
},
\end{equation}
where T$_{0}$ is the mean temperature in the surface layer (z$_{2}$-z$_{1}$
layer).

Usually, the function $\phi_{q} $ is chosen equal to $\phi
_{h} $. Classically, see for example \cite{Garratt:1992} or \cite{Cheng:2005},
integration of the relations \eqref{eq:atmo:eq5}, \eqref{eq:atmo:eq6} and \eqref{eq:atmo:eq7}
between two different levels $0<z_{1} <z_{2} $ gives:
\begin{equation}
\label{eq:atmo:eq9}
\delta u=u(z_2)-u(z_1)=
\left(\frac {u_{\star}}{\kappa} \right)\Psi_{m}\left({L,z_2,z_1} \right)
\end{equation}

\begin{equation}
\label{eq:atmo:eq10}
\delta \theta=\theta (z_2) -\theta (z_1)=
\left(\frac{\theta_{\star}}{\kappa} \right)\Psi_{h}\left({L,z_2,z_1} \right)
\end{equation}

\begin{equation}
\label{eq:atmo:eq11}
\delta q=q(z_2)-q(z_1)
\left( {\frac{q_{\star}}{\kappa}} \right)\Psi_{h} \left({L,z_2,z_1} \right)
\end{equation}

The formulation of the universal functions are given in \tablename{} \ref{tab:atmo:tab4} (two
options available) and the integrated form (expressions of the $\Psi$ functions:
the dynamical profiles $\Psi_{m}$ and the thermal profiles $\Psi_{h}$) are given in the following paragraph:

\paragraph{The dynamical profiles $\Psi_{m}$}

\subparagraph{Unstable case:}

$\zeta_{2}< \zeta_{1}$ \textless 0:
$\Psi_{m} =\log \frac{z_{2}}{z_{1}}-2\log \left({\frac{1+\Psi_{2} }{1+\Psi_{1} }}\right)-\log
\left( {\frac{1+\Psi_{2}^{2} }{1+\Psi_{1}^{2} }} \right)+2\left( {\arctan
\Psi_{2} -\arctan \Psi_{1} } \right)$,
\newline
where $\Psi_{2}= (1-b_{m} \zeta_{2})^{1/4}$ and $\Psi_{1} =(1-b_{m} \zeta_{1})^{1/4}$,
\newline
with $\zeta_{1}= $ z$_{1}$/L, $\zeta_{2}= $ z$_{2}$/L, b$_{m\, }= $ 15.

\subparagraph{Neutral or stable cases:}

0 \textless $\zeta_{1}$ \textless $\zeta_{2}$ \textless 0.5:
$\Psi_{m} =log\frac{z_{2} }{z_{1} }+b_{m} \left( {\zeta_{2} -\zeta_{1} } \right)$,

0.5 \textless $\zeta_{2}$ \textless 10:
$\Psi_{m} =c_{1m} log(2\zeta_{2})+\frac{4.25}{\zeta_{2}}-\frac{0.5}{\zeta_{2}^{2}}-log(2\zeta_{1} )-b_{m} \zeta_{1} -c_{2m} $

$\zeta_{2\quad }>\quad 10:
\quad
\Psi_{m} =d_{1m} \zeta_{2}
+c_{1m} log\left( {2\zeta_{2} }
\right)-\mbox{11.165}-log(2\zeta_{1}
)-b_{m} \zeta_{1}
$
\newline
with $\quad b_{m} = 4.7, c_{1m} = 7.85, c_{2m} = 4.15, d_{1m} = 0.7435$

In the case of \cite{Cheng:2005}, $\Psi_{m\, }$ may be expressed as:

0 \textless $\zeta_{1}$ \textless $\zeta_{2}$:$_{\, }\Psi_{m} =alog\frac{z_{2} }{z_{1}
}-\mbox{a}\,\,log\left( {\zeta+\left( {1+\zeta^{b}} \right)^{\frac{1}{b}}} \right)_{\, }$ with a
$=$ 6.1 and b $=$ 2.5

\paragraph{The thermal profiles $\Psi_{h}$}

\subparagraph{Unstable case:}

$\zeta_{2 }< \zeta_{1}$ \textless 0: $\Psi_{h}
=\mbox{a}_{h} \left( {log\left( {\frac{z_{2}
}{z_{1} }} \right)-\mbox{2log}\left(
{\frac{1+\Psi_{2} }{1+\Psi_{1}
}} \right)} \right)$,

where $\Psi_{2} =1-b_{h} \zeta_{2} )^{1/2}$ and $\Psi_{1}
=1-b_{h} \zeta_{1}
)^{1/2}$, with a$_{h\, }=$ 0.74, b$_{h\, }=$ 9

\subparagraph{Stable cases:}

$\zeta_{2 }> \zeta_{1}$ \textgreater 0: $\Psi_{h}
=alog\frac{z_{2} }{z_{1} }+b(\zeta_{2} -\zeta_{1} )$ ,

with a$_{h\, }=$ 0.74, $b_{h\, }= $ 4.7 for \cite{Businger:1971}

In the case of \cite{Cheng:2005}, $\Psi_{h}$ may be expressed as:

0 \textless $\zeta_{1}$ \textless $\zeta_{2}$:$_{\, \, }\Psi_{h} =alog\frac{z_{2} }{z_{1}
}-c\,\,log\left( {\zeta+\left( {1+\zeta^{d}} \right)^{\frac{1}{d}}} \right)_{\, }$with c
$=$ 5.3 and d $=$ 1.1.

\begin{table}[htbp]
\begin{center}
\caption[fonction]{Detailed expressions for the universal functions from
  \cite{Businger:1971}; \cite{Hicks:1976}, and \cite{Cheng:2005}
}
\begin{tabular}{|p{74pt}|p{116pt}|p{262pt}|}
\hline
\textbf{Function}&
\textbf{unstable} \par $\zeta <0$&
\textbf{stable}

$0<\zeta <0.5,\quad
1<\zeta <10,\quad
\zeta >10$ \\
\hline
unstable:
\cite{Businger:1971} \par stable: \cite{Hicks:1976}&
$\phi_{m} =\left( {1-15\zeta } \right)^{-1/4}$ \par
$\phi_{h} =0.74\left( {1-9\zeta } \right)^{-1/2}$&
$\phi_{m} =1+4.7\zeta ,\quad
\phi_{m} =7.85-\frac{4.25}{\zeta }+\frac{1}{\xi^{2}},\quad
\phi_{m} =0.7435\,\zeta $ \par $\phi_{h} =0.74+4.7\zeta $ \\
\hline
\cite{Businger:1971} \par stable: \cite{Cheng:2005} \par &

$\phi_{m} =\left( {1-15\zeta }\right)^{-1/4}$

\par
$\phi_{h} =0.74\left( {1-9\zeta } \right)^{-1/2}$&

$\phi_{m} =
1+a\frac{\zeta+\zeta^b{1+\zeta^b}^{\frac{1-b}{b}}} {\zeta+(1+\zeta^b)^{\frac{1}{b}} }$
a$=$6.1, b$=$2.5 \par

$\phi_{h}=
1+c\left(\frac{{\zeta+\zeta^d}\left({1\zeta^d}\right)^{\frac{1-d}{d}}} {\zeta+\left(1+\zeta^d)\right)^{\frac{1}{d}} }\right)$
c$=$5.3, d$=$1.1\small \\
\hline

\end{tabular}
\label{tab:atmo:tab4}
\end{center}
\end{table}

But these expressions do not determine explicitly the atmospheric fluxes as
functions of the differences of wind, temperature and humidity between two
levels. Two methods for solving this implicit problem are described in the
next section.

\subsubsection{The exact iterative method}

From the definition of $L$, and with the help of Equations \eqref{eq:atmo:eq9}, \eqref{eq:atmo:eq10} and
\eqref{eq:atmo:eq11} giving $u_\star $, Q$_{0}$ and E$_{0}$, the following
expression is obtained:
\begin{equation}
\label{eq:atmo:eq12}L=\frac{\delta u^{2} \Psi_{h}}{\frac{g}{T_{0}}\Psi_{m}^{2}(\delta {\theta}+0.61 T_{0} \delta {q})}
\end{equation}

Since $\Psi_{m}, \Psi_{h}$ depend on $L$, this
relation has to be solved by an iterative method. We follow \cite{Beljaars:1991}
using an initial value of $L$ that depends on the sign of
$\delta \theta +0.61 T_{0} \delta q$.

This iterative procedure converges rapidly in all unstable situations. In
stable situations for small values of $u_{\star}$ and $L$, the
solution for $L$ tends to zero. For this reason, a minimum value is imposed
for $L$. On the other hand, we have verified that the solution obtained is not
dependent on the initial value of $L$.

Having determined $L$ and $u_{\star}$, the kinematic
fluxes of sensible heat Q$_{0}$ and latent heat E$_{0}$ are then simply
given by \eqref{eq:atmo:eq10} and \eqref{eq:atmo:eq11}, as well as $q_{\star} $and
$\theta_{\star} $.

The iterative method is well adapted to local determination of the fluxes
because this method derives directly from the experimental determination of
the MO universal functions, using assumptions on which the MO theory is
based: quasi-stationarity and horizontally homogeneity of the flow, with the
Coriolis effect negligible.

\subsubsection{The approximate analytical method}
To determine the turbulent fluxes in a meteorological model it is preferable
to use an analytical approximation in order to avoid an expensive iterative
resolution of an equation.

This method is based on a formulation of the universal functions differing
from those presented in \tablename{} \ref{tab:atmo:tab4} and leads to a complete analytical method
(see \cite{Louis:1982}).

In this formulation the fluxes are estimated with the bulk Richardson
number, defined by:
\begin{equation}
\label{eq13}
Ri=\frac{g(z_{2} -z_{1})
({\delta \theta }
+0.61T_{0} \delta q)}
{T_{0} {\delta u}^{2}}
\end{equation}

Using Equation \eqref{eq:atmo:eq12} a relation between $Ri$, the measurement levels $z_{1}$ and
$z_{2}$, and $L$ is established:
\begin{equation}
\label{eq14}
\frac{L}{z_{2} -z_{1}
}=\frac{\Psi_{h} }{Ri\,\Psi_{m}^{2}},
\end{equation}

Then $u_{\star} $, Q$_{0}$, E$_{0}$ according to \cite{Louis:1982}, are
given by:

\begin{equation}
\label{eq15}
u_{\star} =\delta u \left( C_{n} F_{d} \left(Ri, \, \frac{z_{2}}{z_{1}} \right) \right)^{1/2},
\end{equation}

\begin{equation}
\label{eq16}
Q_{0} =-\delta \theta \delta u C_{n} F_{h} \left( Ri,\,\frac{z_{2} }{z_{1} }\right),
\end{equation}

\begin{equation}
\label{eq17}
E_{0} =-\delta q \delta u C_{n} F_{h} \left( Ri, \frac {z_{2}}{z_{1}} \right),
\end{equation}

where $C_{n} $ is defined as: $C_{n} =\left(
{\dfrac{\kappa}{\ln(z_{2} /z_{1} )}}
\right)^{2}$.

The universal functions $F_{d}$ and $F_{h}$ depend on the atmospheric
stability through the bulk Richardson number, and are tuned to give results
close to those of the exact iterative method, at least in the unstable case.

In the unstable case, the functions $F_{d}$ and $F_{h}$ must be
asymptotically equivalent to $Ri^{1/2}$ in order to be consistent with the
free convection regime. Moreover, continuity must be assured when crossing
the neutral point and the slope of the wind profile in the neighbourhood of
zero must be slightly larger than the slope of the humidity and temperature.
Taking all these considerations into account, one obtains,
\begin{equation}
\label{eq18}
F_{d} =1-\frac{2bRi
}{1+3b c_{\star} \sqrt {\left| {Ri} \right|} },
\end{equation}

\begin{equation}
\label{eq19}
F_{h}
=1-\frac{{3bRi}}{1+{2bc}_{\star} \sqrt
{\left| {{Ri}} \right|} },
\end{equation}
with $c_{\star} =C_{n} \, \left(
{1-z_{1} /z_{2} }
\right)^{1/2}\left( {(z_{1} /z_{2}
)^{1/3}-1} \right)^{3/2}$ and $b = c = 5$.

For the stable case, the total extinction of the turbulence is prevented by
a modification of the asymptotic behaviour of the universal functions in
order to avoid thermal disconnection in very stable model layers close to
the ground, but also to take into account subgrid or intermittent turbulence
effects for Richardson number greater than the theoretical critical value of
0.21. The expressions become:
\begin{equation}
\label{eq20}
F_{d} =\frac{1}{1+\frac{2bRi}{\sqrt
{1+d\, Ri} }},
\end{equation}

\begin{equation}
\label{eq21}
F_{h} =\frac{1}{1+ 3 b \, Ri\sqrt
{1+ d \, Ri} },
\end{equation}
with $b = d = 5$.

The functions do not depend on the ratio $z_{1}
\mbox{/z}_{2} $.

\subsection{Estimation of soil temperature and humidity from an energetic budget}

\subsubsection{Soil temperature evolution}
The soil temperature T$_{s}$ (assimilated to the temperature at z$_{ot}$,
thermal roughness height) is computed by using an energy balance equation
for the earth's surface after \cite{Deardorff:1978}.

\begin{equation}
\label{eq22}
\frac{dT_{s}}{dt}\mathrm{=}\frac{\mathrm{2}\pi^{\mathrm{1/2}}}{\rho
_{s}c_{s}\left( \nu_{s}\tau_{\mathrm{1}}
\right)^{\mathrm{1/2}}}H_{at}\mathrm{-}\frac{\mathrm{2}\pi }{\tau
_{\mathrm{1}}}\left( T_{s}\mathrm{-}T_{p} \right),
\end{equation}
\newline
with $H_{at}=\left( 1-\alpha_{g} \right)S\downarrow +\varepsilon
_{g}(F\downarrow -\sigma T_{s}^{4})-\rho_{0}C_{p}Q_{0}-\rho_{0}LE_{0}$,
\newline
where $\rho_{0}\, $ is the air density at $z_{0}$, T$_{p}$ the deep soil temperature,
\newline
$S\downarrow $ the downward solar flux, $F\downarrow $ the downward infrared flux,
\newline
$\rho_{s}c_{s}$ the ground calorific capacity, $\nu_{s}$ the soil thermal conductibility,
\newline
$\tau_{\mathrm{1}}=86400\, s$, the force restore time constant for T$_{p}$,
\newline
$\alpha_{g}$ is the soil albedo, $\varepsilon_{g}$ the soil emissivity,
\newline
$C_{p}=$1005. J K$^{-1\, }$kg$^{-1}$ is the specific heat at constant pressure of air,
\newline
L$=$ 2.5 10$^{6}$ J kg$^{-1}$ is the latent heat of vaporization and
\newline
$\sigma =$5.67 10$^{-8}$ W m$^{-2}$ K$^{-4}$ the Stefan constant.

\subsubsection{Soil-humidity evolution}
The evolution of the soil liquid water content is deduced from the same
principles as the surface temperature:
\begin{equation}
\label{eq23}
\frac{dw_{1}}{dt}=\frac{P_{r}-\rho E_{0}}{C_{1}}-\frac{w_{1}-w_{2}}{\tau
_{1}},
\end{equation}

\begin{equation}
\label{eq24}
\frac{dw_{2}}{dt}=C_{2}\frac{w_{1}-w_{2}}{\tau_{1}},
\end{equation}
Where P$_{r}$ is the precipitation flux,
\newline
w$_{1\, }=$ w$_{s}$/w$_{smax}$ is a non-dimensional liquid water content with w$_{s}$ the liquid water content
for the surface reservoir and w$_{smax}$ its maximum value and
\newline
w$_{2\, }=$ w$_{p}$/w$_{pmax}$ is a non-dimensional liquid water content with w$_{p}$ the liquid water content
for the deep reservoir (which does not contain the surface reservoir) and w$_{pmax}$ its maximum value.

The constants C$_{1}$ an C$_{2}$ are given by:
\newline
C$_{1}=$ w$_{smax}$/C$_{ws}$ and C$_{2}=$w$_{smax}$/w$_{pmax}$,
\newline
where C$_{ws}$ is a non-dimensional constant tuning
the depth of the reservoirs and C$_{2}$ the ratio of these depth.

The surface specific humidity q$_{s}$ is computed as a function of w$_{1}$,
which varies according to the nature of the surface (bare soil or soil
covered by vegetation characterized by veg, the percentage of vegetation in
the mesh):
\begin{equation}
\label{eq25}
q_{s}=H_{u}q_{sat}\left( P_{s},T_{s} \right)+veg\, \left( 1-H_{u}
\right)q_{a}
\end{equation}
\newline
Where $H_{u}=0.5\left[ 1-cos\left( \pi w_{1} \right) \right]$,
\newline
q$_{sat}$ is the saturated value of q$_{s}$, $P_{s}$ is the surface pressure
\newline
and q$_{a}$ the specific humidity in the first level in the air in \CS.
\newline
Here H$_{u}$ will be equal to 1 when a dew flux is present.

